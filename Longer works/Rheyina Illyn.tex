\documentclass{article}
\usepackage{fullpage, verbatim}


%*****************
% Annotations
\usepackage{soul}
\usepackage[colorinlistoftodos,textsize=footnotesize]{todonotes}
\newcommand{\hlfix}[2]{\texthl{#1}\todo{#2}}
\newcommand{\hlnew}[2]{\texthl{#1}\todo[color=green!40]{#2}}
\newcommand{\sanote}{\todo[color=green!30]}
\newcommand{\egnote}{\todo[color=violet!30]}
\newcommand{\newstart}{\note{The inserted text starts here}}
\newcommand{\newfinish}{\note{The inserted text finishes here}}
\setstcolor{red}
%***************************
\begin{document}
	The virtuous king thrust his spear into the belly of the vile dragon, and it roared in pain. It tried to flap its wings and thrash its body to fling off the king and his men, but the more it moved, the  deeper the spear dug. Once, twice, thrice that dragon tried to fly. And then …
	
	the marionette dragon erupted in a shower of confetti and candied citrus. The children shrieked in surprised delight. Then, realizing what had erupted, scrambled over each other to collect their treasures.
	
	The adults applauded enthusiastically at the cleverness of the device. The explosion had been truly magnificent, scattering fruit to all corners of the room. Reina examined the projectile that had landed close to her plate. A small orange fruit, the size of a finger knuckle, wrapped in a thin translucent layer of rice paper. The unexpected combination surprised her. Candied citrus and exploding marionettes were staples of entertainment from the Astrian court. Rice paper came from Clave, and hopefully soon, from Xi. 
	
	Antonia leant over to explain, “Don’t eat the rice paper. That was Igrit’s idea. She didn’t want the children eating things off the floor.” 
	
	Reina caught Igrit’s eye and raised the sweet in salute. The other woman returned the greeting with the traditional double bow of the Clavic. Reina popped the egg shaped fruit into her mouth and chewed thoughtfully, teasing apart the bitterness of the rind, the sweet tang of the crystalized center, and the clove infused sugar it had all been preserved in. She had never had anything like it in her life. But she had never encountered this fruit before. 
	
	Antonia was looking at her expectantly. “Do you like it?”
	
	“Little parrot,” Reina assured, “You have outdone yourself. What did I just eat?”
	
	Antonia blushed with pride and smoothed the pleats of her cotton gown over her just showing belly. The gathering was in honor of her impending seclusion. Unlike the Saran tradition of unceremoniously separating women from the harem just a few weeks before they gave birth, the Astrian custom was to isolate the mother to be with a ‘pampering’ celebration as soon as she started showing. “A kumquat, I think it is called. My husband has been bringing me a box from Xi with every barge. I cannot get enough of them.” A look of doubt clouded over the young mother’s face. “It’s not too untraditional, is it?”
	
	Reina looked around the room. Seven of Riosk’s most promising men sat with their wives from every corner of the empire, and countries that had yet to be swallowed by her greed. Men with diplomatic ambitions or senatorial dreams. No major players in the Saran court yet, but every one of them wanted to be. Some, like Antonia’s husband, came from merchant stock and had freshly bought their titles. The Kokhan’s court would never be catered by anything but the most traditional of Saran food, served to the most Saran of families. Dishes and dynasties whose heritage could be traced back for centuries to where the food or family was known to have been the favorite of a particularly historic Kokahn. This gathering, on the other hand, had children as varied in skin color and hair texture as the fish in the sea. The empire had expanded rapidly in the days of the old Kokhan, and would likely expand again in the next. This, Reina reminded herself, was a gathering of and for the empire. “Untraditional!” Reina laughed. “My darling girl. I hope these … kumquats … become a staple for years to come.”
	
	An eight year old girl with the heart shaped face and hooded eyes so favored as Saran signs of beauty ran up to her to drop off her treasure of golden egg shaped fruit. She had an unusual dusting of freckles on skin the color of freshly baked bread, and Reina knew that left to its own devices, her black hair would revert to tight frizzy curls completely foreign to Saran society. The child wrapped her arms around Reina’s neck, gave her a kiss, and ran off arm in arm with a short stocky girl from Xi. Taria, the Kokhanati, and Quarzen. Reina’s daughter was not officially part of the Kokhani’s harem, as her collection adopted daughters from all extremities of the empire, past, present and future, were not so affectionately called, but she was raised with them. Devisively diplomatic as only a child can be, she chose a new favorite on almost a weekly basis. This week, it was Quarzen.
	
	“They are dressing the fig tree!” Antonia said with sudden excitement. “Excuse me, Mei.”  Someone had brought in a large potted fig tree, and a group of servants were laying out baskets of ribbons, fabric scraps and trinkets with which to adorn the poor plant. Reina wasn’t exactly certain where the tradition of dressing the fig tree came from in Astria. It certainly wasn’t a religious ritual. But if a woman was about to give birth, or once before each of the sowing seasons, the women of Astria gathered around a communal fig tree and decorated it until it looked like a travelling scrap merchant’s dummy. 
	
	Reina watched, choosing not to join the jostle of the crowd. Most of her girls were there, alumni and current, along with their daughters and some, with their mothers. It pleased her to see an elderly Ursk woman help her toddling granddaughter choose fabrics. Judging by her fur lined skirt, she was a devotee of Janma, pracitising a monotheistic religion that believed in the River Janma as its only god. The mouth of the Janma, deep in the mountains of Ursk, was sacred to them, and they had a natural enmity with the people of the ancestor worship Gulab faith, who also held the Ursk mountains sacred. When Sardona set its greedy eyes on Astria, it found it convenient emphasize the religious similarities of between the ancestor worship in Astria and the Gulab faith. Ostracize a troublesome ethnic group and drum up fervor for a new conquest, all in one go. Perfectly typical of the old Kokhan’s style. It pleased her, therefore, to have undone so much of his work less than a decade after he had gone to his grave. But there was more to the warmth she felt at the sight than just spite. It was …
	
	“Kokhan Mata, may Uhr protect your footsteps.” 
	
	She turned around with a start. Two men, one middle aged, another younger bowed deeply. “Why so formal, Hamed? Rise.” Hamed ben Nebab had been one of the first men to marry one of her alumni. Freya, a Velutian girl from the last dynastic line before Velut had been annexed. He was, on a technicality, a half brother to both the old Kokhan and the current Regent, but his mother had been set aside for practicing a sect of the Saran faith popular in the north, from whence she hailed, and Hamed had been raised in the provinces.
	
	“My brother, Ostig, is eager to make your acquaintance,” he said, introducing the younger man. “He is my kinsman by Freya, and the owner of the finest herd of Velutian stallions you have ever seen.” 
	
	Business then, very well. She motioned for the men to rise, and walked with them to a quieter part of the house where she could give them her full attention.
	
	**********************************
	
	“Well, what do you think. Are you ready?” It was long past late, as the women gathered around  in the dressing room donning their abayas and veils over the traditional dresses they had worn to the pampering hailing from all corners of the Saran world. Her own girls had come in matching dresses of the Astrian court, rose, with gold brocade, unified in their display of an as yet foreign culture.
	
	Kripna fussed at the fall of her veil before she put the headpiece on. “It was lovely, Maya, but I’ll never have something like this.” She had been promised to the Bir of the Hasti, a semi nomadic protectorate that hunted on the banks of the Janma river near where the Oniyata joined it. In other words, they were the people who controlled the lands surrounding a key part of the river route to Xi and its rich hardwood forests, and apparently, delicious kumquats. The marriage had been scheduled for earlier in the spring, but then delayed. The last six months had seen rioting in the Janma basin and raiding of the barges to and from Xi. Reina had been told in no uncertain terms that there was to be no union between any of the peoples of the Janma basin and her girls until the matter could be settled. The Regent was being infuriatingly closed minded on the subject, but at least, she noted, the marriages she made now mattered enough to running of the empire that it was worth the Regent’s time to stop them. She would accept that victory for now.
	
	“You are my best, brightest and most talented girl,” Reina stood on tiptoe and gave Kripna a congratulatory kiss on the cheek. “You will make the Hasti the bass of the Sardonan table. The Bir is old. Give him a child before he dies, help maintain the Xi trade routes and use the profits to show the empire your people’s cultural beauty. ”  Kripna did not protest audibly, but Reina could see her looking around with longing at the high vaulted ceilings and the cool mosaic floors. Whatever riches in furs and trade levies the Hasti could provide paled in comparison to the glitter of the capitol. This had been the only home she had known for the last 6, nearly seven years, and she did not want to be sent away.
	
	“Taria, Elanya, Isolde, Aa’ah, Gris, Quarzen and Kripna,” Hazad called out in a loud singsong as an eunuch rang a gong outside the dressing room curtain, arranging her daughters from youngest to oldest. “Your sedans are ready.” 
	
	Hazda was nearly as old as herself, and had once been the first and only member of what the Riosk court had started calling the Kokhani harem. She had also been the first person that Reina had been able to love in all of Sardona. But that had nothing to do with the epiteth. Hazda had never seemed to mind the epiteth, which made it easier for Reina to embrace the insult and remove the scorpion’s stinger. Hazda had married, been widowed, and come back to her to help her manage her daughters. Some days, Reina imagined, that Hazda ran the girls with more efficiency than any officer in the great Saran army.
	
	Reina watched her alumni embrace and say their farewells until the next of these reuninons while Hazda rounded the girls up in groups of twos and threes into ther sedans. With eleven alumni married to husbands with positions in Riosk, another gathering was not far off. The alumni, she noted, did not segregate by birth religion of native language or region of origin. She insisted, upon accepting a child for adoption, that the family provide staff sufficient to appropriately feed, cloth, educate and provide for their spiritual needs. Every girl ate what the others ate, learnt to speak in each other’s tongues and listened to the histories and made each other whole. They only prayed to their individual gods, which Reina knew was a wasted effort. They would have to renounce and convert if they wanted to marry into the imperial court, but Reina could not bring herself to tear them away from the bosoms of their own gods. The entire setup was expensive, as the Regent constantly reminded her. Hedonistic and sacreligious were the polite words used by some in the Kokhanat’s circle.
	
	Whatever the words.It warmed her heart to see her daughters gather like this. Sardona would expand. There was nothing she could do about that. Whoever the army brought into the empire, she would make certain they were welcomed to the court at Riosk.
	
	Hazda hovered at her elbow to tell her that her Sedan was ready. She gathered up her abaya as the eunuch opened the door for her, then stumbled off the stair. Several of the women gathered in the courtyard had caught a glimpse of the inside and cried out. Someone had torn up the cushions of the seat and fastened to the chair a large wooded fallace. 
	
	Reina closed her eyes and fought down the bile in her stomach while Farooq, her host and Antonia’s husband prostrated himself before her, promising to flog each and every one of this guards until he found the culprit and offered to send her home in one of his own chairs, accompanied by his wife if it would please the Kokhan Mata. 
	
	Reina took a deep breath and opened her eyes. This was not the first insult, she reminded herself, and it would not be the last. This was just a shock, there had been no real danger. Moreover, her girls’ husbands were watching. She could show no fear. “Nonsense. Antonia is about you enter her sequestration. She is in no fit state to travel. Clean out the debris, and cover the damage with something pretty. I want the name of the culprit when you have it.”
	
	She watched the husbands huddle angrily around their personal staff of eunuch guards for their wives and children, while the women calmly divided into groups to watch the children and check each other’s carriages for similar vandalism. 
	
	Hazda stood discreetly by Reina’s shoulder. Reina squeezed her hand as if she were a drowning woman.
	
	*********************************
	
	None of the girls knew what had happened, of course, and Reina returned to the happy sound of them gossiping and getting ready for bed, each in their own way. 
	
	Quarzen consulted Kripna on which Xi traditions would best please the various peoples of the Janma basis, for she insisted that she would hold such gatherings for Kripna after she had left the capitol. Taria, insisting on sticking to Quarzen like a shadow, listened, half asleep on the older girl’s bed.
	
	Gris and Isolde, alternatingly the best of friends and the worst of rivals by habit, giggled and decorated each other as if they were both fig trees, increasingly flustering the woman trying to comb Isolde’s hair.
	
	Aa’ah alone was already in bed, breast and buttocks freshly painted with her mosquito net drawn, conspicuous in her stillness. Reina went over to her and untucked the netting to put her pale hand on the girl’s light brown arm. Aa’ah had been bought from the slave markets in distant Pinyeh, one of the few port cities in Xi Sardona had access to before the empire had subdued the Janma basin, by their host. He had offered her to Reina as part of his bride price for Antoinia. There were a miriad of thoughts that could have been bothering the girl tonight. Reina waited.
	
	“If I marry, Malu, will you bring girls to my children’s namings like” she waved a hand over her naked painted body, “this?” 
	
	Aa’ah people were the Kumarands, whos kingdom was not on any Sardonan maps. Rumor had it that it had taken the merchant a full year to find the Kumerands and secure the entourage of tutors and staff that are requisite parts of an adoption tribute. Accepting Aa’ah had possibly been a mistake. Sardona had no interest or relations in the Kumarandi, and Reina knew very little of the girls’ heritage other than what her tutors taught her. Aa’ah felt isolated, she knew. Furthermore, the nobility of the Kumarandi did not clothe their women. Only those who labored on the land wore protective fabrics. Noble women dressed themselves in paint and beads, showing off the unscarred perfection of all their skin. She would be a hard one to wed in Riosk, where women did not appear in public without a heavy veil and surrounded by eunuchs. But Farooq had flattered her, and Reina had given in. And now, here was Aa’ah. “If you can convince your husband to honor your ways, pea pod, I will bring my girls to your celebrations in any way you wish.”
	
	Aa’ah’s high cheekbones blushed. “And if I can’t convince him?” 
	
	Reina gave her a long considering look. “Is there someone in particular you are worried about?” Aa’ah shook her head so vehemently as to immediately give away the truth of the matter. Reina sat with her a long time, just holding the quiet girl’s hand. She was lonely Reina guessed. How could she not be. Raina knew what it was like to come to the Saran court against her will, into a harem that was already filled with its own patterns and intrigues, neither sharing gods or language or any cultural custom with the women she would now call her sisters and family. The girls were kind to Aa’ah, Hazda made certain of that. But just kindness what not enough. “I will find you a husband who will honor your ways,” she promised eventually, and wondered if she could keep it.
	
	She went to her room where Elanya waited, as it was her turn to undress and serve. Elanya was the youngest of her adopted daughters, and she had only entered her care last season. While tall for her age, she was a girl of only ten, yet she almost stood eye to eye with Reina. Hazda stood beside her, still in her outer garments, plump and round and firm as any efficient house keeper. She held three messages in her hand. She looked worried.
	
	“Is there news about the …?” Reina asked. Hazda shook her head and handed her the messages. 
	
	The first was addressed to her, in the Regent’s hand. He was writing to inform her that he had returned early from his inspections of the ports of Tabrisi and Sha’ham. He awaited the pleasure of her company at her convenience.
	
	The second was a formal invitation to the victory parade to be held in honor of Senator Karim’s suppression of the revolts in the Janma Basin, to be held on the occasion of his return to the capitol. 
	
	The third was a request for an audience at the Kokhan Mata’s earliest convenience from the Bir of Hasti.
	
	“An urgent request for an audience and two early returns. Coincidence?” 
	
	Hazda shook her head. “Or I’m a duck. Shall I make the arrangements?” 
	
	Reina nodded, put a hand out to stop Hazda from leaving. “It is late, and I am shattered. You see what tomorrow brings. I must sleep. Would you bring me some milk?”
	
	Hazda gently and firmly removed her hand. “It is late, and we do not what dangers tomorrow brings. You must face them sober.” And then she left.
	
	Rage boiled inside Reina. How dare Hazda deny her her opium on this of all nights, leaving her with only Elanya for comfort. The newest and greenest and most stubbornly incompetent of all the girls she had ever taken in. Hazda was her servant. She had come to her as a slave. How dare she dictate what she could or could not do. 
	
	Elanya chose this unfortunate moment to sneeze. Reina reeled on her. “Undress me!”
	
	Elanya blanched but obeyed, frightened and too new to be familiar with this mood. She moved too slowly, and Reina ripped the rest of the night shift and threw it on the ground. “We aren’t removing the five points of pleasure tonight” she declared and pushed the child to kneel with her face into Reina’s bush.
	
	**************************************
	
	The winter dream plagued her again. As it had during the early days of Antonia’s tutelage. It had nearly driven her mad then with grief and lack of sleep. Hazda had held her hand for countless dark hours listening to the nightingale or parrot and not a goldfinch or blue jay. 
	
	Her daughter reigned in Astria. She was well, and loved her Mrita, as far as she new, but they had developed a habit of not writing during a time now past when personal correspondences would be delivered with seals obviously broken and contents read. Over the years, the regime had changed, Astria had become a tributary and they both had secured their power, but both women knew the Regent well. They had lived with him and under him for long enough not to trust that either a letter sent either by courier or in one of the regular trains of tribute to Riosk would reach the other unread. 
	
	She knew that Illyn had made her a grandmother four times in the last decade and that Astria had two bright princes. That much everyone knew. She wanted to learn more, and it grieved her deeply that she could not.
	
	She had been on the verge of sending Antonia back to her parents, with enough gifts to convince them that the girl had not brought the family shame when the nightmares finally ended. But last night, Antonia had served her her favorite dishes. Her favorite childhood dishes. They brought to mind the grand mountain hall where she had first tasted apples preserved in rose water. And that reminded her not of Astria and her daughter, but of the life she had had with her daughter’s father. And that she did not care to remember.
	
	When it snowed in the mountains, everyone but the extremely foolhardy left for the city. But her husband was nothing if not foolhardy. He enjoyed the solitude, and she chose to remain in his company. 
	
	She was reading in a small library next to  a fire built in a hearth deeper than a man stood tall. She could smell the pine and spruce and fir resins perfume the room. Proper mountain woods, for a proper winter day. She was warm, and the furs she sat on smelt of home. She couldn’t remember if she didn’t have children yet, or they had grown to leave her all the time in the world to read at her leasure. Or maybe, in the logic of dreams, both. What mattered was the warmth of the fire and pleasure of turning thick parchment between her fingers. She laughed out loud for pleasure.
	
	The snow had fallen so deeply this year that they had abandoned their downstairs lodgings and moved everything upstairs. Her snowshoes and cloak were in the next room should she decide to stretch her legs or count the winter stars. 
	
	She remembered napping and waking to hear her husband’s voice in the wind outside. It comforted her to know that he was nearby. Idly, she wondered why she had not seen him at all this season. She heard his voice again, clearer this time, and calling for help. The snow shoes were on her feet but she couldn’t get the straps untangled to get them on properly. Her husband cried again, louder now, in pain. She left the straps as they were and jump stumbled to the library window that doubled as a door. She pulled and pulled, but it would not open. She put both hands on the handle and pulled again until she found herself drenched in sweat and awake. 
	
	She lay in her bed disoriented and unsure of where she was, if not in her mountain home. There was a child snoring in the bed beside her, but she did not think the child was hers. She went to get out of bed and found her way blocked by a spidery substance. She moved it aside without thinking, lit an oil lamp from muscle memory and rushed out of the room. 
	
	She found herself in a long tiled hallway containing six or eight beds. The Kokhani harem came to mind, but she shook it away, that wasn’t right either. She walked slowly past each pair of beds, three strides to a length. She knew the names of the girls. Kripna then Quarzen and Aa’ah. Gris and Isolde. And Taria, her daughter. Elanya, she remembered in a flash. That was the girl in her bed. 
	
	But she could not remember her name. A name came to her from her dream, but that made no sense. That person was dead, it couldn’t be her.
	
	She ran barefoot through the hallways, lost as if in a dream maze until she came to a large high vaulted room with a balcony on the other side. She went to the balcony to escape the company of the sleeping bodies, leant her head against the cool stone railing and wept. 
	
	The breeze brought with it the burble of water and the rustle of oak leaves. She remembered the garden. She was the Kokhan Mata, the empress mother of Sardona, and below her was the Kokhan Mata’s garden. She remembered her garden. She had started it when she had remarried after the old Kokhan’s death. It was a map of the empire, each province with its major rivers, lakes and mountain ranges miniaturized into streams, pools and mounds, carefully manicured with local flora, and, when possible, fauna. The Riosk court had seen it as yet another one of her hedonistic vanity projects and an erosion of the Saran traditions under the regency. But as she continued to add to it with each new conquest and victory, some began to walk its paths patriotically. 
	
	She startled as a hand brushed against her back, draping the thinnest of silk wraps over her shoulder to cover her nakedness. “You are awake, Maya.” 
	
	Her grief disappeared as quickly as a flame doused in water, replaced by a reflexive authority. “So are you.” She tried to remove the cloth from her torso, but Kripna silently stepped forward and put her arms around her, holding the fabric in place. 
	
	The Kokhan Mata struggled irritably “I am too many times a mother. It is the middle of the night, and the sky is clouded over. There is nothing to see.”
	
	Kripna just held her. Silently ignoring the argument. Gently insisting on her infuriatingly reasonable point. Distantly, the older woman marveled at the young woman’s ability to sooth and coerce simultaneously, and at her own complete inability to rage against the equanimity. She would clearly handle a husband well.  Suddenly exhausted and out of options, she stood there in silence for a long moment, Kripna just holding her until the Kokhan Mata’s disorientation slowly dissolved into a raw ennui.
	
	“I heard that the Bir of Hasti has arrived.” Kripna said at last. The Kokhan Mata nodded. “Do you know what he wants?”
	
	“Probably a report of Senator Karim’s activities. He will tell me in the morning. Then he will certainly want to see you. Is that what keeps you awake?” The older woman looked up to peer at the girl’s face, but could not see anything in the darkness. 
	
	“Yes, and … “ Kripna sucked in her breath. “I heard you crying. Are you having trouble sleeping again?”
	
	The Kokhan Mata nodded, uncertain if the admission was going too far. 
	
	“Let me help you, Maya,” the girl said. When the older woman said nothing, Kirpna bent down and kissed her firmly on the mouth. 
	
	The Kokhan Mata felt a safe familiar swell of desire. Not for this body in particular but from the knowledge that after the crest of an orgasm would come a deep healing sleep that she desperately needed. She reached for Kripna’s hand and moved them to cup her breast desperate for a comfort that could not come quickly enough. The girl was skilled, and the older woman quickly found herself led to a quiet private corner where she could wait for the pleasure, the sleep to take her. She just had to let her body relax, and her partner would do the rest.
	
	***********************
	
	Dawn. She sat in the gazebo by an irrigation pipe laid out to resemble the source of the Janma river. The gazebo pointedly did not resemble the temple that stood at the real mouth.
	
	“The Bir of Hasti,” announced one of the eunuchs serving as her minder. 
	
	A tall young man dismounted from the sedan chair she had sent for the old Bir’s comfort, and ascended the gazebo stairs with brisk eager steps. She startled at the unexpected figure attached to the expected title, but she was heavily veiled. The man kneeling before her had not noticed. “Honored Kokhan Mata. May your reign be as pure as the sun’s,” he said in heavily accented Saran. 
	
	“Maya Janma, bring prosperity to the Great Bir,” she replied in the river language. Each of her girls were tutored in the language and customs of their homeland by an entourage of tutors that accompanied them to Riosk. This meant that every member of the harem, servants, Taria, and herself, knew the language and the cultures of all the current occupants. She would great the Hasti ambassador in his language, in his goddess’s name, against all Saran custom. The empire spoke in the empire’s voice, representing the power of the empire’s gods.
	
	The young man at her feet startled at the sound of his native language. “Walk with me,” she continued. “You must be Adya. My condolences for your loss.” 
	
	“Thank you,” he said, rising, and in Saran. “My father died in his sleep. He was in pain before. Now he rests with Her.” Then he looked nervously behind him at the guards and asked “They will not mind?”
	
	As every Sardonan noble woman knew, the eunuch were creatures for their husbands to command. Of course there were tales of the odd wife who had bribed or scared her entourage of minders and body guards into keeping her secrets or to plot against her husband. Every traveling group of players or historical poet had their favorite of the exaggerated tales for the asking. But Reina never knew of any woman having actually had such luck and neither did any of the women she knew. She suspected that the tales were just that, stories. 
	
	The old Kokhan’s harem had suffered harshly under his eunuchs, and Reina had feared them greatly then. But the old Kokhan was dead. She knew who her guards reported to, and what secrets that man had known about her and her co-wives in spite of the old Kokhan’s overprotective tattles. Anything her guards could tell her current husband, he likely already knew. In this marriage, the eunuchs were completely robbed of their power.
	
	She shrugged and laughed. “Oh. I’m sure they will mind. They may even tell the Regent Kokhan. But it doesn’t matter. He likely knows already.” When Adya Bir still looked frightened, she gestured for him to follow. On the other side of the gazebo, precisely over the point where a pumping mechanism burbled water into the irrigation pipe below their feet stood a statue of a young maiden carrying water. It was clearly not the statue of Janma that stood where the river poured forth from a snowmelt pond on the eponymous mountain. But it was also not so different from how an artist might depict the goddess if he so chose. 
	
	The first rays of sun rose over the city wall and lit up the statue's face, much as it would have at the temple on the equinox. Adya Bir paled, then moved to pray, stopped himself, looked over his shoulder at the guards, and blushed. The Kokhan Mata beamed inside, but forced herself to maintain a mask of composure as the young Bir recovered. 
	
	“You honor us,” he finally said, low, in his native tongue.
	
	“You are welcome,” she replied, and allowed her smile to enter her voice. “We are all Sardona, and this garden is for all of us.” She did not know what this young man wanted. All of her suppositions had been upturned by the news of the death of the Jaya Bir. She needed to start from a position of neutrality.
	
	“I had heard about the wonders of this garden, but I had never imagined….” he looked around him, his initial business-like manner dissipating into a new found, almost boyish, amazement.
	
	Good, she thought. Keep him in your thrall until you know what he wants. Outloud she said “Walk with me.” He followed.
	
	They traveled northwest, down the Janma range and up the Phalut range to the gazebo that represented the four monasteries of the oracles of the Golub pantheon. From there, they followed the ridge east to the pillar on the westernmost point of the continent, where Sur, the sun, had wed Uhr and Saguhr, earth and sea, to be husband and wife. Only then did they turn south to take the winding coastal path to the glittering model of Riosk.
	
	Somewhere near the seal pen that represented the Port city of Skurv, Adya Bir found his tongue. “I wanted to speak to you before the parade.” She nodded, but refrained from turning to look at him. He hesitated, before continuing, “I wanted you to hear from me before Senator Karim told you his side of the story.” 
	
	She forced her laughter to sound light. “Then you could have waited. The Senator will tell me nothing. I received my news from the Regent Kokhan.” 
	
	The young Bir startled for a second time at the mention of the title. That was interesting. “Then you have no word of the troops on the Janma?”
	
	“As Senator Karim’s barge arrived yesterday afternoon, and the Regent Kokhan last night, neither of us have had the honor of hosting the Senator for lunch.”
	
	This confused him. The Hasti lived in a far more temperate climate that Riosk. Their days did not revolve around the midday meal as it did in the capitol. But he continued. “They have built camp just north of the Paula wetlands.” 
	
	The Kokhan Mata knew where these wetlands were. They had been an expensive feature in her garden. The real wetlands, with its temperate climate, was a hotbed of mosquitos. In Riosk, well, there was no need to invite more of those pests to an already infested city. She had settled for a handful of reed filled pools instead. Senator Karim had not set camp there foolishly, however. The Janma slowed as it flowed through the swamp, making those miles of river a prime spot for the raiders attacking barges going to or from Xi. Building so close to the swamp was a necessary, if uncomfortable position. “I do not see the problem,” she responded. “If the mosquitos have food close at hand, they will not bother your children.”
	
	The young man looked frustrated at his inability to explain the problem. “It is not the mosquitos, Kokhan Mata. The wetlands are a watering ground for several of the herds the Frit follow. With the wetlands disturbed, these herds are crossing the river into Hasti lands. The Frit cannot follow these herds across the river by the treaty signed by my father, but with his passing and this new watering pattern, Punde Bir does not see a reason to renew it with me. The Frit have accused us of unfairly culling their herds, which was not allowed prior to the treaty, and was the cause of many wars between the Hasti and Frit in my grandfather’s day.” He trailed off, lost in the web of negotiations he was trying to juggle. 
	
	The Kokhan Mata sifted through what she knew of the history of the peoples of the Janma basin. Many of them were migratory peoples, following herds over the basin from region to region. Peace between the peoples depended strongly on the stability of the migratory paths. The senator’s choice of troop placement was problematic indeed. 
	
	The Kokhan Mata pitied him. Receiving the scepter under occupation was hard enough. Receiving it while the occupier was ignorantly disrupting the status quo was a different case altogether. But she did not see what she could do. “I presume you have spoken to the senator about this?”
	
	The young man flushed. “I petitioned for him to mediate between Punde Bir and myself. I did not have to. I could have laid out the Frits' threats to invade our territories to the Senator and demanded my rights as a protectorate. I wanted to negotiate.” Of course he did. He was obviously green. Not only that, some of the Janma basins, including the Hasti, had become more settled. They still followed herds, but their settlements remained occupied all year long. Obviously, this practise was seen as a sign of weakness by less settled peoples. Adya Bir needed to earn the respect of his neighbors. He could not appear to hide behind Sardona’s shield.The man was completely lost. “Senator Karim informed us that if we did not stop the raids in the wetlands, he would have both our heads on display in Riosk. He called us barbarians and the Frit… well, he made his disdain for their migratory customs clear.” 
	
	Typical. Senator Karim had come to power in the old Kokhan court. Its principles were simple. The best way to prove one’s own greatness was to expand Sardona’s border. The best way to prove Sur’s greatness was to increase the believers in The Four. The best way to demonstrate Sadona’s greatness was to spread her ways. For the old court, the best diplomacy happened at the tip of a spear.
	
	As much as she disliked the attitude, there was nothing she could do. “But your problem is solved. The Frit will not invade your lands with the Senator’s iron hand so near. I still do not see what you want from me.”
	
	Adya Bir bit his lip, searching for words. “I … I do not want my family name to end in ignominy.”
	
	Suddenly, she understood. Elanya. In order for Adya Bir to secure peace for his people during this period of transition. He could not ask for Sardona for help to pressure the Frit to be less aggressive. Already inclined to think the Hasti cowardly, his future reign would forever be marred by the fact that he had needed to ask the empire for help. He might have saved face by asking the empire to negotiate neutrally, but Senator Karim had shown no interest in performing that duty. So the only action left to him was marriage. A union with a Frit noble woman would not renew the treaty Adya Bir’s father had created, but it would force a non-aggression pact for the duration of the marriage. Punde Bir would not give up a daughter for that cause, and he may have instructed his court to refuse as well, but no Frit could prevent Elanya’s marriage to the Hasti. 
	
	It was not a bad plan except for the fact that it was an absolute disaster. Marrying out a girl as untrained as Elanya would be devastating for the reputation of her daughters. Possibly more importantly, she harbored many of the same Frit prejudices against the Hasti that lay of the root of this entire matter. “Elanya is not ready for marriage” she said coolly.
	
	“I completely understand,” the young man hurried to assure her. “I know that you do not let your daughters marry until they have been women for several years. I am asking for an engagement.” He knelt before her. “Kokhan Mata, it would mean so much to me.”
	She stared down at him for a long moment. The man had clearly mistaken her meaning, but there could be something to this. With Jaya Bir’s passing, Reina would have to find a new engagement for Kripna, and quickly. Kripna’s engagement to the old Bir had always been a suboptimal solution. She urgently needed someone she could trust to maintain a trade route the Xi that benefitted the Regent’s allies. But Punde Bir was married, and his sons too young. Jaya Bir had insisted on taking on Kripna as a third wife rather than marrying his son, so Reina had had to settle. If Kripna married the young ruler, she could have a dependable set of hands in the region for decades, as well as a pair of ears among Senator Karim’s troops.
	
	Reina sized up Adya Bir mentally. He seemed a cautious man, with a propensity for asking for help rather than forging ahead on the strength of his own resources. Kripna could do a lot with a husband like tha. “Very well,” she replied, making certain that her tone indicated that she was doing him a favor. “You may meet her. But I tell you, she is not ready. I will bring along others for you to meet as well, so you may have some options.”
	
	***************
	
	The girls had already had breakfast by the time Reina returned to the harem, and she was informed that Gris’ cook awaited her pleasure. Pickled fish, then, she thought and shrugged. It was definitely an acquired taste, and it had yet to grow on her. The Clave were an island people,  who lived deep in the southern sea. They were wealth traders, providing Sardona with everything from pearls to chocolate to fine wool fabric softer than the best cotton. Very little of what they traded did they actually produce. The Clavic traders had stories of innumerable lands “elsewhere” on the southern sea. The Clave were not a tributary or protectorate of Sardona. They were a powerful geo political player, and Reina paid dearly to have adopted one of their girls at all times. Sardona, she knew, was interested in stealing their ship building technology, as they were the only people Sardona knew of who could sail the vast ocean reliably. Supervising these efforts had been the reason the Regent had given her for his prolonged visit to the ports of Tabrisi and Sha’ham. 
	
	She passed through the lower verandas on her way to her morning meal. The girls sat at a variety of activities today, supervised individually or in pairs. It was the usual rhythm of study on days where the regular roster of classes was interrupted by matters of state. In a few hours, they would leave for the Senator’s victory parade. 
	
	Isolde, she saw, sat with her philosophy teacher studying the logic of lines and angles. The Ursk girls always took exceptionally well to learning, Reina had found. Possibly a feature of coming from a province with so many monasteries. Isolde was the first of her girls to want to be numerate. Reina encouraged the fascination, though she knew from experience that a fluency with numbers would earn her the suspicion of her husband’s senechal. Reina had initially encouraged the girl because it gave her a chance to occasionally eavesdrop on Isolde’s lessons. Her own education on the subject had been piecemeal and haphazard at best, and it was so rare for her to indulge for leisure. Sadly, Isolde had passed Reina’s skills several months ago, and it became clear to her that she was more of a hindrance than a support when she worked with her on her assignments. 
	
	Kripna was working with Aa’ah on the history of the empire, pointing out when why and how different territories became part of the empire. Aa’ah’s language tutor sat close at hand correcting Kripna’s grammar or pronunciation as needed. 
	
	When Reina walked past the pair, Aa’ah grabbed her hand. “Malu, what are you going to put in the garden for today’s parade?” 
	
	Part of her concession to the Regent for the funding of her garden was that she would commemorate each victory parade with an appropriate monument. She had not given today’s affair much thought. A strong proponent of the ways of the old Kokhan, Senator Karim did not hold much favor with the Regent. Nor did he hold any respect for her. “I have not yet decided, pumpkin seed. Something to signify the trade route to Xi, I think. Kripna, what could I put at the conjunction of the Janma and the Oniyata that would please all concerned?”
	
	It was not conventional to consult the conquered on a monument to commemorate a victory. But Kripna would choose wisely. If it bothered Senator Karim, then it bothered Senator Karim.
	
	Gris sat with an embroidery needle, starting what looked like a swaddling blanket. Letters in the Xi alphabet had been etched in charcoal on the white background. Antonia’s child, Reina guessed, and leant over to read what the inscription would be. Immediately, Gris hunched over her handiwork and declared it to be a secret. Behind her, Taria giggled and sang “Quarzen told me. It says ‘Ma–’ ” The Clavic girl lunged for the Kokhanati’s ankles, and Taria squealed, and ran to hide behind Quarzen, upsetting a bottle of ocre across the canvas that she was working on. 
	
	The three girls met like gulls over a fish head. A week’s work had been ruined, a confidence had been broken, and an innocent attempt at having fun had been trampled on. Any and all of this was reason enough to cry or throw a paint brush or tear a button off a dress. Reina laughed and shook her head as tutors stepped in to clean up the mess and separate the girls. Reina had spent four years as the youngest wife in the old Kokhan’s harem, where the eldest wife was described as a she bear and the fourth wife a cobra. She had known fear and jealousy and the dangers of being the only wife who had borne the Kokhan a son. It was refreshing to see girls fight simply because they were girls. Refreshing and as it should be, she thought, but gave the brawl a wide berth.
	
	She found Elanya sitting with Hazda on a shady patch of grass between the harem and the kitchen where Gris’s cook waited to serve Reina pickled fish. Elanya was carefully looking through the previous night’s dresses for tears or stains and snags, while Hazda performed the repairs. Reina had not intended to eavesdrop on the conversation, but paused when she heard her name.
	
	“Those who call this place the Kokhani’s harem and not your friends, or hers.”
	
	“But that’s how she treats us. She walks around, pretending to be a man, and makes us” Elanya dropped her voice to a whisper “you know .. do what a wife does to a husband.” 
	
	The words stung. She knew what the court at Riosk thought of her, and what they said behind her back. Unwoman, witch, succubus. The loudest voices came from the Regent’s most avid enemies, but as last night’s vandalism showed, the danger could come from anywhere. Even from the mouths of her own daughters.
	
	Hazda shook her head at the girl. “Where do you hear such things?”
	
	“I’m not stupid. I saw her taking private meetings with the men last night. And she’s always talking to diplomats and senators. She acts like she thinks she’s the Kokhan herself.”
	
	“Child, last week, when Igrit asked for your help with Antonia’s sending off, did you ask anyone’s permission?”
	
	“No, but that is because you’ve told us not to. At home, I would have. It’s exactly what I should have done.” 
	
	“You aren’t at home anymore. Here, when you find an opportunity lying unattended on the street, you take it. You don’t point it out to someone else so that they can take it for you.”
	
	“But that's wrong,” Elanya said, sounding like she was trying not to cry at losing an argument. “You don’t just take things that aren’t yours. Women don’t force women to love them.” 
	
	“No one is forcing you to love your Maya. Whether or not you do, is completely your choice.”
	
	“Good! Then I don’t love her. I hate her, and I’m never going into her bedroom again.”
	
	
	“That is not what I said, girl. A wife doesn’t join a harem because she loves her husband. She does so because of the opportunities he provides.” Elanya pouted and crossed her arms, refusing to pass over the next garment.
	
	Hazda put down her sewing. Reina watched her transition from the matron who watched over to her girls to the friend and companion she knew her to be. She looked Elanya in the eyes. “You aren’t the first person who has been forced to pleasure her,” Hazda said, somewhere between bitter and matter of fact. Elanya’s eyes widened, and Reina bit a knuckle. “I’ve seen a few more things than you have, would you agree, child?” Elanya nodded. “Then let me tell you, she’s better than many options out there. Even for one of your station.” 
	
	Reina ducked back into the harem building and shut her eyes, trying to figure out what to do with this unexpected confession. Outside, Elanya protested, and Hazda, again the matron in charge of their schedules and safety, gave her no quarter. Reina loved Hazda. She depended on her, certainly, but she had always loved Hazda. She hadn’t ever forced the woman to her bed, had she? The first time they had lived together, Hazda had been the strong one, and she had barely been more than a pile of discarded flesh. She could not have possibly forced Hazda into anything then. And now, she could not possibly be talking of now. That was not how Hazda served her now. She did not understand the hurt that she had heard in Hazda’s words. But she had heard a real hurt in friends’ voice. It burned.
	
	*******************
	
	When Adya Bir entered the Kokhan Mata’s box at the victory parade, Kripna greeted him with a cup of cold tamarind water, in classic Kumarandi style. She was, of course, covered in a veil and abaya as they all had to be in Riosk in public. But her black abaya was richly embroidered in a darker shade of black with peacocks and cherry blossoms in ways that drew the eye to her most pleasing features. She relieved him of his hat and the offerings he had brought for the prospective brides with a confidence that seemed to come from years of welcoming guests to her husband’s house. As she reached up to put away his belonging, her sleeve fell back, revealing an arm that had been delicately painted for the occasion by Aa’ah.
	
	Adya Bir accepted her welcome with a good deal of appreciation, and looked out at the crowd from the front of the box. She had chosen this box for their meeting particularly for the view it provided. Adya Bir stood at the edge of the box admiring the pomp passing below him. The box stood barely 10 feet directly above the parade route, if he leant over the railing, he could see the faces of every soldier who passed below him, possibly reach out and touch the tops of the spears. “I thank you for the invitation, Kokhan Mata. But is this wise?”
	
	Reina sat shelling pomegranate seeds, carefully wrapped in an air of indifferent laziness. Sitting in an open box like this always put her on edge. She was flouting Saran norms both by sitting in an unscreened box and hosting a man not related to her, but she could not host a dignitary in a woman’s box. That would be unfair to her daughters. The Kokhan Mata went around Riosk pretending to be a man. She shoved the sentiment away and offered her guest a forest green ceramic bowl piled high with a pyramid of glistening red seeds. 
	
	“Wise, Adya Bir? To host you in my box?” Elanya interrupted with a tray of honeyed nuts, fried stuffed flatbreads and a bowl of shredded goat meat dry cooked in cayenne pepper. She placed them before Reina and the Bir and left without saying a word. Reina smiled behind her veil. She could just let the girl be herself and Kripna would win the man’s heart. “Your father was the signatory to the treaty that made you a protectorate of Sardona. It would be shameful not to honor you during your visit to Riosk.”
	
	That was more flattery than truth, and they both knew it. Senator Karim would be incensed to see the Hasti Bir being hosted by her, even more so if the stories of his insulting the young man when he had asked for help were true. But insulting the Senator had been the point. 
	
	“And truly, I am honored, Kokhan Mata. When I had asked to meet the beautiful Elanya, I had not dreamt that it would be under such exalted conditions. But,” he looked around at the crowd below him. He could almost count the stitches on the uniforms passing below him, “We are so exposed. Senator Karim will see me.” 
	
	“My dear friend, so what if he does. You have accomplished your mission and secured my support before the Senator could present his side of the story. If you want him to respect you, let him see that you have won.”
	
	“Your support, Kokhan Mata?” 
	
	Kripna and Elanya stepped behind Reina and the Hasti Bir with large paper fans to keep them cool in the late morning heat. Elanya fidgeted nervously with the red and gold tassels that decorated the frame of the fan. She wanted to be with the rest of the girls in the traditional screened box reserved for Riosk’s noble women, eating figs stuffed with goat cheese and giggling with her friends. She did not want to be serving a stranger, a Hasti, in an open box where the entire city could stare at her. Kripna, on the other hand, glided into place beside Adya Bir as if she were already the mistress of his household, smelling faintly of jasmine. Reina could see the outlines of a set of glass bracelets on Kripna’s fan hand under the sleeve of her abaya. They tinkled pleasantly for the young man as she fanned.
	
	“You came to me asking for an alliance, did you not?”
	
	“An alliance with the Frit, by way of an engagement to Elanya, yes,” Adya Bir asked, starting to understand how much he did not understand.
	
	“A marriage to one of my girls is an alliance with me. They have been separated from their families since they were nine or ten years old. Their families may choose to ally with you after the marriage, or they may not. I place no expectation on them in that regard.” She gestured to the girls. “You may have your pick, either of these would serve you well.”
	
	“An alliance,” the young man said again, slightly fearful this time. “I cannot afford to offer the empire a price for my bride.” 
	
	Reina tsked. “I am neither the Kokhan, nor his regent.” She waved to a box across the boulevard where a line of drummers passed below, beating out a complicated rhythm while a trained bear stomped and turned in time behind them. That box was not much larger than the one Reina occupied, but it held considerably more people. The Kokhanat sat on a raised dias, cross legged on a rich red carpet. The eleven year old boy was currently fascinated by the cleverly trained bear. But when that ceased to hold his attention, he would return to staring at her with a murderous gaze, as he had been intermittently throughout the parade. Next to him stood a huddle of men, generals and Senators loosely allied with Senator Karim, the Regent Kokhan in the center. Place settings and food had been prepared for all of them, but the knot of men were too distracted by their conversation to notice either the hospitality or the dancing bear. “I am the Kokhan Mata. An alliance with me is not an alliance with the empire.” She watched Adya Bir absorb the Kokhanat’s look of disdain, and the backs of the three men consulting with the Regent. Adya Bir was not a stupid man. He would read the disrespect in the poses and the disregard of the Regent. If he were a clever man, he would realize that neither the Regent nor the Kokhanat, nor any of Senator Karim’s allies had offered to host him, the leader of a protectorate state, during an important state affair. 
	
	Reina waited. Her stomach clenched. She took in a breath and forced herself to count to four before exhaling. Waiting was part of the design she was weaving. She needed to give the man before her time to consider his future. He was about to cast his lot either with an empire that wanted to subsume him, or a woman, who had a very unconventional vision for Sardona’s future. They both knew what he had to lose with the empire. Had she made it clear what he had to gain with her? All of her alumni’s husbands, both in and out of Riosk had chosen her over the Kokhanat. But many men she had sought out as potential husbands for her girls had not made that decision. A tight hard knot formed in the pit of her stomach. Waiting for judgment physically hurt.
	
	A bright red tassel fell from Kripna’s fan into Adya Bir’s lap. She stammered and begged the Bir’s forgiveness, interrupting his thoughts. The Bir blushed and handed it back. Their fingers touched over the small red bundle of silk and gold thread. Within moments, Kripna, and Adya Bir were discussing the people he had met in the city, and monuments that he should make certain to see during his stay in the capitol. 
	
	As Kripna’s skilled conversation drew Adya Bir closer to the point of choosing Reina over the old Kokhan’s faction, forced herself to breathe evenly. This was a battle ground she had chosen, she reminded herself. All the stars had aligned, and she had chosen to play this game to because it was the most likely to advance her goals. 
	
	The reassurance helped. What helped more was the realization that beside her, Elanya grew increasingly agitated.  Reina could see her paying more attention to Kripna’s conversation to Adya Bir than to her own duties. Every time Kripna asked for the young girl’s opinion, or Adya Bir asked after her family or her interests, Elanya answered monosyllabically and swung her fan more and more erratically, hitting Reina’s shoulder with the corner of the frame more often than not. Reina could not see Elanya’s sulk behind the veil, but she was certain it was there. The smaller scheme was working, as there had been no doubt that it would. The demands of minding and correcting her daughters’ behavior in public kept her momentarily distracted from the larger political issues at play.
	
	Eventually, the trumpets announcing the Senator and his honor guard blew. A line of four white horses slow gaited in unison down the boulevard, their manes and tails braided with red and gold thread. Behind him, came an elephant. Atop the elephant, in a shaded box no bigger than a single seating sedan chair stood Senator Karim, waving to the crowd, accepting their adulation and catching the occasional well aimed garland of flowers. He had served the old Kokhan for nearly a decade as a general. He was a veteran of victory parades. He played his role well.
	
	He passed the Kokhanat’s box and bowed deeply to the prince. The elephant came to a halt and the horses before it switched to trotting in tight formation in a circle spanning only half the width of the boulevard. The cluster of men around the regent stopped their conversation to convey their congratulations and thanks to the Senator. A man came out of the box, cut across the crowd and climbed up the elephant to pin a new medal onto the Senator’s cloak. The Senator thanked the Regent Kokhan for the recognition with appropriately florid words, and expressed his desire to continue to serve the empire in the years to come. When the Kokhanat invited him to enjoy the rest of the festivities from the comfort of his box, he accepted graciously and with genuine pleasure.
	
	With the ceremony before the head of the empire concluded, while the horses lined up to continue their perfectly choreographed march, Senator Karim turned with a magnanimous smile to face the occupants of the opposite box. To his credit, Adya Bir rose with Reina and saluted. Senator Karim’s face fell with the devastation of a cliff side in a mudslide. Reina raised her arms in gratitude and forced them steady in spite of her nerves. “My thanks as well, Senator. Your bravery has gifted both Sardona and our Hasti neighbors a lasting peace and many years of profitable trade with Xi.” 
	
	The Senator turned ashen with rage. Before Adya Bir could concur with the Kokhan Mata’s declaration, the horses had started moving, and the elephant lumbered away from the box.
	
	Reina grinned behind her veil. Then she turned to face the Hasti Bir. 
	
	The young man looked like he had just realized that his hands were capable of grasping things. “An alliance,” he said carefully. “An alliance with the Kokhan Mata would be very desirable indeed.” 
	
	Reina was pleased, but she kept it out of her voice. “I am glad, respected Bir. I look forward to discussing our future together soon, perhaps over lunch?” That would please the Bir very well. “Then if you will excuse me, I have a few other matters to deal with. I trust that Kripna will be able to see to your comforts?” 
	
	When Adya Bir blushed, she knew that she had accomplished everything she had set out to do this morning. The tension finally draining from her body left her feeling slightly weak in the knees. She hurried out of the box, taking Elanya with her.
	
	********************
	
	“Where are we going?” Elanya asked as they climbed down the rickety stairs leading up and down from the box. 
	
	“The women’s box, of course. Where else would we be going?” She needed to talk to Hazda. Kripna and Adya Bir were in public, guarded closely by a pair of eunuchs. There was nothing improper about a courting pair meeting, supervised, in public. She felt a twinge of guilt at leaving her guests side so suddenly, but she needed to talk to Hazda. In the span of an hour, she had humiliated Senator Karim, secured an alliance with the Hasti, who were well positioned to be important to Sardona as the trade route to Xi developed. And she had married off Kripna. She was giddy with excitement and still dizzy with terror. She wanted to lay it all before Hazda like an array of delicately arranged dishes of food and have her calmly and cooly help her evaluate her handiwork, before she burst from the pressure of it all.
	
	“Why?” No longer on display in a prominent parade side box, the Elanya seemed to have found her tongue. “You said that the Bir of Hasti wanted to talk to ME. Why are you taking me away from him? I’ve barely had a chance to say a dozen words to him.”
	
	Reina stopped short, the urgency of her need dampened, overtaken by her need to correct, teach and care for this child. “You were a disaster in there, Elanya,” she said calmly “I’m not taking you away from him as much as I am saving you from embarrassment.” 
	
	“You aren’t saving me from anything,” the girl yelled. She was so angry that she was spitting. “You wanted me to fail. That’s why you had me competing against Kripna. You didn’t want your Hasti guest to get to know me, you just wanted to humiliate me.” 
	
	Reina stepped in so close to Elanya that the girl faltered in her rant. “You are making a scene,” she said low and stern. Elanya trembled with rage. “Pull yourself together and follow me.” 
	
	“I hate you” Elanya said, but obeyed.
	
	Inside the larger box from where her daughter’s watched the morning’s proceedings, it felt like it took forever to get the five girls and their attendants properly cleaned, groomed and dressed to enjoy the fete that followed a victory parade. Reina had never seen it take so long for a group of girls, already attired for being outside of the harem to clean up, don veils and leave. Isolde still had some tamarind juice to finish, and Quarzen and Gris had put on each other’s veils. Aa’ah had smudged the delicate artwork on her wrists while eating, and no one could do it right but Quarzen, who was too busy disentangling veils with Gris and entertaining Taria’s urgent questions about how the Sardonan breeds of horses displayed in the parade compared to the Xi breeds that the older girl had grown up with to have the time to help Aa’ahs fashion disaster.
	
	Reina had a headache by the time the eunuchs had escorted the last excited girl out of the box to visit the shopkeepers and sweet sellers that would be lining the boulevard after the last of the parade. 
	
	Hazda pulled her back until her head lay in her lap, and started rubbing Reina’s temples. Reina let herself be positioned as Hazda wanted her, and laid herself bare for anything that Hazda chose to give. Her old friend’s touch was as comfortable as a well worn glove. There was nothing they hadn’t done together once, a lifetime ago when she had been gasping for the will to live in the old Kokhan’s harem. A decade later, she was the Kokhan Mata, completely free of a harem’s politics or any duty to produce an heir, and Hazda was no longer a slave. Even now, there was nothing she would not do with her again. Reina sighed in pleasure as Hazda made the pain in her head disappear. Today, she smelled deliciously of sandalwood and rosemary, and Reina wanted to turn her head and kiss her wrists. But she refrained, remembering the overheard conversation of the morning.
	
	“How did you fare, my Kokhani Reina?” 
	
	Reina took in a long deep breath and slowly exhaled. She was suddenly so tired. “Well. It went really well today. I would tell you everything Hazda, but with all of this,” she raised her hand limply and waved it in the air, indicating the chaos that had been in the box just moments ago. “I’m just so tired.” 
	
	Hazda took her hand, kissed the back of her knuckles gently, then laid it carefully over her chest. “Then sleep, Kokhani. You can tell me later. You still have a long day ahead of you.” 
	
	Reina interlaced her fingers around Hazda’s and held her close. Milk and coffee they used to jest, back in the day when Hazda was all ambition, and Reina too pinned down by promises she had made to think that she could ever desire for anything better. They had both come such a very long way. A morning like the one she just had would have been unimaginable once.
	
	Reina closed her eyes and sunk into the cushions and Hazda’s lap. Hazda hummed absentmindedly and rubbed her back as if she were putting an infant to sleep. She was safe here, Reina knew, in her lover’s arms. Hazda would never let anything happen to her, and she was finally in a position to provide for Hazda as well. 
	
	She was almost asleep when a knock came at the door and a voice announced that the Kokhanat desired the company of the Kokhan Mata in the royal box.
	
	*******************************
	
	In principle, Senator Talal Karim was not an unlikeable man. He was intelligent and ambitious and willing to use unconventional means to get what he wanted. As a general fighting the frequent uprisings in Ursk, he had been one of the first Sardonan men to adopt the leather and chain tunics of the mountain people. Only a madman would wear armor like that in the sweltering heat of Riosk, and even in the cooler mountains, a warrior was faster, stronger and far more flexible not wearing a chain shirt than in one. For two entire seasons, the court at Riosk had thought General Karim mad. And then it became clear that he was winning more battles with fewer losses than anyone else leading troops in the north. 
	
	Senator Karim was an observant man, studying the customs and habits of his enemies so that he could turn them to his advantage. He was cultured, well read in Sardona’s history, and a collector of fine Saran ceramics. In principle, there was no reason that Reina could not have been allies with Senator Karim. 
	
	The tall muscular man stood at one end of the Kokhanat’s box, overlooking the fete below put on in his honor. Occasionally, he waved and nodded at a friend or passer by who saw him honored by the Kokhanat and saluted. 
	
	At the other end of the box, the Regent, a round plump man stood looking out over the people of Riosk as they milled about, bought trinkets and sweetmeats or played games in the fete below. Occasionally, he waved or nodded at a passer by who was worthy of his attention. 
	
	Reina’s hatred did not stem from a principled reason. Rather it was the natural byproduct of the Senator’s admiration for the previous Kokhan and everything that man had represented. The Regent and Senator Karim, on the other hand, had very good reasons to hate each other. When the old Kokhan had died eight years ago, the then general had accused the Regent of having poisoned his own half brother. General Karim had not found any proof, of course, and the Regent had inflamed the issue by forgiving him publicly for the slander. The good general was under a lot of strain at the loss of his dear friend and mentor. Who in that position may not make a mistake out of grief. The public statement had put the general in the regent’s debt. But it had not bought the Regent any favors.
	
	The air between Senator Karim and Darius ben Nebab felt as thick as quince jam. In between, the other four occupants of the box arrayed themselves as how they wished to be seen in alliance. The Kokhanat sat on his dias, directly behind the Senator. General Rahal stood to his left, entertaining the boy with adventures of his travels. Senators Al Fahaz and Khoury had their heads together conspiratorially, standing between the two, further to the rear of the box and away from public view. All men turned and bowed when Reina entered the room. 
	
	“Ah, Honored Kokhan Mata, how kind of you to join us” drawled Kokhanat Badr ben Nebab. The Kokhanat had been raised away from the harem at the tender age of three, when the old Kokhan had died. Mother and child were not close. Every time Reina saw him, he looked and sounded more and more like the father Badr could not possibly remember. Down to the sneer in his tone. Uhr protect the empire when he came of age.
	
	Reina swallowed against the butterflies in her stomach and the fluttering of her pulse. These were all men who were loyal to the old Kokhan once. They were not the types of men that she would ever approach to marry one of her girls, and, it would seem, the Regent had not yet brought them over to his side either. She swallowed again and pushed down the growing nervousness. She would not show them weakness, she reminded herself. That would only make the dogs want to give chase. “It is a pleasure to see you well, my son. May your days be as bright as the sun.” She waited for him to release her from her bow, but he didn’t. Reina wanted to shake her head in dismay. This was the old Kokhan’s way, to humiliate guests who came before him, to put them on edge, so that he could have his way. Badr was not as subtle as his father had been, but his father had had half a century’s more experience honing his cruelty when Reina had married. She had to hope that the Regent could sway this child before he inherited the crown.
	
	“I have asked you here, Mata, to answer a question for the good Senator. Gentlemen. Please. Be seated.”
	
	Reina took that as a cue to stand and take stock of her companions. Senator Karim wore a look of triumph on his face, while Senator Al Fahaz looked decidedly nervous. The Regent Kokhan wore an expression of studied neutrality as carefully placed upon his features as a freshly ironed sherwani stole. She wondered what had passed between these men that brought her here.
	
	The four senators and the regent took their places next to the Kokhanat, as if they were in court. The Regent in his rightful place to the right of the Kokhanat, the Senator in the place of honor to his right. General Rahal brought her a low stool to sit while facing the company to the left of the Kokhanat. Then he walked quickly back to his place in the second row.
	
	So there it was. As unsubtle and a drawn blade and just as dangerous. No one sat to the left of the crown seat unless they were to be imprisoned or worse. In neither the old Kokhan’s time, or the Regent’s, the position was not assigned lightly. Badr was taking out a boy’s anger on her. Senator Karim had put him up to this, she knew. 
	
	Ignoring the stool, Reina stood, positioning herself squarely before the regent, to the right of the Kokhanat. She was about to be humiliated, and she suddenly felt self conscious of the publicness of these open boxes. The entire city was at the festivities below. No. She could not let herself get distracted. Let anyone who wished to, gawp.
	
	“Mata, why have we been watching your prostitute seduce the Hasti traitor before the city for the last hour?”
	
	So that was the game. The pieces fell into place, and every one of Reina’s jittery nerves cooled into hard iron. This was an old dance, and she knew the moves well. “Badr, sweet child of mine,” she crooned, “that is certainly not the question the Senator has for me. He is a leader of our people. He would never mistake the son of the signatory of a protectorate treaty for a leader of a band of rebels.” She turned to face the Senator, “Senator, I implore you. Badr is too old for the harem, so I cannot teach him of the world. You are a pillar of the Riosk court. I know he looks up to you. Please, before my sone comes of age, I hope you will teach him the difference between an appropriate and modest public conversation between two affianced and the sullied private actions of a fallen woman.” 
	
	Senator Karim stood up in disgust. “I am not playing games, Kokhan Mata. Riosk will no longer tolerate your flaunting our every custom and belief. This,” he pointed sharply at the courting couple in the box across the boulevard, “is nothing short of public indecency.”
	
	
	Good. She hit a nerve by putting the Hasti Bir in her box. The Senator was mad. And this was his lashing out.  She cast a glance at the Regent to see if she could learn anything from his face. His expression was shuttered shut, which meant that he had likely had strong words with someone in the box recently, but beyond that she could not tell. She took a gamble. “Senator,” she started carefully and smoothly. “I do not believe that I am flaunting every Saran custom. For instance, there is a tradition established during the Kayus dynasty that dictates that a man may either be a general or a senator, but not both. That one, I believe that one is being broken by” she paused and tapped her forefingers together as if trying to recall the perpetrator, “you?”
	
	The Regent’s face stayed frustratingly still, but the men behind him shuffled nervously. Many of these men flocked to the Senator because it thrilled them to see someone seize power against traditional norms. They knew the Regent opposed this. A reminder of the conflict with both parties present was not comfortable.
	
	“You are changing the subject, Kokhan Mata.” The Senator’s words were bitten off at the ends. “You will make your harem behave as women should, or I will call a vote in the senate to exile you.” 
	
	The word exile sat on her chest like an iron maiden. There was too much riding on her ability to stay in Riosk. She could not, by any means, leave. Such a vote would be political theater, she reminded herself. The senate did not have the power to dictate the actions of the imperial family. The only person who could exile her was the Regent. Then she looked at the Regent’s expressionless face and understood what had passed between the two men. The Regent would not allow Reina to be exiled, and while the Senator could not force his hand, he would make him pay dearly for that stand. 
	
	The muscles in her back tightened. The Senator had called her here to punish her for humiliating him publicly. The Regent wanted her to defend herself without costing him any political capitol. She felt cornered and abandoned. Reina licked her lips behind her veil, grateful for the privacy it granted, and thought. She was in a public space. The men in the box were her enemies, but the public at large was not. Moreover, the public would not judge the Senator kindly if he broke too many of his beloved customs and norms. 
	
	“Very well, Senator, you will exile me. But will you also exile every father who allows his daughter to speak to her betrothed in public?” She looked at Senator Al Fazad, who had recently married off the second of his three girls.
	
	“That is not–” Senator Karim started, but Reina cut him off. 
	
	“I have legally adopted these girls as my daughters. Which makes the regent their father. He has been supervising the proceedings across the boulevard from not 40 feet away.” She took a step towards the Senator and looked up at him through her veil. “Tell me, Senator, if this is gross indecency, who else will you exile?”
	
	None of it was true, of course, other than the fact that the girls were legally adopted. But when the Senator turned to look at the Regent, her husband raised his eyebrows in expectation of an answer. Reina allowed herself a small relaxing of the tension she held in her back. It had been a gamble, but she could count on the Regent to support her if it suited his purpose. 
	
	The Senator looked at the Regent, then back at Reina, then back to the Regent again, before realizing that he was trapped. His face reddened as if he wanted to [DO SOMETHING TO HER THAT IS PERFECTLY APPROPRIATE TO DO TO MEN BUT NOT AT ALL TO WOMEN], but could not do so in front of the entire city. After a moment, he collected General Rahal and left.
	
	***************************
	
	A circle of girls sat in on the cool mosaic tiles in the veranda of the harem building, helping Kripna sort through her trousseau. Igrit, Freya and Peisa had joined them, as tradition dictated that married alumni help their younger sisters prepare for their new life. Only Hazda was missing from the gathering, and the absence displeased Reina. But this was a moment for Kripna, and she would not ruin this for a simple disciplinary matter.
	
	The women and girls sat around in anything from chemises and shifts to well worn comfortable clothes that had not been seen in public for years. Some of them looked like they intended to sleep eventually, while others looked ready to while away the hottest hours of the day in the company of old and new friends. 
	
	For now, laughter and chatter filled the courtyard as the company passed around trays of tamarind juice and exchanged news of events in Velut or Clave or Ursk that they had heard from their far flung families. They compared notes and speculations about the fluctuating prices of spices or teas or grain from places as mysterious as the south of Clave or as near as Velut’s granaries while admiring Kripna’s embroidery and critiquing her choice of linens. Bowls of pistachios and fried noodles passed hands along with dates when husband or brothers or nephews would prepare for campaigns in Ursk or the Janma Basin or sail for Xi.
	
	Reina listened and learnt and let herself indulge in the pleasure of the girls she had created. This was not the old Kokhan’s harem, where women were not allowed to read or write or know anything of the world. Her girls understood what Sardona was and what it wanted to become. They gathered to share knowledge to make it greater than the empire dreamt it could be. 
	
	“When are you leaving,” Quarzen asked Kripna suddenly. 
	
	Kripna folded a bed sheet that her peers had scoffed at as too simple, and put it aside for later consideration. “I leave with the Bir when he goes.” Reina knew that Kripna had mixed feelings about her departure. Excitement at being matched to a young and handsome man, fear of living a week’s travel by river from everything she had grown to know and love, pride at becoming a Birna and a first wife, dismay at having to leave the cultured city for a backwards province. For now she was basking in the attention. 
	
	“So soon,” Quarzen wailed and buried her face in Kripna’s neck. 
	
	Before Reina could intervene, Peisa pried the girls apart. “Don’t you dare. You’ll set off the younger ones.” Already, Isolde was curled up in Gris’ arms for comfort, and Aa’ah sat silently picking bits of skin off a pistachio. Elanya sulked a distance away from the crowd, though that had nothing to do with grief over Kripna’s departure. 
	
	Igrit scolded Peisa with a harsh look. “Don’t worry, starling. You’ll see her again.” Then she added conspiratorially, “I hear that Umma is going to visit Kripna soon. She’ll say that it’s a diplomatic mission, but we all know the real reason.” 
	
	With that, all the girls but Elanya started clamoring and pleading to accompany her on the trip. Reina tried to give Igrit a stern look, but the woman was grinning so widely at  the trouble she had caused that Reina found herself laughing as well.
	
	“Very well, Igrit. Have it your way. But for that, you get to tell the first story.” It was important that the girls know about the realities of marriage first hand. It was even more important that they know that they can speak freely about it in this company. Igrit’s marriage was a relatively happy one. It was safe for her to start.
	
	Igrit beamed. She was a skilled musician, with a fondness for wind instruments. During her time at Riosk, she had played frequently for the Regent Kokhan’s court. Her husband was a Velutian nobleman many decades her senior. She didn’t mind his age or being his third wife, as it mostly meant that he was easy to flatter and he already had sons. He’d married her primarily to bring culture to his court, and she did her best to make certain that they constantly entertained some personage of importance from Riosk, or Astria or Xi.
	
	“Don’t you get, you know. Lonely?” Kripna asked.
	
	Igrit sombered for the flutter of an eye. Reina knew that she had frequent discussions with other alumni about the pros and cons of taking on a lover just to keep her bed warm. The warm sociable woman struggled with loneliness daily. “You can’t possibly be worried about that,” Igrit teased, giving Kripna’s cheek a long hard pinch. “With a young, handsome groom already besotted with you before he even saw your face?” 
	
	The girls hooted in mock scandal. Reina settled in for a pleasant afternoon. She placed her back against a pillar and flung a leg over Freya’s lap. Freya, the first and most entitled of her alumni, shoved off the limbs and put her head in Reina’s lap. She balanced a bowl of pistachios on her belly, and started shucking nuts for the two of them. Reina rolled her eyes and reached lazily for a paper fan to cool the both of them.
	
	This is what a harem should have been. Friendship and banter and a place to share your common troubles. Reina knew that most harems were nothing like that, even as she did her best to place her girls in safer situations than the flock of jealousies the old Kokhan bread in his harem.  It was important that at least under her roof, her girls had a place to be safe.
	
	“That’s not what I mean,” Kripna protested. She was blushing deeper than a beetroot. “I mean away from Riosk. There’s nothing but soy fields in Velut. Aren’t you lonely?”
	
	Freya, originally from Velut, opened her mouth to protest, but Peisa and Igrit were already talking. “Every new bride is lonely. Even in Riosk, your husband’s house is a strange new land,” and “Write to me.  Tell me what you can’t get in the Basin, and I’ll tell you how I got it in Velut, or how to make a substitute. Promise?” 
	
	“What if he hurts you?” Aa’ah interrupted.
	
	A hush fell on the room. It was a question that came up every year or two. And try as she might, Reina could not promise that it would not lie in the future for her girls. She squeezed Freya’s shoulder.  Kripna and Quarzen looked awkwardly at their laps. Igrit and Peisa cast meaningful glances at each other. 
	
	
	Eventually, the Velutian woman sat up and took a deep breath. Reina held her hand for support. “The first thing you do is you come home on the first excuse you can find. Doesn’t matter how flimsy. The second thing you do is tell Amma. And then? You figure out what you want.” Freya told her story. At least she sketched it for the girls. There were too many details that didn’t need telling or even, remembering. Hamed ben Nebab had been a very good match. Even an half brother of a Regent and uncle to the Kokhanat by a set aside wife was still a close to the two most powerful men in Sardona. But Hamed was cruel, and Riosk had not yet come to know the Kokhan Mata for who she was. So Freya came home. With Reina’s help, she arranged for the Regent to suggest to her husband that he take on a mistress for his peculiar proclivities. 
	
	When both Gris and Taria expressed terror at needing to involve the Regent in their marital affairs, Peisha said dismissively. “Not everyone goes to the Regent.”
	
	“Not everyone is married to a Nebab,” Reina explained as Freya rankled beside her at Peisa’s snub. She wanted to divert it before it poisoned the atmosphere.
	
	“I came here, then asked Immu to send me to my parents and waited. Immu eventually sent for me again. My husband needed me more than I needed him.” Peisa had always been a very proud girl, and there were details missing from that telling of the work that Reina had to do on her General Khadim in order to ensure that there would be no repeat unexpected visits home for Peisa. But it was her story to tell, and Reina was certain that Peisa’s superiority in this regard would not be bad for her girls to emulate. 
	
	“Mata,” Taria piped, “you never tell your story.” 
	
	Igrit gasped and Peisa looked at the Kokhanat with a bemused expression. No one but Freya and Hazda knew Reina’s story, and that was only because they had witnessed it. No one before the Kokhanati had dared ask before, and Peisa wondered  if the Kokhanati would be the first to wheedle it out. 
	
	While Reina searched for a suitable means of deflecting her daughter, she was saved from the decision by the sight of Hazda walking quietly across the courtyard below.
	
	Freya jumped up and ran to the railing and called down to her old friend “Hazda. You wicked old crone. We’ve all started without you!” Freya grabbed an abaya for modesty and ran down the steps to the central courtyard. A few minutes later, the two women appeared at the top of the steps, arm in arm. “He must have been good, you are never late,” Freya was saying, while Hazda looked like she wished her friend would just shut up.
	
	Leon. A surge of jealousy and anger arose within her. She knew Hazda had a lover. Another middle ranked soldier, just like the man she had run off with seven years ago. Reina did not see why Hazda insisted on having such dalliances. If she wanted a mate, Reina would have happily arranged one for her. 
	
	Freya led Hazda back to where she had been sitting next to Reina, and pulled her down between them. Hazda smelt distinctly of soap in a way that belied her afternoon activities. It was disgusting, and she would have to have words with Hazda. Reina sat up primly, creating more space between herself and Hada. If the woman’s personal life was going to get in the way of her duties, then they would have to come to other arrangements. 
	
	“Amma, you aren’t jealous,” Freya scolded. She reached over to put Reina’s hand over Hazda, who looked equal parts mortified and guilty. “Hazda’s devoted to you. Any fool can see that. But Leon’s been away fighting in the basin for the last two months. You should have given her the day off.” 
	
	There was no possible way that Reina could have survived this day without Hazda. A day off was an impossibility. But Freya had been clever in protecting her friend. By calling her out in public, it would be much harder for Reina to take extreme measures to correct Hazda’s negligence.
	
	
	
	*****************************
	Don’t like this section 
	***************************
	
	“Amma? Amma. Amma! He’s here again,” the fourteen year old Freya whispered, terrified. “Amma!” she shook Reina harder and wished that her personal boogie man had been a child’s imaginings, or even a ghost. Her man, who appeared unexpectedly from under her bed, even when her bedroom changed buildings was not a ghost or a figment of her imagination. He was very much flesh and blood. 
	
	Reina took a long time waking up. She had been sick so often recently, when the nausea finally left her, she was left completely drained of energy and exhausted. “Who, Freya? What are you talking about?”
	
	“He’s here,” Freya squeaked. He pointed her finger to the pale figure standing in their doorway. “Look.”
	
	
	Reina came to alertness with a start, then reached for an abaya as she extricated herself from the mosquito net. She hated that the Minister of Domestic Affairs appeared whenever he pleased in the privacy of her bedroom. It was not appropriate, or safe, or decent or … the impropriety defied words. She didn’t like it. She wished that there was something she could do about it. She had seen first hand what the man would do to people who crossed him. She had provided him evidence to black mail and manipulate the men he thought worked against the interest of the Kokhan his brother. Or those who worked against his own interests. Darius ben Nebab was officially the Minister of Domestic Affairs, but in actuality was the Kokhan’s First Spy. He was more protective of his interests than the Kokhan’s first wife, more devious than the second and fourth put together, and nearly as powerful as the Kokhan himself. She hated that the Minister of Domestic Affairs chose to make secret visits to her bedroom whenever he pleased. But he was a dangerous man to cross.
	
	She reached over Hazda’s sleeping form on the floor for her abaya, which she wrapped tightly about her and quickly tucked her loose hair under a cap before approaching the unwelcome guest. “What are you doing here,” she hissed. 
	
	“I wanted to see you.” The Minister’s voice was low in honor of the fact that there were sleeping women in the room, but otherwise nonchalant. 
	
	“Why?” Her voice quavered. Reina swallowed hard. As far as she could tell, at best, the minister’s visit meant that he had simply come by to watch her and Illyn sleep, a circumstance that had caused to scour her room in the harem for secret doors, or, at worst, meant a very dangerous proposition couched in a situation that made it more dangerous to not accept. His visits were intermittent, but consistent. Sometimes they occurred several times a month, sometimes two months would go by without Darius’s attentions.
	
	The Minister bowed slightly. “My apologies, Kokhani. I seem to have disturbed your companion,” he nodded at Freya, who was whimpering in bed. “Shall we speak in private?” He gestured for Reina to step out the door of her bedroom. Reina bit her lip. She did not want to leave the safety of the company of her women, but she did not have a choice. 
	
	They passed in silence through the two rooms allotted to Reina and her maids, along a short corridor that had three or four other doors and to a narrow rickety set of stairs. Reina stopped, suddenly terrified. Down the stairs were the lower levels of the Round Fort, where less powerful or more dangerous prisoners were kept. Had her husband sent the Minister to remove her from house arrest to a cell? Or had the Kokhan finally decided what he wanted to do with his treacherous wife? She had grown extremely unsteady on her feet with her current condition. The stairs were narrow, uneven and dark. Had her husband just sent the Minister here to get rid of her altogether?
	
	“Up,” her brother-in-law whispered behind her. “Go carefully. I’ll be right behind you.”
	
	Reina fumbled her way in the dark up the narrow, uneven stairs, one hand holding her skirts and the other feeling her way along the wall. Darius followed two stairs behind. She knew she had reached the top stair when her foot came down heavily on the landing, expecting to find a stair that was not there. She let out a small gasp. The Minster crowded her on the landing, reached over her and knocked in a complicated pattern. Reina bit her lips again. He was always standing too close to her whenever he connived to find himself alone with her. And he found means of seeing her in private far more often than she wished. 
	
	The door opened onto a wide moonlit roof. A guard stood on the other side. The Minister passed him something and the guard nodded, indicating that they should pass.  On the far side of the tower lay a pair of cushions on a small rug. Next to the rug sat a pitcher of ginger water and a carafe of wine, along with two goblets. Her intruder indicated that she should make herself comfortable and offered her refreshments. 
	
	Whatever it was that the Minister wanted, he was taking his own sweet time coming around to it. The superfluously long prelude and the apparent lack of immediate danger awakened her curiosity. She worked up enough courage to ask “Darius, how did you get in here? And all this…” she gestures to the cushions and the drinks. “How did you get past the guards?”
	
	Darius ben Nebab chuckled softly. “You are charmingly innocent, my sweet Kokhani. The Round Fort’s guards are nothing compared to the harem’s eunuchs. My brother expects unquestioning loyalty from those creatures, and they are rewarded well. Here? Half the guards here can be bought, and you are only under house arrest. Why do you think I told Raed to put you in here?” 
	
	Bile rose suddenly into the back of her throat. She leant over the edge of the roof to spit it out. The thought of her husband and her brother-in-law colluding about her imprisonment so that the latter could have greater access to her… It was… The prospect of nearly daily midnight visit from the minister was unthinkable. Reina moaned and took her time to find her courage. She had done it for the past several years, she told herself. She was in Sardona for a reason, she reminded herself. There was a lot she would sacrifice for that reason. She steadied her hand to reach for the goblet of ginger water. She had no option but to endure. 
	
	Darius waited patiently for Reina to collect herself. She could feel him watching her every motion. “What do you want with me, Darius?” she asked, resigned. Darius would do to her what he would do, and there was nothing she could do about it.
	
	Darius seemed to not have heard the question. “It’s a lovely night tonight, isn’t it. You can see for miles out to sea.” Reina looked down at her hands twisting a corner of her abaya in her lap. When she didn’t respond, he finally sighed. “I brought you something.” He handed her a narrow suede bag. 
	
	When Reina opened the drawstrings and poured out the contents, she dropped it as if she had just pulled a snake out of the bag. She recognized the scroll case. It was the entire reason that she was in this prison in the first place. Darius picked up the scroll case and pressed it gently into her hands. “I need you to finish what you started,” he insisted.
	
	Reina shook her head vehemently. Her heart was pounding against her chest. She had last held this scroll case a short week ago, along with a summary of the financial discrepancies it revealed. The scroll was one of many that had once belonged to Admiral Noor, in charge of troop and supply lines to the war in Astria. “I can’t. It’s too dangerous.” I don’t want to anymore she wanted to say, but didn’t.
	
	“I need to know what the scrolls hold,” Darius insisted. “There are only five more. I can bring them all by, if you’d like, so you can look through them at your leisure.” Reina never asked how Darius came across such things. She just knew that a few times a year, Darius came to her with a set of accounts that he needed her to inspect. Unlike herself, Darius was innumerate. However, the Minister of Domestic Affairs was also known as the Kokhan’s third eye. He was rumored to know everything ever said in the city. Reina did not know how true that was. Yet, when he needed to uncover an embezzlement plot, he had to come to her for help. 
	
	“Please, Darius. This pregnancy. I’m too sick. Let me rest. Please.” She hadn’t known she was pregnant when she had accepted this newest commission. Darius’s commissions were always well compensated. He smuggled out letters, helped her arrange Illyn’s marriage to Claudio, crown prince of Astria, smuggled in opium to help her sleep or trustworthy girls to keep her company. He always came to her in a moment of need, offering her a solution to an immediate problem in exchange for a job that could get her exiled, imprisoned or killed. It was never a trade she wanted to take, but they were never trades that Reina felt like she could not make.
	
	This commission, however, hadn’t been exchanged for a favor. Her Illyn was in Astria, and the thought of Sardona conquering that peninsula was unacceptable. Reina didn’t ask what Darius’ motivations were for thwarting Sardona’s subjugation efforts. When he had come to her to uncover a circle of embezzlement and racketeering in the top ranks of the army, it didn’t matter why he wanted the same thing that she wanted, just that he did. But the first months of this pregnancy were hard. Between the sickness and Hazda’s insistence that she stop taking opium for the sake of the baby, Reina found herself constantly either tired and aching or nauseous and vomiting. She was neither physically or emotionally capable of closely guarding secrets in the small confined society of the Kokhan’s six wives. When she had not joined the other wives readying themselves to honor an Emissary from Xi, Ghada, the Kokhan’s third wife, had come to check on her. She found her bent over a bucket of her own vomit, the contents of the leather scroll case laid out carefully on a table beside her. 
	
	Women did not read in Sardona, let alone know how to understand accounts. They certainly did not take part in schemes against the military’s top generals and admirals. The only reason she had not been executed on the spot was because she might be carrying the Kokhan’s son. Reina had learnt her lesson. She did not want any more trouble. She prayed nightly that she would find a way out of this. If the Kokhan would just let her live, she would keep her head down and involve herself in any manner of politics. Whatever Sardona did abroad or at home, she would keep her head down and endure. She had to. For the sake of her children.
	
	“But you are in the Round Fort now,” Darius sounded like he thought this would comfort her. “There are no wives or eunuchs to spy on you. You have your girls with you. Raed won’t be visiting you here. You’ll have plenty of time to rest and recover your strength. Surely you’ll have time to look through a few scrolls for me. Aren’t you bored?” 
	
	Bored. The man wasn’t listening. How could she possibly be bored when she was certain to be executed the day after she gave birth. She tried a different angle. “You can’t keep visiting me here like this. This isn’t the harem. How many times do you think you can bribe the prison guards before one of them,” she cast a sidelong glance at the lone guard on the roof. Surely there should be more. He had his back turned to him. “Before one of them tells him.”
	
	“Ah.” Darius stopped pressing the scroll into Reina’s hands and sat back on his cushion. He examined his nails carefully in the moonlight before picking up the odious scroll case and tapping it idly against his knee. He sat there. Tapping. Lost in his thoughts. 
	
	Reina discomfort urged her to speak. She was sitting illicitly with the Kokhan’s third eye on the roof of Riosk’s prison. Where the Kokhan had placed her, not seven days ago. If someone saw them. Would her husband endure another slight? Yet the man before her acted like there wasn’t a care in the world. “Darius, what is it?”
	
	
	Suddenly Darius roused himself. He tucked away the scroll case and rose on one knee. He took one of her hands in both of his and raised it to his lips. “Most honored Kokhani Reina i Nebabi. How would you feel about becoming my wife?”
	
	“What?” Reina jerked her hand out of Darius’ grasp and stood up. “Are you mad. I can’t marry you, my husband… oh.”  She put her hands over her face. Suddenly, the implications of the proposal sunk in. The Kokhan was dead “What have you done?”
	
	Darius stood and put his hand gently on her elbows. She froze at his touch. “I couldn’t let him kill you,” he said softly. 
	
	“How many of my husbands have you killed, Darius.” The words came out before she even thought them. She immediately regretted them. To accuse a man like the Minister of Domestic Affairs, for a woman in her position. She could not afford to get on his bad side.
	
	To her surprise, Darius looked hurt, not angry. “How many…” he muttered to himself. “Is that what you think.” He shook his head and recovered his usual immutable composure. He stepped in closer and bent down to make his face level with hers. “One, Kokhani.” he said solemnly. “Only one. On my father’s grave.”  
	
	Reina pulled away and walked slowly across the roof. She couldn’t process all the implications of this news. Raed was dead. She was a widow. No. Riad was dead. Which meant that he could no longer give her execution order. She would live. By whichever gods watched over her now. She would live! Joy filled her chest to bursting. She wanted to run downstairs, awaken Hazda and sing.
	
	But she was a widow. Before celebrating with her girls, she wanted to understand her situation better. Normally, Saran widows were sent back to their parents’ homes with their daughters as soon as their sons were too old to be in the harem. Badr was three. That gave her only two more years. The joy collapsed like a punctured bladder. Two years was too short a time. She could not possibly find a way to protect her children in two short years. She needed to buy them more time. Suddenly she felt dizzy and put a hand out to ease herself down to the ground. Darius grabbed it and steadied her instead.
	
	The relief, the shock, the helplessness, the constant nagging worry combined with her desire to be free of anything to do with this interfering spy turned into a fury that she had no control over.“You’ve destroyed me!” she cried. “Do you have any idea what you have done? You’ve ended the treaty. My marriage treaty. What am I supposed to do now? Everything I have endured from your brother, his wives, this disgusting court. It’s all for nothing now. You made it all moot. Why didn’t you kill me instead.” 
	
	Within moments of the outburst, the sole rooftop guard was upon them. Darius grasped her arm firmly. “Kokhani, you are distraught. Let me walk you back so that you may retire.” Mollified, the guard watched them cross the roof in silence. 
	
	When they reached the stairs, Darius went down first, waiting for Reina’s eyes to adjust to the pitch black, then taking the stairs slowly, one at a time, waiting for Reina to follow at a pace where she would not stumble. Then he walked with her, still slowly, still silently, down the corridor until they entered her ante chamber. Only then did he ask her to light a lamp. 
	
	Fumbling in the dark for the oil lamp and flint, striking the spark carefully to the wick, filling the empty lamp with a slow trickle of oil so as not to flare the flame all too careful steady hands. It calmed her nerves. So had, she realized, Darius’s deliberately slow return to her quarters. 
	
	She put the lamp between them and asked wearily, “What will you do with me?”
	
	Darius took the lamp and crossed the room to a bench. He lengthened the wick for more light, then motioned for her to sit with him. When she did, he explained “Whatever you would like. I can send you away to Astria to live with your daughter. That would certainly break the treaty, I’m afraid. Or,” he paused and a look of uncertainty passed briefly over his face, “I will most likely be appointed Regent in the next few weeks. You could marry me and stay on in Riosk as the Kokhan Mata. I could probably preserve the treaty in that case. 
	
	Raed was likely not yet cold in his bed when I came to you. Certainly, no one will realize until morning.  The harem should be emptied by a month or two from now. You can move back there if you’d like.” He sighed heavily and rose. “You should probably decide before the mourning period is over though. That would leave you the most options.” He bent down and put the lacquered scroll case on the bench beside her, patted her on the head, and left.
	
	*****************************
	
	Reina stood, bathed and perfumed, looking over the collection of books on the Janma Basin that Elanya’s and Kripna’s tutors had gathered for her that afternoon. Elanya sat sifting through her jewelry bags and boxes, selecting something appropriate for the set of toub’s Gris had selected for the evening’s meetings. 
	
	She selected a book with a promising title from the pile and read. The Bir of the Hasti had asked for her help in the negotiations with their Frit neighbors. Reina had not committed herself, but she was pleased that the man had picked up on her openness to let herself be involved. 
	
	Senator Karim was a thorn in the Regent’s side, grabbing power and setting himself up to control the Kokhanat’s court when her son came of age. He believed in an old expansionary vision of Sardona that went against everything she stood for, and he had made himself her enemy for the very reason that she wanted Sardona to value trade over conquest. 
	
	The afternoon with her girls and alumni had been a much needed respite in her extremely long day. As the sun sunk low in the sky, her alumni dressed for the outside world, collected their children and their nurse maids, and left for their marital homes. She had said her fond farewells and stood on a balcony for a long time watching the slow procession until they got swallowed by the crowds. She enjoyed the large public gatherings of the families of her alumni. They were a time for the families to celebrate, yes, but they were also a chance for the men she had hand selected to come together and perform the types of business that she knew Sardona needed to survive, even if Sardona did not realize the need. But her heart lay in these small, intimate gatherings of her girls. She enjoyed sitting with her married girls, hearing of their successes and troubles. It was a chance for her to watch them grow, and a chance to feel some of the feelings a mother might feel for her own married daughters. Of all the difficult things she had carefully balanced today, that had been pure pleasure. Of all the difficult things she had carefully balanced today, the most difficult still lay ahead. 
	
	“Umma, are you ready?” Gris held out two nearly translucent silk toubs. One was a dark blue with silver and pink birds. The other a pale green with bright red and yellow floral designs.
	
	She closed her book and chose the blue. Unlike the women of Clave, who seemed to vary in color from strongly brewed tea to the color of rich wet soil, she stood pink as a piglet after the bath and scrub. Pale green would never work. 
	
	Hazda brought in a tray of bread, fruit, jam and cheese and stood by expectantly. Reina looked at her irritably. “You are dismissed.” 
	
	
	Gris undressed her until Reina stood completely naked except for a small scrap of cloth of the style Aa’ah’s people wore around the loins. When Hazda did not leave, she added impatiently. “We will talk about it later. Right now,” she gestured to the girls and the books, “I have other things to do.” 
	
	“You are meeting with the Regent at sunset,”  Hazda reminded.
	
	“Yes, Hazda,” Reina snapped. “I will talk to you after I meet with him.” Gris tied the toub around her waist and started tucking the yards of cloth so that it would fall correctly.
	
	Hazda put some jam and cheese on a piece of bread and rolled it before handing it to Reina. “Your plan is to seduce him?”
	
	Reina took the roll without comment. Meeting with the Regent was like wrestling with a bear. He was an absolute force of nature, but one, she was becoming more and more certain, that did not want to hurt her as long as she didn’t get in his way. Yet she had no idea why he had returned so suddenly. She did not believe for an instant that his timing with Senator Karim’s return was a coincidence. And she did not know why he wanted to see her. The entire combination reeked of danger.
	
	Hazda waited for Reina to finish chewing before she asked, “What is your plan, Kokhani?”
	
	She didn’t have a plan. She usually worked through a plan of engagement with the Regent with Hazda before entering his lair, but she didn’t want to talk to Hazda today. Not after… Reina sighed and picked safety over jealousy. “I just put Senator Karim on his back foot. We’ve been trying for years. I want the Regent to press his advantage.”
	
	“By allowing you to negotiate with the Frit?” Hazda probed.
	
	“By allowing me to be close enough to Senator Karim’s camp that I can learn something about his actions. We’ll never take him down in Riosk. But if he left camp in a rush, there’s  a chance …” Reina trailed off. As she spoke it outloud, she realized how small a chance she had to find anything to hurt the Senator.
	
	Hazda had reached the same conclusion. “He won’t let you. He’ll see it as too great a risk for too small a chance.” 
	
	Reina considered how to strengthen our argument. “We’ve been unsuccessfully trying to reduce Senator Karim’s faction’s influence over the Kokhanat for years. His hold in court is just too strong. You saw him today. He is angry, on his back foot. This is the moment to press our advantage.” 
	
	Hazda gave her a pained look. Reina knew that her confrontation with Senator Karim had scared her. It had been an objectively frightening and dangerous situation. But so much of Reina’s life at court had been objectively frightening and dangerous. She could not stop, and she could not have backed down from his threat. This was the only way she had to protect her children.
	
	Gris finished arranging the thin smooth silk over her torso and head. Reina sat to have her hair combed and arranged while Elanya came over with a selection of jewelry. Hazda studied Reina’s attire. “He will object,” she noted. “If his main objection is your personal safety, sex might work.” 
	
	Reina picked at the fruit thoughtfully. If Darius had no other complaints, he would certainly not like her taking the long journey up the Janma to the Hasti and Frit border. She doubted he would be too concerned about the danger she would be putting herself into when engaging with Karim’s men. He himself had put her in much greater danger several times. But safety concerns were the easy objection. Reina had no idea what had called Darius back to the capitol so suddenly, and it bothered her that she couldn’t prepare properly because of that. 
	
	“Then I’ll offer him use of my alumni.” Gris startled and dropped the comb. Hazda gave her a stern look over Reina’s shoulder. 
	
	“You misunderstand, child,” Reina explained. “The Regent has never stopped being the empire’s third eye, even if he is no longer the Minster of Domestic Affairs. He is just working for his own interests now. He always wants more agents, and he knows what you girls can do.”
	
	Hazda looked uneasy. “You should not play that card, Kokhani. You should save it for a more important situation.” 
	
	Suddenly, one of the unnamed tensions that had been sitting inside since her meeting with Senator Karim crystalized into words. In the safety of her room and Hazda’s calm presence, she gave it voice. “You did not see Badr, Hazda. He is growing to become his father. He is cruel. He is manipulative. He is completely under Senator Karim’s influence, and the man will mold him to bring back the ways of the old court. If we cannot change his mind before he comes to power…” Reina paused to take a deep breath, fighting to keep the tremor out of her voice. “I’m afraid that if Senator Karim stays in power, Sardona will expand. He will get Badr  to turn our military’s eye’s westward. There is no more important situation.”
	
	************************************
	
	The chair ride from the harem to the palace took less than five minutes. There was a public way for official guests and public processions and the like. That took one first onto the victory boulevard which one followed to the public fountain and then through to the front gate of the palace. From there, the walk to the regent’s quarters would take another ten minutes of climbing up and down stairs, cutting through courtyards and passages until one returned to the south wing of the palace.
	
	Instead, Reina’s sedan left out a side door, through a small metal gate and along a pleasant path shaded by olive and orange trees in the modestly sized private garden that used to be the associated to the harem, until the regent had gifted her the 100 acres of land just beyond the city walls to landscape into a map of Sardona. Now ,her girls sometimes played in this quaint little spot during the precious few hours of free time their studies allowed them, and she used it for easy access to the palace. 
	
	When they reached the base of the circular stairwell that would take her to the regent’s rooms, she dismounted the Sedan and walked flanked on every side by her mutilated guards. Customarily, these types of passages were for men, as they did not allow for any form of covered conveyance. The Regent did not seem to mind this breaking of norms, or at least, since she had not been seen walking up the palace stairs, he had not protested. These stairs were part of a quieter route to his rooms. As it was convenient for him to be able to meet, discuss and scheme with her without anyone else’s knowledge, it suited him to permit her this liberty. 
	
	Darius was a dangerous man, there was no getting around that. He played people like some people played chess, positioning them into places where they would be ready to be of use to him when the need arose. She had seen him have people who betrayed him killed, helped him destroy the careers of those who longer wished to serve and toss aside those who were no longer of use. She was still of use to Darius, Reina knew, as Darius was to her. She dreaded the day when that would change.
	
	Since her marriage to him, Reina had learnt that it was not that Darius could always controll what a person would decide before he put them in the position where they would have to make the decision. Rather, he was the most thorough planner she had ever known, with contingencies in place for every imaginable reaction. Very few people  ever surprised him, and no one surpised him twice. As the Minister of Domestic Affairs, Darius was  feared as a man not be crossed. As Regent, it was rare that he did not eventually get what he wanted. The fact that, after so many years, he still had not come up with a way to lessen Senator Karim’s iron grip on the future Kokhan worried Reina.
	
	One problem at a time, she reminded herself as they emmerged from the dark stairs onto a wide cool marble passage. Today, she would learn why Darius and Senator Karim had returned to Riosk so suddenly, and gain permission to travel to the Janma Basin. Then she would worry about weakening Karim.
	
	Her entourage stopped at a set of large teak doors set in brass, the daily battle between Sur and Badur engraved in the wood. She took a deep breath and shook out the tension in her limbs. Her goal was to be alluring, she reminded herself. She removed her veil and pushed open the doors.
	
	*************************
	
	The Regent Kokhan, Darius ben Omar ben Faisal ben Azmi ben Ghiath ben Iqbal ben Raed ben Badr ben Nabab sat on a carpet scattered with papers and maps, interrupted from a conversation with Senator Najhi Gali. The senator blushed at the sight of the Kokhan Mata’s face, bowed, and scurried out of the room like rat.
	
	Darius sighed and gave her a reprimanding look. “You do not have to flaunt all our customs.”
	
	She nodded towards the window. “It is at least 10 minutes past sundown. He should not have been here.” When Darius scowled she added more coquettishly “Shall we try again?”
	
	Over the years of their marriage, their initial alliance had grown a substrate of friendship. She did not trust Darius to serve anything other than his own interests, and she did not know what he thought of her. But where their goals aligned, she could depend upon him to support her, and the longer she worked with him, the easier it became to put on a pleasant face to please him.
	
	As he rose to approach her formally, she undid the top several buttons of her abaya. The gown fell easily with a shrug, revealing the silk toub that left little to the imagination.  She bowed deeply. “Badur bless your evening, husband.” She noted with satisfaction the sharp intake of breath and the slight stutter in his voice. 
	
	“May Sur await you in the morn. Rhayina, wife. Please rise.” The Kokhan Mata clenched her jaw. He was the only person in Riosk that called her by that name. It was a dead name, from a past life that he insisted on keeping alive. She hated it for the memories that threatened to reawaken. Darius is a Nebab. All Nebab men are cruel. I will not let their cruelty distract me. “Let me look at you,” he said, and she made certain that when she faced him, her face held as pleasant and affectionate an expression that a husband could ask for.
	
	She wanted to walk to the the other side of the room  to examine what Darius and Senator Gali had been discussing. She had not realized that the two men were closely enough allied for Darius to meet with him in his chambers. Her curiosity would have to wait. She turned herself around slowly for inspection, then let Darius embrace her. She could feel the outfit have its desired effect. The length of his kiss made her casually wonder if he was currently between lovers. She filed the thought away for later consideration. 
	
	“To what do I owe this,” he pulled back so that he could inspect her better, “honor.” Reina giggled and took Darius by both hands to the settee across the room. They both knew that she was not the flighty girl that she was pretending to be. But a little bit of role playing never hurt. It would help Darius relax and make him more likely to at least be garrulous even if it would not, as it often did for the old Kokhan, make it more likely to let her have her way. One never knew what he might let slip.
	
	“Is it not customary for a wife to desire her husband’s company?”
	
	Darius laughed and sat down. “Do forgive me, Rhayina. But you must admit, you are a hard consort to have in this court.” He drew her into his lap. “For instance, someone told me this morning that I have been made a father without my knowledge.”
	
	He was still enjoying her physical presence, she noted. His fingers idly rubbed the silk against her thigh. She should use the mood to get information. She lay her head against his shoulder, letting him have access to all of her body if he wished it. “I had to say something, didn’t I?: she pouted, “It wasn’t that far from the truth.”
	
	Darius’s chest shook in silent laughter. “You were brilliant today. I don’t think I’ve ever seen Taliq Karim quite so shaken. Where did you find the Bir?”
	
	Reina ignored the question and returned one of her own. “When I was summoned this morning. You were upset. What had the Senator asked for?”
	
	“Oh.” Darius turned his head and spat, casually, as if he were expelling an insect that had flown into his mouth, but the gesture was meant for the Senator all the same. “Nothing new. We keep worrying the military with our new budgets and diplomacy. He can’t attack me, and he can’t yell at you. So he tells me exactly how he would like to attack you.” Darius shrugged. “There’s nothing to worry about there.” 
	
	Then why didn’t you defend me? she wanted to ask. Why did you let him humiliate me in front of the Kokhanat and those men? What are you playing at? 
	
	Instead she tried a different line. “And how was your journey?”
	
	“Easy. I came by sea. It only takes four days from Tabrisi if the winds are with you.” 
	
	“That is fortunate, then. Otherwise, you would have missed the parade. What did you think of it?”
	
	Darius shook his head in mock severity. “Absolutely not, my little spider.” He slid her off his lap. “That’s three question for you already, and you have not answered my one.”
	
	
	“Guilty,” she admitted, and pressed herself against his side. She needed to learn more before bringing up a diplomatic trip to the basin. “I didn’t find the Bir, he found me. He came to me complaining of how Senator Karim has treated his people. It was too good of an opportunity to pass up.”
	
	“And Kripna. They are really engaged?”
	
	Reina struggled to give a bright nod. Of course they were really engaged. A lie like that could ruin a girl’s character, not to mention the entire future of her organization. It astounded her that a man like Darius could be unaware of such nuance. “She will return to the basin with him when he is ready to depart.”
	
	Darius slapped his thigh and jumped up in excitement. “You have outdone yourself, Rhayina. Finally, we have permanent eyes in the Senator’s operations.” Permanent? Reina wanted to question her husband. The Senator was not expected to be in the Janma Basin for more than a few more months. But Darius’ mood had already changed. “My dear consort and companion,” he started, entreating, “How skilled is your girl Kripna. Would she…?”
	
	Absolutely not. She would not hand over her girls for Darius to use unless it was a last resort. Reina stood up on tiptoe and kissed Darius firmly on the mouth. “No,” she said with finality. 
	
	“Very well.” Darius shrugged. He hadn’t really expected to win that round. “At any rate, this is something to celebrate.” He rubbed his hands over her buttox and hips. “But I feel I am in your debt. Let me repay it. What would you like?”
	
	“A favor? I am flattered. Let me think” Reina let the moment linger, encouraging her husband to continue his foreplay. She needed to keep him willing to drop useful nuggets. The main question was, did she want to know why Darius had cut his trip to the southern ports short, or did she want to go to the basin? She opted for the first option. She would be better prepared to extract the second if she knew.
	
	“I know!” she announced, stepping just out of reach to let him know that she was serious. “Answer me honestly, husband. Why did both you and Senator Karim arrive in Riosk today?” 
	
	She may as well have thrown the man in a lake of ice water. He walked briskly from her to the where he had sat with Senator Gali and started neatening and putting away the piles of papers, books and maps that cluttered it. He was angry, she could tell. He wanted something to do with his hands. Probably, she hoped, the anger was not directed at her for asking the question. Still. She didn’t think pressing him would help. 
	
	Instead, she opened a cabinet where Daris kept his liquor, took her time smelling and selecting a drink, before pouring to glasses and returning to the her husband. She offered him a drink, sat beside him and waited.
	
	Eventually, Darius gave an exhausted sigh. “The Senate is considering revoking the protectorate status of the entire Janma Basin and annexing it. Senator Gali thought that this would be a good compromise position. The military dislikes the number of protectorates Sardona has. They dislike having to settle their disputes without reaping any of the benefits. On the other hand, it seems we have been successful in convincing the senate of the importance of the river route to Xi. Iskander could not understand why I was not thrilled.” 
	
	The words took her breath away. Annexation. Military conquest of an already allied province. She was about to send Kripna to a country that would very soon be at war. That was unacceptable. No, it would be criminal on her part. It was unbelievable. There had to be a different solution. She must not have heard correctly. “Annexation. Darius, are you serious?”
	
	“As death.” He pressed a knuckle to the bridge of his nose. At least he is also upset by this news. There is hope yet. “The vote is at the end of this week. That is why Senator Karim and I are both in town.”
	
	Reina moved to perch behind Darius, rubbing her fingers into his temples as Hazda had done for her only a few hours ago. She needed to think, but her thoughts flit about like dragonflies, Her mind was gummed up by not only her fear for Kripna’s future and safety, but also her disappointment that all of the hard work she and Darius had put in to get Sardona to see that diplomacy could be just as profitable, if not more, than conquest, and her terror that this might be the first step in the military gaining back the strength and support it had in the old Kokhan’s day. She couldn’t trace down the threads of one problem towards a solution without her mind immediately jumping to another.
	
	When she felt Darius relax under her touch, she asked, “What will you do?”
	
	Darius threw up his arms in frustration. “Do? I will do as one does in the Senate. I will talk to my peers and hope that more of them agree with me than do not.” 
	
	Hope? Her husband never did anything on a hope. Politics in Riosk meant always having a plan. Darius was feared  and successful because his plan was always better than that of his competitors. Why was he giving up?
	
	“I can’t fight Taliq on this one. Not unless I am certain that I will win. He is just too powerful.” 
	
	The pieces started shifting into place. Senator Karim had put Senator Gali up to this. Darius had said the Basin would be Senator Karim’s permanent home. Did that mean he wanted to use this annexation effort to become a governor? That was a horrifying thought. “Darius,” she began carefully, “that is what you said when he wanted to lead the efforts against the raiders in the basin, that he was too powerful to fight. Now he is a Senator with military backing. If he becomes a governor as well, … You cannot let that happen. There will be no stopping him.”
	
	“And what would you have me do?” Darius snapped. “He has long since passed the point where we can hope to reign him in. If I try something and fail, he will wait. He knows that I will eventually stop being the regent. And then he will come for me. If I fall, what will happen to you?” Reina let comment slide. There was too much on the line to be bothered by false chivalry. If Darius wanted to save his skin, then she would have to look for other options. 
	
	The regent must have sensed her thoughts, because he quickly added. “I am not leaving you to the wind.” He shift to face her. “I promise you that. There are still three years until Badr becomes Kokhan. I will find an opportunity to kill Senator Karim before that happens. But for now, I cannot directly confront him.” 
	
	Reina’s mind spun. What Darius was telling her did not make sense. If he was certain that he would kill Senator Karim before Badr’s ascension, why did it matter that he may not succeed in thwarting him now. The Senator could not threaten Darius either way. There was more to this story that she did not understand. It could simply be that the Regent was afraid of the man. But she did not want to believe that option yet.
	
	She would have to figure out Darius’s game later. The landscape had shifted on her. She was no longer coming to Darius to find a way to convince him to let her use an opportunity to find intelligence against Senator Karim. She was helping him come up with a plan for the possibility that he would not get his way in the Senate. Her current idea was too small. But she and her girls were the only ones in Riosk with a good understanding of the peoples of the Janma Basin, and Kripna was a talented charismatic girl, about to be married to her own people. She should be able to use that.
	
	She focused on taking long steady breaths until an idea started to form in her head. She examined, critiqued and adjusted it from all sides carefully until she was certain that she could play the following moves correctly. She would rather have waited until she had filled in the missing details and discussed it with Hazda before presenting it to Darius, but Hazda was not here.
	
	“I have an idea.” When Darius raised his eyebrows, she continued. “Let me go to the Janma basin.”
	
	
	“What?!” 
	
	“When Adya Bir came to complain about the Senator Karim’s treatment of his people, he also asked me to help him in negotiations he is involved in with his neighbors the Frit.” It was only a small lie. “I haven’t answered him yet.”
	
	“No.”
	
	
	“Hear me out, Darius. The conflict is over hunting rights in the wetlands. It is a trivial matter. But the senator is here. He cannot interfere. Right now, they do not trust Sardona. Alone in the wetlands, I can change that. I can make them trust me. If you fail your vote, I will let them know what the Senator is planning. I will give them a chance to prepare.” 
	
	“To what end? That won’t stop annexation.”
	
	“No, but it will give them a chance to prepare, maybe even, to unite. If you loose the vote, it will take the military several months at least to prepare for an operation this large. I can do a lot with that time.” 
	
	Darius put up his hand to stop her. “Even if you could work a miracle in the basin, which I am not granting, you would be uniting a group of rag-tag hunters and raiders. Sardona would just reconquer them.”
	
	“Are you so certain? Securing just the trade route to Xi took years. And the tribes were scattered and disorganized. Now, we work with many of their sons to help them keep their own peace. Surely they have learnt somethings from our great Saran militatry. In the mean while, we have learnt little of how to extend supply lines across vast swaths of grasslands without a village or farm to feed off of. One by one, I’ll grant you, Sardona will certainly succeed, but united?”
	
	Darius gaped at her, slack jawed. “Surely you are not suggesting this. This is. This is. Its treason.”
	
	The word frightened her. She could not deny that. If she knew what he would do to her if she displeased him. She could not allow that to distract her. Darius was distraught, not angry yet. She composed her face to one she would use to talk to a frightened child. “It is only treason if it goes against your interests, Darius.” 
	
	
	“Which it does. It would completely destroy the river route to Xi..” 
	
	“Does it? Yes, the river route will be disrupted. But a long expensive, ineffective military campaign? Surely you can use that in your favor to convince the court that expansion may be a thing of the past? When the dust settles, who knows. Maybe we can go back to the basin, so that you can negotiate a more stable route treaty with all the tribes.”
	
	Darius stared at her, disbelieving. Reina waited, nervous.  At least he wasn’t calling it treason.
	
	“It's a horrible plan.” he said at last.
	
	Of course it is a horrible plan. I’ve only had three minutes to come up with it. But it was a long site better than doing nothing. Besides, she didn’t need him to like the plan, she just needed him to work with her. “Then improve it.” she countered.
	
	Darius pondered the proposal for a long time. After a while, he rose and paced the room. Reina sat on her knees, her hands carefully placed on her knees, immobile. She practiced the tricks she taught her girls to keep level headed and attentive in negotiations with their husbands. She counted his strides across the room. She counted the time it took him to cross the room. Then she calculated his speed, and how long it would take for Darius to walk at that speed back to Tabrisi, where she wished he had stayed.
	
	When Darius finally said “Not tonight,” it felt like a non-sequitur.
	
	“What?”
	
	“I don’t want to talk about this tonight.” It was her turn to stare at him in confusion. He had just wrestled the game out of her control, and she had no idea where he was going to take it. The muscles in her shoulders tensed.
	
	But Darius was not in the mood to enlighten her. He silently picked up the cups they had been using, emptied them, and carefully wiped them clean with a fresh rag. His movements were slow and precise as ever, but his face looked unhappy. It was enough to make Reina wonder again if Darius was afraid of Senator Karim. 
	
	Then Darius walked over to one of the chests still unpacked from his travels and took out an urn wrapped in straw. He dusted off the straw then crossed the room to wet a rag. Darius liked to keep her waiting, Reina knew from long experience. He knew it bothered her, and whether he did it because of that or out of indifference to her did not matter. He was running the game now, and he would keep her waiting. After he had cleaned and unsealed the urn, he poured out two cups of golden liquid. 
	
	When he came back to her, his demeanor had completely changed. “How long has it been since you visited me in the palace like this? Two months, three? Sardona can wait. We still have a victory to celebrate” He handed her a cup and kissed her fully on the mouth. “I’m not saying no,” he murmured, rubbing the silk and down her back. “I’m saying, convince me in the morning.” 
	
	Reina was not ready for this sudden change of mood. She was certainly not ready to be touched like this again so suddenly. But Darius was her husband. Not being ready was not on offer at the moment. She brought the glass to her nose to buy herself time to make the adjustment.
	
	“Apricot brandy..” Darius said.
	
	And in spite of herself, Reina released an exasperated laugh. “From your orchards in Al Usbaliya, I know.” Whether the appropriate phrase for the situation was running joke or constant irritation was a matter of perspective. She hated the sticky overly sweet liqueur. Darius insisted that it was a delicacy from his home region and that refusal would be rude. She supposed it amused him to watch her swallow the liqueur down without making a face.
	
	Reina raised her cup to the health of the Kokhanat and his Regent, and drank it down in two large gulps. It tasted slightly off, she thought, but it had been traveling. She had probably swallowed too much of the sediment. At any rate, she was ready now to meet this new mood. She pulled at a pin at her waist, letting the silk  fall off her body. 
	
	It was only after Darius had dressed her in a man’s kameez, tied her hands to a bedpost and penetrated from behind that she felt the all too familiar wave of euphoric pleasure rise and crest in her. 
	
	It had absolutely nothing to do with how her husband was touching her. And there was absolutely nothing she could do about it in her current position.
	
	***************************
	End here
	******************************
	
	The next thing she was aware of, she was lying on a bed in a small dark room on drenched sheets. No. Not dark. A single shaft of light entered the room that hit her directly in the eye. Every bone in her body ached. The bed moved. She groaned. 
	
	The door opened in a head splitting splash of light. Kripna entered and closed the door quickly behind her. “Maya,” she whispered, “Are you feeling better?”
	
	“Where am I?” she rasped. She was parched she realized. Kripna helped her sit up and produced a ladle of water from Saguhr knew where. 
	
	“We are on your barge on the Janma River.” 
	
	She tried to force the words into some semblance of sense but failed. The Janma River was three days by carriage from Riosk. “H-How?” was all she could manage. 
	
	Kripna handed her another ladle of water. Then another then another, until she motioned her to stop. “There is night soil bucket under the bed. Knock when you are done.” 
	
	Kripna stepped out and the Kokhan Mata realized that it had been eons since she had emptied herself. 
	
	When Kripna returned, she busied herself with tidying the small room and laying out a platter of dried and candied fruits, a variety of hard and fresh breads and several piles of salted nuts. It irritated her that she was being treated like a child. “Why am I on the Janma, Kripna? Where are we going?”
	
	Kripna did not make eye contact. “We are going to the Hasti, Maya. Adya Bir will meet us south of the wetlands and escort you to where the Punde Bir has agreed to meet you for negotiations. Do you not remember?”
	
	The Kokhan Mata sat back and let it sink in. She did not remember. And Kripna sounded on the verge of tears. She contemplated her options then decided the girl deserved her honesty. “No. How long have I been,” out, she wanted to say, but chose “travelling.” 
	
	“We carried your litter by horseback to Sha’ham, which only took a day and a night when we changed horses 3 times. The barge launched before noon on the third day of Badr’s rebirth, and it is now the morning of the fifth. Adya Bir will meet us in another week.”
	
	There were so many other questions she wanted to ask, like how they had smuggled her out of the capitol or why Adya Bir had agreed to this kidnapping farce, but they seemed too complicated. Instead she played with the fruit on the platter before her and asked “Who else is with us?”
	
	“Hazda’s Leon and a handful of his men are providing the protection for this trip.” When Kokhan Mata opened her mouth to protest, Kripna interrupted, “They are being the most perfect of gentlemen. Hazda made certain that Leon saw to that. There are the rowers and the crew, who we avoid as much as possible, and,” Kripna paused, “Elanya is with us.” 
	
	Kripna must have seen the look of surprise on the Kokhan Mata’s face because she started wailing “I tried to apologize as you asked, Maya. I promise I did. But she would not hear of it. In the end, Hazda told Leon to take her, partially to punish me, and partially because she did not know what to do with her.”
	
	The wailing made the Kokhan Mata’s headache worse. She reached over and squeezed Kripna’s shoulder. “I will deal with her, Kripna. Be strong for just a few more days, and then I will deal with her.” 
	
	The sobbing stopped and the Kokhan Mata sat back to hold her throbbing temples. By Sur and Uhr, she ached in every possible way. After a while Kripna moved to her side and pressed a skin to her lips. Milk. She grasped the bag from her hands and sucked the sacrine fluid down like her life depended on it. Her life did depend on it. Every ache in her body told her so. When the skin emptied, she curled up sucking on it, like a babe with a teat. Kripna sighed sadly. She tucked the stray rag back into the shutter, picked up the tray of barely touched food and left.
	
	The Kokhan Mata slept, and woke, and slept, and woke, and realized that she was hungry by noon the next day. Kripna was overjoyed to see that she had remembered the previous day’s conversation.
	
	Over half a round of dry flatbread and a handful of raisins, she learnt that Gris had raised the alarm when she hadn’t returned, and Hazda had found both her and the Regent unrousable in his rooms. The lack of eunuchs guarding the Regents door made it easy enough to get her out of the palace, and a hay cart took her out of Riosk. Adya Bir was simply told that she was visting in secret without knowledge of Senator Karim or the Regent to honor his right as a protectorate, and that if he wanted her help, he should make the requisite arrangements with the Frit. Simple. And Hazda was brilliant.
	
	When she asked for the milk, though, Kripna flinched. She muttered something about running low, put the skin on the bed between them and ran out the door. When the Kokhan Mata tasted it, she understood why. The medicine was diluted and bitter. She spit out the first mouthful and crossed over to the door to demand better, but found it locked from the outside. Hazda, she realized. Of course the Sardona bitch would want her sober for the negotiations. The cursed people would all shrivel and die if a woman ever found pleasure. “Hazda” she cried out loud, then beat the door and screamed until her voice was hoarse. Then she went back to the bed and groped around for the skin. She found it on the floor next to the water. She sobbed in relief and drank a mouthful of the horrible draft.
	
	The rest of that day and all of the next passed in a haze of yelling and gulps of the bitter drug. The skin was empty by the morning, but she still sucked at it occasionally on the off chance she had missed a drop. Her stomach hurt, and she did not remember when she had eaten last. She vomited bile, found the water bucket, drank, vomited water, then vomited bile again. Twice she screamed to the universe at large that the night soil bucket was full. The first time Leon came in to empty it, she tried to push past him and leave her prison. He simply picked her up with one hand while removing the bucket with the other. When he left, she told the door that she wasn’t afraid of him, even if he was as ugly as the bottom side of a barnacled shipwreck. The second time, however, she stayed in her corner.
	
	The third day, she was still irritable, but hungry. More importantly, she was lonely. Kripna came with a book and a bowl of broth and read while the Kokhan Mata sipped at her bowl. She must look like a hag, she realized. Later in the day, she donned an abaya and veil and walked a few times around the deck. 
	
	“Will you go back?” Kripna asked as the Kokhan Mata rested on a coil of rope to catch her breath. 
	
	“Back?” She asked, genuinely confused. 
	
	“To Riosk,” Kripna explained, then whispered, “to the Regent.” 
	
	The Kokhan Mata laughed. “Why would I ever risk my life and yours to visit the Hasti if I did not intend to return to the capitol?”
	
	Kripna knelt before her and grabbed the Kokhan Mata’s hands in supplication. “You can’t, Maya. Not after what he’s done. Not after how you’ve suffered this last week. Please don’t Maya.” She was crying again. The Kokhan Mata just held her. She and Hazda had asked too much of this young woman. Training aside, she had never actually nursed an illness or run a staff of domestics. She had certainly never seen what a husband could do. Having to watch over her like that for the first half of the journey must have been terrifying for her. Eventually, the girl calmed and wiped her eyes. “Maya, please say you won’t. You could,” she cast around desperately for a plan, “you could come live with me. I would make certain you lived in comfort. I would cater to your every whim. I would make my husband agree.”
	
	“Oh sweet potato.” The Kokhan Mata sighed, and held Kripna tight. “You are a good girl for wanting to keep your Maya comfortable in her old age. But I’m not a widow yet.” She stroked her hair until the girl stopped shaking. “Where did you get the idea that one should expect safety and comfort in a marriage? Certainly not from me.” Kripna shook her head against her shoulder. “You are marrying Adya Bir because he will give you freedom.” She pulled away from Kripna and took her face in her arms. “You know that, right?” The miserable girl nodded. “And I have married the Regent Kokan for the same reason.” 
	
	“But you nearly died..” 
	
	“Hush child. I didn’t.” the Kokhan Mata interrupted. She didn’t want to hear it. “Look around you. The barge staff here are all men, and we are sitting in full view under the open sky. Where are the eunuchs? Where is the disapproving court?” She waved a dismissive hand in Leon’s direction. He had been watching her walk the deck all afternoon, but as far as she could tell, his main motivation had been to be on hand if Kripna needed physical help to support the Kokhan Mata. “Leon’s men are body guards, working for me. We have no minders?” 
	
	“But, Maya. Your torment these last few days.” 
	
	“Would not have been so bad if Hazda had been willing to wean me more slowly,” the Kokhan Mata interrupted. “No, Kripna. I will go back. Riosk is the only home I know, and more importantly, it is the only home I choose.” She waved a hand over her recovering body. “Opium sickness and all.”
	
	Leon indicated that dinner had been served in the Kokhan Mata’s quarter. The returned and ate, and Kripna read by oil lamp. The Kokhan Mata was doing well enough that she would not pass out from exhaustion as she had done on previous days, but not well enough that she could benefit from the full pleasures of Kripna’s company. So Kripna read, the Kokhan Mata did not know for how long. When she awoke, she found the girl curled protectively around her body.
	
	***************************************
	
	As she had spent a day with Kripna, and as Elanya was traveling with them, she spent the next day with the younger girl. The child was docile all morning, serving her food, dressing her, walking with her on the deck and reviewing her lessons. The Kokhan Mata wondered if Kripna really had gotten through to her. More likely, if what Kripna had said was true, and she really had nearly died of opium poisoning, Elanya might just be scared.
	
	After lunch, having converted to the northern river tribe rhythm of life, they sat by the coils of rope, as the Kokhan Mata laid out a map of the Janma basin. At the first mention of the Hasti, Elanya’s sank into her very familial sulk. The Kokhan Mata was tired. She was weak. She physically could not have the long verbal matches that accompanied this move. So she took a different approach.
	
	She pushed aside the rope, and cleared the space between them. “What do you want, Elanya?” 
	The unexpectedness of the question surprised Elanya into making eye contact. “We are stuck here on a boat together. It will be at least two weeks until we return to Riosk. What do you wnat with this time?”
	
	The girl considered. The expression on her face told Kokhan Mata that she knew what she wanted, but was considering whether or not to tell her. To put her thumb on the scale, she added. “I promise not to punish you for how you answer.” She picked up a splinter of wood, spit on it, and tossed it over her right shoulder into the river. “See? Promise.”
	
	The performance was enough to convince the girl. “I want to destroy the Hasti,” she mumbled. The Kokhan Mata almost laughed at the ambition, it was so preposterous. But if there was a real drive there for her to work with, this could be how she shaped the girl. Purely for form’s sake, she asked the girl to speak up.
	
	“I want to destroy the Hasti,” she repeated, with a real defiance in her voice that amused Kokhan Mata greatly. 
	
	“Very well, how?” Elanya’s face fell. The girl clearly had not thought it through her plan this far. The Kokhan Mata put on a patient air. “Here’s what I want you to do. I am tired. I am going inside to rest. You do not have to accompany me if you do not want to. When I am done resting, you will tell me your plan, and I will help you refine it.”
	
	“You will?” The girl stepped quickly from awe to suspicion. “Why?”
	
	The Kokahn Mata smiled. “You want to destroy your enemies. I want you to learn to scheme in a harem. I do not see how we are at odds.” 
	
	Realization dawned on Elanya. “You don’t think I can do this!” she cried.
	
	“I don’t think you can do this, yet.” The Kokhan Mata corrected. “But I do think you can learn how, if you are willing to work with me.” That answer did not fully mollify the girl who still sensed that she was being positioned someplace she did not want to be. So Kokhan Mata asked “Is there anything else you want?”
	
	Elanya looked up. “I won’t be punished?”
	
	Kokhan Mata smiled and shook her head. “I promised Janma.”
	
	Elanya raised her chin. “I want to destroy the harem.”  
	
	With flash of familiarity, the Kokhan Mata suddenly wanted to hug the child. She knew what the poor fawn was going through, and she knew what her road ahead would look like. She would not have wished that future on any one of her girls. But that was not a gift that was hers to grant. 
	
	“An admirable goal,” she said, then paused for that to sink in. “I will not insult you by suggesting that you should consult with me on the details of your plan. But I encourage you to develop it, all the same.” I am your enemy, child, as your husband will be after me. It is better that you come to it early, on your own terms, than realize it too late.
	
	Confused into obedience, Elanya helped the Kokhan Mata walk to her berth, then stepped away to take a crack at the task she had been assigned.
	
	***************************************
	
	Inevitably, she found herself awake before dawn and unable to fall back asleep. She left her berth to take the air on deck. They had come significantly north during their journeys, and the nights, while warm, were no longer stifling. She Stood on the deck, and let lungfulls of cool air soothe her.
	
	“It is slippery, Kokhan Mata, let me offer you support.” She found Leon’s arm extended by her side. She took it. She had no idea that he had been keeping watch.
	
	They walked. “It seems, Sergeant, that I owe you my life.”
	
	The large man gave a non-challant shrug. “I just follow orders.The thanks goes to Hazda.”
	
	They walked the length of the barge, turned, and walked half way back, while she considered whether or not she wanted to know what had happened to her. It would not be pleasant to learn of how one had nearly died, if it wasn’t an exaggeration on the part of the girls. On the other hand, it would be easier to face Darius with the knowledge of exactly what had transpired.
	
	“Kripna and Elanya have not wanted to discuss that night. What happened.” 
	
	The Sergeant swallowed. “I was, um, with Hazda when one of the girls realized that it was nearly light, and you had not sent word. We found you in the Regents bedroom,” the man blushed, and the Kokhan Mata recalled what she had been dressed as.
	
	“Go on, Sergeant. I trust your discretion.” 
	
	“Well, we couldn’t wake either of you. You were still breathing, but we couldn’t find your pulse. Hazda had me get a unit doctor while she fed you some ash to make you vomit and meet her outside the south gate as soon as I could. She thought that you had taken enough opium to kill someone, well, of your size.” 
	
	The Kokhan Mata shrugged. She was a small woman, and there was no getting around that. She knew it. And Darius certainly did.
	
	“I don’t know what the doctor suspects. I had told him that I needed help for a relative, but the physic certainly knows that no relative of mine can afford a litter or three changes of horses to get to Sha’ham.”
	
	“Hazda will have paid him off, I’m sure.” It was not a guarantee of discretion, but there was nothing she could do about it.
	
	The Kokhan Mata rested against the trafail. While the worst was over, and she became stronger every day, she still could only circle the boat a handful of times. She reached around her neck and took off a thick gold chain that she didn’t wear exactly. Rather, she never bothered to take it off.
	
	“I want to give you something, Sergeant,” and put the chain in the other man’s hand and held it closed with her fingers. 
	
	The man balked. “Kokhan Mata, I can’t,”
	
	“This is not a repayment of an obligation, Sergeant. I can never do that. This is a whim. Marry Hazda. Take her a away from Riosk. Make her a happy woman.” There was so much she depended on Hazda for. And she could never repay her for her, well it wasn’t love, she had made that clear. She could never repay her for whatever it was that she had given her over the years. But perhaps, if she gave her freedom, it would be a start.
	
	The sergeant kept protesting. “I can’t Kokhan Mata. She won’t have me.” The Kokhan Mata’s hand fell away in surprise. “I’ve asked her. Many times. She will not marry me.” For one brief bright  moment, a sprig of hope that Hazda did love her after all bloomed in the Kokhan Mata’s chest before she plucked it out with disgust. “She says that, while she is with you, she has her pick of a dozen of the best households in Riosk to take care of her during her dotage. What I could possibly offer that would rival that.” 
	
	The Kokhan Mata burst out laughing. Hazda had been her first companion, and the first person she had found to love in Sardona. She had been a broken shard of a woman then, and she remembered Hazda nursing her spirit back to strength. But Hazda had wanted power, as the Kokhan Mata kept forgetting, and they had learnt to grasp it together. In so many ways, she had been the first of the Kokhani harem. 
	
	She wiped her eyes and caught her breath. “She said that, did she. To your face?”
	
	“Yes.” The sergeant chuckled and returned the chain to her. 
	
	“Very well, let me at least find you a better position in Riosk so you can keep her in better trinkets.” 
	
	This Leon would accept, and they walked slowly back to her berth. As they parted, he asked “If I may, Kokhan Mata, why did you do it? Are you not happy at court?”
	
	The question took her by surprise and the truth inside her roared like a caged lion. “Did what, sergeant?” she asked to buy herself more time.
	
	“Suicide,” the man whispered. “Isn’t that….”
	
	The missing details crystalized in her mind. Kripna hadn’t told him what she had correctly surmised. She shook her head and gave a reassuring smile. “No sergeant, it was not suicide. I have every desire to live my life. On the contrary,” she searched for the best choice of words “I believe that we were poisoned.” 
	
	If Hazda trusted this man, then she trusted this man. But there was no reason to go looking for trouble. Better to take any suspicion off the Regent while she was at it.
	
	She went inside. She didn’t know why she was protecting the ancestorless bastard. Not if he had done what she suspected he had done.
	
	***************************************
	
	By the time Adya Bir and Punde Bir met them on the Frit side of the river, just north of where Senator Karim’s troops were stationed, the Kokhan Mata was not quite strong enough to mediate negotiations, but she was strong enough to pretend.
	
	They had laid out luxurious tents for the Kokhan Mata and her girls, the type used by both tribes when they moved south with the herd in winter on the Hasti side of the river. The negotiations were to take place in a similar structure on the Frit side. When the barges crew rowed the girls and Leon across the river to ready their accommodation, the young man’s eyes lingered for far to long. He was smitten, the Kokhan Mata realized. He was going to be useful.
	
	The negotiations started with an invocation to the river Janma, asking for her blessing and praying that she treat both sides fairly. Punde Bir had clearly put on the performance with the intent of making her uncomfortable. The Saran people have a particular opposition to other religions, especially of the people they subjugate. Even the Regent Kokhan, who saw the power of trade and knowledge exchange over subjugation, destruction and theft, who was, by far, the most tolerant of other cultures of any ruler for several generations, would have been made uncomfortable being asked to partake in a ritual to a foreign god. But the Kokhan Mata was not Sardonan. And she had watched this service performed many times at home. When Punde Bir stood before her with a plate of offering from the altar, she took from the platter exactly three slices of the sacrificed goat and put exactly two silver coins on the tray. “I pray for fertility on both banks,” she said and smiled. She would take the invocation for peace at face value, and deny Punde Bir his opening position.
	
	The negotiation party consisted of Adya Bir and Punde Bir and four patriarchs from each side. The men sat in a half moon in the tent while she sat at the northmost side on a small dias. Periodically, and young boy would appear with platters of jerky, or cups of tea. It felt very much like a Sardonan throne room but for the fact that the two Birs sat in equal positions of honor, facing each other, on her left and right. No one sat on the Kokhan’s left, unless it was to be stripped of a title or be sent into exile. 
	
	She started the negotiations in Sardonan style, asking each side for their list of grievances. 
	
	“The Frit were crossing the river into Hasti lands, and act strictly prohibited by the latest treaty between the tribes.”
	
	“Treaties expire with the deaths of the signatories. Hasti were culling Frit herds, prohibited by tradition for centuries before there were any treaties between the peoples.”
	
	Then she read from a Sardonan history what the empire knew of the traditional relations between the Frit and that Hasti. When she closed the book, the Birs took their turns clarifying their different perspective. 
	
	The first few hours of the negotiation she did by rote. It was rigid and codified, tradition dictating who could say what and for how long. It was a ritual of memory, remember the past before considering the present. Then, the group broke for refreshments, each staying cleanly separated from the other. When they reconvened for more freeform negotiations, the proceedings dragged. The Kokhan Mata was exhausted from the journey and her weakened state. The food in her belly was richer than the fare she had been eating in her recovery. It made her head spin. By mid afternoon, she wanted nothing better than to lie down with a cold cloth over her face. She was supposed to listen, she told herself. This first meeting was a chance for her to learn the motivations of both parties. The ones they would admit to, and those they held close under their shifts. And she tried to parse their words, but her head was fuzzy, and her legs ached. She was simply tired.
	
	“The protectorate treaty does not dissolve the moment we go to war,” Adya Bir was explaining, clearly not for the first time. “The Riosk had already sent and army to their doorsteps. Karim’s men could easily impose martial law and tax us dry to support the troops. Not to mention any further pecuniary payments they may further demand. It will destroy us both, as certain as a famine.” 
	
	That didn’t sound right. Protectorate treaties did dissolve the moment a protectorate violated it. Attacking another protectorate was certainly a violation. That was the entire point. A region remained a protectorate until it violated the treaty. If it ever stepped out of line, Saroda stepped in and annexed it. Suddenly, she saw Senator Karim’s plan. He hadn’t antagonized Adya Bir out of his greater than average Sardonan disdain for the outer territories. He was trying to start a conflict. If the vote in the Senate didn’t go his way, his men were already here. He could annex the region and demand the Governorship for his reward. Who would deny him? The Regent Kokhan certainly could not. Oh, the man was dangerous. She had to win this peace.
	
	“The Senator’s men will not stop you from fighting.” 
	
	“I beg the Kokhan Mata’s pardon?”
	
	Every eye in the room was on her now. “I know the Senator,” she explained and began spinning a tale that would buy her room to maneuvre. “He was a general in the previous Kokhan’s court. He enjoys subjugation. He does not hold with the current regime's preference for protectorates and tributaries. He believes that Sardona’s destiny is to span the path of Sur.” She paused and looked around the room to make certain that everyone listening had a proper understanding of what Sardona’s hunger used to be, and what it could so easily become again. Nobody moved. She could hear a fly from across the tent. “Fighting between protectorates is in violation of the treaty. The offending parties are no longer eligible for the privilege of protectorate status, and Sardona is free to annex them.
	
	“When Adya Bir asked for the Senator to step in, as had been his right, he insulted you, did he not?” She made eye contact with Punde Bir. “You, and Honored Bir, who have led your people for fifteen years, a signatory to the protectorate treaty. He called you names to your face.” She looked at the other patriarchs present. “Why would he do that?” The men mumbled and whispered.
	
	Then one of them spoke. “With all due respect, Kokhan Mata. How do you know this? So much of this was before your time.” 
	
	There was so much to undercut her in that question. And none of it deserved her attention. “You are right, Father of Vilal. The worst of the Sardonan expansion had stopped by the time I came to Riosk. And Senator Karim had already been a general for much of it. That is exactly my point.” She looked at Adya Bir. “He is not a spring sapling. Yet even this new born leader showed more sense in brokering a future for his people than the veteran Karim did in protecting the empire’s peace? It beggars belief.” 
	
	The men broke into several conversations that the Kokhan Mata was too tired to follow, so she just sat and waited. Thankfully, within half an hour, both Bir’s called for an early halt to the negotiations so that they could both deliberate on the new information.
	
	*******************************
	
	The second day of negotiations was more productive than the first. With the threat of subjugation looming, there was a renewed desire to renegotiate existing treaties in light of the new herd movements. It would take time, but the Kokhan Mata had faith that it would conclude well.
	
	The next morning was a scheduled break in the talks to mark the Adya Bir’s wedding. It was a well put together plan on the grooms part. Prominent families from both tribes would be attending, almost as if he had married a Frit Birna. Such a mixing of Frit and Hasti happened once a generation. The entire plan made Leon extremely nervous.
	
	She awoke with abdominal cramps and the sudden need to relieve herself. Aside from a general weakness, her inability to hold her night soil was the last remaining symptom of her opium sickness. As there was only one solution to this particular problem, she carefully climbed over the sleeping Elanya next to her and tiptoed out of the tent and into the bushes. 
	
	“Maya.” The Kokhan Mata jumped. She had not expected to see Kripna standing inside the flap of her tent.
	
	“Kripna, why aren’t you with your parents, child?”
	
	“I wanted to see you. But you weren’t in your bed. Where had you gone?” The girls sounded nearly hysterical. She had been sounding more and more erratic, during her time out of the harem. It made the Kokhan Mata question how ready she truly was for the role she needed her to play.
	
	“Night soil, child. Pull yourself together. A Bir’s wife does not go sneaking into the night without a very good reason. Why are you here?”
	
	“Are you giving me away tomorrow?”
	
	“Of course.” 
	
	“Then you are my Maya and my Dura. They agreed to that when they gave me to your care. This is my family, and they need to understand that.”
	
	Dear Maya Janma, sweep me away from another daughter fighting with her parents about her marriage. The Kokhan Mata sighed. “What happened?” 
	
	“It doesn’t matter.” She walked to the back of the tent, into the room that they had occupied together last night, and threw herself onto the bed. The Kokhan Mata sat down beside her and waited. Eventually Kripna sighed in the dark and spoke. “Dura is angry because he could have married his daughter to the Bir of Hasti without any interference from Sardona, and Maya won’t… I mean my other Maya. Not…” she sat up suddenly and fumbled around until she finally found the Kokahn Mata and held her. “Please let me stay Maya. I can be like Hazda. I don’t need to marry ever. I can help you take care of the girl. Please Maya, I want to serve.”
	
	The Kokhan Mata got up suddenly and left the room. The tent had three partitioned areas in it, probably to accommodate three occupants. The two back rooms served as the bedrooms, and the front as an antechamber. Kokhan Mata made herself comfortable on a pile of buffalo hides, lit an oil lamp and started reviewing local histories she had been able to gather from the two tribes. Her efforts notwithstanding, one out of every three of her alumni broke down the night before their wedding. She had to trust that she had given them the tools to come to terms themselves. She had raised them to be self reliant.
	
	She had read three pages of text when Kripna came in and sat down beside her. “I’m sorry Maya. I know that each of the alumni serve. I am certain you did not marry my back in my homeland because you wanted me far from you. I trust that you will tell me what it is in time.”
	
	Oh, Kripna. He will eat you for lunch if this is how you apologize after a fight. The Kokhan Mata waived a hand in the direction of the bedroom without looking up. “Try that again, as you would with your husband.”
	
	Kripna’s mouth fell open for a second. Then she got up slowly and walked towards the back of the tent. After three steps, she stopped. “No.”
	
	The Kokhan Mata put the papers down and looked up. “No?”
	
	“No Maya. You are not my husband. I will not fear you or position you or protect my children from you. That life starts tomorrow. Today, you are my friend.” She walked back over to the table where her Maya was sitting. Picked up the sheaf of papers she was reading, neatened, rolled and put the bundle away. The Kokhan Mata grinned. She taught her girls to understand what they desired and then find a way to procure it. Not every girl mastered this before they left her hand, but most did. She enjoyed watching the manifestation bloom.
	
	“Very well, friend. What would you have of me.”
	
	Kripna filled the oil lamp with more oil and made herself comfortable. “Tell me a story, Maya. I know so little about you. Tell me the story about how you came to become Kokhani.”
	
	The Kokhan Mata’s grin fell off her face. “No,” she said flatly. “Ask me anything else.”
	
	**********************************
	
	“Once upon a time, there was a Hraden and Rhayina in Estrad.” 
	
	“What were their names?”
	
	“Hush child. You take the story as I will tell you, or not at all.”
	
	“Once upon a time, there was a Hraden and Rhayina in Estrad. They had three children, a daughter and two sons. On the whole, they were a happy family, as happy as anyone can hope for. They lived in a mountainous land that the Hraden claimed made their strongholds impenetrable. So the Hraden stopped worrying about foreigners attacking his people and focused on trading with his neighbors.
	
	“This family was not the only ruling family Estrad, but through a series of coincidences, kidnappings, manipulations and dumb luck, the Hraden found himself in a position where the other rulers owed him a debt of allegiance. So he collected, and formed a confederacy of kings that met every year to settle matters of security and trade that affected all members between themselves. The Hraden had not intended for this alliance to threaten Sardona. His interest had been to negotiate fishing rights with Barkus, or maintain a network of roads across the island or secure better agreements with rich neighbors like Astrad.
	
	“But Sardona had long been interested in Astria. Sardona’s border to Astria was across a dangerous mountain range. But from the south east of Estrad? That was just a short channel crossing. Sardona did not like it when countries it is interested in become stronger. It viewed any such action as a prelude to war. And so, instead of Sardona worrying just one king of Estrad, the empire worried all four.”
	
	“The Hraden did not want war with the east. His people had grown rich and comfortable with peace, and he did not want that to change. He sent a diplomat to Sardona, and initiated a conversation. He had a daughter, after all, and she was not so young that marriage could not be contemplated.
	
	“Within a year, the empire sent Darius ben Nabab to Estrad.” 
	
	“Why him? He was the Minster for Domestic Affairs.” The Kokhan Mata shot her a look, and Kripna put a hand over her mouth. 
	
	“Within a year,” the Kokhan Mata stumbled. She could not say his name again. “As you say. The Minister for Domestic Affairs arrived in Estrad. He stayed with each of the four kings, ending his visit with the jewel of the island, the Hraden’s Hall.
	
	“The minister agreed that marriage to an Estrian house was an excellent idea. There had been two other offers of marriage, and while he would consider the Hraden’s daughter as well, a marriage was preferable to an engagement. However, the riches of Hradenhel impressed him, and he was extremely interested in trade.
	
	“The negotiations never concluded. During a hunting trip, a wild rabid dog bit the Hraden, and he was dead within the fortnight.” She was whispering now. She could not find her voice.
	
	“That soon? It must have been a bad bite.” Kripna could not help but interrupt.
	
	The Kokhani Mata shook her head. “The Rhayina killed him.” 
	
	“But why? I thought you said they were happy together.”
	
	“Stupid child,” The Kokhani Mata spat. “Murder does not imply hatred any more than marriage implies love. Where do you get these ideas. 
	
	“He was a  large man, strong and proud. He was hosting a foreign dignitary. Once it was clear that he was infected, he did not want to waste away. It was better to pass suddenly while in his prime than to let his enemies see him weak.
	
	“A thick enough wolfsbane syrup is nearly instantaneous and relatively painless. They discussed it as soon as it was clear the wolf had given him the disease. She was to use her judgment on when his illness would humiliate the throne. She acted sooner. She did not want to judge him. She couldn’t bear the wait. It only took a few drops.
	
	Kripna took her hands. “You loved him.”
	
	The Kokhani Mata nodded. “More than even my children.”
	
	When the older woman did not speak, Kripna picked up the tale. “And Darius ben Nebab decided that an alliance with Estrad in the form of the new young widow would please the Kokhan best. You had to choose a Saran name for your conversion. You chose Reina to honor your family.” She paused. “How did non-aggression with Sardona become part of the marriage alliance. Did you do that?”
	
	The Kokhan Mata shook her head. “The Hraden signed the marriage alliance with the Kokhan’s brother the day after we agreed upon his poisoning. He chose the words carefully. The alliance was not just for the life of the Kokhan. It was for as long as I remained a Nebabi. He trusted that this was enough for me to work with.”
	
	Suddenly, Kripna understood. “Maya Janma. That is why you have to return to Riosk. You married Darius for freedom. Not yours, but Estrad’s.”
	
	*******************
	
	They awoke just before dawn with eyes red and swollen from last night’s crying. While Kripna tidied up the furs and washed the tear streaks from her face, the Kikhan Mata went to check on Elanya. The girl was not in her bed.
	
	Kripna figured that it was just the excitement of the day. Leon would be going to the Frit to escort them to the wedding. She would have to be up and dressed early if she wanted to join them. 
	
	“She had to walk past us to leave the tent. She didn’t make a fuss.” 
	
	“Maya. She is ten, or less. Given the choice between a party and a scene.” Kripna shrugged, but asked Leon’s men to keep an eye out for her, just in case.
	
	They found her on the way to the river to bathe. She had already bathed and dressed, on her way back to the village. 
	
	“Well aren’t you a shiny little carp.” Kripna called. “Where were you?” 
	
	Elanya spun around to show off how the heavy fur trim of her skirts caused it to drape around her ankles. Then she bowed. “I couldn’t sleep, so I started early. I hope you are serving tilapia Elanya. Are you?”
	
	The Kokhan Mata did not trust this ebullience. The little girl was up to something, and she would have to get to the bottom of it. Later. After the wedding. “Go tell Leon where you are. His men are looking for you,” said the Kokhan Mata.
	
	“Oh.” Elanya stopped preening. “Why?”
	
	“He has to leave soon to bring the Frit delegation over. Hurry if you want to join him,” Kripna supplied. 
	
	“Oh, that. I’m not going. I’m going to join the Hasti party.” She turned and looked solemnly at the Kokhan Mata. “I think you are right, Maya. If I’m going to destroy them, I have to learn what they are like.” 
	
	*************************
	
	In spite of the drama of the previous night, the day itself passed pleasantly. A large green near the village had been cleared for games, refreshments and general entertainment, while a large pavilion stood near the center of the village for the ceremony and the feast.
	
	The Kokhan Mata made it a point to speak to Kripna’s family, assuring them that there was to be a second, larger, ceremony in Riosk in the months to come. Whatever the outcome of the Senator’s gubernatorial play, she would flout the fact that she had positioned herself right under his nose.
	
	A long list of Kripna’s friends that counted amongst the most prominent families in Saran went a long way to convince her father that he had, in fact, done right by his family by giving up his  daughter. And while it was always difficult to argue with a provincial mother made uncomfortable by inappropriate Saran beliefs, the Kokhan Mata knew that a chest of gifts to the wife's family usually went a long way to make the capitol's customs more tolerable.
	The Kokhan Mata oiled ruffled feathers, met prominent personages and gathered gossip about the Senator’s operations. She heard unfortunate but not unsurprising stories of the stationed soldiers at the wetlands, but nothing that amounted to a diplomatic incident. She did, however get a description of the regiment’s bursar, and a detailed description of his habits.
	
	True to her word, Elanya did not cause trouble. Instead, whenever she looked, the Kokan Mata found her throwing bones with the groom’s sister, watching the wrestlers practice for their match against their Frit counterpart or watching the the gondolas be decorated for the evening rides.
	
	The Kokhan Mata considered the proceedings carefully. By the standards of Riosk, the gala was provincial, almost childish. But from the faces of her companions, there was nothing lacking in what the Bir had prepared. It was an interesting insight. She would have to think about what it meant for her harem later.
	
	When a man could step over his own shadow, but the Frit hadn't arrived, the crowd started to grow restless. Whispers of snubbing and betrayal passed from lip to eye to ear. The observers of the negotiations gathered then suggested that the proceedings move to the pavilion to start the ceremony. The Frit may choose to sabotage the peace negotiations by not coming to the wedding, but there was no reason to delay the happy couple. 
	
	***************
	
	The Kokhan Mata sat cross legged to the left of the priest on the dias at the northern end of the pavilion. Adya Bir’s uncle sat on the right. The bride and groom sat before them, looking very solemn in a cloud of incense and burning hides. 
	
	She watched as the crowd filter in for the final, public part of the ceremony. Adya Bir and Kripna had been here all morning, being purified and offering thanks, receiving blessings and overseeing sacrifices. The pavilion was enormous, it could easily hold several hundred guests. But due to the proceedings of the last few hours, it felt small and stifling and closed. It stank of blood and tanning and burning hide. Her eyes stung and watered.
	
	After all but a few straggling Hasti had seated themselves before the dias, someone from outside yelled “The Frit! They are here.” 
	
	The pavilion erupted into cheers and crowds rushed out to welcome or watch their neighbors arrive. The Kokhan Mata took advantage of the moment of distraction to wipe at her eyes. Then she rubbed them again. Surely, she was seeing spots. There seemed to be patches of red and black appearing on the walls of the tent.
	
	Somebody yelled “Fire!” and the chaos in the tent increased tenfold.
	
	Adya Bir to her in one hand and his betrothed in another, and carved a path into the middle of the crowd towards the opening. But she was small by any standard. Someone jostled her too hard, and she slipped. She could hear Kripna calling her name, but someone was standing on her back and some else tripped on her legs. She couldn’t get up. Then she heard something come crashing down towards her.
	
	*****************
	
	When she opened her eyes, Darius was sitting on a stool next to her bed. His face was long and haggard. It looked like he hadn’t slept in weeks. She was still in a Hasti tent… her tent. Her entire left side burned and it hurt to breathe. Outside the tent, she hear the sound of soldiers and horses gather up villagers and generally throwing muscle around. She remembered the fire in the wedding, so most of those details made sense. Except … Darius.
	
	“Why are you here,” she wanted ask, “Why aren’t you sucking Senator Karim’s dick.” She was so overwhelmingly angry at him, she realized. This was not a good time for them to meet. But when she opened her mouth,”Why” was all she could whisper. Her throat hurt like someone had thrashed it with the laundry, and she was earth shatteringly tired.
	
	“Great Uhr, she’s awake” he bellowed, presumably to doctors waiting outside, and moved to take her right hand. 
	
	She jerked it away and shook her head. “Why” she croaked again, knowing that she should not be engaging. But here was Darius again, at her moment of desperate weakness, and she would suck the Janma dry before she would let him use her again.
	
	“Why?” the Regent clutched his hair. He looked mad, she thought. He looked like a man who had found his own death only to be denied the relief. “Because, look at you.” He waved his arms wildly. “All of this. This is what I had wanted to avoid all along.”
	
	Distantly, it occurred to her that she had seen that grief before, when she had lost the man she loved. Was it possible that Darius loved her? But a doctor had entered and pressed a pungent smelling cloth to her face. It was being harder and harder to stay awake.
	
	*****************************
	
	The next time she awoke, she was alone. Her left side still hurt, as did breathing. But the pain was localized to a few ribs. She found some bread and water, and found that she could swallow. She hummed a tune and found she had a voice. Encouraged, she tried to sit up, and her head screamed.
	
	She sat in bed and inspected the damage. Large swathes of her left leg and torso were bandaged. She moved them gently to find that she was burnt. A large area of her ribcage was bruised and swollen, and her head had a lump the size of a small potato. All in all, it could have been worse. She was still alive. 
	
	She found that she had been left a bucket for night soil, relieved herself, and tentatively walked out of the tent.
	
	Leon sat watch, whittling. It was well past dark. He stood up and bowed, but then did not did not get out of her way.
	
	“Are you here to keep me in, or others out?” 
	
	“Others out, Kokhan Mata, but,” the man looked embarrassed. “The doctors advised against your moving much. Your rib?”
	
	She started to laugh, then quickly decided against it. Then eased into the stool that he had been sitting on, and he looked much less uneasy. “You sound like Hazda.” 
	
	“I will take that as the complement I know it to be.” They both laughed.
	
	She watched him whittle in the darkness for a while, squinting at his work in the moonlight, cutting carefully, blowing, then squinting again. 
	
	“What happened today, Leon?” she asked finally, not knowing where else to start.
	
	He let out a long whistle. “Today was a manure train meeting a tornado, Kokhan Mata, and sometimes there is nothing you can do about it.” He sighed and put his knife away. “The Regent Kokhan’s barge appeared, along with half of Senator Karim’s men at about the same time that we were about to cross the river. I know that Hazda intended to try to delay him learning about your absence for as long as she could, but it seems she could not keep him away for more than a day or two.”
	
	“The vote. He couldn’t leave until the vote.”
	
	Leon shrugged. “But you know how it is. He wanted to know where you were and what you had been doing, and why I was leading a band of Frit into Hasti lands instead of guarding you. It all took … time. When he was satisfied, he joined the procession, as that seemed for him to be the fastest way to get to you without causing a diplomatic incident.
	
	We were almost in sight of the pavilion when a young lad, Bailas, was his name, ran off into the woods. I did not think much of it, but then I saw a handful of fiery arrows be fired, and a few seconds later, hear shouts of fire. 
	
	Fortunately, there were not a lot of deaths. A few other people were trampled under foot like you, and only an old woman did not survive her injuries. The tent had been soaked in fish oil the night before, so that it would light more quickly. But it was only a pavilion, so it burned out quickly, and there was limited damage. All in all, a fairly amateur plot.”
	
	He looked up quickly to see if he had caused any offense. When he hadn’t, he continued “We’ve caught the boy, and he’s given us names. The Regent has had them all rounded up, as well as the Birs, their families and all prominent officials.” Leon shuffled his feet again. “He, um, name Elanya. Shall I let her out?”
	
	The Kokhan Mata sighed. It made sense. Not being in bed in the morning. The false cheer in the face of what should have made her cry. Her desire to know the people she believed she was about to destroy. She knew the girl. Would she have done the same if she had come to Sardona at that age? Possibly. Only luck had saved her. “No. Keep her imprisoned. I’ll talk to her tomorrow.” 
	
	Leon looked relieved, and took his knife out, looked at it, and put it back again. “Kokhan Mata? There is something else you should know. They have been giving you the milk again. I cannot say how much. I know how much it means to Hazda that you stay sober. When you have healed, I would like to help you off of it? If you will trust me?”
	
	She stretched out an arm for him to help her up. “No hard feelings, Sergent. I’m sure that Hazda had her reasons for what happened on the trip over, and that she is well within her rights.” 
	
	She stopped before closing the flap of the tent. “One last thing. Who are you keeping out?”
	
	He looked surprised. “The Regent Kokhan. I don’t think he’s sleeping. He’s been coming by every hour or so all night. I expect to see him soon, if you’d care to wait.”
	
	She most certainly did not care to wait. “No thank you. I’ll see him in the morning.”
	
	
	********************************
	
	It was past noon by the time she found time to see Darius in his camp. She had talked to Elanya, who had, in fact, had her head turned by the son of a faction of Frit who wanted to use this opportunity to expand. She let her go. There was nothing more that she could do.
	
	Kripna was also imprisoned and worried, but otherwise unharmed. She had not been crying, she noted. Just firmly determined to defend her people against any slander that Darius would throw their way. The Kokhan Mata told her not to worry. Darius was angry, but he was not a rash man.
	
	She met with the Birs, who were two extremely apologetic men. She did not know what would become of them, but she promised to do her best.
	
	And she met with the commander Senator Karim’s battalion, who verified enough of Leon’s story for her to believe that he was telling her the truth. 
	
	Finally, having run out of other people to visit, she turned to the Regent’s encampment. She sat in the sedan until she was certain she could control her emotions. She was so. She didn’t know what she felt anymore. But she couldn’t tell him. She could show none of it. Her head throbbed. She wanted to stay until she had a plan of approach, but that was unlikely to come. So she let herself be announced.
	
	The Regent Kokhan stood by a desk covered in papers, seals, ink wells and scroll cases. She had interrupted his letters. After they had gone through the formalities, she found a comfortable place to sit. 
	
	“How is the Senator?” she asked with a casual curiosity that she did not think she possessed.
	
	“I won the vote” he replied with what sounded like genuine pride. He shuffled through his papers until he found a unsealed letter and handed it to her. 
	
	It was a demand for the arrest for treason. Of the person of Senator Talal Karim. For the crime of conspiring to kill the Kokhani Mata Reina Nababi. 
	
	She laughed. It was brilliant. It wasn’t true. But it could be proved in court. “This is good,” she told him, and thought about the Senator deliberately stoking violence. “And I can make it better.” She offered him the letter. When he took it from her wordlessly, she added “You are welcome.” 
	
	Rage flashed across Darius’s face for an instant before he clamped it down. He was in a dangerous state, she reminded herself, angry, or scared, or both, not having slept in who knows how long. 
	
	“Congratulations,” she added more gently.
	
	He sat down with a huff. “And this mess?” he waved his arms at that Janma basin at large.
	
	She shrugged and shook her head. Small, gentle, calming motions. “I don’t think it is too bad. You have the culprits, and they have more or less confessed or been snitched on. You punish them, and let the rest go with a warning. Keep the emphasis on the military's lack of initiative to put down the dangerous faction, and show the people your mercy.”
	
	He shook his head. “I can’t just let the Birs go. They are responsible.”
	
	“I think you can. The Hasti Bir is not a threat to you. I married a girl here to help us keep an eye on the Senator. Now that he isn’t a problem, she’d be just as useful making her husband get things done. Besides, he’s terrified of you.” Darius gave her a curious look, which she returned with a shrug. 
	
	“And the other?”
	
	“I can’t vouch for him, but I think he’s a good man.”
	
	“The violence came from his tribe, Rhaiyina. Stability in the Janma basis is crucial for the empire right now. I cannot leave in the hands of an unknown quantity.” 
	
	“It was inter-tribal bickering. It wasn’t directed at me.” Darius’s face darkened. “I admit that the situation is less clean for the Frit. But the Hasti are an asset, and I do not think the Frit are truly a threat.” She paused and considered him. Darius was clearly not convinced. “Let me finish what I started here. Before you leave, tell the Birs exactly what you would like to do with them and leave me a commander I can trust. I think they are both frightened after yesterday’s episode. I should not need much more than that. 
	
	I’ll write you daily with my progress, and I promise, if I cannot secure this stretch of river for the empire, I will tell you to begin preparations for annexation.”
	
	He nodded and took some notes. They both knew that she would never give permission for Sardona to annex a new territory. They also both knew that she would not break her word. She trusted him to support her in Riosk as long as she served his purpose. It would have to do. It was all she had. After a while, he stopped writing and started at the paper for a long time.
	
	“There’s something else I have to tell you,” he said, not looking up. She waited. “The Senate has asked that you not return to Riosk for as long as I am Regent. I will build you a summer palace in the Janma basin. But until then, you will return to my estates at Al Usbaliya.”
	
	The Kokhan Mata’s head swam. She was not going back to court. They would throw her out of her home. “They are exiling me.”
	
	He shook his head. “No.” He would not look at her.
	
	“And Estrad?” The words came out without her thinking them. She was terrified. It was dangerous to show it, but… Her family. She had to protect her sons.
	
	“The treaty holds.” He took a deep breath and blew it out. Frustrated. “I am not exiling you or putting you aside. I am just building you a summer palace. Nothing has changed” 
	
	With the relief came the torrent of …. Whatever it had been that she’d been trying to hold back for so long. Absolutely everything would change. How dare he pretend otherwise. “Why,” she yelled. “Why are you doing this to me.” 
	
	“It is what I had to trade in order to get the votes.” He sounded so matter of fact.
	
	“Look at me, Darius.” It was a command, and he obeyed. “I was the only thing you had to trade. Are the rumors are true, then? You are my pimp and I am your whore.” 
	
	“Rhayina.” He moved to touch her and she stepped away. 
	
	“The Senator is defeated now. Take it back. You don’t need the vote.” 
	
	He shook his head sadly. “I cannot.” 
	
	She scoffed. “The most powerful man in the empire, and he cannot take it back.” She turned away to hide the tears she could not stop. He would support her in Riosk only long for as she served his purpose. “Everything I have built.”
	
	“You will build again, stronger, better, safer.” She laughed at that. It had never been about safety. It had always been about freedom. Riosk gave her influence, and with influence she could buy freedom. What could she buy in the provinces? Darius was going on and on about an eastern seat of power in the Empire. The Janma basin was the future heart of the empire. She would be in a position to shape what that would look like. She wasn’t listening. She wanted … what. What could she possibly want from this man? “I know I’m asking a lot of you. But you’ve rebuilt before. You can do it again.” 
	
	And then she knew. She just wanted to hurt him. It was that simple. “Tell me something, Darius,” he stopped babbling. “The last time we were together, in the palace. Did you mean to kill me?”
	
	There is was again. She saw it before he turned away. That look of devastation from before. She had found gold. When he finally turned around, his face was calm. “No. I only wanted you to keep you out of sight until the vote was over.” He took several deep shuddering breaths. “Imogene, I …” 
	
	She struck him so hard that he stumbled backwards over a chair and landed on the floor. She walked over to where he sad, bewildered,  put the heel of her shoe on his fingers and pressed down. “Never from your lips, do you understand?” Darius’s eyes widened, in pain or surprise, she did not care, and nodded. She pressed harder for another second then removed her boot. “I will rebuild as you asked me to. But now I know. Do you understand.?” Darius picked up his hand and rubbed his fingers. Weakly, he nodded again, realizing what he had revealed. “Good. I will not hesitate to use it against you.”  
	
	
	
	
	
	
	
	
	
	
	
	
	
	
	
	
\end{document}