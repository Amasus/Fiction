\documentclass{article}
\usepackage{fullpage}

%*****************
% Annotations
\usepackage{soul}
\usepackage[colorinlistoftodos,textsize=footnotesize]{todonotes}
\newcommand{\hlfix}[2]{\texthl{#1}\todo{#2}}
\newcommand{\hlnew}[2]{\texthl{#1}\todo[color=green!40]{#2}}
\newcommand{\sanote}{\todo[color=green!30]}
\newcommand{\egnote}{\todo[color=violet!30]}
\newcommand{\newstart}{\note{The inserted text starts here}}
\newcommand{\newfinish}{\note{The inserted text finishes here}}
\setstcolor{red}
%***************************
\begin{document}

The heavy gilded collar sat uncomfortably across my back. My hand twitched to move it, then paused when Hinata caught my eye. I'd returned from court this evening to find the green and gold tunic laid out on a chair, where I should have sat to contemplate a half finished game with my duchess. Instead of the bench from which Desmond's tutor should have been reporting on my son's education, there was a large tub of steaming water. The absence annoyed me. I saw little enough of my son as was. I looked forward to the weekly reports.

"This isn't for today," I said, thrusting the expensive tunic meant for the birthday games at Hinata in order to assume my customary seat.

My man accept the garment, gently shook out any wrinkles I may have caused, held it up by the shoulders to fold the arms back into their creases before folding it over his arm. He gave no sign that he may have been mistaken. "Does Sama remember who is coming today?"

I continued inspecting the array of black and white pieces, blocking out my annoyance at Hinata and Desmond's tutor for their failures. It was nearly dusk. The city gates would close soon. "She will arrive in the morning," I said flatly. Sophia was conservative and dependable, especially when it came to our children. She would arrive late and in comfort.

"The Simeta sent word otherwise this afternoon. Her message said that there was no cause for concern. I have sent an escort of five to meet her." Taking advantage of my surprise, Hinata took me by the shoulder to the awaiting bath. This did not sit right. If there had been trouble with horse or carriage Sophia would have stayed at the workman's house. If she had sent sign of danger, Hinata would have notified me hours ago. There had been a time when Eugenio and I would travel long distances, paying no heed to rising or setting of the sun, trusting Rishiki to watch our backs, speed our only goal and thrill. But it was not Eugenio, returning to my hearth, as much as I missed him, it was Sophia. In all the years of our marriage, during all our the journeys to Deyalorn, Sophia, desiring her comforts, insisting on ending the day's journey long before nightfall to ensure that she could be comfortably set up before dusk. I knew she had endured harder travelling for years before she shared my name, forced first, by a tyrannical husband, then Eugenio's growing needs. What could compel her to travel alone at night again, worse, to risk our youngest's, Valira's, safety on a dark road?

I fought against the fear that prowled in the corners of my mind to understand what could possible be the cause of the late arrival. This past year and season had been hard on my family. None of us had quite recovered from Firi's abduction. Furthermore, for the last nine month, the two children in my care, Firi and Desmond, rarely slept at home, but in the White Tower and barracks respectively. Barron Paulis had been set on ridding Cartan of its Liri inhabitants, spreading rumors of a new dark form of wild magic through the town every time a cow died. Duke Griswold the Undying, somewhat placated by what seemed to be a shared interest with this unlikely ally, permitted him access to some of his privy meetings, which the good Baron then promptly reported to King Gustavo.

Nisrita clings to the belief that the Baron gives fewer and fewer matters of importance with each passing missive, that her half brother, Dario, the previous regent consort now in charge of the purge of the White Towers, has his hands full. My poor duchess, living her life daily in the center of the battle between White Tower and army, and bound to the monster of a Duke, cannot be expected to think clearly. Even assuming Dario did not know of Duke Griswold the Grotesque's possible plans for civil war, given importance and influence in the Towers, Dario could not afford to be distracted from his doings. If Baron Paulis were keeping information from him, he played a dangerous game. No, my best guess at Baron Paulis' long term plans was that he would betray both his duke and his dutchess to ride a swell of emotion against the Liri to power. All over Mersea, Liri communities lived in walled ghettos, the worst, in Cortan. The walls around their community were too high and too thick to scale. The two gates to the quarter would only allow one loaded wagon to pass at a time. The residents lived in relative comfort inside, but if the army decided to move against them, it would be easier than laying seige to a town full of fishermen. At the moment, with Marsea deciding not to expand her borders this spring, the army was all too restless.

Which is why I could not let the baron get his way. Which is why he had twice made an attempt on my life. I was saved once by Hinata's overwhelming attention to the detail of who handles the food in my household, and once by a mysterious unsigned missive written in a hand that bore a curious semblance to that of Duke Lucretious, still serving as the Baron's page. I have since advised the boy on disguising his handwriting. My children have not lived with me since the first attempt on my life. They do not know the truth behind why I sent them away. Nor are they among the mere handful of people who know of the second. But today, tonight, they would all be under my roof in preparation for young Duke Griswold's fifteenth birthday games. And my wife behaved as she should not.

No, I did not like this strange behaviour on Sophia's part. I had no clue as to the reason for her actions. I liked that less.

My thoughts were disrupted when my arms found themselves pulled unnaturally behind my body. Hinata had taken off the unnecessarily tight outer vest I had worn to the duchess's court. It is, I am assured by both my wife and my sister-in-law, a new fashion set by our young new king in Deyalorn. "It is good enough for king and duchess, Hinata," I snapped. "It will do for my wife."

Hinata held the vest up briefly, inspecting all sides, found an imperceptable gap in a seam, then put it on the bed for repairs. He did not look at me, he had not all evening, I realized. But worry on his face was not about a missing stitch. "You promised Simeta that you would make an effort," he said at last when he returned to unlace my shirt.

That I had, though the promise had not been made to my wife. The safety of our family had been the topic of every conversation and the source of every argument between Sophia and myself since Firi's brief disappearance. She raged against my ineptitude without knowing a fraction of what I protected them from. Emptying my halls of Firi's quiet grace and Desmond's games had broken my heart, though it was necessary. To Sophia, it was a meager and incomplete attempt on my part to meet her demands for their safety.

For Firi's fragile piece of mind, I had promised to make her mother happy. I would have gladly given everything I owned to hear her easy laughter again. Instead, I could only placate her mother, keep her from wishing to take Firi away from Cortan. My eldest daughter seemed to live in a world of shadows and unnamed fears now. The only I have identified is being separated from her home, and, to a certain extent, from Lyca.

I took a deep breath, held it until Hinata had finished his work, then exhaled as I sank into the bath. There was no point in berating Hinata. We both knew the dangers involved with having my family assemble under the same roof. He was doing his best.

*****

As the gate to the courtyard of my house opened, my hand twitched again. This time, in stead of bringing it towards my neck, I rested it lightly upon my scabbard. Firi fidgeted beside me. I put my other hand lightly upon her shoulder. Following my lead, the dozen men behind me discretely loosened their weapons in their scabbards. I prayed discretely enough for Firi not to sense the danger. The three days that Firi would not speak of had robbed her of her carefree and open nature. She no longer hosted her friends in her room in the Tower, and rarely did she grace the gathering of her classmates. But Nisrita, assured me that my daughter improved daily. Her ability to focus her energies had returned, as did a flicker of her old interest in healing. Still, the slightest sign of conflict sent her to her bed, complaining of an aching head or an upset stomach.

I stole a glance at Firi's face. Rishiki had answered my prayers. She was just a young girl, still a little worn and thin, worried about a visit from her mother and the changes to her life it may bring. She smiled weakly at me when I squeezed her shoulder. She was afraid of her mother, as many daughters are, nothing more. I found myself breathing with more ease at the realization.

The carriage parked before us in courtyard. The dark velvet drapes were drawn. I could hear the faint murmur of voices inside, one possibly male, another distinctly a child's, but could neither make out what was said. Nor could I see within to ascertain the owners. With Firi and Desmond vulnerable beside me, and Valira within, what horror had I let into my house? It had been many long years since I had wished to see Sophia's face, intact and superior, smiling down on me, as much as I did at that moment.

A door opened, and a pale red slippered foot emerged, to be covered quickly by a gown of white silk and red lace. Sophia's many ringed fingers emerged beyond the drapes, followed by her face. A knot in my chest unwound as I quickly stepped forward to help her down and lead her away from the carriage. She showed no outward sign of disturbance, but that meant nothing. "Did something happen?" I whispered urgently when we had moved a few paces away from the carriage.

"Of course not," Sophia replied, meeting my concern with a suave smile. My head spun with disbelief and confusion. Thoughts of possession and witchcraft chased each other around my head. Anything to explain why Sophia would not take a chance to tell me of danger to our children. My wife, in the meanwhile, glided over to Firi and Desmond's. "Valira!" She called out to our youngest after she had received her son's greeting.

A dark straight haired child with a round face, sloe eyes, and a bulbous nose tumbled out of the carriage with the help of her older brother. Her beauty would have been the pride of any family in Lir, if either of her parents had been Liri. The two siblings immediately clasped hands and started spinning each other in circles. I grinned. Even Firi, far too old to join in such games, smiled as she watched.

Sophia cleared her throat and the three children fell in line. Valira greeted me as a dutiful daughter should. I would have protested that for such a rare and complete family reunion one day of indecorum might be allowed, but the question of Valira's safety took precedence over all else. "Then why did you arrive so late?"

"Firi," my wife called, ignoring my question, "I've brought a surprise for you in the carriage. Go see what it is."

I tensed, as, against all other logical explanation, the thought that my wife, who I trusted more than anyone for the safe keeping of my children, could be plotting against me entered my head. I motioned for Firi to remain where she stood and asked Hinata to fetch the present.

But my wife, who managed the southern Romino estates single handed, and kept half the men of Escasaine's Tower happy in her bed, would not have it so. "Firi," she ordered, daring me to contravene, "go to the carriage, pet."

Firi hesitated, then obeyed. In the interest of peace, I let her go, but I could take no more. "What is in the carriage, Sophia?" I hissed when my fragile daughter was out of ear shot. "Why this erratic behaviour? And you know how sensitive Firi still is. Why are you ..."

Firi screamed. I reached for my sword only to find Sophia's hand resting firmly on the hilt. Shocked, I stared at her wrist for what seemed an eternity, before looking up again at the carriage.

Hinata, freed of Sophia's scrutiny by my inquisition, stood, dagger in hand, ready to introduce new holes in Eugenio's handsome physique. After a moment's hesitation, he threw down his weapon and embraced the young Sama with happy cries of welcome home. Firi hung around his neck like a babe in arms. The circle of guards behind him crowded around as if to welcome home a hero. Eugenio had left my halls thirteen months ago to join the northern fleet. As far as I was aware, he had not yet accomplished anything of note.

It took me a moment to gather my senses. When I did, I turned the full force of my fury on Sophia. "How dare you, woman. Do you have any idea..?"

"It isn't mother's fault," came Eugenio's laughing voice. "I forced her to do this." He extracted himself from the adoring throng, still carrying Firi, more appropriately now, as if helping her ford a shallow stream. "To the credit of mother's good sense, she protested long and vociferously. But," he added with a flourish as he put his sister down, "if there is one thing I have learned in the north, it is that women can't resist a sailor." The rogue, I thought, to speak like this in front of his sister, but Firi blushed, and entwined herself around Eugenio's arm. The younger children, following the indecorous lead of their older siblings, ran off to frolick with the men. Sophia coughed disapprovingly. "Don't take it so poorly, mother. Throw it all to the wind for a day, an hour even. How long has it been since we've all been together like this? And look, even father's grinning."

Was I grinning? I tore my attention from Firi's happy countenance to realize that I was. I stepped over Valira's discarded doll to embrace Eugenio. He deserved a hero's welcome. I had not realized how hollow my house had been without him.

***********

It took half an hour of bribing, begging and cajoling by Firi and Sophia to corralle Desmond and Valira inside. It took a similar amount of time for Hinata to convince the cook to open the cask of Firvonese brandy reserved for the Ducal birthday celebration, and change the menu accordingly. Hinata returned to a courtyard recently emptied of celebrating members of my household. Like the last vestiges of syrup in a bowl of compote, my childrens' laughter still clung to the cobblestones, carried from the garden by a warm night breeze. Firi, I could almost make out, was teasing Eugenio about being in love.

Suddenly, my legs would support me no longer. I sat down heavily on the rim of the fountain. I sensed, rather than saw, Hinata perch beside me. I groped for his hand, wanting to draw strength from a man who knew no more than I what to do if Baron Paulis used the birthday celebration to stage an accident against us all.

"The young Sama is right, I think." Hinata said at last. I looked up to realize that there were tears in my eyes. I could hear Firi's voice helping her mother manage the kitchen boys as they laid out the dinner. Her words were lost to the scraping of boards and the thudding of footsteps, but her tone conveyed patience and carried authority. Was my fragile daughter taking on her role as the lady of the house? "Let Rishiki take your worries for the night." Hinata continued when I did not say anything. "They are all under your roof for the first time in years. And young Simena is laughing. That alone should be a sign that the gods will provide for us."

After some time, Hinata left me to my thoughts and resumed his role overseeing my household. He must have said something amusing at my expense, for shortly afterwards, I heard the birdsong of Sophia's laughter, followed by the gentle rolling thunder of Eugenio's mirth, and finally, the short harmonious chimes of Firi's joy.

****************

In the flickering light of the candelabra in my office, now containing a bed to accomodate Eugenio's unexpected arrival, Firi looked pale and tired. Her older half brother, on the other hand, looked flushed and sated, sitting in my customary chair with its feet carved like lion's paws. If this was to be his room for the next few weeks, he argued, so should the master's chair. I indulged his roguish whim. A chair was a small price to pay if I could lure my seafaring son back to this landlocked dutchy I called home. Events had been easier to control with Eugenio lived with me. He was a clever man, both a natural leader and good with a sword. He understood the Liri heart and many in the Liri district trusted him. Some, I knew, even loved him. If I could leave him to manage the anger with in the walls of the Liri quarter, I could turn my full attention to the doings at court. With his help, I could protect my house and my duchess and my home.

But that was a conversation for a different time. The room had too pleasant a domestic feel to it to contemplate politics. The Firi sat at Eugenio's feet with the embroidery. While pale and drawn, she lacked the vacant expression that usually sat upon her otherwise pretty face. The evening's festivities had exhausted her. The spark of life that Eugenio had brought back to her heart had not yet died. Eugenio chatted easily about the landscape of the Selvic coast, while Firi offered her polite opinions on travelling to the same. From time to time, she would shudder, or involuntarily look over at the hearth, kept empty due to the warmth of the season.

I handed Eugenio a cup of wine and Firi a stole, then took my place by the door to the balcony to take advantage of the little breeze the night offered. I had lived alone and in fear for the last nine months. I had neither the strength to send my children away to the safety of my lands in Escasaine, nor the certainty I needed to have them near me. Now, with my house full to bursting, I did not know what to feel or say. It was better to listen to the younster's conversation against the backdrop of crickets and bullfrogs.

"Did you take Yra to this cove?" Firi asked, causing Eugenio to cough in his cup.

He frowned at her with mock severity. "For pity's sake. You keep secrets like a burbling stream!"

Firi shrugged casually, not taking her eyes off her needlework. "There's no point. You know father. He's more likely to approve if you tell him now, before he figures it out himself."

A look of true annoyance passed over Eugenio's face, well above Firi's line of sight. He reached down and playfully pulled her ear. "Now that you have forced the issue, you naughty colt, may I talk to father alone tonight?"

Firi got up cheerfully, bade us goodnight, then left the room with a wink for Eugenio and a mischievous "Good luck."

I stood in the doorway, amused, waiting for Eugenio to speak. "You can lower your eyebrows father," he snapped, irritated for the first time since coming home, "It is not what you think."

"No?" I asked, composing myself. "Then explain this to me." I motioned for him to follow me to the balcony in my bedroom, which had the advantage of being shielded from the open windows of the other bedrooms in the house. While it was late into the night, not everyone in the house was asleep.

We stood shoulder to shoulder for a long time, over looking the city and the Liri quarter, while Eugenio took his time figuring out where to start. It was an easy silence between us, so familiar from when we lived and worked together to protect this city, was it only a year ago? My heart ached with the memory. It would be good to have Eugenio by my side again. "Griswold's invited me to attend these celebrations." I assumed he meant the young duke. The two had grown up together in Lir, and formed a strong friendship that showed no sign of eroding in Eugenio's absence. Eugenio hesitated, then said, "I did not want to come."

This surprised me. "Eugenio, when your duke calls, you do not question your own desires."

"Father," Eugenio corrected me, "When I joined the Selvic fleet, I swore fealty to Duke Edwardo. The duchess released me from her service. Griswold knows this."

I waved away the technicality. "It shouldn't matter. For a man in your position, and a man in Griswold's. There should not be a question."

Eugenio's face twisted in annoyance. He stared out towards the high walls surrounding the Liri quarter while he composed his thoughts. When he spoke, he did not mince his words. "No, father, you are wrong. Given my and Griswold's positions, I must question. He wrote me because he wanted me to negotiate with his mother. He wants her to abdicate." I sucked in my breath. This news came as an uncomfortable surprise. "The duke has been gathering his forces since the Convocation of Heads. With the military not marching the season, Griswold fears bloodshed in Cortan. He asked me to help him stop it."

"I see." I said slowly. "You intend to refuse him gently, or you would not be here telling me this. This isn't why you came home though."

Eugenio looked abashed for a moment. "No." Then he rallied his courage and said, "I came home you ask you to leave Cortan."

Well, that was even more unthinkable than little Griswold plotting against his mother. "That would hardly stop violence in Cortan." I replied wryly.

Eugenio looked pained, but patient. "No, that it would not. But father, forgive me for saying this, but you are blinded by you love for the Duchess. She cannot outwit her husband, even with your help. If she wins for today or tomorrow, he will outlive her and extract his revenge on the families of her allies."

I stared at Eugenio, long and hard, sizing him up. I had not taken him for a coward. Firi's abduction had changed us all. "It is not as dire as you think. The convocation of heads has hurt the Towers' strength. Towers will not supply the army with the number of healers it wants. The army will not march to battle until the Tower agrees to its terms. Meanwhile, frustrated dukes and generals, thirsty for glory starve the Towers within their cities and barrack of resources. There are still more ungifted lords and generals in Marsea. They will not support any rebellion Duke Griswold has in mind." Of this much I was certain. There was also the quest to control the wild magic, I knew, but progress on this, as far as I, or any of my sources could ascertain, had stalled during the last year.

Eugenio threw up his hands in frustration. "It is not just Marsea at stake. I beg you, father. Look beyond this duchy and country." He pointed at the white walls surrounding the Liri quarter. "Do you see how we make them live? The Duke has no desire to suffer foreigners on Marsean soil. Neither do most Marseans. Even in Selvand, where they benefit from the Liri ability to control the seas, the citizens barely tolerate them. Szarvis is furious at this alliance. The word is that Duke Griswold has sent emmissaries to Szarvis. My guess is that he wants to break the northern truce, to threaten our army with Szarvis's. Given the tensions with the Tower, Marsea's army is weak. If the Duke promises to rid us of our Liri guests, I cannot imagine that Szarvis refusing him. But you should have figured this out yourself. How is it that you have not?"

I took a deep breath and ignored his accusation of negligence on my part. He told me nothing new. While Eugenio was as clever and observant as any one I worked with, he was not my spy in Selvand. I valued his life too much. "Master Alerio has it in hand, I began..."

"Master Alerio is an old man who will not see out the decade." Eugenio was yelling at me now. "What will you do then?"

"Enough." I said, to stop the conversation. "You speak out of love for your family, so I do not call you a coward. I have trained you well. You read the political omens correctly. But there are things I know that you do not. No one will leave Cortan."

Eugenio looked suitably chastened. "There is nothing I can say?"

I shook my head. "There never was." We stood for a long moment in a complete and awkward silence. The night had grown old. Even the insects and night crawlers had ceased their symphony. "This isn't what you came to talk about. Tell me about Yra."

Eugenio looked down at his feet. "Her father owns a small merchant fleet in ?????. Her brother manages it now that he advises the Selvic fleet. I had a passing acquaintance with the family in Lir. I was part of an escort for them once."

"Certainly you can to better than this," I interjected. Eugenio had always enjoyed a place in my house equal to his half siblings. It was not part of my original agreement with Sophia, but I had grown fond of the boy over the years. While I never thought of it in such stark terms, he was as much mine as the rest of my brood. "You know you may depend on my influence with a family beyond your reach as a Lazaro."

Eugenio looked only more miserable and shook his head. "I cannot take your help on this. I cannot have my family or any children the Maker may grant me associated to House Romino." I gasped. Had someone thrust a knife between my ribs? "Don't look so pained father. I do not like saying this, and I will have to say this to Griswold after the celebrations. You raised us as brothers, and I think of him as such." He paused to wet his lips. "There is a civil war on the horizon, and no one will hesitate to lop off the head of a sailor named Lazaro. I neither weild the gift, nor hold any land. I must marry by my own..."

"After all I have done for you," I snarled. "You ungrateful mutt. After all you have been given by the Duchess."

"The duchess and her line will be the end of everything you hold dear," Eugenio continued, unphased. "What will happen to us when Alerio dies? You will not take your family to safety. Someone must watch over them. If the war goes poorly, my future father-in-law will smuggle us back to Lir. He will take Mother, Firi and Valira as well. Desmond, for a price."

"They are gifted," I growled. A cold fury consumed me. This was the return on my generosity. "What you say is treason. I will not have it in my house."

Eugenio set his jaw and avoided my eyes, but gave no sign of changing his mind. "I pray it will not come to that," he said.

I answered him with a stony silence. I noticed that his generous future father in law only offered to take my women. What use was Desmond to him if he wanted to bring the gift to Lir. Eugenio was an idiot to trust such a man.

"I will leave your house, sir, if you wish it." I found I could not answer him. I was watching my life long dream of bringing the Liri way of life to Marsea crumble before my very eyes. It filled me with more grief and anger than any action of our current Duke or the disgusting Paulis had ever done. Everything I had worked for, all the suffering I had caused along the way, it could not have been for nothing. By rights, I should have had Eugenio and his self serving merchant friend arrested immediately. If I could only speak.

Eugenio moved suddenly towards the door, and I found my toungue. "Go to sleep, you son of a jackass." I croaked. "It will not come to that."

********************

He had called me beautiful. I looked at my reflection in the tall oval of polished bronze before me, then turned quickly on my heel. The blue silk skirt swished girlishly about my ankles. I glanced back over my bare shoulders. Two large cloth flowers sat on my shoulder blades, tastefully denying the possibility that the right sported an impossibly unsightly scar. How long had it been since I'd worn something so frivolous and fanciful? Ten years? Twelve? I was barely more than a girl then. The slightest movement made the finer hairs on my bare shoulders stand up. It was a queer feeling, having them exposed. I looked around for an appropriate stole, and found that none would sit well over the flowers. I shrugged and gave up, catching the ripple of my shoulders in the mirror dominating my wall. I snorted at what I saw, a thirty year old matron, dressed in a debutante's gown. Many women older than me dressed like this, of course, Sophia especially came to mind, but they all had tall graceful figures, slender shoulders and sculpted faces. No one but Timmon would ever think to call me beautiful, crows feet leaving gentle imprints by my eyes, white hairs appearing at my temples.

Yet he had called me beautiful. Timmon would have the good fortune of having Sophia's equisitely manicured hand on his arm today, as they take their seats at the games. Every eye in the arena would turn to see her, and then the man fortunate enough to possess such a treasure. I imagined, that every night he enjoyed the splendors of Sora's muscular mahogany body. How could I possibly compare with one who once ornatmented the Liri king's court? I shivered. It was best to not let him get to my head. This was yet another of Timmon's incessant and fruitless attentions towards me. He knew of the impossibility. He was one of four people in the kingdom to know of my feelings for my Head. But he would not stop. He would give me flattering gowns for state occasions, and I could do nothing but accept his praise.

**********************

I inspected my hairline at the temples. In honor of the day, I died my hair and took more care to paint over the wrinkles on my face. The effect was, I decided, unnaturally youthful. I turned my waist, felt the silk swirl about my legs, and giggled.

I turned back to find Lyca at my door, a quiet, curious laughter flirting with the idea of expressing itself on his inscrutable face. There was something wrong with the Duke of Escasaine, I thought, not for the first time. He did not react as others did. He seemed not to live among us, but in his own world lost between the dominions of the Maker and Preserver. When he was younger, I told myself that he was simply shy, then that he was studious. Now, as the time of his coronation inched closer daily, I worried. "Yes Lyca?"

"Is your grace ready?" my boy asked, speaking carefully to ensure that his changing voice remained in the lower registers it stubbornly refused to settle into. He offered me his arm.

The gesture touched me, and I refrained from brushing a stray lock of hair off his shoulder. "Where are your siblings?" I asked, slipping my hand through his elbow. He was still shorter than me. It disappointed me. I had hoped he would take after his father and the Baroness.

"Griswold is inspecting the mengerie for the games," Lyca reported flatly. " He wouldn't let Simone and Emile join him, so they are playing with Desmond in Griswold's box. Elena and Mirella are still dressing. I think Cesara and Valira are with them. There were a half dozen girls preening together at General Madriano's house earlier today." He paused uncertainly before continuing. "Father has the rest."

I took in a sharp breath at the unspoken. I had spent the weeks leading to this day trying to deny the fact that I was incapable of managing my own household, let alone calm a duchy. After the disasterous convocation of heads, I had decided that a large celebration of my first born's birthday would be good for Cortan. It would give gifted and warrior something to celebrate; take their minds off the fact that they were jostling for position in Tower and barrack, instead of on the field of battle, where they belonged. But no well intentioned plan could survive the in the poisonous miasma that permeated my household. Griswold, as Baron Paulis liked to remind me, was not the only child born fifteen years ago today.

My first born had not forgiven me for denouncing the changeling he had shared a womb with. He distanced himself from myself, his father, and all of Sophie's children, keeping his own counsel, only occasionally allowing Timmon to come within arms reach of his dark moods. He did, however, do his best to keep what remained of his siblings together. My husband, always the first to champion any cause that opposed me, had taken the boy under his wing. The younger children chose sides, for or against me or their father, with or without their unnameable brother, their moods and allegiances shifting with the wind. Griswold kept the piece, and mostly kept them from overtly taking sides with either parent. He even won over Emil, ever the changeling's favourite, though I suspected this involved allowing him visits with the beast from time to time. Only Lyca, still fiercely attached to Firi, remained outside the shelter of Griswold's palm leaf. He hated his father and his ward with the fiery passion of a lover wronged.

At any rate, Griswold, the man of hour, asked me for a box of his own for the games. He would host his siblings there. It was his birthday, so I could not deny him the request. My first born did not speak to me any more if he could help it, and never appeared in public with me. I tried to comfort myself that at least he did not ally himself with his father. It was a feeble salve under the best of circumstances. Today, in a grand show of unity to celebrate the life of the next Duke of Cortan, the ducal family would attend it three separate boxes. It did not matter whose side Griswold took. This simply hurt.

And what of my dear husband? How did he plan to humiliate me today? Alerio and Timmon had thwarted every plan that Barron Paulis had of keeping me from the spotlight. So he settled for making certain that the public adulation that would radiate upon my husband would outshine that which I would enjoy. Yet we were certain that my husband had something special planned for the day, though for all our efforts, we could not discover what. Shortly after the convocation, my husband seemed to loose interest in the wild magic, a fact that Timmon initially found very suspicious, until Alerio and Carlotta convinced him otherwise. The investigations, as far as anyone could ascertain, in the rest of the Towers proceeded as planned, and Carlotta insisted that he had become completely absorbed by his new ward. I was not permitted to scoff at this idea. After all, she reminded me, he had always taken a keen interest in his sons' well being. Why else would Lucretious be alive. There were several reasons, I wanted to remind her, but left them unsaid. The conspiracy between Alerio, Timmon, Carlotta and myself had enough tensions running through its veins without my starting more. So Timmon let the matter drop, and turned his attention to the more pressing issues of keeping my city at peace.

Lucretious tugged gently on my arm, urging me to pick up my pace. "Head Alerio and General Galderan await you in your box," he reported, with as much emotion as he would use recalling the bones found in a cow's hoof.

I smiled and picked up the pace. "Where will you be during today's festivities?"

Lyca shrugged. "You are more likely to be discussing the Tower's business in your box than watching. I think I will join Baron Romino."

************

General Vittorino Galderan bowed deeply as he kissed my hand. "You look as radiant as when you first became a mother, your grace." Even among friends, this was simply the flattery that went with the office and the day.

"Thank you, Vittorino," I beamed. "I am glad you could join me today."

Alerio took my hand silently and formally, wishing me the blessings of the occasion. When he stood I saw him quickly pass his eyes over my clavicle and half exposed back, and watched the corners of his mouth twitch upward in approval. He was not the type of man to call me beautiful. And even among friends, I counted on him to show no greater sign of our intimacy than this.

For his part, his attire looked as queer to me as mine must have to him. Alerio had devoted his life to Mersea's Towers. He rarely left the confines of the Tower and Barrack on anything other than Tower business, and when he did, it was invariably in the company of other gifted. I could not think of anyone in the city who had ever seen him wearing anything other than white.

Today he offered me a yellow sleeved elbow to lead me to my chair, and held out a green cuffed hand to ease me into my seat. He wore the colors of Firvona, the county of his birth. His clothes were well crafted and gorgeous enough to befit the occasion, but in a style that had gone out of fashion by the time I was born. Today was a day for peace between tower and barrack. I was glad that he chose to appear in my box dressed as a common nobleman, rather than the controversial head of a powerful tower.

There was more to be glad of, I decided as Alerio unclasped an amusingly long summer cloak, and folded it over the back of his chair. Removing the cloak displayed a bright pair of lower garments that padded and flattered the shape and contours of legs. A gentle wind caught his silk shirt, sending it rippling against the birdlike shoulder blades of his thin frame. His new attire would certainly provide an exteremly pleasant, if dangerous, distraction. I turned my head abruptly to pay my full attention to my other guest.

"The day progresses well, your grace," Vittorino commented, passing me a platter of figs and soft cheese.

I looked around the stadium. I scrutinized the crowd any hint of white robe or the blue and white uniform of my soldiers. I saw none but the guards and healers on duty. I exhaled and forced myself to stop fidgeting with the fig in my hands by popping it into my mouth. Then I wondered what to do with the sticky residue on my fingers. Everywhere around me, I saw my subjects eating, laughing, listening to the minstrels wandering through the crowds or sitting with their families. My feeling of foreboding was completely irrational. I forced myself to smile at my general, "It does. But I have yet to give my speech."

Alerio tisked sharply at me without turning his head in my direction. I suddenly longed for five minutes alone with him today. But we had both agreed. Today was a day for reconciliation. I could not be seen to favor the Tower over the army. And on no account could I risk revealing the true nature of my relationship with my head. We would remain chaperoned. "We have practised this moment a hundred times this last month," he reminded me sternly. "You know what to do. Leave the rest to the Preserver."

Blasphemer, I wanted to tease him, coughing to hide my laugh and struck by his ability to speak those words without a hint of hesitation. He meant, of course, that I should leave the rest to him. He had a plan, I knew, for he had reassured me of this at least as many times as he had heard me rehearse for this day. If my husband's plans for the day included trouble, he would handle it.

The trumpets blared, and the black riders pranced into the ring, blue banners fluttering, and black steed turning and bowing in tight intricate formations. They faced the side of the stadium where I sat, and marched forward in a dark ceremonial vee. The Commander Sattore, in charge of the black riders since Carmine Dielo's promotion to a general at Deyalorn, a man who had remained stubbornly neutral in the battle between myself and my husband, reared his stallion up to a long and fearsome pawing of the air. Then, as if one living being, the riders forced their horses to bow, heads low, forelegs extended, mimicking the prone position of respect favored by the Szarvic court. It was a brilliant show of control and horsemanship. The steeds rose, and the riders saluted the box containing my children, then myself, then the box occupied by my husband.

Young Griswold stood and tossed a wreath of laurels towards the commander, signalling the crowd to burst into cheers. I rose and walked to the front of my box. My beloved first born bowed stiffly in my direction, then retired out of my sight into the audible throng of his friends and siblings. I waited for the adulation for my son to abate and surveyed the crowd again. The late morning sun felt warm and unfamiliarly pleasant on my bare back. I told myself that the sensation was an omen of the day to come. At the far end of the stadium, though I tried not to notice, was the box containing my husband. With him amidst the visiting healers from neighboring towers sat the plump figure of the boy Griswold had fostered and the pale shadow of a Nievian wife the child had acquired. I saw less of the Nievian princess than I did of her husband, though rumors of her captivating beauty and ability to turn public opinion in favor of Niev. She would be poison for my Lyca when he inherited Escasaine. Baron Paulis had no right to put her in such a place of prominence. If I could rid myself of that despicable man without endangering my family more, I thought, then realized that I was gnashing my teeth. I forced a smile upon my face again, and began to speak.

It was a short speech. A simple welcome, a few lines in praise of my son and a flattering story from his youth. Gratitude to my subjects for the friendship on display before me this morning, and confidence that Cortan would overcome any tensions for love of duchy and kingdom. I lacked Alerio's natural skill at oration. It was better to speak simply and non-controversially.

When I finished, I blessed my subjects and extended my arm in the direction of my husband's box. We had, against my better judgement, agreed on this. I would only make a fool of myself if I signaled for the games to start, Alerio had said. Griswold would speak. I could not stop him. Like ants before winter rains, it was better to lose ground than drown. I scurried like an insect into the safety of my chair.

"When you first returned home," Alerio gently reminded me when he saw my scowl, "you would not have been allowed to speak at all, let alone first. Do not be so impatient, your grace."

I shrugged in acknowledgement, but my mind was elsewhere. I could hear Griswold's tenor sweeping through the crowds, praising Marsea's healers. I tensed. Vittorino shifted in his seat, but Alerio remained impassive.

"I have liveried men waiting by the stables, your grace." Vittorino said, when it seemed my husband's praise for the gifted would not end. "Commander Sattore will not take sides, but the black riders will keep the peace. Barron Romino assures me that the commander can be counted upon."

Alerio held my eyes for a long solemn glance. Suddenly, he smiled, and a tension I had not even noticed was in him dissipated. "My agents are in position as well. Do not worry, your grace. They will not be needed."

Only then did I realize that my husband had ceased praising the Towers and had moved on to praise of the armies.

"Your agents, Alerio?" Vittorino asked. "That sounds like you have something more subtle in mind than asking the gifted fighters to stand guard."

The head smiled and brushed a speck of dust from his sleeve. "I do. But the details of how my Tower choses to protect this dutchy are not open to the military's scrutiny." There was a friendly note of sarcasm in the old man's voice. This fear and mistrust was, after all, the entire reason Cortan needed today's celebrations. All the same, Alerio how his Tower chose to protect my dutchy was also not open to my scrutiny.

The conversation around me turned towards alliances formed and being forged, fulfilling Lyca's prophecy that I would be discussing politics today. My companions said little that I did not know, they spoke to keep my mind off my husband's words. It helped, but while I could not listen to the oration outside, I could neither focus on the converstaion around me. I picked up on pieces of both, and let my mind tie itself in knots for the rest.

Alerio had a spy in Baron Paulis' inner circle, who reported that the reported little and less to my half brother Dario. My husband began speaking of the importance of purity of Mersean blood, a thinly disguised attack against our Liri citizens. Vittorino shifted uneasily in his seat and bemoaned the lack of support from the crown. It could not be helped, explained Alerio, with pursed lips. There were enough troubles between the towers and the army to keep Master Dario's hands full. If the Baron reported that he had this troublesome Duke to the south well in hand, that freed his attention to sooth the whinings of the powerful, gifted and not, about the extra taxes being raised to support the Towers and the demands that all gifted women march in the spring. I listened, only half believing this state of affairs. Timmon still put his full faith in Dario. And I trusted him... well, more than I trusted either man sharing my box today. But Alerio had planted doubt in my mind. These last year and more, my good friend had been overwhelmed by his crisis at home, for which I held myself more than a little responsible. That much was clear. And, as Alerio pointed out, Baron Paulis successfully kept his attentions focused on the Liri community, urging my husband to impose harsher resctrictions on their movement and trade, and inciting vandalisism and riots agianst them. Though the snake covered his tracks well. Timmon suspected, but had no proof. If not for Alerio's spy, we would have nothing against him either. It was possible, I supposed, given the head's intense secrecy about the source of his information, even from me, that he was simply jealous of my friendship with Timmon. Men have done worse than merely discredit their rival in their lover's eyes.... The mention of Lyca's name brought me out of my reveries.

What was really needed, Alerio was claiming, was a union between Duke Lucretious and a powerful gifted family from Deyalorn. Vittarino pulled out a list of eligible names he conveniently had just received from Carmine.

"No," I said sharply. The spirit of cooperation between tower and barracks can be taken too far. I would not be beseiged like this. "For the hundredth time, this is out of the question."

Alerio's soothing baritone urged me to look at the new list. They had taken all of my previous objections into account, and he was confident that this time Carmine had found a handful of acceptable, and eminently eligible young women. Lyca was barely thirteen, the same age as I when I married. I would not thrust him at this vulnerable age into the same folly. My husband still droned on about the importance of our continued alliance with Szarvis. It was time to put an end to this nonsense. I rose, and went to the front of the box, signalling my son to do the same in his. After a moment, the crowd grew restless and confused, forcing my husband to draw his oration to a close. The birthday games began.

***************

The festivities proceeded delightfully. They opened with a grand show of horse archery from the best of our black riders. Five wooden boars, two quintains, and a full 2 score of wooden pigeons were mercilessly slaughtered by our brave warriors. A handsome young man named Mario sent an arrow through a dozen hoops while charging across the arena, beating both his social superiors and his more senior colleagues. "Only seventeen," Vittorino informed me, "a promising lad of no birth. Signed up to be an ordinary soldier, but, as you can see..." When the young victor presented himself before my son, Elena, eleven and budding into a beauty in her own right, shouldered her brother out of the way to present him with a garland of roses. The young rider's dark skin turned ebony in his embarrassment. Griswold laughed good naturedly and contented himself with presenting the steed with a wreath of white chrysanthemums.  By the time he turned to me to receive my congratulations, he was too taken aback to understand my words. How could a bag of silver compete in a young commoner's heart against the favour of a duchess? I laughed as I settled myself next to Alerio, leaning close across his chest as I reached to help myself to a handful of almonds. The sun was hot, but a few scattered crowds provided intermittent shade. A breeze from the east kept the day from reaching the usual sweltering heat of the season. The worst part of the day had passed without event. My children enjoyed themselves, banded together with no sign of the politics that had stained my court for the last year. Nothing, I decided, would destroy my day.

The foot combat followed, two lightly armoured teams of six faced each other under the noon sun. Twelve swords glinted in the sunlight as they saluted my son before the fight. My heart swelled with pride. The games were interrupted several times by well wishers stopping into my box to offer words of congradulations or present me their intended gifts for the young duke to receive my approval. Sephyat Nagiyi lingered a while to share a cup of wine with me, and expressed his surprise over what he saw. Marsean crowds longed for combats without the aid of healers in the sidelines. These, of course, were illegal, and those organizing death matches, as they were called, were heavily fined. So the Liri priest, biased by local opinions had come to the games expecting a tamer show than what he was accustomed to in CAPITAL OF LIR. Alerio's eyes danced with pride for his art and his country, as I explained the obvious, that the presence of healers allowed our men to fight more fiercely, aiming for glory without fearing for their lives. Eventually, the side favoured my my children won over that favoured by the youngsters in Barron Ramino's box. I caught Timmon's eye as I bestowed the reward. He flashed me joyous grin as he and Firi cajoled Desmond and Valira to leave off taunting Emile over the space between boxes. Lyca sat politely chatting with Sophia, who looked disgusted that her younger children could put on such a public display. Firi looked happier than I had seen her in many months. I returned Timmon's smile as I returned my seat, wondering idly why Eugenio did not share their box.

Wrestlers followed fire eaters followed horse acrobats followed sword jugglers. The human feats of prowess gave way for the bestiary. A dwarf lead a dancing bear into the center as the animal handlers arranged the cages.  The grizzly bear, imported from Szarvis, to everyone's dismay, won against our local lion, but then neatly dispatched the pack of wild dogs brought from Niev. Griswold stood to give the creature a ducal pardon, and free reign in the prairies of Cortan.

As the various barons and generals brought forth their champions for the tournament of horsed combat, Vittorino made his excuses and stepped out to see to the man representing his house. Alerio watched him go, then drained his cup in silence. "I too should leave," he said casually. I protested feebly. I knew what would happen if I was seen to favour the tower on this of all days, or worse, if a detractor were to guess my relationship with the leader of my tower. But the afternoon was so pleasant, and my lover, suddenly dressed as a nobleman was such a lovely sight, my heart overran my head. But Alerio had more self control. "If I must stay," he asked dryly, "would you like my report on how the Duke's investigations proceed?"

"By no means!" I said, scoffing at his transparent attempt to cease my whining and ask him to depart. My husband had, for all appearances, stopped his investigations into the wild magic since the convocation of heads. "Leave, Alerio, if you must talk politics. I intend to enjoy myself today." The old head smiled graciously, put his goblet down sharply on the table between us, wished me a good afternoon, then exited the box.

I leaned over the edge of my box to examine the competitors. The elder barons placed son in the games, the younger competed themselves. In a few cases, where there were no male heirs or the head of household was a healer, a champion was chosen to represent the cause. Griswold, in a show of diplomacy beyond his 15 years, also chose a champion. No baron would dare wound him. It could not be a fair fight. I scanned the group of competitors, expecting to find Eugenio standing in for my son. Instead, I found the young champion of the archery display standing in blue and white livery, Elena blushing and tying a scarf around his black steed's neck, chaperoned by Griswold. I frowned at this last minute change in the program. There was more plotting in my childrens' box than I had bargained. I would have to ask Timmon to investigate this.

A knock a the door pulled me back from the edge of my box. Firi entered, her gaunt cheeks and dark circles under her eyes brightened by a radiant smile, dragging her reluctant half brother behind her. "Sit," she ordered the young sailor, after they had both passed through the necessary formalities, "and stop pretending to be like father."

Eugenio perched at the edge of a bench, suitably admonished, clasping his hands so tightly in his lap that the tips of his fingers turned white. Intentionally oblivious to her brother's nervous state, Firi pecked me on the cheek, then turned the attention to the table piled high with refreshments, putting together a plate of sweets and fruits. She wore a simple green dress, bright and new, but with no embroidery or ornamentation to speak of. She had lost all taste for such frivolities. Today, however, she had gone to the effort of piling her hair in a much more elaborate coif than she usually favoured, with precious stones and bits of gold glinting in the spaces between her curls. I complemented her on her appearance and silently wondered what had brought about this sudden change of heart. Eugenio remained a statue on his bench.

As the silence grew, it became clear that it fell upon me to begin the conversation that brought my two young guests to my box. I turned to the young man. "How was your journey down from the sea?"

"Uneventful, your grace. Thank you for asking."

Silence fell again. Firi pulled a stool intimately next to my chair, sitting so our knees almost touched. She seemed to be willing this to be a pleasant gathering of friends and family, and she delicately nibbled the wings off a sugar dove, ignoring the tension radiating from her half brother. In the arena, Vittorino's champion fought bravely against Baron Cesaro Madriano. Youth against age, strength against experience. If I were to back a favourite, it would be Carlotta's husband. Eugenio absently watched the fight. The tension in the box grew.

"I expected to find you championing my son," I continued.

Eugenio's hands gripped each other more tightly. He looked away from the victorious Cesaro, down towards his feet. "The duke and I have, um, fallen out of favour, you grace," Eugenio muttered. Then hurried to add "I would not have dishonored him in battle. I could never do that. However, he chose a different champion."

Firi snorted in disapproval, then popped the sticky remains of the sugar bird into her mouth. I looked from sister to brother in amazement, wondering what secret had passed between them, and why she wanted Eugenio in my box. The victorious Cesaro did a dignified victory lap around the arena, and disappeared out the north entrance, while white robed healers removed the still form of Vittorino's champion by the east gate.

I turned back to the curious situation of Sophie's children.

"I had also expected to find you in your father's box this afternoon," I pressed on. Eugenio blushed, but remained staring, stone faced at the floor.

After a while, Firi supplied "The idiot has fallen out with him as well," with an uncustomary lack of grace.

"Thank you, dear," I reprimanded. "I believe Eugenio can speak for himself."

The girl put down her plate of treats, glared at her half-brother, then pulled her stool to the edge of my box to better watch the action. The young Mario, Griswold's champion, crowned by Elena's garland of flowers, pranced a long elaborate introductory lap, waving to the cheering crowd and saluting his patron. Firi blushed as the freshly made hero and heartthrob passed below my box. My heart lifted to see her take a healthy interest in men, after everything she had endured.

Eugenio started to fidget in his seat. "There is something you came here to say," I prompted gently.

Suddenly, Eugenio was on his knees before me, grasping for my hand. "Your grace, I wish you to know, that no matter what you may hear from Duke Griswold over the next few days, or whatever my father may say about me, I promise you, I am every your loyal servant. I would never do anything to injure your interests or anyone you ever cared for." He spoke in a low hurried voice, urgently, and clearly overwhelmed by the tangle of intrigue he currently found himself in.

"Well," I said, and stopped, considering how to proceed with this poor floundering boy. I chose the role of the matron. "Sit up, child, a stop being dramatic. I am certain there is nothing here that cannot be resolved with a deep breath and a few honest words."

A shocked uproar outside my box distracted our attention. A short plump nobleman ran a hasty introductory lap around the arena. His face was covered in blue and white war paint, Nievian style, completely masking his features. From the other side, I could see Baron Siasso angrily stomping and gesturing from his box. "That is not my son," I saw him mouth.

My stomach fell as the pandemonium of a well trained strike force broke out on the grounds. Healers and soldiers alike stood ready for some unknown disruption to the day, though we had all prayed that none would happen. A half dozen men at arms ran into the stadium to surround the stranger. Healers in all colors of clothing ran down the various egresses in search of the Barron's son. Another dozen unarmed men and women rose from their seats and walked swiftly down the aisles. They nimbly vaulted the low wall separating athlete from spectator. I watched, horrified and astounded at the overwhelming force Alerio and Vittorino had managed to have in place for the day.

Then foot men fell. By ones and twos, as each reached within 5 feet of the mysterious rider, they stumbled and fell back. A gifted soldier, I wondered, but that did not make sense. I could, perhaps, protect myself all 6 men in this fashion, but my gift still ran strong, and I had enjoyed years of training and experience. In my prime, I could hardly manage 4. The rider, though his features were hidden by paint, looked young. His weilded the gift with incredible strength.

The healers streaming from the audience broke into a run, heading for the struggling foot soldiers or towards the black riders who had streamed out, leading mounts for the gifted support. The masked noble had drawn his sword, desperate now, clearly not expecting this opposition. He rode towards his opponent, Griswold's champion, with weapon raised in a single minded fury. The young hero, also new to battle, panicked and rode away from the crowd of his allies, into the open space at the other end of the arena.

"Fool!" I cried, though there was no hope of my voice carrying over the roar of the crowds. Out of the corner of my eye, I saw Mirella pointing at my husband's box and struggling to get her brother's, Griswold's, attention. She looked horrified. I followed my daughter's shaking finger to find my husband's box empty except for the sobbing Nievian woman.

Recognition was too slow in the coming. By the time I had turned back to the battle, the new black rider, Mario, had loosed an arrow. It swam through the air, it seemed, no faster than I could run, though I was rooted to the spot, unable to stop the carnage to come. The arena, it seemed, watched as one, to shocked to even breath as the arrowhead found the clear jelly of Makala's eye and lodged itself well inside my dear second born's brain.

************************

The unthinkable had happened, though it took some time for that to become apparent. At first, it seemed only that a young gifted nobleman, dressed as a Nievian warrior had taken the young Baron Siasso's place against the black rider Mario, Griswold's champion for the games. This was not a planned substitution, a poorly planned attempt humiliate the young duke on his birthday. In detail or execution, it seemed nothing more than an abysmally timed prank played by one of another of young Griswold's rivals. None of the seasoned players of the day, either on the side of the Tower or the barracks would make such a paltry display for so little gain. Even as a decoy, it attracted attention, but not nearly enough of the force posted to keep order to make a difference.

I frowned at the ill timed prank of a politically illiterate rich brat and leaned forward to watch the response of the men we had all worked so hard to put in place to make the day pass smoothly. There was too much at stake for a mere boy to toy with our efforts. There were over a dozen men in the arena, far more than the circumstances required. Commander Sattore and Head Alerio had also determined that this was a foolish son of a rival. Their goal was to capture, not kill. I watched from my box for a long tense moment while the boy evaded his captors, wielding his gift with unusual skill.

Then, the green boy, Mario, the hero of the day, drunk on his glory and Duchess Elena's attention did was all hot headed young heroes do in battle. The fool shot his noble challenger through the eye. The healers and men in place to corale him hesitated, then futily ran forward to save him. Damage to the brain, even Nisrita in her prime could not save the boy from that wound. Mario sat proud and erect on his stallion for one brief moment until the horror and disgust at his action exploded from the ranks of spectators. Marsean games are not played to the death. Death matches have been banned for generations, and strictly enforced everywhere except at the most savage frontiers. A champion wounds his opponent until they cannot fight, but it is an unthinkable waste of skill and human effort to kill a veteran warrior. The games focus on how swiftly or creatively one wounds one's opponents or how well one continues to fight though maimed. To kill in the arena ...

Behind me I could hear Valira crying, and Sophia quietly vomitting into a bowl. I turned to see Hinata caring for his Simeta, and Lyca, pale and inscrutible, staring at the neighboring box containing his sibling with large dishlike eyes. In the center, young Griswold stood alone and erect, violently rebuffing any efforts from his peers to comfort him, barely tolerating Mirella and Emile clinging to him in tears. His eyes followed his father's figure who had just appeared on the arena running towards the knot of bodies around the fallen. "What interest does he have in this mischief," I wondered.

A few black riders lead their unhappy companion away, stunned now, slouching in his saddle, protecting his head from the volley of rubbish the crowds threw at the killer. There would be rioting soon, I thought, and moved to see what I could do to aid the men already in the stands.

I found my path blocked by Lyca. "Mother needs you sir." He said in his customary strangely distant manner.

I put a hand on his shoulder to move past him. "There will be rioting in the stalls momentarily. Take care of my family, son. Hinata, with me."

But Lyca did not move. "Mother needs you sir," he simply said again, louder this time, with more authority, but still with uncanny calm. Something in his manner arrested me. He knew more than he was letting on. When I did not move, he continued in they same childish but commanding tone "If you cannot help her, the situation will only get worse."

I roared something ugly in my frustration. Curse the duke and his childish mysteries, but he had planted doubt in my mind. "Hinata!" I barked, "Keep them safe," and tore out of my box.

*********************

My duchess was a wild animal. She kicked and scratched and bit at Eugenio, who tried desperately to keep her from leaping the eight feet down into the arena stalls. Firi stood, dumbfounded and scared, just out of harms way, clutching Nisrita's talisman to her chest. A thin scratch ran across her cheek, and a deeper cut on the palm of her hand stained the broken chain of the dutchess's talisman red.

I stepped quickly forward to get between my duchess and the railing of her box. "My son!" she howled, desperately trying to wrench her wrist free of Eugenio's grasp. "Let me get down to my son!"

My heart skipped a beat. I had just seen all of the duchesses children, either in my box or the neighboring. That only left.... I looked across the arena to where space where the duke had made his speach. Only the young Meshkenet sat sobbing her eyes out, a single pathetic maid tried to comfort her lady. On the packed yellow earth below, another scene of parental grief played out in full view of the public. Commander Sattore and a handful of healers stood between the immortal duke and the dead youth, who had now been pulled off his horse and laid on the ground. Head Alerio himself oversaw the futile efforts to tend to the wounds. The duke's minder's flinched and beseeched him to see reason in turn, as he raged and demanded to be let through. There was murder in the duke's manner, and I suspected Alerio was his prey.

I did not have time for my own rage at the ill spoken, ill natured, uncontrollable braggart of a hooligan dukespawn to settle in. Someone launched himself into the crowded space behind me, sending furniture crashing and causing Firi to scream. "This is your doing, woman," young Griswold's voice bellowed. "A bitch killing her runt has more sense than you. That boy is my brother! My flesh and..."

The unmistakable thud of flesh hitting flesh cut the young duke off. Eugenio had released the duchess, who ran headlong towards the railings and into my arms, to turn on his foster brother. "One more word," he growled. Just over twenty, and still in the process of filling out his frame, Eugenio towered over the fifteen year old duke. The younger man stared up at him defiantly, spittle, and now blood from where Eugenio had punched his jaw formed a spray over his lips. The older boy shook with fury. These two had argued earlier today. Claiming blood ties with the monster who had torn our family apart was more than my step son could take.

Young Griswold opened his mouth to speak and Eugenio pulled back his fist.

"Enough!" I put a hard edge to the command. I dragged the now stunned and still duchess away from the railing and pushed her into a chair, where Firi immediately went to hold her hand. I stood between the two quarrelling boys. "Eugenio. As you seem to have forgotten, before you stands a duke. You are a sailor. Make your ammends."

Eugenio glared. Standing an inch taller than me, he was certainly stronger. He looked as if he would fight me as well. Then his eyes darted to his sister behind me. Something passed silently between them, and Eugenio, surly and unrepentant, kissed the younger man's hand and formally asked his forgiveness. The duke barely acknowledged the gesture.

"Take your sister and her grace to safety," I instructed, before Griswold could change his mind. I waited for them to leave before turning on the youth. "And you," I began, realizing my voice was shaking. Alone with the young man, I too found it hard to concealing my fury.

"I want her dead" he raged, totally unaware of how he insulted my family.

I wanted to beat the stupidity out of him, but he was no longer a child of seven. "Your twin" I explained in a hoarse whisper. The word tasted vile in my mouth, "rode against you, boy. He wanted to see you humiliated... He would happily have seen you dead." The youth flinched imperceptibly, but stood his ground. I stared him down, but he would not concede the point. "You wanted to lead." I said at last. It was not a question. When Griswold showed no sign of comprehension, I continued "You wanted your mother to abdicate so you could take her place." His dark eyes widened in surprise at this, the most secret of his schemes. "Now is your chance boy. Keep your people from rioting."

Griswold paled and started to protest, but I gripped him firmly by his shoulders and pushed him to the front of the box, where he could be seen by all. People had started making for the exists. It was a trickle now, but if panick grew, the masses would stampede, and no number of healers or soldiers hidden in the stall could control the losses. Those who had thrown brick-a-brack at the offending black rider vented their anger and fear at each other. Scattered scuffles had broken out across the arena. Griswold scanned the scene before him for a long time before taking a deep breath. Somewhere from highest rafters, a trumpet announced the new speaker.

"Father," he called down to the arena floor, interrupting the old man's verbal assault on the soldiers that formed a cordon around the healers and the dead boy. Griswold looked up, startled. "We all grieve for," the young duke paused. I found my fists clenching in anticipation of the words my brother. "the loss of your ward," young Griswold continued with difficulty. He waved his arm theatrically at the crowds, some still fighting, hurling insults and food at healers or armed guards, others hunched together in a frightened silence looking from father to son. The boy knew how to make an impression. I had seen it on the training ground, and had hoped he would carry the day. "We are all shocked by this afternoon's fight. A young life has been taken, and we all need time to mourn. Let the healers do their job, ask the army to help your people go home, and keep peace within your walls. Those responsible for this will be brought before you for justice, in due time. I hope" he paused, and for a moment I thought he would turn to me. He gathered his courage and continued. "I hope that this is merely the foolish act of a rival wanting to shame me. The young man has paid dearly for his actions, and I do not hold it against him." The young boy turned from the crowd to his father, who now himself to be corralled a distance away from Alerio's band of healers. He looked up again when his son called his name. "Let me help you, father. This is a time for families to come together. A new widow awaits you in your box. Baron Siasso's son remains to be found. Cortan needs our help."

A silence fell over the arena. I admired the speech. He had laid out tasks for the two rival factions of the court, called for peace, and reminded his father of his duties. Young Griswold would do well. Suddenly, his father turned sharply towards the arena gate closest to our box. The young duke heaved a sigh of relief and moved to meet him. Their appearance together in the arena, in a climate where the son rarely appeared in public with either father or mother, would signal a reconciliation and a joint call for peace. "You did well, son." I told him, as he passed me.

For a moment he did not answer. Then he turned on me, suppressed fury turning his face the color of a bloody bruise. "If I learn that you even suspected that this would happen and did nothing to stop it.... By the Destroyer, I swear, there will not be a mousehole on earth or in any your seven unholy spheres where you or your family will ever find refuge."

*******************************

It was dusk by the time I had seen the last of the healers and Commander Sattore's men return to the Tower and barracks respectively. What small fghts that had broken out in the streets had been controlled, the injured seen to, the trouble makers chained to the city walls to contemplate their misdeeds and await the Duke's judegment. The birthday feast had, of course, been cancelled under the circumstances. While a curfew had not been enforced on the city, there were extra guards everywhere, discouraging the general population from leaving their hearths.

I rode home through a city in mourning, some because they sensed a change in the political winds, but many with true feeling. With the raw public display of emotion from both duke and duchess, no one refrained from referring to the nwly dead by his birthright. I wanted to cut a new mouth in the face of the next person who voiced their grief for the lost young duke and sympathy for his family. A younger me would have acted on this impulse, I reflected, and I would find myself either dead, or tied with other troublemakers at the wall. I clenched my jaw and kicked my beast unnecessarily hard. I had a family and a dutchy to protect now.

*******************************

Hinata did not come to greet me when I reached home without incident, though the house was lit up as if we were entertaining the duchess on a formal visit. I paused before crossing the fountained court to my hall, overwhelmed by a sudden swell of pride and warmth for my wife. Sophie intended to make it very clear that we were not in mourning for this afternoon's events, even if we were the only house in the city to take such a stand.

This was not safe, I knew. Even before today's events, having all my family under one roof was tempting the gods' mercy, and Baron Paulis's desire to destroy me. Now, with half the extra security I had set up for my family requisitioned to keep peace in the city, and this display of defiance in the face of a duchy in mourning, well.... Sophia would be not entirely unreasonable to demand that I send her and her children back to our southern lands for their safetly.

I shook off the thought and took in the house. There was activity in the green room, I could hear laughter and the murmur of stories being told. There were two new carriages in the yard, one, barely more than a chair draped in the  Tower's Whites. I took a deep breath and steeled myself for what was to come. What I needed tonight was a long bath, attended to by Sora, and an evening playing chess with Eugenio; anything to forget the bloodshed and betrayal I had witnessed today. But there were guests and, probably, matters of court to attend to.

******************

Sophia emerged from her room immediately upon my entrance. She had posted men to the doors to most of the rooms in the house. One stepped aside discretely as she floated to the head of the stairs. "Welcome home, husband," she beamed at me with uncustomary radiance. I returned her grace, but before I could lead her to my office to talk, she swept down the stairs and took me firmly, but lovingly my the elbow.

She led me to a small table in the garden, laden with fresh fruit, ewers of lemon water and rice wine, along with various other light refreshments. A shallow bowl of scented water lay beside a chair, ready to wash the dust off my feet, along with a pair of fresh sandals for indoor use. Sophia pulled out a chair for me, and looked at me beseechingly. She had never washed my feet in Lir, and I did not expect her to start now. Above me, the curtain to Sophia's bedroom flicked shut, and I glimpsed Sora's brass forebrace glint in the light of the candelabra within. As i immersed my feet in the deliciously cool water, i wondered why my ambitious Liri kept man was in my wife's chamber.

Sophia, for her part, hovered around me with unusual attention to my comfort, pouring drinks and offering cool scented cloths to wash my face. She had changed her elaborate red and white lace gown she had worn to the games for far more modest and, in the summer's heat, comfortable, healer's robes, though by no means did she wear her garments demurely. She belted her tunic with a chain hung with precious stones of every color of the rainbow. Her arms were laden to the elbows with glass bracelets of crimson, teal and chartruse that tinkled and sang as she moved. She wore heavy gold earings, and matching pins kept her hair swept high upon her head, exposing a graceful neck. The simplicity of her dress, if anything drew attention to the richness of her attire. The entire effect was mildly scandalous.

"You will have the full force of the Tower upon my house if anyone sees you dressed like this," i admonished.

Sophia smiled, finished drissling honey and nuts on my dates and ignored my remonstration. "My sources must know that they are free to speak to me against their superior in Escasain's Tower," she said airily. She sat beside me and popped a honeyed almond into my mouth. Then she nibbled delicately on one herself.

Not for the first time in our marriage, i admitted that she was skilled at what she chose to do. She could change between the charm and seduction rivaling Sora's gifts and authority and grace of a poweful baroness as easily as most women changed shoes. Her mind was as sharp and dependable as Hinata's. if only, in these troubled times, i mused, i could get along with her well enough to have her by my side. i was in desperate need of an ally.

"Timmon," Sophia crooned, when i predictably failed to be taken in by her allure. "What did you know about today's events?"

i looked up sharply from my musings, young Griswold's last threat suddenly floating to mind. "What have you heard?" i barked.

"Nothing, dear husband," Sophia said, too sweetly. "i am simply asking."

i looked away from my wife's well preserved face to study the plate of food before me. That the young Griswold would suspect me of allowing such an attempt to humiliate him to continue to satisfy my person desire for vengance hurt. But it wounded me more that i had, indeed, absolutely no prior knowledge that such a pathetic attempt had been planned. "Nothing," i mumbled to the table. i had no one else to confess to. Given the afternoon's events, i did not know how i felt about speaking to the duchess about this, or any matter anymore.

"Timmon," Sophia urged, leaning forward to cover my hand with ten jeweled fingers. "you have been working too hard. You cannot serve Cortan endlessly. You need a rest."

"And you know of a lovely country estate less than a week's ride south of here, a virtual oasis in the grasslands that would be perfect for me," i said wrily. I assumed that this was yet another of Sophia's attempts to show me to be unfit to protect my children.

Sophia ignored my attempt to pick a fight. She had wanted me to leave Cortan for years now, and manage my own estate in Escasaine. Why, i could not imagine. Without my presence, she had the freedom to do what she wanted, both with the land and her guests. "Come home, husband," she said softly. She sounded genuinely concerned.

Her tone of voice gave me pause. The events of the day had worn at my loyalties. My dutchess, for whom i had sacrificed so much, had turned against me, as had her brood. i had overlooked, and gladly, the violence that her offspring had done to mine when she had denouced her second born child. Until today, i had believed her denounciation genuine. Meanwhile, i further endangered my children, who i valued above all other possessions, by exposing them to factions in her court. i had taken up the cause of the Liri people in Cortan, completely of my own volition. i had been Marsea's ambassador to Lir, but my term had passed. in turth, i had not more duty to the people. Hinata and Sora, my two Liri slaves, out of loyalty, helped more and less as they would. Something twitched in my mind from earlier in the conversation.

"Sora," i said. "You've been talking to him."

By way of agreement, Sophia tightened her grip on my hand and repeated "Come home, please."

i could not take Sora to Escasaine, i realized. My kept man for over two years, i had never managed to tame him. He remained ambitious and sulky, longing to leave my service and return to the Liri royal court, from whence Eugenio had acquired him. in all likelyhood, he was harmless, but with Eugenio's recent decision to marry into a Liri merchant family, buying the supposed safety of his gifted female relatives, i could not take any further risks. Sora knew too much about the gift, and too many of the weaknesses of both Escasaine and Cortan to be allowed access to the Liri border. i could not unduly restrict his movements without endangering exposure. That left me little choice. "You are right, Sophia," i said with a sigh. "i cannot protect my family here forever. The atmosphere here is volatile. I cannot see the repercussions of today's events. i will ask Hinata to escort you and the children home before first light."

Part of me rejoiced as i spoke these words, the part of me that still loved the duchess and would not be separated from the earth where Makala's bones lay burried. But the rest of me churned with rage. For the last year, in spite of my wife's constant insinuations of my incompetence, i had fought to keep my two eldest children beside me. My responcibilities to them kept me from being entirely consumed by the constant struggles, intrigues and dangers that otherwise filled my days. This was not a tactical retreat, I had been completely routed.

I looked at my wife ready to rail against a superior expression that was not there. Sophia seemed genuinely sorry. Something worried her, probably me. "Where are the children," I asked briskly, to change the subject.

"Valira and Desmond are in the green room with Carlotta's three," Sophia responded immediately, recovering herself. Well, that explained one of the carriages in the yard. "Eugenio has gone with Hinata to maintain peace in the Liri quarter," she added without pause. If she left him out of the list, she knew I would only ask after him next.

"And Firi?" I asked, after she did not continue.

Sophia blushed, and hesitated. "She is in the temple, keeping vigil." I balked. "When she heard that duke Griswold had washed and prayed over the body over the body of his dead son, but refused to keep stay for the wake, she announced very publicly that she would undertake the duty for as long as none of the deceased family were not present."

My mind, exhausted from the day's peace keeping, battered by the disloyalty of the ducal family, and the false grief of my fellow citizens', failed to comprehend this new turn in events. Firi was not the type of child to publicly act against her family's best interests. My jaw hung open for several seconds before I finally managed to growl, "People will talk."

Sophia's lips pursed, showing that her disaproval of her daughter's defiance and rashness equalled or exceeded mine. "I said the same. But she claimed that her father could withstand it, and rode off. Hinata had left with Eugenio already. I could hardly go after her myself." She looked at me reprovingly, as if it were my fault that the house were understaffed, and my daughter unruly in the face of this crisis.

I sighed heavily. This was the last thing I needed. "I will bring her home later. At least she is safe in the temple." I poured myself a glass of rice brandy to bolster my flagging strength. There was one more unanswered question to tackle tonight. "Now, dear wife," I said firmly, "Would you care to tell me who is in my library, and why it is so important that I meet with you before I go to see him?"

My wife's composure faltered for a moment before the supremely competent face of the lady of my house reasserted
itself on her fine features. She had weathered some conflict on this point, though she would never admit it to anyone, least of all me. "Head Alerio has come to see Carlotta. However, as she came to me hours before his arrival, crying and begging for protection for her and her children, I have taken the liberty of keeping them separated."

***********************

"Good evening, head," I said, a little while later, forcing a smile onto my face. I was in no mood for this interview. This withered old man, who could not keep his own students in order had come to bring chaos to my house. Yet he had come. As I could not simply pick him up by his collar and throw him over my balcony, Hinata followed with some of the Firvonese brandy that had once been reserved for the birthday celebrations.

The old man turned his attention from the map of the Liri coastline hanging on my wall to greet me. He had changed from his laughably out of fashion court dress of the afternoon to healer's robes. Unlike my wife, his attire was entirely white, but far from simple. He wore a high starched collar, mother of pear fastenings, and every inch of cloth was covered with fine embroidery in a ivory silk thread that shimmered in the candlelight as he moved. Simplicity only in name, the effect was gaudy and foppish.

"And to you, Barron. I trust you and yours are all safe tonight?" There was warmth and genuine concern in his voice, and I was in no position to endure it. This twig of a guest in my office was Nisrita's lover. After her betrayal this afternoon, I could not face him. 

I sat down abruptly and grunted. The head was playing at something. I had to hear him out. 

Hinata poured and served our drinks, then retired discretely. The old man sipped his brandy, balancing the liquid on his tongue as he would at a feast. I thrummed my fingers impatiently on the arm of my chair. Eventually, he swallowed, gave a satisfied sigh, and sat on my couch, unbidden.

"The streets are quiet tonight," he mused. "There are three groups of armed men patrolling the streets tonight, which could cause trouble as easily as it could keep the peace. I'm surprised..."

"You cannot have Barroness Madriano," I interrupted. "My foolish wife has offered her shelter. Therefore I must uphold her promise."

"Must you?" The head's lips twitched in mild amusement. "How long do you intend to host the good Master Carlotta?" I could not tell if he toyed with me out of long habit or malicious intent."

I made up my mind suddenly. "One day. Tomorrow, we leave for Escasaine." Sophia was right. I had sacrificed to much for this duchy and its duchess. And she valued it less than the rats in her own cellar. It was time for me to return home and see to my own. 

"That will have to suffice, I suppose," said Aleiro, with an air of mild resignation, then crossed his legs under his robes and made himself comfortable on the couch as if ready to start a long and friendly chat.

He sipped his brandy slowly. I watched, restlessly; unable to act on my desire to rid my house of his presence out of the desire to know the real purpose of his visit. Somehow, I realized, I had ceased to be master of this audience granted in the most private of my chambers. The snake.

"It would seem," he drawled, after slowly emptying his glass, "that we have vastly misunderstood when Griswold will move."

"I beg your pardon?" My voice gave no quarter. I had contacts in every barracks in the country. No where was there any sign of 

I glared at the healer. After everything that had happened today, how dare he, this wrinkled prune of Nisrita's lover, accuse me of failure. Our supposed collaboration had been completely one sided. The old spider guarded his secrets so well that it was a miracle I managed to learn anything of the duke's secrets at all. "I beg your pardon?" My voice came out low and stern.

Head Alerio, oblivious of the anger in my voice, continued to explain "It would seem that Griswold's genius lies, in part, with how he understands the gift. Unlike Nisrita, who was simply a source of raw power in her youth, Griswold understands the gift completely differently from the rest of the Tower's healers. Nisrita has learned to interpret the gift in this manner, and I doubt, inspite of all her strength, without the aid of these new techniques, could neither have regrown limbs, saved your life, or convinced Duke Lucretious to live."

My entire torso ached at the memory of the events of fourteen years ago. I had no wish to dwell on the details of Nisrita's miracles, and a part of me wondered if there were an insult to Nisrita's skill hidden in the heads' words. One thing was certain, there was too much praise of Griswold in this discussion for my liking. "This is old news" I snapped. "Speak your peice."

But, as is the way with old men, he would not be rushed. "I wonder," he mused, "what would have happened if she had not been so terrified of the wild magic. With her strength, and Griswold's  techniques ... " Head Alerio trailed off and shrugged, as engaged in casual speculation of next year's harvests, not of terrifyingly powerful magic. "For various reasons that I will not insult you by repeating," the old man continued, "Griswold holds the most sway in the towers of Escasaine, Cortan and Deyalorn, exactly the three places where, for the past five years, Nisrita, under my encouragement, has been teaching young students to heal using Griswold's techniques."

He paused dramatically. The implications were clear. In five to ten years' time, Griswold would have a body of healers loyal to him whom he could train to be an elite force. Since their marriage, Griswold had been undermining every magical effort of Nisritas, including her attempts to teach his techniques. I had always assumed this sprung from their own domestic discord, and perhaps at first it was. However, it appeared now that if Griswold could limit the schools where his techniques were taught, he could build an army loyal only to him. The implications were shattering. Not only could he split the country, he would turn the tower against itself. But this was pure speculation, Carlotta had failed to find any evidence that Griswold could control the magic in any real way.  I said as much.

Head Alerio assumed a very patient manner, as if explaining a lesson to a very slow student. I bristled at the insult. "He doesn't need fine control it to wreack havock. He has nearly ruined Nisrita's health precisely by his lack of control. All he needs is a bit of creativity. Consider, for instance, the esaphagus. It can be blocked completely by a finger. And the epiglotus is a mere flap of skin. To fuse flesh, why, that takes neither great effort or fine contol, and a man suffocates."

I did not follow the technical details, but the gist was clear. My hand went, involuntarily to my throat. Then, feeling that my intelligence had been insulted, I added grufly. "What does this have to do with today. The children her grace trains will not be ready for years."

The school teacher continued in the same pitying patient manner. "Nisrita taught all her children, and your daughter, to heal by Griswold's means since they were old enough to warm a talisman."

"Djziri's balls", I swore, then composed myself. "Do you have any evidence?" I asked urgently.

Head Alerio smiled sadly. "Five soldiers were found today with their throats sealed. Another had his windpipe crushed. We lost two before we could figure out how to treat them." I slumped back in my chair, too stunned to even swear. Head Alerio studied my responce, still patient. After some time, I could bear his scrutiny no longer. "Anything else?" I asked weakly?

The old man sighed. "One more thing. I have been working with a small group of Nisrita's best young students. None of what I have told you today is speculative."

I felt a surge of angry heat rise through my body. "You WHAT?" I roared, knocking my chair over in my hurry to stand. Nisrita would never have allowed this practise. The head had gone behind her back. Wild magic in all its forms had been banned for years now, not even the convocation of heads had managed to change that. What the head had done was unforgiveable. I wanted to wring his neck. How Nisrita could prefer that snake to any honest, upright man, I could not understand. Not only had he gone behind the back of his coconspirators,  he had committed treason against his duchess and king.

The old man before me remained completely nonplussed. Instead of shrinking from my form before him, he refolded his hands across his knee and quietly apologized. "I'm sorry. I have clearly revealed this to you at a poor time. I tell you this for one purpose only. I had a similar conversation with Nisrita a few hours ago, but she is in no condition to hear me. Someone must make her understand, for her own safety, and the safety of the nation's towers."

He pushed his stool back out of my arm's reach, then slowly rose to his feet. I remained rooted to the spot. Had this two faced backstabber just causally asked me for help? Even if I held any sway with Nisrita or desire to see her, I had no intention of helping HIM. I could not talk in my rage. I remained standing, fists clenched, as he walked to the door.

"And Barron Romino," the turncoat said upon reaching the egress, "you have chosen to shelter Master Madriano from me, though I only wished to speak to her. It was her task, not mine, to report on the duke's progress. I have no evidence of her treachery. But her hue and cry at this very moment speaks volumes, does it not?"

I spluttered and fumed and found my tongue at last. "You should leave," I growled. The old man gave a short polite bow and left.

Alone in my office, I uprighted my chair and placed the length of the room. I needed to think, but the day had proved to be too much. Passion and fury and fear raged through my mind like a storm at sea. The worst of it was the small, far away voice that spoke in my head, reminding me that there had only been one monster in the interview, myself.

**********************






\end{document}
