\documentclass{article}
\usepackage{fullpage, verbatim}
%*****************
% Annotations
\usepackage{soul}
\usepackage[colorinlistoftodos,textsize=footnotesize]{todonotes}
\newcommand{\hlfix}[2]{\texthl{#1}\todo{#2}}
\newcommand{\hlnew}[2]{\texthl{#1}\todo[color=green!40]{#2}}
\newcommand{\sanote}{\todo[color=green!30]}
\newcommand{\egnote}{\todo[color=violet!30]}
\newcommand{\newstart}{\note{The inserted text starts here}}
\newcommand{\newfinish}{\note{The inserted text finishes here}}
\setstcolor{red}
%***************************
\begin{document}

"Moisture" I proclaimed, walking into Alerio's office, dropping a sealed leather book on his table. The old man looked inquisitively up from his letters. "You wanted to know what caused the blight." I tapped the book. "The effects of different levels of rainfall, the frequency of poisioned mushrooms, the changes in how fast they grow, it is all right here." A bony spotted hand reached for the package. I drew it back, protectively towards me. "It stays in my office. It is sealed. Do not make me have to keep it locked."

The hand retreated, a chortle replaced Alerio's curiosity. "After two years of working side by side, I still have not earned your trust."

"On the contrary," I smiled, "these last two years have taught me to trust no one. You, at least, know of the existence of these papers."

The chortle changed into a smile that filled the corners of the Head's eyes with crows feet. "Please, sit. Tell me about your findings."

"Most mushrooms grow in dark, moist environments. The warmer and wetter, the faster they grow. The farmers in Allepo's tower, I surmise, became ambitious, Marsea's towers seemed to always want as many mushrooms as they could harvest. They put a crop in a dank shed, and watered them just short of drowning. As a result, they got a crop of thick, large, lethal mushroom. I've grown crops where as many as one in ten would kill a healer."

"One in ten? We should all be dead with those numbers."

I nodded. "The black powder the tower produced mixed the batch grown in the shed with those grown by more conventional means. Your last letter finally conviced Head Claudio to talk. He admitted that the success of their new methods caused Allepo to change their farming practises to grow them almost completely indoors. If the Towers' demanded more powder from Allepo after the blight, the next harvest almost certainly would have killed us all."

A cup of wine appeared before me. "My congradulations, Nisrita. I drink to your long health and promising career."

I accepted the toast graciously. The wine was white, dry, and very light on the tongue. I sat under the watchful gaze of five previous heads of Cortan's Tower,  carved into the stone where the wall met the ceiling. They were powerful ambitious men, that had led my duchy to greatness. They were a line i could be proud of. Alerio followed my eyes. "Do you dream of becoming one of them?"

I put down my cup and laughed. "I would make a poor head for this Tower. I am far too conservative in my views on what the gift should be used for."

"There are many in this Tower who are far too ambitious in their dreams. You would temper their vision, and they would help yours expand. It could be a good match."

I looked at the old man afresh. He was not being glib. The possibility of heading a Tower had never crossed my mind. Alerio favoured successor had died of pnemonia just over a month ago. "I am embroiled in too many political fights at the castle to want more at the Tower."

"Yes, there is that." Alerio shrugged and reseated himself behind his desk. "Could I entice you to consider one political battle that does not concern your husband?"

My stomach had rebelled this morning, burdened with yet another bout of my mysterious disease. The light wine slid easily down my throat, going straight to my head. The wine and thought of heading Cortan's Tower, the completion of my work all colluded to make me giddy. "I recall a very similar conversation occurring in this very room two years ago," I teased. Alerio had obtained permission from the crown to let me, and only me, given my known distaste for the black powder, and my expertise with the mushrooms, to study them.

"Have I not kept my word? You have not been disturbed with other requests during this time, I have not asked about your progess, no one in this Tower knows what you have been doing, even now, I make no effort to make your findings public. Now that you have finished one project, I ask if you will take on another."

I danced like a puppet in Alerio's hands. He wanted Marsea's healers to benefit from the full extent of the mushrooms. I, and only I, at the moment could help him acheive that, as much as I abhorred the idea. In return, Alerio offered me everything in his power to give, including, it seemed, the possibility of leading this tower at some distant date. He spoke to my curiousity and my ambition, tantalized me until I gave in. The corner's of my mouth curled, unbidden. "Very well, then. I will hear you out."

"If, no, when Master Carlotta discovers the cure to your ailment, and your husband's attack on you is revealed to the crown, no one will be able to stop the other Towers from wanting to learn how to control the old magic. If he tower's ambitions cannot be controlled, it will tear this nation apart."

"I had considered that possibility."

"If the crown is ready for this assault, and has an answer for the Towers, it could make us stronger as a nation."

"No," I said abruptly, losing my temper at his temerity for such a suggestion.

"Nisrita," Alerio sounded dismayed. "You have not heard my request."

"I have heard enough. I will not advocate to the king on your behalf."

"Not on my behalf, for the good of the country. The crown deserves to know what is coming. Otherwise, we are no better than two healers plotting for Marsea's downfall."

"Do not pretend, Alerio," I said, my voice rising, "that because I have done this study, out of gratitude for your help, to satisfy your curiousity, I will let you use me to futher the Tower's mad ambitions. I will not be party to returning this country to the frenzy of three years ago."

"I have never used you. Everything you have done, you have done of your own volition."

Horse eggs, I thought. He uses me to his ends the same way I convince Mirella to attempt healing when she would rather play with her dolls. "How many more men died these last two years due to a lack of experienced healers marching with our armies. The dukes of this land should expose us for the corrupt selfish breed that we are."

"Is that really what you think?" Alerio asked, suddenly somber.

I paused and checked my anger. The wine had gone to my head. "I will not tell any of this Tower's secrets. Neither will I help the White Towers wreck havoc with powers that we were never meant to control."

Alerio pulled out a drawer and produced a scroll. "This is the manifest of Marsea's Towers, as decreed by King ????. Every head recites and swears to honor the law and tradition preserved here when he takes control. Take it."

I hesitated. I did not want to be a part of this. Alerio put it on top of the book I brought in. "The Heads of the Tower control the nation's gifted, not the other way around. We dictate how we will serve, with the blessings of the dukes and kings. If the gifted are corrupt, it is because the Head's have let them become so. If there is a threat to the gift, or an advancement to be learned, it is the heads' role to see that the gifted weather the change as well as possible. These Towers will use the old magic soon. You cannot run away from it. I am giving the the opportunity to forge how we benefit from this according to your will. Otherwise, someone else shape the situation to theirs. There is no third option."

"I will find a third option." I said . "The ban on the old magic held for over nearly two hundred years. It was broken for six. It can hold again."

"Nisrita," my old tutor shook his head sadly, "you are a clever and passionate woman. Will you never learn to rule?"

This was too much. There was no need for this conversation to descend to insults. "As you have pointed out the Tower neglected to breed me for it." I placed my wine cup hard onto the table, sloshing drops onto my robes and the packet containing my findings. "If you will excuse me, there are matters at the castle that need my attention." I brushed the drops of wine from my leather folder onto the rugs covering the head's office, and left.

Head Alerio had been a strong ally these last two years. There is no doubt that my ability to withstand my husband's rage is tightly tied to his support, and even my ability to retreat within the safety of the Tower when life became too dangerous or difficult at the castle. He more than Leila, taught me to take as much as I wanted from my allies, while giving them as littles as I possibly could in return. He taught me to gleen a petitioner's motives before granting a request. I was grateful to him, yet he asked me to do the impossible. I could not put the needs of the gifted over the needs of my duchy. In his mind, this seemed to mean that I could not rule.

\vspace{.5cm}

"I will not allow stand for this, your grace. Those lands have been promised to General Sidro." Barroness Paulis raged from the window seat of my room. Her belly was large with life. The sow did not stand for much these days, I thought, inappropriately. Carlotta sat brooding in my armchair, facing her.

"Promised, barroness? I do not recall making any such promise."

Barroness Paulis gave an angry snort. "That is because you are niggardly with your gifts, your grace. How do you expect my good husband to keep your allies close if you do not reward them for their service?"

"I expect your husband not to make promises in my name without informing me," I retorted. 

"He cannot come running to you for every little decisions, your grace. He has a duchy to and a duke to watch. Those land have been promised."

"Then I wish your husband luck in convincing the duke to grant them." An opportune marriage was all that General Sidro had to gain Barron Paulis's favor. My husband had brought him to Cortan from Escasaine two years ago to replace the disgraced General Galderan. Last year, when the old general spoke loudly in support of Ambassador Romino when my husband tried to shame him for following the Liri gods, the young general opposed him. It had been a tense several months, until I decided that it might be better to take up residence in the Tower for a time and throw the Maker's Daughter's support behind Timmon. My husband had split Cortan's army. There were many in the army who objected to Duke Griswold calling Ambassador Romino a coward and a traitor, given his long and proud service. Others, led by General Sidro, thought it inappropriate to disobey the Duke's will. It ended well. Timmon's family arrived in Cortan last month, taking up residence in their beautiful new house. However, I had no reason to give my favour to General Sidro, however strong Barroness Paulis's feelings may be that the general's marriage into the Paulis family would lead him to betray his duke.

The barroness took a more indirect approach. "Barroness Madriano, Will you grant me that this caprice of the duchess goes too far. She granted your husband a poor mountainous patch of land to call his own, her own personal healer, yet she gives the ambassador some of the most fertile lands in Escasaine. It is a cruel way to treat her friends."

Carlotta did not move from her seat. "I take no part in this argument, Leila."

Barroness Paulis's next words were directed towards me. "General Sidro was at the fall of Escasaine. He played a cruicial role in making the city fall."

"General Sidro married your husband's sister. My answer stands. We are done for today."

"Your grace," the barroness persisted. 

"I said, we are done. Must I escort you to the door?"

"No, your grace." Barroness Paulis rose and proudly waddled to the door. 

I called after her when she reached it. "Next time your husband wishes to ask me to overlook a mistake he has made, have him atend me himself."

The expecting barroness gave an insulted cry and disappeared from my view. I put my head in my hands to calm my nerves. Every discussion, it seemed, ended in anger. I had exchanged regular terrifying encouters with my husband for small, incessant arguments with those that would be my allies. This was a prefereable life by far, but it was exhausting. I had been foolish, I supposed, not to pay more heed to the Paulis ambition two years ago. I had other, more pressing matters at hand then. Now, if Carlotta ever rid me of this disease in my womb, it would be the Barron who would rule in my name, not my husband. The court he had built against Duke Griswold was his. Aside from Alerio, Carlotta, the retired and shamed General Galderan and the head priest of the Liri community, Nagayi, I had no true supporters. In the mean while, men from Deyalorn and Selvand filed into my duchy every month, occupying positions in the Paulis court. A duchess needs kin if she is to stand against her husband and rule. An orphan has nothing. 

I rubbed the frustration from my eyes with the heels of my hands and looked at Carlotta. "You have been quiet this afternoon."

"She has a point." Carlotta spoke with a low voice. "You should not have given lands to the ambassador."

"Are you jealous, Carlotta? Commander Madriano was happy enough with his lands when he thought there was silver to be mined."

Carlotta's eyes narrow and her nostrils flared, but her voice remained low. "I am not speaking for myself. The ambassador is a threat. If you did not give him lands, he would have moved his family back to Deyalorn and served King Gustav's court."

"Giving Barron Paulis complete power over my court. I may not be anyone's favourite for the throne, but I will not have him rule in my name."

"Oh stop being petty, Nisrita." Carlotta snapped. "This is not about lands or thrones. What will you do when the ambassador finds out what we are doing here? He will go straight to the crown, and shame Cortan."

Or he will decide to take matters into his own hands and kill my husband. I was aware of the dangers of my decision. But I had missed Timmon. It had been six years since I had seen or heard from him. How could I not want him near me? "Then he will simply have to not find out."

"The ambassador? Not find out?" Carlotta laughed. "That is impossible."

"No, merely difficult." I insisted against my misgivings. "Three people in Cortan know of our efforts. We have kept the news secret from the entire duchy, including my husband. The ambassador is one man, a mortal at that."

"Stop being flip! Tell me what you will do when he gets wind of my work."

I studied Carlotta. She had not managed to cure me of my husband's mysterious disease, because my husband had not yet managed to find a way control the dreams as well as he wished. For the first year that Carlotta tried to follow his instructions, she killed everything she turned her hand to. As her skill grew, she could attain the results she wanted only some of the time, infrequently enough that neither of us wished her to work with my body. Carlotta enjoyed working with the old magic. It frustrated her, as I am certain it frustrated my husband, that one could not both control the dreams and perform any grand feats of healing and destruction. They both wanted the legendary power of the priestess queens of Deyalorn, as di many Masters of Marsea. It was a selfish and vile ambition.

These were the men and women that Alerio wanted me to hand the secrets of the old magic to. It was impossible, I told myself. It would upset the natural order of our kingdom, it would destroy life for the ungifted. I could not allow it to happen. I put on a smile for my medical attendant. "Do, Carlotta? I will tell him what he does not know. Do you really think the ambassador will interfere with your work if he knows my health is at stake?"

I watched Carlotta flush, gulp, then control her anger. She was right. Timmon would be a threat if he settled in Cortan. I hoped he would be more of a threat to my allies than he would be to me.

"How are you feeling?" Carlotta asked, clearing the air of the tension between us.

Tired, but I did not say that. "Why do you ask?"

"You have been particularly testy these last few days. Does your womb need to be cleaned?"

I had not eaten well for the last two days, my stomach refusing to cooperate. I hated having my womb cleaned every few months. The other option was to let the disease grow insde me until my body could take it no more and expelled it. That was possibly worse. I had miscarried, if it can be called that, a little under two years ago. It had been a horrendously painful and bloody proceedure, resulting in what appeared to be a bunch of fleshy, misshapen grapes. There had been no placenta. Miscast, people would say, or worse, demon cursed, if they knew. They would stone their duchess to death, my husband throwing the first rock. I had suggested juicing the grapes for wine to feed my loving liege, when I saw what I had produced. Carlotta had looked at me with disgust and tossed the mass into the fire. "Yes," I said wearily.

Carlotta got up and gathered her tools while I put a pot in the fire to boil water, and called for sheets for the bed. "You really should tell me when you need this, Nisrita, rather than expecting me to guess."

I gave her an impish grin and started clearing my bed. "It is your job to guess at my maladies, dear barroness. Otherwise, why the exalted position?"

Carlotta frowned and fussed at her instruments. For all the times we did this, I always feared they would not be sharp enough. "You should find yourself a lover." Carlotta said suddenly.

"And they call me mad."

"It has been twenty ... seven months since your husband came to this room? You cannot stay paralyzed by fear forever. It would make it easier for your court to love you."

I removed the knife that I kept by long habit under my pillow and stowed it in a drawer behind Carlotta's turned back. If Carlotta did not appreciate the still pressing danger the discovery of our conspiracy had me in, there was no need to tell her. As for the thought of a man laying atop of me again, well, the less said, the better. "I do not know much about ruling, I admit, but even I can guess that half my court fighting each other to peek beneath my peticoats would not make my days easier."

Carlotta laughed. "Someone else then. I know an..."

I put my hand up to hush her and went to the window to investigate the noise I heard. "What is it?" Carlotta asked after a beat.

"My boys. We will do this tomorrow, I am needed in the yard."

"Who?" Carlotta asked.

"Two on two," I answered. Griswold and Lucretious against Makala and Emile. Emile was seven, large for his age but no match for a twelve and ten year old. He fought like the Destroyer himself. This was the largest of my failures as a duchess and mother. My sons spent more of their free time fighting each other than on any other activity. They drew knives and swords against each other, ambushed one another on their way to and from classes or training. I would have to discipline them, as I did almost every day. If my children did not learn peace, what would happen to Cortan when their time came?

\vspace{.5cm}
****

I missed Lir, I thought, standing before the small snow covered house with narrow windows build into the side of the Pensid mountains. Even coming up the Velta into the highlands, Liri houses had wide veranda's and large rooms. They were not so closely constructed, the air inside so smoky and dark. 

"Well, by my days. The father of this mountain has favoured me today." The master of this house said hoarsely as he walked to the gate. He made no attempt to hide either his joy or astonishment. A stout middle aged captain took my hand, then embraced me warmly. "Ambassador Romino. To what do I owe the honour of this visit? Romolo," he called back to the house, "see to our guest's horses." A young boy of eight or nine ran out of the house to obey his father. My new man, Hinata, a parting gift from Lyo, went with him to our things.

"You must be cold ambassador, come in side. Where are you staying? How long are you in Turina?" my host fussed and blustered as he led me inside. I did not remember the captain to be a loud man. My habits had changed. Lir had changed me.

"I am no longer an ambassador, my good captain of Celona. I owe you my life, I think we can dispense with the titles? I am passing through on my way to Cortan. My man and I have made arrangements in the barracks." 

"Thank you, sir. With great pleasure. But I will not hear of you staying alone in the city's barracks when there is a warm hearth and fresh goat in my house. You will stay with us." Ernesto introduced me to his happy family, a wife and four children, ranging in age from eleven to two. The ease and frankness with which the Captain's women greeted me discomfitted me. I was back in the presence of gifted women, I reminded myself. For the good of Marsea, they needed more freedom than my eyes had become used to seeing. 

"We had heard of your family's return to Marsea, Timmon," the ambassador's wife said, a tall skinny woman with a sallow cheeks and a large mole on her ear. "We had assumed you would have returned with them. This is quite an unexpected pleasure. Welcome home."

Home. Yes, I supposed it was. "My wife has spent a good deal of time and money building up our house in Cortan. I did not wish to deny her the opportunity to reap the fruits of her efforts. I stayed behind to meet the new ambassador and ensure that he was comfortable with his duties. And there was also the matter of seeing my stepson home from fighting the rebellion in Shiroi." I had not wanted to leave. Dress it up as I would, there was no getting around that. I had spent the last year pleading with Lyo to apply for permission to live in Marsea, to leave his hot humid home to make a new one with me on the yellow dusts of Cortan. He would nor hear of it. His wife's health was fragile, his youngerst son still unmarried, and Marsea had no universities. What would he do, scribe for the Tower? In the end, I had to leave him, though I found every excuse I could to linger. My religious guide, friend and lover accompanied me to the head of the Velta, and went no futher. He would not leave his homeland. My heart ached for him.

"You are a devoted father and meticulous in your duty. Would that there were more men like you in Cortan. The duchy will benefit by your presence, sir," Ernesto's wife praised while brining out food and drink. I wondered vaguely whether Lyo's literal moral code would permit me to accept this praise for actions not motivated as they seem. I squashed the musing when I saw the look Ernesto gave the woman, sending her out of our presence to see to my man. 

"What news do you have of" the word stuck in my throat, "home?"

"I do not know much that you have not heard of," he said, dropping a hunk of butter into the hot weak wine before him. It was a goatherds habit in these highlands, one I could never stomach while I had been stationed here. Lyo, on the other hand thought it hearty and delightful at the head of the Velta. I put a small sliver of butter in my drink for the memory. "Commander Griso, the duke, I mean, has put off the conquest of Niev for years, but sent few men to Allepo's cause during the last two campaigns. He has taken the competent generals in Duke Erfat's court back to Cortan, leaving this duchy in the hands of weak or inexperienced men. There are not many in the army who appreciate how this duchy is supported." Ernesto paused to cough and wet his tired voice. "Duke Erfat has lost lands. Niev has a new king who is building a strong army, and chipping away at our eastern border. Nearly every barron has had troubles with peasants who view him as their true monarch, and rise up whenever he wishes. Even this far west, we had trouble in the mines last year. The more our men are occupied with local revolts, the easier it is for Niev to steal our lands."

"Why do you think the Duke of Cortan has withdrawn his support?"

"I couldn't say, sir. The common wisdom says that he is a conservative duke who thinks it wiser to protect his own duchy that help his nephews."

That did not ring true. In the past, Duke Griswold had been devoted to both his nephews. Even if he had fallen out with Duke Erfat, Escasaine would belong to the young duke Makala soon. Why would he weaken his son's holdings? "You served with Duke Griswold for seven years. What does your wisdom say?"

"It has been a long time since I have served with him, I would not trust my judgement."

"I trust it more than the opinion of men who have never seen him work."

"Very well," my host pondered my question. "I cannot say for certain. Turina is far from Cortan and rumours, even more so since the Duke moved his court to Escasaine. I have heard that the Duke intends to give Escasaine to his third son, duke Lucretious, and then there is the splitting of Cortan's court between himself and the duchess. When one meets a stranger from the court of  Cortan, the next question inevitably is 'his or hers?" None of this was new to me, but none of it explained Cortan's behaviour towards Escasaine either. "This is one more thing," my host said after a pause. "Her grace is the only healer in the land who is allowed to touch the black powder. Given the events leading to the Counsel of Nine, it seems suspicious that the Duke's wife is the only woman who can work with the forbidden powder, especially after the blight. But I cannot say that it explains anything. I will not speak treason, sir, but I know the Duke to be a cunning patient man with a long horizon. One cannot always know what he plans."

I drank Ernesto's wine and ate his bread while contemplating this news. I had not heard that Nisrita still worked with the mushrooms. Who or what could possibly have induced her to take that up again? I had not heard from her in six years. I had promised her that I would not seek news of her, and repair my marriage. I had satisfied myself with the rumours the public knew. He marriage was contentious, to say the least. She had surrounded herself with the Duke's enemies, many of who had no more reason to dislike the Duke other than his rejection of their services for their incompetence or weakness. She was master at Cortan's Tower, and head of its school, frequently ill, and had a temper that she would unleash without the slightest provokation. That left a lot of room for imagination about the details of her domestic life. She had always feared her husband gaining access to the gift again. Had the duke forced he to this study? I wished Lyo were still with me. Over the last nine years, I had grown accustomed to testing my ideas against his stronger intellect. He could find wholes in my logic as easily as I found weaknesses in my sons' defenses. He had served Lir well by working my side as ambassadorial staff.  I would serve Marsea without him, but it would not be the same. Lyo had given me Hinata as a replacement. The man was an adequate secretary and had the potentials of being a clever spy, but he could not be a replacement. Lyo had been my equal, no, my better. Copulating with a servant, especially one as unwilling as Hinata seemed to be, matched descriptions I have heard men give of mating with their wives. It was not a thought to be dwelt upon. As Lyo was quick to remind me, Rishiki only blew favourably upon those who did not despair.

"We have the day ahead of us," Ernesto said at last, after I had eaten, and Hinata informed me that my belongings were settled in the house. "What is your pleasure?"

"There is one thing." I began. Ernesto gave me a curious look when I told him, then went to gather the horses.

\vspace{.5cm}

"After the blight three years ago, I started teaching a class of boys at Turina's Tower. Many of the teachers had gone to Escasaine, as that tower had lost many of its masters and healers. I am not a man of letters, but I have been healing and marching for many decades. I teach what I know and serve how I can. I am not a healer of the Tower. Most of my duties involve the castle's walls. It is the smaller Towers of the land that have suffered most from the blight, though they did not lose many people. With openeings in majors Towers across Marsea, no one wanted to stay behind and serve in small forgotten centers. Turina's tower make ends meet how it can. It is not a bad life for a gifted fighter to lead." Ernesto broke off in a fit of coughing. For a man with a damaged voice, he was surprisingly garrulous. For the several hours we had ridden from Turina to this plain, he had spoken of his family, his barron, the campaigns in Niev he had seen that I had not, the doings of officers we both knew, living and dead, pausing only to cough and wet his throat. He was a font of information on the details of life in Turina I had not recieved in letters, or not thought to inquire about. My host was a happy man with a quiet life. Under different circumstances, I would have enjoyed nothing better than his converstion. Under the circumstances, his words tired me. These were Liri habits, I reminded myself. I had grown too accustomed to Lyo's soft spoken nature and well chosen words. 

I beat my cold hands together and brought my mind to the icy mountain path I had not traversed in twelve years. I was in Marsea now, no longer in Lir. My nephew led a barrony without a Tower of its own. It shared a small Tower with the barrony of Lupicas. He had not written of any difficulties in his lands. If here were a clever man, he would have written to his father in law in Selvand, or to the duchess for help.

The road opened before me, onto a shallow slope that leveled off to a high plain. "It is cold, Ernesto, and we have a long road to return along. We will make better time if we let our horses run in the open field." 

"As you wish," my host rasped.

I spurred my black mare into a trot down the hill, and then to a smooth gallop as the field opened before us, leaving a spray of mud and snow in our wake. This was not the horse I had come to Turina on. She used to carry a black rider of Escasaine, but now she, like my host, was to old to serve with that elite band of men. Ernesto filled Turina's guards stables with the powerful black beasts that the riders deemed too old to use.Turina's guards had the second best mounts in the duchy. When he offered me one of those to carry me to mine in the high plain, I could not refuse. Lyo would laugh at me if he could see me like this. Only women and servants ride mares. Only a stallion or the backs of other men were suitable to carry men. The Liri had strange notions about manhood, some less practical than others. They had little knowledge of horses. I had managed to convince Lyo that it was not disgraceful for the Marsean ambassador to be seen atop a powerful Marsean mare. I could never get him ride one himself.

I slowed my horse where Ernesto indicated. A low wide stone structure lay before us under a thin layer of ice and snow. I pushed thoughts of Lir from my head. This was no place to recall other lovers. 

"As you can see, the rift has been covered. The mine is still rich in silver and tin, we could not afford to leave it unused in honor of the dukes and men who gave their lives. Eight or nine years ago, we cleared the bottom of stone and debris, covered the top, and started mining. There is an enterance over there, if you would care to go down."

I looked at the low building at the end of the stone roof. This was where it had all happened, Makala had died, my journey to Lir had begun, though I did not realize it yet. This had been the birthplace of this new duchy, held as a protectorate for thirteen years, to be held as such for six more, until it would finally see its first duke. This was a powerful place, full of memories and potency, the mine had been a vicious maw. "No, I think I do not need to enter the mines."
I looked around the field, now dotted with a few farmhouses. "It seems bigger than I remember."

"It is. We have cleared a good deal of forest on either side. It is good farmland, though the growing season is short here."

Ernesto was proud of his barron's accomplishments in civilizing this land and brining it into Marsea's fold, as he should be. This field was very different from the last I had seen it in the aftermath of that bloody day. Ernesto had been in a self imposed exile then. He had not fought at the battle of the high plain. He had not seen Cortan nearly fall. I hesitated a moment at the battle field, wanting to pray, but uncertain of which gods I should speak to. I had relinquished my habit of appearing before the triumverate since the young Griswold told the elder of my fealty to Rishiki. Yet it felt wrong to ask Rishiki to watch over the souls of the men who had perished here, they were in the Destroyer's realm, not the wind god's. I settled for a moment's silence, and memories of friends lost. 

Eventually, I grew cold and returned to my waiting host. "Thank you." My voice emmerged as hoarse as Ernesto's.

"You fought on the high plain?"

"Yes," I gave the rift that had consumed Duke Lukos' life one last look before mounting. "It was a battle that changed my life."

We rode in blessed silence for most of the return journey, my mind flitting between Makala's quiet laughter and ambitious dreams, and Lyo's serene smile and methodical mind. The two men had been so different from each other in all ways but the two that seemed to matter the most at the moment. I had loved them so deeply that they had changed my life, and they were no longer within my reach. My heart ached with loneliness. Perhaps it had been a mistake to perform this pilgrimage. I did not know what lay ahead of me in Marsea. It was not wise of me to start it out looking back.

\vspace{.5cm}

Young duke Griswold met my camp on my last night on the road. He came with a retinue of a half dozen guards. He had grown taller and broader in the two and a half years since I had seen him last. "General," he said, in a voice that had not quite finished becoming a man's, "I have come to welcome you home. My father sends his regrets for not being able to come himself."

That anyone had come to greet me at all was his mother's doing. I held my tongue and took his hand. If the question truly was "His court or hers," every one knew which court I belonged to. I did not know how the children had allied themselves. "I thank you, my duke. I am glad to find you well."

The boy did not return my affection, but went about the process of instructing him men on the arrangements for the night. No sooner was he within the privacy of my tent than I found myself in a tight embrace.

"Stop that now, my duke," prying his arms from my waist, "you are too old for this."

The boy pulled away suddenly. "Don't say that, sir. My mother is always telling me what I am too old or too young for. I've missed you. There is no one here to see."

I found myself grinning at my ward. I had missed him as well. "Let me take a look at you, then. You've grown."

"I should hope so," he said with the indignant pride of youth. He had also forgotten his manners. I cleared my throat. That was enough to remind him. "I am sorry, sir. Thank you for saying so."

"Tell me how."

The young duke told me about his training, his tutors, his renewed friendship with Lucretious, his ambitions for the cavalry, and his experience marching last year with General Sidro. He offered to show me his improved swordwork, which I said could wait until morning, told me of his friendships with his peers and of the girl he had kissed on the last Day of Unions. He brought me news of Sophia, Firi, Desmond and Valira, complemented me on my house, which I had yet to see, and asked after Eugenio and his flourishing naval career in Lir. He went through the process of asking after all his friends in Buzen, but he brought me no news of his siblings or parents. 

"You are not happy here?" I interrupted.

The smile fell off my ward's face. "I am sir, what makes you think otherwise. This is my duchy and I will rule here. What more could I want?"

As I thought. "Tell me about your mother then. How did she react when she heard that you betrayed my beliefs?"

The duke licked his lips nervously, a gesture that reminded me of his mother. "I never meant to pry, sir. Only, ... Eugenio had dared me. He wanted to know what you wore on that red chord. You were bathing, I slipped into the bathhouse and peaked. I never told anyone else sir, I mean barring my father. I don't think Eugenio ever told anyone either. We didn't want to hurt you, sir. We wouldn't do that." The experience had been rough on him. The boy's voice cracked at the very end.

"You have yet to hurt me, either of you. Tell me what happened."

"You weren't angry with me?"

"Weren't and aren't are different words. I trust you will not make such a mistake again."

Young Griswold sagged with relief. "Thank you, sir. Mother hasn't forgiven me yet. She thinks I may turn on her again given the right price. She had me flogged for betraying you, then healed all but one lash. She said she wanted to leave me a scar to remind me not to return loyalty with betrayal. I won't. I have learned my lesson. I will guard my secrets as deeply as the sea from now. But I cannot convince her that I will never betray Cortan."

His court or hers, which Cortan will you not betray? It was a hard question for the boy to answer. Nisrita's cruelty and lack of trust bothered me. These were not two traits I would have guessed to lurk in her character when I last saw her, or even before I went to Lir. She wanted nothing more than to be left to her studies and her children. What had happened in Deyalorn to cause such a mother to scar her son. "What do you think of your mother?"

"She is a good woman, I don't think she is mad at all" young Griswold said, too quickly. "She is strict and never laughs, but she is fair. She never punishes me more than she punishes my brothers, and only when we cannot settle matters ourselves first. Well, except Makala. She is very hard on Makala, but he deserves it. She wants us to grow up to be good men and worthy dukes, but I don't think she knows how to rule. She's not like you." The young duke realized too late how dangerously he had spoken. "You won't tell anyone I said that, will you?"

"As deep as the sea, my duke. Don't worry." The boy grinned. "Why is Makala different," I persisted.

"He is greedy and jealous. He dislikes me for being a better fighter than he is, and hates Lyca for being a better healer. He doesn't understand his role as a gifted fighter. I've come to terms with him myself, so we don't fight. That was easy. I am bigger and stronger. But I can't stand aside when he picks on Lyca. My little brother is terrified of him. He thinks Lyca uses his position at Barron Paulis's page to get close to father, which is idiotic, and he thinks that I defend Lyca to win father's love as well. He is as stupid as he is stubborn. It is true that father plays favourites with us all the time, but he won't win me over. I will never forgive father for tricking me into betraying you, sir. Not for as long as I have that scar to remember by."

That said a lot about the court of Cortan. Griswold was in her court, Makala in his, or at least wished desperately to be. Lyca, I could not tell yet. With the children split like this, the allegiances and rivalries between the barrons would only be worse. I did not envy the boy his position. The dukes of Escasaine should be close. They were brothers, and shared a long boder. Escasaine had the potential to carry on Cortan's mantle of greatness. If I did, it should be a matter of familial pride for all of Cortan's citizens, not a matter for jealous rivalry. "Griswold, you are always welcome at my house, whenever you need it, for whatever reason."

"Thank you sir. Mistress Sophia has made a similar offer. It will be just as it was in Buzen, won't it, but with Lyca and Firi around as well."

I grinned and sent my duke to bed. We would leave early tomorrow morning. I wanted to be back in Cortan as soon as possible to see what the mad duchess and the immortal duke had done to my home.

\vspace{.5cm}

I arrived at the castle just after noon, my young duke and I riding hard with a handful of escorts while the rest of our enterouge followed behind. His grace still held audience in the castle. I went to pay my respects to the duke and his men. The sight of Duke Griswold took me aback. He looked exactly the same as I had last seen him. The ghost of Makala, my sweet duke, sat upon Cortan's throne, while the stone bust of the dead man peered at him from above. Duke Griswold still looked somewhere between twenty five and thirty years of age, strong, limber, able. Ten years ago, a newly made general, and a practised fighter, I thought nothing of this fact. The intervening decade had introduced grew into my hairs and written lines on my face. A diplomat's duties had limbered my mind but softened my body. I was no longer a match for this man on the battle field. I had grown old, he had not.

"Welcome back, General Romino. I hope your journey has been easy. Cortan is greatful for the service you have performed in Lir. For this, my wife has a few words to say to you." Duke Griswold could have laced his words with poison and spat them at me, and still come off sounding sweeter than he did. 

I awaited the duchess's verdict. After a few minutes, Nisrita entered. I was again struck by what I saw. It was not only that, barring one meeting with her six years ago on the day before King Gustav's coronation, I had last seen her as a girl of seventeen, and a woman of nearly thirty stood before me, it was the austerity of her presentation. Nisrita had never been a girl who had dressed according to the latest fashion, but she had never been averse to showing a bare arm in spring, or a shoulder in summer, or just a hint of cleavage in winter when all else must be covered for the sake of warmth. Her body was hard and firm, she had evidently found time to return to her training, her figure was still attractive enough that she could afford to appear as other than a matron before her court. It was not just her dress, an austere dark blue silk and wool gown that covered every inch of skin barring her hands from her chin to her toes, it was her face. Had I not known my duchess to be capable of warmth, mischeif and generosity, I would have thought her heart to be carved of stone. I could not imagine that face capable of anything but discipline, laughter was completely out of the question.

Nisrita turned to address the court. "We welcome back to Marsea, today, General Timmon Romino, third son of the late Barron Romino, of Deyalorn." She spoke for a long while of my lineage, my father's and grandfather's accomplishments, my personal services to Cortan, in discovering Firvona's treachery, saving Duke Griswold's life, nearly exchanging my own for his, and my services in Lir, forging a peace with that nations, bringing rice and sails to Marsea. It was a long and exalting speach, a hero's welcome, and more than I had expected. My surprise only deepened when she started reading off the boundaries of the barrony she would give me. It was not as large as some of the barronies in Escasaine, but lay near the city, lying almost entirely atop an aquafer. It would be rich arrable land, compared to the dry grasslands that other men ruled over. I knelt, thunderstruck and tongue tied before Duke Makala and swore my allegiance to him. The young duke solemnly pronounced me, in his boy's voice, to be  one of his men, and congradulated me on entering his service. His mother, though she had delivered a long and flattering speach, had no smile for me. His father had only scowls. This had cost her, I realized, as I retreated from the dias. She had been more generous than she appeared. She had given me this gift over her husband's wishes.

I had recovered my composure by the time the court cleared. "Well barron?"

I turned to see young Griswold standing behind me with a mischevious smile playing on his lips. "You knew about this." He nodded. "You did not see fit to tell me."

"I have learned to keep my secrets, sir." I burst into a long laughter, and the boy momentarily joined me. My duke and I stood in the empty corridor, like father and son reunited at long last, sharing our joy and mischeif. "I am sorry though, sir," the young duke said, while I dried my eyes. "Barroness Sophia looked as if she might swoon with the shock. But mother wanted to suprise you. I hope she has come to no ill."

"My wife will suffer no harm. Kindly tell me where you mother, her grace, can be found, I wish to thank her privately." 

Duke Griswold described to me the location of Nisrita's old rooms. It should not have surprised me that she had taken up residence there again after so many years. I bounded up the stairs of the west tower, two by two, as I had as a young man when my duke called. Decades had passed, it was different now, but I could not shake the feeling. As I left the winding stairs, Lyo's face flashed behind my eyes, grey and balding, with his beautiful crooked gap toothed smile and a glass of rice liquor in hand to drink to my success. How would he toast me, I wondered briefly, would it be with an admonishment to serve the gods well with my new power? 

I stopped bounding. He would not toast me. I would write him of course, but it would not be the same. I gathered my courage and entered the rooms of the duchess of Cortan.

Nisrita startled when I entered, she stood with her right hand up her left sleeve. Then recognition and relief flooded her face and she relaxed. Did she always carry blades hidden on her person, or was there something specific she feared today? "Do not come into my rooms unannouced barron. Many a page and servant have regretted the error."

Always then. What had happened to her alone in Deyalorn? "My apologies, duchess, I did not mean to frighten you. I..." I trailed off. What had I come here for? I has spoken to Nisrita once, for only a few hours over the last ten years. What did I want from her now? I kissed her hand and bowed until I found my tongue. She smelt of oranges and cloves, Cortan and Lir. "I came to thank you for your gift."

"It was not more than you deserved for your service, barron," she said flatly, without a smile. Was young Griswold's evaluation of his mother true, that she was all discipline and no joy?

"The lands, perhaps, your grace. But mot the domestic troubles they cost you."

Nisrita's eyebrows twitched and her nostrils flared, but she controlled herself. The rumours of her temper, at least, seemed to have some basis. "Let it not be said that I am not generous to my friends."

I had no answer for her. I had hoped to be met with warmth in keeping with her generosity, not this formal aloof greeting. What could I say to appease her? So much had happened to both of us in the last ten years, we seemed to meet again as strangers. I cast my eyes about her room, searching for a topic to speak on. My duchess felt the unease as well. When my eyes landed on her again, she stood licking her lips, her eyes cast towards her feet. "Is there any way I may yet be of service?"

Nisrita spoke uncertainly. "I wanted to thank you, barron, for how you raised my heir. He is a good boy, you have much to be proud of."

"It was my pleasure," I said in earnest. It truly had been.

"The position of your new lands are not an accident. Makala is the heir to Escasaine, but he has never been there. When you go to your new home, if you would take him with you, to introduce him to Duke Erfat's court. I cannot ask you to do more, of course, though..."

"You wish to send me away?" I interrupted, grief and loneliness tangling themselves inside my head, making me forget myself. I had just returned from ten years in exile from my duchess, unable to come to her aid. I could not be sent away. "Duchess, what have I done to offend you? I beg your forgiveness."

"Offend?" Nisrita asked, giving me a critical look. "Does some of the most fertile lands in the border duchy that is my protectorate indicate my displeasure?"

I opened my mouth to speak, then shut it again, knowing that I had no responce. We had both changed, this past decade had treated us very differently. I had no right to assume that she felt the same about this reunion as I did. "No, your grace. I will do what I can for the duke of Escasaine."

I bowed and left my duchess. "Timmon," she called out, just as I reached the door. Something inside me sang at the sound of my name. I turned to face her. "Welcome home."

I could have kissed her then, taken her into my arms and filled the loneliness inside me with her form, for all the relief those two words brought. But that woul ruin this fragile moment. I yearned for Lyo, not my duchess. I smiled and thanked her. "I am glad to be back."

\vspace{.5cm}

"What do you think, husband?" Sophia asked softly as I swung in a Liri hammock as had been by habit when I had last lived on this land. Pesants swung in hammocks, my wife had grumbled when I suggested stringing it up, not men who dined with kings. I promised that if I took up a position in King Gustav's court, I would forgo this small pleasure entirely. My wife let me swing in peace in the soft winter mist for nearly half an hour after that. She had built a pretty garden. It was not as large as the one in Buzen, but the grounds of this house were smaller. Fountains laughed and tittered their music around me, it smelled of home. No, of Lir. This was home until I decided where to move my family. I swung slowly and pondered my new situation. It had been an ovewhelming day, joyous, certainly, but it had left me drained. "Are you awake?" my wife persisted when I was slow to answer her.

"Yes. Think of what? There has been a lot to consider today." I uncovered the bowl she had set next to me. It contained preserved lychees, fragrant in their own water. My wife was happier to return to Marsea than I was, but I suspected that she missed Lir as well, though she would never admit it.

"Your new home, for a start." 

I smiled to put my wife's fears at rest. "It is beautiful, Sophia. It had occurred to me, on the road here, how ugly Marsean architecture was compared to that of Lir. This it the most becoming building I have seen since passing into these lands."

"Will it suit your needs?"

"My needs, Sophia?" I asked, mildly startled by the tone of the question. "Why do you ask?"

"You are a Barron of Escasaine, with lands ten days away by road. You cannot leave Cortan, given the state of the current court. This should be your residence. I will take the children to Escasaine, so you may stay here to watch over the duchess."

"Sophia," I groaned at the sensitive topic. "It has been a long and tiring day. I do not wish to argue now."

"I am not arguing, husband. It is an offer, given without conditions. I think we would all be happiest with this arrangement."

I turned to look at my wife. The afternoon's mist covered her hair, still thick and purely black, like dew on a field. Even with the care she took on her appearance, once could see the lines that the last decade had drawn on her face. She sat in a Marsean velvets, under a wool cloak lined with Liri silk. Pearls from the Sowene hung around her neck and held up her hair. Even without the lure of a barrony, she was an attractive woman. She would have no trouble living by herself in Escasaine. She would even enjoy the freedom it gave her from our complicated arrangements. "I will consider it, though I doubt we can live as you propose. We have been asked to take the Duke of Escasaine to see the lands he will rule."

"I see," she said, then "which one?"

His court or hers? That was the defining question of these lands. "We are loyal to the duchess, Sophia, have you forgotten? Duke Makala comes with us."

"That is a pity," my wife sighed, "Firi has grown fond of Lucretious."

My heart beat faster at the sound of that name. My daughter was ten, I had not seen her since she was little more than an infant. Lir had no Towers to help her master her gift. What had she grown into? "This is a dangerous game. I do not want her taking sides in it."

Sophia laughed. "You have made our house a second home for Duke Griswold, and soon, Duke Makala. Why should poor Lucretious be left out? I have been helping you appease kings and lords in Lir for the last ten years, husband. I can do the same in my homeland. All of Cortan's children have come and gone as they have pleased from this house for nearly the last two months. We are loyal to the duchess, and she loves all her children."

"Very well," I admitted. My wife was not a fool in this regard. If I could trust her not to be jealous, I could depend on her to be indispensable in keeping peace. "Tell me what I do not know."

"You should meet her first." 

I reluctantly gave up my repose and followed my wife inside. She led me to a large green room with several couches, lit in the grey afternoon by a bronze candelabra. Three girls, two still wearing white robes, sat on a rug in on the floor, playing a game that involved jacks, marbles and rhymes. The youngest, a girl in a yellow wool dress, looked up as we entered and tugged at the sleeve of the eldest. A tall thin girl with Sophia's face turned around and scrambled to her feet. "Good afternoon, father."

Duchess Elena and Commander Madriano's daughter, Cesara introduced themselves and left the room, blushing and giggling. I settled myself into the thickly carpeted room to meet my daughter. 

Firi had grown to be a frank charming girl, who showed no sign of fear or shyness at speaking to a man she did not know. I would not have guessed from her manner that she had spent her first four years in Buzen. She was a gifted barroness, I reminded myself, not a daughter of King Anko's court. It was more than that though. Firi was as confident as her mother was alluring. She told me of her studies, her journey from Deyalorn to Cortan, of her friends and teachers. She surprised me with the information that her grandfather's students had gained in influence after the duchess returned to Cortan. The observation itself was not noteworthy beyond the fact that a girl of eight had the sense to notice it. She had her mother's head on her shoulders. She knew how to gauge which way the winds blew. I gave her grandfather credit for raising her well.

"Your mother tells me that you have developed a fondness for Duke Lucretious."

Firi looked at me, confused. "Why did mother say that? I study with him, and he visits this house often, but I would not dishonor your name."

"Whose court is he in?"

Firi bit her lip nervously. "Don't ask him that, father. He hates that question. Makala teases him with that question all the time. He is Barron Paulis's page and the duke's favourite. He doesn't want to take sides. Lyca says he comes here so often because I don't allow any of the dukes to fight around me."

She was fond of him, I decided, but too young to realize it. "And how, my dear girl, do you manage that, when all the barrons of the land cannot?"

"I don't know, father." Firi considered for a moment. "I call them foolish and stupid when they fight. Makala and Lyca and I have been friends since they visited grandfather every day in Deyalorn. We used to play together in the Tower. I don't think they like looking foolish in front of me. I don't know Duke Griswold as well, but he does not want to fight if it can be helped. Emile is only seven, he will just do what I say. I've told Elena to try the same, but no one wants to listen to their sister."

I chuckled at my extraordinary girl and her ability to take care of herself in the dangerous triangle she found herself in. Did she realize that she had the young attentions of two dukes? "Be careful, child. One day soon, they will not listen to you any more."

"I know father. I don't talk to any other ungifted boys. I would not entertain the dukes either, if they were not such frequent guests in this house. House Romino belongs to her court. I don't want trouble."

I sent my daughter off to find her friends, pleased with what I had learned of her character, disturbed by how pervasive the split in the court had become, and the still precarious situation of my household. I wondered if I should take my family to the relative safety of Escasaine. Duke Griswold would not move to call me back, and Duke Erfat needed good men to keep his duchy from bleeding territory, Duke Makala needed a strong hand to raise him. There was much to be said about that proposition, and one strong reason to decide against it. I had time to make that decision still, I reminded myself. I had just arrived this day. There was a lot I did not know about the duchess.

I left the sitting room in search of Hinata. I found him upstairs arranging my belongings in a small room abbutting my bedroom. It had a door opening onto a balcony that joined this room with my chamber. It contained a sofa, a desk and several small tables displaying objects I held dear from Lir and Marsea. It was a smaller version of my office in Buzen. "You have done well here," I said, interrupting Hinata as he cleaned a lion's pelt for the floor. 

"Thank you, sama." 

I thought about correcting him, but let the Liri honour stand. Hinata gave me a cup of rice liquor and took my shoes away when I sat, but looked away when my hand brushed his cheek. Lyo had also been reluctant when we had first met, admonishing me for misunderstanding, even abusing the Liri custom of male friendships for my own base ends. That I found his beauty and his company irresistable was not an argument that he would entertain. Men fullfilled each other's spiritual and intellectual needs where a wife could not. Sophia was educated, I had no need to seek fullfillment in that area elsewhere. Spiritually, on the other hand, he eventually admitted I knew nothing and condescended to teach me. I pined like a school boy, those first several months, much to Sophia's amusement, until I had proved myself worthy of Lyo's love. It was cruel of him, after so many years, to leave me with an educated and devout servant who shrank from me. Cruel, perhaps, but not entirely unexpected. 

"Hinata, we will be leaving for Escasaine in a month or two. I have been granted lands in the new duchy."

"Congradulations, sama."

"I would like you to stay here when we leave. To manage this house, and attend to a few matters for me."

Hinata stopped working and knelt in front of me. "I have displeased you, sama?"

"Stand up. There is nothing you need beg forgiveness for." The man would perform his duty tonight, adequately and joylessly. He gave me nothing to complain of in meeting my less intimate needs. Lyo had not meant for me to use Hinata as I did. It would be a waste of a valuable gift and a source of useless frustration if I continued like this. "I want you to make immerse yourself in the Liri community. The duchess has some connections there. Make yourself indispensible to her allies before my household leaves. Learn what you can of her friends and her enemies. Keep me informed while I am away."

\vspace{.5cm}

"Is this what houses in Lir look like?" Vittorino Galderan asked, standing on the balcony of my office.

"It is an approximation. Your son lives in a smaller house, that shares a yard with four others. He has a small veranda and large windows overlooking a neat yard." I had helped Nunzio Galderan establish his abode when he came to me over two years ago with a soiled and crumpled letter from the duchess cluched in his hand. He was a nervous man, tall and gangly by build, made pale and jittery by life. I set him up with a room in a house shared by young military officers stationed in Buzen for a year or three or five. When I had last seen him, Nunzio had neither adopted the complicated rituals the Liri undergo to keep the humidity from rotting their walls and their bedding, nor fully recovered from the shock of his exile enough to hire a male servant to keep his room for him. As a result, his rooms were more squalid that I had the heart to tell his father.

Vittorino surveyed my garden and sighed. The smell of jasmine and lime drifted up towards us. It was better that the mourning father think of Lir as my wife's well kept garden, even on this cold grey day, than the narrow veranda littered with goat pellets and chicken droppings his son called home. "I cannot thank you enough for the news your have brought me of him. You cannot imagine what it has been like not knowing how he fares."

He could not be more wrong. I had spent my life imagining every nuance of what the old general felt, and so much more. If lying to him would let his kind heart rest easy, then the words would come easily, whatever Lyo's morality might dictate about white lies and allies. "Letters are small, and words are light. It was no trouble." Nunzio Galderan had served as a marshall of Cortan's barracks. He was an educated, unambitious man well suited a life of records and letters. He came to me speaking no Lir, willing to serve as a pot boy, if that was the position I had open for him. Lyo found him a tutor, and the man slowly and painfully learned the new language. My replacement agreed to keep him on the ambassadorial staff as a translator and transcriber. When I left, Nunzio gave me a stack of letters thick enough to be a book. In his loneliness, he had written his father regularly, but never found the send his messages via the Marsean messengers. I had wondered then, what a man, who restricted himself to as limited circles as the general's son had, could write so many pages about. But the Liri have a strange love of records, laws and beauracracy. Nunzio could do well in the country if he adjusted to the new habits and applied himself. He certainly could to worse elsewhere. "Nunzio is a lucky man, to have a father to write to after his shame."

Vittorino looked pointedly away from me. I found it hard to watch him. There were two ways my father could have reacted if my shame had been revealed in his lifetime. The general's pained face represented the better of the two options. For all my dreams of losing my secret and keeping my father's love, I could never wish that face upon my father. Perhaps it was better that he died in ignorance. "Come inside. The clouds grow dark and the fire is warm."

"Thank you." The old warrior shook off his melancholy. "What is it you wished to speak about, Timmon?" He sat by the fire in a great wooden chair with arms carved to look like lion's paws, exchanged his boots for a pair of wooden sandals and accepted a glass of rice liquor. The rumours that the retired general had developed a taste for Liri habits and delicacies since his son's exile seemed true. I had sent forward with my wife several barrels of the rice liquor brewed by the same family that stocked King Anko's cellar, and his father's and grandfather's before him. No one else in Buzen matched their work in subtlety of flavor or color, to say nothing of the concoctions of the scattered Liri communities in Marsea. I would have had my father drink nothing but this. In honor of his love for Nunzio, I would serve nothing less to Vittorino.

"To thank you for the support you have shown for myself and my family. We may not have returned to Cortan without you." Duke Griswold played games with Cortan's military. Not only did he recruit the strongest and best members of Escasaine's military to Cortan's, but he fractured his own. Revealing my change in gods was one of many small events that finally spilt Cortan's army into those allied with the duchess as the leader of Cortan, and those favouring the Duke as the man who held the scepter. It was a dangerous court to return home to. It would not have been possible but for the general's stubborn campaign, going from door to door of officers who knew and served with me, reminding them of my service.

Vittorino's lips twitched uncomfortably. "There was little else I could do under the circumstances." He lifted his glass to view the world through the clear butter colored liquid, buying himself time to collect his thoughts. He had come here to speak against his duke. I waited. "You have noticed, no doubt, who has marched west to join Allepo's campaign?" he finally began.

I had. Two weeks ago, Cortan had sent less than half its standing army, all of her men and a few of his, to fulfill its duty to the crown. Allepo was displeased with Cortan for this lackluster show of support. Duke Griswold reasoned that he needed his men to help with the uprisings in Escasaine, though Vittorino saw little movement of men heading south. None of the men I had spoken two in the four weeks I had lived in Cortan knew why the Duke gathered so many men near him, and kept them there. If he had intended to move against his wife, or Lir, he would have done so by now. Yet he did not show his aggression. He simply kept a large army at his gates, as if he still lived in a wooden tower on the border of a newly established territory, beseiged on all sides. If he had looked his age, I would have been inclined to think that he had grown senile; he had walked the earth for seventy five years. Did he know no other way to rule but as he had in the years of Cortan's founding before his disappearance? Can a man with the strong body of a fighter have the feeble mind of the aged? There were some in both armies who whispered as much. I kept the possibility in mind, but I could not allow myself to believe that as an explanation. Duke Griswold had a reputation for his cunning, it could be deadly to underestimate him.

General Galderan gave me the names of men in the Duke's army, who went over out of a sense of loyalty to the scepter, who for a price. He told me all I wished to know about the men who had come to Cortan from Escasaine, and who would best be able to inform me of matters in Duke Erfat's court. I learned from him the details of General Sidro's marriage to the Paulis family, and of how my lands had been promised to him. I had also learned that Barron Paulis had bought or promoted a number of men from the Duke's court, and that he and the duchess did not see eye to eye on these promotions. That did not surprise me. The barron played a subtle and dangerous game, while Nisrita, by nature, was not subtle. She was short sighted, preferring immediate securtity from her husband over gambles that may eventually grant her power. The Barron, on the other hand, seemed to collect people who would not unwaveringly choose one court or another. He danced to his own tune, one that I was not aware of. He would not be satisfied with running a secondary court for long. There would be treason in Cortan soon. I wondered if Nisrita knew about it, and how much danger she was in if she did not.

I probed the general on the issue of the duchess, but he could tell me little more than anyone else could. I had learned from rumours that Commander Madriano's wife had become Nisrita's personal healer around the same time that Nisrita became a Master herself. Around this time, the duchess had fallen out with Barroness Paulis, and miscarried badly. Discrete prying by my debonair wife taught me that Duke Griswold also stopped visiting his wife's bedroom at this time. Something had happened when Nisrita became a Master. Sophia doubted that the new title alone could convince the duke to curb his hunger for an ever growing brood of gifted children. I agreed with her judgement. Nisrita, for all my pleading, would not speak of it. In the two weeks between my arrival and the duchess's departure with the Cortan's men for Allepo, she would speak little of herself, unless it was to bemoan the rivalries between her children and her dislike of how Barron Paulis controlled her court. I could not even get her to admit to her pride at wearing a Master's collar. 

Master Carlotta, greeted me with an icy formality that had long become our habit, but she would say little to my wife regarding the duchess in her care. I recieved a flattering and congradulatory letter from her husband, but I could not question him until after the campaign. I doubted that he would know much about his wife's doings with the duchess.

 Barroness Paulis, the remaining woman close to her grace, would not even see me. This ceased to surprise me when I heard that many feared my arrival would wrench power from Barron Paulis's hands to place it firmly in Barron Romino's. The idea had passed through my mind, but Barron Paulis had nothing to fear. I would not act in that direction without Nisrita's expressed approval. At the moment, all the duchess seemed to wish was to have me as far away from her as possible. That desire stung me more than was good for my family. Out of love for my children, and respect for my duchess, I would obey, once Duke Makala returned from serving as Commander Dielo's page. It was possible, I supposed, that Lucretious's mother strove to hide information from me as did Nisrita and Master Carlotta, but from what I had heard of the woman's intellect and the state of her friendships with Nisrita, this seemed unlikely. If my current task was to learn about the duchess, I would do well to look elsewhere. 

In my desperation, I even sought an interview with Head Alerio. My efforts earned me a superb cup of dry white wine served with a dish of salty fried fish served with peppers that brought me back to King Anko's court, but it gained me no information. The Head's hospitality was impeccable. He spoke highly of Nisrita's abilities as the head of his school, and as a powerful symbol to lead healers out of a quiet domestic life to serve the army for two months every year. He gave me a detailed description of her current mannerisms, her peculiarities as a teacher and mother, and her recovery from the frightened woman who had come to Cortan to be coronated. It made me oddly jealous. I had been replaced in Nisrita's heart by this crafty politician. I had neither written nor spoken to her in six years, let alone be of service to her. It was only right, I told myself, that she form a friendship and allegance with one of her own kind. I could do nothing to help her with matters of the mushrooms, or what she believed to be her husband's true ambitions. Yet the Head's intimate knowledge of my duchess's behaviour irritated me more than a bed of nettles. Furthermore, I could not get from him anything more than the vaguest explanations for why King Gustav allowed Nisrita to work with the black powder, when a strict ban, enforced by inspections by the military exsisted for everyone else. I could learn nothing about her personal studies, her duties in the infirmary, or the state of her children's education. When it came to matters of substance, Head Alerio was as mysterious as the moon. 

Pehaps, my wife had finally suggested, when my fretting at the mysteries behind Nisrita's new behaviour threatened to fray the strands that patched our queer union, the duchess had taken to the mushrooms as a matter of self defence against her husband. It was possible, I admitted, but she had her talisman, I knew the damage she could do to a man with that crystal. If she had been too frightened to use that against her husband in Deyalorn, it did not make sense for her to turn to the black powder in Cortan. Sophia had laughed and accused me of being too good of a man to understand. Oddly enought, her laughter gave me pause. She was a valuable ally, my wife, and her waters ran deeper than she would admit. When a husband's behaviour towards his wife turn from demanding to vulgar, what might a woman not do to defend herself? I had no reason to know, but Sophia and my step son certainly did. The thought of Duke Griswold raising a hand against my duchess filled me with a rage that, a decade ago, would certainly have ended both our lives. But the last ten years had tempered me with three children and the sagacious company Lyo. My life was not solely my own to give to a cause of my chosing. I could not orphan my children for the love of my duchess. Even so, I spent the next two days hatching dark plots of revenge against the duke I had come to think of again as my rival, and my nights prostrating myself before the pool in the temple of Sento, begging the god of the open seas for means of controlling the storm within me.

When I finally bid the General good day, I sat in my office contemplating my position. Duke Griswold and Makala had left with their mother for Allepo. They would be away for two months. When they returned, my family would leave with the young duke Makala for our lands in Escasaine. That left me with little to do for the next two months but try to learn what I could of the courts of Duke Griswold and Duke Erfat. I had a growing list of names of men who could help me with the latter, but it seemed there was little I could do penetrate the depths of secrecy of the former. 

My man Hinata performed his task diligently. He had become the tutor of the sons of a man named Daigo no Sokune, one of the first engineers to migrate to Cortan to help establish a proper irrigation system in these dry lands. His efforts in Cortan had not worked as well as they had in Allepo, yet he enjoyed a closeness with the towers of both duchies. Hinata had learned that both the duke and the duchess had visited Allepo's tower over the last two years, separately, and had met with different people. He could learn little of the purposes of either visit. The duke schemed and the duchess hid. To what ends, I could not tell. That ignorance would drive me mad. 

Hinata had futher troubling news. The Duke had attempted last year to move the Liri quarter beyond the city's walls. He failed due to the duchess's support. In the duchess's absence, the Duke intended to try again to harrass the citizens. It troubled me that her absense mattered. Barron Paulis remained in the city. The duchess would not stand silently by to watch her Liri citizens abused, why did her regent? I No one had taught my duchess to rule, she could not even control her allies while away. For all her good intensions, she alone could not offer her protection to anyone in her court. The question for Cortan should not be his court or hers, but Griswold's or Paulis'.

A child's cry and the sound of a gong interupted my thoughts. I went to the balcony to overlook my now crowded garden. This was not a time of year I was accustomed to being in Cortan. With most of the Tower marching with the army, the children of the school went home. Sophia filled my garden with almost every child in the duchess's court, it seemed. Women talk as well as men do, my wife reminded me, and they speak more freely when they do not have to worry about their children. My afternoons were normally spent at the barracks, while my house filled with the company my wife chose to keep. Today's unusually long visit with the general had left me at home while the cacophony filled my house, driving out all possibility of rational thought.

Today, Commander Madriano's Cesara and Panfilo joined Elena, Emile and Mirella, of Cortan, as well as Ladislao and Charmina of Paulis joined my three in a grand pagent around the hibiscus bushes. Firi, the eldest of the gaggle, had made herself their queen. No, there was another child, lurking in the branches of the lime tree, watching the game from above. If I still wagered, I would put my money on the proposition that his eyes followed my daughter. Duke Lucretious had placed himself by my daughter's side like a shadow on a sunny day. The more I learned of Cortan's court, the less I liked their friendship. 

My daughter called her court, and my thoughts to attend her in a loud clear voice. She maneuvered and positioned her followers with the ease and assurance of a captain dispersing the squadrons under him. The scene pleased me in spite of the noise. My daughter was born to rule. There was no question that the blood of queens ran through her veins. If she had been born a boy, I might have gladly have given her control of my new found lands upon her maturity and thought nothing of it, knowing them to be well guarded. As is, she would grow to be as beautiful as her mother, and as charismatic as the gifted officer she must have been sired by. Given her character, I could not imagine any other option. Duke Lucretious watched her from the branches of my lime tree. In a different world, I might see this as an opportunity to make my daughter a duchess. In this world, I would do better to marry her to a well positioned Liri family. 

\vspace{.5cm}

I entered my house on a warm spring afternoon with a pain in my side and the taste of blood and fear in my mouth. I sent the stable boy to ask my wife to rid our house of our regular afternoon guests and retired to my room to change my torn and muddy clothing. I did not have the luxury of Hinata's assistance. Without him, the painful task of cleaning myself might be the easiest challenge ahead of me in the next few hours. The cachophony in the garden dimmed as the young musicians put away their flutes and drums to answer their mothers' calls. I was almost sad to heard the garden fall silent. Normally chasing me from my house, today the sound came sweet to my ears, far more pleasant than the din I had just escaped from. Young barrons and barronesses, sons and daughters of commanders and priests filed out of my house as I winced and donned a Liri tunic with a wide neck. I did not have the strength in my right hand to fiddle with laces. My rib was probably broken, my wrist certainly sprained. I counted myself fortunate not to be bleeding. 

I stepped into my office, lifted the parcel I had recieved this morning from my desk into a cupboard. It was starting to smell, but it had not served its full purpose yet. I checked the room for anything else that might unduly unsettle Sophia, then eased myself into a chair to hide my wounds and support my aching back and waited impatiently for my wife. She appeared, unhurried, at the threshold of the room, wearing an air of concern and mild annoyance at my imposition. "What was so urgent, husband, that I had to turn Father Anglius's daughter from my house? It took nearly three months to convince her to grace this house."

"My apologies for spoiling your efforts so abruptly, Sophia. How soon can you leave for Escasaine?"

My wife inspected me sharply. I could see the accusation form behind her eyes. What had I done today to put her children at risk? As if the mere fact that my blood did not run through their veins made me callous towards their fate. Her words emerged angry and terse. "You have been fighting again."

"I will send a messenger ahead to announce your arrival. Can you be ready to leave in three days?" If Hinata were here, I could trust him to have my family safely out of Cortan's gates by the next afternoon. Without his help, I would have to slog through a long domestic dispute. I owed my wife more of an explanation than I was ready to give. The events of the morning and my wounds made me irritable. They put me off treating Sophia as my useful gifted companion. I wished she had adopted more Liri habits than she had. I needed her to listen, and I needed her to leave. My side throbbed at the throught of a long discussion.

"You promised." Sophia hissed. "Before I left Lir, you promised that you would do nothing in Cortan to endanger our life here."

I felt my anger rise. She had no idea what I had been through or why. "And I have kept that promise, barroness. It takes more that a morning of even my most diligent efforts to ruin the prospects of a richly landed house of Escasaine. You will take no more than three servants, the better to travel quickly. If I know the duchess, you will find our new house already appointed to a tolerable level. I will send everything else you need from here along in three weeks as we had originally planned."

"You forget yourself, husband. You have spent too long in that foreign land. I am not..." Sophia stopped suddenly to look at me anew, at my still shod feet and the mud stains they left on the carpets, the glass of lemon water missing from my hand. She took a step into the room. "Where is Hinata?"

"At the Tower." He would likely be there for over a week, but Sophia did not need to know that.

Understanding slowly dawned on Sophia's face. "You were in the Liri quarter this morning." I nodded. I had been in the barracks when Hinata brought news of the riot. Most of the dozen men I brought with me took a wound during the quarter hour until the city guard finally came to establish the peace. Dozens of Liri were arrested and the rest denied the Tower's services for instigating the riots that Hinata insisted they had not beenthe cause of. I had hoped I would never see the day when I drew my sword to defend foreigners against my own. I was a member of Marsea's military, not a city watch. My place was to instill fear in foreign invadors, not defend legitimate citizens. I would face retribution from the Duke for breaking his peace. Ideally, I would have protected my children from seeing their father humiliated, but I could not get my family away by tonight. 

Sophia tisked as she poured the glass of boiled, cooled water and cut a lime. "You cannot leave well enough alone, Timmon. It does not matter whether you swear by my gods or yours, you cannot keep from seeking trouble." I opened my mouth to protest when a glass of lime water appeared in my hand and I felt the warmth of Sophia's talisman searching my body for injuries. When she spoke, her words sounded resigned. "You acted nobly, I am sure. I would rather be chased to Escasaine by rumours of your heroics than of your disgrace. I would rather still not have to flee." I kept my mouth shut. She would let me win the day without revealing the contents of the cupboard. She finished inspecting my wounds. "How much danger are my children in?"

More than I would like to admit. "The sooner you could leave, the better I will feel."

"When will you join us?"
 
"With Duke Makala, as arranged."

My wife's face twisted in distaste, bringing the wrinkles at the corners of her mouth into focus. "You must serve your duchess. Very well. I will see what I can do." She walked to the door, then paused to call back "Do not endanger us further by walking to the Tower. I will have someone to come to you."

The door closed and I found myself alone for the first time since before dawn. I had called Hinata to my room when I found the bloody buffalo head beside my bed. The Liri sacrifice buffalo to Zinmenzu of the fertile banks. He was the only god that requires a sacrifice, and nothing but the greatest of the Liri beasts will do. The Liri revere their buffalo as the Marseans love our horses. This was an insult to myself and my gods. It was also a warning. Had my benefactor wanted me dead, a small man could have slipped through the space it took to insert the head into my chamber. I could not tell Sophia any of this. I depended on her as much as I did the armsmen who would travel with her to guide my children to safety. She could not do that if she were frightened.

I spent the morning speaking to other Marseans I knew to be supporters of the Liri community: Head Alerio, Vittorino, the head of the masons' guild, old Barron Quiro. All but the Head had been warned or threatened two weeks ago to chose Marsea over Lir when the question came. Two weeks ago, the duke raised a levy on fish, barely a hardship for those with traditional Marsean tastes, but a steep increase in the price of food for all Liri. No one protested. Vittorino, one only had to mention his son to scare; Gallo, the stone mason who enjoyed his women feared his wife may learn of his habits; Barron Quiro, the head of a noble but no longer wealthy house had spent the last fifteen years in the Duke's service, in exchange for the treasurer overlooking his debts to the treasure. He feared that the new Duke may no longer require his services. Those threats were consistent with the Duke's habits. They may be men of the Paulis court, but Griswold owned them. 

The buffalo head, I decided, did not come from the duke. If he had wanted me gone, he would have sent an assassin, not a butcher's boy. There was one other man would wanted me dead in this court. I would have to find a way to thank him for his present.

A boy appeared shortly to help me to bed to await healing. I lay there, listening for the sounds of the Duke's men coming to arrest me. I would spare Sophia this humiliation if I could. I was personally not in any real danger, I would spend a few days in the cells under the castle, but I doubted the duke had the support to do much more to me than that. Still, Sophia's accusations were not all wrong. My family would be humiliated  by my actions, by my inability to control my anger. The weak needed defending, Sophia would chide, but so did my children. It would be better for my family and myself if I left for Escasaine with them. 

Vittorino had shown a great deal of sense this morning. While I had run towards the rioters, he had sent messengers to summon guards. He had avoided a conflict with the Duke, while I had walked into his trap. I may still be able to avoid it by leaving immediately. But that I could not do. I could not answer the bull's head with flight. I had pulled three Marsean boys off of Hinata's still and bloodied form before the city's guard arrived. I could not leave Cortan without knowing who had pushed the boys to attack the Liri community and injure my man.

The guards' delay bothered me. The Liri quarter lay closer to the castle than to the barracks. It should not have taken the city watch the time it did to quell the rioting, unless they had been specifically instructed to delay their responce. As I replayed the morning's events in my mind, it seemed that they possibly did not move to intervene until after they recieved Vittorino's message. There lay the starting point for my queries. If I could learn who had given the order for the delay, I might learn which court had started this violence. From there, I could avenge Hinata, or warn the Liri community, or ...

Firi entered timidly, disrupting my thoughts. "Mother says that you have fractured a rib and broken your wrist. Lyca would like to see to your wounds."

It did not surprise me that the young duke had found a way to stay by my daughter's side after I had sent away my guests. His flaunting his disobedience, however, annoyed me. "A sickbed is no place for children."

"Father," Firi hushed me, "he is here. He will hear you."

I looked beyond my daughter to see a shadow across my doorway. Duke Lucretious spent more time in my home than any of Cortan's children, but he seemed always to be lurking, always near the children, but never playing with them. I do not think he was liked by the others, small as he was, with a frail, almost birdlike in stature. By the mercy of his gods, he wore his father's face, not his mother's. They had also seen fit to grant him the gift, so he did not need to learn to weild a sword. Lyca's eyes had a peculiar quality, he walked with a constant squint, as if he gave everything he looked upon his full attention and concentration. He never spoke to anyone but my daughter unless adressed. He kept his answers polite but brief, offering no more information than was necessary. I could not tell if he were timid, just short of fearful, or aloof, just short of disdainful. The distinction would make all the difference. The young duke was his father's favourite for the Duke of Escasaine, his mother spoke highly of his qualities as her student but dispared for him as a potential leader, and he was Barron Paulis's page, though rumoured not to be good for much beyond writing letters and pouring wine. He was either an unimposing child, either used by both courts to their own ends, or the only one of Cortan's heirs to not be allied with one parent or the other. He was young yet, the answer would not matter for a few years. As a man, this mysterious young duke could prove to be dangerous.

"Your mother should have sent for a fully trained healer, child," I said more gently for the duke's benefit.

"She did, father. Men have just arrived from the castle. They said no one may come or go. Mother is seeing to their needs."

That was how it was to be done, then. We were all under arrest for the moment. The public shaming of the inquiry to follow would guarantee my wife's self imposed exile, if not my own.

"Lyca says that Barron Paulis will be here soon," my daughter continued. "He thinks you should have at least one session of healing before you meet him."

Paulis? That surprised me. How was the boy so certain of the upcoming visit? The young duke knew far more than he let on to anyone but my daughter. He was another mystery to contend with. "Very well, let him heal me."

The young duke entered, Sophia's talisman in hand. He was too young to have one of his own. The process would have put me more at ease if there had been a healer standing by. Lyca was quiet, but at ease with himself and the situation. His silence seemed to stem from a mere lack of desire to speak, not from a sullenness or shyness found in boys his age. He checked my wounds manually, painfully and silently, not with the hot wave of fire that healers normally use. The rib was easy. My swollen wrist, he ground and twisted and moved until I thought of telling him to leave it be. I would bind it and have it healed later.

Firi muttered something to the duke, and I felt the warm probing fingers of the healer's gift in my hand. "I'm sorry, father," Firi apologized as if speaking for her friend. "The duchess has taught us to heal and locate all the bones and organs in the body, but not to detect what is wrong with them. I promise you, Lyca is very good at healing broken bones. He spent the last two years working with the duchess on various skeletons." 

I smiled at the memory of the duchess's first campaign, when I helped her dig through the rubble in Bayam in the hopes of finding wounded but living enemy soldiers for her to study. "I trust the duchess's training." I said to my attendant. Firi smiled, but Lyca did not. He remained as silent and emotionless as ever. It was as if he had cast a spell on my daughter and spoke only through her.

The hot probing finger disappeared, momentarily replaced by the familiar hot searing pain of healing. Firi sat down on my bed and took my other hand. "Mother says that we are leaving for Escasaine the next morning but one."

Sophia was effiecient, I had to give her that. "You will leave as soon as the duke's men will let you go."

"Father," she began, hesitantly, "I would like to not leave Cortan."

With the boy who had cast a spell on her on my other side, I could see why she would not wish to. "That is out of the question. You go with your mother."

"You are sending us away because you fear for our safety. I can think of no place in Marsea safer for me than in Cortan's Tower. I am certain Head Alerio would allow me to reside in the Tower, if he knew the situation. Not even the king himself could protect me ..."

"You are going with your mother." I snapped. The girl wanted an excuse to stay with the duke. I could not permit that.

Firi quieted. We sat in silence until prespiration beaded on my forehead from the pain. I was getting old. I had recently noticed it becoming harder to sustain a healers efferts. The searing gift stopped flowing through my wrist. I turned to the silent duke. "You may continue, your grace. I can take more than that."

My daughter looked up sharply to corroborate, "His pulse is strong, Lyca." 

I smirked at the deceit I had fallen for. My daughter did not hold my hand solely out of filial affection. She felt for my pulse. The two children coordinated with each other as easily as my right hand did with my left. It was uncanny how they seemed to communicate without speach. The young duke ignored us both, and started wrapping my wrist. Was he preserving his strength, I wondered, or did he not have much power? I had never been healed by one so young. It was impossible for me to know what he could or could not due. His impassive face gave no clues.

"Father," my daughter continued before Lyca started on my ribs. "One more word, I beg you, then I will stand by your decision." I nodded my permission to speak. "Master Adele heads the women's Tower in Escasaine. Head Corino lives in the town. The healers of Cortan's Tower that Head Alerio replaced when he took over mostly have positions in Allepo and Escasaine. My success as a student will be greater under the supportive care of Head Alerio and the duchess. I do not know what danger you fear for me. If there is any chance of retribution within the Tower against my grandfather or yourself for supporting the duchess against the duke, I would be safer under Head Alerio. If you fear for me outside the Tower I will promise not to leave its grounds."

My daughter's keen awareness of her enemies and friends impressed me. She had thought this through carefully, probably with Lyca's help, but that mattered less at the moment. I did not like the position I found my family in. Sophia would not be pleased to be separated from her daughter again so soon. "I will consider it. Go get your belongings together." 

"Father, your chest..." Firi protested.

"It is a broken bone. His grace will not kill me while healing it. Off with you, and keep out of the men's way."

I wanted some time alone with my young healer. I had not had such and opportunity in my two months in Cortan. I watched my daughter obediently leave the room, then turned to my attention to the boy. "Why do you think Barron Paulis will visit me?"

The heat stopped searing my chest. "I am his page. I know the men loyal to him by sight. My father has not yet sent his men to arrest you." The boy delivered his words with certainty, but inspired no confidence by his flat tone. "I beg you not to interrupt again. I am not as practised as full healers, and cannot speak and heal at the same time." I waited, disappointed, for my next opportunity to interview the Duke. The Barron arrived as the boy finished with me, depriving me of my opportunity. I cursed the luck Rishiki had chosen to grant me today. Not only was this boy still a mystery, I now had to face the man responcible for the bloody head in my office. I had no plan to win this confrontation.

My office, when I entered it, smelled ripe with rotting meat. It would have to do for an advantage, I thought, and awaited my guest and warden.  Due to the duke's efforts, I could move with a tolerable amount of pain. Experience told me I would not be able to lift heavy objects or move quickly if I needed to. This would be a civilized meeting. I hoped it would not come to that. The garden outside was empty and quiet. I could hear my wife's tense voice tending to the barron's men, and the hum of their conversation, but nothing more.

"Good after noon, Barron Italo." I said brightly as he entered, a round faced man with beady eyes, a thin long neck and stooped round shoulders. I watched his snubbed nose crinkle at the smell of the gift he had left me with. He was younger than me by some years, a man who had not come to the end of his ambitions. I handed him a cup of wine, and placed a bowl of fruit before him, which he eyed uneasily in the rancid atmosphere. "You have my undivided attention, your honour. What would you wish of me today?"

Italo Paulis downed a mouthful of wine, and sniffed the air with an expression of disgust. "I would have thought this morning's gift would have made that clear. A confrontation with the duke would be unnecessarily inconvenient for both of us. I have come to escort you from Cortan before the duke comes to arrest you."

I had guessed correctly, then.Barron Paulis did not play a game of hide and seek. We would come directly to the point. "The buffalo? It was a fine beast. Zunmenzu would certainly have granted you any request you had of him. But I am not he. Why should I leave?"

The barron put his sleeve to his nose to guard against the smell in the room. I had seen many men, Liri scholars, mostly, react so delicately to the scent of death. Barron Paulis was not a fighting man. Wounded as I was, I would almost rather have faced a man of arms today than a man of wits. Today's events had not left me with the time to ready myself for this opponent. "Why? I should have thought that much obvious. My men fill the lower floor of your house."

"Bring all the men you want. You will not hurt me, or you will have to answer to the duchess."

"The duchess?" Barron Paulis shrugged his round shoulders. "There are no witnesses to what happens here today. The duchess will believe you, but would you really wish her to spend her very limited resources protecting you while the Liri community needs her protection? The duchess needs my support, you see. I intend to keep it that way."

"You underestimate..." I began my weak defense when I heard a ball thud onto my balcony.

Barron Paulis looked sharply to the window, while I opened the adjoining door. Slow, to both of our amazement, Lyca's small face and round eyes appeared outside the trellis.

"Your grace! What are you doing here?" came Barron Paulis's voice behind me.

I grabbed the boy roughly by the shoulder and helped him stand, wincing as I did so. "Lyca! I told you to stay out from underfoot."

The boy stared at me mutely with blank round black eyes. He pulled his thin arm from my grip, slouched his shoulders and spoke to both of us, looking at neither. "My apologies, your honours. I lost my ball. I did not mean to intrude."

"This is most disturbing behaviour, Lucretious," Barron Paulis reprimanded. "Your mother will hear of this." 

The boy flinched at the mention of his mother's name. "I am sorry, barron. It will not happen again. I did not know ..."

His mentor studied him long and hard. The boy studied his feet intently under the gaze. "Out with you. We are returning to the castle. Now." 

The barron barely controlled his wrath at the eleven year old page, but there was an unease in him as well. As he left the room after the duke, our conversation degraded to threats and insults. "You had me fooled, barron. I almost believed you to be a man of courage. But I see that you hide behind women and children. Do not return to Cortan, or this conversation will have to pick up where it left off."

I went to the window of my bedchamber and watched Barron Paulis escort the young duke and his men from my house. It had been an educational afternoon. The duke had not thrown his ball onto my balcony by accident. No one had been playing in the garden when Barron Paulis had entered my garden. Where had the duke been? Perched on the iron supports of my balcony? It would be a challenging climb for the boy who watched the children play from my lime tree, but not impossible. Then what had he heard? I would have to assume everything. He had positioned himself there to spy. He had planned for this, shepherding his strength while healing so he could make the climb. Who did he spy for? His attentions towards my daughter, his warning me of my visitor, the care he took towards healing me, the timing of his interruption all seemed to indicate that he acted to protect me. Yet that may purely be flattery. He clearly acted against Barron Paulis's interests, he may have only acted for himself. My guests words rang in my ears in the the empty room. I had hid behind women and children today. My duchess, I had hid behind, shamefully, but knowing that she would not mind. I had not intended to hide behind her son. His appearance had been embarrassing. 

I shook the shame from my head and sat down to write Head Alerio about my daughter's education. I did not know if the duke's men would follow the barron's. It would be inconvenient for Barron Paulis if they did. I could not count on that for my safety. The movement of the Duke's men would tell me much about the barron's motives. As for the boy, he seemed to cultivate an air of shy incompetence to everyone, his mother, his father, his mentor, but kept his eyes and ears open and his own counsel. He was an impossible boy, and would grow into a dangerous man to make an enemy of.

\vspace{.5cm}
***

I was proud of Makala. Of young Griswold, those words went without saying, but of Makala, I was supremely proud. Commander Dielo spoke well of him as a page. He had not complained of the hardship of the road, nor the rigors of the campaign. Compared to the sulking crying boy I had brought with me on last year's campaign, I felt that this year, a man rode with me. He would never be a warrior his namesake was, but he would be a warrior, and not one Marsea would be ashamed of. 

Cortan's towers and its White Tower appeared on the horizon, black against the dawn's brilliant oranges and blues. My children, my students and patients, Alerio all called to me from there. So did my husband, my court, my ongoing tensions with Barron Paulis. I had recieved news of the Liri riots and the taxes on fish, a product that few Marseans eat in these arid land, but the Liri seem to thrive on. The Liri priest, the sephiyat, Nagayi had written, asking for support while I was still on the western front. As if I could have done anything to help him from there. I would have to speak to the Barron about this. It would be one of the many battles we would have ahead of us. If only I had the courage to send him back to Deyalorn. But then I would have to weather my husband alone for the months it would take to find a replacement. No, there were times in this campaign that I felt that I had a better life on the road that I did at home. What would I not give to live as the kings of old, conquoring new Marsean territories, always on the road, always far from one's tower or holdfast, always living with the respece accorded me as a healer. But that was pure fantasy. My life and duty lay before me. I would escape again next spring.

\vpsace{.5cm}

"Why are you still here, Timmon?" I asked, irritably. For the four hours that I had been in Cortan, every man I interviewed told me of Timmon. Master Alerio complained  of his constant questions, Sephiyat Nagayi praised him for his courage and bravery during the Liri riots, General Galderan did the same, going on at great length about how he, not a member of the duke's inner circle had known about the new fish tax before anyone else. Barron Paulis went to great trouble to impress upon me how inconvenient and difficult it had been to keep the duke's men from arresting Timmon. I had only been away for two and a half months. How could one man become so important in such a short time. One would think that he did not realize that both Barron Paulis and Duke Griswold wanted him dead. Timmon had to go. Let the Destroyer take my soul out of the dance, I did not want him to leave. 

"I do not understand, your grace. Did you not just call me to see you?" the offending barron said with some confusion.

I sighed and offered him a seat. It was dark, I had been marching since dawn, the mass inside my womb often sapped me of my stamina. I was tired and irritable. I tried again, with more warmth. "When I heard your family had left for the new Barrony of Romino, I thought you would have gone with them. There was no need for you to wait alone for Makala's return. The boy is old enough to travel by himself."

Timmon's jaw clenched under his beard. "I had other matters to attend to."

Other matters, the twich of his jaw, someone had threatened Timmon's loved ones. Even after six years and a host of strange new Liri habits, there were some mannerisms of this man I could still read. "Is your family well?"

"Yes." He laughed. "Sophia wrote complaining of how sparcely you have appointed her new residence. She eagerly awaits my arrival with a few luxuries."

"My ..."

"Do not worry about it." He waived away my incomplete apology. "My wife will spend three years' incomes on furnishing the house, at the end of which, she will decide she needs new rooms added on. It will become a small palace and the talk of the duchy."

"Timmon." I laughed. "You are too hard on her."

"Not at all. I value her sense of taste and ability to entertain immensely. But they are not my tastes, and a man must find something to complain of in a marriage." He paused to let me control my giggling. They were happy together again, in spite of Timmon's impulsive nature, he and Sophia had resolved their differences. It was what I had wanted for him. "Firi is here," he continued. "She prefers Cortan's Tower to Escasaine's. You will watch over her as you can?"

"Of course." the prospect delighted me. "It will be my pleasue. Elena will be thrilled." I looked at the man before me who I did not wish to send away. There was no one else in Cortan, barring my daughters and little Simone, who could make me laugh as he could. Life would be bleak again without his company. "Makala will be ready to leave as early as tomorrow morning. He awaits your pleasure." 

"Tomorrow morning then. I will speak to his grace." Timmon accepted his duty as a disciplined soldier accepts an undesirable post. "May I have the pleasure of one last game of chess before I take my leave?"

My back ached, my feet rebelled in the tight shoes I wore for court, my bodice pinched, my head felt heavy and unbalanced in its coif, the need for sleep dulled my mind. Since I had arrived home, I had done nothing but grant interviews to various members to my court in this dreadful attire. My bed called to me, and my body yearned with all its attention to be done with its time in this stiff elaborate dress and give in. "I am sorry, but fear I am too exhausted to prove much of an opponent to you tonight." He looked disappointed by my answer. I was not surprised. I was disappointed by my answer. "It is not that I wish you to leave Cortan, Timmon. Some say it was a foolish thing for me to do, granting you a barrony. I have made you many enemies in the process. You will be safer in your barrony."

"It was a foolish gift. But I am grateful. I would rather advise the duke of a border duchy than serving the crown in Deyalorn." Timmon rose to take his leave. "Your husband holds a secret over most of the men loyal to you in particular, not just to Barron Paulis. Be careful how you rule."

This news did not surprise me, though I had no evidence of my husband's weakening of my allies other than General Galderan. I gave him my hand to bid him goodbye. "Thank you. I will walk carefully." The Barron took my hand but he did not raise it to his lips. He brought his face towards mine and planted a kiss on my forehead. I closed my eyes in surprise. The tight curls of his beard brushed against the bridge of my nose. He smelt strongly of sandlewood and lavender, as if he had applied the scents recently. I became keenly aware of the warmth of his beard, the smallness of my hand inside his, how soft and uncalloused his years as a diplomat had made his fingers. Then the moment ended. 

The stubborn man spoke, "Until we meet again, duchess," and quickly left the room. What had he done? Why could he not simply stay away? The impression of his lips stained my forehead, his scent lingered in my nose. Why could I not insist that he never return?

I walked as if in a dream to my door and called a maid to undress me. I had been a fool to grant Timmon a barrony. If he had gone to Deyalorn as a general to serve the crown, he would have fewer people wanting his head. I would not have reason to fear for him the next time he visited. But I had no guaruntee that he would go to Deyalorn. What would I have done if he had stayed in Cortan, asking his questions, trying to learn the truth behind what my husband had achieved, and the danger it posed to my health. Carlotta was right, as much as I hated to admit it. The barron could ruin everything. I had to send him away, for his good and mine.

I yelled at my maid for digging a hairpin into my scalp. I knew what I had to do, and why I had to do it. I hated the barren prospect it had left my future days. Timmon had taken Makala and left me Firi. She was a charming girl, with a good head on her shoulders and not the slightest sign of squeamishness. I could take her under my wing in the infirmary in a year or two. In the mean while, Elena, who had never let anyone else tell her what games to play, would be delighted to let the young barroness rule her afternoons.

\vpsace{.5cm}
%Mid June

"Are you ready?"

"Ready, Alerio? I do not recall volunteering myself for these services in the first place."

The old man smirked. "Our guests are not keen on attending this meeting either. Come."

I picked a long dark hair off my simple robe and followed the clattering pearls of Head Alerio's ceremonial gown down the long corridor between my office and his. There he made himself ready for the meeting while I saw to the refreshments for our guests and cleared the clutter in the office. Alerio was a neat, well organized man. There was not much to do. After a while, I sat at the table laden with flagons of wine, bowls of fruit and platters of Marsean cakes, and waited in silence. 

The new school year had been in session for a month, my third as the head of the Tower's school. I had faced significant resistance in changing the way in which our students learned to heal. My husband's approach to healing was so intuitive to young students who had not had their minds molded to the traditional methods, and the process of learning how each organ and bone in the body felt in its various states of damage so onerous to them, that I had decided to teach the young students how to heal various injuries before teaching them how to identify what was wrong. This had been the thesis of my third book on my husband's teaching methods. It meant that young students could work with experienced healers on patients at much younger ages than ever possible before. Ten year olds mastered healing complicated injuries to small bones or the intestines, previously thought too difficult to teach in all but the last year. The difficulty in these injuries were not in the process of healing, but in the process of identifying the cause. There was even one brilliant girl, Marcella, in Lyca's class who could regrow tails on geckos. It was the easiest of all regeneration processes, granted, but no one had thought it possible to master at such a young age. As the children grew older, they would learn the theory and physiology behind the art that the Preserver had granted them the power to perform. In the meanwhile, Cortan had classes of youngsters who could heal as well as I could, if someone were to tell them where to direct their fire. 

There had been resistance, of course. The older children, used to the old ways of teaching did not like working with six and seven year olds. The teachers complained of the difficulty in teaching classes comprised of children of vastly different ages. I had been forced, in the end, to increase the number of classes offered by the Tower' school, separating students learning the same material by ages. This of course, caused my teachers to complain even more. Alerio let me deal with these disputes on my own, and I managed, somehow, to unruffle feathers and buy myself time. It was a good opportunity for me to learn to lead, Alerio had said. I had no desire to lead, I would have argued, but I did. The prospect of heading Cortan's Tower lay unspoken and silent in the corner of all our interactions now, ever since the day I presented him with my findings on the mushroom blight. Not to be the head after him; he had picked his favourite for that position months ago. Master Manlio from Belisal, Alerio's nephew, would lead after him, if the Tower agreed. I was young yet, I could follow Manlio, or become the head of the Women's Tower. There was still time for him to groom me. My presence at this meeting was a part of his attempts to train me.

I was not entirely certain I wanted the position, for all its temptations. The issue of the black powder still loomed large over the nation's Towers, Carlotta had not yet found a cure for me, and she despaired of ever being able to. She shared my workspace in the Tower, in order to have access to the mushrooms I grew. She plagued me with requests for students and aids to help her with her work. She feared she could not do this alone. I could not allow her that. We had agreed that this would be a tightly held secret, known only to three people. The more who knew, the easier for the news to leak. Timmon, at the very least was still a fear. He had left his man Hinata, and who knows who else, to spy on me. It was more than that. Over the last two years, my condition had proved uncomfortable, but it had not deteriorated. Once cured, and my husband condemned, what would I do to fight Marsea's avaricious masters? A horde of white robed men and women who could inflict consumption and Preserver only knows what else on their enemies was unthinkable. I feared unleashing on Marsea a host of men like my husband. What would become of my country. It would be better if Carlotta continued to work in solitude and never found a cure. 

Alerio disagreed, of course. He thought the masters could be controlled. Perhaps they could, but it would take a stronger politician than I to do so. Alerio would not listen to my excuses. Two weeks ago, he recieved word that Head Eliseo and Master Cleto would visit my husband. He asked them to pay him a visit and ordered me to attend.

Two wary grey headed men entered the Head's office, both in ceremonial robes. The elder, Master Cleto had a long oval face, stooped shoulders and leant heavily on a cane. Alerio stepped forward to help him into his chair. The younger, Head Eliseo, was a man of Alerio's age, perhaps a bit younger. He was a head shorter than his companion, barrel chested, with weak eyes, bald, but walked with the firm gate of a man twenty years his junior. He greeted his host with a stiff formality in his loud baritone voice. When I sat at the table with the rest, he gave me, and my simple robes, a look of disgust, then turned to Alerio.

"What is this, Alerio? You said this discussion was private."

"What you say here will be held in the strictest of confidence," came Cortan's head's even staid responce. "You have brought a master to my table to aid you in your arguments. So have I."

"This whelp?" Head Eliseo gestured rudely in my direction. "I will not discuss matters in front of a self serving youngster who dishonors the collar given to the experienced. This is an insult to our Tower. Remove her, or we leave."

"My master stays." Alerio insisted calmly. I felt myself flushing, yet the head showed no sign of distress.

"Come, Cleto," our younger guest said. "We will not discuss matters with children."

I watched the older man struggle to his feet. Alerio would say something, I was certain, to bring them back before they reached his door. But then the old man would have to hobble back to the table. I pitied him. "I am sorry, Head Eliseo, that you do not see fit to speak in front of me. Please give my regards to my husband, when you see him today. Also, would you be so kind as to inform Head Luce, when you return home, that I cannot give leave for the Mother's Balm to serve at your tower. She is, however, welcome to send as many students as she wishes here. If you will not meet with me here, I will see you at dinner at the castle tonight."

Head Eliseo stared at me, dumbfounded, his square jaw hanging slightly open. Master Cleto struggled to his feet and reached for my hand. He spoke in a soft quavering voice. "Your grace. Please forgive my companion's rudeness. You wear no mark of your station. We will, of course, be happy to conduct our buisiness with you."

Happy, I doubted, given their intensions to speak to my husband, but I believed that he wished to talk. I did not, however, offer him my hand. "On the contrary, Master Cleto, I wear my master's collar. I have no other rank here. You may bow to me as duchess at the castle. Please, make yourself comfortable."

Head Eliseo made a sour face and returned to the table. But he returned. The conversation started uneasily, but it started. Alerio asked our guests whether they supplied my husband with mushrooms. They would not answer that question, not with my presence at the table. I had warned my head of that danger, but he had ignored me. He then asked about their continued harvesting of the mushrooms, which they again denied. Similarly, they denied having any Masters working on the old magic or healers practising with the black powder. The conversation continued fruitlessly until Alerio informed them that Commander Dielo had recently become one of his magesty's Tower inspectors, a title that Barron Paulis had only reluctantly allowed me to write my nephey the king for. We had two other inspectors in Cortan's armies, both commanders newly brought from Escasaine, men loyal to my husband. Reluctantly, the men admitted that they did supply my husband with mushrooms, though their harvesting program had been scaled back severely. They also supplied Escasaine, two towers in Selvand and three in Voltain. They denied supplying the mushrooms to their own people, the price of the mushrooms being so high, it was more lucrative to sell than to toy with the well controlled substance. I took down the names of the towers and held my tongue. It was better for Alerio to conduct this interview. Alerio pointed out that the harvesting program would have to stop, but did not mention anything about interfering in their supplying my husband. He asked if they knew of why my husband wanted the mushrooms. They said the that duke did not speak of his work during their meetings, only of his plans to overturn King Gustav's ban on the substance. 

When the men from Allepo's tower finally left, and I was left in relative peace and solitude, I realized that I was filled again with the all consuming fear that had ruled my life for so many years in Deyalorn. They would meet with my husband, and they would tell him of my presence at this meeting. They may even twist the tale to make it appear as if I had instigated this, and not Alerio. And then what would my husband do? Would I have to fight him for Commander Dielo's position, or life? Would I have to face the jeering faces of my court, would I have to retreat into this Tower again? With Makala gone, would he turn to Emile as his price for my disobedience? The possibilities were manyfold and terrifying. I stood abruptly and started clearing the dishes from the table, cups and platters clattering in my shaking hands as I did so.

Alerio looked up at my sudden surge of motion. "Leave those, Nisrita. A student will come by to clear that all." He set on the table  two crystals, side by side, one brimming, the other merely full, with a dry golden wine of much finer quality than the drink he had served during the interview. "Sit. You are not the serving girl they took you to be. Tell me your impressions of the interview."

"My impressions?" I asked angrily. "They will go straight to my husband with news of your interefereing questions, only in their version of the story I asked the difficult questions, not you."

The old man shook his head. "I doubt that very much. Eliseo and I go back a long time. I doubt that he would prefer to soil your name over mine." I sulked at my overful glass of wine. Even if that were true, my husband would not exact his anger of Alerio, safe and untouchable in his tower. He would take it out on his politically feeble wife. My friend and mentor sighed. "You are still afraid of him, after everything you have won." 

I could feel the pity forming in his mind, hear it whispering between his ears as he sat beside me, even if it did not express itself in his voice. The stupid old man. What did he know of husbands? What did he know of my husband? For as long as he lived, and my husband planned to live forever, he would be a source of terror for everyone under him. I had accomplished a lot these last two years. That much was true. I could control the sun and the sky itself, but I would still be his wife, and he would still have control over my children.

As promised, a sixth year student entered the room and cleared away the debris left over from the meeting. He startled when he saw the head of his school in the room, then continued with his duties. I watched him work, in a neat, orderly fashion, and sipped my wine. Isaia, his name was, I reminded myself. He had come to us three years ago from Gissal when they discovered his gentle touch. His father was a blacksmith, his mother a trader's daughter. No one had suspected him of weilding the gift until he went with his elder brother went to a tower to tend to an arm maimed by the bellows. The healers at the tower handed both boys a talisman and asked them to perform certain tests, as is standard practise with children. Isaia exhibited an ability to weild the gift. He was, unfortunately, only weakly gifted, and a poor student. I had taken over the school early enough in his career that he knew how to heal several varieties of wounds, but he did not possess the intellect to learn how to identify them. He was a problem; his teachers tired of working with him. Perhaps I would have to send him to the barracks for a few years to learn the discipline he would need to be a gifted fighter before re-enrolling him in the school to learn to heal properly. 

The boy left. I knew each of my student's histories, strengths and weaknesses as well as I knew Isaia's. I had created something to be proud of in this school, and yet, I still feared my husband.

"You put on an effective performance in calling our guests back to the table. Eliseo is rarely at a loss for words," Alerio said when we were alone again. 

"It was clumsy and heavy handed. I should have worn a mark of office." Head Eliseo's attitude had been humiliating. Without that prod, I would never have exposed myself to my husband's wrath as I did. "You would have found a more elegant solution had I held my tongue."

"Perhaps." The old man shrugged. "Your insistence on appearing as a healer in the tower and a duchess in the castle is tradional, if not as common a practise now as it used to be. I know your staff appreciate being approached as only the head of their school, and not their liege lady as well. Word will get out of your habit as you appear in public more frequently. Eventually, you will be famed for your devotion to the Preserver." More frequently, I thought. Alerio would not let me be. "I was curious" he continued, "how you would react to the insult. You did well."

"Thank you," I sulked. The man had set a trap for me, and I had walked blindly into it.

"What will you do with what you have learned today?" Alerio reached across himself to refill my crystal, full but not brimming this time, and his own. He was alway generous, both in the quality of his vintage and amount of wine he poured for me. I had learned to be careful not to meet with him in the mornings if I planned to meet with students and teachers later in the day.

"I have the names of the towers Allepo supplies to. I will wait three days, then write to my brother about those in Selvand and Voltain. In a week, I will ask Commander Dielo to investigate Escasaine. That should give them enough time to destroy their current work, and hopefully scare them enough that they will not start anew. You have given our guests permission to keep supplying my husband. That is understandable, we need him to keep working. Beyond that, I do not see that there is anything to do."

"You have missed the fact that Master Cleto is working with your husband in his research."

"What?" I cried. "They said they knew nothing of my husband's work. Did they lie? How do you know?"

Alerio looked at me with his phenomenal calm. Then the corners of his mouth rose slightly. We sat so close to each other, I could count the wrinkles, three  on the left side of his face, four on the right. But his eyes remained still, almost sad. "My apologies. I have an advantage of over three decades on you. Master Cleto and your husband were healers and friends in Cortan. I studied under Master Cleto as a youth. He had brilliant mind for study, I have met few better, but was never good about writting down his findings. I have long suspected that much of your husband's two books on skeletons were based on Cleto's work. No matter. They were close friends, Cleto as mild mannered and apolitical as your husband was ambitious and imposing. My mentor would not have objected to someone like Duke Griswold taking his work and publicizing it. Cleto was crushed when your husband disappeared, or died, as he believed, overjoyed when he reappeared, and furious with me when I opposed him. He is an old man now, and frail, as you saw. He does not travel much any more. Yet this is his second visit to see his old friend since your husband became the duke. What would draw him out this far other than a chance to learn, in his dotage, how to control the old magic that may grant him health and youth again?"

I took a large swallow of wine. It was so difficult to think of my husband, large, imposing, spry, as the same age as the doddering old man who had sat before me for two hours this afternoon. Master Cleto looked well over seventy. I did the calculation. My husband had would be in his midseventies if here were not perpetually stuck at twenty six. It was cruel of the Preserver to allow such a vicious man as he to live forever while good men like Alerio must wither and die. 

I took a second swallow of wine to find the courage to face the next line of quesitons. "If they lied about their involvement, what else did they..."

"They did not lie, Nisrita. They misled you. They said the duke does not speak of his work during their meetings. That does not mean that Eliseo or Cleto do not. Nor does it rule out the possibility that the Duke invites them here for an opportunity to compare notes on ongoing work that is too sensitive to send via more convential means of communication." My head swam, not just from the alcohol. I drained my cup and dug the heels of my palms into my eyes to think. "You really should find out more about your political opponents before you engage them. You had ample time and opportunity before this meeting." Alerio spoke as my tutor again. I had disppointed him.

"Who else?" I asked, not lifting my head.

"Pardon?"

"Who else do you suspect of knowing? How big is this conspiracy!"

"I was not certain of Cleto's involvement until today." Alerio began cautiously. " I suspect Adele is involved in this. She the heading the Women's Tower at Escasaine, I believe. I hope cutting off Escasaine's supply from Allepo will slow her down. I would not be surprised if Master Govind of Deyalorn is as well. He was a strong supporter of your husband's, though it seems he gets his mushrooms from else where."

"Preserver defend us," I whispered. With four experienced healers working with my husband, what chance did Carlotta have of finding a way to cure me before they did. If someone succeeded before my husband, before Carlotta found out and gave us a chance to act against my husband, every healer in the land would know how to control the old magic. Even if she learned the secret first and exposed my husband, there would be four others to contend with. I could not hope to keep word of this quiet.

"We cannot simply pray our way to a solution. What do you intend to do?"

"I can ask the crown to investigate all three of them. My brother Dario has no love for either Head Adele or Master Govind."

Alerio looked at me doubtfully, tenderly. "Just a moment ago, you were afraid of what Cleto and Eliseo would say to your husband, and how he would react. Is this really what you wish to do?"

"I don't know." I said, and blinked to rid myself of something irritating my eye. I found moisture rolling down my face.

Alerio cleared his throat and filled my cup. Destroyer take the man. Let him take both of us. The man beside kept pouring and I kept drinking. I would not have been crying had I been sober. I took a swallow anyway. I had no way to face my husband. Without standing against him, I would let him control the old magic. The implications of either course of action devastated me.

Alerio sighed deeply and spoke in a quiet voice. "You seem to think, Nisrita, that you are the only young healer who's name and career your husband has tried to destroy. He is long lived and has always been vicious. I tried, after gaining my healer's rank, to establish my own body of study on human skeletons, completely separate from and sometime contradicting Cleto and Griswold's work. Cortan was young, border raids a constant reality. It was easy to find bodies to study and prisoner's to work on. Your husband would not hear of it. He physically threatened me. I ignored him. What are a few broken bones to a healer? He pushed me out of Cortan's Tower, so I found a position with Firvona's. When I still refused to give up my line of study, I think Griswold took it as a personal affront. He told my father, the then duke of Belsale, of my longstanding affair with a young Nievian woman. My father took my love as a sign of my trechery to my country and disowned me. I was his fourth son. I never stood to inherit much, but I had his name and his support, and that of my brothers. Without even that, I had little choice but to join Firvona's Tower in service to the Perserver." The old man paused and shrugged slowly. "You are right in one respect, I have not been completely honest with you in my motivations towards wanting to destroy your husband. They have not been the same as the reasons behind your desires, but they are not so different either. You cannot hope to hurt him by giving ground."

"You want me to approach King Gustav and ask him to let the nation's Towers have access to my husband's work. It would end in slaughter for Marsea's ungifted. It would end Marsea as we know it."

"I do not want you to let the Tower's run free with this power. I want you to forge it, help the king forge it, if you like, before someone else does." A cold spotted hand appeared over my own. "Oh, if only I were thirty years younger-- I would have rid you of Griswold long ago. You would have been the temperance to my ambition, the chastening of my avarice. We would have approached the crown together and faced this threat to our nation as one." Alerio sighed. "As is, I can speak my mind in Deyalorn. My words will be heeded. But I am an old man. Who will see my will preserved when I am gone?"

"You are drunk, Alerio." I said, looking pointedly at my hand. "Look to someone else. I will not support you on this."

"Perhaps." The spotted hand withdrew. Its owner sighed again. "I would never wish to force your will on anything. I hope you know that. And now," Alerio said, rising, "I believe you have a dinner at the castle to prepare for, so our afternoon's guests may address you in your other role. I have kept you too long."

\vspace{.5cm}
%Aug

"Your husband knows of our activities." Carlotta stormed into my office and slammed the door shut. She removed a large bundle of papers from under her arms dropping it heavily on the floor before my hearth.

"That is impossible. And I have meetings this morning. You cannot stay here."

"I've chased them off. You are otherwise occupied today."

"Carlotta," I moaned then paused to look at her hunched before my hearth. "What are you doing?" She was crumpling the papers she had brought with her, and throwing them onto the cold hearth. 

"I am burning our records."

"No." I bent down and helped my once friend and current healer to her feet. She did not resist. Nor did she rise willingly, but let herself be lifted, shaking and crying to a chair. "What happened?"

"He must know," she said through her sobs. "Why else would he suddenly turn upon my family like this? We have done him no harm. He is a cruel sadistic and manipulative. I don't know how you live with him."

Oh, Carlotta. You are normally such an intelligent woman. How is it you are just now learning what my husband is like? "Has he threatened you?"

"Yes, I mean no. But he will if we refuse him. My husband is questioning me, and I don't know what to do." She sat with her head in her hands quivering and drying her leaking eyes.

I poured my guest a glass of lemon water. It was already hot, the mid summer sun burning even the cool white stones of this tower. The Liri custom felt more appropriate than the Marsean one of wine. "Drink this, calm yourself, and don't leave my office. I will return in a moment." I paused with my hand at the door. "And please don't burn those notes just yet. You have worked too hard for them."

I left Carlotta to cancel my morning's meetings in a more orderly fashion, then went to find Alerio. Whatever Carlotta had to say, it was more likely to be linked to my role in the arrests of Master Cleto and Head Eliseo last week than my husband discovering anything about Carlotta's research. In either case, Alerio needed to be aprised of the situation. It took me a moment to get his attention. He was in a meeting with the stone mason, Gallo, probably for some repair work to the walls. Why the head did not leave such matters for groundskeepers, I did not understand.

In the two months since our interview with Master Cleto and Head Eliseo, Alerio had made no further embarrassing overtures towards me. When I approached him on the subject of turning over the names of my husband's four accomplices, he had shown a great deal of resistance to implicating his old mentor. Master Cleto was an old man. Out of respect for his age and Alerio's feelings, I agreed to write my half-brother only about Master Govind and Head Adele. It turned out not to matter. My half brother's raid on Escasaine's tower was thorough and humiliating enough that I was glad that Firi was not there to see the humiliation of her teachers. A half dozen members of that Tower were arrested, and made examples of. Ten arrests were made in Deyalorn, at around the same time. Commander Dielo echoed my half brother's zeal, or acted under his command, it makes little difference, and undertook a surprise raid on Allepo's Tower. They found evidence to make only three arrests, Head Eliseo and Master Cleto among them. The crown was angry at Marsea's Towers for what it saw as a betrayal of our vows to protect our citizens and aid our army. The military, especially our gifted fighters felt betrayed. The ungifted men came short of accusing us of treason. It made sense that the military would want to be appointed inspectors of the nation's towers, not trusting us to govern ourselves. What should I have expected when I made Commander Dielo, the leader of Cortan's Black riders, the unit most supported by gifted fighters, than that he act zealously against those I named? It was what I wanted, I told myself. Marsea's Towers needed to be remined that we served our country, not our own ambitions. Yet when I saw Alerio mourn his mentor's arrest, I could not help feeling like a traitor to my own. Master Cleto was an old man. He had give an long life and  dignified service to the Towers. He had the right to a quiet, comfortable final few years, not the shame of a trial and the ignomous death in a prison cell.

When Alerio and I had entered my office, we found Carlotta, now dry eyed, sitting at my desk, unrumpling the pages she had so hastily thrown into my hearth. She stood hastily as I entered to vacate my seat.

"Now, Carlotta, could you please start from the beginning?"

Carlotta and Alerio found seats while I served more lemon water and perched on the end of my desk. "His grace spoke to my husband last night," Carlotta began, sounding as if she may break into tears again at any moment. "He wanted my husband to leave Escasaine to accept a general's post in Cortan. But he had a condition, that I leave your service, and return either to Escasaine or Allepo as Mother's Balm. My husband was furious with me this morning when I refused. He feared that I would be arrested by the crown for whatever dark conspiracies I engaged in with you in your room. He is convinced that I am researching the old magic. I have three children to care for. What will happen to my family if the duke tells the crown of my work here?" 

Alerio and I exchanged looks. It was not my husband Carlotta feared, but her own past. Every healer who had ever played with the old magic now feared the crown's anger, more sharply now after the recent raids.

"Listen, Carlotta. Think for a moment. Why would my husband write the king to tell him of your work. He needs to keep his work a secret more than we do. You have the cover of my workspace, the only legitimate space to study the black powder. My husband has nothing. He is in even greater danger now, with the recent arrests and crown's paranoia. You are being frightened by shadows. He told your husband nothing."

"Someone infiltrated Escasaine's Tower and told the crown about Head Adele, Master Ines and Master Guerino. They were all good people, none of them the traitors the crown makes them out to be. No one knows who started the rumours, or set the inspectors on Escasaine's Tower. It may have been Barron Romino for all I know. Who is to say that he isn't spying on us here?"

I closed my eyes and took a deep breath. Carlotta was frightened. I could see why. She had spent years at Escasaine's Tower with colleagues who thought themselves beyond the crowns reach because of the powerful company they kept. They had all been arrested. If she had not been working with me here, she may have been in their number. "Timmon, is not spying on us here, Carlotta. It was not him who leaked the information to the crown. I gave the names that led to the arrests."

"You?" she shrieked. "How did you know about Escasaine's activities?"

"Because I told her," Alerio growled, breaking his inemicable calm. I stared at him, surprised by his vehemence. He grieved Master Cleto, I realized, feeling guilty for my role in his pain.

Carlotta's eyes went wide as she stared from head to duchess and back. "You? Why? Why did you tell her of all people?"

Alerio rose. "Because, Carlotta, I happen to think her the better heir. She may be as politically naive, in your words, as a ten month old babe, but she stands up to political pressure better than you have displayed in your illustrious career. One can be taught to be political. Not even the Maker's Daughter can regrow a spine. Now if you both will excuse me. I have more important matters that demand my attention today."

Carlotta and I both sat stunned by our head's anger. Neither of us had ever seen him enraged. He was a man known for his level headedness. The sound of my door shutting called me to my senses. I ran after him, catching him before he left my antechamber. "I am sorry, Alerio. I did not mean for Master Cleto to be arrested as he was. Believe me on this point."

My old tutor closed the door he was halfway through and turned to meet me with a long grave face. "Your actions have consequences. You are not a stupid girl. You can think them through as well as anyone if you chose to. You chose to alert the crown. I did not stand in your way. Do not cry like one of your students because you do not like what the crown did in responce. Now clean up your mess with Master Carlotta so you can go back to running my school."

I let him go. What could I say? He had let me make my mistakes knowing the consequences, now he expected me to learn from my actions. It was no more than I would ask of my children or students. I walked out of the cramped antechamber and into my office. "Well?" I asked Carlotta. "How shall we fix this?"

"You traitor," she spat.

"Yes," I acknowledged. There would be many white healers who would think that of me now. "But you have always known my views towards the old magic. I have made no secret of it. I am sorry for the arrest of your colleagues. I will fight my husband on this and demand that my kind not consider themselves above the laws of the land." I paused to see if Carlotta had any further words for me. When she did not, I continued. "As for your husband, you should resolve your domestic issues as you must. I make only one condition. If house Madriano allies itself with my husband, I keep Cesara. No harm will come of her. Elena is fond of her. She will have the same education and care I would give my ungifted daughter. I will see her married to a good man. But I keep her as a guarantee of your silence."

"Bitch," Carlotta spat at me and stormed out my door. I closed my eyes as the spittle landed on my face. 

I closed the door gently after her and sat at my desk until my heart stopped pounding in its chest. Carlotta, in her haste, I noticed, had left her notes, our notes, on my desk. I put the papers carefully away, washed my face with water, and patted it dry with a binding cloth. I walked out of my office in search of a student. I had teachers to meet with and student's futures to mould. This morning's dramatic events put me nearly an hour behind schedule.

\vspace{.5cm}

%September

I lay in my darkened bedroom, praying for night and the cool it would bring. My maid replaced the wet cloth from my upper back with a fresh cool rag, then continued fanning my prone body. In the three weeks since my argument with Carlotta, she had maintained her distance, and her husband remained unalligned. In the mean while, my womb needed cleaning again. It had not needed cleaning since I had marched in spring. I had been grateful to the triumverate for giving me over four months of peace. My gratitude had been premature. The Preserver simply waited for my healer to be indisposed towards me to let my illness flare. This time, by the same perverse mercy, I did not suffer from accute vomitting, but a constant nausea made worse by the late summer heat. My duties at the Tower had become a nightmare. My existence seemed only bearable starting an hour before sunset, and became miserable again within two hours of sunrise.

I moaned when I heard a commotion outside my door. My maid placed a fresh rag on my hips and went to quell the noise. As soon as the door opened, I heard Firi's panicked voice, begging to see me. My maid, Destroyer be kind to her soul, insisted that I was indisposed and could not be seen. I sat up wearily and found a light shawl to throw over my shoulders to cover my nakedness. "Let her in, Elsa. I'm up."

Firi dashed past my maid at the sound of my voice, entering my room like a wild animal released from a trap. Then she stumbled in the darkness. "Your grace, come quick," she blurted out, not letting her blindness stop her. "Please, you have to come."

"Firi, calm down child. Why are you here? You are not supposed to leave the Tower without escort." It was not like Timmon's daughter to appear perturbed, let alone panicked.

The words tumbled out of her mouth as she fumbled in the darkness towards my voice. "I came here with Lyca and Elena's guard. Master Eulalia knows. Please, your grace, you have to come."

"Tell me what happened, girl. Come here." My maid had drawn the curtains to my room, by this point letting in some light. Firi stood before me, imploring. "From the beginning now."

"I came to the castle with Lyca and Elena's guard, as I often do. We play by the turtle pond. Cesara is usually waiting for us there, feeding crumbs to the birds. She wasn't there today."

"Child," I soothed, not understanding the normally calm girl's strong reaction, "that is no reason to make this fuss. She has been kept late by her music teacher, that is all."

"No!" she screamed, then poured out the rest of her narration as fast as she could. "We spoke to her teacher. She let all the girls go over an hour ago. None of her friends have seen her. They all said she went to the turtle pond to wait for us. None of the men on the walls have seen her either. I went with Lyca to ask. When Lyca said that they were all his father's men, I came to you. You have to believe me. No one else will. Elena's crying. Something horrible has happened."

I found myself impressed. Firi had been as thorough as her father would have been in her search for her friend. It was possible that the child was right. "Calm down, child. I believe you. I need you to do something for me. Tell one of the guards at my door to get Barron Paulis for me. Then step into the next room while I dress. The barron may need to question you, I will need you to be calm if you do. Can you do that?"

"Yes your grace." The child took a deep breath and dropped the neat curtsey for me that she had neglected on her rush to enter, then walked out of my room almost as composed as she normally appeared.

In the meanwhile, I put on as many layers of stiffling clothing as I could bear, then sat down to battle my nausea and await the barron.

He appeared, visibly irritated a few moments later. "You wished to see me, your grace?" he said, bowing stiffly. 

"Yes, enlighten me on a point, if you could. Are the men on the walls facing the turtle pond yours or my husband's?" Lyca was a shy retiring boy, smart, but I did not trust him to be aware of his surroundings, certainly not as much as Firi trusted her friend, nor, for that matter, as much as I truted Firi's judgement. 

"How should I know your grace?" The barron's snub nose twitched in a sneer. "You interrupted my work for this trivia?"

"I assure you it is not trivia, you honour. As for how you should know, I suppose you can either ask the guard captain, or go to the wall and check." I gazed up calmly at my barron. The test of wills had begun. I would win this. I usually did, but for any request I wanted that did not please Barron Paulis, I had to endure a humiliating contest of stares until he deigned to comply. It took five long heartbeats today.

When he finally left, I recalled Firi to my side and we waited together, silently, for the reply. "His men, your grace."

"I see." I had to give Lyca credit for having learned something until the Barron's care. Firi did an excellent job of not looking vindicated. "Barron, we have a problem. Commander Madriano's daughter, Cesara, has gone missing. All of her friends saw her leave for the turtle pond. None of the guards overlooking the pond saw her arrive. I want you to find the girl."

"This is ridiculous, your grace." Barron Paulis spluttered. "I have more pressing things to do than find a girl. She has become bored with her friends and gone off to play elsewhere."

I closed my eyes and fought off a wave of nausea that threatened to swallow me whole. I was tired of being thwarted by this man who ran my court. "Very well." I snapped. "You know the situation with House Madriano as well as I do. Pray that the worst that comes out of this is that my husband wins them to his side by kidnapping their daughter only to return the girl to her frightened parents, safe and sound. Otherwise, if even a hair of her head has been hurt, you cannot imagine the extent of the fury I will unleash on you for failing to protect my healer's first born child. Now leave me, before I believe you to be in league with my husband."

The man hurried from my room as I roared the last sentence. Firi stared at me wide eyed. "Your grace?" she whispered, putting a hand on my arm.

"Leave me, child. I am feeling unwell. We will find your friend. Do not fear. No, wait." I swallowed a mouthful of bile. "Find Elena. Bring her to the next room. Take care of her if you can. When Lyca comes for you tonight, tell him that all three of you are to spend the night in the castle. My men will see that you make it to your morning classes."

"Yes, your grace." Firi left me with more calm than I could manage under the circumstances. 

I gave my orders to my guards, and sent word to the tower, and returned to my impossible battle against my nausea. In a little over an hour, I learned from Barron Paulis that Cesara had been probably been kidnapped. His men had found a doll in the garden, Carlotta identified it as her daughter's, claiming that she was inseperable from that toy. When she learned where and how it had been found she fainted.

"The men on the wall saw nothing?"

"No, your grace. We questioned them. They claim they saw nothing, not even the child enter the garden. I fear you may be right."

I did a poor job reacting to my vindication compared to Firi. "You fear I may be right, barron? You should fear what my husband is doing to her."

"I do, your grace. It was only a manner..." 

"Find the girl, barron," I interrupted his blithering mincing words. "Bring her to me before my husband brings her to the Madriano's. And if anything happens to her, I swear, by the gods' sacred dance, the Destroyer's vultures will be kinder to your soul than I will be to your body."

I stood, half a head shorter than the man, yet I felt like I towered over him. I watched him back out of the room, paying me the respect due to my station for the first time in the three years we have lived in Cortan, and more than a little fear. I spat on the floor as he shut the door. Rage filled me now, not nausea. My husband had gone too far. Cesara was a child. She was Carlotta's child. The mother was a snake, but I had promised her that I would keep the daughter safe if the mother turned from me. She was as much a child of Cortan as Firi was. 

I turned to my maid cowering in a corner. "Up girl. I have a message for you to deliver." I scribbled down a list of items, sealed the note and gave it to Elsa. "Take this to the sergeant at the gate. If he hesitates, or seems to be one of my husband's men, tell him what you saw here just now, and that there will be a repeat performance if he does not obey. The man will give you a parcel to bring back to me. It will be heavy. Can you manage?"

Elsa left me. The day had started to cool, my body started gaining its strength. I changed into more suitable clothes and awaited my delivery.

\vspace{.5cm}
 
The sun hung low on the horizon when I stood, fully armed at the south wall of the castle. A pile of ropes and grapples lay at my feet. Three floors above me, my husband sat in consultation with his men. Makala, my first husband, had once told me that there isn't a mountain face or castle wall in all the world that Crotan's scouts could not scale safely in the dead of night. I had never been trained as a scout, but neither was it night. I had, however, spent much of the last three years teaching Lyca to climb. First, to help him escape the bullying of his older brother, then because we both found it a pleasant way to spend time together. I would never have imagined that the frail eleven year old would learn to scale a wall faster than I could. 

I tied a rope around my waist and began my ascent. The castle guards would do nothing to stop me. They would think that the duchess was in her madness again. That she climbed without her son this time, and at an odd time of day, may appear odd, but what was that irregularity compared to the sight of the head of their duchy dressed in man's clothing scaling the castle wall?

Between the second and third floor, I was grateful for my training with Lyca. I perched, panting on a lion's head and considered my route. I did not dare make any noise to warn the duke of my coming. The windows to his counsel chamber had ornate borders that would serve as a handhold, but throwing a grapple up to them would give myself away. My husband would understand my madness for what it really was, even if his men did now. I decided in the end to climb to a different window, where I could risk making noise, up to to the fourth storey, then over to window above the one my husband sat behind, to lower myself slowly onto the desired ledge. 

My knuckles bled and my arms ached by the time I had positioned myself where I wanted. There was only one thing left to do, and then the gods may do what they wanted with me. I peered my head around the window to see inside. No one thought to look in my direction during the heartbeat the my head intruded upon their light. My husband sat at the head of the table, at the far side of the room, in direct line of sight of my perch. He was fifteen feet away, I was pecariously balanced. I had one chance. I could not count on making a delicate shot. 

I calculated my aim, pulled a poisoned star out of my sleeve and threw it at my husband, then hoisted myself through the narrow window. "That was a warning, my liege. Do not frighten my friends." I said in the chaos before me.

"You mad whore! You took my ear off."

I looked at the damage I had done. The duke's ear was indeed bleeding, as was his jaw. I cursed my arm. I had been aiming at his shoulder. "No, my liege. I gave you a warning. Return Barron Madriano's daughter to him, or I will take your ear off. I do not miss."

"I have no idea what you are raving about. Arrest her." Some of the men in the room, howevere, seemed to. Among the looks of honest confusion, I saw Barron Farone's angry stare, and General Ciretino exchange a look with his duke. I wished Timmon were in the room. He would have known what to make of those looks.

"No, my barrons and generals, an abductor of children stands before you. Your families are in danger. Arrest him before me." There was a moment's hesitation in the room until my husband cried out in agony cluching the side of his face. Half the men rushed to my husband's side to see if they could be of assistance. The other half ran to prevent me for leaving. Chaos erupted around me, the men near my husband calling for healers, others for bandages or cushions, and those around me calling me a number of unseemly names, and dragging me to the armed guards. It amazed me, how a simple creature like a jellyfish, almost the lowest of all beasts, could produce such an extreme effect.

Outside, in the calm of the hallway, I spoke to my escorts. "You will find in the pouch on my right hip a large vial. It contains vinegar, and will ease my husband's pain. It is the antidote to the poison I used. Take it to him." The men eyed me suspiciously. I lost my patience. "Did the Maker cast you both as idiots? You can smell the liquid to see for yourself and ask his healer if it will help, but your duke is in agony."

A hand went to my hip and a vial opened, stinging my nose with the scent of strong vinegar. A page took the antidote to my husband. If this came to trial, and there was some chance it would, I had to let it be known that I did not try to kill him: I barely wounded him, I did not use a lethal poison, I gave my husband the cure. That would be my only chance at keeping my life. 

\vspace{.5cm}

I shortly found myself under the castle, in the upper level of cells, the dry tight spaces, not the large wet ones of the lower dungeon. They intended to make me uncomfortable, not kill me slowly by some lingering disease. The warden stripped me naked in front of his men while they laughed and sneered at my scarred and sagging body. They knew who I was, yet did not feel the need to respect my station. They were my husband's men. Griso's men, I thought, and spat at them. They roared in laughter at the feisty hussy who thought herself a man. Then they jeered and joked about the large pale puckered scar on my left shoulder blade. The duchess with the barmaid's body. I grit my teeth and showed them no shame. My anger fed my courage. My goalers, poor men though they were, had lived near a Tower all their lives. They served the castle, thus had ready access to the Tower's services. None of them had a scar on their body that wasn't recieved in boyhood.

I hated my husband for my humiliation, but I had put myself in this positions. Duke Griswold thought me too weak to defend my allies. He had tried to scare General Galderan from me, and failed. He did not know that I had thwarted him, so he tried again with Carlotta. He had to know that he could not cut my support out from under me like this. I would not allow it. Through the castle's dungeon's was my only path to victory.

The guards gave me a thin scratchy robe to cover my deformities and put me in a cell. A man could not have stood upright in the space I occupied. I, short as I was for a woman, could not squeeze my fingers between my head and the ceiling. The floor was less than three feet by three feet in dimension. But there was a bucket in one corner, and a pile of straw in another. The door to the guard room closed and I was enveloped in complete darkness. I was comfortable in darkness, I reminded myself. I sat on a pile of straw, closed my eyes and listened to the rats move about me. 

After about an hour's time, perhaps two, but not three, it was hard to tell in that complete darkness, the door opened and a figure carrying a torch entered. I held up my hand to sheild my eyes from the sudden light. 

"Mother?" young Griswold called.

I stood to face my son. "Good evening, son."

"Mother, what happened?"

I smiled at my thirteen year old first born. He had his father's face, my first husband's face. I had been so wrong about the nature of my twins. Griswold, the larger, was also the gentler of the two, and I loved him more. For the first time since Firi burst panicked into my room, I regretted this mad course of action I had embarked on. There was a risk, a small risk, that because of how I had acted, I would be separated from my children, not by prison walls, but by the Destroyer's arms. Then this may be one of my last chances to look upon my good, sweet Griswold. I loved my children. I did not want them to suffer. I shook the thoughts and hesitation from my head. My son awaited an answer. I knew the consequences when I had embarked on this path. There was no turning back now. "Your father has kidnapped Barroness Cesara Madriano. Barron Paulis is looking for her."

"Yes, I know," Griswold interrupted. "Half the men at the barracks have been drafted into search parties. They say you've gone mad. That you took off father's ear and poisoned him."

I burst into a loud roar of uncontrollable laughter. There was so much anger coiled tightly inside of me that it adopted its own will, releasing itself an unending string of peal after peal of laughter that sprung from my throat and echoed through the hall and the surrounding empty cells. I leaned against the bars, panting for breath when I finished. Griswold looked frightened and wary. I must have seemed mad to him. "Your father exaggerates, son. I wounded him, yes, but his ear was whole when I last saw him. I poisoned him, but not lethally, then had my prisoners send him the antidote. He should have been whole before I was safely locked away."

"Mother," came my boy's surprise, as he backed away from me, his face twisting in fear or disgust.

"Your father is a cruel man, and Cesara is a young girl. That makes for a very bad combination. He has to be stopped."

Young Griswold stared at me in silence for a moment. I had placed him, and all my children in a bad situation. "What do you need mother?"

"Need?" I asked, surprised. "Nothing. Take care of your siblings for a while, and please don't involve yourself in this. You will only get hurt."

"Mother, I ... " Young Griswold shut his mouth suddenly and angrily left me to my darkness. I sat down and resumed my waiting. My son was in a hard position. I hoped that he would find the strength to dicover his path out. 

I do not know much of what I did during those next hours. Anger numbed my mind. I sat on the straw and stared at the dark wall before me, or closed my eyes and stared at the darkeness behind my lids. There was little difference. Perhaps I slept, or perhaps I sat in a trance as the world continued its conspiring at its searching above me. Rats scurried around me, shy at first, then scampered over my still fingers and toes when I became familar to them. I discerned the different smells of rotting hay and acrid urine floating faintly in the air. Darkness enveloped me. Darkness was my comfort. I could live in the darkness for an eternity and not be afraid.

I cannot say how much time passed until the door opened suddenly again, this time letting in a group of men with three torches. I startled at the sound. When I determined that it did not seem to be young Griswold visiting me again, I sat with my eyes closed against the light. I did not need to see anyone other than my children. 

I heard the loud sound of metal against metal as a key turned a lock. Then Alerio's voice spoke. "Get up, your grace. You have been released into my custody."

"I am fine where I am, thank you, head." The Tower did not need to involve itself in this madness. Barron Paulis would see to me later. This was between my husband and myself.

"You are of the Tower, your grace, not the clergy." Alerio's voice reached me, coated with a layer of sarcasm. A large hand grabbed me roughly by the shoulder. "Our order does not perform penance by self abnegation." A light cloak fell on my shoulders, to cover the dirty garb I had been given by my goalers.

I rose and silently followed Alerio out of the cells and through the dark empty castle to a waiting carriage. When it started moving the old man spoke. "Well? What was that about?"

"Has Cesara been found?" I asked urgently. When Alerio shook his head I explained tightly, "My husband needed to be taught a lesson."

"When I last checked," Alerio spread his arms wide, then folded his hands on his knees. "Your husband still lived, and roamed free in the castle, while you lay in custody."

I gave the man across from me a disdainful look. "If my husband were dead, he would not have the chance to learn his lesson." The old man chuckled. "He will not scare my allies in the future. If he is smart, he also knows that the day will come when he will push me too far. Then he will lose his life."

"And I had mistaken you for a model of temperance and chastity," the head muttered under his breath.

"We were both drunk on that day, Alerio. Do not allude to it again."

"No, of course not."

We sat in an awkward silence for a while, listening to the carriage rattle around us and the clip of the horse's gait. It was late. Possibly past midnight. We were the only movement on the streets. 

"May I ask what you had planned?" my companion finally asked. 

"I hoped that my scaring my husband would make him return Cesara sooner than he planned, or before Barron Madriano gave in. Barring that, I pray that Barron Paulis finds the girl before any harm comes to her."

"And after that?"

I sighed. My anger was starting wane. I was tired. "My actions have consequence. I know. I had thought this through. Barron Paulis would not let me stay imprissioned for long. He needs me as a figure head, if nothing else. I would spend a few nights in that cell before my barron freed me. I trust the barron's ability to negotiate himself into a good position. That would keep me from facing the worst punishments, though my madness and bad name would be smeared through the court. I may further lose Emile to my husband for this." I shrugged. "It doesn't matter. I will find a way to win him back somehow. He is still young, and this battle ongoing. If Baron Paulis could not pull us both out of this conflict in one piece, then I would face trial. I barely scratched my huband, I gave him the cure for his pains, it would be almost impossible to convict me for attempting to kill him. In the mean while, I may have sewn a seed of doubt in his followers minds. A kidnapper of children is a distasteful ally."

The carriage rattled to a stop. Alerio got out and I followed him silently into the tower. The entrance hall remained lit. A few scattered men waited for our return. One handed the head a torch and we made our way through the still passages and winding stairs of Cortan's White Tower in silence and shadow. When we were well away from evesdropping ears, Alerio asked, "You did not expect your allies to come to your aid?"

"Allies?" I almost spat the word. Timmon had warned me that my husband owned most of the men near to me. My husband had scared or weakened everyone who wished to be my support, and I had let him. I had no allies in Cortan. I turned to the man next to me to express the thought out loud and stopped myself. He carried the heavy torch high above his head, to make it easy for us both to see. The position could not have been comfortable for him for long; his shoulders stooped as he walked. He had grown balder, I saw suddenly, in the three years I had lived in Cortan. The shadows of the smoky torchlight made him look older than he was. Yet he was beside me, as faithful and dependable as the Tower walls themselves. "No, Alerio." I said softly, "I had not forseen your intervention."

"Your son came to me," my friend explained, "young Griswold. He was quite distraught. He thought you were in your madness. He did not know how to free you. I reminded him that he would be duke if something happened to you, not his father, with Paulis as his regent, and as such, he had some power over the situation. It was your son who ordered you released into my custody."

"Thank you." I whispered. My anger had almost completely drained from me by now, leaving me shaken and overwhelmed.

Alerio shrugged. "It was nothing. Come." He opened a door and I found myself, inexplicably, in gleaming white room of the Master's quarter. I had followed my friend blindly through the Tower, not heeding my steps.

He handed the torch to a sleepy boy in his fifth year, named Lapo, then indicated that I follow him through the room. "Alerio." I protested. "I can't."

The old man put his spotted hand behind my back, not touching me, but ushering me forward all the same. "You have come this far already, it is just a little further."

He led me through a small door on the opposite side of the round white room I had dined in once, and up a small winding set of stairs. The landing opened onto a small sitting room, with a door on the far side, leading, presumably to the bedroom. Alerio went no further than that landing. "My chamber is through there. I have asked a female student to attend you. I have a room in the dormitories tonight. The boy downstairs knows where, if you need anything. And, ah, I am particular about keeping my bedding free of lice and fleas. I know you have only been in that cell for a few hours, but I have had a bath drawn for you, if you would not mind."

"Alerio..." I stopped, unable to decided if I should laugh at his insistence on hygene, or protest his overgenerous hospitality. I was weary. The distinction blurred in my mind.

"This is the best arrangement. I do not trust your husband or his men to keep from trying to avenge themselves on you tonight. This is the most easily defended room in the tower. Try to get some sleep."

I watched his bald head wind down the stairs. When it disappeared, the last of the anger that had supported me through the night did as well. I was in a dream. That would be the only possible explanation. What else would cause Firi to lose her phenomenal poise, me do display such a frightening temper to Barron Paulis, my husband to do something as unsubtley cruel as kidnap a child, and end with myself standing alone in Alerio's private quarters. Everything about today was impossible. 

Yet, here I was. I shrugged my shoulders wearily at my surreal situation and looked at my surroundings. Alerio, for all his power and leadership, was a very private man. Few in the Tower knew anything of his youth other than he had been made a healer in Cortan and had been born a Duke of Belisal. We knew he had taken his vows to the Preserver, but few knew why. His office he kept decorated as one would expect a head to keep his office, pieces of the Tower's history adorned his shelf, souvenirs brought back from battles he had marched on, honours he had recieved from his duke or king. It revealed nothing about the person who sat in it other than his tidy habits and exquisite taste in wine. Before now, I had been as far as the floor below me. Even there, stark white by the tradtion of the Tower, I had learned little of the man except for his fondness of Liri fabrics and food, a fact that surprised no one, given his support of the Liri community in Cortan. But this room, and the bedroom beyond it told a different story, the real story, of the man. I found myself on rugs of green and yellow, the colors of Belisal. The house's coat of arms hung on the wall of the outer room. A map of the lands in the bedroom. The shelves of both rooms were arrayed with small gifts, of varying levels of monetary value given to him by students, everything from a small wooden flute with Alerio's name and the gifter's name on it to a small marble statue of the Head as I had known him to look as a child. Lying on the table beside the only chair that looked used in the outer room was a book of poetry. I examined its contents. The author still lived, but I had not heard of him. The poems were not particularly good or clever. Some praised the dukes of Belisal, some were lurid, some on nature, some praised the gods. I closed the book, feeling like I was prying. The bedroom contained a wood carving of a young woman with a thick mane and sloe eyes, a beloved sister, an old lover, I could not tell. The bed was large and soft, the bed clothes made of green and yellow cotton. They were crisp and freshly washed. Three scented tapers stood in a candelabra by the bed. They smelled of winter roses. That surprised me. I could not recall Alerio ever appearing perfumed before the Tower. 

The feeling of being in Alerio's rooms, seeing these private habits of a man who rarely spoke of or revealed anything about himself, without the man there beside me felt like an invasion of his privacy. It was a more intimate experience than almost anything else I could imagine sharing with him.

"Master Nisrita?" a girl's voice called from the other side of the bedroom. "The water is getting cold."

I tore myself away from the spell that Alerio's personal rooms had woven on me. I shrugged off the cloak he had given me, and left it on the floor to be cleaned. I placed the thin grey, rat eaten robe in the hearth and stepped into the bath as requested. When I convinced myself that I would not infest my friend's bed with undesired invaders, I donned a set of white robes and climbed into it. 

Sleep would not come. I lay still and listened to my student Isotta fall asleep, but my body refused to follow her example. My anger from the afternoon and evening, as well as the surprise and curious tenderness I felt at finding myself suddenly in these rooms had all worn away. What remained was a grating weariness that would not let me sleep. Where was Cesara? Would my husband hurt her? He was too clever for that. Were his men? They probably were, but I did not wish to take that wager.

I rose from the bed and fumbled at the oil lamps at the base of the alter to the triumverate. A glass stopper slipped from my hand and rolled noisily across the floor.

"Master?" Isotta cried, startled. "Are you there?"

"I am here, Isotta. Go back to sleep." My husband would not get me here. "I am only praying."

I lit the lamps, annointed the gods, knelt in the comforting odor of the burning scented oils, and prayed for Cesara and Carlotta.

\vspace{.5cm}

Cesara still had not been found when Alerio interrupted my morning appointments with an unscheduled visit. I found myself annoyed with him. The morning would become unbearably hot in half an hour. I wanted to fit one more student in before my nausea made it impossible to think. He sat himself in a chair squarely before me, before I had a chance to protest.

"I did not expect to see you at breakfast this morning, Nisrita. Did you sleep well?"

"I wanted to see that my children had arrived from the castle safely. I could not sleep last night, though by no fault of your rooms."

The old man stretched out his arms and crossed his legs, finally placing his hands neatly in his lap. They would be on his knees, I predicted, though my desk blocked my view. I forced my mind back to Alerio's face. Where he put his hands while he sat across my desk was not my concern. "Cesara's disappearance is not your fault," he said.

"Of course its not." I replied, rising quickly to pour refreshments and hide my unease. I had spent the greater part of last night in anger and frustration for not being able to better protect my allies. "Does anyone think it is?"

"You do, I think." Take and curse the man. He was proving to be more of a mind reader than Timmon. If only he could prove to be as good at finding the girl. "You felt responsible for Ezaro's death, though I cannot say why, you felt responcible for Carlotta's ailing health, and sent her to Turina because of it."

I put a glass of lemon water down sharply on his side of my desk. "Who told you that? I have only told two people what occurred that day."

Alerio smiled tenderly. "Carlotta knows what you did for her, even if it is hard for her to admit it." He stood abruptly and drained his glass. "My apologies, I did not come here to upset you. I only wanted you to know that you should not hold yourself responcible for this incident as well."

He turned to leave my office. I felt hollow. After the events of yesterday, the worry, the fear, the anger, then feeling so close to Alerio in his room, yet being completely alone, it all took a toll on my self control. I had put it all aside to face my students and my work today, then this meddling old man invaded my office to remind me of my tortured prayers. "What would you have done?"

Alerio paused his retreat. "Yesterday?"

I shook my head. "About telling the crown about the suspected activities in Escasaine, Allepo and Deyalorn."

My friend and mentor walked slowly back to his chair and repositioned himself across from me. "That was three months ago. You have new battles to fight."

I shook my head again. It felt so heavy, I could not stop it moving. "Master Cleto died on the way to Deyalorn's Tower. I have heard from a scribe in Deyalorn that the crown could not accuse most of those arrested for doing anything other than taking the black powder. The crown only accuses Head Adele and Head Eliseo of using the old magic. There was so much terror and destruction caused for so little gain."

"Do you believe the others arrested to be innocent of your charge?"

I considered the question. "No. Carlotta has as much as admitted to trying the old magic continuously for the last twelve years. I cannot believe that she was the only Master at Escasaine to do so."

"Then the terror should be counted among your gains. You have caused others to think twice about disobeying the crown. I would say you have gone a good way towards accomplishing your goals."

My head shook again, almost of its own volition. "You have not answered my question."

"No." Alerio stretched out his arms and folded his hands again. "I suppose I have not. In Deyalorn, during the Counsel of Nine, you accused me of using you to my own purposes. I admit, I knew that your well being would count among the costs of my actions. That did not stay my hand. I did what I could to see to your safety, then moved my attention to the other needs of my Tower and Marsea's gifted. What I would have done in your shoes does not matter. If you are to lead, either in the Tower or the castle, you must learn to accept the consequences of your actions, and those of your allies, for better or for worse, then move forward. There will always be those who will use your regret against you, as Head Corino did the day you saved Ezaro's life."

I stared blankly at my desk. I felt scolded. I should have known this would happen if I asked Alerio what he would have done. The man was so closed fisted with information about himself. Why did he insist on making me his heir? I did not know how to rule. I was better suited to managing students and spending my time studying mushrooms and animals. "Why me, Alerio? You could have chosen anyone in Marsea to see your will preserved, why me?"

A long bony forefinger reached out to tap my desk. "Another day, Nisrita. I recommend that you retire either to my rooms or the dormitory to sleep for a few hours. You have been ill recently, and Carlotta has first been sulking, then too worried to attend to you. Your teachers can wait."

I shrugged off the recommendation and turned my mind to the papers before me. "Later. I have a few matters I wish to attend to today."

The sound of tolling bells from the temple drowned out the end of my sentence. Alerio stood at the window before I could rise from my seat. "They've called off the search it seems. Men are heading back to the barracks from all directions."

I held my breath. My heart teetered on end, balanced on the answer to my next question. "Is she alive?"

"I don't know. I will go see what I can learn." Alerio opened my door and a fifteen year old gifted fighter stumbled into my room.

"The girl has been found, your grace." he panted. "She is shaken but unharmed. General Ciretino is taking her to her parents. Barron Paulis wanted me to deliver this message to you."

General Ciretino is one of my husband's men, I thought, before relief flooded my body, leaving room for no feeling other than complete exhaustion. I sank into my desk and sobbed on my arms, only vaguely aware of Alerio ushering the confused messenger from my office, only to follow him and close my door gently behind himself.

\vspace{.5cm}
***
%Sept

"We will miss the Day of Unions in Escasaine if we do this," Makala whined as he rode by my side among the dry tall grasses that reached past our horses flanks.

"You will attend in Escasaine or Cortan next year," I replied. We had been traveling side by side for three days now. The Duke of Escasaine did nothing but complain. He was short boy, and round, though neither of those characteristics at this age indicated what he would look like as a man, with a face of neither his father nor his mother. His voice had changed into a mans, but the soft round cheek and the constant pout he wore on them made him look a boy several years younger than his thirteen years.

"I wanted to taste my first whore this year. I hear they are as salty and tart as those disgusting pickled fish you love."

"If you dislike the taste so much, you will not mind waiting to discover it." I had been tempted sorely, over the last four months I had lived with him, to beat some honor and dignity into this boy. I had once, to comic results. It was the day after Eugenio had come home. Makala had yelled and screamed in fury at my audacity to punish him. He threatened to write his father about my abuse, when I put a hide belt to his back. He reminded me that it was treason to hurt one of the ducal lines of Marsea, and swore revenge. I threw him down the stairs of the cellar and locked the hatch, leaving him to cool his temper, then forgot about him. One of the cook's boys discovered him down there when he went to fetch wine for dinner. Eugenio gave me a shocked look when the young duke emerged dusty, wrinkled, sullen and late for dinner. Desmond and Valira could not control their laughter, sending the duke deeper into his sulk. Sophia hid her mirth valiantly behind the back of her hand, feigning a light fit of coughing. I did not like resorting to shaming young men. It never ended well, their egos are too fragile. But his boy needed to learn some discipline. What had he done to provoke me that day? I could not remember. Duke Makala was a constant stream of misdeeds that required reprimand. 

"I will not sign the agreement, you know," Makala continued after a spell.

"You do not need to. Your great uncle is your governor, not your regent. His seal will serve. You are on this journey as my page and nothing more."

"Cousin. Duke Erfat is my father's nephew, making him my cousin. My mother is not blood of Cortan."

"You mother was Duke Ergino's heir and duchess, Duke Erfat is your uncle by lines of ascension. If I hear you disrespect the duchess's station again, you will feel the back of my hand."

Makala sat sullen for another half mile, then said "Duke Erfat is my cousin."

My sword was in my hand before I knew it, and the pummel made contact with the bridge of his nose before I knew what I had done. His bay mare startled and threw him, leaving him unconcious among the dry prarie grasses. I calmed his horse before she could trample the boy, then went to inspect the damage I had done. There was a large bloody bruise on his nose, but he was breathing. It looked as if I had broken it. I rolled him onto his side to keep him from choking on his own blood, then checked for broken bones. Aside from a bruise on the back of his head, he was whole. 

I had to learn to control my temper, I reminded myself in Lyo's voice. That had been more than foolish. We travelled alone to the border barrony of Siasso to negotiate with the young Nievian King Menkaura. We had no healers. I looked at my short shaddow in the trampled grass. The barron expected us tomorrow at the latest. Makala travelled slowly. Between the nose and the need to cover ground, I had only one option. I lifted the boy onto his horse and tied him firmly to it. I took his mare's reigns with mine, and led both horses at a comfortable trot eastward. The trot would leave Makala with a pounding headache when he came to, but he would complain of his head even if I did not do this.

I had arrived in Duke Erfat's court to discover that I disliked the man almost as much as I had the last time I had seen him thirteen years prior. He was a weak, decadent ruler who reigned by his brother's favour. Duke Ergino had saddled him with the best generals and given the new lands to his best politicians to keep the border duchy firmly under Marsean control. 

Yet the healer's blight four years ago left our armies so short of healers, yet our men had fought bravely as if the Tower could support them as well as they always had. Well over a thousand men had perished, noble and common soldier alike, that first disasterous year. The second year saw fewer deaths, but still twice times the number of we normally saw. The numbers would have been much higher but whole brigades of men turned upon their commanders and returned home at the sight of vast number of their fallen comrades. The campaign west of Allepo that year had been a route. Only last year did the number of casualties returned to what our armies consider acceptable levels. Efforts of the queen mother Anyesa and other prominent female healers, like the duchess my unconcious page is so quick to disrespect, was the only factor that kept an irreperable rift from forming between the army and the Towers. For the first few years, the new female healers they recruited scared easily, or killed men accidentally, lacking a sufficient number of skilled healers to guide their work. With time and training, the new women became adequate for the job, but poor substitutes for the skilled men and women the healer's corps had lost. Now the White Towers, or at least Escasaine's Tower, complains that it will go years without a class of first year students, given the lack of gifted children born in Escasaine. All the women have been busy keeping the country's soldiers alive. The complaint falls on the deaf ears of Escasaine's ungifted barrons, generals and commanders, as I suspect it does in almost ever other duchy as well. There is little trust for the Preserver's arm by the Destroyer's arm these days in Marsea. The one exception to this rule being Cortan, where the duke's men know of Nisrita's opposition to the excesses of the Tower and her incessant efforts to work in its interests, regardless of their commander's opinion of the duchess.

The end result, for Escasaine, as all over the country is a sudden opening of new lands to be run by ambitious generals or third sons of power barrons, such as the new Barrony of Romino, once held by House Piset. Over the last two years, Duke Griswold has filled these new barronies with weak men, or those loyal to his cryptic interests, rather than those who can serve to keep the new duchy coherent. Instead of this crisis making Duke Erfat a stronger leader, or less decadent in responce to the hardships of his people, it has made him scared. And thus, when I offered my sword to Duke Erfat four months ago, he chose my abilities as a diplomat over my abilities as a general and a gatherer of information to treat with the Nievian King Menkaura. When I argued that Marseans do not sign peace treaties, he pointed out that while not a treaty, my pressence in Lir for the last ten years had been to serve the same purpose. My arguments about Niev being weak and barely an independent nation any more did nothing to persuade him to the contrary. The new king was a menace, the crown had turned its attentions elsewhere, Escasaine and Cortan would remain border duchies for the forseable future. He did not want his city threatened. He wanted to know that he would still have lands to hand the new duke who had come to visit his court in five years when Makala reached his majority.

The boy now stirring and moaning on the horse beside me knew all of this history. It did not change his attitude towards his mother, or his father, or make him see the need to carefully husband his new lands, or learn to be liked by his men. The boy saw himself as his father's son, and his father would give him everything he needed. 

"Undie me. My dead hurds."

"Try to get some sleep, Makala. You will feel better, and we will travel faster if you do."

"Bhad haddened?"

"You fell of your horse. I have tied you to it so we may travel quickly and get you to a healer."

"I did dot."

"You most certainly did. I would not lie to you." I stopped both horses and took out the bladder. "Drink. It is hot today." The boy drank, tended to his own wound, then moaned for a while, and eventually slouched in his saddle and drifted back to sleep.

What the boy did not know, and what worried me constantly were the rumours and whispers that Duke Griswold kept Cortan and Escasaine border duchies intensionally. I had guessed as much during my months in Cortan, but the army in Escasaine, not as divided, and thus freer with their talk, and especially the gifted army, with connections in the Tower, say that the Duke is amassing forces for a war of his own, unsupported by the crown. It seems implausible that even someone like Duke Griswold would do such a thing. Of all the rumoured ends of the duke's personal war the most believable is that he wishes to establish Cortan and Escasaine as an independent state, opposing Marsea and making himself king. Even that sounded thin. Cortan's army and Tower were great, but it could not hold its own for long against the joint forces of the other duchies. Yet Sophia seemed to corroborate that Griswold had something along these lines planned. Upon coming to the new Barrony of Romino, and then the city of Escasaine, she made herself an important part of Escasaine's Tower's social life. It was not hard. Head Adele, of Escasaine's women's Tower, remembered her from Cortan, as did several other colleagues who had moved here from there. She has long made a habit of overcoming her connection with me, and her exotic Liri tastes make her popular in Escasaine, given that the duchy shares a border with that land. Her old friends and companions no longer speak to her as freely as they perhaps once did, but she has heard them make oblique references to Griswold's uprising, once she had  poured enough of my rice brew down their throat, or pleased them sufficiently in bed. She has learned too well, I sometimes fear, from living with me, but I cannot complain. She has used her skills only to my benefit. 

I made the camp while my page stumbled about ineffectively, complaining of his nose. He swallowed his dinner in three large bites, then rolled over and went to sleep.

I stayed by the fire for some time, gazing at the stars above me and my surrounding grasses. For some reason, these grasses made me melancholy. Perhaps it was because I lost Makala in Nievian territories, or because I had nearly lost my life in these plains, a week's ride to the south and west, perhaps because in the eight months since I had come back from Lir, I had not yet found a purpose in my life. Training Makala, helping Duke Erfat keep the peace, these were all tasks that I was willing and ready to do, but they did not fullfill me as leading Duke Ergino's men, or serving as a diplomat in Lir had, or for that matter, as training Eugenio, then Griswold, and now Desmond did.

Every night away from my family, my thoughts turned homeward, and also to Lyo. It had been eight months, and I still missed him as I had the day we parted. I had grown so used to having a partner in everything I turned my hand to. By the end, though Lyo and I never ceased moving in different spheres, I felt that when he spoke to the Sephiyats, and I spoke to the princes that we spoke as two men with one voice, and that voice was neither completely his nor mine. By the gods, I missed him. Sophia, for all her usefulness, could never be the same. She followed my instruction, read my mind for what I wanted at times, and uncovered useful pieces of gossip better than any spy I could hire. When it came to caring for my family, I could ask for no better wife. I valued her, I would be lost without her, but I did not love her. She had no ambition of her own beyond her luxuries and her children, she would never mould me, she was satisfied to serve. She was a good woman, but a woman in the end. 

Then there was Sora. That was a completely different story. Admiral Isumu had given my son a parting gift for his years of service, a slave from King Anko's court. When my stepson arrived home a month ago, he announced that he no longer had need for such a man in Marsea, and offered him to me. Eugenio must know. He spent nearly eleven years in Lir, entering a young child and emerging almost a man. He was well aware of the subtle differences between friendships among men. That he himself accepted a slave as beautiful as Sora, well, that was a matter of some concern for the guardian of a poor son of a commander. I could protect him some with my new lands and title, but that protection could not reach him in Selvand where he wished to make a life for himself. Eugenio knew his station in Marsea, he knew his ambitions, and had the sense to leave the Liri custom behind when it interfered with them both. He made me proud. It was more than I had once done in his place.

Sora reminded me of Makala, my duke of old, in many ways. Small, lythe, athletic, he moved like a dancer, for that is how he served the king until he had broken his leg and been replaced. He looked so young, I almost had him shave his beard off, though he was a year older than Eugenio. Sora was ambitious, that much was clear. He would have had to be to dance for the king, but he was a slave, and not a man. Sora pleased me in every way that he could, and he pleased me well, but I found myself, entering middle age, longing for the company of a rumpled old body, one that fit me as comfortably as a well worn glove than face the nightly delights and challenges of breaking in a new colt to my stables. 

\vspace{.5cm}

Rumours of King Menkaura's presence did not do him justice. He was a tall, fair, muscular man barely twenty years of age, with a voice like the sea; frightening in his rage, and deep and luling in his calm. He was as handsome as the sun is warm, and as proud as these plains are wide. He was not a quick tempered man, he listened to his advisors, and weighed his options carefully. I have met with two kings in Lir and served under three in Marsea. None of them were Menkaura's equal. It was a pity that he had been born to the wrong country, he would have numbered among the great kings of Marsea.

On our second day of meeting with King Menkaura, Makala knelt at a distance from him, as custom dictated, instead of trying to approach to kiss his hand according to Marsean custom. He had not succeeded in this attempted act of disrespect yesterday, a dozen spears barrred his way, but he had tried. We had talked about the incident since. 

"Do you realize, your grace, why I have brought you first to Escasaine and then on this trip?" I had asked the sullen boy last night when we were alone in the room we shared. His nose had healed nicely, and he suffered no ill effects from the accident.

"Because mother made you, and you cannot refuse her."

"Tempting as that story may be to you, no." Makala sat in the one chair in the room given to us, while I leant against a bed post. He sat with the chair askew, his side facing me, staring blankly at the opposite wall. I walked behind him, and turned his chair so it faced me. "Look at me. These are your lands. One would hope that you have an interest in ruling them. I brought you here to Escasaine to learn about your duchy, and on this trip to learn to rule. Why do you refuse to be helped."

"I will have men to deal with the details. I am a gifted fighter, I have served with the Black Riders. My men will respect me for that. If they do not, my father will see to the rest."

The boy had paged for the commander of the Riders, his men would know the difference. "Unless your father gives these lands to Lucretious."

Now I had the boy's full attention. "You wouldn't. Mother wouldn't let you. This is my birthright," he spluttered. The child feared losing these lands more than he feared anything else.

"I have no desire to see these lands ruled by anyone else, not your brother, neither King Menkaura. But once you are in Escasaine, what will you do? Your father has no interest in conquoring the remains of the Nievian territories, neither does the crown. You should learn to live with your neighbors."

"I will have generals who will do that for me. Leave me alone, barron." He pushed me out of his way from my position crouching before his chair.

I moved, but persisted. If this boy continued to show the Nievian king the disrespect he had today, I would have to go home and admit defeat. "Your grace, your peers adore you, but your elders do not. Your generals may not listen to you as well as you think. You should take a lesson from the young king in how he listens to his advisors."

The boy turned on me with a surprising fury. "I knew it. You are a traitor. People whisper it all around me, but no one has the balls to say it to your face. But you sir, are a traitor. You wished to come here to sell Marsea out to the Nievians, just as you tried to sell us to the Liri. Traitor!"

He finished with a dramatic flourish and a finger pointing accusingly at my chest. Spittle frothed at his lips, but I could not fear this boy, I feared my inability to control my own rage more. "Write your father, your grace. I pulled out a quill and parchment and set it on the table. You will find that even he believes sizing up one's enemy, admitting him to be a better man, and learning from him before cutting him down is not treachery, but a general's duty."

The boy's hand went to his side at the sight of the ink and parchment. He stared at me in confusion, I had not risen to his bait. Anger still seethed inside him. I pressed my advantage. "If your generals do not love you, who will protect you if Emile sees your weak court, wins your father's favour and moves against you? He loves you as his brother now, but closer siblings have been known to rise against each other when a large duchy is at stake."

I watched the possibility filter into the thirteen year old's brain. He had never thought of treachery from his siblings. Makala had three younger brothers and a divisive father. How did he hope to remain his father's favourite? I watched him battle against his anger and his hatred of me. Nisrita had said that she thought Makala had improved greatly over the last two years. If this was improvement, I shuddered to think what he had been like before. It was possible, I allowed, that he reserved this behaviour for me. He was as devisive a person as his father, already playing favourties in Escasaine's court before he has taken the throne. He had a hatred for me that echoed that of his father. I feared that I may be the wrong man to teach this boy.

Reason won in Makala's internal battle. "What do I need to do?" he asked in a calm respectful tone, but kept his distance from me.

"First of all, you must kneel to the king, from afar. He is the son of the sun, therefore he cannot be aproached."

"That is ridiculous. How can he be when the sun is not a person to father children."

I sighed. Makala's ability to thwart everyone astounded me. He had no fear or respect for his elders, or anyone other than himself. If only that could be focused in battle, he would make a great warriour instead of a sulking child. "Be that as it may, the proper means of address is either radiance, or son of Roth. Do not call him magesty again."

The lesson continued for some hours, and I went to bed nervous about how the duke would perform. Yet, on the second day, he behaved admirably.

The discussions, on the other hand, continued to go poorly. King Menkaura had no intention of signing a peace treaty with Escasaine when he seemed to have the upper hand for the moment. It did not matter that Marsea clearly had the greater force of arms. Marsea was not interested in turning its eye towards him, and he did not believe I had the power to sway the king otherwise in the near future. He was correct in his assumption. Even if I could in five years, given the gains he had already made, who knows what he could attain in that time. 

The third day, we paused negotiations for celebrate the Day of Unions. Barron Rudolfo Siasso, King Menkaura, and I spent the day at the celebrations. Barronies do not traditionally hold Days of Unions. Marriages are arranged by either Dukes of Kings, yet given the timing of this visit and the need to show hospitality to his royal guest, the barron had gone out of his way to invite a set of performers to mark the day and gave a large feast. We three watched the performers and enjoyed what the small border barrony had to offer. Makala enjoyed himself as he could, I did not see much of the boy.  Queen Kanika, and the king's sister Meshkhenet spent their time with the ladies of the house. They were both young attractive women. During their brief appearance at the feast, they held the attention of almost every male eye in the room. It was no wonder that the King feared for them and kept them hidden away in the company in Siasso's women.

The young duke  was sleeping soundly on the floor of my room when I turned in late. It had been a profitable day, on the whole. His radiance seemed to soften towards me as we spent our time admiring dancers and drinking wine rather than arguing terms.

\vspace{.5cm}

A loud banging at my door brought me sharply out of a deep sleep. "Barron Romino, open your door. The duke has been arrested."

I jumped out of bed and opened the already unbolted door. "On what grounds." I roared into the messenger' face.

Far from flinching, Siasso's guard met my anger with a stern coldness of his own. "The barron arrested him himself, on charges of attacking the family of Roth. The barroness's maid found the duke with the princess."

By the seven harpies of hell, boy, I thought to myself, did no one teach you common sense? Out loud, I said, "Take me to the barron." 

Barron Rudolfo was meeting with King Menkaura. Neither man was pleased to see me. My page had brought dishonour on both houses. 

"Where is the duke?" I began.

"Under guard by my men in the house." Barron Siasso began. "This is disgraceful, Barron Romino. I would have expected you of all people to keep better eye on your page." I should have, but I did not. 

"I should have his head for this," his radiance rumbled in anger. 

I should have his head for this, but that served nothing. I turned and knelt before the angry king. "Son of Roth, if you hurt his grace, you will have the full force of Marsea in your lands by spring. And I will lead the charge."

"You are in no position to threaten me, barron." he spat the last word. "No man touches the daughters of Roth without my consent."

"The armies of Marsea back me, your radiance. I do not make threats. I apologize for the damage the duke had done. Name your price, and I will see that your demand is met." I could not let the young king hurt Makala. The more time passed, the more I wanted to keep that pleasure for myself.

"Price?" his great stormy voice raged. "You cannot defile the children of Roth, and pay for it with filthy gold. His price is his head. Or every Nievian man woman and child suffering under the Marsean yoke will riot when the hear of this. I will make certain that this barrony will be the first to fall." 

That threat I believed. The charismatic young king had an uncanny ability to start riots and uprisings in Escasaine. It kept the duke's men busy while he conquored territory. Barron Siasso believed him too. "Noble son of Roth, certainly there is a third way out. You are here as my guest. I would not have ever wished this dishonour upon you. My house is yours take anything you wish from it."

The King considered the offer. "Your daughter, Ana."

Barron Siasso paled and stammered. "Sh-she is too young,  a child of ten."

Ana was gifted. We could not give a gifted girl to a barbarian king. It was a foolish move on the Barron's part to offer something he did not wish to deliver. The man was frightened and angry. "Take the duke as your brother-in-law."

The young king turned his stormy gaze upon me. "You are not his father to promise his hand. You both make empty promises."

"I am not his father, your radiance, but I have served the house of Cortan in one capacity or another for most of my life. The duke will listen to me." Duke Erfat and my duchess certainly would. I would have to pay a price for my negligence in keeping the duke locked in my rooms, but they would listen. Duke Griswold would be harder. 

King Menkaura considered the option for a long time. The storm on his face slowly wore itself out. It was not a bad proposition. It would grant Duke Erfat the peace he wanted, and Barron Siasso will not have dishonoured his guest, and the King would make a good match for his sister. "I will keep the boy until I hear the agreement from Cortan."

I could not allow that. The boy was in my charge, loathesome as the duty way. I did not trust him to foreign keepers. We argued and negotiated until we finally agreed that the boy would remain in Barron Siasso's charge, while his young gifted daughter Ana would go to the king until I could arrange the marriage. All three men felt as if they had lost. When the king learned that the wedding could not be held for another year, the arguments began again, the key point of contention being his mistrust of what would happen if Makala had already sired a bastard in his sister's belly. He would neither believe, nor trust that Marsean law made arrangements for these events. To prevent the country from being over run by bastards, children born out of wedlock were not bastards if the father married the mother before the child reached half a year's age. If the princess were pregnant, her child would be heir. King Menkaura would have none of it. The discussion turned heated until I offered myself up as collateral, if the occaison arose. In exchange, the princess would live in Cortan, with her afianced, as was custom. It was not an ideal arrangement, but Duke Erfat had been a fool to try to treat with such a weak, if noble, opponent. Several months in the King's court would help me learn his weaknesses for when the time came to conquor this patch of land.

\vspace{.5cm}

I found Makala in a foul mood at dawn, sitting in a small room at the top of Barron Siasso's house. Both his door, and the grounds below his window were guarded. However dark his thoughts, the could not rival the anger I felt for him. When I entered the room, however, I was completely alone with him.

"Congradulations, your grace. You have made peace between Niev and Marsea."

"What are you talking about?" the boy yelled at me. "What took you so long. Get me out of here."

"You are to marry her radiance, Meshkhenet."

"What?" The boy's eye's grew wide. "That little whore?"

I lifted the boy from his chair by his collar and threw him to the floor. He landed heavily on his ass. "Is that not what you meant to do when you raped her at knife point? She is of royal blood, not a whore."

"She is a barbarian." Makala scrambled to his feet to face me. He was furious, and he would take his anger out on me. "All barbarians are whores." 

My fist flew across his insolent face, his cheek bone cracking under my knuckles. "You will stay here until I can make the necessary arrangements with your great uncle and your parents. Then you will return to Cortan."

"You hit me." he howled, cluching his face. "You hit me on the plains, and lied about it. You hit me now. This is the last time, barron. By the seven names of the Destroyer, this is the last time you lay a finger on me."

"If you will not treat your women with respect, give your men their due. I will get you a healer." I turned to leave the room. 

Makala raged behind me. "You will not leave me here, barron. If you leave this house without me you will regret it."

\vspace{.5cm}
***
%Oct

I watched the fire blaze, hot and high against the dark Autumn evening. Timbers crackled, and the occaisional piece of sealed pottery cracked and popped, sending hot shards of baked earth into the surrounding crowd. A half molten silver chalise had been ejected by the flame, not half an hour ago. Declared indigestible by the flames, it rolled to the feet of a bystander, displaying half a soot lined face of a foreign god. Women wailed and men cried out for revenge at the ill omen. Commander Dielo, at my command, kept the peace, but just barely. If the men and women behind me did not get their pound of flesh, and soon, there would be trouble tonight. 

Firi slipped herself under my arm and wrapped her hands around my bandaged wrist. "Why are you here, child. You should be at the Tower."

"It was my father's temple, your grace. I came to mourn in his place."

I brought my other arm around and held her in a light embrace. "It still is his temple, child. It will be rebuilt. With stone this time, as is proper." Preserver grant me strength to defend these foreign gods, I prayed. My city wanted peace, but it would not have it without our foreign guests.

The first signs of trouble had come the previous day, long before I would have thought it would end in this. A Marsean wife of a Liri trader had died in childbirth. She was a gifted woman, weakly so, a carpenter's daughter. Her name was Carla. She had stayed in the Tower until she was in her seventh year, then been sent home to lead the rest of her life. This was the second child she born the man, and in both cases, he had insisted that she not seek the services of the Tower, but give birth by traditional Liri means. Barbaric, but a husband's perogative, even I had to admit. I doubted, that under normal circumstances, there would have been many Marsean's who would see this as anything but a minor tragedy. The offending husband certainly would not wed any gifted Marsean women. Fathers across the land were already learning to question their prospective wealthy Liri in-laws about their views towards the using the gift's healing powers. The girl's family would grumble, but they would let it go. 

For some reason, that is not what happened this time. The girl's family was not a powerful family, but for some reason, people flocked to their cause. Young healers and older students had walked out of classes and missed sessions with their mentors to march enmass to the Liri quarter, armed with talismans and sticks. By the Preserver's grace, I had happened to glance out my window to see a band of thirty white robes walking through the Tower gardens towards the Liri quarter in the middle of the day. By the time I reached the Tower's north gate, I heard my younger students talking about the walkout to the Liri quarters. I found two girls I trusted, Isotta, one of my own shadows in the infirmary, and Gisele, one of Carlotta's trainees, and told them to get Head Fabia and Master Eulalia to lock down the school. Head Alerio was to send healing wagons and a core of trained, level headed healers to the the Liri quarter as he could spare. Then I ran.

I found Commander Carmine Dielo in his office. He had just seen the band of young white robed men leave the barracks east gate. It did not take much explaining for him to call his black riders and their gifted support to my aid. It was not the city guard, but I was not Timmon. This nonsense, I had thought then, would be stopped now, permanently. The Commander had his men organized by the time I had changed into riding pants and armoured myself. I admired Commander Dielo's efficiency and loyalty. If there must be a third General in Cortan, it should have been him, and not Barron Madriano.

My accusations towards my husband had been no secret by the time Cesara had been rescued. Her victorious homecoming, and the heroic tale of her discovery by General Ciretino did little to sway some, including Carlotta, that my husband lay behind the plot. Officially, a stone mason from Firvona, Vicente Vicenza, wanted my husband's patronage. The duke refused him, so, he acted on his own to kidnap Barroness Cesara to bring her father in line with my husband. The mason lost his life for hs efforts, and the duke washed his hands of guilt. I will believe the story the day I grow two horns and a tale. In the aftermath, Carlotta continued as my healer, her husband became my husband's third general, and a rift appeared in their marriage that seems to grow and shrink with the wind.

I put away the thoughts of Generals and Madrianos and my husband from my head. They would be involved in this matter sooner or later. For now, I needed to ride with Commander Dielo at the head of his Black Riders. 

The fighting had begun by the time we reached the Liri quarter. The Liri community was young, it had only been eight years since Duke Ergino had signed the treaty that allowe settlement in these lands. There were perhaps two score grown men living in the Liri quarter, and few young adolescent boys to make trouble. There were twice at many Marseans, healers and ungifted alike. Still, we nearly matched them in numbers of horsed men. It was a heavy handed show of force, but I did not know who lay behind these riots. I meant to show my invisible opponent, possibly my husband, that I was serious about this nonsense ending. 

Some of the rioters stopped fighting at the sound of hoof beats, but not all. Stupid, hot headed boys, and angry scared men. Those would be my problem in the end. The black horn blew low and loud inside my city walls. It was a terrible day. That horn strikes fear into the hearts of our enemies where ever it was blown, gives hope to our friends. Which is which when it is blown against our own people? 

The fighting stopped. I dismounted and walked between the two crowds of my citizens, Marsean and Liri. They parted before me, and stayed apart. I found a white robed student slow to crawl out of my way. I stepped on his fingers, feeling them crunch under my steel clad heel as I went. This was how sergents trained young recruits. If my students wanted to be serve the Destroyer, I would treat them as such. I made certain to also kick a Liri man in what looked like a broken leg. I had to appear fair. He would be healed later. I got some satisfaction out of the act, though I wished it were Sephiyat Nagiyi I kicked. I had trusted that man to keep his people peaceful. This was how he repaid my trust.

"Beloved citizens, old and new." I began, then took it back. "Your behaviour today has not put me in a benevolent mood. What in the name of the triumverate and the seven sacred spheres has caused the best men that Lir has to offer, the brightest students of my school, and the boldest pick of my subjects to go at each other's throats like this?" I was not an auspicious beginning. Both sides started talking at once. The black horn blew again. I picked one representative from each side to tell their version of the events: Sephiyat Nagayi from the Liri, and Healer Christoph, the leader of the group of white robes that had left my Tower. They spoke over each other, contradicted each other, argued accused and bickered, but eventually, I learned the tale about the deceased Liri wife. In the mean while, the gifted fighters supporting my men took wounded from both sides away to tend to their wounds, as they would see to the wounds of every man loyal to Marsea, on any battle field, until the healing corpse could make them whole.

"Sephiyat Nagayi, need I remind you that the Liri community exsists in Cortan at my pleasure. Continued rioting like this, and I will have to send the immigrants home. As there is no treaty allowing for those born in Marsea to settle in Lir, your women and children will have to stay behind. You will keep your community under control from now on, or there will be consequences." I let the unhappy murmurs from the Liri side, and the excited whispers from the Marsean side die down. "As for you, the men with whom I share the country of my birth. This is the second riot in seven month. The first occured while I was marching for Marsea's, your, glory. I was not pleased by my citizens betraying my trust, and breaking my peace." 

Somewhere, a youth called out "They started it."

"I am aware of who the duke put at fault for that incident. However the duke and I do not see eye to eye on all matters. As I do not have the names of all who attended the first two riots, any man born on Marsean soil found at a third riot in the Liri community will face charges of breaking my peace at the highest possibly penalty." The angry rumblings switched sides.

When they paused, I continued. "There were at least three groups of men who came to this conflict. Who speaks for the rest?" A  middle aged man name Guiseppe, a tanner by trade, stepped forward and declared his hatred of the gift spoiling invaders to the cheers of his compatriots. "I commend you for your leadership and innitiative, tanner. You seem to have quite a following. Commander Dielo, please arrest these three men for questioning." 

The crowd errupted in anger, but the black riders kept the peace, and I did not need to draw a weapon. I waited my soldiers to scatter my citizens and for the healing corp to arrive with wagons and healing tables. With the wounded attended to, I rode with Commander Dielo to the White Tower. The autumn sun hung low in the sky.

As I expected, Alerio awaited my guest at one of the smaller rooms reserved for the public in the Tower. I had not expected General Vittorino Galderan to be with him. "General, Head, thank you for anticipating my need." I sat down before my small counsel to review and learn.

Carmine provided the details of the encounter while a student provided refreshments. The room we gathered in may be less comfortable that Alerio's office, but the fare was clearly of his choosing. Bowls filled with Cortan's orange wedges and Lir's pomogranite seeds sat before each of us, a platter of bread, a hard cheese from Voltain and dried salted fish from Lir in the center. Alerio would put Sophia to shame in his ability to merge Marsean and Liri food.

"This has to do with the walls," Vittorino began.

"The ones that Mason Gallo is postponing?" I asked. My husband, failing to move the Liri quarter outside the city walls wished now to build a wall around them. It stood to reason that these riots would only support his argument for protecting his Liri citizens.

"Yes. I am not a good a man for finding plots as Barron Romino, but it seems odd to me that this riot come so soon of the heels of Mason Gallo's sudden labour shortage." 

"He has been dragging his heals for two weeks," Carmine objected. "Why would the duke act now?"

Vittorino shrugged. "Perhaps he has just lost patience. Just as he suddenly lost patience with Gallo a few weeks after he suddenly found his nerve again, and tried to hire the traitor Vicena to replace him."

Color drained from my face so quickly that I frightened Carmine. "Your grace, you have gone pale."

"I am fine," I shook my head. Simply an idiot. There were times that I wanted Timmon by my side, his meddling and all, just to help me see the nose on my face. Mason Gallo had sent his family to Belisal shortly after the sudden searches of Marsea's Tower had occurred. That timing was purely coincidental, other than the fact that I had seen him meeting with Head Alerio often during the same period, and put it up to repair work, blind as I am. Three weeks later, Cesara disappears, and there is some truth to my husband's story, though I will eat my toungue if he did not know something of the matter ahead of time. "Talk of motives will not help us much now, I think. No one here thinks that the trouble has been put off for long. What do we do now?"

The conversation moved on to a critique of how I handled the situation, Vittorino and Carmine thinking that I had come across as determined and strong as I had wished, Alerio objecting that I had given neither party reason to love me. The concensus emerged that I had little reason to believe that while Sephiyat Nagayi may still trust me, no one else there would, but they would obey. Carmine proposed posting regular guards in the Liri quarter. I objected that that would be as good as a wall. Vittorino suggested that I visit the community with some regularity. That seemed a better idea, though Alerio pointed out that I was already in a poor position to be showing favouritism.

The speculation for the next riots varried from six months to a year. Only Alerio surmised that the next act would not be a riot, but something different. We were all assuming, he pointed out, that the men we arrested were truly the men behind the event, a fact we knew to be false. He was right of course. Carmine took his leave to find out what he could from his men doing to the questioning. Vittorino left shortly thereafter. It had long grown dark outside. I picked up my wine goblet, brought my knees to my chest, but my chin on my knees, and brooded. By almost any measure, the day had not gone well.

"You were hard out there." Alerio finally said.

"You said that already."

"You are not nearly so hard in here."

I have you to back me here, I wanted to say, but did not. "I do not have to fight my huband's will here."

"Your teachers respect you more than your subjects do as a result."

The infuriating old man would neither leave me alone, nor give me clear instructions. "They are lucky. I do not aim for respect. I aim for fear. Which reminds me. You knew more than you let on about Cesara's kidnapping."

"I did not know that helping Mason Gallo's family resettle would lead to the child's disappearance."

"But if you did, it would not have stayed your hand," I said in disgust. I trusted this man, and he did nothing but toy with me, holding back information to feed me as only he saw fit, long past the time when it could do me any good, or him any harm. "That is not what I asked."

"Raising your voice will not help you here, Nisrita. You would do better to have my respect. You will not have my fear."

Curse his half answers and his hidden games and mysterious motives. I had other men to contend with today. I uncurled my legs and put down my cup. "I should go. Barron Paulis will want my head for bringing out the Black Riders instead of waiting for the city guards."

I rose and fastened my black riding cape. My mood had turned as dark as my attire. 

"It would not have helped," Alerio said as I left him. I turned to face the speaker. "If I had told you what I know, it would not have helped your mood that day, nor would it have brought the child home any sooner."

"Alerio," I said, dumbfounded at his lack of understanding. "That was a month ago. At any point between then and now, it would have made all the difference in the world."

\vspace{.5cm}

I was right in predicting Barron Paulis's anger at my showing such an overwhelming show of force. We struggled and argued over that point, as if the riot were my fault every time I mentioned that he should investigate who lay behind these uprisings, and at every other possible moment when I did not make him look the hero of the hour. I did not recall seeing him, or his men of the city watch at the Liri quarter at all while the trouble brewed. In the end, I lost my battle with him that day. He would not investigate, I knew. Or if he would, he would do so with the same lackluster effort with which Mason Gallo build the wall for the Liri enclosure.

I arrived at the Tower the next morning to find that Alerio had canceled classes for the day. He wanted myself and a panel of Masters to interrogate the thirty students who had attended the riots to find out who had instigated them. It was a standard proceedure after students in the school embarrass the Tower in any significant way. I have seen two or three inquests in my time as a student. They are dreary affairs, where all the students are kept in their rooms for at least the morning, if not all day, waiting and wondering if they would be called, or angry and frustrated at their classmates for the trouble they caused. I considered laying into the head about having the common courtesy of letting me know that he interfered with my schedule. The castle was less than a mile away, and I the head of his school. It would not have hurt him to tell me what day he wished to conduct his investigations. But this was his Tower, and Alerio enjoyed being illusive. It was not worth a battle so early in the morning. 

We assembled in an empty class room, Head Fabia, and Alerio, myself and Master Eulalia and Master Leon. The morning started out promising. Most of the younger students pointed to one of five young healers, Christoph, Juaquim, Leonardo, Dante or Frederico. Between Master Leon and myself, we confirmed that it seemed plausible that the youths spoke the truth, given our knowledge of who was friends, or in classes with whom. Next we called in the five perpetrators, and my head began to hurt. Christoph claimed that he had heard of the call for a riot in town from Dante, who claimed that he had heard of the riot from Leonardo, who claimed he had heard of it from Juaquim, who had heard of it from Frederico, who had heard of it from, who else, Christoph. No ammount of calling back, cajoling, threatening or pleading could get the five to change their stories. All five eventually admitted to instigating the younger boys, and accepted responcibility for their action, but they would not change the rest of their story. Finally Master Eulalia had a brilliant insight. She called all five healers at the same time, physically arranged them in the appropriate circle, asked them to point at all the men in the room who had told them about a potential riot. Head Fabia and I could not control our giggles at the ridiculous sight of five men in their early twenties pointing in a circle at one another. Eulalia, more experienced at dealing with youthful antics than I, then asked the men to raise their other hand if anyone not in this room had notified them of the event. The five men stood in the middle of the room, staring at their feet or awkwardly at each other, clearly aware of how ridiculous they looked in their current positions, and of how unbelievable their stories were, but not a single hand went up. Eventually, Master Leon lost his temper. He had seen to each man's education as their school's head, he said, and he had taught them all better than this, if not in honor or common sennse, then at least basic logic.He locked the boys together in a room for an hour to settle their stories. Otherwise they would all be released from the Tower.

The meeting adjourned and I found myself in Alerio's office with Fabia. "That was a brilliant maneuvre on Eulalia's part." I said.

"It was least quite funny," Fabia concurred, giggling again.

"It showed them that they have to stop wasting our time." Alerio said.

And idea formed nebulously in my mind, pulling and tugging at things that did not make sense. "Someone has put them up to this," I murmurred.

Grey haired Fabia stared at me, laughter still dancing in her wrinkled eyes. "That is the general concensus."

"No, the consensus is that one of them is the ring leader, there is a difference. Alerio, they are your healers. See if any of their recent habits have changed, recent expensive purchases, extra trips to brothels, gifts to their families, that type of change."

"You think they have been bribed?" Alerio asked. "These riots are not your husband's style."

"We have been assuming that my husband is behind this directly, and not someone working for him, as was the case with the Madriano kidnapping. We have also assumed that you have build a Tower where he cannot establish a foot hold. We may be wrong in both cases."

Alerios thin lips tightenend. "That is impossible. I know each of the men in this Tower. They are not corrupt."

"I told you, Alerio" Fabia teased at the same time.

I did not have the inclination to untangle Alerio's plots or Fabia's cryptic comment. Alerio may hold strings on all of his men, or he may think they all respect him too much to act against his will, but as Carmine points out when disgruntled and in his cups, not even the Black Riders are without their trouble makers. "Prove me wrong then. Nothing would please me more. I am going for some air."

"You have the remainder of the inquest this afternoon," Fabia objected.

"I have most of an hour. I won't be late."

I went to my office and changed into the clothes I had come to the Tower wearing. Following last night's discussion, I decided to start my routine of visiting the Liri quarter today. I had brought my riding pants and marks of my station with me. It would do no one any good if I appeared as a simple master. I put on the riding pants and thought about the ridiculous scene with the five men. They were trying to waste our time. Someone must have paid them very well for them to put up with that humiliation. Why would they do that? Male healers in their prime are renowned for their prickly egos. There were far better ways of avoiding their guilt than that childish display. 

Because they were trying to waste your time, Timmon's voice whispered in my head. My stomach tightened in fear of the possibility that the idea was correct. I dropped my shirt and reached for my armour. I wore the shirt over it, put on my jewels, and left for the Liri quarter, praying I was simply paranoid.

Anyone familiar with the Tower's practises would have known that there would be an inquest such as this after the riot. They would have known that the head of the school would have been involved. Could they have planned an massacre at the Liri quarter on a day when the I, the duchess, could not interfere? Alerio had warned that the next catastrophe would not be a riot.

The Liri quarter was deserted when I rode my palomino through it. Sounds of wailing and chanting came from the temple, along with the smell of burning wood and buffalo butter. Of course, Carla's funeral. The Liri community was small, its temple large, built with the expectation that the immigrant population would grow with time. It would make sense that everyone would attend. The enclosed funeral pyre, on the other hand, seemed pure foolishness to me, but my role as duchess was not to judge Timmon's religion, only see it respected. The Liri showed much more sense in Deyalorn. The Liri sent the spirits of their dead to the sea. They walked their loved one to the nearest river, put them in a boat, and built a pyre over them, to ensure a quick destruction of the flesh to free the spirit within. The pyres I've seen floating along the Tulsi were beautiful, if ephemerial, tributes to their dead. Cortan has no rivers. So the Liri temple had built in it an open enclosure with a large pool to place the body, and the boat and the pyre. The pool would be cleared after the services, and some of the water  and charred remains reverntially taken to the Velta by the next party to head for Lir. I've often asked Sephiyat Nagayi what he would do if the fire got out of control, but he has assured me that the room was large, and the pool of water protected them. 

Foolish as I thought the practise was, I heaved a sigh of relief. With all the men engaged in a religious ceremony, it would be difficult to make trouble today. I paused at the house of the temple's silver engraver. His Liri wife had just visited the Tower three months ago to give birth. 

I found the woman well, and the child strong. She felt that I had insulted them gravely by arresting their priest yesterday. I apologized and explained that I could not show favours during the riot, which she seemed to understand, but I gave that little weight. She was an ungifted woman who did not walk much in the word beyond her hearth. I left her house to return to the Tower, deciding to return when the community had finished grieving. I had an inquest to attend.

I walked my horse towards the temple, intending to listen for a moment to the exotically rhythmic chanting of their services, when I heard the cry of fire. I cursed Sephiyat Nagayi for the foolish design of his funerial chamber and ran to the scene. 

I could not believe what I saw. I could see the toungue of the funeral pyre flickering periodically above the roof where the services had been held. The wide veranda the surrounded the wooden temple burned, forming a wide ring of fire, burning down the stairs and eating at the outer walls. Someone had started this fire, it had not been caused by a stray ember from the pyre. The sounds of chanting and singing had been replaced by that of screaming and panic as men women and children stood and pushed at the door for a way out. The ground lay three feet below them, through a pile of burning timbers and rubble. This would end in stampedes and death I thought, and took off my shirt. The leather of my armour underneath would provide me with some protection against the flames, but my arms were bare. It would have to do. 

I watched three brave men jump to the ground through the flames, one falling as he did. His companions dragged him out of the flames and beat down the 
flames on his leg, and helped him hobble to his feet. "You two," I called. "Can you ride?"

One man nodded. "Take my horse to the barracks as fast as he will run. Here is my signet ring. Tell Commander Dielo of the Black Riders of this fire. He will know what to do." A woman screamed and I watched a child fall, slipping under the pressure of the crowd within. "Hurry."

I turned to the other, unbearded still, barely more than a boy. "Heavy cloaks and blankets, wet if possible, as many as you can, quickly."

When the wounded man found himself alone he turned to me and asked, "Can I help, your grace?"  I was shocked for a moment by his lack of accent. He was Marsean, not Liri. What was he doing at a Liri funeral. A member of Carla's family, perhaps?

"Yes, guard these." I tossed my shirt, jewels and talisman on the ground by his feet. I did not want any metal touching me. "And care for the wounded I bring out." I ran into the flames to find the child. 

We call the heat of the gift the Preserver's fire. Standing in those flames, searching for the child, I realized that what I was so accustomed to pulling out of the air and through my body was nothing like fire. Fire, real fire, is far more hot than it is searing. The flesh recoils from the hot air, eyes tear and close, it comes as wave after wave of impossible heat, dessicating a body, cooking a person from the inside out as the air enters the lungs. The preserver's fire, in comparison is tame. It sears, but is not hot. It is staid, not greedy, it, and the pain can be controlled. 

The flames and smoke around me rose and flickered, only allowing me glimpses of the fallen child. He was frightened when I reached him, badly burnt and trying to find a way up the burning stairs to his mother. I grabbed him by the waist and threw him over my shoulder. "Stop kicking" I barked, and navigated my way out of the flames. 

By the time I emerged as few women and servants stood with cauldrons of dishwater and blankets for me. A few other men had tried to run through the flames to freedom, and now used the water to damp the flames. "Clean water for the wounds," I panted, putting the wailing, struggling child down, and fresh sheets for bandages. 

I soaked a blanket in water and went back in. I put it over my arms for the moment. It would protect the next child from the fire on its way out. My arms had blistered already, it would be hard to carry another child. There was less of a push at the door when I approached, the adjoining wall had caught fire, flames licking at the cracks in the charred wall, orange toungues slipping inside to desecrate the santified ground. People stayed back. How had the fire spread so quickly, I wondered. This must have been arson. "Childen!" I called out over the flames. A thirteen year old youth jumped towards me, knocking over a seven year old girl in his way. I helped him up. "You are too big to carry," I growled. "Follow me out on your own feet." The seven year old would not jump, so her father threw her towards me. She landed on her hands and knees, her silk dress immediately becoming a sheet of flame. I threw the blanket over her, put out her flames, and tossed her on my back. By the Preserver's grace, or the mercy of whatever god she worshipped, the child held on. The cowering youth followed.

I went in again, this time bringing forth a six year old boy on my back, and leading a fifteen year old girl carrying a toddler in her arms. Lines of men in black leather armour filed past me as I picked my way amoung the embers. Commander Dielo's men had arrived. The Riders were better trained than I to ignore physical pain and trust in their healers. My feet had blistered. I could not run with the weight I carried. 

I put my burden down and turned to douse the flame on the girl's skirt. My palms left bloody prints on her now nearly naked and blistering body. It was a pity she was not gifted. She had not flinched as she had followed me, though her skirts were on fire. With that type of courage and discipline, she could do well serving in the healing corpse.

A Liri woman came to help her to the rest of the wounded, and I turned for my fourth trip into the flames, only to stumble into Commander Dielo.

"Move, Carmine," There was only smoke, fire and children on my mind. I could hear the sound of screaming women over the roar of the flame. 

"You are wounded, your grace. My men are fresh."

I saw a black rider carry a pregnant Marsean woman out of the temple. Her hair was on fire and she was screaming. He was quickly followed by another man carrying two children tightly wrapped in leather cloaks. There was little I could do compared to these men. "How many gifted do you have?"

"Twenty, it will not be enough. More will come, and I've sent word to the Tower."

"Help me to the wounded. Any of your men have first claim to my healing."

Carmine bent down and scooped me gently in both arms. He put me down by my talisman, and introduced me to the man leading his gifted support. Then he left us to the task of organizing the wounded. We saw to my wounds first. The buckles on my shoes had melted and seared into my flesh, large bits of flesh and skin clung to the leather when the gifted fighter Jacopo cut them off. The rest of my feet and legs werered and blistered, but relatively whole. Jacopo bandaged my limbs as I saw to a badly burned toddler. He had breathed too much smoke and was hazy and disoriented as a result. There was little I could do for that. I saw to the flesh. My role here, I reminded myself, was different than it normally was. I was to act as a gifted fighter would, stabilizing, and even then, not completely. Simply long enough for the healing corpse to arrive. I tended to the worst patients. Gifted fighters are trained specifically to stop bleeding, internal and external. It is one of the simplest and most effective ways to save a man's life in battle. Burns are different, both more difficult to treat and more subtle. The gifted support poured through their strength on those that needed their attention, not knowing how to tend to the most difficult cases. 

I was on my second patient when the healers arrived, the pregnant woman that no one wanted to touch. Her head and neck were burnt, but the rest of her seemed whole. Her face would be brutally scarred if we did not tend to her. I explained to her the risks of healing while pregnant, and told her that if she still wished to be treated, she should appear at the Tower, with her husband, and appeal to the Mother's Balm, giving my name. Carlotta would do this for me, I was fairly certain. The woman's wounds were no where near her womb. It was not that the Women's Tower categorically did not heal pregnant women, we simply did so with extreme reserve. 

I had sent her off to be bandaged when I heard the sound of fresh horses. I looked up to see four of Fabia's healers and one male student approach. All of the women were in or just past their prime, some of the most powerful in the Tower at the moment. The student was known for his skill. The wagons would be coming, they explained. They had been sent forward to help. I wondered what it meant that Alerio did not send his men. Had I been right about a my husband gaining a foothold? Had the man who had planned this been lucky or clever? Did he know Alerio well enough to know that the inquest would be held on the day of the funeral, a perfect excuse for arson? If so, how long in advance had this been planned? The thoughts made me sick. I could not answer them. I could not pretend to be Timmon, no matter how hard I tried. I turned to the fresh healers. I explained the different way in which they needed to serve in this setting and put them under Jacopo's command. The student I kept for myself. I had too many exposed wounds. One of them would lead to a fever if someone did not heal me.

Within an hour, there was not much left to do. Barron Romino had come to see the blaze moments before the wagons arrived. Furious with his delay, and the lack of the men of the city guard supposedly still loyal to me, I chased him off to speak to the Commander. I had a Black Rider with a badly burnt shoulder and hands to tend to. When the building started to collapse, Carmine called his men back. The Liri community stood and watched.

"How many did we lose?" I asked him when placed himself my side. 

"The priest says that there are three people unaccounted for. Out of over eighty people in the temple, that is not so bad."

"They will want blood for this."

"Not today, I think. You have saved many lives and limbs today."

I shook my head, watching the wagons take away the wounded. For every two wounded on the wagon, one had refused treatment beyond bandages. Some of those would not survive the night, and then the Liri would leave their homes to trouble the rest of the city. "How is Barron Paulis?"

"He is going to get the men from the city guard to help keep order here after we are finished. He too thinks that there will be a call for blood."

The situation was bad, and I did not know how to mend it. I did not understand these people or their needs. I did not know how to console them, or what they would want from me. Their ways were foreign. I respected their traditions, but I did not understand. The one man who could help me was too far away. "Help me stand, Carmine. I should be with them, watching their temple burn."

Commander Dielo helped me to my feet. "If you would permit my saying, your grace. I do not trust your barron to keep the peace. Timmon would have been better."

"Barron Romino is not here," I said, more sharply than I needed to. I had not sent the barron away on a whim, though I sorely regretted the decision now. He would know what to do with these Liri who would not let their loved ones be saved, but who boiled over in anger and grief when they died.

As if on cue, Hinata came running through the crowds. "The Sephiyat's daughter-in-law has succumb to her wounds," he gasped. "They say she was pregnant. The Sephiyat's son wants her avenged." 

"By your leave, your grace," Carmine said, and disappeared into the smoky evening before I nodded. 

Hinata helped me to high visible ground between the temple and the crowds. I had nothing to say to them. I simply wanted to be seen in solidarity mourning their gods and this savage act. Firi found me there, and slipped herself into my arms. Her father would murder me if anything happened to her here, I knew, but I did not have the heart to order her back to the Tower. It had been a long wearing afternoon. The warm brave child by my side gave me strength. As she stood with her head against my chest, she reminded me of my softer duties in life.

"The Head is coming," Firi said suddenly.

I looked over the dusky fields to see Alerio, Barron Paulis, and a ring of city guardsmen escorting a chained healer. I could not recognize the man in the smoky light. I waited for them to approach.

Alerio spoke first when the party ascended the knoll. "Greetings, your grace. I owe you an apology. You know Master Cretius. When I consulted him yesterday, he felt strongly that we hold the inquest today. As I had no strong opinions on the matter, I relented. He is yours to do with as you see fit."

I looked from head to barron to prisoner incredulously. Cretious stared dumbly at his feet. He was a young master, only a few years older than me, soft and round, but tall and sturdily built. He looked as if he might have made a good warrior if he had not been so gifted. I seemed to remember a story about his sister, or a cousin being courted by a Liri merchant, but the details were vague in my head. I had no strong reason to believe that he hated the Liri any more than any other man. "Why is he here and not in a cell?"

"The men and women behind you want revenge, your grace," Barron Paulis explained. "This is the man they should take it out on."

"No!" I said firmly. "Marsean justice is not conducted without a hearing before the gods and king."

"But the Liri allow for immediate public stoning when the crime is great and the guilt is clear, your grace," said Alerio's calm, quiet voice.

I stared at him, horrified. The guilt was not clear. Not to me, and I would be his executioner. "This is not Lir." Death by stoning was considered a severe punishment, reserved only for certain miscasts and witches. Even traitors and heretics faced exile.

"It will keep the peace tonight, your grace," my barron persisted.

"So will your men." I said tightly. My barron shrugged. He would not guarantee anything, he had no love for these people, and I could not ask the Black Riders to oppose my city guard. I looked to my mentor, stoic, silent and inscrutible as always. What would you have me do, I wanted to ask. But he would never answer that question. He simply stared back at me with sad wrinkled eyes. Timmon, I begged. Your voice brought me here to keep today from being a slaughter. Do not desert me now. But his god, Rishiki of the Four Winds, neither heard my prayers, nor ferried my message. At this dark moment, the decision was mine alone to make.

"Rider," I called to one of the men guarding me. "Your helm."

The disciplined man took his head gear off obediently and passed it up. I prayed I did not see him in the Tower later tonight because of this. I took the half helm and started fastening it to Master Cretius's head.

"What are you doing?" Barron Paulis asked.

"We will allow for a stoning, but not to the death. Your men will see to it that the Liri do not kill him. If he dies before he reaches the Tower, all of their lives are forfeit. You, Barron Paulis, will give the order. Gather what support you need from Commander Dielo, then begin." I would not have him go to my husband with news of this event and twist it against me. He would be just as responcible as I for this foul deed.

The Barron balked, then bowed. As he walked away from me, it struck me that he wanted to do this. He had obeyed too quickly. "By your leave, your grace." I turned to see Alerio bowing. "I will not stay to see this bloodshed. I have a house to clean." Did he disapprove of my choice? I could not tell from his tone, and he did not wish to let me know. No. I had made this choice alone. This much was certain, he would not want me questioning myself now.

I watched my friend's stooped shoulders descend the small rise. This morning I had thought that Alerio and Carmine were the only two allies that my husband had not touched, hoped that his failure so far would mean that he could not touch them. The Tower had barely escaped. How long until Commander Dielo and the Black Riders fell?

"Your grace?" Firi interrupted when only she and I remained atop the knoll.

"Yes, child?"

"There is something you should know. Barron Paulis wants these riots. He uses them to further his other goals. I would not trust him."

I stared at the girl. "What do you know?" I asked sharply.

"I do not know of any plans, your grace. I" the eleven year old girl gathered her arms about her budding chest and bit her lips. "I overheard the barron speaking to my father after the first riots. It is simply the impression that I had."

"You spy on your father?" This could open a new set of problems that I did not have the strength to deal with at the moment.

"No, your grace, I would never repay him with that crime. I was on the balcony of my room when I heard him talking to the barron."

I looked at the girl for a long moment. I could not tell if she was lying, or merely frightened by the knowledge of what she had overheard. It was not like her to be so ill behaved. "Who else knows?"

"No one, your grace. I... I was frightened. I did not tell anyone until now. Please believe me."

"Your father will hear of this," I bit off. Timmon needed to get his house in order. First my Griswold, then his Firi. The man had too many secrets to guard to be so lose with his children.

"Yes, your grace." 

Barron Paulis, what would I do with you? You were involved in the first riots, were you? Is that why our men are never around when we need them? You are not the man I thought you to be, and your wife not the faithful companion. I would send you back to Deyalorn, if I could. If I did that, how safe would Lyca be? Do I trust my husband's hold on your wife, or does she hate me more than him now? You are a snake in my garden, but I cannot remove you.

I saw the Barron walk before the crowd as a handful of black riders joined the handful of guards he had brought to keep Master Cretius alive. A makeshift stretcher lay nearby. The four men guarding me at the bottom of the hill moved up to form a tight circle around myself an Firi. 

"Citizens of Marsea," the Barron began, "you have suffered a heinous crimes against your gods and your community today. Bound before you is the man who planned this atrocity." The Barron then began to speak in great detail how Master Cretius had put rags soaked in lamp oil and buffalo butter under the stairs, veranda and in the rafters of the temple with the cold and calculated plan to burn the building down while most of the community mourned one of their dead. He claimed that Master Cretius had wanted nothing short of turning their temple into a mass pyre. I cursed the barron, then myself for allowing him this pulpit. He intended to frenzy the crowd. As Firi said, he wanted to use this to his own advantage. I had no idea what that could be. Then I shuddered as he raised his arm and threw the first stone. 

I wanted him gone from my court. I wanted him dead, I wanted his eyes gouged out and his toungue set on fire. How dare he resort to Liri brutality as a citizen of Marsea, a member of my court, and a representative of my people to this community. He made me sick. 

Firi whimpered and buried her face in my leather vest. "Don't watch child. It will be over soon." I, on the other hand, had to watch. I had condemned this healer to this fate. I could not turn away from his suffering.

Stonings, as a rule, do not last very long. In a quarter of an hour, Master Cretius lay limp and bleeding on the charred ground, surrounded by Black Riders and city guard. The rest of Commander Dielo's men ordered the Liri back to their homes. In another quarter hour, the mounted archers held the streets. My four guards helped me tend to the victim. He was unconcious, his spine broken near the base. The man may never walk again. He had heavy internal bleeding, but his helm had held. I worked with a gifted fighter and Firi to stabilize him enough to move him to the Tower. The process took nearly an hour and sapped me of all my remaining strength. I slumped heavily against a tree as men carried Master Cretius's stretcher away.

"Your grace?" my unhelmed guard asked. "Should we help you home?"

I peered through the smoky light at his young, loyal face standing over me. I could depend on these men in black around me for the time being. Given time, Alerio would clean his house and I could depend on the men and women in white around me as well. As for my house, I did not know how to clean the castle of my husband's friends and Barron Paulis's allies. The thought of returning to the castle suddenly filled me with a sickening dread. "No, take me to the Tower. I will see the young barroness home."

"Very well, your grace." Firi rode before me, holding the reigns loosely in her hand. My palomino found himself guided by my knees, surrounded by black companions, mounts of the mounted riders. 

I entered the Tower and walked its cool white corridors, barefoot, Liri style. My burnt ankles would not permit me to don shoes for quite some time. I walked past the dormitories, bid my ward goodnight and spoke to the steward. I made my way past my cold dark office, and that of Alerio, the up the winding stairs, up and up and up, until there was no more up to go. I had seen so much violence, and so much anger today. I had seen far more wounded from a minor skirmish on the field, but these had been my people, in my city, burnt and betrayed by my own kind, not wounded heroically on a distant plain by a barbarian defending his home. I had been betrayed by my allies and hoodwinked by my  underlings and aided by young students with illicitly obtained information. Was this the life of politics my head wanted to prepare me for? I wanted no part of it. I wanted a workspace, access to kennels and students and the freedom to support my army. My husband and the Destroyer could take everything else. Yet I climbed those cold white smooth triangular stone stairs. My feet knew my desires, unsteady and barefoot though they were, covered with new uncalloused skin, freshly healed of blisters, even if the rest of me denied it.

Lapo opened the door when I knocked. "He is expecting you," he said quietly.

I followed the boy through the large round white room, uncertain of my decision, yet completely clear that every step I had taken to reach this point had been correct. I climbed a final set of narrow winding stairs onto a soft green and yellow rug. Alerio looked up from his book with a long weary face. He shut and placed it carefully on the table beside him, before speaking "I've been expecting you."

"I heard." I said wearily. "I've made arragements to sleep in the dormitories tonight."

Alerio gave me a wide melancholy smile. "I expected no less." He rose and removed the cover on a silver goblet set with small white stones sitting across the table from his chair. He had already poured my drink. Its mate stood by Alerio's seat, within easy arm's reach, just beyond the newly closed book. It too had waited, it seemed, for my arrival, for something to break the solitary monotony of this nightly ritual of reading by the fire. Alerio offered me his arm for the last few steps of my odessey to his quarters. As I leant on him for support, I felt as if the room had shrunk since the last time I had been here. Today, it was not a comfortable sitting room to entertain the powerful and the gifted. Today, it had space enough only for the two of us. Conspiracies, fires, stoned colleagues and loyal soldiers would all have to wait outside. "Come, rest your feet. You have had a long, hard day."

\vspace{.5cm}
%Late Oct

"You should not have let Master Cretius live, Nisrita." Alerio said when he saw my head appear around his door.

I groaned internally and stepped inside his office. I had come, as had become my habit over these past two weeks, to wish him good day before riding to the Liri quarter, then home. If Alerio had wanted me to allow Master Cretius to die, he could have told me so the night of the fire, or killed him himself in the infirmary. "It is not to late to change that, Alerio. It is better I acted as I did. Had killing him been a mistake, not even Griswold's findings would have helped."

The head's lips remained unmoved by my wit. "His cause is mobilizing men against you, as further proof that you are a traitor to your own kind."

"Then fix it, Alerio." I snapped. "This is your Tower, they are your men. If you want me to rule as I see fit, you will need to control your men according to my vision."

That, of all things, brought a smile to the old man's lips. I would never understand him, the whithered sphynx, not to the end of my days. He rose from his desk, walked to where I stood, and took my hands in his cold fingers. "I will see what I can do." Then he raised my hands to his lips, not as if he were greeting his duchess, not as if he were greeting a lover, but somewhere in the middle. Alerio was a subtle and intoxicating man, as suave and alluring in private as he was quietly manipulative in public. There was little I would not do for him once he had led me to the position from which he could ask. "Good day, Nisrita. I will see you in the morning."

The sun was well past its autumn zenith when I rode my horse out the north gate towards the Liri quarter, nursing both the warmth of Alerio's parting words, and the bitter feeling the rest of my parting interview had left in my mouth. My encounters with the head frequently left me both thrilled and conflicted of late. For all his warmth and kindness, I hated his cryptic nature. He had his schemes for the Tower that I would never be a part of. He had his schemes for the nation that he was determined to draw me into. He cared for me as no one else in Cortan did, he aroused my curiousity and tempted me,yet I wanted to hold him at arms length more than I did any of the men loyal to me. Life with him would be full of secrets and unspoken instructions, only to be followed by conflict and quiet disapproval afterwards. 

The aftermath of the fire had been complicated. A half dozen people had died, three in the fire, two from refusing treatment, and an old man the Tower could not take. A dozen more would retain scars on their hands or feet that would remind them fovever more of that vile day, and drive them to anger all the faster the next time. I was currently favoured by the Liri again. Commander Dielo and his men spread the stories of my running into the fire bare armed to lead children out throughout the barracks, so that I was wildly popular in the barracks. My Tower did not love me, but for the last two weeks, they seemed to love the five young healers who had stood in a circle pointing at each other less. They stood trial before my husband for the fire, and were, of course, exonerated. Mason Gallo could not both fail to find men to build my husband's wall around the Liri quarter, and find men to build the stone temple, so construction had begun in wood again. Fights broke out all over the city over whether the Liri community should be allowed to remain, given their attempt to kill a healer, as well as what was not being called the murder of a gifted mother. Somehow, the city guard were never around to keep the peace. Eventually, I commandeered a handful of cavalry men Carmine swore were not corrupt to patrol the streets as a second watch. Two days later, Barron Paulis found the men to keep the peace. And now this about my treachery to my own kind. I could not win, whatever I did. 

All of this is to say nothing of the letter I recieved three days after the fire, asking my approval for my son's marriage to a Nievian princess. Even if I had the inclination to permit such a marriage, I could not give my approval now. With my husband's help, my actions at the fire had earned me a reputation for loving my foreign citizens more than my own. Tying my family with that of Niev's new upstart king would only weaken the faith my subjects had in me. What had Erfat been thinking, negotiating with his barbarian neighbors. He was the caretaker of my son's domains. He should be crushing Makala's future opponents, not bedding them. Timmon had been very clear on what Makala had done, and that he would be held by Barron Siasso until the marriage could be fully approved. Timmon had manuevered in far more delicate and complicated situations. I did not understand why he failed me here. He should have directly brought my son back to Cortan, not left him in Barron Siasso's care. I had made a mistake in sending Makala with Timmon. I should have directly sent him to Erfat to learn about his new lands. My husband reacted to this chaos by permitting the marriage, and appointing Lucretious to be the next duke of Escasaine. Given the way he had acted with Niev's royal family, I had no desire to prevent Makala's demotion. My second born would be home in two week's time, with a bruised ego, and a viscious desire for vengance against his younger brother. I would have a long road ahead of me, to make peace between the two. I did not know what to do with him. If Makala could not be brought to heel in any setting other than as Carmine Dielo's page, what was to become of him?

I stopped my horse suddenly at the unexpected sight of movement inside Barron Romino's house. I stared in angry surprise at the commotion in his yard.Timmon's return was the last addition I needed to my long list of troubles. He had failed me in Escasaine, tied my line to that of doomed barbarian house, he would meddle with Carlotta's work, my husband and Barron Paulis would both try to kill him, to say nothing of how Alerio would react. I was furious with him for returning. He had not even seen fit to write me of his coming. I would do better to turn my horse from this house, and continue  as I had intended to the Liri quarter, I thought. I did not need to entangle myself now with this unwanted Barron. Yet I remained, gazing, transfixed, at the candelabra being raised through the large windows decorating the front of the house. Hinata saw me there as he coordinated men and firewood and wheels of cheese between the kitchens and the house. He bid me pay his master a visit.

Sophia had built Timmon's house to look like that of a Sama or Lir, with a roof, or so Hinata has told me. Beyond a small entrance vestibule where I saw four sets of shoes, including Firi's, was a large entrance hall that wanted to be a central courtyard, but for the roof that held a large lit candelabra. To the right lay the dining hall, to the left a lush green sitting room, opposite the hall an enterance to the garden that had become the talk of Cortan, that had served as the gathering place of my court while Sophia resided here. The large windows let in the autumn air, making the foyer feel brisk at night, and comfortable in the summers. A wide curving wooden stairs led to a long railinged hallway that wanted to be a veranda, with five doors leading off it. Servants bustled up and down the stairs and through the foyer with brooms and crates and furs and firewood as Hinata put me through the long and elaborate Liri process of removing shoes and washing feet. Everything around me was in a state of disarray. Walls sat unadorned, seats covered and uncushioned, floors cold and uncarpeted. Men and women tripped over each other in their attempt to make their master's house liveable again. It would seem Barron Romino had not seen fit to tell his household of his arrival either. A young bearded head poked shyly and curiously out of the left most door above, mirrored by another one the far right. The left face quickly disappeared, while the right door sprouted that of Firi. 

"Your grace, what a lovely surprise!" the girl exclaimed and ran down the stairs to greet me, pulling the hesitant fyoung man behind her. 

A loud crash and the sounds of shouting and arguing came from the kitchens. Hinata winced as he brought me my slippers. Firi glided into position before me, chasing Hinata off to manage the household staff while she took over her role as the woman of the house. "You will forgive the state you find us in, your grace. My father arrived quite unexpectedly not two hours ago," she bowed prettily and introduced me to her male companion. "You have met my half brother Eugenio, certainly." 

The years fell away from the bearded young man before me, and I saw again the ten year old child who had played with my twins during King Gustav's coronation. He was tall and broad shouldered, not as tall as his step father, but already almost as muscularly built as Timmon had ever been. Eugenio was almost as dark as the Liri were, though his features clearly Marsean. I had not known Commander Lazarro. He would have been a handsome man if his son looked anything like him. "Eugenio," I beamed, raising Timmon's stepson from his bow to look at him better. "You have grown into a man since I have seen you last. What brings you to my castle?"

"I am passing through, your grace. I wish to swear myself to the duke of Selvand." Eugenio had a quiet confident voice, he carried himself like one used to commanding men, but he did not meet my gaze. It was a measure of respect shown my by my Liri subjects. He had grown up there, I reminded myself. Under Timmon's influence, he would be almost more Liri than Marsean now.

"You wish to serve in the navy then?" I asked, impressed by the young man's courage. "Brave man. My army will miss your strength."

Eugenio smiled at the compliment but kept his eyes averted. "I thank you, your grace, but I am more at home atop a mast than most Black Rider's upon their steeds. This is how I can best serve Marsea."

I lifted Eugenio's chin until his eyes met mine. I had known this man as a child, he had played every day with my twins, been both their idol and their tormentor. I found his foreign formality hard to take. "Welcome home, Eugenio. May the Preserver guide your path." I walked around the young man once to fully take in how he had grown. It was hard to believe that he had changed so much since I had seen him last. "You will have to dine at the castle one day to tell me your plans for incorporating healers onto vessels," I gently teased. The problem of healers and sea battles had been one that had plagued Marsea's admirals and crippled our navy for centuries. 

To my surprise, Eugenio took the the comment seriously. He blushed and stuttered at the suggestion. "I would be delighted, your grace, but as for healers, I , er, I cannot..."

"Do not let my stepson's current bashfulness fool you, your grace. Eugenio has served aboard ships with healers, and has his own ideas for a stunning carreer." I looked up to see my host standing atop the stairs smiling down at us. He beamed at his stepson as if he were his own blood. Timmon would see Eugenio through to a successful career in Selvand, as he had seen to his training in Lir. His efforts had paid off well in the handsome boy before me. No father ever looked prouder.

My heart lept at the sight of Timmon with his family. I had a priviledged position to witness this scene. But I kept the smile from coming to my lips. The warmth I felt for his children not withstanding, Timmon had failed me in Escasaine, and had returned without my leave. He would complicate the delicate situation of my court. I turned a cold gaze to my host. "You did not write of your return, barron."

Timmon grinned, his joyous return not to be spoiled by my moods. "I assure you I did, your grace, on the off chance that we would be slower than the courier." He took my hand and bowed with a playful flourish unique to him. I was his duchess, and this was a formality, it said. I had no reason to doubt the depth of his respect and loyalty. The Ramino household's joy was infectious. I smiled in spite of myself. By the god's sacred dance, I had missed Timmon. The memory of his parting kiss burned on my forehead. If only he had not chosen such an inconvenient time to return to my duchy. "Come to my office, duchess. At the moment, it is the only room in this house that can even hope to receive guests." Firi stole a glance at her father's face, and disappeared with her half brother.

Timmon's office was barely habitable. While the chairs and sofa had been cleared of dust, there were no rugs or furs on the floor. The shelves sat dusted and mostly barren, the walls still unadorned. But there was a warm fire in the hearth, and a flagon filled with wine. I sat on the sofa and let Timmon serve me. "What have I done to distress you, Nisrita? Your welome was less warm than Hinata's."

I was neither his lover not his servant. One would hope that under any circumstance, Hinata would provide Timmon a warmer welcome than I. "What brought you to my court so suddenly, racing the wind?"

Timmon settled himself on an uncushioned armchair beside me, the warm welcoming demeanour of my host slowly transforming into the serious manner of a man of my court bringing new. He handed me a weak cup of sour red wine. His pantry and cellars had not been restocked to prepare for his return. "I heard about the fire," he said gravely.

Take and curse the man. Not now, not when the Tower was involved, not when any investigation by Timmon would certainly lead to questions about my husband and his motives, not when Timmon was the only man in Marsea who could effectively make the Liri understand that their lives would be easier if they accepted the Tower's services. He would be the solution to so many of my problems if only his mere existence would not cause half a dozen more. "I have the situation under contol, Timmon."

"I am certain you do, though I do not understand what led you to stone a Marsean."

"Nor do you need to," I said quickly, and more harshly than the situation called for. The words rumbled out of me, pushed forth by my anger before sense could stop them. Alerio had escorted Master Cretius to the scene of the fire, and remained silent while I decided his fate, only to critisize my decision over two weeks later. Now this man, who had not even witnessed the affair questioned my judgement. I would not be balked and judged by my friends and allies. I stood to leave. "I had only paused on my way to the Liri quarter. I had not intended to intrude upon your household. I am glad your family is well."

Timmon rose as well, opening his arms in abjugation, blocking my path as he did. "Nisrita, please do not take offence. You are my duchess, I a soldier. It is not my place to question your command. I have come to help. I cannot do that if I do not understand." I stood and stared at my opponent. He, with his questions, could be the most dangerous of my allies, the undoing of everything Alerio, Carlotta and I have worked so hard for over these last three years. 
"Will you tell me what happened?" he asked.

If I did not, he would ask questions. If I did, it would give him an opportunity to read my mind. My husband had been trying to gain a means of weakening the only true seat of power I had for years. The arrests across the country had provided the perfect opportunity. Master Cretius was not the only man of Marsea's White Towers to wish me ill for my actions. He was the only man we knew of who had gone to my husband against me. Alerio did not know what agreement had been struck between the two men, only that they had met in private several times over the last several weeks, and the unconfirmed rumours that my husband wished to take on a second personal healer. As far as Alerio could learn, Cretius acted on his own initiative, hoping to weaken myself and the Tower, while furthering one of his duke and parton's causes. What neither Alerio nor I could learn was why Barron Paulis had been so eager to stone the master. "Barron Paulis would not keep my peace. It was either the master, or a riot that night."

Timmon considered my half truth carefully, his jaw clenching and reclenching under his beard. What could I do to get him to leave well enough alone? "Let me come with you to the Liri quarter tonight," he finally said.

Oh of all the stupid things. What would I not have done for an hour of his company under different circumstances, but I neither required a body guard, nor a spy at the moment. "I sent you to Escasaine for a reason, barron. You are not welcome back here to interfere with my court."

Timmon bowed his head. "I failed you with Makala, duchess. You have my sincerest apologies. It was a poor --"

"You horse haired idiot!" I yelled. "What will become of me if my husband kills you while you ask your impertinent questions around this court?"

The words that came out of my mouth startled both us of. Timmon looked at me, first in surprise, then with something between anger and resign. "I have asked myself the same question every day that you have been married to the duke. You will learn to live with my absence, Nisrita, as I would learn to live without yours." He opened the door of his office, letting in the chaos of the house beyond. "Your Liri subjects await you, your grace. If there is anything I can do to assist you, you will let me know."

\vspace{.5cm}

***
%end Nov

I stood in the gathering darkness, outside the tall spired gate that demarked the Liri quarter, waiting for my stepson. I had finished my meetings early tonight, the gurya in charge of the boy's education having suddenly come down with a fever. Eugenio and I worked well together. Between us, we could speak privately to most of the members of the Liri community every fortnight or so. He had insisted on coming with me, over Sophia's objections that anyone at all leave for Cortan. The Liri were his people as well, he had said, even if the temple was not. His naval ambitions were a mere excuse for the moment. Eugenio knew how to talk to Liri men as a Marsean. He had fought and served and sailed with them. I could not ask for a more able assistant in keeping this peace.

It took less than two weeks after my arrival to make Sephiyat Nagayi see that his choice was not whether to have a wall separating the Liri from the city, or not, but whether or not to delay the inevitable in exchange for a proper marble and silver temple to Sento. After that, the concensus in the community followed swiftly. The other problem, of the Preserver's gift, posed more difficult. In Lir, a few were awed by the miraculous gift that Marsea's gods bestowed upon the nation, and flocked to make a temple to the triumverate in Lesoko, in the hopes of gaining the gods' favour, and thus, the gift. In Marsea, surrounded by Towers, the immigrant population distrusted the painful healing process, calling it witchcraft rather than a gift from the gods. Especially in this time, when many ungifted Marseans did not trust their Towers, the Liri saw little reason to trust in the powers of foreign gods at their times of greatest need. Even for the men who have married gifted wives with the hopes of having gifted grandchildren, the thought of letting their wife leave the house for childbirth was so abhorent and unnatural that they would rather see their wives and children die than let them avail themselves of the services that every wealthy Marsean woman took to be her right. King Anko's first born was delivered by the Mother's Balm, of course. But that was different. The Tower would visit a merchant's house, for a price, and a woman's life in Lir was rarely worth so much. Still, I wrote the king for a word of his support, and awaited an answer. 

I learned from Carmine Dielo the details of what had happened the night of the fire. He had gifted fighters with connections to the Tower, word of Nisrita's perceived betrayal and Master Cretius's association with the Duke eventually escaped the tight confines of those white walls in the form of whispers and drunken boasts. Given the incident with General Madriano's daughter, it seemed to me that the best course of action would be to keep an eye of the duke. Any indication of his needing to change his personal staff or hire a man would be reason to keep an eye on the duchess's allies. That was easily enough arranged, but it bothered me that Nisrita had hidden so much information from me. She granted me permission to stay in Cortan when she had satisfied herself that both Barron Paulis and her husband feared for their lives should something happen to me or mine. Her initial behaviour could possibly be explained by fear for my safety, but her continued refusal to speak of her husband or of her work with the black powder disturbed me. Was it possible that she was protecting him? I had dismissed that idea immediately. I would sooner think of her as plotting against the interests of the crown than protecting her husband. But I could not come up with a different explanation for my duchess's behaviour. 

"Have you been waiting long, father?" Eugenio appeared underneath the gates. 

I explained to him the change in my schedule, and he summarized his meetings with two traders. Both men complained of the increased fees Marsea charged for security along the Liri road. Eugenio did his best to explain away the costs without detailing Escasaine's current compromised position. When he left them, the men still grumbled about the robbery of Marsea's military, but the did not complain about their duke. "Where are you going tonight?" I asked when he finished.

Eugenio looked at his long shaddow and quickened his pace. "To the games. I may still catch a few matches tonight, take a chance at fattening my purse."

General Sidro and Barron Madriano organized nightly evening games between the younger recruits in the barracks. This kept them from roaming the barracks in opposing gangs, fighting small battles in the empty training yards, as they had nearly every night since Duke Makala's return. The young duke hated his brother for still having his father's favour and his inheritance. Duke Makala would kill young Griswold if he could, barring that, he would kill every one of young Griswold's friends. My stepson counted among that number. "Are you taking part?"

"Ha!" my son laughed. "I suppose you want me to enter as Griswold's champion. I leave you to play the fool with the duchess, father." Eugenio gave me a rougish look that would once have earned him a hiding. "I only go for the gold."

And to promote himself as one of young Griswold's most avid supporters, no doubt. Eugenio and Griswold had gone back to spending time in each other's company as if the interevening three years of separation had never occurred. I held my toungue. The games stopped the open fighting, at the cost of dividing the sons of barrons and young officers into Makala's camp or Griswold's. My stepson knew how to keep out of trouble. "Keep your nose clean, son," I said as we parted ways.

Eugenio did a little dance in anticipation of the evening's winnings. "Only as clean as you would, father," and he was off.

I turned homewards in anticipation of a solitary evening avoiding Hinata's grave disapproval and Sora's pouting. I left my servants to their domestic troubles after dinner, and retired to my library to think. The sound of the games had ceased from the barracks, there had not been any trouble tonight, quiet held the streets. Eugenio was not home yet. I stood on the chilly balcony listening to the stillness of the city and the arguments in my house and lost myself in thought. Duke Griswold had given in too easily to Makala's marriage. I had half expected him to decry me as a traitor, and urge the crown to clean the stain that Niev posed on a map of Marsea. He had done no such thing. He had plans for Cortan and Escasaine to stay border duchies. They would pay less in taxes to the crown and gain the best and most ambitious Marsea's soldiers while they remained so, but I could not see the end this arrangement served. Whatever it was, he hoped to accomplish it in the next five to ten years. He could not count on keeping these borders as they were for much longer than that. 

Whispers of Griswold's uprising haunted my ears, as did Nisrita's secrecy about her husband and the Tower. Had he frightened her? She did not seem frightened. If anything, she appeared more confident of her role in the castle than she had when she had returned from this year's campaign. She had learned to make Barron Paulis obey her in some matters, though the man would never be the ally she had initally hoped for. I found myself jealous of the man who had wrought that change in her. Alerio, another spider, weaving another web. Carmine did not trust him, he saw the Head bring Master Cretius to the Liri temple to be stoned. It was possible, I supposed, that the old healer had coaxed Nisrita to study the black powder where her husband's threats and violence had failed. If so, Nisrita may be trying to protect him and not her husband when hiding information from me. Alerio was a sly man, fully of the Tower, not one that Marsea's military should or would trust. I did not like to think of my duchess as a pawn in his hands. He was too secretive and Nisrita too guileless and naive to be safe in his hands. There was something uncomfortable going on in Cortan's Tower, and Nisrita stood dangerously close to its center. It frustrated me. I did not even know what I was looking for, let alone how to go about uncovering it. 

I heard the door to my chamber open, and Sora enter loudly, his candle casting flickering shadows onto the balcony. I went inside, not wishing to be disturbed by the jealous young man. I smiled when I found the warm drink Hinata had left for me by my chair. I made myself comfortable and sipped it slowly by the fire. Eugenio would laugh at me if he were home. He would accuse me of living with my two squabbling servants as the Liri kings of old lived with their arguing wives. "If you cannot control these two, father, what will you do when mother comes to Cortan to see you sulking at the duchess's displeasure?" I was not sulking. I merely dined alone at home while my children entertained themselves at the castle of a duchess who held me at arms length. Still, the boy had a point. Nisrita's displeasure stung more than it ought. Any other loyal man would have begged her forgiveness and returned for home, not persisted in disobeying his duchess's wishes. Unwanted though I was, staying in Cortan, watching over the Liri community, and yes, even the duchess who scorned me, provided me with more satisfaction than almost anything else that I had turned my hand to returning from Lir. If there was some good to be done for Marsea, I would do it here. 

There was a knock on my door, and Hinata announced a visitor. This was unexpected, though not extraordinary. I had set my traps and laid my nets over the last month, searching for information. Late in the evenings, after the castle has settled into their houses usually proved a good time for moles to come out to tell their tales. I rose to meet my visitor. "Good evening Barron Madriano. I am pleased you have accepted my offer. Tell me, how is your wife?"

\vspace{.5cm}
****

%mid Dec

"Are you coming?" Alerio asked late one winter afternoon from the door to my office.

I looked at him, dazzling in this ceremonial robes, ready for the evening's celebrations ahead. I stood in my office, in similar robes, but with my hair still undone, my face barely made, jewels still in their bags and boxes as I had brought them this morning. I had students in my office until half an hour ago, and the girl I had to dress me was not as efficient as my ungifted maids in the castle. "As I am, Alerio? Do you wish me to offend Head Fabia's replacement before she starts her position?"

Alerio passed his eyes critically over me and smiled. "Master Ada will see you appear in far more scandalous attire soon enough. It will not hurt for her to get used to your queer habits early." The girl brushing my hair, an eighth year student, Gemma, started giggling and tangling the comb.

"Out, Alerio," I ordered through my laughter. "Before I show you the true face of my madness."

The head chuckled, then bowed his apology. "I leave you to your dress. I came to remind you that none of what happens tonight reflects poorly on you, or your standing in my tower."

I smiled weakly. "Of course it does not." May the triumverate be good to the man for his kindness. Tonight would be difficult for me, though we had discussed all the reasons that I had not been chosen to succeed Head Fabia a hundred times. The rational explanations did nothing to ease my jealous yearning, though I told myself they should. One last reminder would not hurt my attempts to make it gracefully through the night. 

Alerio still stood in my door, looking at me, rather, at my hair, usually tied back sternly, now falling thick and straight to my waist. "Head, I believe you have duties to attend to tonight?"

Gemma burst into another peal of giggles as Alerio stepped, embarrassed, out of my office. I joined her. It was the best way to avoid rumours from spreading about Alerio's affections. 

Head Fabia had tripped down the stairs only a few weeks ago. She broke her hip and her leg badly. She was a few years older than Alerio, closer to seventy than sixty, the Tower could not heal her. Women, in our collective experience, do not heal easily from hip injuries in old age. Fabia announced her resignation from her position as the Head of the Women's Tower. Her favoured successor had once been her student, Master Febe from Escasaine, but Alerio refused to promote her, given her close ties to the now disgraced Adele. Instead, Master Ada came from Allepo to take her place. My name had been put forward, but given my current reputation as a traitor to my own, and my young age, there was no real hope of my becoming the next Head. Alerio had not even supported me for the position, putting more favour behind Carlotta for the position than myself. It had hurt. It did not matter that Alerio still saw me as heading the entire Tower someday, the Women's Tower had been my dream for as long as I could remember. 

There would be other opportunities, Alerio had reminded me, I was not yet thirty. No Tower would appoint a head that could lead for thirty years. That had not prevented him from suggesting Carlotta, I had objected. But Alerio gave me a disappointed look with his large grave eyes and refused to explain himself. Instead, he told me to tend to my teachers and quell the new controversy over how to train weakly gifted students. 

It was not simply a matter of a badly bruised ego. I had no assurances from Alerio that Master Ada would not turn herself over to my husband's cause. After Master Cretius's betrayal, it took Alerio several weeks to root out two more masters who sympathized with my husband, and another handful of students and healers. Unlike in the case of Master Cretius, where he had wished me to divine his wishes and kill, Alerio decided to let these men and women know that he watched them, and let the matter stand. The head promised me my continued safety, a claim I would not have believed if not for the change in our relationship over the past several months. Now, when Alerio said he would see me safe, I trusted him as I would a father or brother, or even a lover. 

Yet men can be wrong. Unlike my husband, they are merely mortal. Alerio had thought his Tower impenetrable, but my husband had found allies. Now there was a new politically powerful woman that Alerio allowed into his Tower, from Allepo, from where Head Eliseo and Master Cleto had hailed. It made me nervous. 

Gemma put the last pins in my hair and gave me a mirror. I had to give her credit, the girl may be slow, but she was skilled with her hands. I quickly tidied my office of my toiletries, then walked down to the Tower's Great Hall to congradulate Master Ada, and watch her be made head. 

I was among the last  to file past the rows of students and healers on my way to the raised table for the Masters. Alerio gave me a long appraising look that sent a chill down my spine and brought a coy smile to my lips. He approved. It had been such a long time since it mattered whether a particular man approved of my appearance. There were moments when I could imagine myself completely sustained by the head's company. But the man had a Tower to run. By the time I had taken my place at the table, Alerio had turned his attention to his new acquisition, and I was forgotten. 

The ceremony formalizing Master Ada as head of the Women's Tower lasted for the longest hour ever recorded by man. Alerio could not sing enough of her praises. Neither could Master Eulalia who spoke in Head Fabia's stead, as she was still too unwell to attend the ceremony. Everyone spoke of her inspirational character, and patience, yet all I knew of her was that last year she had advocated against women from her duchy serving in Allepo's campaign, claiming that our sex had other gentler duties to Marsea. It was easy for her to say, withered and well past her years of child bearing, having produced only four children, of which only one is gifted. She would bring that battle to Cortan, and I would fight her on it, patience and inspirational character be taken. The Tower had promised the military its support. I meant to honor it. 

I caught Carlotta's eye, questioning my scowl. This was not personal, I reminded myself. Alerio knew what he did in promoting this woman. He agreed with my views on supporting Cortan's military. She would not get her way. I drained the wine cup in front of me and called for more while I listened to Eulalia drone on and one about Head Ada's childhood in Allepo. The wine was of better quality than that normally served in the hall for dinners, but it was not of Alerio's personal stock. That it was sour did not matter, only that it was strong.

The five courses of the dinner dragged through an eternity. Conversation during the soup course focused mainly on the new Head's qualities and the old head's health. By the second course, those topics had run their course, and those around me asked about my controversial plans for the training of our weakly gifted students.

"Children like Isaia cannot be taught. I will stake my house on it," Master Tobias insisted for the fiftieth time this month. He taught the sixth year students, and had struggled all year with the boy Isaia until I took him out of his class and sent him to the barracks. I intended for him to train for a few years until he had developped the discipline to learn the theory lacking in his education. The ungifted segeant in charge of him stood ready to steer him towards his books, recognizing the importance of creating another gifted fighter. It was my kind that rejected the boy.

I smiled mischeviously at my colleague. The wine made that easier. "Watch your words Tobias. I may accept your wager. I am certain Mason Gallo would put that beautiful rose tinted gate of your to good use."

Master Fernando, who had coveted Master Leon's position before I stepped into it protested while Tobias laughed, "What you propose is impossible. Most weakly gifted children simply cannot learn. Your new methods may work well for those that can go through the rigors of a full education, but it will deprive our armies of gifted fighters."

The arguments continued, publicly at that dinner as they had in countless private meetings in the weeks and months before. My idea was not impossible, simply undtried. It ammounted to the same thing. One may have said that once about regrowing limbs. That point was completely irrelevant. Our armies take barely literate men and turn them into scouts that can read maps and send messages. Did I really think that reading a map required the same amount of skill and education as diagnosing and healing a punctured lung? Several of my neighbors joined in. Eulalia supported me, for what it was worth, though my decisions would not effect the education of her girls. Carlotta remained quiet, her husband did not approve in theory of my plans, though it would have little immediate effect on him as a general. Alerio remained completely absorbed with Head Ada.

After an hour, and several cups of wine, I finally turned to Fernando in frustration. "My good master, if you will not deign to train these boys, I will bring in gifted fighters teaching at smaller towers who will. It may be time for this school to be run by people willing to work once in a while."

The table around me hushed. Carlotta shot me a look that would have sobered a sailor. I ignored her, and waited for Master Fernando's enraged reply. Tobias spoke first. "You cannot mean that, Master Nisrita. There is not a man in this school who as served as long or well as Fernando."

"As long, yes. As well, perhaps, but one may do better." I looked around the table at the scowling faces, and took another swallow of wine. The wine had given me courage to face all their disaproval alone. Now it made me prescient. It was not hard. I knew what my teachers thought of me. "You are all accusing me of betraying my own kind." I shrugged. "If I sat among you only as a teacher, I might agree. But I carry the seal of Cortan when I am not in this Tower, and I must look out for my military, and all my students. The gifted fighters, last time I checked, were as welcome amoung our ranks as learned men like Master Ferdinand are." Master Fernando's scowl deepened, but I had interested the of the rest of my audience. I swallowed more of the dark red liquid that seemed to turn my tongue to silver. "I will place a wager, Fernando, with our guests here as witness. One student of my chosing, among those you think cannot be taught, one gifted fighter of my chosing, and one year. If he cannot be taught to heal as well as he would learn in this Tower, I will shift the responcibility of teaching the weakly gifted to the military, and you will hear no more about it. If he can be taught, I want your resignation."

My comrades and colleagues whispered and murmurred amongst each other, shocked at hearing a woman place a bet. Some prodded Fernando to take the wager, others calling on him to proceed carefully. Eulailia leaned in towards me and hissed "You are drunk, Nisrita. Take back your words before you do something foolish." I smiled and waited. My political woes had stacked against me. I was no longer popular enough to lead the Women's Tower. If I did not already have this position as the head of the school, I may be too unpopular for even that. I had little to lose. Why I wanted to lead such a group of self important, power hungry, flacid good for nothings, was not clear to me at the moment. It was not that I betrayed my own. My companions did so every day. Few among them fully recognized the vital role our gifted fighters played in our military. If the weakly gifted were not our bretheren, it was hard to betray them, wasn't it?

Fernando put away his scowl and looked at me. "We are not of the military to wager like drunken fools."

"Well argued," I congratulated him. "If we were of the military, you would be demoted or sentenced to hard labour for questioning my command. Instead, we are a collegiate order, and I drink to your health."

The company drank to his health, while Master Fernando gave me a look that said clearly, we would speak of this later. But he dropped the subject. I had won for the night, without resorting to threats, violence or bloodshed. Perhaps Alerio was right, I was a better leader within the Tower than I was without it. I drained my cup and called for more wine.

\vspace{.5cm}

I found myself, several hours later, in the round white dining room below the head's private quarters, drinking Alerio's wine and still congradulating Head Ada. The night had proved an excersize in manuevering through a gauntlet of men and women who looked down upon weakly gifted and the gifted fighters we trained, and myself for putting the needs of my military on par with the needs of the Tower. These were Head Ada's views, and tonight brought out her supporters. The general crowds had dissipated into the night. Only a half dozen masters remained in that gleaming room at the pinacle of Cortan's Tower. It was not personal, I reminded myself for the twentietth time that night. None of this reflected poorly on my abilities as a healer, or indicated Alerio's lack of support of my vision for his school. Yet the constant adulation of Master Ada's skills and personalily around me sickened me. As the head of the school, I could not easily leave. I would have to work with her and against her for years to come. It would be a poor performance if I could not stand her company for one night.

The conversation turned, somehow, to praising my husband's accomplishments during his early years in Cortan. I glanced at Alerio, who remained as inscrutible as ever. I was possibly the only person in the room besides the head to know of his youthful conflict with my husband. If the head did not object to this conversation topic, I hardly could. Alerio busied himself by playing the attentive host to Head Ada. He spoke to her with such intent attention, even in this small company, that my face flushed with jealousy. It was not simply the need for the head of the Tower to become accquainted with the personality and ambitions of the head of the Woman's Tower, the person he expected to work most closely with in running the gifted in his charge. I was a fool, I told myself, as I stared at Alerio's unwavering attention to Head Ada. The head was a man of many plans and schemes. He had invited her to join his Tower with a purpose in mind. Did I really think that I was the only woman he had seduced into playing his political game? I knew nothing about him, or his goals or his plans. I did not even know in entirity what he planned for me. What reason did I have to believe that I would be the only woman for whom he would spare an appraising glance, that mine were the only hands he kissed, or I the only one of his masters he tenderly wished good evening every day.

I rose from the table and made my excuses, only swaying mildly as I headed to the door. I could no more listen to that party praise my husband than I could watch Alerio staring intently at the new head's face.

I found myself at the stables, rousing the drowsy groom. I had meant to ride my palomino home, but I found myself leaving by the north gate, out of habit of heading towards the Liri quarter, or seeking something else, I was too drunk to say. 

Lights still shone in Barron Romino's house, inspite of the late hour. In the two months since Timmon had returned to Cortan, I had visited him only a handful of times, though Eugenio and Firi were frequent guests at the castle. My encounters with Timmon inexplicably ended in argument, no matter what the intended purpose of the meeting. I had just spent the night with men and women who scorned Cortan's great military. Yet it had been the military that had saved the Liri while a gifted had started the fire. The men of the barracks respected me without question. I knew who was loyal to me, and who was not. They did not balk me, or question me, or test my will. At the end of the day, Timmon had been invaluable in establishing an understanding between the Liri quarter and the rest of my city. I had shunned him for a company of white, thinking that my future lay as the head of the Tower. At the very least, I owed him an apology. I dismounted to knock at the gate. 

Hinata met me promptly at door, and led me directly to his master. Most of the household was quiet, yet Hinata acted as if a late visit to his house was nothing unusual. I wondered what games Timmon played at, then pushed the thought from my mind. I had come here to apologize to a friend, not to start another argument.

\vspace{.5cm}

My maid woke me the next morning from a beautiful dream about the unexciting life of a general's wife. She lived in the mountains, walked with her four children in the hills, raised them on goat's milk, and marched with her husband every spring. Her children loved each other, and her husband loved her. She kept his household neat and tidy to his satisfaction, and he asked about the students she taught at the Tower every night. As evening came, she watched her sons learn to beat her husband at chess, while her daughters sat by her feet and practised her embroidery. Her children all looked like her husband, a handsome man with long now greying locks that still fell in soft curls about his face and shoulders. They had once been rich and dark, a feature of her husband's body she had delighted in. Only her youngest daughter had her hair, straight and thick, falling to her waist. They all had her husband's fine features and broad shoulders. They were handsome creatures, good, honourable children, that she had raised. They were not gifted, but it did not matter.  She could be proud of their characters and her husband had great dreams for the men and women they would become. She and her husband sat every night in their small house nestled among the mountains. The windows were small, the hearth was large. It snowed in the winter, and there were apples in autumn. Her husband loved apples, and he would hum softly to himself as he passed among the trees picking the fruit. Every night, he put his arm around her shoulders and kissed the top of her head. She thanked him for watching over their family, and let him lead her to their bed.

The feeling of inexpressible contentment and joy that permeated the general's wife's every day flickered for a moment as I registered my maid's voice calling me. I clung to the whisps of that quiet peaceful world, without knowing fully what awaited me when I awoke. My maid persisted in stripping me of the feeling of warmth and comfort, of having my heart swaddled in lamb's wool. I would be late for morning prayers. The cool of the winter night penetrated my conciousness, the sound of the morning rain replaced the call of mountain birds. In stead of a husband's kiss, placed gently upon my head, my temples throbbed, and my mouth tasted of dead rat. When I threw back the covers, I found myself inexplicably completely naked underneath. I had no recollection of how I had come home.

"Where are my clothes, Elsa?"

The girl looked abashed. "You came home late last night, your grace, wearing nothing but a cloak, asleep in a carriage. When the men carried you here, I did not disturb your sleep to dress you. I beg your pardon if I judged wrongly."

"I am cold." I snapped. "Give me something to wear." I pulled the blankets over my goose bumped flesh and put my throbbing head in my arms. What had happened? I remembered nothing. 

A splash of color caught my eye as my maid lit candles by my bed. Three long sprigs of lavender and a wilting red hisbiscus flower lay huddled together on the table. Timmon, I thought. I had paused by the Barron's house to apologize. It would seem that I had done more than just that. It was not fair, I groaned to myself. Of all the trials of the previous day, why did I remember with painful clarity every comment made by my colleagues against the weakly gifted students in their care, but nothing of what might have been a very plesant evening spent with the barron. 

I picked my way carefully and painfully down the stairs to the temple that served the castle, trying to remember and praying that I had left Timmon's house with a shred of dignity in tact. I must have I argued, when the sublime tranquility of this morning's dream pressed itself on my memory. That alone must be proof that I had left the barron's house on good terms with him. 

Timmon had been awake and happy to recieve visitors, I recalled, as I found my place with the rest of my family near the front of the temple. He said that I was not imposing upon him at all, he used the late hours of the night to think. Now sober in the small smoky room, I thought myself a fool for entering the barron's house. I was drunk, Timmon was not. What secrets had I let slip? One I could remember quite clearly. I had realized my error as I had made it. Timmon had offered his condolences for not being chosen as the next head of Cortan's Women's Tower. I had said something, intending to shrug of the sligh and sound full of bravado. "I am young yet. I will rise far in the Tower. Provided my mysterious malady does not kill me, I will long outlive Ada and serve a long tenure when she is gone." Timmon ignored the bravado and latched onto those two cursed words. What had made me think to speak of my illness to him.

There were few details of our conversation I could remember. I had the sense that we had talked for ever. Timmon talked about himself at great length, I recalled a sense of relief at being with a man who did not hold details about himself tightly to himself, to be pried and coaxed from his lips. It came with the feeling that I could share anything with this man and still be held in high regard. Not only had it been foolish, I now had to visit Timmon and learn, as tactfully as I could, what it was that I had revealed to him. Then I had to find a way to make him stop acting on the foolishly given information.

The gongs sounded, making the inside of my throbbing head wish to explode. A priest walked by with sticks of burning incense to offer to the gods. He was a tall man, I had to look up to thank him as I rose, my fingers brushed his hand as I reached for the incense. The motion brought back another memory. There had been a kiss. I blushed to think of it, and bowed quickly before the Preserver to hide my face. It had been more than a simple kiss. I remembered my surprise at Timmon's fervour, and the urgent need it awakened inside me then. The same need stirred uneasily between my kneeling legs, reawakened by the memory of Timmon's arms. His soft beard tickled my toungue, and I had clung to him in my desire to be something and someone I was not. I had never been kissed like that before, not even as the newly made Maker's Daughter by healer Ricardo under the organge trees. There was a longing in it, and a skill, planted by a man who knew how to be a lover. The kiss, it seemed, filtered now by an uncertain memory and a strong sense of longing, had seemed to last forever. There must have been more than that kiss, or I would not have found myself as I did this morning. The details I could not remember. 

It would be easier, I told myself, if I let the events of last night slumber. There was not need to disturb the pleasant evening Timmon and I had spent together with questions and politics. It would only cause us to argue again. It would certainly reduce any chance of a reoccurrance of the events under circumstances that I could remember. The dream of the general's wife floated back to me. I had no more desire to argue with Timmon than I had to send him away. I loved the man, and I loved his family, and I had for almost as long a I had known him. 

I rose from my prayers, savouring my morning's dream and the sweet sense of loving peace it gave me, along with the delicious memory of the passionate kiss. What had I done, I wondered, to provoke such a reaction from Timmon? I regretted the question, almost as soon as I had asked it. In my inebriated curiosity about his life, I had asked the barron what it was like for a man to kiss another man.

\vspace{.5cm}

"You missed a most informative evening last night, Nisrita," Eulalia said, intercepting my path from the infirmary back to my office. 

"I had seen all I had wished to see, I believe." I replied curtly, slowling down my steps to match those of the grey plump woman beside me. She winded easily, and walked painfully, many of her joints gnarled and red. To look at her, one could easily jump the to conclusion that she was the type of grandmother who would coddle and cosset a girl, slipping her sugared orange rinds and an extra helping of cream when her brothers weren't looking. Certainly many of her girls did. They could not be more wrong. Eulalia was sharp, keen eyed, and could outwit and foil the best laid plans of any thirteen year old, even if she clould not outrun her. "How is the new head?"

Eulalia raised her eyebrows in disaproval. "Still nursing a headache I believe. One must learn how to pace oneself to serve with Alerio." Eulailia shrugged dismissively, "She will learn. You, on the other hand, surprised me. What possessed you to put forth a wager? You do not need to provide our teachers with more scandals to gossip about."

I laughed. "What will they say, Eulailia? That I am mad, or miscast? They say that already about me. Those whispers come from the court. You and Alerio will keep them in line."

Eulalia scowled. I would have hated to see that scowl on the face of my headmistress while a student. As her superior, I found it comforting. It did not balk me, but kept me honest. "You do not make our task easy."

"No," I admitted, "but neither have you been able to convince Fernando to stop his grousing. I have bought myself time. Small doses of public shaming can be good for a man, ask any general." I could see the thought gathering darkly in Eulailia's head. The Tower is not the military. It would be better if it were, I would argue, but let it be. "Besides, I think the idea of having gifted fighters train other gifted fighters may not be a bad idea. I will talk to my army about it."

"You chose a horrible way of announcing your reform." I could feel Eulailia's disapproval radiating off of her in waves. I had been brash, unfeminine and drunk. In short, I had disgraced the school. 

I changed the subject. "What happened last night that was so important?" I asked.

Eulalia lowered her voice, but kept her tone casual. We were in an empty corridor, but still some way from either of our offices. We looked for all the world like the two heads of the school discussing our students. "The revelry ended shortly after you departed last night, leaving Head Ada, Alerio and myself some time to talk. I think Alerio was disappointed that you had not stayed longer." The old woman shrugged. "The new head admitted to knowing more of your husband's allies in the Tower than Alerio had uncovered so far."

I did not understand. My head swam. The new head of the Women's Tower belonged to my husband. "Alerio got the new head drunk so she would talk?" I asked incredulous.

Eulailia laughed a dry papery cackle. "How uncouth! You should know your head well enough to know that he would never stoop to such a level. Alerio brought her to the Tower because she thinks she can bring your husband's faction under control to avoid any further incidents like the burning of the temple in the future."

"Bring them under control?" My voice rose unintentionally. Eulailia tisked at me. "What is Alerio doing?" I continued quieter. "This is his Tower. It is supposed to be free of my husband's interests."

"Supposed to be," Eulalia wagged a gnarled bent finger at me. "That is differnt from is. He failed. So he will bing someone in to control the damage."

I was frightened. Alerio had promised me a place of safety. Now it seemed that I would have to fight my husband even in the Tower. "How long has Alerio planned this. Did Fabia know?" The thought of the frail cheerful head of the Women's Tower being part of a dark conspiracy seemed wrong. Schemeing seemed to go against the very fabric of her open likeable nature. But she had been head under Alerio. It was impossible that she did not know. I seemed to be the only Master that answered to him not privy to his convoluted plans.

Eulailia tisked again, scolding me this time. "If you are implying that our head had anything to do with Fabia's accident, you have mistaken the man for your husband. Of course Fabia knew. She warned him of chinks in his wall long before anyone else thought to do anything about it. Who do you think discovered Master Cretius's treason?"

I held my tongue as we passed through a passage containing several full classrooms. I seethed with anger. How long had Alerio planned this and kept it from me, soothing me with promises of safety while planting spies in 



\end{document}
