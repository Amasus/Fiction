\documentclass{article}

\usepackage{fullpage, verbatim}

\begin{document}

"Did I ever tell you, little one, about the time your father stripped me in front of his men? I suppose I haven't. You have only been with me these five days. We have not had much chance to talk, have we? Well, settle down for a little tale. It is late, and I should be sleeping, but the bed in the other room is too big, and the night is too warm. I would much rather be speaking to you. 

"What's that? You are rubbing your face in your sleep. Does that mean you do not want to hear that story? Very well. It is not a very nice story, and it happened well before your time. What can I tell you that you do want to hear. Let me tell you your name. That is an important thing to know. I have named you Griswold. I didn't have much choice.  It is your father's name. I named your brother after a better man. I am sorry I could not do the same for you. If I could name you any thing in the world, what would I name you? Well. That's an easy question. I would name you after another good man I know. I have not seen him for months, but he is still a good man. Don't let rumours of his actions fool you. 

Oh, but you are crying. Don't do that. Here. Let me pick you up again. Is that better? I suppose you don't want to know about the good men I have known, especially when they don't effect your life. Who do you want to hear about? I don't know much about your brother, he's as new to me as you are. I don't like to talk about myself. There hasn't been much to say recently anyway. I haven't seen anyone or done anything for months, between traveling and being too big. That leaves your father. 

Your father, your father. There are so many things I can say about that man. Do you want to know how he gave me this shiny locket I wear around my neck? I know it caught your eye the other day. It catches the sunlight just so, doesn't it? It's not just for decoration, you know. It's called a talisman. Maybe someday you will wear one too, if you are lucky. 

I haven't always worn it around my neck, you know. I used to wear it low, where it was unobtrusive. It was also more comfortable. Then, when I found out I had you growing inside me, I took it off. I didn't want to hurt you. Well, I didn't know what I wanted to do. But I took it off anyway. Then one day just after we arrived here, while I was helping your father with his letters, your father asked me if I would be interested in trying the mushrooms again, after you were born of course. I refused. It was an old argument. I knew how it would go. He would ask me every few days, and I would refuse. He would get angry and call me all sorts of names. They are the types of names that I will have you thrashed for, if I ever hear you using them against a woman. We had a terrible row. It was lovely. I think the yelling must have woken you and your brother up. I could certainly feel you kicking inside me. He yelled and screamed and threatened. I yelled and screamed and cried. What to do? I am the daughter of a low born gifted woman. No ammount of breeding would clean up my nature. He said that too, you know. Don't ever insult anyone's mother, you hear. I'll make sure to thrash you for that too. 

Anyway, I woke up in the middle of the night, as I often do. I don't know why I thought it. I got out of bed, found where I had put my talisman, the old one, walked up to your father's sleeping form and burned him. I burned him long and hard until I could hardly stand. He lay in bed choking and gasping and thrashing. I burned him as he had made me do to others once. I burned him and I did not let him go until I saw stars. And then I fled before he could recover. I must have dropped my talisman somewhere. I was stumbling blindly down the stairs, dizzy.

You know, I don't know where I ended up that night. The next thing I remember, I was in the infirmary of the tower. They haven't let me out since. When I asked for my talisman, your father showed it to me. The crystal was shattered. He said I had dropped it in my hurry to be off. I don't believe him. That crystal does not break easily. It is meant to keep us safe when we travel with the army. It can't just crack every time we fall off a wagon. 

Anyway, he gave me this one you seem to like so much, the day you were born. He wanted me to wear it where he could see it. I think it is because he wants to know when he should be afraid of me. I'm glad he is afraid. He also got me this soft velvet band to hang it off of. I didn't think much of it at the time. But if you like it, I'll keep wearing it. Just for you.

***

\vspace{.5cm}

I did not return to Cortan until late spring. The duchy did well in the aftermath of its conflict with Firvona. The Duke of Firvona had been found guilty of plotting to kill a duke. The crown gave the duchy to the duke's brother, while the duke and his sons, he gave to the Destroyer. Several second and third sons of Cortan's barons now hold positions in Firvona's court. Duke Ergino had returned to Cortan with the legendary Griswold as his uncle, son-in-law and advisor. His house had not fallen. When there had been a fear of Cortan passing eventually into the hands of the unpopular Erfat Cortan, brother to Duke Erigino, the crown had considered carving up Firvona, and Cortan into three, leaving Erfat with an interior territory to rule. But that did not come about. Griswold returned in the hour of Cortan's need, as had been prophecied. I know nothing about the second part, where the mysterious man, who does not seem to age, will destroy himself in saving Cortan. Duke Ergino, from everything I have heard, does not seem to fear his uncle's demise. He is overjoyed to have his uncle back with him. 

I left Deyalorn to return to my post in Turina. I had thought that the habits and pressures of my duty would help me forget the turbulent events of the late summer. It did nothing of the sort. I found the mundane administrative tasks given to me as a commander dull and unfulfilling. The captains and officers of lower ranks doused the constant flares of violence and destroyed the pockets of resistance that incessantly troubled the new holdings between Turina and Bayam. I sat within the walls of the stronghold and watched over Duke Erfat. Officially, I was the officer in charge of the territory's intellegence. Not a bad title to hold for a man not yet thirty. In reality, my duties were closer to being a tutor to an exceptionally dull boy. My job was to explain to Duke Erfat, painfully and slowly, the details of our military's movements, and the conditions of our western border with Niev. It fell upon me to keep my charge from making any decisions that would interfere with General Madriano's running of the duchy. We had not lost so many good men to gain this land, only to lose it to Duke Erfat's incompetence. It was a bitter job. I had won these land's for Makala's glory. Cortan was to be his in time. For his brother to be gifted these new territories was one thing. For them to go to Makala's fat decadent uncle was loathesome.
 
I found myself regretting my promotion to commander. I saw no battles, I did not march with men among the mining towns and hamlets of the Pensid mountains. I sat in a stronghold, gathered and deciphered reports brought to me by men doing the real work of keep this territory secure, pieced together an image of the new land from maps and second hand reports, and boosted the ego of a meddling middle aged man, to keep him for losing these precious lands. This was not the work I was cut out for. 

By the time the mountain winter drew to a close, General Madriano had reprimanded me several times for the my lack of proper conduct before Duke Erfat. I had failed to appear before him a few times during those cold frosty winter months. There was little in Duke Erfat's court that could entice me to eagerly attend him. These lands were not meant for him. He was serving as regent and caretaker until Duke Ergino could make further arrangements. I resented his presumptions to the contrary. 

Distasteful as it was, I was there when Duke Erfat's court celebrated the milestone of holding Turina for a full year. I sat near the front of the hall listening to men from Gissal sing Duke Erfat's praises, then the Duke claim the credit for General Madriano's smooth running of this new territory.  It was not my place to protest. As the soldier's order says, \emph{Officers order on the king's behalf}. General Madriano served the Duke.

I could not, however, sit idly by when one of Duke Erfat's courtiers, the second son of a Gissali baron, I think, suggested that the duke would not be in this position if not for his nephew's hubris. Only an inexperienced diplomat would attempt to marry his brother to the daughter of a duke his father had so recently quarreled with. I deny, as General Madriano claimed, that I was already foolishly drunk. I had drunk no more than had become my custom. Bound and shackled within the walls of Turina by my new duties, there was little to do but drink and gamble in the evenings. I do not, however, deny the force with which I turned on Duke Erfat's lacky. Makala was a gifted diplomat. His father trusted him. In his short twenty three years, he had brought more glory to Cortan than his over stuffed uncle had brought in his four miserable decades on this earth.

After I had bloodied the Gissali baron's face, and said those words in the Duke Erfat's hearing, it was a miracle that General Madriano managed to have me sent back to Cortan with my rank in tact. I should have been stripped of my command, or exiled. Instead, General Madriano sent me to the one place I could not face. I suppose he meant it as a kindness. He did not know that he had set a punishment for me worse than time working in chains in Turina's dark mines.

I spent a few days in Bayam gathering my courage for the last leg of the journey. I would return to my colleagues and traines, and officer without an command, to sit idle in the lands I had given nine years of my life to. I contemplated returning to my father's house. There was nothing that bound me to Cortan anymore. My nephew clearly needed a father's hand that he was not getting from his grandfather. I could resign my commission in the border, and seek my fortune in Deyalorn. It would be a better option than facing the men I had once worked with or trained in my current state.

I decided in the end to pause for a while in Cortan, before continuing on to my father. There was the matter of Cortan's heir. I had told Nisrita when I saw her last that I wished her well. I had meant it. I did not begrudge her a marriage to Makala's living image, though I did not know how I would stomach the sight of Nisrita and Griswold together at Duke Ergino's table. I wanted to see her child, when it was born. I had no claim to it. I had released that when I left her standing in Deyalorn's training grounds. I found myself drawn to the idea of the child. Except of a handful of days that autumn, I had never seen myself as the type of man settling down to a wife and family. I wanted to at least see what I had given up. I would stay in Cortan until Nisrita was delivered of her child, if she would permit it. Then I would move on to carve a different path in this world.

I left Bayam the morning after I heard that the Duchess had delivered twins. Cortan had two male heirs again. The news abruptly cut short my indecision. I did not care who the father was. A year ago, I had thought Makala's death had killed me. Three months later, when I learned of Nisrita's rumoured death, I realized that only half of me had died. If news of this strange young woman's death could bring me so much anguish, it only stood to reason, that news of her bringing forth new life would burn me with a desperate desire to share her happiness. 

I spent my four days traveling to Cortan listening like a hag at a fishmarket for news of the Duchess. I had wanted to hear news of her health, of her doings during the last several months, even of her marriage. On that front, there was little to be learned, even from those well connected with the Cortan's Tower.  She seemed to have ducked behind a veil of secrecy. I could hear wonderful tales about Griswold's wisdom, and the great impression he had made in Cortan and Deyalorn, but little of the lonely child he had married. I did, however, hear whispers about her character. The children were born only seven months after her marriage. She had traveled for so long alone in the company of men. I lost my temper in a small tavern a day out of Cortan. Nisrita had not wanted her character questioned, in spite of her mistake. I had assured her that no one would. I was stone cold sober when Baron Farone's marshall's son called the duchess what she was not. My sobriety saved his life. 

I slept among the grasses that day, and paid for the man's transport to a Tower of his choice to heal. My temper was not helping my career. I would resign, I swore. I would resign, leave this painful life behind, and start again.

\vspace{.5 cm}

I did not know what to expect when I approached the Duchess's rooms in the Tower. I did not think I would find her sitting demurely sewing in bed with her child sleeping by her side, as women normally do eight days after giving birth. I had tried to polish the rough edges from her that Makala had so irresponcibly ignored, or even encouraged. She had too many. I thought I would not find her a quiet and  blushing new mother. I did not expect what I saw.

I gave my name to the healer attending her door, and waited my responce. She emerged a short while later to inform me that the duchess would not see me now, but that I was welcome to visit the new dukes. I rose stiffly from my seat and sighed. I had expected this as a possibility, though I had hoped against it. After all, it had been I who had left her in Cortan. I could not blame her for bearing a grudge against me. 

I turned my feet down the corridor to the stairs that would lead me out of Cortan's infirmary. The attending healer stopped me. "The children are this way, commander."

I may as well, I thought. I had journeyed to Cortan to see the new heirs. I should pay my respects before I leave this land forever. 

The new born Dukes lay curled tightly around each other on their bed in the warm spring air like a pair of miniature lovers. They looked very different from each other. One was darker, and larger, the other fairer and born with a head full of hair. I searched their faces for traces of the strong resemblance that ran through Duke Ergino's family. The children were simply plump and sleeping, their features too small to find any familiarity. I wondered what it would be like to be in Cortan in a few years, with little likenesses of Makala and Lukos playing in the fields between the castle and the barracks, while the men I knew so well would no longer walked this earth. I would never get to find out.

I asked the woman attending them if their names had been chosen yet. They would not be officially named until they had lived for two weeks, at which point they would exit the Maker's domain to enter the Preserver's. I wondered if I was still considered a close enough friend to be told the children's names ahead of the ceremony. The larger son, born first, I was told, the duchess called Griswold. The smaller, she called Makala.

The words stung. They felt like a reprimand, though I am certain I was far from Nisrita's mind when she named her children. These children could have been mine, the names reminded me. Nisrita could have given me two children named Makala and Timmon. I handed the two gifts I had brought for the two new dukes to their attendant and watched the sleeping forms as she passed through a side door to give them to the duchess. I could not take my eyes off the smaller, fairer child. Nisrita was calling her husband's soul back to this world through her son. I am not accustomed to praying, especially not to the Maker, who chose to created me miscast. Up until now, I thought I had nothing to say to that god. I sat watching the children sleep and prayed that the Maker be merciful to Nisrita where he had been so cruel to me, and allow Makala to return. 

"You should not have come to my private chamber, Commander Romino." 

I startled to see the dutchess stand before me. Not the boyish child that climbed and ran and trained with first the Black Riders, and then with me. Not the uncharismatic, unfeminine healer who walked for six weeks to Turina, too interested in her art to pay attention to her appearance. Not even the awkward, forcibly goomed, young woman I had courted last autumn. The duchess stood before me. Her face and dress had the polished and smooth look of a woman who valued the art of inspiring awe. Her hair was tied intricatley back and jeweled. Her eyes, outlined in black, were dark and aloof. Her waxed arms elegantly carried several bangles, her fingers were ringed, their nails shaped and lacquered. She even wore her talisman as an ornamentation, or warning. The large crystal and silver amulet now sat on a wide black band at the base of her neck.  She held herself with the upright ease and grace of an acrobat, and the confidence of a woman who knows that she should be feared, if not for her skills, then for the political power she weilded. The troubled, gangly girl I had left behind in Deyalorn was no where to be seen.

I bowed before her, deeper than had been my habit. Her bearing demanded it. "I beg your pardon, your grace. I did not wish to offend."

When she gave me her hand to kiss, I felt a small piece of paper in her palm. I had played this game countless times with Makala. I took the missive quietly. 

"Are you staying in Cortan long, Commander?" she asked when I rose. Her voice was distant and formal, there was no trace of childishness left in it.

"No, your grace. I am passsing through to my father's lands. I will leave in a few days."

Her grooming had not taught her to hide emotions completely. I saw her face twich at my words. It took her a moment to compose herself. "That is for the best. There is talk that you have been my lover. It is not safe for either of us to been seen together. The children, however, you may visit as you wish."

She left me with that short audience, and I left her children, wondering what strange circumstances had led her to this new form. I was not entirely comfortable with her new presentation. I had wanted to tame her, someone needed to. But the Nisrita I knew was so different from this refined courtly vision that had appeared before me, that I did not know if the girl I had once called a friend still hid inside somewhere. It was for the best, I supposed. She was Cortan's heir. She needed to act it. Her previous previous boyish comporture had been Makala's irresponcible fancy.

The note, when I opened it discretely, said "Thank you for returning, Timmon." I held it over a candle flame, as burning such messages had long become a habit, and watched the flame lick then brown the paper, eventually convincing it to curl, blacken and flake away into ash to protect both myself and the sender.

Nisrita was not angry with me, it would seem. The realization came with relief. I did not want to leave her with anger between us. I had not done wrong by her, I told myself. The children were Griswold's. They should be raised by him. That was the best possible outcome. That I have lost her in the process should have been inconsequential.

\vspace{.5cm}

I wrote my resignation that afternoon, but did not send it. Instead, I spent the latter half of the day wandering Cortan's barracks. I had lived within this complex for the last nine years. I had fought and risked my life for this duchy for that long. Boys I had trained were now men walking its walls. I had loved here, my first and only love. Makala laid burried under the castle that rose high above the yellowing grasses to the east. Everything I had built for myself in my adult life had been built here. I had once thought that only Makala bound me to these dry dusty plains. I had been wrong. He was gone, but the ruins of my life his absence created were all gathered here, in these few square miles.

Twilight found me in the officer's hall. Friends and colleagues met me with a cordiality covered with a sheen of distaste. I was a commander without a command. I was a part of the great machinery of Cortan's army that did not have a job. My past decorations held little weight if I was a man without a future. I bought them drinks liberally to wash away the disdain, and did my best to keep my temper in check. I would leave tomorrow, I thought. There was nothing for me here anymore. 

Sunrise came too early, and found me too unwell. Nisrita's invitation to visit the new dukes as I pleased rang in my mind. I would not dishonor the new dukes by appearing before them in this state, I decided. I went to watch the new recruits train. I knew the man in charge of training the boys. Sergent Matteo Londi, a man of twenty two had passed through my hands eight years ago.The ambitious boy I knew then had turned into a strong authoritarian. Sergent Londi worked his charges hard, his skill with the sword and axe were greater than mine. But he played favourites. I watched him consistently pick one or two of the stronger, abler boys for demonstrations. That type of behaviour just breeds resentment in the group, where they should learn to work together as one body with their cohort. There was nothing I could do. I had given up my post as trainer long before I lost my command. It was not my place to interfere. Cortan army would train its recruits without me, just as it would win wars without me.

I watched the youths go through their excersizes until I felt like I could present myself to the young duchess, held the crying young Makala in my lap, exchanged a few words with his mother, then found myself riding in the familiar terrain around the castle where Lukos and Makala used to hunt the small deer peculiar to these lands. Twilight found me in the officer's hall again, my resignation letter still unsealed and unsent.

This pattern continued until the new dukes had completed their first fortnight of life. I did not receive an invitation to the passage ceremony, which should not have surprised or hurt me as much as it did. I found the courage, during the ceremony, to leave my letter for Duke Ergino's inspection after he had returned from his celebrations, and gathered my belongings to leave.

"Are you deserting her again, Commander Romino?" I turned to see Nisrita's friend Carlotta, now wearing the badge of a fully trained healer standing in the center of my way out of the stable. Her normally polite but reserved tone towards me had turned icy cold and accusatory. 

"I am no longer a Commander, healer. I have resigned my post to return to my father's lands." Carlotta said nothing. I should not have felt a need to justify my actions to her. Nisrita knew that I would not stay for long. I was deserting no one. "My presence is generating talk at court. I am only presenting a problem for her grace," I explained.

Carlotta's eyebrows arched sharply. "Is that what you think, Mister Romino?" Had she been a viper, there may have been less venom in her words.

I felt myself lose my grip on my temper with the healer's tone. "It is what she has told me, and it is what I have seen. Do you deny it?"

The young woman responded to my raised voice with a yet icier tone. "The duchess is ill." She spared my shock at this news no sympathy. "You know of what I speak, Mister Romino. Do not feign surprise. The Tower cannot help her. The only salve we have found for her madness is her love of physical exertion, embarrasing for the court at best, an outright impossibility for the last two months. Your appearance seems to be a second balm."

I nodded, easing my weight onto the stall of the stable. I knew what ailed Nisrita. The tower could not touch this malady. I had seen many soldiers suffer from it. I had hoped, once, to cure her of it by letting her rest in the sheltered, quiet life of my father's house. Then I left her. No wonder Nisrita's friends accused me of desertion. "Can you arrange an audience with her?" I asked.

"You?" Carlotta laughed. "No."

I ran out of patience. I would not be teased and balked by a woman, of the Tower, or no.  "I know more of this malady than you do. Why tell me all this only to keep me from the duchess?"

"Because," the healer continued with a sardonic smile, "you seem to be incapable of maintaining the composure necessary to attend the ill."

I nearly hit her. If a Tower's woman cannot behave like one, she did not deserve the respect given to that sex. I pushed past her instead, leaving Carlotta to think what she would of me. She was not right. It was not true that lost so much of myself this year that I could not be of service to the duchess. She had no right to judge me or what had passed between Makala's wife and his lover. I found a tavern beyond the barracks and the castle that would serve me wine, and drank until I had calmed my nerves. 

I did not leave Cortan that day. I recalled my letter form the castle before the day's celebrations ended, and considered my next actions. I did not have access to Nisrita during her confinement, certainly not while she stayed under her friend Carlotta's thumb. I would simply have to wait until she had recovered. That meant that I would have to find something to do with myself for the next few weeks. 

Any squadron leader or sergent worth his rank knows the importance of not leaving wounded companions behind. Nisrita may have given up calling herself a soldier, but if I had not let her go wounded and alone into the world while she still served Cortan's army, she would not still be ill now. I would not leave her again.

\vspace{.5 cm}


The next three weeks crawled by. I found little work to keep myself occupied. I had been stripped of my command. I was not to be trusted with tasks that required the coordination of more than a dozen men. I did not have a place in the natural order of the military. I was to be left to cool until I had proved that I was fit to return to service. How could I prove my worth without work? 

My situation left me scrambling to find positions with small intelligence missions, or the mundane task of rounding up ruffians and theives from our farmlands. Sergent's work. I could not bring myself to take most of them. I would be answering to men who had once answered to me. I had not sunk so low. There were a few pieces of work I found to occupy my empty weeks. One involved finding a band of petty theives and arsons in the part of the eastern hills that I knew particularly well. Those hills were a newly accquired territory when I had joined first joined Cortan's forces. I had the fortune to be assigned intellegence work in the area. As I was promoted, I voluneered myself and my forces to help keep the peace there. I knew it already, why waste the skills? I had enjoyed working on the ground with my men then. It was not the menial nature of the work that irked me now, but the fact that I answered to Captain Piento, a man who had been passed up for promotion many times in his career, including once by me.

On two other occaisons, I sought out missions under the command of Captains that had come to Cortan from other duchies, hoping to make their fortune in the expanding border land. I could not take Piento's sneer. It was better to be under one who did not know me at all.

That kept me away from Cortan of ten of the twenty one days that remained of the duchess's confinement. The rest of the time, I drank and brawled, slept and sobered, saw the two young dukes that were not mine, exchanged a half dozen words with the duchess and did not think too hard of the future. I could not learn much of her life during my absense. She had not returned to Cortan until a month before her confinement. She was already the personification of Cortan's pride and greatness when she returned. People seemed to either not to speak well or ill of her appearances in court. She presented herself well, held no opinions, was an average conversationalist, and an adequate hostess. Of her habits or actions, there was absolutely no news.

I returned to my rooms far later than I should have a day or two after Nisrita had returned to the castle, when I seamed to have scared a young soldier doing Destroyer knows what in a dark corner behind the kitchens. The boy took off running. Certain that he was at best having an affair, and at worst committing some act of vandalism of theft, I gave chase. 

I encoutered him amid the grasses at the other end of the yard, cooing to an animal. "There's a good girl. Now if you had only come when I had asked you nicely, I would not have had to hurt you. Don't come any further Timmon. It is late, I do not want to chase this dog a third time tonight."

I stopped. I felt a simlutaneous surge of relief to hear Nisrita as familiar unpolished self again and a spasm of fear at the sound of her cooing, in a very motherly fashion, to the creature she held cradled in her arms, purportedly a bitch. Was this a manifestation of Nisrita's illness, or could I explain this as the less dangerous madness that the Tower breeds in her. At any rate, I waited for her lullabye to cease before approaching.

"May I ask what you are doing?" 

"Putting an unconcious dog in a sack," she answered as if she were doing something as ordinary as combing her hair. "Actually, Timmon, could I ask you to take this to the Tower. Tell them it is for me. The night staff in the infirmary will know what to do with it. It is late, and she is heavy, and I ran more than I should have today." When I hestated in my confusion, she added "You are sober enough to do this?"

"I am perfectly sober, duchess." I bit off.

I heard her sigh in the darkenss. I saw her shadow rise. "Which is why, commander, you could not catch up with a mother of five week old twins." I saw her hoist a bulky shadow to her shoulder. "Never mind. I'll take care of this myself. Good night, Timmon." She walked off in the direction of the Tower.

Something was very wrong. This was certainly Nisrita before me, but not as I had heard her since the ambush on route to Turina where she had killed her first man. "You are right, Nisrita. I am not sober, but I can deliver your beast." I heard her stop, and watched her shadow lower the burden from her back. "You should not be here," I ventured, speaking as if I were approaching a wounded animal. "What will the court think if it learns that the young duchess is chasing stray dogs in the middle of the night?"

"It will learn that the sixth year student assigned to capture this dog had failed miserably in his task and is currently being punished for it. Furthermore, the court will not learn of this matter." The last was a command. 

I wanted to laugh at the situation I found myself in. But the undercurrent of this conversation was too serious. Carlotta had said that Nisrita turned to physical exertion in times of trouble. "Your dreams are still haunted." There was no possible other explanation for this queer midnight foray.

She stood in the darkness for a long time, contemplating my statement, or her recent dreams, or the silouhette of the Tower against the sky. I could not say. When she spoke, she her voice sounded hoarse and small. "This is not the time for this conversation, Timmon."
 
It was not an outright rejection. I pressed what little advantage I had. "Can I see you at a time when we may talk?" 

"No, Timmon. You know why that is impossible. I heard that you had changed your mind about leaving, for my sake. That was very kind of you. I wish you had not. You should have done what you needed to gather up the pieces of your life. I am sorry I could not ask you to attend the dukes' passage. I could not for the same reason I cannot meet to talk over old times with you. Good night, Timmon, and thank you."

She walked back towards the castle. I took the dog to the tower. I was certain that Carlotta had a hand in her feelings towards my presence. She was a meddling troublesome girl. She had decided for reasons of her own that she did not like me, and was set on making the duchess feel the same. Who else would have told Nisrita that I had planned to leave three weeks ago. Pick up the pieces of my life indeed. I was not chasing dogs in the middle of the night to escape the demons of my dreams.

\vspace{.5 cm}

I was ready to turn in my resignation the next morning. There was not doubt in my mind this time. It would be better if I left. I could not return to my former carreer by begging for odd assignments from men I outrank. Sitting still only drove me to worsen my case. The reason I had extended my visit did not wish to give me her time. If Nisrita would not cooperate, there was no reason for me to stay behind.

But I found a message from the duchess requesting that I attend her that afternoon. She had been kind enough to give me plenty of time to make myself presentable. 

I found her in Captain Dielo's office. The room was decorated with images and and trophies of the black riders. This was the office of their captain. It was as large and well appointed as that of any commander. To lead the black riders is no small honour. One wall was devoted to the names of those that had fallen in service. I saw Makala's name lie third from the bottom. That high valley in Turina had claimed many good men's lives. 

"Good day, Commander Romino, I am glad you could come." Captain Dielo rose to greet me. Nisrita stood, the courtly image of her I had learned to become accustomed to over the past month, rather than the boy in soldier's clothing I had tripped over last night.

"Duke Ergino," the captain started, when I had seated, "has decided that a second trip to the Velta valley is necessary. Since I am known to Prince Keno, he has asked me to organize the mission. I have selected a small group of Black Riders for the task. We will be traveling as emmisaries to the land, this mission is purportedly a diplomatic one. I am searching for a diplomat and a military officer to travel with us."

I was not in a position to turn down ways to serve my duchy. Yet this was a strange request on many fronts. I started with the strategic. "Pardon my questioning of your purpose. Why does the Duke wish to reinvestigate the Velta now? A year ago, he had scouted it as a possibility to put before the crown in several years, five at least. Has he changed his mind? Our resources are still fully engaged in Turina. The kingdom of Lir is wealthy and powerful. Cortan is not in a position to propose funding a war there yet."

"There is not yet a proposal before the crown, Commander Romino," Captain Dielo corrected.

"No, captain, there may not be one on paper, but we both know your mission is the prelude to its drafting. A war with Lir would leave our flanks and our supply trains exposed to Niev. This is a rash expansion. If I may ask, who is behind this." I no longer had the ear of the generals at the Duke's table, let alone the Duke. Where once I may have been consulted on planning a foray such as this, I found myself begging for information from the mouths of underlings. I did not know why I could not cut myself off from this constant humiliation.

"Duke Griswold has a expressed a desire to learn of the fertile valley," Captain Dielo responded.

That answered one question while posing another. If Duke Griswold had a plan for taking the valley, I would trust it to be sound. It was a more aggressive move than his nephew would make, but Duke Griswold had helped found Cortan with his brothers over forty years ago. "If you will be traveling with the Duke, why do you need me for rank and diplomacy?" My current position in the army held no power, and even a healer such as Carlotta saw the value of my diplomatic skills.

"No, commander. The Duke is the inspiration for this mission. You will not be travelling with him."

Nisrita spoke for the first time this interview. "Prince Keno has had an unfortunate experience with a certain Commander Griso, commander."

That explained the Duke's absense, at least. If the Hundred Horsemen had dealt poorly with the Prince, or even fought against him, Duke Griswold could not show his face in the valley. "But why me, captain. Would you not be better served by General Galderan, or a man of Cortan's court? Baron Sylvo, perhaps?"

"Excellent men, commander. You were once better qualified than both. I have never had the opportunity to work closely with you. I am curious what it would be like."

A pity position then. I contemplated whether my self esteem could endure such condesention, when Captain Dielo whas abruptly called from his office, leaving me alone with the duchess. I looked at her evaluating me. This was her doing, I realized. She must have her hand in this matter. There was no other reason for her to be in this room. I wondered at the change these last several months had wrought. The girl I knew feared politics. I wondered if she would answer me directly, or if that too had changed.

"Is there more I should know about this situation before I give Captain Dielo my answer, your grace?"

"My husband thinks he, and the Hundred Horsemen who have settled in Marsea, still know enough of the valley's weaknesses to have an easier time taking the valley than any other duchy."

That was a surprising ammount of information for the duchess to have. "How do you know this, your grace?"

The duchess's lips curled imperceptably. "The Duke is  a boastful man. He has made no secret of his ambition."

"And you wish me to help him realize his desire."

"No. I believe you to be a different type of man than my husband. I wish you to draw your own conclusions."

I stared at her. Nisrita wanted my help in working against her husband. She was young and inexperienced, my position precarious. This was not a game I would be drawn into. "I took an oath of loylaty to Duke Ergino, your grace. My conclusions will support those of his uncle."

"I am aware of your oath to Cortan, commander. I am also aware that you are a brilliant young officer who has shot through the ranks of her army on the strength of his intellegence and insight. My two husbands have both seen the Velta valley. They have come to very different conclusions. Surely, a man of your abilities can come to your own."

She was so cold in the way she reffered to him. She barely mentioned him at all. He meaning was clear, even if her tone was heartless. Makala had seen something of an unspeakable hope in Lir. She wanted me to see it for myself, then decide whether it was worth challenging Duke Griswold. It was a generous offer, and an interesting idea. I did not know that I still had the courage. I could not consider it here. I could not think of Makala in front of his cruelly formal widow. "May I ask you a few questions before I decide, your grace?"

"You may continue asking as you please, commander."

I started delicately and broadly. She had found time for me as I had asked, last night. I did not wish to frighten her off. I continued the fascade of formality."You appear different from when we last met in Deyalorn, your grace. May I ask what has transpired?"

Her jaw set harshly for a moment before she answered. I feared I would be yelled at for deserting her. Her voice, when she spoke, remained cold. "Much has happened in these last eight months, commander, chose a less broad question."

Very well. "Your carriage and dress then. You have refined it greatly since we last met." 

"I am Cortan's heir. Having spent a year embarrassing my parents with my behaviour, I could hardly be a considerate daughter and wife without attending to those details."

She left unsaid that she could not endure her mother's displeasure without Makala to protect her. I should not have needed her to tell me this. I continued with a less obvious question. "And your talisman, your grace?"

"My old one is damaged." 

I watched her carefully. She had not met my eyes as she spoke. There was more to this story. Her answer made no sense. She had put her talisman away at my father's house. She would barely have used it between then and when I saw her. Nor did her answer explain the new very visible style of wearing it. "How?"  

Nisrita fidgeted as she explained "My husband broke it." She paused. When she continued, he had regained her cold composure "I used it against him."

"Why?" I cried, rising from my seat. What would possess Nisrita to take such a dangerous act. Domestic disputes be Taken, I did not expect Nisrita to be a tame wife. This, however, risked her life.

The duchess mistook my anger. "Do you question my domestic life, commander?" Her temper still remained, I noted. It had just been polished to burn cold instead of hot.

"No, your grace," I replied, reprimanded. I understood why Carlotta had been so worried as to intervene on the duchess's behalf. When I had left her, she had been reckless enough to run away, but I had not thought her the type of woman willing to play with her life. If she had deteriorated ...

Nisrita interrupted. "If you were to question my domestic life, commander, whose side would you take?" She was not a child seeking help from a friend. She was a duchess, potentially plotting against her husband. I would not be drawn into this.

"I would chose Cortan, duchess. I am a soldier. I serve my king and my duke."

If my answer displeased Nisrita, she did not show it. "Loyally said, commander. But duty, as we both know, is not always straightforward. Have I answered enough of your questions for you to make a decision?"

"You have, your grace. I will ask Captain Dielo to find someone else." I would have to let Makala's vision rest with him. I was no longer the man to help Cortan make her military decisions.

"Really, commander. That is a pity. I would have thought that the rising star of Duke Ergino's army would be the best candidate for such an mission. General by thirty, I have heard some say."

She was taunting me. She was cruel and malicious and taunting me. I did not have the heart to be angry with her. "I am no longer that man, your grace." I bowed my exit. Neither did I have the stomach to endure more this interview. 

"Prove it, Commander Romino."

Her command confused me. I halted my retreat. "Your grace?"

"Prove that you are the washed up man at the end of your carreer that you think you are. Go with Captain Dielo to the Velta, see the country that had inspired and moved Makala, and destroy Mersea's relations with Lir by your incompetence. Bring about my husband's desires, and trample the hope Makala held on his last night alive. Then, and only then, will I believe you to be the man you say you are."

"Good day, your grace." I left her. I did not have the strength to face her. She had learned to be manipulative. She schemed and plotted as her first husband had. Whoever had taught her this new trick had neglected to line her iron fist with Makala's velvet glove. She drove me to ride with Captain Dielo at knife point, where Makala would have walked me there, grinning and eager to prove my worth to him. 

\vspace{.5cm}

\begin{comment}The Velta Valley was the northern most strip of land of the kingdom of Lir. I had encountered their traders a few times as child in Deyalorn. They would come to the winter fair with exotic warm spices and expensive dyes. They would leave with ivories from the the traders from Szarvis, and perfumes from the southern duchies, such as Gissal. I never saw their caravans pass through Cortan. Their route took them further west of those lands. I have never seen their lands. 

Makala, had, if anything understated their riches. We were met in Turina by an escort worthy of the great Griswold himself. We traveled down the Pensid mountains, through days of hamlets and herds of mountain goats claimed fully neither by Niev or by Lir, until we came to the head of the Velta river. A further days travel brought us to a fleet of river boats. The noble houses along the Tulsi have river boats they use during the warm summer months. They are simple structures with a small sheltered alcove to shade on from the midday sun, or keep one dry from rain. They were not meant for long distance travel. It would not be pleasant to stay in such a craft overnight. As an afternoons outing, they are very pleasant. Training as a boy in with the recruits at Deyalorn, I have had several opportunities to indulge in this pleasure. 

Those boats were fisherman's dinghies compared to the fleet that met us on the Velta. These were wide and flat bottomed. Each had a small house built on it. Each house had three rooms, one for entertaining, two for living, and small galley. One could travel for weeks on such a craft and never want for anything. Our enterage of thirty two men were piled onto a dozen such craft, their servants, secretaries and pages piled onto a more familiar barge for transporting people and goods up the Tulsi. We travelled mostly by night. During the day, we would stop in the houses of local noblemen, refresh ourselves, listen to their daughters play beautifully on stringed and wind instruments, and see the luxury of the Velta's people. 

No sofa was complete without silk cushions, no curtain without gold embroidery. The vilagers and farmers we passed were tall and healthy, greeting us from their lands in cotton shirts and dresses died in brilliant hues of green and yellow. This was a rich land, with four crops a year. They had found a way to measure the rain and rid themselves of famine, I was told by my escorts.

Some of this was a performance put on for my benefit, I was certain. How much I did not know. Certainly, their sailing technologies were far beyond anything Marsea knew. They had better knowledge of astronomy and navigation that Marsea did. I did not at speak their language, I could not gauge their literature, but the music I heard in the halls of my hosts surpassed in beauty anything I had ever heard in Marsea. Where we had schools training children in the gift, they had schools training their children in the arts and mathematics. The first was a decadent luxury. The second was crucial for the development of what I would later learn was their naval strength.

Their horses paled in comparison to those of my home. Most of the farm labour was done by great black buffalo, slower than horses, useless for anything other than pulling a plow, but they were strong. A pair or Liri buffalo could plow as much wet water logged soil as a pair of our horses could in the fertile Tulsi valley. I wondered at their strength, though I could not justify brining them home for anything other than a curiosity. While it was useful to breed a beast that could pull a plow and put milk on the table, they lacked the versatility of horses.

Given the awe inspiring wealth of the Velta valley, I was not surprised to see the decadence of the doubled walled city of Lesoko, where Prince Keno held court. The walls were high, but the stones were soft. They would fall easily to our catapults. Double walls not withstanding, the land defenses of the city did not overwhelm me. I thought I understood why Duke Griswold thought he could take this land easily. I wondered why other duchies had not ventured forth to claim the riches of this valley for their own. Our boats floated beyond the city gates, and entered a large quay. Then I understood. Lesoko's defenses were not the walls. It was the river that curled around over half the city, and the war ships harbored just inside the gates. To take this city, we would have to fight their ships. We could march from point to point, but they would meet us at every ford with their armies of oarsmen. Progress would be devastatingly slow and costly. If Duke Griswold had a plan, I did not see it. Our strength lay in open plains where our horses were unstopped, or in hills where our infantry moved faster than anyone elses. We could not fight in water.

Even if we were to take these lands, the Tower of the Velta would have to learn to fight and heal differently. This was a land based on a network of rivers and tributatiries. To keep it, we would have to learn to use the waterways as the Liri did. To support a navy, even a freshwater navy, our healers would have to learn to rescue wounded drowning soldiers. I did not even know if the gift could cure someone of having taken on too much water. It would require a completely different, radical change in our tactics. I could not see it happening over the course of a few years. But I was not Duke Griswold. I did not know what he had planned.

We stayed for two weeks in Prince Kento's palace, but saw little of him. We were clearly the honored guests of  a man who knew the dangers posed by sudden visits from distant aggressive lands. Aside from his company, we were spared no courtesy. Or accomodations were lavish, our questions answered. We met with governor's and priests and tradesmen and artists as we pleased. We were granted access to his city, as long as we remained accompanied by people of his court as well as translators. 

Our engineers told me that LIr's war machines were weaker than ours, Captain Dielo discovered enough to surmise that if we were just to fight Lir's land army, it would fall before our own. Our traders reported that Lesoko seemed to be stronger in fabrics and woodworking than Marsea, but they could not compare to our skills in stonemasonry, or ironworking. I learnt that the prince had an immense court. Every second son of a noble house served for ten years in the court of either a prince or the king. They kept records on absolutely everything, from rainfall, to farm output, to births of serfs, to the size to the herds of livestock and flocks of geese on each nobleman's lands. It was an expensive and fantastic endevour, but in return, the army of accountants could collect taxes efficiently and accurately with hardly any effort. That was a system worth investigating and bringing back to Marsea.

I wished I had access to maps of the valley. I had in my possession the rough picture that Marsea had of the region, from traders and the few surveys of the region that we had conducted in the past. It was not enough to work with. It was impossible to put together an accurate description of the valley, and of Lir without knowing where cities I had heard of lay, where roads led, where quarries and forests could be found. Perhaps Griswold knew enough of this to plan a battle, but I did not. It was frustrating work, trying to put together a picture of this valley detailed and accurate enough to be believeable, and convincing enough to lay before Duke Ergino, or perhaps the crown, to ensure that Marsea would leave Lir alone. I needed access to the Prince, and that, as a mere commander, and third son of nobility, I did not have. I was left with my peers in Lir's court. It was frustrating work, but a sweeter sort of frustration than I had encountered in months. I took pride in constructing the story I would return home with. It had to be complete, and thorough. I would be recommending an action against the wishes of the great Duke Griswold, and I could not reveal my motivation. 

I had spent much of my first few days in Lesoko visiting its temples. They were large open air complexes with great domed gates made of marble with beautifully carved figures spiralling up to the apex. Inside were vast fields of marble tile, still geometric ponds set with precious stones or lush fragrant gardens, each demonstrating the glory of a chosen god. Etched silver plated the inner walls of the temple complex, telling in pictures the stories of the dieties they housed. The display of wealth was amazing, the stories they told even more so. Warriors fought, arm in arm with each other against hordes of demons, while gods bestowed gentle benedictions on them, and granted them great strength. Gods came down and coupled with women to give the Lir a hero unite their kindgom, but they also came down to lay with heros to give them courage to face the demons that lived in the sea. 

I found myself weeping at the sight of a great bearded god bending down from the clouds to speak to a king with his eager army at his back. The deity looked for all the world like he may kiss the mortal king. I did not know if it was my imagination that gave these fantastic interpretations to the scenes before me, or if there was something to these gods that made them kinder and more accepting. I could not believe that the Maker was stronger than the pantheon I saw before me. If it was the Destroyer's will to chase these gods from their lands, he would not do so by my hand. Stronger or not, the Maker had seen fit to make me a miscast, one to be shunned and cast from his sight. In these lands were gods that had embraced my kind. What loyalty did I have to the Maker that I would serve that triumverate in the face of this kindness.

I struggled with myself for days. Had I really just declaired, even to myself, my infidelity to the gods of my birth? I had felt no strong love for them, this is true, but aloofness and disloyalty are two very different stands. I needed to know more about the gods that ruled the lands south of the Pensid Mountains. I wondered if Makala had stood in this very temple and been moved beyond words at the hints that a different set of gods could be better than the ones we held dear. It was treachery of the worst kind. The constant demanding allure of the thought was stronger than anything I had ever experienced. It was stronger than my youthful desire to see the world beyond my father's lands, stronger than my desperate need to be alone with my duke during those first heady days of our love, stronger even than the despair that led me to climb Mount Turin, only to be rescued by a vigilant goatherd's wife. To think the gods of Marsea inferior was to think that everything my family loved to be inferior to that of this strange new land that did not know me. My father, though I could not tell him my situation, did not deserve that disrespect. And yet. The image of two warriors meeting their enemies arm in arm, back to back, what could Makala and I not have accomplished together if we had been allowed on the battlefield as one. Who would fight more fiercely to protect his duke than I, if only I could have stated why I fought. After three days of indecision and turmoil I decided that I needed to know the stories of these gods. 

When I asked in court \end{comment}

***

"Carlotta is angry with me, Makala. She really shouldn't be. Its not as if I have done anything wrong. Not really. All I have done is buy a house. What is the use of being a duchess if I can't do something as simple as that? I didn't even use the duchy's money. I sold some jewely and the books the Tower had given me as a wedding present. I have the man himself living with me, why do I need his books. It fetched a nice sum, enough to pay for half of a house, and borrow the rest from the treasury. It's a modest house, but I like it. It has two large rooms and a smaller one, as well as a little yard. I chose it because it has a lovely view of the north face of the tower. Maybe I'll take you there sometime. You might like the walk. We would show the court what a devoted mother I am. But  you would know that we are going to see this house that I scandalously bought. 

"As your aunt Chamila says, pooh on scandal. I cannot bring myself to talk like you aunty, no matter how hard I try. It just won't come. I can't say pooh in public. I can say it to you, but this is hardly public, is it? Your nurse and your older brother are asleep, and you are curled up in my arms trying to figure out whether I am going to feed you. Sweet Makala, it is so easy talking to you. The only hardship you know in this life is that my breasts do not give you milk, no matter how much you paw at them. Stay that way forever, my precious child. If you can do one thing for your mother, do that for me.

"You know what? I think she is jealous. Carlotta wanted to be married and pregnant by now, but she isn't. She's been asked to wait. I do feel sorry for her. No, really, I do. I don't care if you pull my necklace in protest. It isn't right. If she just had to wait for one year, that would have been one thing. It would have been hard, but a year goes by fast. You'll just have to trust me on that, won't you. A year must seem forever to you now, but it is true. The world can change in a year, but it still goes by fast. You can stand at the other end of it and wonder how it all slipped away. Or, as I have been told countless times by Carlotta, I can look forward, and think what to do from here. 

"What will you be doing in a year? Talking I suppose. It unimaginable. My gift is so much better at healing wounds than nature is, but when it comes to making a simple seed sprout, I am helpless compared to nature's gifts, let alone helping a fetus become a life. I want one of your first words to be Timmon. That's too hard. Immo then. I want you to know him and love him. He's the only good man in this court. Can I tell you a secret? You came a hair's breadth away from calling him father. Would you have liked that? You wouldn't know. I would have liked that. I would have feared for you less then. Your father is a hard man. I don't want you to grow up to be like him. You would have liked calling Timmon father. You might have been a great general's son. Now I don't know who Timmon will be.

"Hush now, don't wail. You will wake up your nurse, then I won't be able to talk to you anymore. Are you soiled? Let me fix that. If you keep crying, I will have to go away to sleep with your father, and you will have to sleep with your brother. Isn't this better? I'm a much better companion than your brother is, though that will change with time. Carlotta thinks I'm insane. I'm not though. Don't believe her. I just have some trouble sleeping. There's nothing wrong with talking to my son in the middle of the night, is there? It is supposed to be what mothers do. There, now, isn't that better? You can stop crying and go to sleep in my arms. It is a warm night. I'll rock you. We can rock and talk about the world around us. No one else listens to me when I talk. I'm glad you do.

"Do you remember the rabbit I brought for you last week? Of course you do. I have her fetuses now. All six of them, nicely preserved in a jar. They are a bit small, but rabbits breed all the time, and they have large litters. It is easy to study them. It is amazing how similar the rabbit pups look to the dog pups. I wonder if you looked that way too at one point. No, you are right, that isn't a nice thought. I'm learning a lot though. That is nice. I think Horatia Joris was right. Well, I can't say anything for certain about her theories on healing. I haven't actually tested her suspicions on a living creature, but looking at these creatures, and the similarities between them, it makes sense. I will have to try on a duckling or a kitten at some point. There is time for that later. I can't do that with your father still here. He'd be furious if he found out.

"Did I tell you about Horatia? I must have. No, that was your brother. Listen carefully then, I'll only say this once. Horatia was the head of the Women's Tower in Deyalorn about fifty years ago. That's a very powerful position. I once wanted to be like her. She wrote a book on women and childbirth, where she studied how different animals gave birth, and what happened to different creatures if they were born too early. She had this theory that the Maker made us all in the same image in the womb, and that it is the mother's juices that make us different. She went so far as to speculate that if one could put a dog's fetus in a horse's womb, just as the fetus was forming, then you could turn that dog into a horse. Impossible to test, of course, but a fascinating idea. No, don't spit. The church didn't like it either. They denied that all creatures are the same in the womb, there is a hierarchy of creatures, and a man is always a man, and can in no way be confused with a dog. The priests have never cut up a rabbit and a dog to see what they hold inside. What do they know?

"You have to promise not to tell anyone this now, my son. No one can know. Least of all, your father. Don't worry. Your mother is not a heretic. I will not publish my findings to back up Horatia's claims. I don't need to be condemned for my beliefs. How the Maker made us and why is not my concern. He is great, and I trust in his work. If the priests deny the evidence that is before us, there must be some simple misunderstanding. That is not my fight. I work for the Preserver. After the Maker gives a life to a creature, it is already a dog or a cat or a man. That is all I care about, and how to keep them alive. Not how they started inside the mother. But, and here is the bit you have to promise not to tell, but if they start out the same inside their mother, Horatia argued,  perhaps all creatures starting as identical being, heal the same way in life. A rabit's paw, and a human's leg all started from the same point. There must be something fundamental about how a mother grows a child that will explain how a man or a dog can regrow a leg, if it can be done at all. Okay, now do not tell your father I just said that. He thinks, he is convinced, that the only way an amputated limb to grow back is to bring it back whole, inch by inch, as it first last appeared on the wounded man. Now is that how a lizard regrows its tail, I ask you? No no, I'm not angry at you. Of course you wouldn't know, my sweet. I am mad at your father. I won't yell at you again. Hush, let me tell you about the lizard. A lizard's new tail first looks like that of an infant lizard, it takes weeks for it to grow to an adult tail. I can heal a lizard's tail. When I do so, it grows back the same way as it does in nature. Horatia noted this in her book, though it was far enough after her theories about changing species that I do not think anyone read it. Your father was furious with me when I pointed this out. Men are higher creatures than lizards he said, we cannot be healed the same way. He had laughed at me when I had said something like that about fish. I thought I loved him for his bold new ways of thinking. Don't remind him that he once wanted me to study fish. It will only get you in trouble. 

"He is a stubborn man, your father. You will never convince him you are right when he does not agree. Be careful of that, my sweet. He will hurt you if you aren't. It doesn't matter. I have a house now. I didn't buy it for myself, but there is a chance that we may live there someday, or go there for a few days to rest if your father gets too overbearing. I promised Timmon that I would never run away again. Otherwise, I would take you and just leave. I don't know where I would go. No where as dangerous as last time. You wouldn't like that at all. Perhaps I can find a nice goatherd and his wife, like Groto and Maya, to take us in. I could milk goats and be an extra hand on the farm. You would like goat's milk, especially with honey. It is wonderful stuff. 

"Do you think anyone would miss us? Timmon and Carlotta maybe. Not many others. The duke and duchess would just declair me dead again. Your father wouldn't notice. He is too busy with his new favourite student Ezaro. Have you seen what Ezaro has turned into? Of course you haven't, and a good thing too. The man has gone half out of his mind. Your father is trying to teach him to control the dreams for wild magic. It can't be done, I say. No one can control their dreams, especially not those dreams. Let him try though. Ezaro can have his own mad quest while in his prime, and I will have mine. As long as I do not have to touch the mushrooms again. I pity him, Ezaro that is. He spends eight out of ten days caught in the wild nightmares those mushrooms bring on. One of these days, he will just take too many, and never recover, or your father will not be able to keep him from dying in his dream. Either way, it will be a waste of a good healer. Not that your father would care. That is the type of man he is. 'Ezaro did it willingly,' he would justify. 'Griswold in not responcible for the actions of others.'

"I hate him, Makala, I really do. I thought I was bringing a man to save Cortan. I fear I have brought her doom. No one else sees it. Only I do. They all think it part of my illness that I cannot see the greatness in him. But your father will kill Ezaro, you mark my words, then he will walk away guilt free. Nothing is ever his fault because in his mind, he has never forced anyone to do the thing that caused them to die, just as he has never forced me to kill anyone. I know different. He cannot wash his bloody hands clean on my guilt. 

"Oh Makala, you are sleeping already. How can you sleep when your mother is crying? I haven't even told you what I came here to say. It doesn't matter, I suppose. I don't think I'll do it now. It was good to be able to talk to you, precious child. I think I just need to be able to talk to someone about your father, some one who will not think me mad for hating him. You don't think I'm mad, do you? You just think I'm cruel for not giving you milk. Go ahead then, think me cruel, but don't keep me from talking to you. I think if I can talk to you and your brother at night, I'll be okay."

***

\vspace{.5cm}

I was a changed man when I returned to Cortan. At least I thought I was. Over the past three months, I had only been involved in as many fights, usually over my gambling debts, not due to my drinking. It was the work, I think, and the gods. I cannot deny the role the gods played in my life those three weeks. I had added scholars and tradesmen and engineers to our diplomatic mission when I set out with Captain Dielo. If I was indeed to be exploring Makala's dream, I needed to do it properly. Captain Dielo did not mind. He too, it turned out, did not wish for Marsea to enter a war with Lir, though not for the unutterable reasons that Makala did not wish it. He simply did not think it practical at the moment. He could not, and would not, say against Duke Griswold's wishes, but he was happy when I wrote a report indicating that a war with a Lir, a country rich in swift wide rivers, that uses river boats to move its military and keep its peace is not the type of land Marsea could take and hold easily. Even if, according to Duke Griswold's plans, Cortan spearheaded a successful attack, so much of the country's defenses was built upon clever use of water, that our occupying army would have to quickly adapt to a navy, a change I did not think we could make. 

The report itself would not be enough. I needed there not to be war. I needed it more than I have needed anything since the first few days after Makala's death. My need tested my loyalty to my gods and those of my fathers. Had I been forced to chose, I know how I would have chosen. I perhaps would not have returned at all as a result. I could not return, serving in an army that called itself an arm of the Destroyer, when I no longer prayed to that god. It had been close, however. 

Instead, I returned with two Liri men. One, a boy, barely sixteen, to be my companion and a reminder of the greatness of that country's ways, the other, a man older than I, a cousin to Prince Keno on his mother's side, who would meet with Duke Ergino, and hopefully the Regent Consort to discuss the possiblity of exchanging ambassadors. In return for Sama Gorou, we left behind five black riders and the head of Cortan's ironmonger's guild. I was not doing my career any favours by chosing to balk Duke Griswold. I could not let him threaten Lir's gods. 

I also returned with one copy of one of Lir's holy book, and another of an epic poem that had moved me beyond words even when I heard an ad hoc translation. I did not speak a word of Lir. My companion, Nobu, did not speak a word of Mer. We would teach each other the languages, Prince Ken had decided when he gifted me the servant. I may not ever learn Lir well enough to read the poem in its original, but possessing the book that contained such fantastic deeds and ideals gave me great joy. I am not a man to own books. It gives me pride to say that these will be the first two volumes of my personal library. 

I returned to Cortan, and presented my report and my guest directly to Duke Ergino. The news was recieved as coldly as I had expected. A year ago, I had stood before him to be decorated for saving his duchy. Now, I stood contradicting his uncle's wishes. Whether the Duke still put weight in my judgement or not, I did not know. He did still trust Captain Dielo. The news was recieved and considered. The rest was out of my hands. Let Nisrita and Captain Dielo do what they will. 

When I arrived at my room in the barracks, I found that my room had been emptied of my possessions. My belongings were in my house, the porter informed me. I do not have a house, I told the man. Impossible, commander, the porter replied. The duke's treasury had left the deed and the agreement with him only a few weeks ago. I must have made the agreement and forgotten about it before I left on my journey, the strains of the road being what they are. 

It was impossible. I would not have been so drunk as to have bought a house without remembering, would I have? Yet the deed was there in my hands, including terms for repaying the outstanding loan. It would take half my salary for a year to make good my debt to the treasury. It was a significant investment. I would have to be watchful of my gambling losses if I was not to ruin myself with debt. What madness had driven me to do this? I walked to the property on the north side of the barracks with my tired new servant plodding beside me. He at least did not know that his master had encountered a new and mysterious property. He only wished to know when his journey would come to an end. I tried to put together the pieces of the days before setting out to Lir. The day after my interview with the dutchess, I spent in a fog, much of it drunk. Could I have done such a foolish thing then? I did not remember any gambling winnings that would have given me a house. Try as I might, I could not recall anything that might have lead to this sudden change in lifestyle. It worried me. If I was capable of  buying a house without my knowledge, what else had I done that I did not remember.

The property itself was modest. It sat back a distance from the houses of other officers, tradesmen and healers surrounding the castle and barracks. The distance and a short row of dwarf oak protected my privacy. It had a small lawn, where the previous owner had once put up a fountain. It was not a particularly pretty structure. The deed had said that the house had belonged once to a member of Cortan's carpenter's guild, who had the misfortune to have had six daughters. He had died last winter without heirs. The house itself looked well built, if small for such a large family. It was much more than I needed. What could have possessed me to buy such a place. The beams were unornamented, the door was simple and sturdy. The roof looked high enough to keep the rooms cool in the summer. If I removed the fountain, I could plant a row of trees on the south side of the house to further shade it. I walked around the house, I am certain to my boy's confusion. It was simply built and unadorned. There were, however, curtains on the windows. When I entered, I found that the house was alreay sparsely furnished, though no mention of that had been made in the deed. It had a bed, a table, a chair, and a few necessary items for day to day life. I walked through the three rooms baffled. There was a small cellar. Nobu went down and found a few items that could be turned into a meal. He set to work. I pulled back the curtains in the bedroom. Before me, the White Tower loomed large. It was not a bad view, all things considered. I could have chosen worse, though I did not understand how I had chosen at all. 

The contents of my room in the barrack lay in a neat pile in the bedroom, probably the porter's work. I set about organizing my life in this strange new setting. Nobu interrupted me with a piece of paper. He indicated the spot where it must have fallen off the table, unnoticed when we had entered. It was unsealed and unsigned. It did not matter. There was only one possible source. "You have shown that you are not as you had feared, commander. Congradulations. Welcome home."

I sat down on the bed, stunned. She was a meddling troublesome girl, I decided, suddenly glad I had not married her. A woman that could buy a man a house without his permission could drive him to an early grave as a wife. Yet she had meant this kindly. We had seen so little of each other in the past year, I could not hold this act of generosity against her. I did not know how to accept it either. 

Nobu called me to dinner. The boy had proven himself to not be a bad cook on the road. Given the odds and ends he had found in the cellar, it was a passable meal. I settled into my first evening in my new domestic life. I tried to converse with the boy over food, and watched him as he cleaned up and readied the house for the night. He was tall for his age, though slightly built. He shaved the light growth on his cheek, he would not grow his beard until it could cover his face in a dignified sign of Liri manhood. Nobu was not a strong boy, but he seemed clever and eager to please. It would not be an unpleasant life, I thought, my days occupied with the needs of the army and the worries of court, my evenings, quiet and alone with this lad. I could have a sort of domestic peace. A bachelor's life, to be sure, but also more. 

I tossed the duchess's welcome into the kitchen oven to be consumed the the morning's fires. I found Nobu sweeping the cobwebs out of the long unused house. I put my hands around his bony ribs, under his upstretched arms. My servant startled, dropping the rag in his hand. A set of crooked teeth gleamed white between his shyly parted dark lips. I met them with my own, and led him to the only bed in the house. The Liri consider the companionship of a man nearly as important for a grown man's spiritual health as the sacred bond of marriage. Older men, to frail to mingle with their peers, or who have outlived their closest friends surround themselves with boys such as Nobu for their spiritual, emotional and physical needs. Nobu had spent his life training to care for someone over double my age. The barrier posed by our lack of common language disappeared when I found myself sitting on my bed with the boy standing between my knees. He bent over my face, his lips teasing me as they brushed my cheeks, my eyes, my neck. A desperation grew then exploded within in me forcing my hands to fumble at his belt. The boy laughed then knelt to unlace my trousers.

Nobu's skilled mouth was warm and sweet. I had not known this pleasure since I had lost my duke, nearly a year and a half ago. I gasped and shuddered in the privacy of my house,  safely behind curtained windows. My Liri servant had no concept of a miscast man, only that men need each other for comfort. I dug my fingers into his hair and learned what it felt like to release the fear that had ruled me for nearly half my life. 


\vspace{.5cm}
***

Good evening, Makala, my sweet. Will you comfort your mother tonight? I cannot sleep, my dear. And I have no one to talk to. There you go. Stay asleep on my shoulder. I just need to hear you breathing next to me. 

I tried talking to the other women in the Tower, as we had discussed. None of them are interested. They do not believe in the possibility that your father is wrong. He is a living god to them. They will not even look at the evidence I have collected over the months. I think they all think me mad. They all think I am ill. They all think my dislike of my husband is part of my illness. They think my theories are part of my dislike for him. Carlotta flat out told me to give up my work. She said that my theories and my desire to follow Horatia Joris's ideas is a twisted extension of my hatred of my husband. She does not understand. How could she say that to me? If I cannot talk to my colleagues about our craft, what company but you and your brother is left to me in this world?

I will have to start work on this on my own. I have a fresh litter of kittens. They will have to do. I hate the idea of hurting one so young, but if this is going to work, it will work when the creature is most fresh out of a mother's womb, when the body still remembers how it became what it became. That is what Horatia thinks. I have no choice but to trust her. Your father is wrong, and he will not admit it until I can show others that he is. I don't dare start this for a few weeks, not until he leaves for Bayam. I don't have the courage to do this while he is here. Not if I have to do this alone.

It is so infuriating Makala. He is there in the next room, the bane of my career, the most frightening man I have ever met. I am expected to love him as the court loves him. I am expected to sleep by his side as a good wife, knowing the risk that poses to my art. He seems hardly to care for that anymore. He has Ezaro now. Ugh, when I think of the sad state he has reduced Ezaro to. You know what I heard him say today, Makala? He said that he could not wait for you two to grow up so he may teach you about the gift. I would rather die than give you over to him, my precious. You deserve better than that.

*** 

\vspace{.5 cm}

Life passed quietly and peacefully for the next few weeks. Captain Dielo offered me a position training those wishing to be Black Riders. His time was needed dealing with the new situation with Lir, making sure our efforts had not been in vain. I had never trained archers before, and I had little skill with calvalry. It did not matter, Captain Dielo said. He trusted my abilities as an instructor. 

It was another offering out of pity, I knew. Captain Dielo, possibly under the duchess's instructions, was determined to pull me back from the abyss I had fallen into. Months of traveling with him had let me know and like the man. It seemed that he did respect me, or at least the man I had once been. I accepted the position. It was much better than grovelling for scraps from strangers' tables, and it was good work, if below my station. 

I mostly kept myself out of trouble, though it was harder at home than on the road. The evidence of my failure lay around me, taunting my discipline and Captain Dielo's good will. I had spent so many years in these very fields training boys and watching Makala ride with the Black Riders. It was hard not to wallow in the memories of my loss. There were many days when, had it not been for my ability to retreat to the quiet domesticity of my home and Nobu's company, I would have tripped on my shame and fallen deep into the disgrace I was so desperately trying to dig myself out of. 

I was grateful to Nisrita for what she had given me. Her company, however, brought out the worst in me. It was her illness, I told myself. I had to be patient. She could not help herself. I wanted to find a way to get her away from Cortan, to someplace safe where she could rest and recover. I wanted her at my father's home. Apples would be in season again. It had been a year since I had last tried. Why was it so impossible to help her. 

I did not see the duchess for several days after my arrival from Lir. I found her running laps in the pre-dawn light in the lawn where the new recruits train. It was to be my first morning with the Black Riders. None of the army had yet come out for training. I found the vision terrifying and infuriating. When news of this episode would return to court, it would only make life harder for her. Why could she not control herself. If she had been a soldier in my command, she would have been reprimanded severely for this outrage. Then again, if she had been a soldier in my command, I would have a command, and therefore more power to help her, and she would have more outlets for her frustrations.

"Good morning, your grace. I wished to thank you for your generous present, though I must find a way to pay you back."

She stopped running, and untied the weights from her body. She was nearly carrying as much weight again as she had when she had last trained with me. "It is a gift, commander. You are simply welcome to it." She said through her panting breath. 

"Is there nothing I can do for you in return?" Whether financially, or otherwise, at that point, I was still determined to repay her kindness.

Nisrita smiled warmly at me for the first time in nearly a year. "I lost nothing in procuring it that I would not have given away gladly. I hope it brings you peace."

Perhaps it was the smile, perhaps it was the relative calm of the last few months, but I could not let it go. "Will you at least tell me what I did to deserve this?"

My foolish persistence broke through the thin wall of joy she had built for herself by running. Her manner darkened, and she looked down and her feet. "They need a father, Timmon."

"Who?"

"My sons. I cannot raise them alone."

"Nisrita," I said cautiously, wondering where this delusion came from. Duke Griswold was alive and well. "They young dukes have a father. You are married to him."

"Don't talk to me like I'm an idiot, Timmon. I am perfectly well aware of what I am married to. The Duke is only capable of raising monsters like himself. My sons need a father if they are to rule Cortan like men. You cursed them when you left them a year ago. I do not want you to leave again." She gathered together her weights and put them away, leaving me reeling with the force of her accusation. When she returned, she had regained her serene exterior, though I had not. "Good day, commander," she said cheerfully as she left the yard.

I would not get involved in the duchess's domestic disputes. We were known as lovers. Meddling in this matter would only make matters worse. It was none of my buisiness. Except, of course, it had once been my business. She held me responcible for her misery. I may not have been able to rescue her from this fate,but I could have tried. I returned to my work, a stung and hurting man.

My mood improved greatly by the end of my time with the trainees. They were good riders, disciplined, eager. My job was not to beat into them a habit of discipline and a give their bodies strength and stamina, as it had been with the new recruits. With these men, I was to hone their skill, curb their ambition, train them to be proud of their abilities, but not arrogant. It was much more refined work than I was used to. I enjoyed it.

I shrugged Nisrita's reproaches off that day. I had no proof of Griswold's ugly character other than her word. No one else spoke ill of him. I had just allowed myself to get caught up in a simple domestic dispute with a tempestuous young woman. I resolved to tread more carefully around her.

My pre-dawn encouters with Nisrita did not cease. They occurred every few days, with Nisrita either climbing, or running, or once walking the beams in the ceiling of the armory. It was disgraceful display on her part, and it always ended poorly for me. I told myself that she was not to be held responcible for her actions, but I did not see a way in which I could help her. The encouters left me ragged and raw, either grieving for her and the pain I could not help her with, or angry at myself for giving her reason to make the accusations she did.

\vspace{.5cm}
****

I am so happy, my pet. Duchess Cybeline has given me the yelling of a lifetime, but I don't care. I would gladly disgrace Cortan with my calloused and unthinking, reckless behaviour a thousand times a week for another day like this. And look! you are sitting up. Just five months old! What a precocious child you are, my darling Woldino. Are you going to be a brave warrior like Timmon? You are certainly built for it. 

Here, come into my lap. Yes, you can have my talisman. I know you like it. I won't be needing it for a while. We won't talk of anything unpleasant now, will we? Uma. Of course not.

Do you know what I did this morning? You will think your mother so clever. There was a girl, a washerwoman's daughter, I think. She was playing near the vats of boiling water and lye, or maybe she was helping her sisters. I don't know. She was very small. She fell in. They fished her out, but she was so badly burnt. No, I won't  tell you. I don't want to remember how she looked. 

Her mother had the good sense to bring her to the Tower quickly. I had just left your father's lecture. It was one of his more useful ones. I was going to come home. I had one of my rare appearances in court to attend. They occaisonally have to bring their mad heir out to parade in front of the barons to show that Cortan is not yet lost. I skipped it this morning. I've been barred from leaving my room for the rest of the day. It doesn't matter. I am a healer before everything else. 

The child was thrashing in pain when I saw her. Pieces of her skin remained on her mother's arms when she put her down on the table. The girl should not have lived. It would have taken two average healers nearly a day to make her whole again. She is so young, the pain and shock may have killed her by then.

I am glad I was there. You don't know this, but I am not an average healer. My body is changing. I can pull down more than twice as much fire as the average healer can, and I can sustain it for longer. Do you know what that lets me do? Okay, that's enough. You can put my talisman in your mouth, but you cannot throw it on the floor. Where's your horse?

Oh my precious child, you don't know what it is like to bring a person back from the Destroyer's clutches, to know that I have played my role in the gods' dance, and that I have performed well. I can heal so quickly now. I can tend to wounds that could not be tended to before because of the time it took. How do I explain this to you? If I put a knife in a kitchen oven, it may turn red, but it will be difficult to even bend it. If I put it in a forge, it will soften so I can weld it with another blade. I am the forge. The child's skin melted and reformed before me, where others would have to coax and tear, and try again. And the child will live! Een Een! It is that funny. The child will live! Een Een. I had to hurt her of course. The guards had to restrain her mother when she saw what I was doing to her. I could not have done this last month. Everything inside me is changing so fast. This month, the child will live.

I have to thank your father for this day. Eowe. Yes, that is how happy I am. I am speaking well of your father. His lectures, when he talks of controlling the gift, are actually quite lucid. He has a host of theories about the skeleton and the brain that I think are utter rubbish. At least no one has been able to get any results following his advice. But when he talks about controlling the fire, about focusing it precisely, of making two streams meet at a point deep inside a person's body, Woldino, I tell you, I feel like he has opened my eyes to my potential for the first time ever. Not everyone understands his metaphors. I think Carlotta does some of what he is trying to teach without knowing it. I can't think of any other explanation for why she is so naturally gentle. It doesn't matter that no one else understands. I understand, and because of his lessons, I knew how to run my stream of healing fire only over the girl's skin so that I did not cause her pain anywhere else. 

His lecture was so brilliant today. I felt like I did on the boat to Szarvis. That's what started this all in the first place you know. That long story that ended with you and your brother being born. Wok um. Fine. You don't want to hear about my day any more. Lets walk. Neither of us can leave this room. Put your feet on mine, and tell me what you did today.

***

The Day of Unions came, as it does every year, with its frenzy of pleasureable sins. That year, it also came two days before I was to accompany Duke Griswold and a group of the men he commanded as Commander Griso into the lands west of Bayam to plan the next stage of Cortan's expansion. It was not a command position, but it was that of significant trust. 

I learned that I was no longer in line for taking over General Madriano's position in Turina. The General was an old man. He wished to retire to the comforts of his estate on the border with Firvona. Had I still been under consideration, he may have put his retirement off for another year or two, until I could be promoted. Instead, Commander Morel was promoted and given the lands. He too would ride with us until Bayam, then we would part ways. I tried to tell myself that this was not a disaster. I was young still. I was regaining my position in the army. Commander Morel was a good man. He was the key to gaining Bayam, he had played a crucial role in preventing Niev's retreat on that high plain. He was better suited to life within a stronghold than I was. The news stung.

I was however, invitited before the duke for the formal ceremony of unions. It was the first invitation I had recieved to appear before the duke since I had lost my command. I told myself it was a good sign. My report from Lir was not being taken as a personal act of thwarting Duke Griswold's wishes, but as an interpretation of my honest opinions. Duke Griswold would be going to Deyalorn this winter with our Liri guest to argue his case against my report, and that of Captain Dielo. It should have been me arguing before the crown, not Captain Dielo. The man had been good to me, he had helped me regain my respect. I should not hold these slights against him. 

Nisrita caught my eye during the cermonies. She sat next to her husband, playing with Makala in her lap. The child was fussing. She had bent down conspiratorially to whisper stories into the child's ear to keep him occupied. She looked, happy. Happier than I had seen her in so long. I felt a sudden wave of what I had lost wash over me. They were not my children, I reminded myself. I have no right to that vision of wife and home when their father is alive and can provide for them better than I could. I watched Nisrita give the fussing infant back to his nurse, who took the twins out of the hall. All joy left her face, and the cold composure of the duchess I had seen the few times I had encountered her in a courtly setting returned. I wondered what I was doing in Cortan still. I had returned to say goodbye to my lover's widow. I had stayed to help her. I had bumbled for the last five months watching her only from a distance. I had not been able to do anything for her. It was impossible for me to get news of her doings in court or in the Tower. I could not even arrange to speak to her. I feared aggravating her delicate situation by fueling the rumors that we were lovers. 

I felt impotent. From what I saw of Nisrita, her condition was deteriorating. The Tower could not help her. I was certain that no one in court was trying. I tore myself away from the sheet of ice the duchess's face had become, and watched the happy couples before me exchange marriage chains.

As afternoon wore on to evening, I left the company of officers watching the dancers for the officer's hall. This was usually the point in the day when Makala and I would slip into his room for games of our own. Those days were over. I had someone waiting for me in my house today. He was not Makala, but he was young, and eager to please. His Mer and my Lir were improving to the point where we could have some basic conversations. I would only stop for a few drinks and stories before heading home to Nobu's arms. 

It must have been at least two hours later when two soldiers on duty informed me that they had been asked to take me home. "By whom!" I demanded. I had not done anything wrong. They could not tell me, though they reminded me that it would be best not to fight. I did not need reminding I pushed past them and into the late summer night. The air was as hot and stiffling as the inside of a donkey's ass. I would not suffer to be escorted home. I left the guards at their post.

Nisrita waited for me at my house. Of all the cruel and painful things she could do. Why, today of all days, when the memories were so raw, had she chosen to meet me at my house? "You should not be here, duchess."

"No, commander." She answered coolly. "And you should be ashamed of appearing before me as you are."

Ashamed her cunt. I was not going to be ashamed before this presumptuous girl. I had not asked for this audience. "Why did you wish to see me, duchess?"

"I did not. I could not see you throw away five months of recovery in one night of reckless pleasure."

I was done with her accusations, her meddling, her demands, her refusal to see me. I was done with the frustrated dance we danced, the impossibility of my speaking to her when I wished, the dangerous capriciousness with which she came to me as she pleased. "Who are you to decide that, duchess. I have thrown nothing away."

"Really?" She threw a ball of twine at me. My hand went up to catch it as it rolled on the ground, unwinding itself by my feet. 

"Out." I roared. "I did not marry you last year. Thank the gods I did not. I am done with you meddling in my life. You tie up half my income, you conspire with my underlings to shower pity on me, now you dictate whether I drink or die of thirst. Who do you think you are?"

She did not move. She did not flinch. She did not get angry. "The only person left who still believes you still worthy of Makala's love."

I heard the clap of skin on skin before I realized I had told my hand to move. Her reaction seemed to take a decade to play out. She raised her hand to rub her jaw. Then her hands went behind her neck. She wordlessly unfastened her talisman, took my hand, turned it palm up, put the large cystal trinket in it, and closed my fingers around it. I watched her, spellbound by her actions, horrified by mine.

At last she spoke. "Now I am completely defenseless, commander. Do that again, and show me that you are not the man I think you are."

"Get out, Nisrita." I oredered quietly. "Leave." She obeyed that at least. I sank onto my chair a defeated man. 
 
\vspace{.5 cm}

It did not occur to me until Nobu had put me to bed that I still had the Duchess's talisman. I rose the next morning to return it to her. I had to resolve my argument with the duchess. If I could not help her here, if there was no way in which we could arrange for me to be useful to her, then there was no reason for me to stay. I had reestablished enough of my carreer, that I could apply to an interior duchy for a command. It was not the life I dreamed of, but it was better than this.

I found her in the Tower as I had expected. A young boy in white robes led me to a large room with tables and shelves lined with jars filled with various amorphous forms. Nisrita sat in the midst of this, head bent low over something on the table before her, cutting and pinning it open. "Good morning, commander," she said without looking up. "Give me one moment." It took me years to figure out how she always knew I was there before she saw me. That morning it still surpised me. I contemplated her uncanny ability and her surroundings while I waited. The class jars were filled with a clear liquid. The each contained several pieces of flesh floating inside. I wondered who the poor sots were that had sacrificed their organ's for the pleasure of Nisrita's study.

"How are you, commander. I'm sorry for the state of my surroundings." She looked around her and smiled apologetically.

I gathered myself up to have the difficult discussion I had come here to have. "You left your talisman with me last night, your grace. I've come to return it."

She smiled, and I relaxed. This would go much easier if I had found her in a happy state. "I know. I'm sorry. I'd meant to get it back from you, but I- I did not want to disturb you."

I put the trinket down on the table by her where she indicated, then looked over her shoulder out of idle curiosity. My stomach lurched. All thought of the conversation I intended to have with her disappeared at the grotesque sight before me. I looked from the pinned and peeled form on the table to the objects floating in the glass jar closest to her. "These are new borns, Nisrita!"

"Not quite, Timmon." she continued in that calloused academic way of hers. "I took these kittens out about a week before they should have been born."

"You. Took. Them. Out." I repeated dumbfounded.

"Yes. The cat is fine. The Tower is very good at cesarian sections." She did not seem to understand the horror of what she was doing. How do the healers live with themselves?

"Do you have any human children here?" I asked looking around, expecting to find a jar with the head of an unborn child lurking in the corner. This was beyond grotesque. This was madness. It was the Destroyer's work.

"Of course not, Timmon." She sounded annoyed with me. SHE sounded annoyed with ME. Then she finally looked up at me, confused. "Why would you think I would have human fetuses here."

I prayed, I did not know who I prayed to any longer, but I prayed for salvation from this girl's madness. "You are torturing animals and stealing their unborn pups for your pleasure, Nisrita. If cats and dogs, why not humans? Where does your cruelty end?"

"Torturing for pleasure, commander?" Her tone was suddently icy. "Would you care to say that again?"

I softened my alarm. I would get nowhere by fighting with her today. "What else is this Nisrita?"

She laughed. She whispered laughter for such a long time, that I thought she had finally lost her senses. "This is torture, Timmon? I see. My kind torture when we learn from the beasts. Fine. If I am correct in my theories, and skilled enough in my art, then children maimed by the crippling fever may be able to walk again. That is what I gain from my torture." She put down her tools and rose to meet me. As she spoke, her body seemed to expand to fill the room. I found myself backing towards the door, though she had not left her position behind the table. "And what of your kind? Soldiers torture, I argue. Ruffians and smugglers torture. Healers do not. Your General Madriano tortures. You think Carlotta tortures. She only follows orders. Your General values her skills so much that he will not let her marry until she is past her prime. What does he gain from it? How many more lives does he save with her torture than she saves by healing? A good, innocent, warm hearted girl suffers a lonely life because of your General's desire for a torturer. You know who else tortures? Your great revered Griswold tortures. He takes young, foolish gifted girls onto pirate ships, sells his service and hers a husband and wife team of information extractors. He frightens her into hurting men, boys, that she does not know. He wounds prisoners, forcing the healer to heal only enough so the man may still talk. He does not let her seal the wounds. The prisoner becomes infected and feverish. She, because Griswold will not, may have to amputate the hand. The man begs to die. She leaves the last of the Black Rider's venom she has within he suffering man's reach, and leaves. Why? Because healers do. Not. Torture. The sooner you understand that, the easier this will be."

She sat down again, and turned to the unborn kitten splayed before her. When she picked up her knife, her hand shook. She waited for it to calm. I took a deep breath. Whatever other stories about her conflicts with her husband I could not believe, this one rang true. This was what she could not bring herself to tell me a year ago in my father's woods. There must be more. "Will you tell me about Griswold?" I ventured.

"No, Timmon. this is not a good time." Her hand had calmed. She returned to cutting and proding at the preserved flesh.

"You are ill, Nisrita. You have been for the last year. I am trying to help." 

She would not look up at me. Her hands stopped moving, but her eyes remained fixed on the form pinned to her table. Eventually, she sighed. "You are travelling with my husband tomorrow. From what I have seen of him, he is much the same on the battle field as he is in the bedchamber. When you return, you will know him nearly as well as I do."

***

"Stop that incessant noise, Griswold! There is nothing I can do to help you! Anything I do will just make it worse. You just have to wait until they come out. Oh, now look what I've done. I'm sorry my little darling. I'm not really angry at you. My nerves are just a complete wreck today. Please stop yelling. Here, have my finger to chew. Is that better? Your teeth are bothering you. I know. Its horrible, being a healer and not being able to help you. Hush now. I won't yell any more.

"You must think me a horrible mother, don't you. It is times like these that I am glad I am married to your father. No, don't whimper. Come into my arms, I'll walk with you. If I had not married your father, I'd be even worse to you. Timmon would not be able to give both you and your brother one nurse each to watch over you. I would have to stay up all night while you both teeth, and then I'd be a complete mess for my presentation today.

"Do you like that cushion? Its good to chew, isn't it. Nice and soft. Your teeth will get better soon, I promise. Can I tell you about my presentation today? There's a good boy. The kitten is growing its leg back. Its almost back to where it should be. I had to kill two kittens before I could get it right. They are so fragile. It is easy to hurt them, like your teeth are hurting you, but worse. I only have two left now. I showed Master Adele. I didn't trust anyone else. I told you that last month, remember? I showed her what I had done, and told her what I planned on doing, and she agreed to supervise me. Not that she had much to add. But I am in my prime, I need a tutor. I don't think she liked what I was doing. They all like your father too much, Woldino. None of them would talk to me at all about my work. They all thought this was something to keep me occupied, the mad duchess. Master Adele nearly said as much. But she believes me now.

Noo nii. Yes. There's a good smile. You are happy that someone believes me, aren't you? The leg has grown back almost completely. Its a small shriveled thing, no muscle on it. I have no idea if it will be any good for the poor creature, but the bone has grown back. No one has done that before. Isn't your mother so clever. Master Adele thought so. She apologized for misjudging me. Isn't that fantastic? No, I am sure it is better than that tassle. Master Adele has been picked to be the Head of the Women's Tower, when Master Rosaria retires. Master Adele apologized to me.

Never mind. You wouldn't understand. Play with your tassle. It's probably more interesting. Oh, you want my talisman now, do you? Give me a kiss, or I won't give it to you. I'm so nervous, Woldino. They want me to talk about my kitten in front of the entire Tower. All the Masters will be standing behind me, looking over my shoulder. Everyone else will be in front of me. I've never given a lecture before. I'm not even sixteen yet. What will I do if I make a mistake? This kitten is a month older than the last one. What if that is too old? What if it doesn't work this time?

Thambu? Is that your answer? You don't care, do you. You're just happy to have your shiny toy. You'll still love me if I come back having failed. Well, at least I am glad you father won't be there. I don't think I'd be given a chance at this lecture if he was still here. He'd be angry that someone had proved him wrong. Or if I did give this lecture, half the Masters wouldn't come. Doesn't matter though, does it? Your father's still exploring the southern front with Timmon. Can you say Immo yet? Im-mo. No, Bala is not even close. You'll get there. 

You know who is going to be there though? Ezaro. It's taken him nearly a month for him to recover from the mushrooms enough to make a public appearance. He's been recovering in you father's rooms in the Tower. No one but a handful of masters, and I, know that he's been there. Everyone thinks he'd fallen ill with consumption or something and sent home. He looks much better. He's been eating two meals a day for three days now, but he still has nightmares. This is why I don't ever want your father getting involved in your training. I don't want him to tempt you into wasting away like that. Ezaro is still in his prime. He was a skilled healer even before that. He should be practising and curing, not giving himself bad dreams.

Maybe now that he's better I can talk to him. What do you think? You think you want to play on the floor. Go ahead then. You can ignore me too. No one else has been willing to talk to me these last few months. I've no one to talk to besides you and your brother. That might change today. I'll tell you a secret, even though your aren't listening. You are my first born son. You should know these things. I promised the Preserver last night, that if people believe me now, if people are willing to talk to me about the gift again after this, I'll stop hurting myself. It's a mad habit. It must be part of what everyone calls my illness. I think it is just because I've been so lonely. But I've promised. If Carlotta, and Analise, and Ezaro, and Claudio and Penna will let me join them in their discussions as before, I'll stop my habit of cutting and healing myself at night.

***

\vspace{.5cm}

We returned to Cortan as the first of the winter rains set it. It had been a fruitful mission. The ex-Hundred Horsemen knew eastern Niev well. We made a detailed map, and a plan for the next campaign. The Duke appreciated having an experienced eye look at the information he had gathered over his decade living in that part of the world, supporting his conclusions. I had to be much more careful when presenting him with information he had missed. It surprised me how much he had missed, given how long he had lived in these plains. I was glad, for the sake of our army, that I had been brought along.

I kept Nisrita's words about Griswold being the same in the bedchamber as he was in the battlefield in mind. He was a good leader, not a great leader, but certainly a good one. His was a very different style than General Galderan's, or even General Madriano. He was, if anything closer, as a man, to what I had seen of Commander Lazarro when I had served under him. He was demanding, strict and officious. He inspired by fear. He praised success little, and reproached failure generously. He had a sense of humor that involved humiliating his men. His Hundred Horsemen knew him well enough that no one held this against him, but it was not a habit that would gain him the respect of his men. He was ruthlessly brilliant and cunning however. Given a good set of commanders to sheild the lower ranks from his cruelty, Duke Griswold would make for a good General. 

Nisrita did not have a good set of commanders to sheild her in the bedchamber. I tried to imagine a man like Duke Griswold with my boy Nobu. I shuddered at the thought. Nisrita was not even a boy, tempestous and bullheaded as she was. Women were not built to endure such cruelty. I owed her an apology.

I felt eyes stare at me as I entered the castle. They held surprise, or disgust or admiration. These were not the eyes that had seen the struggling commander depart. I asked the guardsman that took my horse what had transpired. 

He blushed as he took the reigns. "Everyone knows, commander, about you and the duchess."

My heart stopped. What new villanious rumour was this? "Then why are you blushing, soldier. There is nothing to know."

The man stammered. "I- I  be-beg your pardon, commander. The duchess has admitted to much herself."

My ears betrayed me. They must have. Why would Nisrita lie about this? Had she lost her senses? She must have. I gulped and stammered until I could find a responce. "The duchess is ill, soldier. One cannot believe everything she says." I watched him take my horse away. I did not understand what happened. Was she trying to destroy my carreer? Was she trying to destroy her marriage? She was a disturbed, erratic, impetuous girl, but this was ... "Romino." someone called urgently. I turned to see four strong hoofed legs charging towards me. I barely had time to register that Duke Griswold rode them when I fell. 

I awoke strapped to a healer's table with the familiar agony of healing. When I moaned someone put a leather strap between my teeth. Better to bite that than my tongue. I kept my eyes closed. I found the pain easier to bear that way. I must have broken my leg. When the burning ended, I opened my eyes to see Nisrita and a few others fussing over a splint on my leg. I closed my eyes again and sighed. This girl had caused me so much trouble, yet I could not leave her. She would be the end of me yet. After a moment, the pain started again. I protested. "I do not need to be fit to march tomorrow. Leave it, healer."

The burning stopped. "I am sorry, commander. I had not meant for you to be injured. Please believe me."

She sounded pentinent. I groaned, and uttered the old refrain. "You should not be here, Nisrita."

"If you let me work on you, I can have you well enough to hobble home by the end of the day. Captain Dielo has arranged for guards to be posted at your door. It will be harder to keep my husband from you in the infirmary."

A broken bone takes a week to heal during peace time, three days in a battle field. I did not believe her noble offer. "Even you cannot do that, Nisrita. Save your strength. I'll manage here."

"I can, Timmon," she insisted. "I've entered my peak strength, or near enough. My husband has taught me to be more efficient. Let me do this for you, as an apology."

The claim surprised me, but what did I know of healing. I did not think Nisrita would idly boast. I flexed my muscles and determined that I seemed to also have injured my arm. At least I kept my life. A broken arm and leg is a very light price to pay given I had just been ridden down. "Give me a moment to rest, then." I opened my eyes to see that Nisrita had taken a seat beside me. She sat with her hands in her lap, looking at her knees, for all the world the image of a girl scolded by her nurse. "Why did you tell the court we were lovers, Nisrita?"

"I am sorry, Timmon. I am not trying to manipulate your life. You've remade your career by yourself. I am sorry for what I said on the Day of Unions. I didn't mean it. I was scared. I didn't know what else to do."

She was agitated, I thought, though she was trying to keep a calm front. I thanked my luck for her composure. I could not take one of her outbursts at the moment. "Just tell me why."

She took a deep breath and, if anything, looked more intently at her hands. "Your servant, Nobu, does not know our ways. He implied things, in jest, I am certain, about your habits that could not be ignored. I ... I did not know how to stop the rumour without starting one of my own."

I closed my eyes and lay back on the hard healer's table, feeling my world teeter on the edge of a deep abyss. This young girl had decided to make herself my tether, regardless of the cost to her. It was unpermissable. I needed to take her away from here. I needed to put here someplace safe where she could rest and recover, out of danger herself and no longer a danger to me. She was married, I had no right to interfere between her and the duke. That did not dull the need. She should not be in the position of taking care of me. She was barely a woman, I nearly old enough to be her father.

"Are you mad at me, Timmon?" Her voice emmerged small and childlike from beyond my eyelids.

"No, Nisrita. Thank you. Feel free to heal me as you see fit." I did not have a choice but to let her help.

\vspace{.cm}
***

Dear Makala, why are you not with me now, my son. Why did they take you away from me. What will I do without you? Everything has gone wrong. This was not how it was supposed to play out. I don't know what I will do without you at night, my darling.

Are you sleeping now? Are you curled up tightly with your brother? Is the bright green blanket over you both? The night is cold. Is your father in the next room, alone? He won't hurt you, will he? I hope not. I do not think so. You are his blood. He values that. I am nothing to him. Less than that. I dared show the Tower that he was wrong. I don't know why I dared. It would have been easier if I had not. Then we would still be together, wouldn't we? Will you forgive me, my son? I did it because now we may be able to heal crippled children. We may even be able to help soldiers who have lost a limb, though it is too early to tell. I put you aside because I thought this was where my duty lay. I thought I was a healer first. But this hurts so much, my son. I am so sorry.

Why are you still reproaching me? Why are you still mad? Is it because I did more than prove your father wrong? I had to help Timmon, Makala. I had to. You do not know him now. You will, though. He will be the man to guide you to manhood. He must be. Then you will understand. The rumour is false Makala. Your mother has only known two men, the two she has married. I have not shamed you. The gods know that, even if the court does not. 

Do I love my work and Timmon more than I love you? Please, Makala. Don't say that. How can you think that? Forgive me, Makala. Please forgive me. I did not mean to desert you. 

"May I enter, duchess?"

\vpsace{.5cm}

***

She did not answer. She was talking to someone, Makala, it seemed. Her husband or her son, I could not tell. There was no one in the room. She stopped when she heard my voice, but did not answer.

Nisrita's maid had opened the door for her, without announcing me, thinking what ever she thought about my intensions. It was the middle of the night. My bones had been healed completely, but the bruises on my body still troubled me. Nobu did what he could for me, but I found lying down for long periods uncomfortable. The past three nights, I had developed the habit of rising and studying the newly made maps of Niev. Griswold no longer wished me to work with General Galderan in the upcoming campaign. He had his reasons, based on lies though they were. If I were to salvage my career, I would have to convince the General that I was the only person who knew the lands well enough. None of the ex Hundred Horsemen could be trusted to advise the General. Whatever else my sins, I had never betrayed Marsea. 

I knew the duchess had left the castle to live permanently in the Tower. The past three days have filled the court with uproar and scandal. I have not had the courage to approach the castle. I was too close to the center of it all. When I saw a lone light and a figure in a window, surrounded by the dark windows of peacefully sleeping healers, I knew who I saw. I put away my maps and I went to her. What would people say? That we were lovers. I had stayed away from her, wanting to protect her from those words. Now that it was upon us, we had nothing to lose. 

"May I enter, duchess" I asked more firmly. Whatever the rumours, I would not enter her room without her permission. It was a small room, fitting only a narrow bed and a small desk. Shelves lined the four walls containing books, clothes and other possessions. It was an ascetic's room, too small even to let her maid sleep inside. Nisrita stood at the narrow window, her back to me. A candelabra stood beside her, the light I had seen. She was doing something to her hands, I could not tell what. She shook her head at my request.

I tried again. "I heard of your acheivements with the cats. Congradulations."

That, at least, got an answer. "You should not be here, commander." 

I was sick to death of that cursed refrain. I had been in Cortan for seven months now. For seven months I watched Nisrita suffer the soldier's illness, helpless because of that one gods forsaken phrase. "Why, soldier, because some will think we are lovers?"

I saw her shoulders rise and fall in a deep sigh. Then she turned to me. "Come in, Timmon. What can I do for you."

Her eyes were red and swollen. She had been crying. Her normally lacquered nails were chipped and uneven. She appeared completely unadorned, as as healer, but unkempt, as a madwoman. She fidgited where she sat on her bed, pulling at her robes anxiously. She was a vision of anguish. "Tell me about Griswold," I said sitting at her desk in that dark tiny room.

She did. She started with the wedding night, and told the story of the last fourteen months. She told me about his neglect, his pressure for her to use mushrooms, her fears for her sons, Ezaro's illness, her husband's pride, and how she feared for her career. She told me about her isolation both at home and in the tower. She told me that everyone thought it part of her madness that she did not love her husband as a savior returned. She told me that it was not her choice to move to the Tower. Duke Ergino had only let her remain in the castle only to await his uncle's judgement on her. She had been forced from her house. The price of her protecting me had been her sons. The children needed to be with their father, they would not come to the Tower. She told me about her midnight vigils with her children, and of the solace they brought to her. 

"You need to leave." I told her when she finished. 

"I cannot leave my work. And there is a campaign this spring. Who will watch my sons?"

"I will." It was the least I could do. This hue and cry would die down in a few days. I could find a way into the young dukes' lives. "Duke Griswold leaves in a week for Deyalorn. As you have said, the children need a father." 

She broke down completely at that. She put her head in her hands, her shoulders shook uncontrollably. I sat, watching her sob from my seat on her chair, our knees nearly touching in that cramped private room. If I had been a man in the habit of having, or had any experience with, relationships with women, that night would have gone very differently. Lies have the danger of spawing truths. As it happened, I found myself sitting beside her on her bed while she cried her loneliness, fear and relief onto my chest. I had seen her carried of the field, bleeding on a stretcher, stunned by the reality of her first kill, trembling atop a wooden post, throwing poisoned weapons at me, terrified at the realization of a life changing mistake. I had never seen her so vulnerable as I did that night.

By the time I left her, we had reached a compromise. She would wear her talisman in her preferred manner, against her husband's wishes. She would not march in spring on the campaign. Neither would she leave Cortan for a quieter Tower. We would meet regularly when I was not away serving Cortan. We would find a way to combat the rumours later, together.

\vspace{.5cm}

"I think you should marry, Timmon." I looked up into the shadows above my head for the speaker. I saw two black braids hanging above me, like ropes of a belltower. 

In the month weeks since Griswold had left, Nisrita had taken to a routine of physical training in the dark hour before Cortan's army emerged for practise. She would not touch a weapon for all my encouragement, but she kept her body limber, agile and potentially deadly. I met her in the darkness before meeting the men in my charge. I told her about her son's doings from the previous day. She told me something of the health of her crippled animals. She had moved on from kittens to dogs to an adult sheep. She found that the dogs and cats could eventually use their recovered limbs, though their bodies did end up lopsided from favoring the weak limb for too long. She had a hideous dream of trying this new trick of hers on children. She sensed my unease with the subject and did not press the issue. I did not ask.

I would not say that she seemed happy every morning, but she assured me that she was better than she had been for a while. Her work proceeded well. Her theories and results were not popular, but her colleagues did not shun her as they had before. The regular exertion calmed her nerves, and news of her children calmed her fears. She could carry on, she claimed.

For my part, I found I enjoyed my time with Nisrita's children. Griswold was still the stronger, dominant boy. Makala, smaller, sly and clever. I found myself wondering if she had indeed called her husband back to this world by naming her son. They both looked like Makala, or the children he could have had. Playing with them, it was easy to slip into a fantasy world where the dukes were Makala issue. I had no urge to sire children. Raising Makala's sons would be a pleasure, barely different from raising my own.

The thought of marriage had entered my mind. It would give my various questionable habits a veil of propriety. It would even make my climb back through the ranks of Cortan's army easier. The problem was finding a suitable, tolerant wife. "I take it you have someone in mind?" I called into the semi-darkness. 

"In fact." The bell pulls disappeared, I heard a grunt, a rustle, then a thud beside me as Nisrita landed. Then I saw her crumple into a string unladylike words. 

"Nisrita," I started firmly, and cut myself off. I was as guilty of encouraging her bad habits now as Makala had been. She had shown herself to be capable of carrying herself appropriately in court. I swallowed my reprimand and asked after her health.

"Hand me my talisman, please." I did and contemplated her condition. Part of my purpose for attending her morning practises was to ensure that she did not push herself harder than I would allow any soldier to push himself. She only wanted the slightest excuse to take dangerous and reckless risks with her body. "Barely sprained." She announced. "I'll be fine soon."

"Good. You are done for today, soldier." She whined until she had determined my resolution, then turned her attention to her ankle. 

"Come to the Tower after dinner tonight," she continued after she finished. "She would like to see you."

"You wish me to marry a healer." I should have expected this. She had little occaison to socialize with other women. 

"You've met her before, Timmon. She is more refined than I am. I think she'll suit you better than I ever could." I found myself grinning at her wicked false modesty. In responce, she bounced up on her now healed leg and ran back to the Tower without giving the name of the woman I would court.

***

I would like to say that I was not jealous of Timmon and Sophia's happiness. I had thought myself beyond being jealous of Timmon. I was wrong. 

He met us on the evening I had arranged, dressed almost as if he were coming to court. He was nervous. I bit my lip to bury my smile. It would not do to scare him off now. Timmon still drank more than was good for him, though privately now so as not to hurt his recouperating image, and gambled more than I liked, though he would never admit it. He had failed to make his full payments to the treasury on more than one occaison. I paid his balance. I could not replace what he had lost in Makala. Neither could his Liri servant Nobu. I had thought that by burdening him with the trappings of a comfortable domestic life, I could give him an anchor to return to his old life. Clearly, I had been wrong. 

I had learned from Nobu's broken Mer that Timmon had all but given up on the triumverate. He had teetered on the edge of conversion in Lir, brining back with him holy books filled with stories that could not be spoken in Marsea. The sky god lay with the mortal warrior king Ayumu to grant him courage in this battle against his rival and brother, the god of the sea, and his favoured king. Two brothers and lovers, Haru and Haruto fighting an army of demons from the southern seas back to back and arm in arm. I would have been happy for him if it gave him peace, but I feared that it did not. Nobu assured me that he did not pray to his new gods. He did not have an idol of the triumverate in his home. He had been an agent of the Destroy. Timmon swore by him when he was upset, even if he neglected him at other times. He had stopped that habit. He had made himself into a man without a god to watch over him. No good could come of that. 

Someone needed to guide his steps. I could not. I thought Sophia could. I did not wish to scare him off the idea of marriage. I had told Sophia little that was not public knowledge. Timmon needed to marry for the sake of his reputation. He needed a woman's touch to stop destroying himself. He had certain habits that would be distasteful to most wives, but he would not marry a wife that would not tolerate them. In return, he would probably be willing to tolerate many habits in his wife. 

Sophia laughed at my attempted delicacy. She was a woman with many lovers. She had suffered greatly at the hands of Commander Lazarro. Something changed in her when she became a widow, and she gave herself as freely to men she likes as she had been constrained in her marriage. But her son was nearly six now. He did not seem to be gifted. Even with a name like Joris, the Tower would not support her son for long. She needed a name. Timmon had one. I am certain she thought at the time that I was asking her to be tolerant of his affections towards me. I did not tell her anything more. The rest of Timmon's tale was not mine to tell. I would leave that to them to work out. 

On the evening Timmon met us, I had invited Ezaro to join us. It proved a convenient decision. With my husband's absense, Ezaro was finally healthy enough to return to practise. He too had listened to Griswold's descriptions of funneling the gift. It seemed only those who had experiences the nightmare mushrooms could truly appreciated Griswold's lession. Ezaro had always been a more skilled practioner than I, even if not as strong. If anything, he was able to apply Griswold's instructions with more effect than I could, even if he could not teach me what he was doing differently. So far, no one could reproduce my results. Everyone saw what I did, and agreed that it was possible. I could not be the only one to heal amputees. That was nearly useless. If this new skill was to be of use to Marsea, I had to be able to teach it to others. The popular theory was that only one as powerful as I could perform this act. I could not let that be true. That would mean an end to this type of healing as soon as I left my prime. There had to be an element of skill involved, or a modification of my method. I spent my days talking to Master Adele about possibilities and techniques, I spent my evenings with Ezaro discussing how to apply Griswold's new wisdom to this problem.

I reintroduced Commander Timmon Romino and Healer Sophia Joris to each other, pointed out that Sophia had grown up near Deyalorn and left them in the garden. It was a clear if slightly chilly night. They would be perfectly comfortable if they chose to conduct their conversation walking the lit paths in the garden grouns. For myself, I turned to Ezaro with questions of my own.

"A Joris, Nisrita? You want me to marry a Joris?" Timmon exclaimed when I saw him the next morning. 

I was struck by the vehemence of his displeasure. I did not want him to marry anyone in particular. We both knew he needed to marry. I thought Sophia would make a good match. Did the Rominos and Joris's have some long standing feud I was not aware of? "You do not have to marry her, Timmon. It was only a suggestion."

"Even I know the name Joris. It is the greatest family of gifted healers in Marsea. They claim Lucretia as a matriarch. The third son of a barren line can never hope to marry into such an illustrious family." 

I could not take the criticism that day. Some days are easier than others. Some days, I can remember to consider my words, on others, I find myself still overwhelmed by grief. It must have been one of the latter days. "You once thought you could marry a duchess and sister to the Regent-Consort." I spat back. "What's the matter Timmon, was my mother of low enough birth to suit your needs?" I heard him start to apologize. I could not hear it. I left him then, without gathering news of my two precious children. 

Why couldn't Timmon just say what he was thinking? He did not want to marry. Makala's memory was still to fresh. His life with Nobu too new and pleasant. He did not like Sophia. I would have accepted any of these excuses better than his false humility. 

We fought. We argued over everything for the next two months. We argued over Sophia. We argued over whether I should accept night shifts in the infirmary. We argued over the presents he brought the dukes. We argued over everything but what was important. I said nothing when he failed for a third time to fully pay the treasury, or when Sophia mentioned the problem of his drinking. He did not criticise when I failed to save a potential recruit's life because I was too drained by my previous night's insomnia. Nor did he question my lies for the next few days about a winter cold keeping me from rising to meet my duties.

Some time, a few weeks before the campaign started, I still had not found a way to teach anyone else to regrow a limb, but my work with livestock had proceeded well enough that Master Adele gave me permission to work with humans. I had held a wild dream that I would be able to give this gift to Cortan's army this spring so that every man who returned from the front would return whole. It was not to be. I sent out word that I searched for a crippled child and an amputee, and waited. 

I heard no news by the time Cortan's army left. I watched Timmon leave the gates at the head of his column of men again. It took him nearly a year to regain his command. I was undeniably proud of him. I watched most of my friends leave for the front, Ezaro and Master Adele being the only notable exceptions. Ezaro was not physically fit enough for the journey. Master Adele claimed old age. I did not believe her. Master Alerio was just as old, and he marched. She had wanted to watch me work I think. That made me very happy. I admired Master Adele. 

Carlotta did not say goodbye to me before she left. I watched her stand with Cesaro Madriano in the yard. They would be married in a few years, when she was past her peak gift. The general had bought her service as a torturer with his son. I hoped she was happy with the bargain, though she did not seem it. Who was I to comment? I had barely spoken to Carlotta for the past several months. Where other colleagues had returned to the habit of conversing with me once I had proven the my ideas were not insane, Carlotta remained aloof. It was not that she loved my husband more than I, she claimed. She said that I had grown testy and impossible to talk to. I do not think I have changed. I still think she is jealous. I know she is furious at me for leading Sophia to Timmon. Carlotta truly believed the lie that I had been Timmon's lover. In fact, she believed that I still was his lover. Sophia deserved better than to be tied to a man already spoken for. What a laughable idea, given Sophia's long list of lovers. It was impossible to deserve better than Timmon. What could I say without being accused of being testy. I let it go. I had met Sophia through Carlotta. Sophia was four years older than Carlotta, seven older than me. She was Carlotta's friend before she was mine. It was natural the Carlotta want to protect her.

I watched Cortan's army leave the gate. It took nearly all morning. I stood at the top of the White Tower and watched the long fat dark worm inch itself along the yellow grass. There went my entire life. My husband had elected not to return to Cortan for the spring campaign. He was needed in Deyalorn by the crown. I did not believe him. It was quite possible that he was simply too much of a coward to stick his neck out in battle, now that he had a comfortable home in Cortan again. I was still not allowed to see my children. With Timmon gone, I no longer had a means of gathering news about them. I was not welcome at Cortan's castle. I was only tolerated at Cortan's Tower because of my power and my unfortunately unique skill at regrowing limbs.

I think I slept for the next three days. I certainly cannot recall how they passed. I do remember Master Adele storming into my room one morning and thrusting a letter in front of my sleep filled eyes. I had a volunteer. Baron Farone had found a young club footed child whose parents were willing to let me work on. 

I sat up. "I'll go get him now."

Master Adele snatched the letter from my hand. "No, Nisrita."

"Why not?" I cried. This was the opportunity I had been waiting for. 

"When was the last time you practised?" 

I thought. "Four days ago?"

"Our infirmary is not empty. If you bring this child here, you cannot disappear for three days while he is in need of healing." 

That was an unfair accusation. My skills were not needed in the infirmary. There were no life threatening cases. The few healers who remained could see to them easily enough without my help. "I won't neglect my duties if I have something intersting to do, Master Adele. There's been so little to keep me occupied here."

"The Tower has not closed, Nisrita. There is always something to do." She folded the letter and walked out of my room. 

I sprang out of bed, shrugged on a robe and chased after her. "Master Adele, give that back to me, please." I needed that child. 

"I will, when I know you can be trusted with it."

"I can do this, Master Adele." I whined. "This is what I've been waiting for. I'll do anything for this chance. Please."

"Prove it." She said, and walked off to her office. 

Proof. Curse and Take proof a thousand times over. Healers do not touch humans without proof that our techniques will work. It is an integral way of how we think about the world. As a result, we use it where it does not apply. How could I prove that I could be trusted with that letter? I grumbled and went to the infirmary. I spent a week healing sprained ankles, scrubbing floors, running laps and cleaning out stables before I was given that letter again.

***

By the time I returned with the child and his mother, I had a blacksmith with a hand deformed by fire and hammer awaiting my services. I could not tend to both at once. I asked the  man to wait. The child was young, he was just over a year old. Cutting off one of his deformed legs was quite possibly the hardest thing I had ever done. Hee was only a few months older than my twins. I had not seen them in over three months. I wanted to embrace this child and keep him safe from the cruelty I was about to inflict on him. Would I ever have the strength to give my sons up for something like this? If I could have closed my eyes and plugged my ears to cut the child I would have. Instead, I found the sharpest sword in the armory. I watched the mother's eyes grow wide as I severed the leg quickly with one neat blow.

Two days later, I saw a growth at the end of the infant's stump. I was extatic, as were Master Adele and Ezaro. One or both of them watched me work every day, recording my progress. I had done something no one had done before. I had worked on lower animals, but this was a living human child. There were so many in the tower that said it could not be done. With the Tower empty of its inhabitants, it was essential that notes and pictures be painstakingly recorded for when the healers returned. 

The new leg started as a small version of a normal leg with a flexible ankle that looked like it could support weight. I watched with spellbound delight as it become an fully human leg, wondering at the slowly evolving form. This was what had happened to my two sons inside me. I had taken apart so many fetuses of animals in these past several months, measured and drawn and compared them to each other, all to try to guess what I would see before me today. The physiology of the changes before me did not surprise me, but nothing could have prepard me for the awe inspiring sight of seeing the Maker's work, recreated outside a woman's body. It was as if I had been given a gift by the gods themselves.

Over a month after they arrived, I sent them home. It did not work. The leg grew back, but it grew back as I had found it, only emasciated. It took me a while to realize the enormity of the injustice I had committed. The new leg grew to the size of an new born's then the ankle turned again. There was nothing I could do. I had hurt the child for nothing. I had kept them at my expense at the Tower for weeks after I stopped healing the boy, apologizing to the mother, and teaching her how to help his son gain back the muscle in his leg. What else could I do? The child was only a few months older than my boys.

The tower being empty, the mother and child slept together in a private room where healers sleep. I found my feet being drawn to their door in the middle of the night. I would stand outside for a while, content to be in the proximity of a sleeping child. I could not let myself stay there long. Master Adele would never forgive me if I failed the child by not sleeping when I should. I could force myself to turn from their door once I had reached it, but I could not keep my self from walking to their door. It had been so long since I had held my sons. The mother never found out. I doubt she even knew who I was. I was a nameless powerful healer who had given herself to service in the Tower. For all the right I had to my sons, I may as well have been. 

I saw mother and child leave through the gate.  I could not bear to let them leave Cortan. I did not want to lose the company of this child. I waved goodbye and went to Master Adele's office. It would seem this had all been a foolish venture on my part, unreplicable by anyone else, good only for healing livestock. It was wasted effort, wasted resources, I would have served the Tower better had I stayed and tended to the infirmary and scrubbed the floors. 

Master Adele, of course, would hear none of it. She led me to a private room, where the blacksmith awaited my convenience. I sighed. I could not claim I was too tired to work today, though I felt like I could sleep for a hundred years. I had not healed anyone yet.

I cut off the deformed hand, and tried again. Master Adele stayed in the room. I could not shirk my efforts. The gift is fickle. It varies in power from gifted to gifted, according to their nature and their age. Similarly, it varies in effect, though less so from patient to patient. It also varies by age. The youngest respond to the fire the best. At the same time, a child is fragile until he reaches about the age of ten. We tend to them carefully because they are young. They have their entire life before them. The tower does not generally accept patients over sixty. Their bodies are too fragile, and it takes too much power to coax the body into repairing itself. We tire ourselves, and risk their lives for little gain. Somewhere near forty, a body starts resisting the Preserver's fire. It is as if it knows that its time on this earth is drawing to a close; the body passing from the Preserver to the Destroyer. The man on my table was well past thirty. He had lead a hard life. His had resisted my efforts.

It took nearly three weeks of exhausting, gruelling work to return his hand to the appropriate form. I often went to bed completely drained, unable to attend my nightly discussions with Master Adele and Ezaro about the future of this work. Two weeks into my endeavour, stood in the Tower, surrounded by the sound of celebration in the barracks. I could hear the joyful bells sounding from the castle, a mile away. I did not need a reminder of the day. I was a mother in fact, if not in name. One does not forget these things. I closed my ears to the sounds around me and turned my attentions to my blacksmith. I added his heartfelt cries of pain to the laughter and song from the grounds below my window. It was the perfect medley. I cried when my children were born while the court celebrated. I cried when they turned one without their mother. It seemed only right that someone should give voice to my pain. 

I picked at my meal in the Tower's hall that night. Had Master Adele not been watching, I would have slipped away without eating. She would accuse me of not being fit to heal my patients. "I congradulate you on your achievements, Nisrita." Master Adele said, as the meal drew to a close. 

"I have not healed him yet, Master Adele. The hand may grow corrupted again in the end, as the leg did."

"I do not think so. I've been looking over the records for the leg, and the mother's story. The child's leg refused to straighten shortly after birth. The leg you regrew bent when it reached the size of a newborn. There may be something to that. The bone may remember. Your current patient did not deform himself until well into adulthood. His hand may remember what it was like to be whole."

It was an interesting theory. It would require a lot of work to prove, but the logic was sound. It offered a faint glimmer of hope on the miserable day. "Thank you, Master Adele. Good night." I was tired. I was lonely. I did not wish to discuss physiology.

"Come to my office, Nisrita."

"Forgive me. I am tired. Perhaps another night."

"Come to my office, child." Master Adele had served for years as Headmistress to the girl's school. Years of training had drilled into me a reflex to follow that tone of voice meekly where it led. 

I could not believe what I saw when I entered her office. The two infant dukes, my two precious sons lay sleeping in their nurses' arms. I froze in the doorway, unable to approach this miracle before me. They had grown so much in these last five months. Their faces looked so handsome, their hair had grown so thick an long. I wrapped a lock of Makala's hair around my finger and stroked Woldino's cheek. What were they doing? I wondered. Were they walking yet? Could they stand? Makala could not crawl when I had seen them last, though Timmon had told me of the strange seal like motion he had adopted as his preferred means of locomotion just before he marched to Niev. I would never see him move like that.

"Give them each a kiss, your grace. We must leave." Woldino's nurse spoke. She was so cold and heartless. She had children of her own. Did she not understand. I had not seen them for so long, how could I satisfy myself with just one kiss?

"Nisrita, they cannot stay for long," Master Adele urged. 

I picked up my first born's hand and laid it on my cheek, as he used to do. I was startled by the memory of the soft caress. Then I put his tiny fingers in my mouth and nibbled them as I used to. Had he been awake, it might have made him giggle. I had last made him laugh nearly half his lifetime ago. He may not even remember. I turned to Makala, my sweet, gentle boy, and the balm of my sleepless nights. I kissed his forehead as I did every night I laid him back in his bed after he had given me solace. He murmurred in his sleep. Did he know me still?

I had barely straightened from my kiss when the nurses carried the children out of the Tower, and back to their life. The were only a mile away, and I could not reach them. What had I done to deserve this? Master Adele handed me a large goblet of strong wine, and asked me to sit. "I did what I could, Nisrita. They are beautiful boys."

"Thank you." I said instinctively. I did not know if they were still mine to claim credit for. 

"There was a note this morning" Master Adele continued, "from the captain in charge of the grounds outside, asking that you stop your work with the blacksmith. It stated that his anguish dampened the celebratory mood of the day."

"I- I am so-sorry." I stammered. "I got no such note. I would have .."

"I burned it." I stared at my old Headmistress. "They do not understand us, Nisrita. They would just as soon call us torturers as healers. The tower does what it can for its own kind. I've seen you grow up here since you were only a little older than your sons are. You are as much a daughter to me as any I will ever have. I do what I can for my kind. I will do more for you if I can."

I thanked her, finished my drink and left. She meant well, I knew, but the family I had dreamt of as a child had not been in this Tower. I had lived and dreamed beyond these walls. I had made friends with those who had no idea that the gift existed. They were good, kind people. I had seen my husband's cruelty, once one of the greatest among us. If indeed, I had to retire to a life in service to the Tower, I would do so because I wanted to serve the Preserver and Cortan, not because I loved only my kind.

Time proved that I had made the blacksmith's hand whole, only two days before the Tower filled with residents again. I was asleep when they entered the gates. Not because of the soldier's malady. The exertions of the past several months had reduced my body to a fevered state.

I heard the familiar clamour and sounds of a living, celebrating army from the yard outside through a haze of aches and shivers. The Tower could do so much, but the Preserver had not seen fit to give us power over even such a common illness. At one point, I felt two familiar hands running their fingers through my hair. I sighed and let Carlotta continue her caresses. We had not spoken much in the past six months. I did not want to chase her off. 

"You are awake then, Nisrita?" I nodded. "Will you eat?" I shook my head. I did not think my throat would let me swallow food. "Will the greatest healer since Griswold perish by starvation from a simple summer flu?" Is that what they were saying about me? I would not starve from one day of fasting. I think it had only been one day since I took to bed. "Eat." Carlotta commanded when I did not respond to her coaxing. I opened my eyes, sat up in bed, and took the bowl of hot liquid she proferred.

It was an old ritual, from my first days back in Cortan after my marriage. After Griswold had decided that I needed to be in Cortan's infirmary, it was Carlotta who cared for me. She made sure I ate on days that I did not wish to. She forced me to get out of bed and walk the grounds every day. She even managed to finagle permission from time to time to take me to an interesting lecture. She had and inexhaustible well of patience for me those days. I do not know what happened to change that. We started fighting as soon as I was released from my confinement. She said that I had changed since my marriage, that I had become hard, argumentative and unpleasant. She did not like the line of questioning I pursued, in spite my assurances that I did not prescribe to Horatia Joris's more heretical thoughts. We argued, eventually we stopped speaking to each other. The Tower is large. It is not so hard to avoid one person.

Now she was next to me again, feeding me and nursing me. I wondered what had happened. "You are not angry with me anymore?" I whispered, when the hot liquid had soothed my throat somewhat. 

"No. Are you?" I shook my head. I had never been angry with her, confused and hurt, but not really angry. "Can we start over?" I would like nothing better I wanted to say. I smiled, nodded and squeezed her hand.

"Good." She went to the window, pulled back the curtain and looked out. "Come here, Nisrita. Look out the window."

"Its cold," I protested. I had kept the curtains closed because even the warm late spring breeze had caused me to shiver.

"This is worth it." She found a blanket to wrap around me, led me to the window and pointed. I gasped when I saw the figures in the garden. She was right. It was worth it. I stood at the window and gazed that the miraculous sight I had not hoped I would ever see again.

\vspace{.5cm}
***

Makala gave Griswold a mighty blow to his head with the wooden sword in his hand. The larger boy fell heavily onto his rump and wailed. Makala looked at his vanquished brother, sat down on the grass and put his hilt in his mouth. Sophia laughed at my attempts to train the young dukes. I helped the crying boy up, dried his tears and said "Now that is no way to react when beaten, your grace. You have a sword as well. Hit him back." He boy did not like my advice. He turned and toddled to the gentler arms of his nurse.

I watched Sophia watch her six year old son Eugenio play with the still happy Makala. She was a very different woman than Nisrita. She wore the air and grace of Deyalorn's court about her. She had refined tastes, was well educated in the arts, and enjoyed her pleasures. I could talk to her as I could talk to any other woman, she held few surprises. She was literate, but did not see the need to flaunt her knowledge. She was reaching the end of her natural beauty but willing to work to make up for what time had worn away. She did not speak of her art unless asked directly. When she did touch on the subject, she did so delicately and gently, without Nisrita's academic bluntness and callousness. Everything about her was refined and feminine. She was what a man should want.

Sophia would be an expensive, but elegant wife.  She would keep my household appointed far better than I could hope to. She was the youngest daughter of a minor branch of an old family of healers. The Joris tree has produced some of the best healers in the land over the centuries. Sophia assured me that in spite of my barren line, she could provide me with gifted children. I did not question her. We agreed that she would keep to her affairs out of my house and the public eye. I would do the same by keeping my affairs strictly within my house. She would not question my loyalty to the Triumverate. I would allow her to practice her art. She would keep a quarter of her incomes for herself, the rest for the househould. I would adopt Eugenio and educate him from the small inheritance left him by Commander Lazarro. He would be fed and clothed at our expense. I would not touch the boy's inheritance. What money did not go to the boy's education and training would become his when he reached majority. 

Eugenio was not gifted. The Tower would not house him anymore. Sophia could not possibly join me for the five months until the Day of Unions. During the campaign I had proven my worth to General Galderan, who promised to speak kindly of me to the Duke. When I received news, shortly after returning to Cortan, that my command had been reinstated permanently, I agreed to let Eugenio into my house. Her family had influence in the country's Towers, but little land. Mine had some land, but little sway among the healers. I had a promising career again. It was not a very uneven match for a widow with a child.

Sophia looked up at the Tower and pointed Makala's mother out to him. I had come with the boys to these gardens for the last week to play in the late afternoons under Nisrita's window. She had taken ill at the same time our army returned home. I have been assured by several Masters that there was nothing abnormal about this illness. She had worked hard performing her miracle in our absense. He body had succumb to a spring flu. The Duke had still not given her permission to return home, or for her to visit her sons. He did not, however deny Master Adele's request to let the children play in the Tower's garden, provided the duchess did not approach. I would not visit Nisrita in her sick room. I had promised myself to another. I would not enter her chamber. I took up my part in this complicated charade by bringing the dukes to the garden with my afianced and playing below her window.

Makala started fussing, and the two nurses rose to take them home. Sophia rose as well. "I'll walk Eugenio home. You should stay here for a while longer."

"There is no need for that. Let me walk with you." The days were getting longer. It would not be dark for a while, but Sophia was too refined a lady to be allowed to travel unaccompanied beyond the Tower's walls.

"The Duchess was well enough to come down to breakfast today. She may wish to speak to you. I know the way." I nodded and walked her out into the barracks where I found an escort for her. Conversations were easier with Sophia than Nisrita. She was older, age and experience had mellowed and wisend her. It was not that she was not a woman of passions. There was ample evidence of that. She knew better when and how to act on them. 

Sophia was very different from the young woman, the miracle worker, I met by the garden path in the Tower's orange grove. Nisrita sat on a simple stone bench with her back to me. Something in me barred me from approaching. In the two months that I had not seen her, she had done the impossible. Everywhere I turned, people spoke her name in hushed incredulous tones. My friendship with her, which had once threatened to unravel my career, suddenly made me a popular man. It was hard to believe they spoke of this small unruly woman. 

It was not just respect or awe that held me back, however. There was a faint twinge of guilt as well.

"Is something the matter, Timmon?" she said, without looking at me. "I promise not to cough on you." 

I grinned. The most famous healer in Marsea still spoke like an ungroomed child. "Nothing is the matter. I had just paused to admire the Maker reborn."

Nisrita laughed and then coughed. "That's blasphemous, Timmon. No one says that."

That was simply not true. "If you spent time among the officers, you would hear many men say that."

"They should not. I cannot make a limb any different from what the Maker intended. I can only regrow what was lost." 

"You have given hope to hundreds of crippled veterans, Nisrita. You do not know what you have done."

Nisrita shook her head. "I have done nothing yet, if I cannot teach it. Their hope will die with me, if not sooner, otherwise."

I sat down next to her. She was never satisfied with her accomplishments, whether in her physical training, or in her healing. "You are hard on yourself, soldier. Take each victory as it comes. General Galderan wants you decorated."

"Me?" she sounded genuinedly shocked. "I did not march in this campaign."

"No, Nisrita, you did something much more useful with your time. Should I tell the general that you well enough to appear before the Duke?"

She balked. "Father will decorate me?"

"That is traditionally how it is done."

"It is absurd, Timmon." She laughed bitterly. "It is completely absurd." I waited for her to finish her laughter and reveal to me the impropriety of the General's intensions. Miracle worker or not, there was still too much sorrow in this young woman's life, and not many who could help her. "You don't see why, do you?" I had to admit that I did not. "I am worthy enough to be decorated by the Duke in front of his court, but not worthy enough to see my children except from afar, hiding behind curtains and shadows? This is a perverted farce."

I took in a deep breath and let it out slowly. Put that way, the act of decorating her was more than absurd. It was cruel. "I will ask the general to speak to the duke."

Nisrita shook her head sadly. "He will not listen. Do you think the Tower has not tried?"

"Then I will simply have to find a way to involve the clergy in this matter as well. How much of a threat does a sixteen year old girl pose to his grace, that he cannot bow to the pressures of the three pillars of his court?"

Nisrita laughed until she broke down coughing. It took her a long time to gain the breath to thank me. I grinned. Sophia would be an quieter wife, less troublesome by far. But she would never inspire me to move heaven and earth for her as this curious young woman was in the habit of doing.

***

"I've done it!" Ezaro thundered into the room where I was resting. I had spent the last two months healing a steady stream of cripples and amputees. Master Adele helped me organize my day and my strength so I could heal three patients at a time without draining myself as I had done after the blacksmith. Even at this rate, it would take me half a lifetime to tend to all the men who wished healing. My schedule left my evenings free to answers questions from healers from distant Towers about this new act. Timmon had worked a miracle, and allowed me to see my sons, just as the demands of the Tower denied me the time to see them. I decided to continue to live in the Tower, to better be able to tend to my duties. My children visited me every morning between my first and second patient. Every five or ten days, I could steal time away to socialize with Carlotta, or meet Timmon and Sophia during one of his now regular visits to the tower. I had neither time nor energy for anything else. 

"What have you done, Ezaro?"

He thrust a limp rabbit onto the table. "I've regrown its front leg. Look." 

I rose to examine the creature. Ezaro was literally quivering in his excitement. "The rabbit is dead, Ezaro."

"Yes, but that's not the point. Look" He showed me an amputated limb with a characteristic lump on the end of it. "The bone started to regrow. Give me one of your goats tomorrow. They are sturdier than rabbits."

"You will not touch my goats, Ezaro. This death is exactly the point. What is the point of regrowing an arm if you kill the patient in the process." This is why healers have the reputation of being torturers. Ezaro's arrogance was infuriating. 

"A goat is hardly a patient, Nisrita. I admit, I still need practise. But this is better than anything anyone else had done."

I looked at the rabbit's stump, while Ezaro jittered beside me. It was true, the newly forming paw did look as it should after two days of work. Something about Ezaro's nervous composure bothered me. "Let me see your eyes, Ezaro."

He turned away from me. "Your are a hard woman to please, Nisrita. You should be grateful that I can do this, with the long line of patients waiting for limbs."

The bastard. The cruel, short sighted, idiotic bastard. "And how many of my patients do you think you will serve while consuming mushrooms. You do this correctly or not at all. This is not a task for the old magic."

"Your patients, Nisrita? They come to the Tower for healing, not you. I will serve them any way I can. Master Alerio will give me a goat when I show him what I've done." He stormed past me with the dead rabbit. I sat down and thought. This was my project, and my achievement. If I could teach this to others, it would be to serve Marsea, not for my personal glory. I could not allow this to be an act only done while using those horrible mushrooms. It was unreliable, and dangerous to both healer and patient. 

The Tower's hall was buzzing with Ezaro's acheivement by the time I seen my third patient and entered for dinner. I took my place between Master Adele and Carlotta. 

"What do you think of Ezaro's rabbit, Nisrita?" Master Adele asked.

"I do not have hopes for it, Master. I do not think he can reproduce it."

"Why not?" Master Adele asked. 

I was still not used to being in a position where my teachers turned to me for knowledge. I licked my lips and proceeded carefully. I did not want to say anything without proof. "Was Ezaro ill yesterday?" 

"I do not know, why?" 

I sighed. I would have to speculate. "I think he did this using old magic. I do not think he can repeat it while lucid."

"You are being jealous again, Nisrita," Carlotta chided. "Why can't you have his moment of acheivement?"

"I am not being jealous. I may be wrong. Maybe he'll do this with a goat tomorrow. If he does, I'll be the first to congradulate him. All I am saying is watch him. Make sure he is lucid when he replicates this act."

Carlotta made a face to show me exactly how much she believed my claims about jealousy. "Girls." Master Adele commanded. We both stopped arguing. "It may not matter what state he healed in, Nisrita." 

"What do you mean?" How could it not matter if he had been drugged or not.

Master Adele spoke in a low voice. "Your husband has been working with Ezaro on developing ways to control the wild dreams. They had come a long way towards actually creating a table of doses and effects of the mushrooms. If Ezaro can do this with a safe dose, then I think it is a valid means of proceeding. We have far more men wanting limbs than you have time."

"Impossible." I said too loudly. "It is impossible to control those dreams. To try is just foolishness. He'll kill healers and himself in the process."

"Listen to yourself, Nisrita." Carlotta said quietly. "Since when have you stood against learning about the gift? You cannot simply block a new means of healing  because your husband has a hand in it. If we can control the old magic, then think of what we will become. There will be no limit to what the gift can acheive. Your domestic troubles have blinded you."

I opened my mouth to tell her that she did not understand. Then I closed it again. She did not understand, but I would not lose her company again over this argument. Griswold did not want to learn about the gift or expand the boundaries of healing. He wanted to regain his healing powers and retain his eternal youth. He was far too selfish to dream beyond that. Every report of my husband that I had heard from healers from Deyalorn that came to visit, told me that he had not returned to Cortan because he was busy answering the questions for and being studied by Deyalorn's Masters. Cortan was safe. All Griswold wanted now was to bask the adulations of Deyalorn's elite healers.

It turned out that I was wrong. Ezaro amputated a goat the next morning. By the evening, the charachteristic stump had appeared. Everyone who watched him claimed that he was lucid. I kept my word and congradulated him on his acheivement. In a week's time, the Tower had added another limping goat to its growing menagerie of misbalanced and maimed creatures. 

Headmaster Corino gave him permission to start work on patients. He would only get one at a time. The rest of his day, Ezaro would spend teaching others to do as he had done. I attended his lectures to learn his techniques. I learned that I had missed several subtle details in my eagerness to heal. I did not need to pay attention to them, my power was great enough that I could ignore details others could not. Ezaro had always been a more skilled healer than I. I admired the finesse he had introduced to this art of regeneration. It was clearly inspired by the insight Griswold had given him.

I trained with Ezaro's students when I could. There was no reason to work inefficiently, when I could learn to conserve my skills. The first week was grueling agony. My three patients drained me under normal circumstances. The additional practise on rabbits again left me barely able to stumble into bed.

Master Adele reproached me. I was not husbanding my resources correctly. I would harm my patients if I did not take care. I begged for a week of time to improve myself. If I could not become more efficient in a week, I would stop attending trainings until I could remove a patient from my roster. The good Master relented.

By the end of the borrowed week, three patients were merely tiring. By the end of three more weeks with Ezaro, I took on a fourth, and found more time for my children. Ezaro's miracle was nearly as great as mine. With his insight, any strong healer could now regrow a limb. It was not an act only for the extreme outliers like myself. By then end of the sixth week after he had healed his goat Ezaro taught the dozen strongest healers in Cortan's tower on a daily basis, as well as several visitors from other Towers. Healer Melina, a young woman from Carlotta's class, and Healer Delphina, an older woman from Deyalorn had taken on a regular roster of two patients each. Every week visitors from distant Towers arrived to learn, to study, to take the skills we had invented in Cortan back to their homes. We had almost become a pilgrimage sight. It was an incredibly exciting time to be in the Tower. None of the old Master recalled a period like it. 

I cannot deny that part of the allure for me was being in the center of the activity. People watched me work. They asked questions of each other and speculated over my shoulder. Master Adele became very possessive over my rest periods. My conversation was in such demand. I met healers from different towers every day. Many were young, in or just past their prime, but I also conversed with the Head of Allepo's Tower, and the head of Selvand's. Never in all my time in Cortan's court had I been the subject of so much attention and praise. It was a giddy exuberant time for everyone. I was a young girl of sixteen. It would have been impossible for my head not to have been turned by this flattery.

At the end of the sixth week, Ezaro did not appear for his evening class. We waited for fifteen minutes before becoming restless and curious. Melina and I went to see if he was still with his patient. Another few left to search his room and the animal pens. Melina entered the room and was at the patient's side before I could take in what had happened. Ezaro lay convulsing on the stone floor of the small private room while his patient lay pale and motionless on the table, with a ten year old's leg exposed on a man's body. Blood dripped from somewhere below the large grey form. There was more of it on the floor than there had any right to be if we hoped for the man to still be alive.

I threw my arms around Ezaro and spoke firmly into his ear. "You are safe, Ezaro. Whereever you are, you are safe. I have you. You will not die now." I did not know what his dream was, but I was certain that he was in one. If he died in his nightmare, he would die in life. Cortan could not afford to lose him. 

After five minutes of repeating this mantra, Ezaro's siezures did not stop. Healer Melina stepped away from the man on the table. "What can I do? The patient cannot be helped." 

"Make him vomit." I said. I did not know why my words were not helping. I did not know what else to do. The actual dangerous part of the dream only lasts for a few hours. Clearing his stomach was the only hope I saw of helping him now. Melina did as I asked, helping me pry open his jaws and gag him with a thin dowel. I was amazed at the contents of his stomach. There was very little food in his stomach, and many more flecks of black fungus than I liked the sight of. He had not brewed it like Griswold had done for me, but eaten it whole. Some of the pieces were more digested than others. It looked like he had been taking small doses over the course of several hours. I guessed that he might have consumed and entire head over the course of the day.

Melina left me to get help. I sat in the small room, covered with blood and vomit for what seemed like ages until two more healers arrived to move the now still, but breathing Ezaro to the infirmary. When they took him from my arms, I sat on the floor in a stunned angry silence. Ezaro had no right to be playing with his or his patient's life like this. This was too dangerous a game. He had built so much without the help of the old magic in these past few months. How could he be so power hungry to throw away for a toss of those horrible dice. 

Melina brought me a bucket to wash my hands and feet, and a clean robe. She had already made herself presentable. I changed and looked at the man on the table. I recognized him, though I could not remember his name. He had lost his leg on the march to Turina. I remembered him in particular because he could not stop himself from grabbing every healer's hand he met and begging them to save his crushed limb. Timmon had said I had given hope to so many crippled veterans. I had done nothing of the sort. By letting Ezaro work through his madness, I had killed a good man.

Master Adele and Headmaster Corino called us to the head's office to ask for a report of what had happened. "You will speak of this to no one, of course," he said when we had finished. He impressed upon us the shame and scandal that would befall Cortan if news of this incident made it out amongst the visiting healers. "Ezaro has had a an epileptic episode. He will be sent home until he has recovered."

"Of course," we agreed. Healer Melina was excused. I was asked to stay behind for a few further questions.

"What do you know about your husband's work with Ezaro?" Head Corino asked.

"Only whispers, Master. My husband did not discuss the matter with me." 

"And what is your experience with the old magic?"

My stomach tightened. What had I done? I had just admitted to the head master that I had been able to guess at the quantity of the mushroom he had consumed and how recently by looking at the contents of his stomach. Furthermore, I had admitted to trying to bring him out of his dream. It had not occurred to me to keep secret my knowledge of the subject under the circumstances. I licked my lips and lied. "None, Master Corino."

Master Adele's eyebrows rose to the peaks that had once terrified me as a child. The Head Master spoke. "I find that very hard to believe, Nisrita."

"Only once, sir," I mumbled. "As a brew, months before I married him."

"Thank you, Nisrita. I think, under the circustance, it might be best if you cease your services to the Tower for a few weeks until this blows over. I am sorry, Master Adele to slow your program to a crawl like this, but under the circumstances, I see no other choice."

"What!? But I have done nothing wrong."

"You have admitted to playing with the old magic. I would be within my rights to expell you from this Tower."

It wasn't fair. I had been forced to. Everyone else in this tower wanted the old magic more than I did. It wasn't fair. I turned to Master Adele for support. "My patients, master. I cannot just leave my patients."

"Your patients are stable," she reminded me. "We have enough people who can regrow limbs to take care of them in your absense." 

"This is wrong, Master Adele. If you had listened to me when I told you what I thought Ezaro was doing originally, this would not have happened. A good man would still be alive."

"Nisrita!" Habit shut my mouth at her sharp tone. She continued more calmly. "What happened cannot be undone. It is a pity that a life was lost in the process. We will work to fix our errors from this point forward."

"It was more than a pity, Master Adele. It is a travesty. That man begged for his leg when he lost it in Niev. Now he is dead because of our carelessness. This is why the ungifted fear and hate our kind. This is why we are called torturers."

Head Master Corino and Master Adele exchanged meaningful looks. Master Adele rose and asked me to follow her outside. "You are upset, Nisrita," she said when I was alone with her.

"I am, Master Adele."

"You have every right to be. What you saw must have been hard." I nodded, though it was more than just that. "I would like to give you some advice, as an old experienced healer to a young promising one. Keep some distance from your patients. You rise and fall too much with their recovery or deterioration. You will lose people in your care. You will not be able to help it. In the end, you will help more than you hurt. Distance from your patients will give you perspective in your work."

"Thank you, Master Adele." That may have been good advice, but I did not see how it applied to me now. "Please, Master Adele, let me return to work tomorrow. Don't send me away."

"Two weeks, Nisrita. Maybe three. Then you will be back. You are prone to nervous episodes. You have been pushing yourself hard recently. I would hate to see you relapse. Take the time to spend with your children. They are sixteen months old? They must be talking by now."

I shut my mouth. I had lost. Master Adele walked me to my room and left me there. I slammed the door and sat in my room and sulked. Nervous episode indeed. The Preserver must have cut out her heart and replaced it with stone when she gave herself over to service. One man was dead. Ezaro guilty and possibly mad, and I was the one reprimanded for pushing myself too hard. This was madness.

I paced the length of my tiny room until I could take the captivity no longer. It was nearly dusk. No one would be practising in the yards. I had neglected my training since I had devoted my life to healing limbs. It would do me good to get out and do something physical. I had just changed into suitable clothing when I heard a knock at the door. 

I turned around to see Carlotta's head peering around the door. "Master Adele said you might be doing something embarrassing."

Is that what she said. "Nothing embarrassing, Carlotta. I had just realized how long it had been since I had trained."

"Embarrassing enough Nisrita. Change into something nice, there is someone I want you to meet."

Something nice? "What are you planning, Carlotta?"

"Nothing much. Just meet me in the hall wearing something nicer than that" she jestured at my current outfit, "and more feminine than your healers robes." Her head disappeared.

Nothing my horse's ass. On the other hand, I had nothing else to do, and she had arroused my curiosity. I made myself presentable for polite company.

I found Carlotta, her afianced, Sergent Cesaro Madriano, a tall handsome man, the son of General Madriano, who had his eye on a new barony in the freshly conquored territories of Niev, two girls from two classes below me accompanied by the head girl of the tower, and another fair young man, about twenty, slightly built with beautiful finely carved features I had seen before but did not know. 

"Ah, the Maker's daughter has come to grace us." said the stranger, bowing low with a flourish. "It is a pleasure to make your accquaintance, your grace. I am Healer Ricardo Almena, son of Baron Almena from the duchy of Selvand." 

"The pleasure is mine, healer." I said, giving him my hand. Carlotta looked at me with a wicked glint in her eye. 

"There has been a market in town for the last five days. I would have invited out to join us earlier, but you have been busy. Shall we go?"

Healer Ricardo told me about his fathers lands on the north sea and asked me about my adventures as we walked to the fair. I wondered how long he had asked Carlotta to arrange a meeting with me, and why she had agreed. By the time we reached the market outside the castle gates, I decided I did not care what Carlotta's motivations were. I was glad that she had introduced us. Many men, healers, had showered me with their attentions regarding my art these past few weeks. Several had flattered me as a woman. None had done both as elegantly as this man by my side. He was a pretty man, his body thin and soft as healers' tend to be, not athletically built as military men are. He took my hand to help me step over  a ditch in the road. When he retained it for longer than necessary, I did not object.

The market was a bright affair, well lit by torches in the now gathering dark. Vendors sold spices, sweets and fragrant woods from Lir. I selected a few candies and toys from my boys, accepted a bright blue silk scarf from Healer Ricardo, and settled down to a selection of spicy roasted meats and sweet fruit drinks before a company of musicians. 

"And to what do we owe this reemergence of the miracle worker?" I heard Timmon say over the din of the crowd. A tray appeared with nine cups of alcoholic fruit drinks. I moved over to make room for Sophia. Timmon sat across from her, next to Sergent Madriano. If he thought anything of the presence of the fair Ricardo by my side, he did not indicate it.

"The usual, commander." I laughed. "Tragedy and ill chosen words." I would not spoil the boisterous mood tonight.

Carlotta scowled at me. "And Healer Carlotta's cunning, commander." Sergent Madriano added. "I fear she may be a wily wife." Carlotta took her afianced's words more gracefully than I would have.

"For which, I am ever grateful, healer." Timmon bowed to Carlotta. What had been an icy relationship before the last campaign seemed to have softened back to its original cordiality. "It is good to see you out of doors, dutchess. I was beginning to fear those white stones would bleach your skin." He grinned. I bit my lip to keep from doing the same. He was not drunk it would seem, just happy. He was happier than I had seen him in I did not care to recall how long. Was it Sophia? Was it the memory of Lir? Did it matter? Why was I jealous?

"Where is your boy, Sophia?" I asked.

"Comander Romino's man is watching him tonight," she responded.

"You have man of many talents, commander." Sergent Madriano said. I do not know how Timmon did not react to that comment. I blushed. I caught Ricardo looking at me for blushing, and blushed more deeply. 

"Is this Liri market your doing, commander?" I asked to rescue myself from my embarrasment.

"I cannot take credit for it, duchess. The crown has agreed to set up an embassy, which has helped. The details remain to be settled, but they will be revealed in time. Our conquest of the lands between Cortan and the Velta Valley has certainly made a trade route safer. The Liri traders have done the rest." 

"A toast, then, to the brave Liri traders that have made this week possible," Sergeant Madriano called. We drank. We talked. We listened to the musicians.Timmon explained the songs. We returned home far later that I was accustomed. 

I was glad I had gone. The tragedy that had instigated this outing was a mere distant memory, easily forgotten under the pleasant pressure of Ricardo's company. He asked me if I would walk around the garden with him once. The rest of our companions went their own ways. Ricardo and I went ours. He kissed me under the orange trees. I did not resist. It was hard, at that moment, and for the many moments between then and when Griswold finally returned from Deyalorn, to keep myself from risking a pregnancy. I understood why so many gifted women do not practise during all of their prime, chosing to marry and bear children in stead. Celibacy, when surrounded by the attentions of flattering men one works beside as colleagues every day, is hard and lonely. It would have been so easy to slip. 
 
"There is a small discussion group among visitors that meets every evening on the northern terrace. We would be more than delighted if you could join us, duchess." Ricardo said when I had finally pulled away. 

"Of course," I said disentangling myself with great difficulty from his gaze. I had nothing better to do for the next few weeks. 

\vspace{.5cm}


\end{document}
