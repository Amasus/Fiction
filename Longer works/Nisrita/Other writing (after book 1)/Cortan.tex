\documentclass{article}
\usepackage{fullpage, verbatim}

\begin{document}

The Great Hall sat still and empty. Court would not be held today in preparation of the celebrations tomorrow. Flecks of dust danced in the rays of late morning sun that lit the room, the smell of dry summer grasses and baking yellow eath in the summer heat wafted into the cool hall, welcoming me home. The stone arch above my head transformed the whisper of my skirts to those of ghostly disapproving courtiers. Tomorrow evening, the hall would fill with the the formidable living version. They would not speak their opinion of me in my hearing, but I knew what they thought of the Mad Dutchess. The stone heads of three generations of Cortan's ruling family inspected me from atop the pillars. Duke Ippolito, the first Duke and conquorer of this Duchy over half a century ago, looked sternly down from the pillar directly behind the dias. His gaze fell intently upon the throne. I knew little of the man beyond the tales of his military genius. I did not like the manner in which he looked possesively and critically upon the throne, like a Master frowning upon an unconcientious student. These were his lands, this was hall. None would rule in it without his approval. To his left, sat the young face of Makala, lovingly elevated to the position traditionally reserved for ruling dukes by his father. Yet my first husband was not as I knew him, laughing and optimisitic. He looked proud and ambitious, a young man urging Cortan's ruler to conquor new lands for the glory of the duchy. There was no room for mischeif in that carved face, no hint of a desire to teach a girl to climb poles and ride bareback on destriers. This was all wrong. To Duke Ippolito's right, sat the face of Duke Ergino, old, wise, benevolent. That, I supposed, was how he wished to be known. Perhaps it was true. The man showed little kindness to me, but Makala loved him, as does my husband and Cortan's court. He was a popular duke, every man woman and child I have passed during this last week has stopped me to offer their condolences and say that Marsea will be lesser for his passing. I seem to be the only one of his subjects who does not truly mourn his loss. I should be more grateful, he gave me a name. Somehow, I cannot attribute any of the good fortune of my first marriage to anyone but Makala. It did not matter that the Duke looked upon the throne with benevolence, I would feel him staring at me from my left with the aloof disapproval I knew he held for me. I may have called him father, but his was Makala's father in truth, not mine.

I took the three steps up to the dias, and sat down. My blue train spread itself nicely in a half moon by my right foot. My feet reached the ground if I did not sit back, but the back of the chair towered over my head. I felt small and scrutinized by my predecessors. The great hall stretched on eternally before me, pillar after pillar after pillar formed two neat rows. Hundreds of barrons, generals, masters and priests would fill it tomorrow with more critical gazes. I had not stood in front of such an audience since the Counsel of Nine, eight years ago. I did not think I would enjoy the experience any more tomorrow. It did not matter. I had to sit on this chair for a day before passing it over to the ruler my father's court wanted. I had served Cortan to the best of my abilities. I had brought her back the lost duke to lead the duchy greatness, and I had born the line five sons. Nothing more was required of me but to accept the silver crown, and pass the scepter to my husband. 

My mother, Duchess Cybeline, had spit on me when I met her last night. She raised me from my bow and drew me close in a show of great motherly affection. Her whispered words sizzled in fury that I had arrived in Cortan nearly two weeks too late to pay Duke Ergino my final respects. I felt the thick wad of moisture land on my cheek. As the difficulty of traveling with six children, one of which only three months old would not move her hardened heart, I held my peace.  I doubt the rest of the court saw the act. My mother is far too discrete for that. I wiped it away as reverently as I would wipe away my the tears my widowed mother might share with me, and turned to meet the priest. I disliked her, as she disliked me. But I was no longer a child. She no longer held for me the terror she used to before I married a second time. My husband had done me the favor of teaching me what that word could possibly mean.

"Are you ready for tomorrow, duches?" a voice asked from behind my field of vision. I startled. Barroness Leila Paulis stepped before me. I was glad not to have to sit over this court, if even my friends made me nervous, how would I unite my reluctant barrons beneath me. "My apologies, your grace. I did not mean to startle you." A strong alliance had grown between the barroness and myself over the last eight years in Deyalorn. It was not as strong as the friebdship I had shared with soon to be scribe Malia, which at times ad strengthened and comforted me s much as my friendship with Timmon  once had.  Leila and I no longer spoke of the events that had thrust us together, we were wordlessly united in our hatred of Duke Griswold. Leila took care of Lucretious and Elena as if they were her own, fostering them, and helping them navigate the world of Deyalorn's White Tower during the long and frequent stretches of time when I was too unwell to perform my duties as a mother. There was an unspoken understanding that I would help her family when the time came. Thus far, I have kept my bargain, our mutualhatred of my husband has prooved mutually beneficial to both of us.

"As much as I will be. Are the children settled?" I asked, stepping down from the dias. 

"Makala, Lyca, and Elena are still at the Tower. Makala claims that he knows the way, and will bring them home in the afternoon. Emile has busied himself exploring the walls with my Ladislao with his nurse. Mirella is with her nurse by the turtle pond and Simone was a sleep when I last saw the nursery."

I sighed at the long recitation. How did I come to be a mother of eight, with another possibility inside me? My husbands zealous attentions and vigilance, of course. I returned from the first campaign after the Counsel of Nine, heralded as the reborn spirit of Sveva, the warrior queen of old who had given her life using the gift to protect Marsea from Szarvis's troops, for fighting with the men defending the base at Andona. I found that my husband had been released from his room in the Tower to live in a house on the bluffs where he could not move or hold an audience without the company of an observer from the Royal Court. I was, of course, to live with him there. My brother would not appear to interfere in our domestic disputes. Though Duke Griswold was still a prisoner of sorts, it would be cruel not to allow him his wife. Thus I became a prisoner as well, not of the crown, but of my husband's wishes every night. I learned first not to fight him, then to please him. It was easier, with Griswold, to let him have his way. I could live in peace, in my ever shrinking scrap of land, if I would only do as he asked. As my husband did not trust me to realize when I became pregnant, he posted two guards to check me every morning. I fought them at first. I was stronger than both women combined, both physically and as a healer. There was not much they could do against me. The second day, I found three armed guards outside my door as well. The humiliation was too much. I let the women examine my womb with their gift for three months until they found Elena inside me. Having done as my husband wished, he left me alone to write my book about the wasting sickness and other side effects of the mushrooms. 

Elena was born in spring. My husband was upon me again as soon as I left my confinement.  I tried every witch's trick to keep from Griswold's seed taking root. They worked for six months, and no longer. Perhaps they did not work at all. I spent that half year working tirelessly researching the effects of the mushrooms. They were clear for anyone who wished to examine Ezaro's health, but few cared to admit publicly that he still consumed the food regularly. It was a matter of my producing evidence against them faster than my opponents could  produce ways of circumventing the problems I discovered. In the end, I lost the battle. My guards found Emile inside me, about the same time that Cortan's Tower found that irregular use of mushrooms did not lead to the wasting sickness. So our gifted fighters did not lead to the wasting sickness. Before the spring campaign, Selvand had found a way to safely give the mushrooms to our gifte fighters, using my animal tests, the same ones that healers had laughed at me for performing. 

My husband made it clear that my life would be easier if I stopped fighting him on the mushrooms. I complied. He had showed me how brutal he could be if I refused. There are many acts between husband and wife that the world will never know of due to my ability to heal myself. There are scars on my back due to the fact that my guards took away my talisman each time they discovered a seed taking root in my worm.  Bearing a child, no one in the Tower would touch me. The humiliation proved too much again. I spent that winter teaching my two eldest sons to heal. Makala took to it, Griswold did not. I saw no one but Malia. By the time she finished her studies that spring, she moved into my house and served me as a companion and friend. It was she who noticed that increased efficacy that Griswold's new teaching methods had on my children, and suggested that I use my time to write a series of manuscripts on the methods. She would help.

Malia's suggestions help appease my husband's anger. He liked the idea that his way of teaching the gift proved superior to those currently used by the Tower. His nightly visits became less brutal, though no less frequent. It felt like I conceived Ophelia almost immediately. That pregnancy was nothing but a disaster, from start to finish. I had been isolated from my work with Master Joris's and in the infirmary for too long. Leila had kept me from disappearing completely from the view of Deyalorn's court by obtaining dinner invitations for me on a regular basis, but the social life of Deyalorn, as appealing as it might be for some like Sophia, was not enough to sustain me. I felt isolated and lost with Ophelia, in spite of Malia's and Leila's best efforts. Then my physical health deteriorated, and with it, the state of my nerves. I sent Griswold away to live with Timmon and Sophia in Lir, and gave birth to Ophelia a week later. It is hard for me to speak of her. She lasted a mere six months. It was hardly worth the Preserver's time to bring her into his fold if he would just let her go so soon. The gods dance an intricate and mysterious dance. If I had been better, stronger of mind and body, I might have saved her as I had saved Lyca. Instead, it was a waste of a life, and the last year of my prime. For the first time in eight years I regretted my decision to turn my back on the lethal poisons of the Black Riders. Isolated as I was, I had no means of accessing anything of the sort either from the White Tower or the barracks. I had no more use for my life. It would seem that Marsea did not either, unless it was to bear more sick children. It was Malia's love and devotion that helped me recover from my grief and remorse. I lost so much when I lost Ophelia. I never fully recovered my physical health, nor my courage. The years of my prime were gone forever. 

Leila touched my shoulder. "You are looking backwards again, your grace."

I blinked, then wiped my eyes. "The past weighs heavy. I wish Ophelia's grave were here."

"It will be easy enough to exume her body and bring her bones here. What is it that really bothers you?"

"My children. Mirella is two, Simone three months, I've had two miscarriages since Ophelia. All four pregnancies have been difficult. My husbands wants me out of sight so he may rule in my name. My body is determined to give him his way, whatever my mind may say."

"Again?" Leila asked, dismayed by the news.

I put a finger to my lips. No one else knew. Not even my husband this time. Under the grave circumstances of our hasty departure from Deyalorn, bringing my guards seemed uncouth even to my husband. It was unseemly to imply that even in this time of great grief, he would be so strongly interested in creating another heir, when Cortan already had so many. The need for appearances spared me my dignity, if not the nightly visits. That my husband did not know of my pregnancy was the one sliver of hope I had at the time being. 

Leila put her hands around my waist and led me gently from the hall. "Then we will attend to the matter ourselves," she said. "In the mean while, you have other matters to see to. Master Carlotta wishes to see you."

"I do not wish to see her," I snapped. I had recommended Carlotta to Turina in an attempt to see to her safety. She spent so much time with Ezaro and my husband, presumably under constant pressure to use the mushrooms that her health had started to deteriorate. She had lost a worrying ammount of weight when she returned from her lecture tour with Ezaro. Timmon had reprimanded me for the politically foolish decision. He said that I made friendships with my political rivals as if they were men I had served with in war. My instincts were wrong and dangerous, he had warned. He may have been right, but I was born without a family. I had no blood bonds to guide my loyalties. If I could not be loyal to my friends, what type of a person was I? Whatever the misbegotten reasons for my promoting one of my cheif political rivals, I had seen her safe. She had three children and a successful career as a Master and the head of the women's school in Escasaine's Tower, while I was a mother of eight and has spent my last three years a school teacher for first and second year students in Deyalorn's Tower. I had possibly saved Carlotta's life. What she did with it was not my concern.

Leila's embrace tightened. "Then she will take her request for an audience to your husband, your grace. I do not know what she wants, but in your husband's hands, she would certainly be an useful ally." 

I sighed. I had pressed my brother to assign Barron Paulis to be his eyes and ears over my husband in Cortan, not because I thought Barron Paulis to be particularly suited for the job, he seemed no more or less effective to my untrained eyes than any other courtier, but because I needed Leila by my side. She and Malia had been the bulwarks for my well being in Deyalorn. Malia had lived with me, slept in my bed every night after my husband had finished with me, reminding me of the reasons my life was not a wasted effort. Leila was a married woman with three small children of  her own. While she could not provide me with that comfort, she could continue to look after Lyca, advise me, and keep me from wandering down the dangerous dark alleys my mind had developed the habit of lingering in since Ophelia's death. "I cannot see her alone." I did not have the courage to see the woman I had thrust into fame and success while I sank in the forced anonymity of my husband's desires. 

"Then I will attend you," Leila asserted with a reassuring smile. "The Duchess of Cortan need see no one alone."

Leila led me to top of the west tower to the two rooms Makala and I had once shared. His room, the larger of the two was now the nursery for my six children in residence. The other I now occupied again. I liked the room. Even during the years I lived in exile in Cortan's Tower, or with my husband, or in Deyalorn, it had been my chamber. I liked the view it had of the White Tower. I could see Timmon's house from it, now not longer a small three room building, but a large two story house with a lavish garden that most Barrons would be happy to have as a second residence near the Duke's castle. The call of the conch from the Liri temple calling the community to gather rang through the morning twiglight, as I dressed to attend morning prayers to the triumverate. The room reminded me of the things I still held dear in my life, as much as it held memories of the life I could have led. I like the rooms. Choosing to live in them again had been a gamble, but I hoped to be uplifted, not weighed down by what they represented.

The Mother's Balm sat in the antechamber to both rooms, playing with two year old Mirella. She appeared before me in a white healer's robe, adorned with a master's collar, I did not fail to notice.  Nearly thirty, motherhood had made her plump. Broad hips bulged under her loose healer's robes, her talisman sat high atop a pair of full breasts. Her once flawless round cheeks drooped a bit; no longer rose tinted, they looked firm and matronly. She would be as formidable a head teacher as Master Adele ever had been. This was not a school room, I was not her student. Carlotta left off talking to Mirella and bowed deeply before me. Why did it always bother me when healers bowed before my station as a duchess instead of address me as my rank in the Tower? I let Carlotta stand bent before me while I asked Mirella about her morning with the turtles and sent her off with her nurse for a meal and a nap. With my duties as a mother attended to, I gave the Master before me my hand. 

"What may I do for you, Master Madriano?" I asked, ushering her into my room. Barroness Paulis poured two cups of wine, served, then seated herself behind me. Carlotta glanced at our chaperone. "Anything you have to say to me, may be said to the Barroness, Master. There are no secrets between us."

I had no desire to make this interview with Carlotta easy. She had played on my weaknesses last time we had met, evoked in me a strong sense of pity for her for having been maltreated by my husband. I had no intention of letting her play upon my once fond feelings for her again. Carlotta took a swallow of wine and began uncertainly. "I would offer you my condolences and my congradulations, your grace, but I know that you were never close to the duke, and that you have cared little for the life of a duchess."

"Then what is it you wished to see me about?" I asked with a hint of impatience. 

Carlotta looked at Barroness Paulis again, then washed down her pride with another sip of wine. "I wished to apologize, your grace."

"Apologize?" I repeated. I was glad of Leila's company. If Carlotta was about to prey on my emotions again, Leila would not let her.

"I was wrong. We all were. Last autumn's blight proved us to be. You were right about the mushrooms, they are too dangerous for Marsea's Towers." 

"You should apologize to the Masters who lost their lives to make room for your promotion, not me. I have little to do with the debate anymore." When my husband and Ezaro had discovered what the called the safe powder, healer's flocked to the black powder like swans to bread crumbs. Every two weeks a healer could get a chance to enhance the ammount of gift the Preserver had seen fit to give her, and play with a fire that was not hers by rights. Then Allepo's Tower produced a bad harvest of mushrooms. No one knows why the year's crop was poisonous. It was one of the most terrifying blights Marsea has ever faced. Without warning, great Masters had strokes and young promising healers fell convulsing on infirmary floors. It took two weeks before most Towers figured out the source of the strange illness, another two months for the crown to completely outlaw the mushrooms again. By then, Marsea had lost one in four of its masters and over one in twenty of its healers. Many heads of Towers died, their positions filled by men who had spoken against the mushrooms from the first, or who had never tried the fruit themselves.  Most of the White Towers feared the mushrooms. I do not know what Dario or my nephew, King Gustav, had been thinking. They closed the town gates only after half the town had been razed. Did they really not think that the seven years of access to the mushrooms would not change the attitude towards the fruit for at least one generation of ambitious healers? The fire still smouldered beneath the ashes. Moreover, my ambitious husband still wished to weild the gift again.

"I wrote you," the master offered, as if that could paper over our the rift in our friendship and atone for the success of her carreer and the destruction of mine.

"My husband recieves my letters. I only see the ones it pleases him to show me. Yours did not please him." I found myself wishing this pointless interview to end. Carlotta had once wanted a quiet life surrounded by children and domestic duties. Instead, she was well poised to be the next head of Escasaine's Women's Tower. It would seem that we had fulfilled each other's dreams as we had once finished each other's sentences. "Your husband wants a barrony in Cortan or in the lands to the south. He has sent you to plead with your friend. I bear you no grudge, but you have no favour with me either. I have no lands to spare for Commander Madriano." 

"You are correct, your grace," Carlotta said, caught off guard by my accurate guess. "That is why my husband wished me to speak to you. That is not why I came. I only came to apologize."

A cynical smile turned up the corners of my mouth. Carlotta was a wily woman. The tide had turned. Those who had called me mad, or afforded me the respect due to a frightened girl, now wished to ally themselves with the duchess who had stood against Marsea's Tower. It made sense for Carlotta to be among them. The problem was, the duchess had removed herself from the debate. She had become a mother and a school teacher. "I accept your apology, then, my friend," I said with a formal warmth, "but too much water has passed between us these six years."

The Mother's Balm looked disheartened. "I know," she sighed. "You have suffered greatly at your husband's hands. I never meant you to. I wanted to help you make peace with him." Of course she did, I thought. I had not forgotten that she had thought me mad for knowing my husband to have sinister aims. Crocodile tears sprung to her eyes. "I wrote you when I heard that Ezaro died. I wanted to thank you for caring for him." She stopped to steady her choking voice. "These intervening years have been cruel to both of us."

I had heard enough. Whatever Commander Cesaro Madriano had done to Carlotta for having an affair with Ezaro could not have compared to what Griswold subjected me to. However much she may have loved Ezaro, however painful it may be to hear that a lover had poisoned himself, I could not imagine it comparing to the agony of throwing dirt on the six month old face of one's own daughter. I did not wish to hear of Carlotta's suffering. I took a deep breath and smiled. Leila was right. If I would not keep Carlotta as a friend, my husband would use her as a weapon against me. "It has been," I sympathized. "Ezaro spoke often of you during his last days. You were lucky, not many men love as he loved you." He was twenty five when he had killed himself. Even in his paranoid last year, he was a skilled healer and a good teacher. Marsea is lessened by his passing. Griswold would not acknowledge his part in this tragedy. Ezaro had taken his own life, his death was his own fault. Someone had to see to damage my husband left in his wake.

"Thank you," Carlotta said as I took a sip of wine. "I wish to serve you, if I can." 

I believed her. I also believed that she would serve my husband if I refused. I should count myself grateful that she came to me first. I rose and pressed Carlotta's hands warmly in mine as I led her to the door. I felt tired. "We will speak more often now that you are here."

Leila moved silently around my room after I closed the door and turned my back on Carlotta. The curtains of my bed flew open, those before the window flew closed, leaving only a pale shadow of light on the stone wall where the cloth ended. My barely touched wine cup appeared in the comforting darkness. I tasted the sour drink and the salt of my tears as I stood with my back against the door, bracing it closed against the world. They called me the mad duchess. It was not for standing stubbornly for years against the will of Marsea's Tower. It was not for defying a husband twice my size with a reputation for cruelty. My nerves had a habit of giving way when faced with the failures of my past. The displays were frequent and often public in the year containing Ophelia's death and Ezaro's suicide. I brightened some with Mirella's birth, but have never been cured. Malia and Leila have learned to care for me and sheild me during these times; to keep me hidden from critical eyes and to keep me from burying myself completely in the darkness I so often craved.

Leila removed my outergarments somewhere between my silent sobs and led my shaking shoulders to bed. I sat hugging my legs, my eyes pressed against my knees. It had been so hard to see the Mother's Balm, her title given to her by a method I had devised on a child Leila had borne. Carlotta took the credit and never looked back. A warm woolen blanket draped itself around me in the hot noon air. It weight comforted me. Carlotta was the head of the girls school at the main Tower of a promising border duchy, I a teacher of first and second year students. Escasaine's Tower was already a place mother's visited when pregnancies became difficult, or when one feared a hard delivery. Carlotta had birthed princes in Lir as well as countless Marsean barrons and dukes. Her husband was not jealous of her abilities. Commander Madriano used them to his advantage. She, I was certain, did not sport a scar on her back from her husband's rage while she bore him a child. She wore a master's collar while I did not even don the whites of a healer most days. 

A pair of hands picked pins from my hair. My braid uncoiled itself from atop my head and unraveled slowly under the patient coaxing of those loving hands. The shuddering of my shoulders slowed. This was an old routine, comforting in some ways. "There are dozens of women in Deyalorn who would not be healers but for you," a voice whispered. Malia would always start with those words as she ran a comb through my hair at night after my husband had made his use of me. "They look up to you, even knowing how you suffer," she would often add. Malia was not here. She was less than a year away from becoming a scribe in Deyalorn's Tower. For the last two years she had been engaged to a scribe for the triumvertate, a man named Paulo, a few years older than her. They would likely marry during the next Day of Unions but one. I could not ask the girl to leave house and home and happiness to follow her miserable friend to these sunbaked lands she did not know. Leila came instead. "There will be dozens more in Cortan's Tower over the next few years, if I know you," she said. I felt her fingers massage my scalp, coaxing the tightly bound hair to lay at ease. "You have written five well recieved books. There are three classes of exceptionally skilled youngsters at Deyalorn's tower due to your ability to teach and embelish upon the ideas your husband started." I felt the comb work its way down my back. I had not reacted well to this means of calming my nerves the first time Malia had tried it. I had snatched the comb out of her hand and beat her with it, breaking the comb and leaving the girl bruised. It had not been two months since I had burried Ophelia. Nothing I could do would brng my daughter back, I had yelled. None of it could keep my husband off me. It did not matter what I had done. The world called me mad and the gifted did not want me. The girl's terror at my sudden rage called me back to myself. I begged her forgiveness as I healed her. I loved her, I would not survive without her, I had said. That was all true. I did not wish to turn into a monster like my husband. That was even more true. Malia forgave me. The next night, she tried again. I had no anger left at myself anymore. What had happened had happened, whether I had let it happen, or it had been forced ,upon me was not a question to ponder. The Preserver led me down the path he would, and I must make the best of it. That is it unfair is not mine to question, but to bare. "Lir and Marsea would not be as close as they are without your work in Deyalorn, either to obtain the settlements, or in supporting their arts and religion. You have helped bring peace where your husband wants war," Leila concluded. My accomplishments ammounted to little compared to the ambitions I had once had sitting in this very room. "We will rule as the first kings did," Makala used to say. "You will be a warrior queen by my side. No one will stand between us when we ride to add lands to Cortan." How long had it been since I had trained? Elena was six. At least that long. I had once been the most powerful healer in Marsea. What could I not have done for Cortan's Tower had Makala lived? Leila pressed a second wine cup to my hand. I had stopped shaking by this time. I took the wine and drank. I lay down when she removed it from my hand, falling asleep with her hand resting on my shoulder.

\vspace{.5cm}

My husband came to me the night after I became the duchess of Cortan. The White Tower gleamed in the light of the full moon, rising white and glittering above the dark walls that made up Cortan's barracks. It felt so far away compared to colourful riot that had congradulated me throughout the afternoon and evening in the castle below me. Mine was a life in court now. The life I had once had in the Tower that had raised me was over. I was wife, mother, and now duchess. There was little room for anything else. Head Alerio had asked me to dine with him in a few days time. I had opened my mouth to decline his hospitality when Leila accepted for me. My husband did not want me practising, I reminded her. I needed powerful allies, she reminded me. I accquiesced. Accquiescing had become a habit of late, whether to my husband, or to the restrictions of my feeble mind and body, or to Leila's instructions on how I lead my life, my own will had little to do with my actions anymore. 

"That was a passable performance," I heard my husband's voice say.

I jumped. He had, inspite me best efforts to instruct my guards to the contrary, entered my room unannouced. I rose from my seat by the window and bowed. "Thank you, my liege."

"It seemed to me you hestitated an instant before handing me the sceptre," my husband said, giving me his hand. 

"No my liege," I did not rise after kissing his hand. He was angry. I was frightened. "I did not mean to. I have no desire to rule Cortan."

"A good thing too," he said, strolling through my chamber, inspecting it. "This duchy does not need you madness to bring it to its knees."

As he was no longer standing over me, I looked up. My husband walked through my room with a quiet terrifying confidence, as if it were his own. He touched everythng as he went, opened drawers, moved objects. He took books off their shelves, flipped through them casually, then dropped them carelessly on the ground, bending their pages or damaging the binding. He paused at my dressing table to rearrange its contents, knocking over a crystal of rose water in the process. The room filled with a cloying smell of roses as the vial poured its contents onto the ground. At my desk he pocketed my seal, uncapped the inkwell, then lay it on its side. He waited five heartbeats before righting it again. The pool of black spread across my table until it dripped, then poured onto the blue upholstered seat before it. His hands he wiped behind me on the blue curtains that decorated my windows. This was a lesson for my tardiness, a reminder that this duchy was his, not mine. I had nothing here that belonged to me, even this room was mine by his sufferance.

"Are you pregnant yet?" my husband asked from the window, his hands now clean.

"No," I lied. These last two days had been hectic, filled with audiences and ceremonies either of ascension or mourning. I had not had time to remove the seed inside me. My husband had not yet had time to find guards to examine my womb. 

"Then why are you still dressed?" he snapped, then started a second round of my chamber.

I swallowed the bile that had collected at the back of my throat and worked quickly at my laces. The ransack of my chamber would end when I was ready. I had been crowned in an austere blue gown that protected my modesty and made me sweat, rather than the sleeveless gowns worn by Cortan's women, that revealed too much of the arms and back but protected them from the late summer hear. I had chosen to be known as harsh than scarred. My choice teased me now, the numerous buttons and laces took time to undo. A white robe found itself into the hearth, and a pile of vellum onto the black pool of ink by the time I was in my underclothes. 

I closed my eyes as he mounted me. My life was easier if my husband got his way. There was no escaping this act. Even if I cleaned my womb of this pregnancy, these nights would contiue. There would be  other pregnancies. Would I void myself of them as well? Griswold would eventually notice, he would evetually ask. What would he do then. When he realized that I destroyed his seed, he would not react as he had to todays percieved slight, calmly teetering on a cliff of cruelty. Would I relive the fury he unleashed in my direction when I returned from Escasaine, a hero, to the house where he lived imprisoned, or would it be closer to the quieter torment I faced after I lost the pregnancy after Ophelia, when he found countless ways to smear my name and earn me renoun for my madness? As dire as I knew the consequeces of this pregnancy to be to my health, did I have the courage to find out?

"Our son is coming," my husband said, grunting and rolling off me. "I would imagine he will be here in four weeks time." He rose to start dressing. I did not dare move. "I look forward to raising him. He will rule Cortan, not after me, but after you. I will teach him everything I know." My heart sank. I had send my first born away four years ago to keep him from my husband's claws. I had not spoken or written to Timmon in that time, not wishing to be the force that destroyed his fragile home. Why had the ambassador sent my son back? Was his sense of duty to Cortan stronger than his friendship to me? I could not take care of the sons I had in my care. How would I care for yet another? "It was wrong of you to send him away," my husband continued with an ophidian smile when he finished. "A child of Cortan does not belong in Lir. Still, I will forgive you. The boy may have many interesting tales to tell about his guardian and your lover." He stood over me, a cobra over its mesmerized prey, waiting for me to fall.

I closed my eyes to escape his gaze, but I could not move away. My husband could not possiby think Timmon a threat any longer. I no longer had the courage to defy him, let alone take on a lover. Yet his face filled with lust at the thought of tearing this man I had once loved from limb to limb, feeding his reputation to the court's vultures. Had I ruined Timmon, I wondered through the panic flooding my mind? Had I ruined him by the simple request that he protect my children? If I had written him in these last four years, instead of stubbornly maintaining this empty bleak silence, might I have warned him of my current condition, impressed upon him the need to keep my son away, and thus saved his family?

"No word of joy at the return of our son?" my husband asked, with a syrupy sarcasm dripping from his voice. He tisked his disappointment when I did not open my eyes or respond. "You are a poor example of a mother. Perhaps you should spend more time on your children, and less on your dabbling in the Tower's business." He stood over my bed for a moment longer, letting my anger and frustration bubble up inside me, find no release and fester into a madenning fear.

When he finally left, there was no one there to draw the curtains and douse the candles for me. I drew the curtains around me against the too bright room, pulled my knees to my chest, and burried myself under the pillows and blankets on my bed. I could tell no one of what had passed tonight. In Deyalorn, Malia would have comforted me after these encounters, but even then, I spared her the details. Malia was barely a woman, with no knowledge of men or marriage, still in her prime. She was bright, loving, ambitious, she had the life I wished I still lived. So much of what she had that I had lost was her innocence. I could not take that from her with stories of my husband. In Cortan I had Leila. United as we were in our hatred of my husband, she was more of an ally than a friend. I could not share with her my domestic shame. 

\vspace{.5cm}

It took me a week to gather my courage to accept Head Alerio's invitation to dine with him. My husband did not wish me to practise my art, he had made that much clear, though he had not expressly forbidden my meeting with Master Alerio. Over the last eight years, I had become a healer against his will, I marched against his will, I had worked with Master Alyosus against his will, I had fought against him for each of these endeavours. When I gave all these up to teach and tend to Ezaro after Ophelia died, my husband granted me a modicum of peace and a private room where Malia could stay with me at night. Everything I had, I had at his pleasure. He could make my life easier if I only did as he asked. As duchess, he thought it demeaned our great military to have the a school teacher sign orders for their deployment. It would be better, in his opinion, if I restricted myself to a more domestic domain. In the end, I accepted head Alerio's invitiation only to appease Leila. I would not meet with him publicly, and I would not meet with him in the castle. I would hear what he had to say, but I would not join Cortan's Tower's life. My future lay with the ungifted. 

I touched the curved white wall of the Tower's steps reverently as I climbed the spiralling stairs to the Head's quarters. I passed the door that led to what had once been Master Adele's office. As a child of this Tower, that landing held terror for me, I often stood trembling upon it, called to speak to the Head of the Women's School for sneaking into the library, or out of my dormitory and into that of Carlotta or some other class. I paused there now. I had never climbed higher on these stairs. As a young woman, the door before me had become my gate to Master Adele's knowledge and wisdom. I had performed my first miracle under her guidance. It seemed a lifetime ago, before my husband came back to Cortan to remove me from this Tower. I turned my back on the door and climbed further. My heart beat quickly in my chest as I approached the Head's rooms. Head Alerio had been good to me once. He had taken me as his student when no one else would touch me, we had differed in our goals during the Counsel of Nine, but he had offered me protection when few others would. Yet I would gladly face the disapproving faces of the leaders of my father's court on my corornation again rather than reach the top of these winding stair. Growing up within the walls of this Tower, the Head's chambers symbolized all that was awesome and wonderful about this institution. It visited both my dreams and my nightmares. With the Tower now closed off to myself by my husband's will, my feelings for the Head's chambers remained frozen in that childhood fantasy. I reached the top of the stairs and knocked timidly at the door.

A boy in healer's robes let me in. The sight of his simple white robes made me feel self concious of my courtly regalia. I would appear before Head Alerio in the giuse of  a woman not of the Tower. Duchess or no, only a completely ungifted woman ranked lower. I followed the boy into a dazzling white room. Not only were the walls and ceilings the white stone of the tower, the floor was whitewashed. The windows hung with white curtains, embroidered with ivory thread, of Liri silk, I noted with pleasure. The furniture in the room was made of ivory, marble or granite, the cushions made of white velvet. Everything gleamed or glittered or shone, like a landscape from Deyalorn the morning after a heavy snow. The only color beyond the conspicious blue of my gown was the black face of the Destroyer in the enormous marble carving of the triumverate that took up nearly the entire wall adjascent to the door I entered through. 

I sat on a white stone bench by a window and waited, nervously, for Head Alerio. I stood when I heard the soft clatter of the pearls adorning ceremonial robes. Head Alerio seemed to have bleached with his new station. His hair had greyed to nearly white over the past eight years. Only his brown face and hands showed that he was a living man, not a permanent fixture of this gleaming white room. "Good evening, Head Alerio."

The old man bowed. "My condolences and congradulations, duchess. I am honoured that you accepted my invitation." I gave him my hand. I would longer engaged with the Tower, I reminded myself. The Head honoured me by adressing me a his duchess rather than the healer I was not. He did not push me away from the world of Cortan's Tower, I chose not to be a part of that domain any more. "It seems there may be more condolences to offer you than congradulations since we have last met. I was sorry to hear of the loss of your daughter."

The old healer's word's touched me. I had been hearing words consoling me for the loss of my adopted father for so long, I had not thought that there may be someone who would think to speak to me of anything else. "Thank you, Head Alerio."

He led me to a white high backed armchair at a small circular table mosaiced with ivory and mother of pearl then took his seat across from me. The boy who had seen me in produced two crystal goblets, both of which he filled with a clear red wine. The liquid cast a long red shadow in the late evening light, two red quivering tear drops on the patterened white surface. Head Alerio raised his glass to me. "Now that you are in Cortan again, your grace, I hope this will be the first of many meetings in this Tower."

Glass chinked against glass, but I did not second his hope. "I doubt that very much, head. My duties as duchess threaten to keep me away."

"That is a pity, your grace. This Tower was hit hard by last autumn's blight. We are in need of Masters. A woman of your experience would be invaluable to us."

I took a large swallow of my wine. Being hit hard by the mushroom blight meant either that a Tower was staffed with an unusual number of sick and elderly healers, or there were more healers experimenting with the mushrooms than was strictly allowed by the rulings after the Counsel of Nine. The vast Majority of Cortan's healers marched every spring. I suspected that the latter situation had put the Tower in its current position. "I am sorry, head, but my time is spoken for."

"I am sorry to hear that, your grace, and also for not being able to do more for you since the Counsel of Nine. It could not have been easy living with your husband after you denounced him."

Apologies. What use did I have for apologies such as these? To any other man I might have reminded that I had seven healthy children; what more could a woman want? Head Alerio was not any man. He had fostered my ambitions for a life beyond hearth and home. He had stood by my side once when almost no one else would. He treated with me honestly and frankly, as is the habit of the Tower. He deserved my honesty in return, not the winding words of court. "I thank you for your concern, head. These last years have been difficult, but do not trouble yourself with guilt. We parted last on unhappy terms, but you gave me sound advice, for which I am grateful. "

I watched a shadow of tension leave the old man's face. The apology had been in earnest, but I did not know what this man wished of me. Since arriving in Cortan, I have been beseiged by people of Cortan's and Escasaine's court vying for my favour either because my husband would not give it, or because of my once outspoken defiance of his goals. None of them were earnest, many of them I rejected under Leila's watchful eye. I was not so naive as to think that Master Alerio did not seek advantage for Cortan's Tower in this interview. Head Alerio was a shrewder man than I. Cortan's Tower had been deeply involved in researching the effects of the mushrooms over these last four years. It was they that discovered that sporadic use of the mushrooms were not harmful to gifted fighters. It was their discovery that led to the eventual abuse of the gift enhancing powder, and the deaths of so many gifted. Their masters had tried the wild magic, illicitly,for years untold, a practise that must only have increased after the Counsel of Nine. Master Alerio knew my views towards the fruit. Why did he wish me to be a part of his Tower?

Dinner arrived, served in Liri style. We washed our hands in a wide crystal bowl, water poured from a white enameled urn, with a conch shell for a spout The serving boy brought two white stone plates, each with a large mound of white rice in the center. Around each plate he set seven bowls containing courses ranging from vegetables on the left to meats and fishes above and ending with fruits and milky sweets at the right. Seven dishes for the seven books to the Liri gods, for the seven celestial orbs, Sama Araki had explained to me once. This was the first Liri meal I had had since leaving Deyalorn. My husband did not approve of my supporting the Liri cause as much as I had done in my days before becoming the duchess. My loyalties should be to my Marseans subjects only, he claimed, not to the Liri newcomers. 

Head Alerio turned the subject of the conversation to the most recent of my three books on my husband's teaching methods, though he gave me full credit to my ideas. It terrified me that my husband hear my work referred to as anything other than an expansion of my husband's theories. Duke Griswold did not take kindly to my successes. "You are too kind, head. The ideas in my three volumes are due to my husband entirely, I am hardly more than a scribe to his genius." 

"I disagree, your grace," Head Alerio's tone stopped short of calling me a liar. "I have read what you have written, as well as Master Alyosus's praise for your diligence and accumen. I have also heard your husband speak. What you have written shows an insight into a child's mind that I have only found in experienced teachers. Never having children, I would not know. Might a dedicated intellegent mother also have such insights?" I had nothing to say to this dangerous praise. I could not accept it. I hid my discomfort with a large swallow of wine. 
 
"Master Leon, the head of Cortan's school, wishes to step down soon, your grace." Master Alerio continued, changing the subject. "I am looking for a replacement."

"I wish you luck. There are some names from Deyalorn's Tower I could recommend if you are interested."

"Thank you, but no. I would like you to hold his position, your grace. With your insight, we would attract more promising students that Deyalorn's Tower does."

I ceased eating to look at my old tutor. His mouth curled slightly at my surprise. No woman had ever been head of the school in any Tower in the country in Marsea's history. Women may teach men, but they can only lead the girl's school and the Women's Tower. I opened my mouth to argue that I was not qualified, then shut it again. He liked my teaching techniques, he wanted me to continue my work. It had been so long since someone had recognized me. I wanted to thank my old tutor for watching over his student. I could not. My husband would not approve. "I am sorry. That is impossible."

The wrinkled brown brow before me drew itself into deep furrows. Head Alerio spoke quietly. "When I left you in Deyalorn, I offered you my protection. I am glad that you never found yourself in a position to need it. Now that you are in Cortan, the offer still stands. I have spent the last eight years purging this Tower of Duke Griswold's influence. He will not be able to touch you here."

But I would have to go home, I thought, and he would be waiting. I was duchess. I had duties to my court that had to come first. I could not retire to the Tower to live the rest of my days there. Still, the offer pushed so hard against my desire to be recognized that it hurt. I could be a Master in Cortan, the head of its school, respected among the gifted again, so much more than simply a mother and wife. Was it worth the risk, I wondered? If I left for the Tower, who would protect my children. Woldino would come home soon. My husband would not hurt his son as he hurt me, but he would corrupt him. He would use him to try to get at Timmon. If I had any strength to fight my husband anymore, it would have to be to protect my son and Timmon. I could not aggravate Duke Griswold on anything else.

Head Alerio waited a long moment for me to mull his offer over. Then he continued. "You once said that any Tower would consider itself lucky to have you amoung their healers. You may be past your prime, but you have written five well recieved books. With death of so many masters in Marsea, now most Tower in the land would consider themselves lucky to have you as a Master. You were frustrated then, I recall, at the lack of courage of those around you to champion your cause. Your situation has changed now, my words may come too late. Do you still have the courage to stand against the duke?"

\vspace{.5cm}

I spent the next few weeks putting off Head Alerio's question. I could not bring myself to reject his offer for a life in the Tower. Head Alerio offered me everything I had once dreamed of. At the same time, I could not find the courage to flaunt my husband's wishes so obviously. In Deyalorn even at the end, I had occaisionally gathered the courage to support a group of minstrels, or sit as one of a panel of judges to elevate a student to a healer. I would venture out of my domestic domain, usually with Leila's help and find the strength to weather my husband's jealousy of my public image or anger at my disobedience. This was different. Being the first female head of a Tower's school was not the same as encouraging an ambitious female student against her family's desire. This was a larger proposition, a more sustained opposition. I did not have the resources to mount this war against my husband alone. I did not have enough faith in the support my scant few allies could provide. 

I put the decision off. I told myself I needed to help my children settle into their new home. I had been too ill, my last years in Deyalorn, to look after them all, each pregnancy since Emile forcing me to retreat to my bed for months at a time, my mental frailty confining me for much of the rest. My husband was correct, I had neglected them. In my last years at Deyalorn, under Malia's reccommendations, I had spent most of what energies I had on my teaching, trusting Timmon, or Master Alerio or Leila to care for my older children. She thought it crucial to my recovery that I keep a foothold in the Tower, however feeble. I did not have Malia to guide me here. I had left her safely behind to her life as a scribe in Deyalorn and what I prayed would be a happy domestic life. In Cortan I was duchess, mother and wife. I had Leila's counsel on how to move in court, her husband's protection, as the crown's representative, against the Duke's attempts to harm me, and Griswold's expressed wish that I spend more time with my children. My life would be easier if I let my husband have his way.  In Deyalorn, I had fought to be a healer before all else, fought my husband and lost. In Cortan, I would be a mother.

Makala, Lyca and Elena were adjusted in their own ways to their new rooms and companions in Cortan's Tower. Emile made friends with Commander Dielo's son Maurius and the two, along with young barron Ladislao Paulis spent their days playing under the feet of men at the barracks who had better things to do than avoiding tripping over dukes and barrons. I would have to find a good house to foster him if he did not display the gift in a year or two, I could not let him stay with his father to be corrupted. Emile was an impish rowdy boy, he would brighten the home of any barron who took him on. When not in school, Elena could be found in the company of Carlotta's daughter Cesara. I did not worry about her much. My husband did not concern himself with his daughter, even when she displayed the gift. Leila had extended her protectiong over her as well as Lyca. Several times a week, Leila visited both children in Deyalorn's Tower, giving them the comforts of home and the loving attention their mother could now. Elena learned her manners more at the Paulis table than at Cortan's. Nine months older than Cesara, and prouder by nature, Elena enjoyed the role of duchess and older sister she got to play with the Master's daugter. Carlotta lingered in Cortan after the coronation, though her husband had returned to his post in Escasaine. Hers was no longer the happy marriage their courtship had promised to be, but Commander Madriano was only jealous of her love for Ezaro, not her gift. Carlotta was resourceful. She would survive. In the meanwhile, she spent time around my children, making sure Panfilo and Serena both played well with Mirella. It was an obvious attempt to garner my favour. I did not object to Elena forming a bond with Cesara, they were children, I could not stand in their way. I had no intention of letting Carlotta get anything other advantage out of my misery.

Lyca, a small boy who never outgrew his quiet nature from his infancy, made no effort to make friends in his new home. He had been shy and reclusive in Deyalorn, but Leila assured me that he did have friends. He spent time on Leila's father's barges, and was well liked by the lesser members of the Barron's household. The barron had a younger cousin at the Tower who had taken him under his wing, protecting my small slight boy from being bullied. In Cortan, my son seemed less happy. He spent his free hours playing by himself in the dormitory. He missed Master Alyosus and Firi he told me, when I visited him. Master Alyosus's office at Deyalorn's Tower had become half a creche during our last years in Deyalorn. When Sophia discovered Firi's gift, she sent her daughter to live with her father. The girl often spent her free afternoons reading in her grandfather's office. Makala saw more of the saw more of the kind old man than he saw of his father, often interrupting the old man's discussions with his students to be spoiled by his attention. When Lyca started attending the Tower's school, he accompanied his brother on these visits to Master Alyosus. Firi was in a class behind my son, but they studied an practised together in Master Alyosus's office all the same. These images painted by my old tutor heartened me. My family and Timmon's mmingled again, though he was not there to see it. "The ambassador will be back in just over two years," I comforted him. "I am certain he will bring Firi to Cortan then. That is not so long to wait." My eight year old looked at me with doubtful eyes. I laughed at my foolishness. My words had been meant for someone else. "I have ordered fresh cider and walnut pies for this years Day of Unions, so you need not be homesick."

Lyca's face brightened. "And pears?" he asked hopefully.

I straightened the belt of his white robe, drawing him close to me as I did. "There is always fruit in the castle's kitchen for glazing the meat. You should see if you can steal some from time to time."

My son grinned as he threw his arms around my waist. "Thank you, mama." 

He walked with me down to the garden, promising to try to play. He never bounded as other children his age did. He was always careful and methodical, almost hesitant in his actions. As an infant, I had feared his quiet nature meant something was wrong with him. By the Preserver's grace, he has grown up to suffer from nothing more severe than shyness. Duke Lucretious was second in line to the duchies of Cortan and Escasaine. I had cured him of his weakness once. I had to find a way to cure him of his shyness. Someone had to teach him to rule. "What do you think of Barron Paulis?"

"I like him," Lyca returned without guile. "he spent a long time introducing me to the barron's at the coronation that father left out, and he crosses swords with me sometimes when the barroness or father isn't looking. I don't think father likes him."

"No," I laughed. "I imagine he does not. How would you like to be his page?"

"His page?" my son sounded startled. "But I have the gift. I am part of the Tower. It is beneath me to be an errand boy."

The Tower, and I was certain my husband, put these thoughts of aloofness into my children's heads. It served my husband's purposes. Lyca was a small, shy boy. He would never be strong or have a commanding bearing. If he spent his life in the Tower, as I had, he would never learn to earn the respect of his barrons and generals. His father, of course, would help him, as the need arose. My beautiful, fragile, gifted son would become a pawn in Griswold's game. "You are a clever boy, both skilled and strongly gifted for your age. Everyone says you study hard. Your schoolwork still leaves you with five free hours every day to mope in your dormitory. It will do you good to see how the ungifted run a duchy."

"I don't mope!" he cried. "I do not play with Makala's friends because they are all bigger than me. They send me high into the Tower's orange trees to pick fruit from the smaller branches, but they run away and won't help me down when the head boy comes to scold us."

I bit my lip to avoid laughing at my son's misery. There had been a time when I was the youngster chosen pick ripe fruit from narrow branches. I was fortunate that Makala had taught me to climb. I ran with the rest when the head girl tried to catch us. I never needed a hand down. Lyca was small and weak, born a duke and still bullied. I could not blame him for having been born early. As his mother, I had to help him learn to rule. "You are gifted and a duke. If you have time to sit idle, you are either failing the Preserver in your duties, or neglecting your people."

"That's stupid. Makala doesn't have to page! Father always says you love him more than me." Lyca turned on me with an anger he should have saved for his older brother's friends. His words stung for the truth that he could not know. It had only been a little over a year that Lyca had not been in a class with me, that I had not overseen his education. Was it possible that my husband had turned my quiet retiring child from me so fast? I could not imagine how else Lyca, raised under the careful eye of Barroness Leila, could speak to me as if I were only a woman and mother, not his gifted duchess.

"Be careful how you speak to me, lest your father's words become true," I growled, sending my small boy into a deep and surly sulk. I checked my anger and continued. "Expect to hear from Barron Paulis in the next day or two." I left my son pouting at the injustice of his situation. He would learn, I hoped. He was a good student and applied himself well in the Tower. He was a good boy. Baron Paulis would be able to teach him how to rule.

Lyca complaints that Makala's life was easier than his proved false. I found my second born skulking along the walls high above the turtle pond one morning, while I sat with Carlotta, Leila and our children near to cool waters below. At least, that is who I surmised the short chubby silouete standing between two posted guards belonged to. He should have been in classes in the Tower. I left off trying to teach Carlotta's four year old Panfilo the rudimentaries of healing to find my son. The other two women would watch over my youngest children for the moment. This was my new life, watching over my children, disciplining them, running to their aid. It was more of a father's role than a mother's. I ought to have been a source of comfort and tenderness for my boys, rather than the one forcing them to look to their future. I did the best I could. 

"Why aren't you at the Tower, Makala?" My son did not answer. He was a willful, petulant boy, raised mostly by his father's pride and Master Alyosus's dotage. As the Duke of Escasaine, no one dared discipline my son when my husband would not. What little control I had over his character as his teacher at Deyalorn's school, I lost when he passed out of my hands three years ago. I had let too much slip while I was in Deyalorn; instead of teaching him the art that was his birthright, I had stayed with my five and six year old students, and left his education to my old teacher, and his father, who he now adored. I would not find discipline and respect in this boy. "Very well, your grace," I said after a long sullen silence from Makala, "we can speak of this civilly, or I can ask a pair of guards to escort you to Master Leon's office to face punishment for truancy."

My boy stood with his back to me, short, pudgy and defiant. I counted twenty heartbeats, then turned to accquire his escorts. My movement drew a responce. "Master Leon will not punish me." Makala sounded despondent.

I stopped walking. "Why not?"

Again, my son stood silent and surly, staring at the masonry. Something had gone amiss at the Tower. He could not admit it to me, or in the hearing of the guards. I had little sympathy. Makala had chosen to sulk in such an unsightly manner in this public place. He would to learn to admit to his failings with dignity here, and control his displays of emotion better in the future. I cleared my throat impatiently, appalled at how poorly my son carried himself when troubled. I had trusted Master Alyosus with my son's upbringing.  The old man had raised three boys of his own, he was the only friend I had in Deyalorn's Tower. I had clearly misjudged my dignified tutor's abilities. I wished I could have sent Makala to Lir with Timmon. He woud have made a man out of Makala. "Master Leon sent me away." Makala finally admitted, hurling a stone down one hundred feet to where his siblings and their friends played near the turtle pond below. I heard cries of alarm from several women, but no sounds of crying children. Whoever he had aimed for in his fury at his teacher, Makala had missed. This was unacceptable, it reeked of his father's callousness. "He said I should not have been enrolled this year at all. I am not gifted enough to study."

Had I not been so dismayed by his temper, I would have pitied my son. Makala enjoyed his life in the Tower, surrounded by books and admiring classmates. My husband had built for him dreams of greatness all predicated on Makala ruling with the strength of the Tower behind him. If he had already learned all that his gift would allow, then he would have to rule as a gifted fighter, a class of men my husband endured but did not respect. I did not envy my son his position, but I did not have the luxury to coddle him as a mother should. If I did, who would teach him how to overcome defeat, and rule as a noble duke? "It does not become the Duke of Escasaine to meet failure by skulking on his father's walls." 

Makala turned his indignant fury on me. "The Duke of Escasaine does whatever he pleases. Go away." he screetched, his face contorting in anger, spittle frothing at his lips. "I am waiting for father to finish in court. He will punish Master Leon for this."
  
It was one thing to be treated this way by my husband. Another, completely, to meet this contempt from my son. He dishonored himself as well and I in this childish outburst. I had to find someone to train my son to be a better man and a duke. Until I could, I would have to take him in hand. There was a difference between power and the natural order. My husband did not know it. Makala would have to learn. "Master Leon has done nothing wrong. The Tower does not turn away students without good reason. Your father cannot make your gift stronger. Come away from the wall. We will speak more of this in private." 

"What would you know? You are a school teacher." my son spat, his face flushed and lips still frothing in anger. "My father will always protect me, myself and my heirs." 

I would have burned him where he stood, my blood or not, burned him with my talisman until he fell to the ground and begged for mercy for his tone and his words. I could not bear to see my flesh and blood, my sweet Makala, the child I had doted on as a child before Ophelia's death and my illness had forced me to turn my attentions to other things, stand before me as a corpulent, quivering coward, corrupted by my husband into a over proud puppet who would eat out of his hand, and let him rule Escasaine in his name, as Griswold ruled Cortan in mine. He dishonored the kind optimisitic man he had been named after. In the privacy of his room, I would have hurt my son until he learned his lesson. The castle wall was too public. "This is not a conversation to be had in public." I said sternly. "Meet me outside the barracks in half an hour." 

My son came at the appointed time, escorted by the man I had asked to ensure his compliance. He was still surly. By the time he arrived, my anger had cooled somewhat. Makala was not a bad child, I told myself. It was not that he was normally cruel or overly proud. If it were easy to overcome failures, it would not be said that they forged men's characters. I was his mother. I had failed in raising him for the last three years. I would correct that now. My husband would not approve of my picking up arms again, but I could not allow him to turn each of my sons against me, as if I were nothing more than an ungifted nurse. Could I find the courage to fight my husband, Head Alerio had asked. I had been reduced to a wife and mother. Even the meekest animal, when backed into a tight enough corner fights back. Without my children, I had nothing left.

Makala's future was as a gifted fighter now. He would have to train. Soft and bookish as he was, his life would be hard if he went directly to the barracks. His charisma and his station may prevent his peers from picking on him, but it would do nothing to help him meet his training sergent's demands. I had not trained since I had carried Elena, perhaps this would be a good lesson for both of us.

I led Makala to a patch of cleared grass beyond the west wall. There were few houses on that side of the barracks, a small dip in the ground protected us from view of all but the top windows of the Tower. Few would have a good view of our activity. I wore the black leather jerkin and pants, designed in the manner of the Black Riders, I had worn during my last campaign to Escasaine. I could not move freely in this outfit anymore. Five pregnancies, and seven years of idleness had not been kind on my body. "Why are you dressed like that?" he asked churlishly.

"I think it is high time you learned to fight," I replied calmly, tossing him a sword. 

"Fight who?" he asked, looking around for soldier that might challenge him.

I drew a dirk in one hand and a throwing knife in another. "You?" he said, with a mixture of disdain and disbelief. "But you are a girl."

"I am glad to see they did not kick you out of school for being dim witted." I retorted, stretching my back under the too tight leather. I was losing my temper. Still, I spoke more kindly than whoever would train him in the barracks. 

"Father is right about you," my son spat. "At times you need to be reminded of your position." My anger grew, yet his rotund form did not move against me.

If they will not respect you, I reminded myself, let them fear you. That was the lesson this ground had taught me as a child. It filled me with grief that I would have to apply that lesson to my own son. Yet it did not grieve me half as much as the words that he had spoken. Despite all my efforts, he was his father's son. I could not let my husband win. "Let me even the odds for you, Makala." I took my talisman off my neck and tossed it on the ground. 

That proved to be enough provocation. My short chubby child raised his sword and ran towards me. We must have made a comic sight that day, I years out of practise, the young duke never properly trained. Makala swung wildly at me; I dodged or parried easily. It was never really a fight, I was older than him, more experienced, more skilled, and stronger. It was never meant to be a fight. I corrected my son's stance and posture as Makala and Timmon had once corrected mine. He had to be good enough not to humiliate himself fully when he went to train with the rest of the army. Twice I pushed him back and forced him to yeild without causing him serious injury. The third time, he pressed me into a corner until I cut him without thinking. It was not a bad cut, I would heal it completely before we returned to thee castle, but my heart tightened with guilt when I saw the blood on his arm. "That was good," I panted, as Makala leapt back in alarm. "Keep pressing next time. You have healers, your enemies do not."

My son dropped his weapon. "I yeild." 

I sighed, cleaned my bloodied knife against my arm, sheathed it, then bent to pick up his sword. "No," I said, handing it to him. 

Makala would not take the blade. "These are real weapons," he shrieked, "My arm is bleeding."

If there was any shame, I told myself, it was mine for not teaching him to fight earlier. I knew his father would fail him in this regard, he had been interested only in his son as a healer. I would have to be patient with Makala. I shook my head, and tried again. "Our enemies do not use wooden swords against us, so we do not train with them. You will be whole before you leave this field, I promise you. Take your weapon."

"No," Makala yelled, his fear clearly in his carriage. He turned his back on me and ran towards the Tower as fast as his pudgy legs would carry him. 

The sight of my son, the Duke of Escasaine, fleeing in terror from a woman who had not picked up a weapon for most of his life filled me with fear and shame in equal measure, driving me to anger. My son had to learn how to stand and fight. Someone had to make a man out of my boy. I threw the star I found in my hand. It stuck itself in his right leg.  Makala screamed and fell to the ground, clutching his thigh.

"Get up," I yelled.

"My leg," my son cried. "I can't stand."

I hated Makala for his cowardice. How would he rule, I thought, if he could not withstand a beating from an old woman? I reviled him as I despised myself. How did my husband, like a giant leech, sap courage from those he touched, until we sat quivering before our courts, great anemic cowards on a throne,  handing him the scepter so he could rule in our names? "Get up and fight," I yelled again. "Or the next one I throw will be poisoned." I no longer carried lethal poisons on my weapons, but I had poisoned three stars out of reverence to traditions associated to the uniform. 

"You are crazy," my son sobbed as he pulled the small blade from his leg. "This is why they call you the mad duchess. They should kill you and make my brother duke."

Makala found his strength and ran away from me, limping towards the Tower. I did not throw a poisoned star at him. I did not answer his angry words. I sheathed the blades in my hand. He would come back, I knew. My husband would disapprove of Makala's failure, he would meet a sergeant from the barracks who would force him to run harder than he had ever run before, and he would be back, having realized that I was his best option. I would teach Makala to respect me, and I would cure him of his cowardice. 

I picked up the weapons scattered across the ground. I cleaned them, oiled them, put them in the stars and knives in their proper place on my vest, Makala's sword I slung across my back. Then I started the journey back to the castle from the barracks under the blistering noon sun to face the consequences of my actions. I did not weep. My well of tears had dried for the moment. I had none to shed for my son, and fewer to spare for myself.

\vspace{.5cm}

The Duke's table, my table, hushed when I entered the great hall for dinner. All eyes turned to my high backed blue and white silk gown, some in shock, others with embarrassment. I gathered that my son had confessed to his healer about our comical display this morning. I swallowed my fear at the oncoming humiliation and walked to my spot beside the duke. I had expected to encounter something of this type when I donned armour this morning. 

The company seated themselves after I did, and the converstaion continued where it had broken off at my enterance. I heard the word miscast uttered in hushed tones across the table, spoken quietly, but within my hearing. Eyes glanced at my husband as the word was spoken, others, including Barron Paulis, avoided mine. Duke Griswold carried on his convesations as if nothing untoward occurred around him. In his opinion, nothing was. It had been like this in the weeks after Ophelia's birth, when I was well enough to be seen in public, the word mad sat on the lips of everyone I met, planted there by my husband's wishes. How was this any different? I could not be exiled for sparring with my son, when I had been known to fight with the army for nearly the last two decades. There was no other evidence of my acting against the natural order set forth by the triumverate. I could be shamed however, I could be shamed until my will to fight my husband on this matter shattered as it had in so many other instances. I had not let Griswold have his way. What could I expect but that my life would become more difficult.

I ate in silence, as had long been my habit at this table. No one wished to speak to the unpopular duchess. My husband had seen fit to seat families around us that favoured him and scorned me. My son, I noticed, sat on the opposite side of his father, also ate in silence, no longer sullen, but frightened and ashamed. Would that I could shower him with a mother's compassion. It hurt me to see him so miserable. I could not take his father's disrespect from him. He would neither respect nor fear me if I came to him as simply a mother and a woman. 

Course after course arrived. Conversation eventually changed from my behaviour to the matter of next year's campaign. Allepo wished to seize territories to the west of Turina, adding to their domains, forming a border with Escasaine. Duke Erfat and General Morel wished to continue expansion to the south and east, reducing the Nievian threat to Escasaine. I listened, but said little. I had given the scepter over entirely to my husband, decisions such as these were his alone.

"What is your opinion on the matter of next spring's campaign, duchess? South or west?"

I looked up to see General Galderan's hooded wrinkled eyes staring at me benevolently from across the table. This man had been an ally once, the last time I was in Cortan, when I had been naive enough to think that I could defeat my husband. Why did he trouble me now? I wanted nothing more now than for my sons to grow up to be strong honorable dukes. I did not want to interfere in any matters beyond this small scrap of terrain my husband had left to me. "I have no opinion on the matter, general, I trust your greater experience and wisdom to lead my armies."

"Will you march under my command, your grace? It has been many years since I have marched with a healer who can save the lives of my men with her talisman and her blade." Laughter circulated the table as men thought again of the spectacle I had made of myself this morning. I heard men whisper the name Sveva, a name granted to me after defending Andona, derisively. The blood rose to my face, but I held my tongue. This was part of my husband's game. His men would have their fun with me, and I would endure. If I fought for my honour here, there would be no chance that I would be allowed to fight for my sons' love and respect. I must let him have his way.

General Galderan raised his voice and put an end to the snickers and murmers. "I see nothing amusing in the idea of marching with the Maker's Daughter. I can think of no woman in the history of Marsea who has served our armies as well since Queen Sveva defeated the giant Oddo. " The table fell to an awkward silence. The company of the duke's men had not expected opposition from one of their own. My husband glowered at the general as he turned me. Old age may have wrinkled General Galderan's skin, and removed his hair, but it had done nothing to his courage and loyalty. "I would be honored your grace, if you would join me in this spring's campaign."

A tense silence smothered the table, making it difficult to find my voice. "Thank you," I managed over the angry gaze of my husband, and the unspoken derision of his friends. 

"I heard, Barron Tristina, of the villages your father lost in the Nievian raids," Barron Paulis shouted across the table to change the subject. "I hope my duke will do everything in his power to come to your father's aid." 

The convseration moved on. I finished my meal in silence, and fled to my rooms at the earliest opportunity. I pulled every curtain in my room and snuffed every candle, emmersing myself in protective darkness. It had been a long trying day. I would not march with General Galderan, I would not serve as the head of Cortan's school. I would do nothing to aggravate my husband, but try to help my sons. My poor loving Makala had been turned into a proud coward to suit my husband's needs. My husband did not want me to teach or practise or march. I would give him that. He did not wish me to help my son become a good man. He did not wish me to earn his love and respect again. I could not give him that. Griswold had taken everything else away from me, I could not give him my children. I knelt before my idol of the triumverate, fumbled in the darkness for a stick of incense, flint, a knife and tinder. A small organge dot of fragrant light appeared between myself and my gods. I laid it at their feet and prayed to the Maker to guide this mother's hands. 

\vspace{.5cm}

Sunlight streamed through the windows and painfully on my face when Leila mercilessly pulled back the curtains of my bed. "Will you honor Cortan with your presence today, your grace?"

"No," I snarled, burrying my face in my pillows to escape the light. I had lain awake for some hours in the warmth and darkness granted me by my curtains, safe,  in my little cocoon, from the sneering court and derisive laughter. "It's too bright. Go away."

Leila moved to draw the transluscent curtains to my windows, but not the heavy ones that would block the light. I heard her walk to my dressing table, pick up a comb and bring it to my bed. She seated herself on it, and started to undo braid and remove the pins I had neglected to see to last night. "You are not ill today, your grace."

I could not face the world. I could not face a life where my son was too much of a coward to fight with sharp weapons but had the temerity to insult his duchess. "I fear that I am. Please give my apologies to the court." I needed a day or two to recover my strength and my nerves after yesterday's battering.

Leila's skilled finger's worked at my scalp. "You should not have challenged your son yesterday, your grace. You knew it would lead to an altercation."

"It was a mistake." I admitted to my pillows. "I was angry. I thought I could teach him. I was wrong. Find someone to raise him to be strong, Leila. I cannot teach him strength."

I heard my companion sigh. "I will do my best, your grace. Will you march with General Galderan?"

"No, of course not." How could the woman be so dense?

Leila lost her patience. "How many more allies do you need, your grace, before you stand against your husband?"

Two, I wanted to say. One was in Deyalorn, the other in Lir. Both of their happinesses depended on their maintaining a distance from me. I was friendless in Cortan. I suddenly ached with loneliness. I sat up to face the barroness in anger. "I need one champion, barroness. One man who will stand before me as my sheild, as a sworn man ought to his duchess. My court is filled with my husband's cronies, and snivelling half men who will only stand by me when it serves them. I should have them all hanged for disloyalty, and let my husband do what he will with this land."

Barroness Paulis defended the insult to her family. "My husband, your grace, he has sworn himself to the crown to protect you against the duke. You may rely on us."

"Your husband," I yelled, "said nothing last night to protest the words the duke laid against me. He waited for the old general to speak. I should send your family back to Deyalorn disgraced for this insult, as my nephew to give me someone actually able and willing to protect me."

It only took a short moment for the barroness to remove the defiance from her eyes and replace it with a condescending compassion. She was a clever politician, he family's position was my doing, it behooved her to bear my insults with grace. "You are upset," she cooed, "your nerves have borne more than they can. You have nothing to fear. Your husband cannot hurt you. It is treason to harm a duke of the land. My husband will see to your safety."

I waited for the strokes of the comb in her hand to sooth my anger into a deep sense of defeat. "It is treason to harm a duke of Marsea, it is no crime to beat one's wife." I corrected sourly.

Leila tisked . "You know how the crown will side if it comes to that. That is a large part of why King Gustav sent my family to Cortan. We are ever your loyal servants." 

I shared none of Leila's certainty. My half brother Dario had done little for me when Master Alyosus had complained of his ordering guards to lay hands on me. Such matters were not for the crown, he had said. I had seen how readily Barron Paulis had jumped to my aid in the face of my husband's insults. My husband could not murder me to put his son on the throne, that was the only assurance I had for my safety. I held my tongue and let Leila dress me. Malia would have understood, but she was not here. I only had Leila by my side. If I sent her family away, they would no longer help me raise Lyca. It would do no good to fight with her. 

I did, however, protest when she tried to take me from my room. 

"You have to clean your womb, your grace. In your current state of illness, it has slipped your mind. It is my duty to see to these matters for you."

"No. He will only plant another in me. What will I do with that? How long until he discovers that I am destroying his seed?"

Barroness Leila sighed. "You are not making my task easy, your grace. How will you establish a new life for youself if you refuse to ally yourself with everyone who tries to stand with you?"

If the task of living with my husband were easy I would not need the Barroness's assistance. "I no longer wish for a new life for myself. I only wish for a bit of peace in which I can raise my sons into dukes Marsea will respect." 

"Very well, then, your grace. Clean your womb. If you carry this child to term, you will be too ill to do even that. We will deal with the issue of a second pregnancy when the time comes."

There was logic to her words. I followed obediently to the Tower.

\vspace{.5cm}

Makala entered my chamber sheepishly that afternoon. I sat, resting, by my window, examining a picture in the book Malia and I had prepared on fetal development. There seemed to be a mistake in the copy that Cortan's Tower owned. Whether the error was in this transcription, or an error in the original work that I had missed, I could not tell. These matters were no longer my concern. Still, it bothered me.

"May I enter, mother?"

"Of course, Makala." I put away the book. 

My son hesitated as he crossed the threshold then seemed to gain his confidence. His manner had certainly changed from yesterday. My proud sullen child walked and spoke with the dignity I had hoped Master Alyosus would have instilled in him. "I wanted to apologize, mother, for what I said yesterday. I do not wish you harm."

I looked at my son, proud, but curious at the same time. My husband had not put him up to this. What had transpired between the duke and his son to engender this change in manner. "Thank you. I do not hold it against you. How are your wounds?"

Makala winced at the memory, rather than at any residual pain. "Completely healed, thank you. I wish to speak to you, mother, if you have the time."

A ten year old should not come to his mother in such a formal manner. My illness had taken me too far from my children. This was not right. I offered Makala a chair and poured him a small cup of wine. He wished to speak to me in the manner of the court. He had acted nobly in his apology. He deserved to be met on the grounds he wished for. "What is troubling you?"

Makala took a sip of his undiluted drink, struggled to hide his reaction, then lost his dignified resolve. "Father is displeased with me," he muttered to the table. "He told me to work with Captain Farone until I was ready to train with the men in the barracks. I met with him for two hours this morning." Makala raised two pleading eyes to me. "I would like to go back to training with you, if I could."

I shook my head. It was not that I did not feel sorry for my boy. Captain Teadoro Farone, second son of Barron Farone was a large strong man who worked with Cortan's cavalry. He was over six feet tall if he was an inch, built like and ox and famed for his strength. My husband had meant to punish our boy for the limits of his gift more than train him. My son had to learn to overcome his fears and obstacles. I, merely mortal, would not stay with him forever. His father, he had to learn to keep at bay. Learning to overcome Captain Farone's brutal strength may do him some good. "You may train with both of us."

"But..."

"You are not at the Tower any more." I reminded Makala gently. "Four hours of training will not be an unreasonable demand on your time. You will join the ranks of the military faster for it."

Makala resisted my logic. "He frightens me, mama."

"More than my poisons do?" The jape missed its mark. Perhaps Griswold was right. I was a poor mother. My son had come to me for a mother's comfort, and all I offered him was a father's discipline. Makala stared at his knees. "Four hours then," I declaired, settling the matter. "I will meet you every morning, when Sergent Farone is engaged with his cavalry. You will meet with him at his convenience."

The matter was far from satisfactory in my son's mind. Makala shifted in his seat, then drained his wine in search of he courage. "Would you be angry with me," he asked at last, "if I said I did not wish to be a gifted fighter?"

Oh, my poor sweet Makala. You wish to be a duke from the Tower. You have always been a curious boy, better suited to study, not swordplay. If only it were in my power to grant you your dreams. "Angry? Why would I be angry?" I laughed. "We are not all cut out for the roles we have to play. I will not be angry with you in the slightest. Unfortunately, I will entertain your wish even less. You are a duke of a border duchy. In a few years, Escasaine's lands, and Folgino's new conquests will make Cortan into an interior one. We will have neither the men nor the resources to fight for your glory. Your southern border is with Lir. The crown does not want a war there. You will eventually have to compete with other border duchies for new lands. You must march with your armies, either as a healer or a warrior."

My son, who had started out encouraged by my speech, now looked thoroughly disheartened. "I prefer books to fighting," he mumbled.

"There is no conflict there," I said, offering him what solace I had to give. "Being apart from the Tower does not prohibit you from persuing other studies. You may have as many tutors as you please, provided you do not neglect your physical training."

"Thank you," he said, but this assurance was not sufficient to comfort my son. He lingered silently in his seat, his eyes flitting restlessly about my room.

"What else troubles you, child?"

Makala's soft round face turned to me with shining eyes. His lower lip quivered. "Father was very angry last night. He said he favours Lyca over me now."

What would I have given at that moment to no longer be the Duchess of Cortan. Would I have killed my husband if I could? My ten year old son sat before me playing politics against his eight year old brother. His father had threatened to deny him the lands that were his birthright. The duchy of Escasaine had been meant to be governed by the second son of Cortan since Duke Lukos lived. Duke Ergino had argued long and hard with the crown not to have the lands of the Pensid mountains given to Gissal or Allepo when he found himself without heirs. That those lands belonged to Makala had long been decided long before he was born. My husband had no right to take them away from him. My husband had no right to sew the seeds of corruption and rivalry between my boys. "I will not pit myself for one son, and against another, Makala. Not for your sake, nor your father's. You and Lucretious are both my sons."

Makala looked at me in shock and despair. "You won't help me?" he nearly sobbed.

"That is not what I said,"  I started, but my son's anger cut me off.

"I knew I should not have listened to Head Alerio. The old fool said you would make a valuable ally. How can you be an ally against Father. You are so weak and afraid. I wish I had a proper mother who would listen to me instead of piercing me with knives."

Makala stormed out of my room as I returned to my window. How could I protect my children from their father indeed. I could barely protect myself. Who did I have to build a life around in Cortan? My sons did not respect me. Elena, six and gifted, had so far escaped my husband's notice by virtue of her gender. How long until he turned on her as he turned on me. Emile, soon to be five, would either manifest the gift or be sent by my husband to page at one of his barron's estates. Either way, he too would soon learn to scorn me. Mirelle and Simone would soon follow in the same lines. Where did that leave me?
If I was to be a mother before all else in Cortan, without my children, I was less than an ungifted woman.

I had no allies to help me, only a small group of leeches and half sheilds. I needed the Barron and Barroness Paulis, but I had seen the extent of his support. General Galderan had come to my aid gallantly, as a man of my army should, but he was an old man, how many more years would he remain in the army. How much longer could he support me against my husband? Master Alerio, likewise, seemed to offer me good position and a power base in the Tower. The Tower and the military were two domains that my husband did not wish me to touch. If I went to either General Galderan or Head Alerio, what would my husband do to me when I came home? Then there was Carlotta, who stayed close in the hopes of advancing her children, who I did not send away only to keep from becoming close to my husband. Where were the men who would swear themselves to my sevice completely and entirely, to protect me from all my enemies? They had all sworn themselves to my husband, I supposed, when I had handed him my scepter. He had made himself into my protector. What champion beyond a husband did a wife need?

The depths to which I had sunk astounded me. I had once sat in this room, gazing at the Tower, dreaming of heading it once. I now sat at the same window, cowering from my court. What had become of that girl? That child had not wanted a large family, but she had wanted to give children everything her dead parents could not give her. Had my husband stripped me even of that dream? My son had come to me for help, I had turned him away empty handed. My children I sent away to be cared for by friends and allies, myself playing minor roles in their lives. Would I leave my issue to be raised by strangers as I had been raised, in the hopes of sheilding them from their father? I could not be disloyal to even that childhood dream, when everything else had been ceded to my husband's will. I had nothing but half measures to protect myself, but at the very least, I could use those half measures to protect my sons from my husband's plots. Otherwise, what use was I as a mother to them.

I rose from the window, to walk to the barracks. I owed General Galderan a word of thanks. After that, I supposed I owed Head Alerio a visit.

\vspace{.5cm}

My first born came home a few days later. My husband rode out to meet him on the road, forbidding me to accompany him. So I waited, a patient mother, in the East tower, searching for a hint of Cortan's blue and white banner, the sight of dust on the horizon, kicked up by the hooves of my husband's men and my son escorts. My children waited with me each excited for what the day may bring. Mirella and Simone simply excited by the air of jubilation that adorned Cortan's castle. Elena and Emile excited at the prospect of a new sibling, though neither were old enough to remember living with him. Makala and Lyca were the only two of my sons who knew Woldino, their expectations as different as night and day. Makala, no longer seeking my help as an ally, had returned to his surly manner. I had spoken to him, during our morning's trainings together, of his brother's return, how a strong friendship between siblings would lead to strong alliances between the duchies they would rule as adults. He was fearful, I knew, of another sibling to vie for his father's attention that he had so recently lost. My arguments that his brother posed no threat to him fell as effectively against his misgivings as clods of earth would against Cortan's walls. Lyca, on the other hand, remembered Woldino fondly as an affectionate older brother. During Makala's first two years at Deyalorn's Towers, my first born had turned to Lyca to replace his twin's companionship. Lyca, tempted out of his usual reserve by the auspicious day, spoke voluminously of his outings with his brother along Deyalorn's cliffs, or their trips down to the river bank. It seemed to me that Lyca hoped that his eldest brother's presence would protect him from Makala's bullying, which had grown more intense during the afternoons, Lyca spent in the castle as the Paulis page than it had even been in Cortan's Tower. I watched my boys and tried not to despair. These would be the bonds that would write Cortan and Escasaine's futures. They squabbled and bore grudges and hated each other, sometimes with a greater passion than many of my barrons did. My children were not as close as Makala's siblings had been. I despaired of the task ahead of me. How would I raise these future dukes against my husband's corrupting influences? 

The guard beside me saw the dust on the road before I did, then the band of small figures on the distant horizon, and called the greeting. I hurried down to the outer yard to greet my son after four long years, with my troupe of children and nurses in tow. He had been six when I had sent him away. Would he still remember me? Would I recognize him? What had Timmon taught him? What did he look like? These and a thousand other questions flooded my mind as I waited on the yellow dusty yard, with a feeling in my stomach not very different from that of a wife or a sister waiting for her husband or brother to return from a long campaign. I wanted to know that he was safely by my side again. I wanted to know who he had become. 

When I saw him, in the end, I was breathtaken. He dismounted nimbly from his horse, tall and dark, a beautiful broad shouldered boy who wore his father's face. He bowed politely before me, and addressed me as his duchess. When I embraced him to welcome him home, he returned my warmth with equal ease and addressed me as his mother. I felt as if I had never heard that word spoken with more grace and warmth than I did that day from Woldino's, young Griswold's, lips. He carried himself as if ready to become a man, knowing the responcibilities to this castle that he would one day bear. I could hardly call that proud figure by the childish nickname I had given him as a babe. He knew the names of each of his siblings, and greeted them warmly by name, as well as the names of many of the members of court there to welcome him home. If he was nervous, he did not show it. This had been a well rehearsed encounter. He performed well in the yard that afternoon, he was a credit to Cortan's line, and Timmon's ability to raise children. 

I felt undeniably proud of the newercomer to the ranks of my children, my first born, the son who would inherit these lands that were mine. I looked forward to spending time with him in the days and weeks to come, getting to know him, helping him adjust to his new home, learning from him news of the dear friends I had not seen or heard from in as long as I had not seen this child. For the moment, however, I bowed to the wisdom tha he be allowed time with his siblings and peers on his first day home.

Cesara came running to me the next morning by the turtle pond where I sat in my usual place with Carlotta, Leila, our six youngest children and their nurses. "The dukes," she panted, pointing to the winter garden, "come quick." I dropped Mirella's doll and ran. 

I did not know what I expected to find, perhaps one of my sons lying with a broken leg, having fallen out of a tree, or bleeding as the result of a game taken too far. I had expected to find one of both of my boys in need of healing, and perhaps a reprimand before being sent on their ways. I had not expected to find them at each other's throats. My two boys fought on the packed earth paths between the freshly dug up flower beds of the garden, their shirts torn, their noses bloodied, knuckles scratched and knees skinned. This was not simply a wrestling match that had grown out of control. I saw a small glint of light in Makala's hand, he had drawn a knife on his brother. Young Griswold, always the the larger and stronger of the pair, and now also the better trained, had his soft round brother pinned to the ground, and was trying to pry the blade from his hand. 

"Griswold, Makala, stop this, both of you, I yelled from the gateway to the garden." Both boys, locked in their own struggles, paid me no heed. I stepped briskly to within five feet of both of them, and unleashed the Preserver's fire upon my first born.  "I told you both to yield," I demanded. 

My talisman's fire had barely touched young Griswold when he understood the warning and yeilded. Makala siezed this opportunity to gain an advantage on his stronger brother, rolled him off of himself, pinned him in turn to the hard ground and stabbed his arm with the knife. "Take it back, you cow mounting Liri born bastard." 

Young Griswold cried out in surprise from the unexpected attack, then struggled to regain the upper hand. Two guards had come down from the walls by this point to come to the young dukes' aid. I easily lifted my struggling soft son from atop his furious brother and let the men ensure that Duke Griswold did not attack again when he got to his feet. "Cowards attack men who have yeilded. Do I know my son to be a coward?" I asked my son's kicking and flailing form.

"He insulted you!" Makala yelled, digging his heal hard into my shin.

I threw my son face down onto a freshly dug flower bed, away from his fuming sibling. "Blades are drawn only on the training grounds or in battle. Is this any way to treat your brother?"

Makala spat dirt and blood from his mouth. "I should have known you would take his side. That is the last time I ever defend you!" He picked himself up ad ran. 

I took a deep breath an turned to my remaining son, who had not calmed himself at all during the intervening exchange. I sent the guards to see that the Duke of Escasaine cleaned and healed himself before presenting himself more civilly before his brother again. 

"He hit you!" young Griswold exploded, when I turn my gaze on him. "He is craven for hitting a woman."

I blinked for a moment in surprise. It seemed that I heard Timmon's chivalrous outrage escape from my son's lips. I forced myself not to dwell on my two sons' disparate attitudes towards their mother, and turned to the matter at hand. "Mind your words, duke Griswold. Are these the manners you bring into my house?" When the boy bit his lip, I added. "You may have noticed that I am not unarmed."

The young duke turned his eyes to the ground, his proud shoulders slouching in shame. "I am sorry, your grace."

"Mother will suffice," I corrected curtly. "What happened?"

"I meant no offence, your.. mother," young Griswold continued. "I did not mean to insult you." The child paused. I waited. "I said something about all these lands being mine one day, when my brother attacked me, calling me greedy and ill bred. I had made the comment in admiration. These lands and castles are yours, I wish you a long and healthy..."

"Enough," I cut him off, I had no need to hear my son's flattery. Makala still smarted from the threat his father had hung over his head that Lyca may inherit Escasaine. It would not do to explain that to young Griswold. I turned to Cesara, who had stood, now with Leila behind her, gawking, large eyed at this spectacle. "Is that what occurred?" The girl gave a frightened nod, staring at the growing red stain on Griswold's sleeve.

I sent Barroness Paulis and the Master's daughter away so I could punish my son in private. "You will apologize to your brother such that he will accept it. I will not have any more fighting between my children. If I catch you at it again, you will clean the stables for a week."

"Yes mother," Griswold said obediently, then paused. "Only I have apologized to him already, but he would not accept it unless I admitted that Ambassador Romino taught me to have weak morals."

Those sounded like their father's words. Having lost the Tower, Makala would do anything to please his father. I had to find a way to save my boys from the elder Griswold's corrupting influence. I raised my first born's chin until I met his gaze. "You are the future Duke of Cortan. Makala is the Duke of Escasaine. The ambassador has shown you a map of Marsea?" My son nodded. "Then you have seen the long border you will share. If you cannot live in harmony as brothers how will you keep peace at your border?" My first born gulped and nodded. "Find a means of apologizing that suits both of you, then tell me the terms of your treaty."

"Yes, mother." 

"Now let me see your arm." Young Griswold looked at me in surprise. I smiled and softened my tone. "You are bleeding, son, let me see your arm." My son offered the wounded appendage. "Do you need a healer's table or can we do this here?" I asked.

"Here is fine," my son responded taughtly. 

He had done this before, and he would not admit to his dislike of the painful proceess. "Good lad," I said, and examined his wound. It was deep, but clean, the blade Makala used had been small. It would not take long to heal. "Ready?" My son nodded. I contemplated him as I worked. It was not right, I knew, to feel a love for this boy's polished polite manner, when I felt so much shame and pity for Makala, and worried for Lyca. Yet Griswold was what I had hoped alll my sons would be, an honest, brave, decent boy who could own to his faults and treated me with respect. It was easy to love him. my favouritism would only sew further discord between the already turbulent relations between Cortan's children. I was a mother. It was my job to love and comfort all of my children. Yet I found myself more often disciplining my older issue than drying their tears. If I did not, my husband would rule their duchies in their name. 

"Mother?" Griswold asked when I had finished with his arm. "Thank you. That was easier to take than Mistress Romino's work." 

I smiled at this impeccably polite boy. "I am younger than she is, and have always been a stronger healer. Get yourself cleaned up and speak to your brother." It would be impossible not to love him more.

\vspace{.5cm}

General Galderan found me cleaning my knives in the shaddow of the armory after training with Makala. My son had survived over an hour with me today. Slowly but surely, he was getting stronger. I sat atop a barrel beside a wide wooden bench, arrayed with a variety of vials, weapons and rages. I liked the spot, it offered me shade and a good view of Cortan's armies training under the ferocious the early morning sun. As the armorers were mostly busy seeing to the needs of the men on the field, it also offered me privacy.

"Good morning, your grace. I just heard of your upcoming promotion. I congradulate you."

I carefully scraped a fleck of dust from the ivory and black hilt of my first husband's knife, put it down, then wiped the oil from my hands before offering one to the old General. "Thank you, general." I beamed. "I look forward to donning a healer's whites again. It has been too long." By this time tomorrow, I would be entitled to wear a Master's collar on my healer's robe.

"I take it there will not be a celebration at the castle or Tower to mark this great achievement, your grace? A Master at twenty five is no small feat." He sat, as I indicated, on the bench where my knifes and stars lay arrayed for cleaning and storage. 

"Times have changed, general. After Allepo's blight, Marsea's Towers need masters. As for my age, it is by the Maker's pleasure that I have the gift that I do. I only serve his purposes." My humility to the gods was the only appropriate response for a healer. I could not politely express my pleasure and pride in this accomplishment. I would suffer for this tonight, I knew. Leila had convinced me that my husband would not kill me. Furthermmore, when I awoke the next morning, it would be as a master of Cortan, not simply the school teacher and mother that Makala scorned me for. I would be something that Lyca could be proud of. I was terrified of this afternoon's ceremony, terrified and excited. I had not had something to be proud of since my second book was accepted in Deyalorn's library. 

The general saw through my polite humilty. "I am glad that the Maker's Daughter has returned to Cortan to serve our Tower, duchess. I would be more pleased if you would march with us this spring. As you say, Marsea is hurting for healers."

"I cannot, general," carefully arranging the three poisoned stars for cleaning and storage. "The head of Cortan's school steps down this spring. I cannot both march and be ready to take over his position when classes start again in summer."

The general looked disappointed. "Your duty to your kind comes first, I understand your grace." He fidgeted with Makala's knife in his hands, tossing the heavy blade from hand to the other. 

"You misjudge me," I protested. "Marsea's military has always been good to me, I owe it almost as much as I owe the White Tower. Your support, for instance, I cannot thank you enough."

"Yes," he gave a long troubled sigh. "You are welcome to it." General Galderan played with the weight and balance of Makala's knife aimlessly, his mouth moving silently as he counted the nine stars that lay before him. "Does your grace for Duke Makala to serve with the Black Riders?"

I laughed out loud at the idea. That son would never rise so far in the army. "No. Simply to be brave. I fight this way because it is the only way I know. When Makala overpowers me with his sword, I will stop teaching him."

"A unique woman." the old general muttered, then put down his blade in his hand. "May the Maker bless your motherhood."

There was something in the tone of the blessing that gave me pause. The general wanted something from me, but he would not say what it was. His nervousness started to rub off on me. "Not many will wish that upon me any more, general. My husband has convinced a few priests that I may have been miscast, that my ability to fight is evidence of that." I laughed nervously. "He wants me to stop training my son. I do not think I could continue without your support."

"You have not heard, then, your grace?" the general asked with a sudden sadness in his eyes I had failed to notice earlier.

"Heard, general?" I asked, frightened. A discussion of miscasts is not when one wishes to hear surprising news. "I have heard nothing."

"Your husband has turned the clergy against me as well, your grace. I am afraid I will not be of service to you much longer." He made a fist and brought it down on his knee. "They have pointed their accusing fingers on my son."

In the Maker's merciful name, I thought. What a terrible punishment for reminding the court that I was once called the Maker's Daughter. This was a message for me. My husband meant to frighten me. He meant to show me that he could take away my allies as fast as I found new friends. He meant to weigh me down with my feeling of responcibility towards them. He was succeeding. "I am so sorry, general. What will you do?"

"Do, your grace?" He laughed bitterly, but there were tears in his eyes. "I will retire and give my position to a man with a less shameful family. Then I will mourn my son." 

I sat with the General in silence, feeling guilty for his pain. Unlike in Deyalorn, where I had conciously entered a conflict with my husband, my very existence here attracted men to my cause. As inadequate as I felt their aid may be to the real danger in my life, I could not sit idle and watch my husband exile their sons. I wished there were something I could do for the general. My husband had accused his son of a crime I could not offer him pardon from. The triumverate, having given us the ability to heal ourselves, demanded that we cast away those unfit to serve, the idiots, the irreperably maimed or disfigured, and those that went so far against the natural order that their service corrupted the will of the gods, men who lay with men or beasts, ungifted women who spoke too loudly against their husbands. Our gods did not suffer the disobedient and the weak. 

"I fear I may have sealed my son's death by my military success." General Galderan continued sadly. "With the fall of Escasaine, the Hundred Horsemen disbanded. I will not even have the comfort of knowing Nunzio to be alive as a mercenary." He laughed at the irony of the situation. Commander Griso had once sought out miscasts exiled into Niev. He had served with them, led them even, without compunction. 

"Where is your son now?" There had to be something I could do.

"I do not know, your grace. I only heard the news at dawn. Nunzio fled this morning while it was still dark." 

My husband, the master of terror. What better way of frightening me than greeting me tonight with the news that I had no allies, to snatch away the moment of my greatest pride. My husband would make me pay for my accepting the master's collar. It was not right that I drag others unwillingly into my domestic war.  "Buzen," I said suddenly, quickly picking up my arrayed instruments on the bench. "The young duke Griswold's Liri escort only left the castle yesterday. Get word to your son, if you can, to travel with them." 

"Lir, your grace?" the old man asked with genuine confusion. "Our settlement treaty with Lir does not allow for Marseans to become Liri citizens." 

"I am aware of the terms of the treaty I asked my brother to sign, general," I responded, letting my anger flare more than was necessary. My husband had frightened me. It made me impatient. "You came for my help. A letter to the Liri ambassador is what I have to offer. Do not ask questions. Do whatever else you see fit to help him."

I left the general abruptly, to return, deeply shaken, to the castle. Allies, I needed allies, that was undeniable. My husband would endanger them as fast as I could form them. If Duke Griswold got word of this conspiracy between General Galderan and myself, how much worse could he make my life. Did I really have the courage to protect the men who came to my aid? My husband did not leave me with much other choice. If he demanded that I sacrifice my sons to his ambitions, what life did he leave me if he did not even leave me the life of a mother? I had no choice that I saw but to make allies, and protect them as I could.

I prayed that Leila was correct in her assessment that my husband could not kill me, that I would live to see a morning where my sons would know me as a master.

\vspace{.5cm}

The long anticipated hour came at last. My heart beat somewhere in my throat, and my stomach turned itself inside out as I stood in the that Head's office before Head Alerio, Head Fabia of the Women's Tower, Master Leon and Master Eulalia of the school. Master Otho formed the fifth member of the committee. I answered questions about my books calmly, with certainty. I spoke with pride and assurance about my role in helping Malia become a scribe. I stated my goals for Cortan's school. These were not the questions that made me lightheaded and nervous. It was the white suede bag that lay demurely on the table before Head Alerio's spotted folded hands. The bag contained my master's collar. 

Unlike the process for becoming a healer in Deyalorn's Tower, this meeting of Masters had no unspoken animosity lurking behind me. Cortan needed Masters. This was my home. This questioning was half a formality, and half a chance for the men and women before me to get a private audience with me, when I had no other duties to attend to. I eyed the white bag, and answered questions. To my surprise and relief, the question of mushrooms did not come up. I had expected it to. I was not so naive as to think that Cortan's Tower was not interested in the black powder anymore. As question after question came about teaching, or healing, or working with Marsea's military, I relaxed. There was no danger for me within the Tower's walls. I started to believe in Head Alerio's promise of safety within these white walls. The memory of General Galderan's visit from the morning faded from my mind, the fears of what my husband would do to me stopped tugging at my mind. Even the white suede bag stopped drawing my eye, and I lost myself in the frank and challenging conversation that was so characteristic of the Tower, that I had been cut off from in Deyalorn. 

The sun hung low in the west when my five friendly inquisitors led me to the Tower's great hall to be collared before the assembled healers. Leila had suggested that I keep news of the ceremony quiet for as long as possible. The Duke could do little to prevent the Tower from promoting me to a Master, but his power to intimidating me from attending was great. No one beside the five questioners, Leila and myself knew of this affair until late la st night. As a result, the celebrations were modest, comprising entirely of healers in the Tower. Still, the wine flowed freely, and the company warm. I spent hours speaking to friends I had not seen for years. Stories of battles seen and peculiar injuries healed passed as freely from mouth to mouth in that high ceilinged white hall as did congradulations. 

The hall filled for dinner. Strangers, new to the Tower, and younger healers from classes behind me filed into the room, starting a new round of congradulations. Students came next. Most eyed me curiously. I was a curiosity, and for them, a potentially dangerous one. My rules would soon dictated their lives. Elena caught my eye, unabashed and waived, until an older girl scolded her for her frankness. Lyca looked at me, blushed then looked away, bu it seemed to me that he walked a bit taller after his embarrasment than he had before. The meal passed in a flury of questions about my books and my plans for the school that I barely had time to finish my courses before the servers came around with the next dish. I certainly did not have time to be awestruck by the fact that I was eating at the master's table, for the second time in my life, this time wearing the thin white collar. I was too busy making plans for the future. 

When the meal finally concluded, my jaw had tired from speaking. My mind buzzed with possible ideas to increase the participation of women in the spring's campaign to conterbalance the loss of healers, means of rearranging the class structures to allow older students to work with the younger and thus benefit from what I still thought of as my husband's teachng methods, and manners in which to keep youngsters far to much like my youthful self from climbing onto the Tower's domes and turrets. I felt as if the volumes of wine I had drunk that evening had not dulled my wits as much as it had lain open a world of possibilities in which I could play. Nothing was impossible from where I sat on the low dias, surrounded by Cortan's Masters and teachers. Escasaines and Allepo's recent successes not withstanding, Cortan was still second only to Deyalorn's Tower. Between the loss of teachers from the blight and my efforts as the head of the school, I would be lead the best school in the land. I was a Master of Cortan. The Destroyer could deprive me of all other lands and titles. I would still be happy.

After dinner, Master Leon and Head Alerio took me to what would be my new office. It was a small well lit room, with two windows, already furnished with a variety of furs, and rugs, a large desk, three rows of shelves for books, simple cabinet for other belongings, a moderate hearth, and a magnificent view of Cortan's gardens. A turret to my east entirely blocked my view of the castle. A tiny antechamber led into the room, the type that I spent countless hours fretting in, wondering if my visit to this particular teacher's office was for admonition or praise. It would do nicely for now. It had enough space for me to work, I could watch the next generation of Cortan's healers grow an play in the garden under my window, and completely ignore the fact that my ungifted life as a duchess and a wife awaited me beyond these walls. Conversation and wine continued in that room, as Carlotta and Head Fabia joined us in the office. I left when a student knocked timidly at my door to remind me that the castle gates would close in half an hour.

I arrived home alone, the brisk walk sobering my mind. I hesitated before entering my chamber, turning first to that of my children. I stood in the dark doorway, listening to the sighs and snores that filled the room. Five children, two nurses, some snoring, others sniffling. Someone large, Griswold or Makala rolled over in his sleep. All of my children were home now. Yesterday the thought still terrified me.  Today, it seemed right that they should all be near me on the first dawn of my new accomplishment. I had no more solutions than I had yesterday, I thought, fingering the bit of cloth that sat unfamiliarly on my healer's robe, just a new title, and possibly a few more allies. 

"Is something the matter, your grace?" Mirella's nurse asked, sitting up to peer at my outline in the darkness.

"Nothing at all," I whispered in return. "Just visiting." I turned my feet to my room. I would live to see another day, Leila had promised. Whatever my husband had planned for tonight, tomorrow would be a good day. 

I took a tallow from my maid, and dismissed her. I wore healer's robes. The simple white garment was as comfortable to me as was my own skin. I would not need her help tonight.

"You took your time coming home."

I cried out out and dropped my candle at the unexpected sound of my husband's voice. The flame guttered, leaving me in utter darkness. 

"How does a woman so clumsy think she can train a duke to fight?" My husband tisked. He was sitting on my bed, I realized. If I had a knife, I thought. "It was superfluous of me to call you miscast. You are simply mad. You have no concept of the limits of your abilities, do you?" I heard him stand. "Or the extent of mine." Any brave thoughts I had foolishly entertained fled at the sound of his footsteps walking towards me. It was not completely dark in the room, there was some light coming from the window, there was a moon overhead, it hung high in the sky, far from my window. How long had my husband sat in my room waiting? His eyes had adjusted, while mine were still blind. 

He paused when he  stood close enough that I could hear his breath. Griswold's hands reached for my neck, then his fingers reached for my collar. "How many times have I told you, I do not want you working with the Tower? You really do have dung between your ears." I felt a jerk and the sound of ripping cloth. "Will you never learn that I will always be here? That I will always win." A large hand closed around my neck, the thumb just below my larynx. He led me backwards by this grip until I hit a wall. "This really is getting tiresome, you know. Coming to you every night, hoping to find you pregnant." Another hand fumbled for my crotch, and a thumb dig itself deeply into my bladder. "It is time I took matters into my own hands."

Fear, like a ferral cat scratched and pummeled my ribs, desperate for a way out. My husband would kill me tonight. I did not know how, but that much seemed certain. "My liege," I whispered. "I will do anything you wish. Please, I beg you, tell me your pleasure."

"It is a little late for that, don't you think?" Griswold said, banging the back of my head against the wall, and digging his thumb deeper into my stomach. A spray of stars went across my field of vision. "Don't move, and you probably won't die. This will hurt some. Don't scream, and I won't have to cause you further pain."

"What are you doing?" I asked hoarsely. The threat to my person was clear, but I felt no blade. It seemed he was alone in my room. How he would kill me? Could I overpower him in the act? I still wore my talisman. I was not unarmed.

"Doing?" he laughed. "You stupid woman. I making certain that you give me another child."

I did not understand. I drew the Preserver's flame from the air around me into my talisman, but before I could burn my husband a searing pain errupted in my stomach, just beneath my husband's thumb. The world went white. I was vaguely aware of voiding my bladder and falling to the ground. 

\vspace{.5cm}

I awoke to the sensation that someone had carved out my inner organs with a blunt object, entering rudely from my birthing passage, and left me stuffed with rats hair. In addition my head ached, my mouth tasted of bile and rotten eggs. My arms and legs lay leaden beside me, when I opnened my eyes, I found myself in a painfully bright white room. 

A soft hand appeared on my forehead. "Good, you are awake, and  you don't have a fever." Carlotta's voice said by my ear. One soft arm lifted up my shoulders while the other placed pillows to support my back. "Drink this." She put a lage mug of sweet cold water to my lips. 

"What..." I began.

"You lost your child, your grace. Your maid found you yesterday in a pool of blood, urine and your own feces. You were miscarrying. By the time you arrived here, you had lost a lot of blood. It took all day to stop your bleeding. It was very unusual.  What happened? Why did you not come for help immediated when you realized you would miscarry?"

I lay back and closed my eyes. I could hear the sound of the army practising in the barracks outside. It was morning. I had lost a full day. What had happened two nights ago? My husband had said that he wanted to make certain that I would give him another child. How could he make certain? He had not entered me that I could remember. "My stomach hurts." I said. "A sharp pain radiating from what may be my uterus."

Carlotta looked at me quizzically. "That is impossible. You have no other injuries, other than a bruise on the back of your head where you must have hit it as you fell." Impossible or not, it was true. Carlotta's finger probed my stomach until it reached the area of my bladder, where I cried out sharply. "Give it time," she said with a touch of uncertainty. "It has been a bad miscarriage."

Master Madriano left my room to bring me food. My situation did not make sense. I waited for Carlotta and counted days. "I could not have miscarried," I told her when she returned with a plate of bread and broth. "I had my womb cleaned three weeks ago."

Carlotta looked at me with confusion for a moment, then it passed. "It is hard to believe, but it is true. I checked your womb when you arrived. There was a seed still planted in you. I cleaned your womb before I healed you."

"I did not miscarry," I nearly yelled.

Carlotta looked at me with patronizing eyes. "You have had a shock. You will recover in a few days time."

I sank back on my pillows and closed my eyes. Think, Carlotta, I wanted to yell, but I held my tongue, loathing her condesention. It was nearly impossible that I have a child in me again so soon after cleaning my womb. Even if I did, it would not rid itself from my body in such a violent manner. Something else had happened that night. My husband had met me in the cover of darkness. I had not seen what he had done. He must have used the darkness to his advantage somehow. Even I could tell that my thoughts sounded mad. At worst my husband had inexpertly tried to clean my womb. That would explain the bleeding. Why would he do such a thing? He had said that he wanted to make certain that I concieved. Why would he then clean my womb? How would he make certain? Certainty was impossible. Whatever my husband had done, he had been very clever. He had nearly killed me and bound my silence by the very improbability that he had a hand in my ill health.

The sound of my door opening interrupted my ponderings. "How is her grace today?" I heard Leila's voice aske Carlotta. 

Bitch, I thought. Snake, cur, traitor. She had promised that my husband could not kill me. She had promised that she and her husband would keep me safe. It was on the strength of their assurances that I had moved to take my master's collar. She and her husband deserved to share a damp cell below the castle until their eyes and skin went white from the eternal darkness.

"She is awake, barroness, and has eaten," Carlotta informed my guest, then in more hushed tones, "Still weak, and a bit disoriented, but she should recover."

Leila entered the room. "That is such a relief. Yesterday was a frightful day, your grace. Your children are all..."

"You!" I roared. "You said he would not try to kill me. You lying whore. Get out of my sight."

"Your grace!" Carlotta gasped in surprise. "She is not strong enough for visitors yet, I fear." she said, bending over my bed to fuss at my pillows and blanket. "You know that duke has nothing to do with your current condition. You miscarried badly. Why do you persist in this madness?"

Leila's face appeared over Carlotta's shoulder, peering at me in aprehension. Carlotta bent nearly double over my face and chest now, holding my face in her hands, forcing me to look at her. I could see the bulge of her talisman against her robe. "Listen to me, your grace. You must be rational. You are frightened, but..."

I grabbed the talisman that hung from Carlotta's neck and hurt Leila with all my strength. The small white recovery chamber filled with the sound of Leila's screaming and gasping as she fell to the floor clawing at her neck, trying to breath through the fire, and Carlotta's struggling to free herself from my grasp. I was not strong. I could not keep up my punishment for much longer. I felt Carlotta's hands around my own, trying to pry my fingers loose from her necklace. My world narrowed and seemed far away, as if I were looking down a dark tunnel. Carlotta succeeded in pulling two fingers off her talisman. I heard Leila scrabble backwards, moving out of my reach. I turned my fury on Carlotta. The world grew more distant and smaller. I heard the sound of a woman sobbing, then a warm darkness and a blessed silence covered me.

\vspace{.5cm}

It was still dark when I next became aware of the world. It was still warm, and it was still silent. I was still horizontal, but this time, I was bound to my bed. I heard a muffled stirring when I struggled against my bindings. "Who is there?" I yelled in alarm. I heard the sound of hurried footsteps. A thin strip of light appeared in my wall, and a figure slipped out through it. The door closed, leaving me again in complete darkness.

I relaxed and considered my options. I was still in the Tower. The world outside my door was white. I was on a narrow bed, not a healer's table. I still did not have my talisman, and my womb still hurt worryingly and inexplicably. If Carlotta had cleaned and healed it, there should be little residual pain after the first hour or so. There was something wrong with me. My husband had done something to me. I did not know what, and the world would think me mad for suggesting it. I turned my mind from thoughts of my husband, and focused on my own body. There was something wrong with me. However it had happened, I needed to fix it. To do that, I needed my talisman. I wondered if there was any chance of my getting it back after I had hurt Leila and Carlotta as I had.

I closed my eyes and shifted in the bed. This darkness was probably Leila's idea, as the only known balm for my madness. The healer who had left had probably been posted to watch until I recovered my senses again. Who had he gone for, I wondered, and how would I convince them that this was not an episode of my illness? My last meeting with my husband played itself over and over in my groggy head. I felt my husband's breath on my face, and the pressure of his thumb on my neck. His words rang clearly through my head "You stupid woman. I making certain that you give me another child," each syllable crisp and well pronounced. His hands fumbled along my leg, one finger curving aroung my pelvic bone. In the warm darkness, my throbbing pain was the only sensation that let me know I was still in this world. I imagined I felt his finger digging into my flesh at the exact spot where I now felt any sensation. Why had he put his hand there? The hand and the scene took on a surreal quality in my mind, raping me, it seemed, but through my flesh, and not by the conventional means. My husband was so large, I felt like a child before him again, as when we had first married. His hand pressed down on my stomach and my throat. Why had I not carried a knife? 

I startled when the door opened, letting in a thin strip of light and a person whose features I could not make out. I had drifted off, I realized, as the cobwebs of the nightmare I had fallen into disappeared from my mind. The door closed. "How are you feeling?" asked Head Alerio's voice.

"Tired, and in pain," I answered. It was true. "I would like to be unbound, light, and my talisman."

"Very well," I heard the old man move slowly around the room, towards the window. "This Tower has done you a disservice, I fear, though it was in no way intentional. We raised you according to the station we found you in, a poor orphan, with nothing to her name but an exceptional ability to weild the gift. Adjusting to life as a duchess must be difficult for you." Curtains flew open, flooding the brilliant white room with painful sunlight. I shut my eyes tight and turned my face away from the glare.

"I am not mad."

The old man walked to my bed and loosened the leather straps that pinned me to my  bed. "If I thought you were mad, master,  I would not have asked you to lead my school. You are angry, and perhaps a bit frightened."

"You promised me protection." I said as my hands came free. I opened my eyes to see a plate of fruit and a jug of water on the table next to me. My talisman sat on the far side of the food, just out of reach.

Head Alerio sighed. "I can promise your complete safety if you chose to reside in this Tower. I have little control of life in the castle." He unbound my legs and handed my the talisman. "Why are you still in pain?"

I checked my bladder, womb and surrounding tissue for injury. There was nothing. No bruising, no internal bleeding, no foreign objects. As far as I could tell, I was whole and fit. I tried healing myself, and winced as the pain shot through my soft tissue and into the surrounding bone. "I do not know. It hurts to touch it with the Preserver's fire."

Head Alerio nodded thoughtfully and lowered himself into a chair by my side. "Master Carlotta said that she had trouble stopping your bleeding, that she had to work very delicately. It almost appeared, that by healing you, she made your blood flow more freely. But only to your womb. The bruise on your head healed normally." He drummed his fingers on his knee. "What happened when you left us the other day?"

I told him everything I could recall of my meeting with my husband. "Why did he meet with you in darkness, I wonder?" Head Alerio mused when I finished.

"It was purely by accident. I had a candle when I entered."

"But he did nothing, when it guttered, to light another. Curious." The drumming on his knee continued. "Master Nisrita, your love of darkness does not stem, by any chance, from any experimentation with the mushrooms?"

"No!" I bolted upright at the question, wincing as I did. The world spun.

I found the Head's arms steadying me. "You must learn to control your emotions. Not only because you are weak now, but because the Duchess of Cortan should be better presented."

Everyone had an opinion on how I should rule. What I needed were suggestions for how I should continue to live. "My husband rules in my name. How I carry myself is of little consequence."

"Wrong." My old tutor corrected his surly student. "If we can prove that this was an attempt on your life, the crown will gladly remove your husband, leaving you to rule. But we were discussing your ailments." He handed me a mug of sweetened water and put a bunch of grapes in my lap. "Those who suffer from the nightmares caused by the old magic seek the quiet and the dark." I recalled the comfort of the dark room in Maya and Groto's house after my only attempt to weild the old magic, and the tom that had slunk under a cabinet after sinking his teeth into me. "Your husband has long tried to regain his skills by working with the old magic. Is it possible that he succeeded?"

The question terrified me. The idea of Griswold, long lived and a healer again. There would be no end to his power. "They all said I was mad to even contemplate such a possibility. The world has called me mad so many times, that even I began to believe it." I put down my drink, my hands shaking too violently to keep from spilling it. Whether I shook from fear or anger, I no longer knew the difference.

"It benefitted your husband to call you mad. It drew the attention off of him, and silenced your voice. You made his job easier by behaving as you have these last few years."

How dare he, an unmarried old man, accuse me of weakness after everything I have endured with my husband. "You have never had children, Head Alerio," I bit off. "You do not know what it is like to bury them." He had never lived with my husband. How could he know what it was to endure him night after night?

"You are correct," the old man answered, giving no ground. "All I know is that you will turn you allies into enemies is you attack them without reason. Barroness Paulis will still serve us, though less willingly after this morning's events."

"Us?" I snapped.

Head Alerio sighed. "You and I have our own reasons for disliking your husband. There is no one else who would believe that your husband hurt you intentionally. If we wish to rid Cortan of him, it would seem that we must work together again. Put aside your politicial misgivings for the moment. Is it possible that your husband was in the grip of the mushrooms when he hurt you?"

I thought about the night again, the clarity with which we conversed, the intentionality of his acts, the confidence with which he moved in the dark. "He was lucid. I am certain of it." I said, though there seemed to be no other possible explanation. 

"Was he lucid," the old man started slowly, "or did he learn to control his dreams?"

"That is impossible! Those dreams cannot be controlled."

"Master Nisrita, you are an intelligent girl. When you chose to study a subject, I trust no one else in Marsea to understand the depths and intricacies of the matter better. Do not disgrace yourself by making baseless claims about a subject you know nothing about." I shriveled at my tutor's chastisement. "I leave you now to rest. If you learn anything more about the reasons behind your pain, let me know." He rose magestically from his chair.

"What will you do?"

"For the moment, I will make certain that no more than four people know of your suspicions that you did not miscarry naturally."

\vspace{.5cm}

I spent the next three days in the infirmary, studying myself. I repeated Carlotta's experiment. My body, unwounded as it was, would not respond to healing anywhere but in one particular spot, where my body responded with a pain of its own, different from the pain of healing. I had never encountered anything like it. I had no infection that I could detect, suffered from no other illness, simply a mysterious pain that would not go away. I spent my days recovering my strength, testing my strange new ailment, and greeting a long line of well wishers, which included my husband. 

On the fourth day, as I had recovered my strength somewhat, I put aside my frustrating and painful exploration of my body, and spent the day with Master Leon, learning from him what I needed to know to take over the school in a few months time. It felt good to be working in the Tower again, walking the halls I had grown up in, watching children do what I had done at their age. I did not see much of my children during the days of my recovery. I had not expected them to, children generally are not admitted into the infirmary. Still, I found myself trying to catch a glimpse of either Elena or Lyca on my first day out of bed.

Lyca satisfied my mother's need after dinner that night. He found me at my desk going through the rolls of women who had once studied in Cortan, or had once marched with our armies. This year's campaign needed competent healers to march. So many had been lost in the blight, Cortan's army needed women to leave their families for a few weeks. I had many dozens of letters to write.

"Are you busy, mama?" sticking his small head around my open door.

"Lyca," I beamed. "For you, I have all the time in the world." My son came in and put an orange on my desk. "From the tree, Lyca. Really," I scolded. "I cannot condone..."

"From the castle's kitchen," my eight year old interrupted hastily. "You said I could?" I laughed as I picked up the fruit. I would lead this school soon. I would have to be careful what mischeif I put my children up to. "Griswold helps me not get caught," Lyca continued, putting an orange wedge in his mouth. 
"How are you liking your time at the castle?" I asked, sucking orange juice from my fingers. 

Lyca hesitated. "Barron Paulis said that you might send them away soon. Is that true?"

I had spoken to Leila Paulis once since I had attacked her, to accept her apology for failing me. She did not believe that my husband had a hand in my illness, but she did believe that Head Alerio thought so, and that he did not wish the matter to reach her husband's ears. "I haven't decided what I will do regarding them. You haven't answered my question."

Lyca gave the question the same weight as the important matter of chosing his next orange wedge. He picked a thin slice without seeds. "I don't like walking there in the hot sun, but I like Barron Paulis. He says I read and write better than Cornelio does, so I get to write the letters and Cornelio delivers them. Cornelio is ten," he confided proudly.

"And what do you think of the court?"

Lyca shurgged as he started making a boat out of a quill and two orange peels. "They treat me well enough, but I like the men at the Tower better." I sighed and watched my son play. It was not an ideal response, but better than the disdain he had shown for the ungifted life a month ago. Satisfied with his construction, my boy started carving a design onto his sail. "Grandmother says that you stay here because you like our kind better than you like the court."

"Our kind, Lyca? All of Cortan are our people. That is what being a duke means."

My eight year old looked uncertainly at his sail, still convinced that the gifted and ungifted were fundamentally different. "Father says you don't come home because you are frightened."

What had my husband and mother said in front of my children? "I'm not afraid, Lyca. I was still in bed yesterday," I lied. I was terrified of the idea of facing my husband again. 

"Will you come home tomorrow?"

"Possibly, if I feel strong enough. What does Barron Paulis have you do?" 

Lyca prattled happily for some time about his adventures as the barron's page. When he started to yawn, I sent him to bed. I stayed awake writing letters late into that night, wondering what Malia did in Deyalorn, and what she would say to me in my current predicament.

\vspace{.5cm}

"You cannot hide in the Tower forever, your grace. You must show your face in the castle," Leila urged from outside my door on my seventh morning in the Tower. I wretched the last of my breakfast into the chamberpot. "You are nervous, that is natural, but you have to go home. We agreed last night that it was time. The court expects you."

I walked to the narrow window of my narrow room, and placed my head against the cold stone. The fresh air calmed my nausea. "Did you send for Master Carlotta and Head Alerio?"

"I did. your grace," came the exasperated answer from beyond my door. "What you are saying is impossible. You are just putting off the inevitable."

Impossible, yes, but true. I had no idea what my husband had done to me, but I could not deny what I had learned this morning. Another wave of nasea shuddered through my body, and I spat a mouthful of bile out the window. I held my togue, hoping it would induce Leila to do the same. When my door emitted silence, I craddled my head in my arms, and gulped down the cool air. My stomach was determined to wring me dry this morning. Without the help of the wall, I would not have remained standing.

It did not take long before I heard a knock and Head Alerio's voice. Four people filled the small room in the infirmary I had occupied for the last week. Carlotta verified my outrageous claim as I explained to Head Alerio that I had decided, this morning, to prod at the mysterious ailment that had stopped hurting when I stopped trying to heal it. It hurt when I studied it, it hurt now as Carlotta dicretely surveyed my womb, and I was overcome with a powerful wave of nasea that had e confined to my room for the last hour. 

"This is impossible," Carlotta gasped hoarsely.

"It was impossible last time as well," I corrected. "But something has taken root in my womb."

Leila looked dumbly from Carlotta to myself. Head Alerio wore the inscrutable face of a practising healer. 

"I can clean your womb again, if you would like, but ..."

"But you will not heal her," came the quiet instruction from the Head.

"Cleaning a womb is a dangerous proceedure." Carlotta rose proudly to defend her position. "If she hemorrages, her death will not be on my head."

I looked at the Mother's Balm, defending her reputation for never having lost a patient, the miracle worker of the south. She had once accused me of being too close to my patients. I accussed her now of not seeing them as humans at all. Head Alerio looked past her and spoke to me. "There is precedent for healing causing lingering pain. It is rare, especially in one so young. Most people do not encounter even one such case in a lifetime, and you lack most of the symptoms."

"Consumption," I whispered. It was a strange disease, leading to a painful and lingering death. It was elusive, without a set of clear symptoms defining its stages. It was the only disease the Tower knew of that reacted to the Preserver's gift. We could kill a man by trying to save him. The Preserver only knows what damage Carlotta and I did to myself by healing me at all. I felt as if I sat in a bubble. The world beyond had nothing to do with me. 

Carlotta looked pale. Leila's voice rose sharply in disbelief. "That is impossible. The duchess has no symptoms. She is strong and healthy." 

The two healers in the room ignored her. "If the duke wished to kill her, why do it this way? It could take years." Carlotta asked.

Head Alerio remained the only unshaken person in the room. "Your mentor did not wish to kill her." Carlotta bristled as the old man continued. "As Master Nisrita has said repeatedly, his grace wished to make her pregnant. He misjudged his ability."

"The Duke!" shreiked Leila's hysterical voice. "How could he possibly do this to her grace? The gift does not inflict disease."

"How would you like to proceed?" Master Alerio's word's reached me like a pit falling through oil, leaving eddies of confusion in its wake.

I blinked. "Proceed? Yes." I paused, clawing at the layers of straw that muffled my thoughts. "Will I be able to continue weilding the gift?"

"Your grace, you are dying!" Carlotta sharply reminded me. Somewhere I heard the sound of Leila sobbing. Was it for me, I wondered, distantly, or for the fear that I would have her family sent away for failing to protect me?

"I cannot say." Head Alerio said calmly. "The Tower has never had the opportunity to study a member with the disease young enough to practise."

"I am at your service, then."

Head Alerio gave me a tight smile and a nod. 

"Your nausea," Master Carlotta reprimanded her stubborn patient.

A sobbing voice accused me of killing myself. Head Alerio gave Carlotta an impatient look. She led Leila out of the room.

After they had left, he said "You are still in danger. If you submit this as evidence that his grace is practising the old magic, you will never learn how to undo what he has done."

I found it easier to think in the less crowded room. "Undo this? Disease cannot be undone."

"You have never tried, master. Until today, I had never heard of disease being inflicted by the gift."

"You think the old magic can cure me of this?"

The old man shrugged. "It is a possibility. This is an unprecedented situation."

I shook the last strands of bewilderment from my mind. "Then you do not need my husband. There are plenty in the Tower who are experienced with the old magic. Cortan can cure me."

"We stand a better chance of killing you." Head Alerio showed the first emotion he had since this conversation began. There was anger in his voice, but something very different in his manner. What was it? Lust? "We do not yet know how to control the dreams the mushrooms bring. This is a delicate proceedure. Anything could happen." 

A picure of Cortan's head's dreams formed in my mind. It seemed I was to be a pawn in his hands again. The benefit of my compliance could be my life, or I may serve the Head's goals with no benefit of my own. I had no idea what grew inside me. It may not be consumption. I had the symptoms of an impossible pregnancy, and nothing more. I heard Carlotta enter the room quietly and stand by the door. "What do you propose?"

"Your husband, his grace, cannot control the old magic as well as he thinks. He will try to improve his techniques. We will watch him. He is a man of studious habits. He will keep a record of his attempts. When he succeeds, he will study his methods, and cure you. After that, we will have no need for your husband." 

With my husband out of the way, I thought, Head Alerio would lobby the crown to allow for the practise of the old magic, showing Griswold's work as evidence of the benefits to the nation. Cortan would be the first Tower to have mastered these new methods and used them to their advantage. "No," I said. "I am dead as soon as my husband masters the dreams. I am likely dead if the crown kills him before he learns more. I would rather die knowing that his body is actually in his grave."

"I agree with her grace, head." Carlotta said from the door. I wondered what she had to gain from my misfortune. "Your plan is too slow. You need someone to study alongside the duke. To learn his methods as he does, and cure the duchess as soon as it is possible." Head Alerio gave Carlotta an expectant look. She shook her head. "I cannot spend that much time in Cortan. My husband grows impatient with my absence. I must return to him before he marches."

That left one option. The fewer people knew of our plans, the safer I would be. Baroness Paulis could not be trusted to study my husband's work. She was never a good student. "I will do it." Head Alerio looked at me, surprised. "This is my ailment, head. I will cure myself."

"You have found your courage?" he asked. 

My courage. My husband had left me with little ground to stand on. He had taken away from me my art, turned my children from me, played with my very life. I am sure seasoned generals would have had wise words to admonish me on the follies of giving ground to readily. I only knew that even the most timid animal, when pressed into a tight enough corner will fight. "I have," I replied.

Head Alerio took his leave. Carlotta lingered. "Your grace," she began, gently, "this is brave of you, but not wise."

"A wife disobeying her husband seldom is."

"We were friends, once, your grace. Until you married, and we went our different paths. In the name of that friendship, let me do this for you." Carlotta sounded so sweet, I was so lonely, I wanted to believe her.

"The commander is impatient," I reminded her. I would not be taken by her pleas for our friendship again. I had been foolish once. To be fooled twice would reflect poorly on my intelligence.

"Help me reason with him, your grace." Two large suplicating eyes looked at me from above her plump cheeks. Carlotta had always been so effortlessly beautiful. Even now, she carried her extra weight with an air of magesty and confidence, instead of the shame with which I bore the softness of body motherhood had forced on me. "You do not know what you are about to turn your hand to. It takes time and practise to get used to the mushroom's madness. You will not be able to hide your nightmares. You are under too much scrutiny. The duke will find out what you are doing." She knelt before me, graceful and benevolent, a friend begging another's forgiveness. She would care for me again, as she had before, we could be each other's confidants."Believe whatever else you will about me, your grace, I do not want his grace to hurt you. Let me do this for you. I have the greater experience and skill in this matter. I could sheild you," she smiled impishly, "as I once did against Master Adele when you snuck into my dormitory."

The little viper knew the right words to say. Not just Cortan, but Carlotta would be the one to master the old magic first. The Mother's Balm would be the heroine to save the duchess' life. All for the price of a barrony. "I have no lands with which to buy the commander's patience."

Carlotta had the temerity to look hurt by my statement. "My husband is aware that lands cannot be had at a moment's notice. If you were to appoint me your personal healer, I am certain he would see the wisdom in waiting for the appropriate occaision to arise, trusting in your good wishes for our family."

She had it all planned in her head. Her family would rise, while I struggled to keep my life. There was no love in my court, only loyalty on a lease that must be constanly renewed. "Leave me." I ordered. "I am expected at the Duke's table tonight. Call for me in the afternoon so I may ready myself."

I drew the curtains when she left and blanketed myself in darkness. He had come to me in darkness, stalking me like a panther stalks its prey. He had planted in me a seed, as corrupt as his own ambitions, a disease that would kill me in the end. He had been so clever, he could not lose. If he had given me a child, my position in Cortan's Tower would have been greatly dimished. He may have frightened me into resigning. If I had died that night, before my maid found me, no one would know of his crime. Even living, I would have suspected nothing if not for the pregnancy I had destroyed. The one my husband did not know about. The fetus that ruined all his insidious plans. I brought my knees to my chest and lay in the comforting darkness. My husband had to be defeated. Otherwise, he would leave me no ground but that which covered my grave. Cortan was my home. I could not let him take me from it. 

\vspace{.5cm}

I ate little during the next few days, between the persisent nausea, my fears and my plans. Carlotta and I had made arrangements. She was a cunning woman. That much could not be denied. If only as a child I had paid more heed to her manners and learned from her. Perhaps then, I would have married an ambitious commander, and she been saddled with as many children as she could support. That was neither here nor there. I put the past aside for the moment and learned to use my personal healer's cunning to my advantage.

Finding the Duke's notes had been easy. Carlotta left her children with the barroness and myself by the turtle pond, then left, complaining of a mild headache. This was such a commonplace scene. The three women met every morning with six or seven children running about our ankles, and on nearly half the days, one of the three of us would retire early, due to ill health, a hurt child, a matter to attend to at the Tower, or within the castle. No one would think it odd that Carlotta was not present with us that day. Nor would anyone likely see her enter my husband's unguarded door. My husband was at court, and Carlotta, by now, such an integral part of Cortan's life, that no one would think it odd to see her at any part of the castle. Still, I admired the way she lied to the Barroness about her ill health. When I saw Carlotta later that afternoon, I learned the location of his notes, that he had been developing these techniques for as long as we have been married, that his entries of late have been timed about once a month, and that at his current rate of progess, it might take several years for Duke Griswold to gain enough control over the old magic to undo the damage he had done to me. It was a good morning's work, and would lead, I was certain, to a long heated discussion about future plans between Carlotta, Master Alerio and myself. That would come later. I had other things to attend to first. 

With Carlotta gone, Leila and I talked about that families in Cortan still loyal to me, or, more precisely, her husband. The barroness did not know how my husband had made me ill, she believed Head Alerio's assertion that he had, and that he needed time to find a means of helping me. The details remained vague, and Leila remained under the impression that my husband had no need for a talisman. I did not know what exactly she thought, or what Head Alerio had told her. She thankfully asked me few questions. 

In the Barroness's mind, I was certain, she sought allies for me in preparation for the day that she would confirm to her husband the events she had witnessed in the Tower, support my accusation of an attempt upon my life, send the Barron for help to Deyalorn, so he could return to help me hold the scepter I had taken the Duke's hand. In my mind, the question of whether I wished for Barron Paulis to rule in my name or not could not even be asked yet. I needed him to buy me time. A day, a year, three, I did not need to fight my husband forever, just long enough to know what he had done to me. What I did with my husband afterwards, who ruled in my name, and who lusted for power in the dark corners of the men I would have to draw around me could not be my concern at the moment. None of that would matter if I did not live.

It was a dangerous plan, so openly living in a circle of men that flaunted my husband's will. It was not something I would have dared think about a week ago. Now, the danger seemed less pressing. It would take Barron Paulis time to speak to the men he had chosen as his, my, supporters. If his efforts lasted for long enough for my husband catch wind of them, I will  have succeeded in convincing the duke that he should not threaten me for the act of defiance. Otherwise, the Barron will have left off his endeavours to ride for Deyalorn.

My personal healer met me in my chamber on my third night in Cortan, as she had the first two. "How are you feeling?" picking pieces of lint and dirt off my healer's robe.

"I am still nauseous. My episodes of vomiting are restricted to the mornings, with nothing to violent as what you saw that day in the Tower. It must have been because I tried to heal myself."

Carlotta made an annoyed tisking sound. "That is not what I meant. Are..."

"It doesn't matter." I interrupted brusquely. "I am what I am. I will be fine tonight."

She surveyed the contents of my dressing table. Makala's knife and one small white bag sat on the table. It contained three glass vials, one, empty. This was the centerpiece of my argument. Everything else, vials of perfumes, pins, powders, jewels had been removed. I needed space to think. "What if he can weild the gift?"

"He can't."

"You don't know that for certain."

I laughed and looked up at my companion. "He would have checked to see if I were pregnant himself if he could." Disbelief stared back at me from a round matronly face. I sighed and tried again. "If he does, I am a stronger healer than he is."

"You are no longer in your prime."

"Neither is he, and he was never as strong as I was."

"The man has given himself eternal youth. There is nothing to say that he has not made himself a stronger healer."

Nothing indeed. This was a fine time for Carlotta to develop a healthy fear of my husband's ambition. "Leave me to my vigil. You will make me nervous."

"I am sorry." My healer walked to the door. "Nisrita." she called back to me.

"Hmm?" I sat with my back to her, almost hidden from her view by the high backed armchair.

"I will see you in the morning by the turtles."

'Yes, of course. Good night, master." This was the ritual. Carlotta would not leave me without my uttering a promise to see her again. As if my words would stop my husband's hand. Fear, it seems, cripples the best intellects.

I spun Makala's blade on my dresser, waiting for my husband to arrive. He had not the last two nights, for reasons I cannot read. It was a good blade, heavy, well balanced, sharp, perfect for throwing. It was hard to resist cutting myself with it, testing my power, reminding myself that I was stronger than my husband. I resisted. There was no point in wasting my strength, not even on such a small wound. I may have need of it at any moment. 

I longed that night, as I had in previous, to slip next door and see my children. Lyca and Elena were safely at the Tower. Barron Paulis had secured berths for Makala and Griswold at the barracks, away from immediate danger. That left my three youngest. If I went into the next room to give Simone a kiss, the sight of his six month sleeping face would sap me of my courage. I had chased Carlotta from my side for less. The master would watch over my youngest three, if it came to that. I would have to depend on the speed of Barron Paulis' horse. 

I tossed the blade, shaking morbid thoughts from my mind. The ivory inlaid handle spun smoothly through the air, landing easily in my hand. The waiting lasted for untold ages. I had not slept the last two nights until I was certain my husband would not come to me, and even then, poorly. Just this suspense would wear me down, if I let it continue. I picked a spot on my bed post, then sent the blade through the air to land, quivering where I wanted it. I stil had good accuracy at less then ten feet. I walked over to retrieve the weapon, cleaning it absently against my sleeve.  How many more nights would I spend like this, dithering. My husband would not hesitate to kill me, why did I? I coated Makala's knife with a thin sheen of oil. I may come to depend on this weapon. I had not sunk to Griswold's level. My first husband would not have hesitated to poison this blade if he needed to. Neither should I.

I returned to my seat and tossed the blade in the air again, wondering if my husband would not come. I heard the sound of my door opening, the hilt landed in my hand. I put it down on the table, took a deep breath, and waited for my husband to enter.

"Get up," the Duke ordered. "You are going to Turina." I stopped myself from obeying instinctively. This I had not expected. "I tire of your presence here."

I swallowed my fear and rose from the shelter of my chair to face him. "I beg your pardon, but I am not."

My husband turned purple. "You disobedient little cunt!" he said seeing my healer's robes. "You think Alerio can protect you? Dress. The old man cannot reach you where you are going." 

I ignored him and continued in a pleasant tone. This discussion had to be on my ground. "My knife is poisoned."

"Is it?" My husband paused to put an amused look on his face. "Then kill yourself with it, and save me the trouble of paying mercenaries."

"It is meant for you, my liege."

My husband laughed, took a cloak from its hook and moved towards me. I brought Makala's knife from the table behind me into my hand. My husband stopped, just out of reach of my talisman. "You are more of an idiot than I thought. Your life if forfeit if I leave here wounded."

"You will not leave here wounded, my liege." I tossed him the white bag on my table. "You have given me the gift of augmented power. I will wager that I can wound and heal you before you reach the door." My husband emptied the contents of the bag, and stared at the two full vials of black powder, and the empty one. "Who will rush to move against me if you show no blood?"

"You don't have the courage to try the mushrooms," he said at last.

"Perhaps. Perhaps I don't need them. I was recently the most powerful healer in Marsea. Do you have any concept of the extent of my abilities, or the limits of yours?"

My husband raged, but he did not move. "You dare twist my words against me?"

I raised my arm. His neck was the easiest target. I could see his pulse throbbing. "And you test my patience."

Duke Griswold looked at the blade in my hand, and then at me. "You crazy whore. Do not think I am done with you," he said backing up.

"You know where to find me, my liege," I smiled. Then he was gone.

I sagged onto the floor. It was over. I had won the battle. The war would just become more heated. I trusted my husband to find new clever ways of threatening me. But for now, this room, at least, was mine. I would sleep with Makala's knife by my bed. I would carry blades on my person, I would arm myself for the days to come, until I could surround myself with allies, and make my accusation against my husband. I the meanwhile, I would regain the ground I had ceeded to Griswold, yard by poisoned yard.

\end{document}



