\documentclass{article}
\usepackage{fullpage, verbatim}

\begin{document}

The Duchess of Cortan's story for the years I spent in Lir was an unmitigated tragedy. For the rest of the country, one may say those years were a phenomenal success. My father considers himself lucky to have lived through such times. I doubt the duchess agrees with him. He did try his best to protect her, I have no doubt of that. But he is an old man of a noble line, and she has suffered the misfortune of being orphaned and raised by the Tower. One cannot learn manners late in life that were not learned as a child. For all of Nisrita's efforts to refine herself, her breeding does show.

Her situaton did not start off poorly. When I left for Deyalorn, she was in high spirits. She had a terrible week after the Counsil of Nine revealed its findings and shamed her husband. I do not know how my father managed to gather the requisite two Masters and three Healers from Deyalorn's Tower to make her a healer. Perhaps Healer Ferdnando was right. She was given her rank because the Regent Consort wished it, and at the moment, the did not dare back Duke Griswold against her. It was an ugly business though. It was fortunate that Nisrita stayed away from the Tower during the few days immediately following her attaining her rank. The things my father heard. One would think that the Duchess had come within inches of being tried for treason, not her husband. 

Master Alerio put an end to it quickly, at least nothing was said within my father's hearing in the Tower after the Day of Unions. Nisrita's ability to sail haplessly through life amazes me. Between her station and her gift, she could easily be the most powerful woman in the country. Kings and generals and powerful masters fall over themselves to come to her aid, yet she simply floats by them, like a boat without a rudder. If I had half the resources she had at my disposal.... Then again, I was blessed to be born a distant cousin to the Queen Regent. Nisrita was raised by Masters and Healers who only had an eye for her gift. They neglected to teach the poor girl anything else. If I did not have such a fragile domestic life, I would have liked to take her in hand, to see what Timmon and I could make of her. Ah, but I am not complaining of my life with Timmon. I could have done much worse.

After the Day of Unions, father asked Nisrita to continue her research on the effects of the mushrooms. I am not sure which I found more amusing that day, the shock on my father's face when she admitted to killing all her beasts in a fit of pique, or Nisrita's attempt to control her outburst when my father explained that his previous lack of support sprouted from his  dislike of young women taking part in poltics. As I said, he is an old man from a well respected house, and she a young orphan who trained with Cortan's army. They made for an interesting pair. I would have loved to be a fly on the wall, watching them work together. My father was fond of her though. They grew quite close to each other, inspite of their differences.

Nisrita threw herself happily into her work. She accepted her shifts in Deyalorn's infirmary. She visited frequently with Baroness Paulis, and drew the attentions of many of the Barron's friends that had previously kept their distance. With her friend's help, she managed to be one of those rare women who manage to flourish at the downfall of their husbands.

When I gave birth to Firi, Nisrita and Carlotta both visited me daily. It was quite sweet of them, really, to go through so much trouble on my account. They never came together of course. The two women hadn't spoken to each other since the dramatic performance that started the Counsil of Nine. It was really quite tiresome. Nisrita would not mention Carlotta's name. Carlotta would grow dark and sullen every time Nisrita came up. In the first few days after a long delivery, it is hard to remember the rules regarding what subjects can be mentioned to whom. My memory failed me when Carlotta visited the morning after Firi's birth. 

"Good morning, Sophia. How are you feeling?"

"Much better after having slept for twelve hours. I am so sorry I was rude to you yesterday evening."

Carlotta tisked at my comment while she made a show of examining me. "You speak as if I have never encountered a tired mother after giving birth. You seem to be recovering well." She paused as a spot of yellow caught her eye. "Sophia, why are there leaves scattered all over your room?"

"Nisrita brought them to me last night." I bit my tongue too late. The taboo name had already been said. I carried on to undo as much of the damage as had been done. "The changing colors have enchanted her. She wanted to share her joy with me. I think it sweet."

Carlotta made a sour face. "Its childish. Not everyone new to this city at this time of year goes about gathering leaves all day."

This was an unusual reponse. "What is bothering you, my dear?"

Carlotta bent to pick up the leaves that had fallen from its bouquet. "Nothing. I just think you are too indulgent of Nisrita, that is all."

"Did you forget Cesaro's scarf at home today?" I knew the signs of marital discord. Carlotta had worn a red and gold silk scarf that belonged to her husband on her person every day since she married. She made a great show of being in love with him. 

"No." Carlotta threw herself down on a chait next to my bed. "We fought last night. He wants me to go to Turina or Cortan."

"I thought he had transferred to Deyalorn." 

"He told me he would, but he never did. My husband has his eye on a barrony in Niev. He fears he will not get it stationed in an interior duchy."

"You could do well in Turina, one of the founding memebers of the new Tower? Or in Cortan? Master Adele may no longer hold favour there now that Master Alerio is Head, but she is just as knowledgeable about fetuses as Head Isadora is."

"I do not want to go," Carlotta insisted.

I smiled. "Ezaro?"

Carlotta did not answer me. She did not have to. "Have you decided on a name yet?" she asked. We went through a long list of names and their histories, considering what would please Timmon and speculating on what would be pronouncible in Lir. The priest's daughter had changed over this last year. When I married, I could not imagine her not wishing to follow her husband to the end of the world, let alone having an affair.

She did not ask me for advice on how to handle her husband. In a few weeks time, Sergent Madriano left Deyalorn to escort the representatives from the House of the Triumverate back to Voltain.  Carlotta was a clever woman who knew her way in the world.

I left my confinement for the Barrony of Romino. I would spend most of the winter there, then head south to Lir before the spring campaign began. My father suggested that I take Nisrita with me. She was making friends outside the Tower, due to the close friendship with the Barroness Paulis, but her interactions with the members of Deyalorn's Tower wore on her. A break would do her good. 

I worked on Nisrita's reluctance. I admit, it was an awkward position for her to be in, visiting the house of the man she loved, who had proposed to marry her, but did not. Barron Romino knew none of the complicated history between my husband and Nisrita since she married. He did not hold it against her that she married Duke Griswold, as opposed to his son. If anything, I imagined that he would wish to thank her for saving his son's life. 

Nisrita showed more reluctance than I expected, or understood. In the end, it was her assitant, Malia's, promises to work with her tirelessly on her manuscript and care for her animals that seemed to convince her. Something had passed between Malia and Nisrita in those first few weeks after the Counsil of Nine declaired their rulings. My father noticed it as well, though he had no more idea of what had happened than I. The two had become inseperable. It was all for the better. The girl seemed to bring some life into Nisrita's careworn face. 

Nisrita spent four days at the house of the Rominos helping to settle me in. She did not leave the manor, preferring to spend her time by Barron Desmond's side, listening to the sick old man's stories of his youth. It was warm in the old Barron's rooms, she claimed. Not being used to Deyalorn's winters, she felt cold too easily. Nisrita is a terrible liar. There was more than a fire's warmth she sought in the old man's rooms. She served and cared for my father in law with a devotion that most men can only hope for from their daughters. Barron Desmond doted on her in return, and Barroness Miri spoke gratefully of the relief she provided. Neither of them reproached me, of course. I had just had a child, they understood. Nisrita apologized, as she left, for the possibility that she had overstepped her bounds and made my position awkward. I accepted the apology, for she had made me look undutiful. There was a history in that house that I was not a part of. I understood why Nisrita had not wished to accompany me to the place, and I was glad for her to be gone. Barron Romino's family meant more to her than was entirely appropriate for one in her position. 

When I saw her next in Deyalorn, during the few days I spent readying myself for the long trip to Lir, Nisrita was ecstatic. She had submitted her manuscript on fetal development in various species to Deyalorn's library, yet she barely had time for me. She spent most of her waking hours in my father's workspace in the Tower. He explained to me that while Nisrita was with me at my father in law's house, she had ceased her experimentation on the animals. In her absence, half of them refused to eat and died. She thought, and my father was inclined to agree, that she had found the reason for the wasting illness that had plagued so many of my gifted forefathers.

It was exciting, certainly, but not nearly as exciting as Nisrita found it. Nisrita shared with me a vision in which she would show the world enough of the ill effects of the mushrooms that the Crown would be forced to reinstitute the ban. Proof of the wasting illness, as well as the documentation of its exsistence in Heads of old, and Master Alerio's testimony to Ezaro's condition would certainly go a good deal to this end. I did not have the heart to correct her. The times were changing, for the better, and not as she wished. Harnessing the power of the mushrooms, if not the old magic, was inevitable at this point. Duke Griswold had shown Marsea the way.

My father did point out that she spent a lot of time with the Liri ambassador Sama Araki. He did not approve of Nisrita entangling herself in the still somewhat contentious politics surrounding the Liri embassy so soon after her conflict with her husband. I spared my father the complicated details of my domestic life. It was clear to anyone who cared to look that Nisrita missed my husband. Naturally, she wished to know of the lands he now lived in. Not many people cared to examine that old whisper, for which I am grateful. However understandable her motivations, I did not like the news of Nisrita's closeness with the Liri ambassador. Not that I did not think enjoying the unqualified attentions of a man would not be good for her, if that was her intensions with the man. Timmon's nephew seemed eager to give her just that, but Nisrita would not hear my encouragements. No, Sama Araki disturbed me. As the Marsean ambassador's wife, I had reason to meet with him several times during my confinement. He was a tall dark man, with wide features and a bulbous nose. Certainly not a handsome man. He towered over my bed, answering my questions with an air of supperiority I often see thrust upon non-gifted women. Nisrita, in her inexperience, can be a poor judge of character. She did not need another domineering man in her life. 

I told her as much as I left Deyalorn. She simply smiled and said that it was not what I thought. Far be it for me to judge someone else's intimate life. I wished her well.

\vspace{.5cm}

"I cannot quite believe that I am finally surrounded by a family of my own, Sophia. Did you have an easy journey?" Timmon asked when I appeared before him in our small palace in Buzen. My new house was surrounded by fountained gardens, had two verandas, three balconies, and beautiful rooftop terrace shaded by mango trees in full bloom. It was on this terrace that my husband met me with a broad welcoming smile and two crystals of a light lemony drink. These were the first words he had spoken to me in the two hours since I had stepped off the luxurious river boat he had sent to bring me here from Lesoko. I had expected to find him in a foul mood. His geniality and obvious joy surprised me

"It was quite comfortable, Timmon. Thank you." I had traveled to the capital by what I can only describe as a small floating house that Timmon had sent to the northern City of Lesoko to meet me. "The Liri seem to use rivers as we use roads."

"Yes. There are many customs here that I am afraid you will have to adapt to. I hope you will find life here pleasant all the same." He handed me my glass and kissed my forehead before leading me to a stone bench. This was more physical affection he had shown me since our wedding day, when certain public displays were necessary. I was intensely curious about this sudden change of manners.

"Would these strange customs explain your behaviour today, my dear husband?" I asked with a hint of coyness. 

He allowed me an embarrased smile. "My sincerest apologies. Liri women are not accustomed to the same liberties as Marsea's gifted women are, I fear. Not having the gift, they can only serve their families in the conventional manner. Beyond the privacy of this house, I must observe the customs around us."

"Of course." That explained the silent greeting at the quay, the perfunctory introduction to his excellency, King Ayum, and the swift bundling into an enclosed palanquin with my daughter, while Timmon went to great lengths to introduce Eugenio to the King. It would take some swallowing. Even ungifted women in Deyalorn were given more leeway. I would have to accept this. My family was here on my instigation. I would do as my duty dictated.

"There are other adjustments we must make to our arrangements," Timmon continued his uneasy apology. "To avoid scandal, I cannot allow you to meet with male guests without my presence." I let out a small startled gasp. "I will conduct my affairs elsewhere. The house is yours to do as you please. It will take some adjustment, but I am certain we can come to a new understanding."  I did not like the idea of Timmon knowing who I entertained, whether they were my lovers or not. I wondered how Liri women ever managed to have an affair under these restrictive customs. Perhaps that was the point. "With that unpleasant business settled," my husband continued, exhalaing the matter from his mind, "what news do you have of home, Sophia?"

I laughed. Requests for news are never the questions they seem when one's husband made his name in the military as and adept intellegence gatherer. "What have your spies not told you already?"

"Little. I value your perspective."

I told Timmon about his father's health, Nisrita's friendship with the Baroness Paulis, Captain Dielo's marriage, the plans for a new Tower to be built in Turina, rather than waiting for the conquest of Escasaine, and my dislike of the Liri ambassador before settling down to retell the long and complicated tale of the Counsel of Nine. A maid brought out a plate of figs, cheese and something deliciously rich and sweet while I talked, and refilled our glasses with something stronger. By the time I finished, the sun had sunk low on the horizon and the hot day had cooled to a tolerable temperature. 

"He is a visionary, one must admit," Timmon said when I finished. 

"Duke Griswold? Do not let the duchess hear you say that."

"He is corrupt, ruthless, short sighted and a brute. But he is a visionary," Timmon ammended.

"Why this sudden admiration for the Duke, Timmon?"

Timmon emmited a soft burst of laughter. "Several years ago, Duke Griswold, then of the hundred horsemen, came to Lir to fight some battle or another. He was here for a month, perhaps? In that short time, under those trying conditions, he learned of a grain that grows in the wetlands to the south that could stop famine forever in Cortan. The stalks grow so closely together in the watery fields, that they say one can place a new born babe on top, and not have it fall. But this only grows in the hot wet climates of Lir. I believe Commander Griso wanted it for Cortan. He knows to hold Lir, one must have ships, so he steals plans for sea going ships, hoping to entice Selvand into training healers to work on river boats. He tries to turn public opinion against Lir. When that fails, he reccomends me as ambassador to Lir in the hopes that in my drunken incompetence, I allow him to try again in a few years. All this he does by himself, while, as a side project, he finds a way to improve the Tower's power. One cannot help admiring him."

"Put that way, one cannot," I allowed. "Be careful who hears you speak that way in Marsea. Admiration for the Duke is currently not a popular opinion."

"And what is your opinion, Sophia?"

Timmon knew better than to ask my political opinions. A woman with strong political positions decreases her pool of potential lovers. "It is my opinion that the Duke has grossly overestimated my husband's incompetence."

Timmon smiled broadly and led me down to supper. The meal was a light, pleasant affair, conducted on the northern veranda, on the side of the house not facing the main road. The sound of crickets and frogs accompanied our meal, my children whisked out of sight by one set of servants, the meal cooked and served by others. It felt more like the luxury of being a guest in Barron Romino's house, than being in my own. Timmon explained to me the customs I would most immediately need to be aware  of, which my Liri escorts, provided by Sama Araki, had failed to explain. Then he told me in broad strokes how he ahd kept himself in the months since I had seen him. We talked until well past dark, sitting in the pleasant night air long after my children had been put to bed by others. 

"Do you think you will adjust to life here?" he asked as we finally went our separate ways to our chambers.

I recalled the beautiful table I had just dined on, the quiet garden it looked upon, and the silk sheets I had slept on in my river boat. There promised to be drawbacks to living in Lir, certainly. I had little hope of power in this country, not as I had known it in Marsea.  On the other hand, on what was left of Timmon's salary as a commander after paying his gambling debts, we could afford little more than Nubo and a maid for myself. I had despaired of even affording a nurse for the child growing inside me at one point. "Yes, Timmon, I think I will adjust nicely."

\vspace{.5cm}

I learned much of Lir in the following weeks and months, more of my husband, and a surprising ammount about the politics in Marsea, albeit dated. One does not expect to make any pleasant discoveries about one's husband after a year and a half of marriage. The man I shared a house with in Buzan was, to my great delight, not the man I married.

One of the easiest adjustments in my new life was the abundance of artists and scholars in Lir. Timmon and I attended several different orations, plays told through song or dance,  or musical performances a week. The subjects were always religous or historical. That such works are never as moving in translation as in the original, was betrayed by the audience around us. I found the performances were extremely pleasant. They had an exciting, exotic aura, completely unlike attending plays in Marsea, or lectures in the Towers. I could sit for hours and let the exotic customs wash over me, losing myself in the beautiful dark bodies smelling of sandlewood, parading themselves before me in silks embroidered with silver and gold. As my husband explained, lacking Towers, the time Marsea spent on the art of healing, Lir spent on the art of storytelling, drawing and sculpture. The beauty and decadence of Buzen entranced me. 

This is not to say I did not miss Marsea. I missed my homeland and its Towers  more than I had expected. Timmon had two healers in his retinue, which was more than enough for the embassy staff for fifty and their families. It was not my safety that I feared for. I missed the intellectual life that surrounded the Tower. This surprised me. I had married before I became a healer, immediately became a mother and followed my husband on those miserable campaigns so he could keep a distrusting eye on me. Upon becoming a widow, my father encouraged me to become a healer so I could support my son. I returned to Cortan, not because I had any love for the intellectual life of the second greatest Tower in Marsea, but because I knew peole there, and I could not ask my father to help raise Eugenio. When I married Timmon, I continued practising because our household needed the money to make up for his gambling debts. At no point have I ever had any ambition within the White Tower's structure. I enjoy the company of the Tower's men. I had thought that my sole interest in the intellectual life of Cortan's Tower was as a means of meeting, and pleasing them. I had been wrong. No ammount of filling my days with exotic performances and the company of Marsean and Liri wives erased my hunger for the familiar sounding of the gong at the start of a Tower's lecture.

By the end of my first month, I found myself spending more and more of my time alone in the shade of our mango tree. The weather was impossibly hot, and the customs around me a little too strange. Performances accompanied by a husband uninterested in both oneself and the subject matter cannot solely sustain a woman, however delicious it may be to watch the Liri nobelmen and their wives strut around in their finery. I was lonely. I did not mention this to Timmon, of course. I did not wish to appear ungrateful for his uprooting his life according to my desires. It surprised me, therefore, when I awoke one morning to find an open letter addressed to my husband and a pair of silver earings on my dressing table. The letter, I guessed, was one of the regular messages Timmon got from Marsea, keeping him abreast of affairs at home. It contained news of battles won or lost two weeks ago in Niev, along with bits of gossip from Cortan's court and its army. The simple news brought more comfort than I had expected. I read the letter thrice before returning it to Timmon's desk. 

When I asked about the unexpected fndings, my husband averted his eyes and said only that I had appeared wilted from my long separation from home and the flattery of men. My dear, sweet husband. He was aware that he had crossed a line in the arrangement of our marriage, however well intended. We did not speak to each other of our lovers. I found his embarrassment on the subject endearing. I also learned that my movements were under greater scrutiny under my second husband than they had ever been under my first. While Julio watched my every movement, he would not have noticed my loneliness. The letters continued to appear in my room, though the acompanying gifts did not. Eventually, I simply joined Timmon in his office and we discussed the tidings from home together. Soothed by the goings on of my countrymen, I found my days of entertaining Liri nobelwomen easier to bear.

For his part, Timmon seemed, as the Liri would say, like a fish returned to the river. I had married a man haunted, by his own admission, by the spirit his previous lover. An edgy man by nature, the unforgiving memories drove my husband to fits of wrecklessness and rage when he could find nothing else to occupy his mind. Only in the first few months of his physical recovery last spring, did I see him shake off his need for constant activity. His lover's spirit did not seem to follow him to Lir. He surrounded himself with the plethora of gods of this new pantheon whose temples he could now visit, if not worship publicly. I also found him surrounded by men. We could not walk for five minutes in Buzen without being accosted by a handsome young naval or army officer wanting his attentions. Timmon rarely spent a night at home. Unlike in Cortan, this did not cause me to fear for my family. When I saw him, he was always at his best, cheerful, attentive to my self and my children, meticulous about his responcibilities to the court in Buzen. 

Liri men, I am told, do not expect their wives to satisfy their higher needs. They have a custom of keeping close intimate friendships with one or two other men, to satisfy their spiritual and emotional requirement. In the world I grew up in, a woman's touch was considered to be the only balm that soothed men's souls. Unaccustomed to my new role, I was tempted, in those first few lonely months in Lir, to lash out jealously against my husband's outings. I had no ground to stand on for creating a scene. I could not complain of neglect or endangering our family, as I had suffered so often in Cortan without complaint. Timmon respected the rules of our marriage perfectly. He went further than the rules prescribed. From my open window, in the occaisional mornings he awoke at home, I could hear him playing with and singing to Firi in the garden as I had watched him play with Nisrita's children in Cortan. He had a deep comforting voice. I had never heard him sing in Cortan.

\vspace{.5cm}

Through my husband's network of informants and letters from home, we learned that Nisrita had taken part in the a defense of the small village that had been turned into a base in this year's campaign in Niev. 

"That was wreckless of her," I said when Timmon read me the letter.

To my surprise, he shook his head. "We lost half the healers stationed in Rialotte when it was razed last year. She has some training as an archer. It is unclear from this report whether she endured a greater risk hiding among the healers, or raising morale amongst the men and helping them fight."

"A fighting healer to complement our gifted fighters, Timmon?" I asked wryly.

"No." Timmon turned away from me. "That is not something I would propose for Marsea." My poor, loyal husband. He wanted to be with his army, I knew. 
Before leaving for Lir, he had questioned whether it was right of him to lead a life of leasure as a diplomat rather than leading his men to battle. Sitting idle, so far from home while Marsea's armies struggled in Niev could not be easy for him.

%Late May 5
We recieved a letter from Nisrita that she had returned safely to Deyalorn, as I began to discover the miserable long hot months of summer rain that would end the impossible heat of the Liri spring. Upon her return, the duchess found that my father,  with the girl Malia's help, had finished documenting the effects and the causes of the wasting illness. She sounded excited and eager. She would start lecturing on their findings immediately. She spun happy fantasies of writing another manuscript, including her findings with the hints found in biographies of old heads about the effects of the mushrooms on higher animals and man. She wanted my permission to approach my father on the subject of our family's history. I gave it, of course. She did not need my permission to speak to my father. My husband could not have been prouder had Nisrita been his own daughter. I, on the other hand, could not share in his joy for her accomplishments. I doubted that it would take many years before the Towers could find a way to resolve any issues she illuminated. There was too much at stake for the Towers not to put their full efforts against her. Still, Nisrita sounded much happier than she had been before the Counsel of Nine. She wrote fondly of a number of friends in Deyalorn's court, she was adored by the army, Lyca seemed healthy and strong. She was practising and studying as she wished to be. There were many other reasons to be happy for her.

%June 5
It was only a few days after that happy letter that we heard news that Duke Ergino would go to Selvand to apologize in person for this uncle's behaviour towards the northern duchy. I did not understand why the late apology upset my husband quite so much as it did. "At the time of writing," Timmon explained, "Duke Griswold had been held by the Tower for nearly nine months. No one spoke for him in all those months. That his nephew is moving in his favour indicates that he is gaining favor in Marsea. The Duke's attentions will turn southward again soon." He did not mention the duchess by name, but she was in his thoughts as she was in mine. I saw several letters addressed to commanders and generals in Deyalorn's army sitting on his desk when I asked him later that night to send a message to Master Alerio from me. There was little either of us could do for her from afar that her friends near by could not supply. I hoped that I had been wrong about Nisrita's character, that in the months that she had developed friendships in Deyalorn's court, she would be able to withstand her husband's rage. 

%July 5
Evenually, the interminable rains stopped in Buzen, bringing a pleasant late summer. My family went on a long trip down the Sowene River to the lands of the rice paddies and banyan trees. We stayed at the house of scholar at Buzen's university, a slender tall man named Minoru. He had elloquent long fingers he used to draw landscapes, exquisite taste in printed silks and oysters marinated in lemon and garlic. I had learned, with time and Timmon's encouragement, to seek from the University much of what I had once found enjoyable in Cortan's Tower. Of all the Liri men I met in my first six months living among them, Minoru was the most attentive to the details of my desires. His flattery was not the banal variety paid by men to women they think little of. He knew the names of my father and brothers. He took pains to learn the customs of Deyalorn. The University was not the Tower. I knew nothing of their philosophers or their history. Even if Liri culture allowed me to be a colleague of a man, I could not be his intellectual equal. If I must resign myself to being an ornament for my lover's pleasure, I chose to be a well cared for ornament. When I took him as my lover, I was not disappointed.

Our visit to the southern wetlands overlapped with an exotic holiday to a god with an unpronouncable name involving smoke, drums and the graphic sacrifice of a water buffalo. The ceremony took place in a Liri temple, each of which, I have learned is of a different construction to the rest, even to other temples of the same god. It made the process of recognizing and learning the pantheon extremely difficult. The only common theme identifying the temples were the silver spired gates which lead into a large, usually open air, square. On the interior walls of the gate almost always hung large sheets of metal etched with pictures telling stories of the god in question. Beyond that, nothing was constant about the contents and the design of the temple square. The floor of this square was lined with an elaborate black and red mosaic, at the center of which sat a dry well. At the bottom of the well, stood a large statue of the diety in question, only the top of his head and shoulders visible to the devoted peering down. I failed to see my husband's fascination with this religion where the idols of the gods cannot be admired. The ceremony itself was not one for the weak of heart. Drums beats and lowing accosted my ears, while smoke and the smell of incense mingled with that of burning butter and flesh stinging my eyes and filling my nostrils. The Liri around me swayed to the rhythm of the drum beat with a glazed look in their eyes, as if enchanted or hypnotized. I left the temple as soon as I could, which was after the buffalo's blood had been poured onto the god at the bottom of the well. The devoted went to pour their offerings on the bloody stone. My husband, I was certain, would also give his offerings in private. 

It was odd to think of myself as married to a heretic. In Cortan, I kept the secret for Timmon, as we agreed. My husband rarely attended morning prayers to the triumverate, offering his fealty with Nubo to these foreign gods. I performed the duties for both of us, lighting candles at the feet of the Destroyer, for Timmon, and the Preserver, for myself every morning. In Lir, Timmon appeared at morning prayers with the rest of the ambassadorial staff for the sake of appearances. How he settled appearing before two sets of gods with his concience, I do not know. Under the guise of his public duty, my husbad appeared frequently at religious events such as these. I feared for the scandal it would cause if our staff learned that Timmon worshiped the Liri gods. This habit of his was more dangerous to our family than his drink and gambling combined.

Outside the temple, Eugenio battled Minoru's  nephew with the wooden swords curved as the Liri use in battle. I watched at the edge of the crowd of gathered children, fearful that my son would be sent back to me bloodied by the larger boy, in need of healing. Timmon had personally taken on Eugenio's training, against my wishes. I did not like the number of bruises he left for me to heal on my eight year old's body every day. To my horror, Eugenio showed no recognition of the fact that he attacked a larger opponent. He ignored several blows about the shoulders, came in close and gave the older boy a good scare that sent him running. Eugenio gave chase to the children's delight.

Even I had to admit that Eugenio had put up a good fight. I looked around for my husband to congradulate him on the fruits of his training, but he had not yet emmerged. It was not like him to linger for inappropriate lengths of time before these foreign gods. He knew the risk of whispers in Buzen and Marsea. I stiffled my impatience and annoyance and mingled with the family of the local lord. 

I could not speak freely to my husband until we returned to Minoru's house, where he led me immediately to the chamber we shared for propriety's sake.

"Did you enjoy the ceremony today, Timmon? You were in the temple a very long time." I began.

"This arrived today," he said huskily. He handed me a letter adressed to both of us in Nisrita's pronounced handwriting.

I accepted it with a sense of foreboding. It was the same pattern as had emmerged in Deyalorn, that I had been warned of if Cortan before I wed. There it had been drink and violence. Here it seemed the passion took the form of a religious ferver. The effect was the same, and it went against our arrangement. I put the letter down, unopened. "Timmon," I said firmly and patiently, "you cannot put your family's name at risk every time there is bad news from the duchess." 

"I stayed too long in Zinmenzu's temple." my husband snapped. Guilt and helplessness are not a pleasant combination for him. " I needed to pray. It will not happen again."

I had long learned not to listen to those words when spoken by a husband. At least with Timmon, I knew they were not meant as a simple attempt to placate me. "She has friends in Deyalorn and Cortan who can help her better than we can." I insisted. "Nisrita was correct in thinking that our presence near her would only endanger her." 

"Read the letter." Timmon said, and left the room.

It turned out we were wrong in our fears for Nisrita. It is understandable that Timmon would be wrong on such a matter. He is a man of honour. It would never cross his mind to act as Duke Griswold had. I, on the other hand, had suffered from a man like Duke Griswold, had been grateful when the time came to don my widow's veil. Who should have been able to guess was merely an academic excersize. Even if I were in Deyalorn, I could not have helped her. Short of riding to Deyalorn and assassinating the Duke for doing what was well within his rights, neither could Timmon. 

We had both expected Nisrita's troubles to come from a political front, punishment for standing against her husband, a seclusion from her Tower or her sons, possibly a smearing of her name. These, she had friends everywhere to help her recover from. No one can protect a woman from the interminable nights created by her husband's desire for children she does not wish to bear. No one can save her from the humiliation of his hired healers checking every morning for a life in her womb. No one will take the child from her body when he announces that he wants it born. The letter I read was written by a broken women. One of my father's students would give the lectures she had planned on the wasting sicknes side of Deyalorn. Her husband would not let her travel. He had taken her talisman away. She would not be able to march in next year's campaign against Niev, and with that, she feared being forgotten by the army. Nisrita lived in fear of a natural miscarriage because of the suspicion it would plant in Duke Griswold's mind. She saw no ray of hope in her situation. The impossibility that the situation held for me lay in the fact that it was dated over two months ago. There was much I could advise her to do for herself, but my words would reach her five months after the crisis. Timmon's actions at the temply may have been inappropriate, he was right in surmising that there was little we could do for the duchess but pray.

\vspace{.5cm}

It is a queer situation to be separated by such a long distance from those that one loves, to know that one is simply an observer in their lives, and they in one's own. I had a sense of this when I moved to Cortan from Deyalorn. It did not become a harsh defining reality of my existance until I moved to Buzen. The distance of two weeks along a well maintained road is a far cry from that of two months. Timmon and I lived our lives with the knowledge that we would always laugh or cry for our friends long after they had experienced the moment. This left us with a choice. We could either create our own joys and sorrows, or we could exist in a constant state of anxiety for the news we may find out happened two months ago. 

It was hardly a choice, really. The former was the only sane possibility, as tempting as it may have been for my husband to lapse into the latter. As such, Timmon enrolled Eugenio to learn to fight in the Liri fashion while continuing to train him at home in the Marsean style. Firi learned to walk talk and splash in the fountains of our house's gardens. I learned enough Lir to stop needing a translator around the house, and started learning to draw in the Liri fashion, not landscapes as Minoru did, but simple scenes from nature. I brought tailors from Cortan to make dresses for the women of the court in the Marsean style for the winter. Timmon continued his work privately. It comforted me to see that did not miss a beat due to the bad news from home when he returned to Buzen from our trip south, but returned to his meetings with naval officers and courtiers as always. 

In the meanwhile, we learned that Turina had established a Tower, and sought healers to fill it. In Deyalorn, Carlotta had found a way to replicate the miracle she performed with Nisrita which saved Lyca's life. She would spend the next several months lecturing at the country's Towers with Ezaro. I laughed when I read the news. She was a sly woman, to manage consent for that from her husband. I heard from my father that he, with significant help from his student Malia, had convinced Nisrita to spend her pregnancy working on the manuscript she had dreamed of, documenting evidence of the mushroom's side effects. We heard little from Nisrita. 

Some time during that winter, Timmon came home with a look on his face I aspire to put on that of my lovers. Curiously, after its appearance, he spent more nights at home than had previously been his habit. His morning games with Firi took on an ebulience I had not known my husband capable of sustaining for days at a time. I gave in to my curiousity, and broke one of the rules of our marriage, when he appeared one evening, book in hand, on the terrace where I sat with charcoal and easel. "May I congradulate you on your new lover?" not taking my eyes off my drawying

"You may, when I have one, Sophia." He replied, ignoring my breach of conduct. "Do you remember our appointment at the quay in the morning? Captain Isumu is due to arrive." 

"Of course, Timmon." My husband had a book in his hand. No lover from the navy inspired Timmon to pick up a book. I put down the charcoal to look at him. "Will you tell me which of your old lovers I have to thank for dressing my husband as a school girl meeting her first pair of lips?"

Timmon laughed a loud boisterous laugh the startled a pair of parrots from the mango tree he sat beneath. Then he composed himself with a deep contented breath. "I have met a man, a friend of one of yours, in fact. I cannot call him my lover until I can tell him there is no one else. This, at the moment, I cannot do."

A scholar then. That explained the book. I was jealous. It had been some time since I had been so besotted. "If you will forgive my saying so, Timmon. You are a fool."

My husband chortled. "Perhaps, but I am a loyal one. I must see to Marsea's needs first. Lovers talk, as I am sure you know. At the moment, I need to listen."

I joined in his mirth. "I am glad, husband, that I will never have to encounter you as a lover." Hitherto, I had not considered the risks of falling in bed with a man so skilled at extracting information.

He tisked. "And I you, Sophia. You have learned too much already, all without the advantages a lover has."

He turned to his book, and I to my easel, intensely curious about the man who could bring such a drastic change to the one I married. I met Lyo shortly thereafter one of the innumerable religious ceremonies the Liri hold. He was a religious scholar at the university. I had met the man before, he was several years older than Timmon, married with four living children. Timmon said nothing, Lyo said less. All the same, it was not hard to read the signs when I knew what I was looking for.

\vspace{.5cm}

%Jan 6
About two months later, mid spring in Lir, a time when the trees were just budding in Deyalorn, Timmon heard news of his father's death. He entered my chamber early one morning, looking uneasy and disheveled. He put a letter in my hands and said "This came yesterday evening."

I unfolded it and read with growing alarm. Baroness Miri explained that her father-in-law had succumed to a winter fever. She gave the particulars of the timings but spared her brother in law details of the illness. Timmon's brother was with his father at the time of death, as was the duchess. She said that her son Guiseppe was ready to take control of what were now his lands. She concluded with a few other trivial pieces of news of the family.

"Why didn't you show me this earlier?" I asked, unable to mask my grief. 

"We dined at the palace last night." Timmon said flatly. "You have waited two months to hear this news. What is another day to that."  There was no question as to the grief and strength behind this calculation. I would not have been able to hide the news as he had last night had I been in his position, let alone protect him from it.

I put the paper down and let the tidings fill the air around me. Timmon stood with that complete stillness that only overcomes him when his mind is fully engaged elsewhere. 

"I will pray for him," I offered, when he finally shifted and sighed. 

"I will join you." my husband said in that same dull flat tone. When I looked surprised, he showed a hint of annoyance. "They were his gods. They will watch his soul, not mine."

We knelt before the idols in my chamber. I guided his hands and prompted his words when his memory failed him, working slowly and meticulously through the ceremony. It must have been years, I realized from the lapses in his memory, since my husband had prayed to the triumverate. Timmon held to the Liri gods with a fierce loyalty that he had to mask outside our home, but I knew it to be akin to his love for Mersea. To love his father enough to leave his new gods and kneel to the triumverate again was no small gift. 

He annointed the Destroyer and, after a moment's hesitation, the Maker, then sat still and silent, consumed by his prayers or memories. I knelt by his side and said my own prayers for Barron Desmond. When I finished, I sat back and waited.

"I spent my evenings with Lyo arguing the values of different types of valour in Rishiki's eyes while my father lay dying," Timmon finally said. 

"We cannot live our lives constantly trying to guess what is happening at home." I reminded him. 

He ignored my ill fitting coment. "I have lied to my father for half my life, fearing he would betray me for what I am. I will never know." 

I turned to remind my husband of his father's love for him and pride in his accomplishments, but he was crying. I took his hand in mine and let him sob like a schoolboy at the foot of my bed. We were so far from home. There was nothing else I could do.

\vspace{.5cm} 

%March 6
Several months later, as the weather grew unbearably hot, Timmon handed me a letter as I sat fanning myself in the high ceilinged room I used to recieve Liri noblewomen. The letter, addressed to me in Carlotta's hand said that she would move to the Tower in Turina as the head of the girl's school, where she will be named an honorary master. Her husband was eager to finally move to a border duchy. They would not be returning to Deyalorn after the current campaign in Niev, but head directly to Turina. She hoped to use her new teaching role to start a family. My husband failed to share my joy when I passed the letter to him. "That position was offered to Nisrita," he said sharply.

"When did you start having agents in the Towers, Timmon?" I asked lightly, shifting to keep my blue silk tunic from wrinkling in the humid air.

My husband's anger ignored both the heat and my attempt to diffuse the situation. "I have no agents. There are still those in Marsea's Towers who believe that being part of three miracles and producing two manuscripts by the age of eighteen is sufficient to earn an exalted position."

I chose to ignore his harsh tone. These displays whenever Nisrita's name entered our household were becoming tiresome. This was more than simple duty to Cortan. "Carlotta is a good candidate."

"Nisrita is better." he snapped. Then he continued almost to himself, "We know that Duke Griswold is still unpopular but no longer captive, Nisrita had enough support in Deyalorn to conterbalance the joint efforts of Cortan and Selvand, it is unlikely that Healer Carlotta heard this news on the field, so something stronger than the army's support came into play in shifting this decision in Deyalorn. Sophia, what do you know about how Healer Carlotta's lectures were recieved?"

"Stop it, Timmon." I hissed in anger. By the time I had his full attention, I calmed myself. "You cannot pretend that you know everything happening in Marsea through your agents. Consider how much our lives have changed this past year."

"Yes, of course," my husband responded gruffly after a pause. "You are probably right." He left me abruptly. I succumbed to the heat. There was no point in trying to console him. He had failed to create a position for Nisrita. My husband was not a man who liked to lose.

One of the supposed advantages of our strictly codified marriage was that as long as we both held to the agreed upon rules, we maintained an easy domestic peace. Nisrita was not in our arrangement. There were times when I rightly drew attention to my husband's wreckless behaviour surrounding her because it threatened our family income, or name. In a situation like this, I could not protest against his behaviour. Strictly speaking, he had done nothing wrong. 

%April 6
A few weeks later, we heard of the birth of Elena, a new duchess of Cortan, shortly followed by news of Cortan's continued failure to take the city of Escasaine. The rumour, as we heard it, was that given Cortan's multiple failed attempts at subdueing Niev, the duchy of Foligno would be given the honor to leading the next assay. It hurt Timmon, I knew, to see Cortan fail time after time. Timmon believed in Cortan's greatness and invincibility like a child believes his father's invincibility. He took these losses personally.

%May-June 6
At some point during the summer rains, Timmon and Lyo became lovers. At least, at the time, that is what I attributed the sudden and brief reappearance of my husband's ebulience. Around the same time, Carlotta announced her pregancy from Turina. Timmon released Nubo from his service, marrying him to a maid in an admiral's household. We continued to hear from Nisrita about once a month, about her family, her work, her doings in court. On the whole, the letters painted a happy picture of her life in Deyalorn. She made no mention of the position in Turina's Tower supposedly offered to her, though she spoke at length of the washing away of the bluffs overlooking the Tulsi River, just north of Deyalorn, the loss of lives and property to the people living below, and the role she played in the recovery efforts. To hear her tell the story of her life, Nisrita was learning to become a successful politician, leading a contented life. My father's letters contained darker tidings. He had no light to shed as to why Nisrita gave her position in Turina to Carlotta. Not only were the two women not close, Deyalorn's Tower viewed them as rivals on the issue of boosting a healer's power artificially. My father said that Nisrita would not speak of it. He confessed that since she became pregnant, Nisrita seemed to loose confidence in her work, the writing of her manuscript would have ground to a complete halt if he had not asked Malia to leave her other studies behind to help Nisrita. 

Timmon took my father's report news calmly. There were no outbursts in the house as I had feared, not acts that could lead to public shame, only a period of a eight or ten days when my husband did not spend the night at home. I prayed that this peace was due to Lyo's calming influence on him. There seemed to be something to the Liri belief that men need men's company, at least in Timmon's case. Lyo succeeded in reforming my husband at many points where I had failed. Very quickly, under the religious scholar's influence, Timmon turned his back on dice completely. My husband grew more like the Liri, even growing a beard as all Liri adult men do. He settled into his dimplomatic life style, spending more time with books and less time with his sword.

%July 6
I had gardeners from Cortan and Deyalorn brought in to showcase Marsean flowers: Cortan's during the hot relatively dry months of late summer and autumn, or Deyalorn during the mild winters. Liri garden's were pretty, and while traders told us that fountains had suddenly come into high demand in Marsea, I missed the flowers of home. The gardeners did a reasonable enough job of recreating Cortan that I spent my free time on the southen balcony, overlooking the garden that abutted the main road. This had the disadvantage that anyone entering the house could see me. Timmon put up a trellis around the balcony to avoid the scandal of a wife appearing unchaperoned in public, and let me have my pleasure. From my point of view, my perch had the advantage of knowing whenever courrier's arrived with news of home. I would leave my view of the garden, if my husband was home, to eagerly await the letters he would hand me. 

"There are ill tidings today," my husband said on one such occaison. "It would seem Captain Madriano has been cheated on."

"You are spying on Carlotta." I said more coldly than I should have.

"Not at all," Timmon handed me the letter. "I asked questions in Turina about her appointment. The men who served under me previously learned nothing, so I dropped the issue. An old friend wrote me unexpectedly. He gave me this news."

I glanced at the letter in the unfamiliar hand. The signature at the bottom revealed it to be from the black rider who had helped save Timmon's life over two years ago. I felt bad for Carlotta. She had grown ambitious over the years, as many healers had with Duke Griswold's discoveries. I knew that there was still a part of her that wanted a quiet, respectable family as well as her promising new career. This scandal would sit badly with her. It was unfortunate that her husband had to be so public about his disgrace. When I looked up, Timmon was lost in thought. I begrudged him his obsession with Nisrita. I had hoped that his love for Lyo would put an end to this nonsense.

"Let me try a hypothesis," Timmon said slowly. "Healer Carlotta has a lover, either from the clergy or the tower, knowing her nature."

It was disgraceful. "Ezaro, Timmon." I said shortly. "Now I beg you, stop prying into Carlotta's life."

"My apologies," he said sincerely, removing the contemplative look from his face, "this must be difficult news for you." 

This obsession had to stop, for the sake of our domestic peace. "Does Lyo ever get jealous of the duchess?" I asked.

"No." Timmon said in a matter of fact manner. "He sees no reason to be jealous of a woman. Ah, I see. We will not discuss matters concerning Nisrita any further, if that is your wish," he finished apologetically.

So far from home, we both clung to old friends for comfort. We not only both cared for Nisrita, our family owed her much. "That is not my wish." I could no more excise her from life than I could demand that my husband stop loving her. 

"Very well." Timmon returned to his correspondences, and I left the room.

I had been looking forward to the play we were to see the next evening for some time. A Marsean troupe of actors had come from Deyalorn, if they pleased the Liri public, my husband, the ambassador would become their patron. They were rumoured to be quite popular in Deyalorn. I hoped they would succeed here. I longed for a taste of home. 

When I went to dress, I found a silver chain holding a large white and green opal on my table. It was an expensive apology, certainly. My husband had even gone through the trouble of learning my choice of gown and finding a stone to complement it. That did not change the reliability of gesture. I had had two husbands and many lovers. I knew when men made empty promises.

That evening, Timmon sat, silent and distracted during the play. It irritated me that he still lingered over Nisrita, so soon after apologizing. I sat in our box, unable to enjoy the performance. Timmon sat beside me, his mind elsewhere. "How will you judge the quality of the actors, husband, if you do not watch the performance?" I asked sweetly.

"Hmm?" Timmon asked from the distant lands of his thought. "I leave that in your hands, Sophia. You are better equipped to judge such matters."

"You are still obsessing about her," I said calmly, but firmly.

"Her?" my husband asked, then returned from his fantasies to our box in Buzen. He continued sternly. "As a point of fact, I am not. And this is no place for a scene."

He had no right to accuse me of causing a scene after his behaviour. "This is not a scene," I informed him. "I am feeling unwell. I will make my way home." I took the silver chain off. "It is not an apology if you cannot keep your word for one night."

\vspace{.5cm} 

%Nov 7
Our domestic life became stony and formal after that. Letters from home addressed to Timmon made their way from his desk to my table after he had read them. Letters addressed to me I sent to him after I had read them. Letters from Nisrita, Timmon allowed me to open first. Winter came. The front garden bloomed with Marsean roses and lillies. We recieved news of Nisrita's pregnancy, the success of her manuscript, the birth of my nephew, belated news of Commander Dielo's marriage, and that Cortan's Tower had found that sporadic uses of the mushrooms boosted a gifted fighter's power without any ill effects, or bringing on the wasting sickness. We did not discuss any of these matters with the other. We had retreated to our seperate lives. Timmon spent few evenings at home. What time he did spend in the house he devoted to my children. I barely saw him beyond official functions.

%Jan-Feb 7
By spring, I learned that Selvand's Tower had started a routine of testing mushrooms on animals before distributing them to Marsea's gifted fighters. It was exorbitantly expensive, and the crown offered a healthy reward for the Tower that could find the cheapest method for allowing the nation's healers to use the mushrooms by the following year's campain. Nisrita's letters, by this point in time had grown short and vapid. They stayed cheerful, but she restricted herself to news of her children, news of my family, or Timmon's, gossip from court, or Malia's promise as a healer, and her efficiency as her assitant. She said nothing of herself or her work. My father's letters filled in the details that her silence outlined. Nisrita's mental state was deteriorating. With her most recent pregnancy, it was as if she had given up all her ambitions. He could not make her see that she was on the wrong side of history on the issue of the mushrooms. She spent her days with her sons, trying to teach them to the rudimentaries of healing. It was only due to a supreme force of will and bravery on Malia's part that Nisrita started writing her understanding of her husband's teaching methods. Healer Ezaro had lectured frequently on the topic, but he was ill so often that no one expected him to be able to undertake writing a coherent manuscript. There were few others in the Tower as well qualified to write down Duke Griswold's wisdom. My poor father sounded frustrated. It would be wearing, I imagined, for his gentle manner to grate, day after day against the hot headed young duchess. In my mind, he would sit calmly in her rooms in the Tower, trying to reason with her. As soon as the topic of Marsea's armies, or gifted fighters came up, she would accuse him, baselessly, of plotting against her. My father was right. Nisrita was on the wrong side of this battle. Even when I had last seen her, she had been obsessed by the idea that she could not be wrong about the mushrooms. In her mind, because it was associated to her husband, it must inherently be wrong. I found myself wishing that she shared some of my husband's reserved respect for Duke Griswold, and his visions for Marsea.

Eugenio ran excitedly onto the terrace one morning, as I sat drawing before the day became too hot. "Father has said that I am to sail on the Yori when she leaves the city next week. He has made all the arrangements. I am to be a Captain Isumu's page."

"Good morning, Eugenio. How are you this morning? Have you forgotten your manners?"

My son left the terrace, and reentered more appropriately. When he finished retelling his news, I rose. "I will have to speak to your step-father about this."

"Don't say no, mother, please?"

"It will be a discussion among adults about what is best for your future. Don't pout, Eugenio, you are far too old for that."

My ten year old son pulled himself together and faced me proudly. "Father says I am ready," He was tall and skinny, his face over darkened by the hot sun of these foreign lands. He had his father's features, and much of Timmon's authority in his eyes already. He would make a handsome man, if he did not get himself killed in a naval battle without healers first. 

"Come here, Eugenio." I took my sons thin hands in mine. "If your step-father says you are ready, then I believe him. But he is not the only one who has a say in your up bringing. We both want what is best for you." My son looked at me with the uncertainty of a young boy who knows his adventure is about to be stolen from him. How I have dreaded the coming of this day. "When do you meet with your Marsean tutors?"

"In an hour?" my son looked sheepish.

"Have you finished your exercises?" He shook his head. "Off you go then."

I followed him out of the terrace and down to my husband's office. I did not pause by my chamber to check my appearance. The days of sweetly requesting favours of Timmon were over. "Good morning, Timmon, I just heard of your arrangements for my son from his own mouth."

My husband looked up from his papers for long enough to say "Yes, Captain Isumu has been kind enough to take him on personally. It is a good opportunity for the boy."

I was furious at Timmon for not coming to me with this news himself. But a woman of Deyalorn's court knows how to control her anger. "My son is Marsean, he has no place on a Liri warship."

Timmon deigned to put his papers down and face me. "He is more Liri than you know, Sophia."

"By which you mean you have corrupted him," I said tightly, but evenly.

Timmon rubbed his hand along his hairy jaw, and controlled his temper. "What have you come to say, Sophia."

"I do not want my son going to war unprotected by healers. He is Marsean, if you need to train him, send him to Deyalorn."

"He is not going to war." my husband replied, then ended my audience as if I were a Liri ungifted wife. "You have said you piece. I have listened, I believe we are done here."

%"He is not your son. I should have been consulted."

%My husband sighed, releasing enough frustration to check his anger at my overstepping the bounds of our agreement. "When you have spent twenty %years training men and leading armies, Sophia, you will have an equal say in your son's training. We are done here."

I left defeated. By the cold rules of our marriage, I had been defeated when I entered the room. Timmon was in charge of Eugenio's militay training and education. I had thought, once, when our marriage was more amicable, that I would not be blindly handing my son's fate over to my new husband's hands.  In this vastly different state of affairs, I walked Eugenio to Buzen's quay a week after our argument and watched him board a Liri dromon. Timmon had not lied. The ship was not going to war. It would perform a routine patrol the Sowene and part of the coast line and return in two months. That did not stop my mother's heart from quaking. Eugenio was only ten.

%Feb-March 7
I spent the two hot months of Eugenio's absense in stumbling from one anxiety to another. I recieved word that Carlotta had given birth to Cesara. In the mean while, I learned that I was pregnant. Timmon took the news so coolly that I wondered if he would not accept this child as his own. I turned to my lover for comfort. Minoru assumed the child was Timmon's and turned his back on me, suddenly self concious of touching another man's wife. The Liri have such awkward views towards their women. I argued with him, foolishily, instead of letting him go his way. I was lonely and worried. I was frustrated with my cold and rule bound marriage. I was furious at being tossed aside at my lover's convenience. I was unhappy in this society where my gift brought me no advantage.

Three days later, my husband asked me to see him in his office. He sent me a formal written request via my maid. I found him seated behind his desk, bearded and barefoot, dressed in a silk tunic, looking more like a Liri naval admiral than a Marsean ambassador. "You wished to see me?" I asked, mildly off put by the depth of his conversion to this new land's customs.

"Did you fall out with Minoru?" He demanded.

"Have you been watching me as well?" I curled my lip slightly into a sardonic smile. He had no right to this information.

My husband's tone did not budge. "Lyo told me that your name is being dishonored at the university. Do you have anything to tell me?"

I sat down delicately, shocked by Minoru's betrayal. I had always thought that a danger to our family would come from Timmon's lack of self control, not from my discrete activities. "He wished to leave me three days ago." I admitted. "I was upset, I lost my temper. He called me a Mersean whore. I called him a long list of unbecoming names in response. I have not spoken to him since."

The muscles under my husband's beard clenched then loosened. "I am going to the university this evening to accept his apology. If you wish for him to apologize to you as well, accompany me." He dismissed me to think about my choice.

"I am sorry," I said softly as I left. Timmon was absorbed in his own thoughts. I do not know if he ignored me, or simply did not hear.

In the end, I decided I could not bring myself to be displayed in front of all of Minoru's colleagues or hear the litagy of things he had said about me to them. Instead of accompanying my husband to the university, I lit a lamp and waited on the balcony for his return, hoping that he would send word to me of the outcome, rather than going straight to Lyo's arms after his ordeal. The garden below me smelled of foreign Liri flowers that flourished in the impossible heat of spring. The night grew late, the number of people on the street dwindled significantly, the sounds of human conversation replaced by that of crickets and frogs. Eventually, I saw two men enter the gate to walk down our path. I left my vigil of shame to hear what my husband had to say. He met me in the entry hall. He would not come into the house. We talked in the dim light of the lamp I held. His voice was tense, with fear, or disgust, or something else, I could not see his face well enough to judge. "There will be a duel, tomorrow morning, in the square before the palace. You may come if you must."

"A duel," I gasped.

"It is a barbaric custom in these lands, but one that will protect us in the end. The winner has his name fully restored."

"Timmon.." I said, and stopped, robbed of words by the unthinkable situation.

"I will not fail you." He smiled tightly and left me for Lyo.

I slept little that night, tossing and turning in my bed to the barbs of guilt, shame and fear. My illicit actions had brought shame on our house. My husband may not object to my relations, but to the Liri company I kept, I would be shunned as a fallen woman. If I had just let Minorou go peacefully, if my pride at being scorned had not interefered, this would not have happened. Dawn found my praying for both men's safety. Timmon was a warrior, Minoru a scholar. My husband would likely win, but took on significant risk to his person. What had posessed Minoru to accept this challenge. We had argued as lovers. Was that enough reason to risk his life. I hated this brutal custom and the country that birthed it. I felt no love for Minoru anymore, but I not want his blood on my hands. This would never have happend in the civilize court of Deyalorn.

When the sun had risen, I went by covered palanquin to the square. I could not bring myself to walk the ten minutes in full view of Buzen's nobility. At the same time, Timmon would put himself at risk for my sake. I could not stay away. I learned, from the careful questioning by my maids, that Timmon had demanded not only an apology for soiling my name, but a denial that there had been an affair at all. Minoru refused to be known as a liar. I would have thought that no man would risk the wrath of a cuckolded ambassador, that Timmon could have used his influence to make Minoru's life rather difficult. If the Liri thought in such reasonable ways, Minoru would not have betrayed me in the first place. It was easier, it seemed, for Timmon to insult and enrage Minoru until he felt he had no recourse to save his honor but to duel. It was uncivilized and foolish. No Marsean would throw his life away like this.

Minoru had trained, I knew, when he was young with the curved Liri sword. Almost all Liri nobility do so at some point in their life. It was an awkward weapon for my husband, who lived his life by the straight thin Marsean blade. The two healers from the embassy stood by to help my husband if it came to that. I prayed it would not. I found myself wishing I were not pregnant that I may add my skill to theirs. The two men squared off, both unarmoured by law.  The man I had not loved, and the man I no longer loved. It was a terrible choice. I would lose one of them today. I doubted I would have the courage to leave the shelter of my palanquin to face either of the men I had driven to this. 

The justice from the palace ordered the men to ready themselves. Then he blew sharply on a conch shell, and the duel started in the already hot, dusty square. Timmon and Minoru circled each other twice before Minoru attacked. Timmon deflected blows to his head and chest so they landed lightly on his arms and shoulders, pushing Minoru back with every blow. It was a similar tactic to what I had seen Eugenio use against Minoru's nephew. Timmon seemed impervious to his wounds. I closed my eyes when I saw blood on Timmon's shirt. I heard the skuffle of the fighting men and the ring of blade against blade, but I could not watch. Shortly, I heard the conch blow again. I opened my eyes to see the two healers bending over Minoru's still form on the ground, while the Dean of the university and one of Timmon's ambassadorial staff stepped forward to support my teertering husband. 

I told my palanquin to take me home. I could neither watch the crowd's voyeuristic jubilation, nor face my husband in the immediate aftermath of what I had forced him to do. The image of Minoru's still form floated behind my eyes. Timmon had sent his healers to his opponent rather than tend to his own wounds. Minoru, had he had the chance, would certainly have left me a widow. I resumed my solitary vigil of shame on the balcony overlooking the path up to the house as I had the previous night. I was glad the terrace hid my face, I did not have the courage to show it to the world, but I wanted to see my husband safely home. In less than half an hour, I saw an open sedan arrive carrying my husband, still accompanied by the Dean, and now also Lyo. After a short time, enough to make Timmon comfortable in the house, I saw the Dean leave. I waited on my balcony, lacking the temerity to appear before a man who loved Timmon. After a further while, I saw Lyo leave as well. One of my husband's men called me to his office. My husband wished to see me. 

"Thank you," I said as I entered his presence.

"He is stable and under the care of the healers in the palace for the moment. He should survive." My husband informed me from his reclined position on the sofa. Both his arms were bandaged, as was his left leg. I had not asked for a report on his rival. I had thanked him for what he had done for me.

"What of your wounds?" I asked quietly. There is a good reason for the Tower's reccomendation against friends and siblings healing each other. For all my husband's nonchalance, the sight of his blood disturbed me. "Should I tend to you?"

"You are pregnant," he snapped. Seeing the red on my husband's bandages, recalling the scene from the morning, I no longer wanted Minoru's seed. Timmon was more conservative. There was little difference for him between the seed I carried and the living child he would adopt. "My wounds are minor. Our healers will see to me tomorrow."

My husband's lack of concern for himself, the sight of his blood, the memory of Minoru's body in the square, the knowledge of how close I came to shaming my family wore on me. "Why?" was all I could say calmly.

"Why did I heal him, or why did I leave him living?"

"Both."

"What he did was not worth a life. In Marsea, I would have him tried for libel and cut off his tongue, possibly sent him to the galleys. As for healing him, he would have died of his wounds in a few days without help. I could not allow that. I could kill him swiftly, or heal him. This has the advantage of showing of Marsea's greatness." He shifted his weight and winced. "I am afraid we will both have to be a bit more careful with our conduct for a time."

"My noble Marsean husband," I smiled and leant towards him. I had put my children as risk. I wanted his forgiveness.

"Stop simpering, Sophia," Timmon said ordered firmly, but not cruelly. "You are not a wife begging her husband's forgiveness for dishonoring him. You made a mistake. I have certainly made my share. I protected our family, as we had agreed." 

I removed my smile and sat up straight. "You are right, Timmon." My instincts were completely wrong with him. We were not man and wife, but two renegades seeking protection from each other. "Is there anything you need at the moment?" I spent the rest of the morning tending to my husband's needs until Lyo arrived in the afternoon to relieve me. 

\vspace{.5cm} 

Warmth crept over the threshold of our marriage, but came no closer. It took Timmon the better part of a week to be able to keep his appointments again. I cancelled my engagements during that period, as a dutiful wife should, and stayed in the house. Even in a small palace, it is impossible to completely avoid another person for days on end when neither have anything to do. We fell back into the habit of talking.

"Stop pacing, Timmon. Your leg needs to heal," I told him on the terrace the day news came of Foligno's disasterous retreat in Niev. His agitation surprised me. He had taken news of Cortan's failure against Niev better last year.

The stubborn man sat down suddenly on the stone bench in the blazing sun. I left my drawing and went to him. "You are disturbed by Marsea's retreat" I ventured. It had been a long time since we had discussed news from home. 

"No." Timmon fiddled with his cane impatiently. Then he threw it down on the ground, adding "But I may as well be." I picked up the stick calmly and placed it on the bench between us.

"The crown will give command back to Cortan after this failure."

"Who will fail again unless they learn to change their tactics. I have said as much, but not one listens to a general gone soft from ambassadorial duties." He looked distantly beyond the garden to the north, his eyes focused on the palm trees on the horizon. It was the old thorn. My husband had earned the title that had been his youth's ambition, only to leave the army for a diplomatic post. I thought Lyo had helped him release the sense that he betrayed the men he had trained.

"You would have made a good general, Timmon." I cooed.

He gave me a broad smile, suddenly shaking of his gloom. "Another in the long list that have failed to conquor Niev? No, I am doing Marsea more good where I am. Now, if you do not mind, I will go inside." He patted my arm affectionately and limped into the shade of the house, while I marvelled at the transformation Lyo had engendered in my husband.

My son arrived three weeks later full of sailor's songs and tales of storms along the southern coast. He boasted to his step-father how, a few weeks ago, he was the first to see the raiding vessels on the eastern horizon, even though the Yori did not have the highest mast in the fleet. Then he terrified me with his recollection of the ensuing battle, where the raiders rammed the Yori's hull, how they took on water for hours while huge black men with small eyes poured onto their ship, and how the Ai came to their rescue. I sat quietly and watched Timmon proudly encourage my son's actions on the sinking ship. My husband may have taught my son to swim, but there had been no healers on that ship. But for the Preserver's grace, I was listening to the retelling of a mother's nightmare. Eventually, my husband told Eugenio to stop scaring his mother, and sent my proud sailor off to bed. 

"He has a good eye. I should see how he takes to archery," my husband chuckled. "Perhaps he'll be a black rider."

"You knew of this battle already, Timmon," I said coldly.

"Yes, before it occurred infact," my husband explained calmly. "I heard news of problems with raiders in the area the Yori patrolled on that day on the terrace you were so kind as to distract me from my worries."

I am afraid I lost my calm in the face of my husband's callousness. He had not consulted me when sending my son to sea, he had not informed when he had been in danger. "Need I remind you that I am his mother, that I am a healer, and we are Marsean. You should have told me. I will not be treated like a Liri wife."

My husband stood wearily. "I do no wish to argue again. You were already anxious when I first heard the news. I saw no need to agitate you further. When I heard that the crew of the Yori had survived, I saw no need to cause you worry in retrospect. Next time, we will not be embroiled in this constant bickering, and I will keep you abreast of all developments." I fumed while my husband went to his office.

To calm my anger, I went to Eugenio's room and helped him to bed as I used to when he was young. Eugenio happily told me that Captain Isumu thought him to be braver than any Liri boy his age, due to his Marsean training. He told me that he wanted to be serve in Marsea's army like his father and step father. He promised he would never complain about Timmon's trainings again, he liked being stronger and braver than the other pages he had served with. He told me again that he had missed me and was glad to be home. He fell asleep with his head in my lap as he used to when he was a small boy. 

I could not question my husband's devotion to this family. I could not question my son's admiration for the man, nor his need to make his name as a warrior. Eugenio's father had left his son precious little when he departed. Timmon had plans for his greatness. By agreement, I gave Timmon control over my son's education. If I did not like being kept from knowing about his welfare, I had no option but to restore our domestic peace.

\vspace{.5cm} 

%July 7

By the time the summer rains stopped, we learned that Deyalorn's Tower, specifically, Duke Griswold and Ezaro discovered that by creating the gift boosting doses out of a mixutre of at least one hundred mushrooms, the poisonous effect of the one poisonous fruit in that mix would not be felt by the gifted. Deyalorn's Tower recieved the promised reward. As Ezaro's health was still too weak to permit him to take any long tours of lectures, the tower pressured the Crown to release Duke Griswold from his confinement within the walls of the captial city. At a similar time, Allepo's tower devised a way of quickly and efficiently farming the mushrooms. The Tower's head was recognized for this innovation, though not rewarded. Between these two innovations, I could only imagine Nisrita's dismay. I hoped my father, or his studen Malia could provide some comfort. 

I decided that I would not travel to Marsea to deliver my child. Timmon sent for healers from Turina to care for me when the time came. Lyo became a larger part of my husband's life. He all but left his position at the university to take on the role of a cultural and religious liason to the Marsean embassy. He visited the house occaisionally in an official capacity. We dined with him regularly as we did with all of the ambassador's key Marsean and Liri advisors. For my part, I resigned myself to the fate that I would have to turn to men from lower stations in life for comfort. It was a bitter pill to swallow, a tactic used by older women, well past the prime of their beauty, who could not hope to attract men by anything other than their wealth and power. I was no longer a blushing maid, but at twenty-seven, I was, with care, still attractive. After the incident with Minoru, the beautiful, intellegent men of Buzen's University were out of my reach.

Timmon bounded down the already dusty path of our front garden late one morning, where Firi chased her per cat behind the sage bushes. He scooped up his daughter and threw her high in the air. "Congradulate me, Sophia. Today will go down as a turning point in Marsean history, and your husband stands in the center of it."

I took my bewildered daughter from my husband's arms, and pointed out where her precious cat had escaped to in the mayhem. "This is too public, Timmon, come inside." I wanted to share in his joy, but our Marsean habits were too extroverted, he would shock any Liri citizen passing our house on the main road.

Timmon had barely crossed the threshold of the veranda into the house when he grabbed my waist in a manner I feared might lead me to be flung in the air as well. He controlled himself at the last minute and spun me around in the narrow enterance hall. "Not one small word of congradulations for your husband, dear Sophia?"

In the privacy of my house, I laughed, overwhelmed by his exuberance. "I am certain there will many laudatory words for my accomplished Marsean ambassador, once I know what you have done."

He practically danced into his office to seat himself in his chair. "Without shedding a single drop of blood, I have accquired from Lir the one thing that Duke Griswold wanted," he said over the lemon water served on these hot dusty days. 

Even Timmon could not convince King Ayum to peacefully recognize the Marsean crown as his sovereign, so I asked "What did Duke Griswold want?"

"Rice, Sophia. Not simply for trade, but to grow. A delegation of farmers and engineers leave by the month's end to show Cortan how to cultivate rice."

"But Cortan is so arid. The rice paddies I have seen are flooded." 

"Lir has better irrigation techniques than Marsea has ever dreamed of. The scholars and engineers at the university believe that the underground streams and aquafers surrounding Cortan's castle can be used to maintain a rice crop. Alleppo is just as hot, but less dry. If this doesn't work in Cortan, Duke Ergino will sell the technology to them. It doesn't matter. Marsea will have rice. This will bring an end to famine." Timmon drained his drink and put down the crystal. "I recieved the letter from Duke Ergino giving his, and the crown's, approvals to the terms proposed by Lir this morning. King Ayum gave me the names of the envoys two hours ago. You are the first person not on that select list I have told today."

He sat grinning like a schoolboy in his office, unable to sit still in his excitement. "My heartfelt congradulations, Timmon." I said sincerely. "And I am honored by your choice, though I do not understand what I did to deserve it."

My husband could not remain sitting anymore. He rose and pranced circles around the couch I occupied. "This is a Marsean victory, Sophia. It deserves to be celebrated with Marseans, not Liri. There is no Marsean in Buzen I would rather share my joy with. I have lived in this country for nearly three years now. Finally, I have shown that my time here has not simply been an excersize in idle posturing and cultural exchange. At last I have done something to earn my keep."

The qualifications he placed on his complement did not escape my notice. "His excellency did not give Marsea this gift for free. What are the terms?"

Timmon waved his hands abstractly at the question. "A pittance. That is the glory of it. A paltry sum of gold, prioritized trade with Turina's mines, and the right for a limited number of Liri to settle and marry in Marsea every year."

I was horrified. Liri men marrying Marsea's gifted women. It was unthinkable. "Marsea does not permit settlers. It dilutes the gift."

Timmon shrugged off my protest. "Both Marsea and Lir seem happy with this trade. One cannot hope for better in matters like this." 

"I cannot see how this benefit's Marsea, Timmon. Liri men will have gifted grandchildren, who will return to Lir after training in Marsea's Towers, and give Lir the gift."

Timmon sat down beside me, pleading. "Don't be like the others, Sophia. Times are changing for Marsea, there is a greater vision here. The Liri must settle and swear oaths to Marsean Dukes. There is no provision for Marsean citizens to return to Lir. And even if there is," he added when I started to object, "when there is, even, this can only be good for both Marsea and Lir. A war between two such great powers will be expensive and hard fought. I do not know that Marsea will prevail. If both powers have the gift, then war will be less likely."

I was shocked by my husband's words. "This is heresy, possibly treachy, Timmon." I said sternly. "I respect your right to worship Lir's pantheon, as we agreed, but do not ask me to love you for betraying Marsea to them." To doubt Marsea's military superiority was to doubt the superiority of Marsea's gods. After the drummed up fever in Marsea against Liri gods several years ago, the scandal this threatened to generate in our delicate situation was hard to bear.

Timmon's joy vanished. His bearded jaw clenched. "I have never asked you to love me, Sophia. I am sorry for coming to you with this. I will not trouble you any further today." 

I took hold of his arm as he rose. I did not want to fight any more. "You misunderstand me. Do not leave in anger. People are already whispering that the Marsean ambassador has become Liri. I only fear for the backlash we will face if this new treaty turns out to be unpopular at home." The bloodshed following Minoru's betrayal had made me fearful for my family. Every whisper seemed to have the potential for violence behind it.

"I have already foregone my habit of visiting Rishiki's temple regularly." Timmon said tightly. "I have declined invitations to all but the most important of religious holdays since the scandal you nearly brought upon our heads. What more do you want of me?" He left me, presumably, for Lyo.

\vspace{.5cm} 

%Aug-Sept 7
The next month brought news of the birth of Duke Emile of Cortan, the following, news of the new Barron of Romino's, marriage to a duchess of Selvand, and Carlotta's arrival in Buzen. She arrived with her student, Amile, a clever, patient girl of seventeen, extremely skilled, according to Carlotta, the bastard daughter of a gifted Allepan barron and a goat herder's wife. The arrival of a friend who brought me news and stories of home filled me with more comfort than I can express. My marriage with Timmon still quavered on the brink of animosity. By the time Carlotta arrived, I found myself frequently nervous and weepy. Eugenio went on another patrol with Captain Isumu. My husband and I kept our relationship cordial enough for him to feel comfortable keeping me informed of what information he had regarding my son's movements. The information did not serve to calm my nerves, but it eased tensions between myself and my husband. 

Carlotta and Timmon have always had a tepid relationship at best. For the nearly three months he hosted her at our house, he made himself scarce. Neither of us wanted to disturb the fragile situation that passed for our domestic peace. Left to my own devices, I introduced Carlotta, now known as "The Mother's Balm" for her work with early born children to Liri nobelwomen, invited priests from various temples to the house for Carlotta, born to a respected clerical family, to engage in religious discussions in, even introduced her to Lyo one day for this very purpose. I took her, and the terrified Amile on a three day trip in a floating house down the Sowene, and taught her the art of cooking rice to suit a Marsean palate. Carlotta commented on how Liri I had become, in my dress, my manner and the presentation of my house. I denied her claim, explaining to her the needs of adjusting to this stiffling culture for the sake of my husband's post and my family. She was, as ever, a sympathetic friend, at a time when I was sorely in need of sympathy. 

For her part, Carlotta seemed happy enough in her situation. Cesara was healthy. She had patched over her troubles with her husband. Her new position was not only prestigious, but well paying. Her husband supported her work because of the influence it gave him in Duke Erfat's court. She did not enjoy her teaching duties as much as she enjoyed healing, but it was nice for her to have a position in the Tower to fall back on when she coud not practise. She laughed at the simplicity of her childhood ambitions, before she got her healer's rank. She had disliked torturing for General Madriano. the conventional path of marriage provided and easy and pleasant escape. But when she became a healer, she explained, and saw the excitement Nisrita's first miracle had caused in Cortan's Tower, a new world of possibilities opened for her. 

Eventually, I could not contain my curiosity any longer. I took my friend away from Buzen for a day's liesurely float in a gondola along the Sowene. The gentle rocking of the small craft as I lay cradled on soft down pillows eased the pains of my now heavy form. Carlotta sat with her toes dipped in the cool water, just beyond the shade I lay in, her face turned up to the pleasant winter sun. 

"There's something I never understood, Carlotta. What happened between you and Nisrita in Deyalorn that made you leave for Turina?"

She turned her head lazily to face me, "What makes you think Nisrita had anything to do with my position?"

"Come now. My husband is hideously in love with the duchess. I think he is paying half the Liri embassy to keep track of her doings. He says that the position was initially offered to her."

Carlotta took a pillow from me and lay back, luxuriating in the sunlight. She had no intension of letting this conversation interefere with her day's pleasure. "He is right. The offer came while she was still pregnant with Elena. Griswold refused her permission to go. It was late in her pregnancy, she did not have the energy to fight him. Nor did she have the support in the Tower. Your father, I believe, was the only Master even willing to consider backing her against Duke Griswold. She must have woven quite a spell on him, Sophia, for the old respectable man to be willing to go so far for her."

"My father can be a very sentimental man about some things, Carlotta. But why did she offer the position to you? You were hardly close when I last saw you."

"We have made a sort of peace, I suppose." Carlotta sighed uncomfortably. I heard the splash of her feet kicking in the water. "It was an unhappy time for me, when I returned from lecturing on my work with Ezaro. His health was already clearly deteriorating. Being with him started being," Carlotta searched for the right word "well, a bit frightening, I suppose. Then, within a few weeks of returning to Deyalorn, Duke Griswold found out about our relationship. I don't know how, but he was furious." Carlotta shivered in the sunlight. "I wouldn't like to face that fury again. He threatened to have me stationed in a remote tower for having an affair with his favourite student. I think he feared further scandal to his barely recovering name. I was, of course, devastated and terrified. I turned to the only person I knew who could possibly help me."

I found myself a bit stunned by the story. I could neither believe that Nisrita was so naive as to help Carlotta simply because she had been misused by Duke Griswold, or Carlotta so calloused to prey on her childhood friend's weakness. "What was her price?"

"Nothing, really," Carlotta said lightly. The unpleasant part of her story over, she threw an arm over her eyes to shade them from the sun and let the boat rock her into a state of drowsy stupor. "She made me promise not to advocate the use of the mushrooms for any purpose. That was, still is, I suppose, her one cause. I agreed. She had not asked me, after all, not to use them if the opportunity came." 

"Will you keep your promise?"

"I suppose." She said lazily. "Given the winds in Marsea at this point, my voice would just be one in a throng. There is no need to antagonize the duchess."

I let Carlotta doze. We had all changed so much in these last three years. My husband, Carlotta, Nisrita, myself, even Marsea herself. It was the natural order of things, but it made me unaccountably sad.

\vspace{.5cm} 
 % 8
Carlotta delivered me of a healthy boy in the middle of winter. She and Amile had to cut him out, as he stubbornly remained facing the wrong direction. My devoted husband dozed in a chair by my bed for that first deliriously painful night. When I saw him again that evening, after Carlotta and Amile had healed me twice more, and I had a chance to sleep, he entered with a letter in is hand.  

"How are you feeling, Sophia?" he asked, gingerly taking my hand in his.

"Better than this morning, thank you. You need not carry that stricken look on your face, husband. You have called the Mother's Balm to my bedside. Our son is healthy. No ill will befall me." 

Timmon laughed. "I remember jealously saying the same under similar circumstances to a lover once." He sighed at the memory. "You are right of course, I have seen many men in agony in battle, seeing a woman suffer the same is ... harder."

I smiled and let the sentiment lay softly around us. We were still a far distance from the easy friendship we once shared. Moments of true tenderness were skittish and delicate. 

"Is that news from home?" I asked, when the silence grew heavy.

"Yes," he passed me the letter. "King Gustav's turns eighteen in six months. We are, of course, invited to the coronation."

It was wonderful news. It was a chance to be in Marsea again. We would pass through Cortan, sleep in that tiny little house Timmon and I called home when we first wed, see friends in the Tower and the military we have not seen or heard from in years. Perhaps we could leave a bit early, and linger a week or two there before continuing to the capital. How I longed to see my father and brothers again. We would both go to the house Timmon grew up in, let him say his final goodbyes to his father at his grave, an act I know he longed to do. Eugenio, I was certain still remembered the young dukes. My son would be excited to see them again. I wondered if they remembered him.

My thoughts must have shown on my face. Timmon nervously cleared his throat. "I have several negotiations in progress here. I will do what I can to finish them, or hand them off to others of my staff, but some diplomatic matters cannot be rushed or trusted to underlings. I do not think I can stay more than a month in Deyalorn, or linger in Cortan on our trip there. You, on the other hand, are welcome to stay in either place for as long as you wish."

My husband's words filled me with an irrational fear. "You wish to be rid of me?"

"Not in the least." The pure shock in his voice comforted me. "Sophia," he said more tenderly, "you have not been entirely happy here. Some time in Marsea will do you good. I know you have missed the towers."

"Yet you would find yourself unhappy in your homeland." This country had become such a part of my husband. It sat so ill with me. Were were fated, I wondered, to this miserable existence, one of us at peace in a country that slowly drove the other of us mad. 

"Do not adorn my words with your suspicions, Sophia." My husband scolded, loosing his patience. "I meant what I said. My work beckons."

I squeezed his hand to keep him from leaving. I wanted to return to that moment of tenderness I had blown away so carelessly. Timmon sat awkwardly silent by my bed. "How is our son?" he asked, attempting to find neutral ground.

I gave him the only peace offering I had. "I had thought of naming him Desmond, Timmon. If that would please you."

There was a long pause, then Timmon lifted my hand gently off the bed and put it to his bearded lips. "Thank you, Sophia" he said, with a heavy voice.

\vspace{.5cm} 

We spent two nights in our old house in Cortan, enroute to Deyalorn. It felt dingy and drab compared to our abode in Buzen. Timmon gave me permission to have it built up as I saw fit. He gave me a generous budget, a peace offereing of sorts, I suppose. We lived our lives from peace offering to peace offering, trying to mend the argements that break out, almost out of habit, in between. I spent most of my day in Cortan in the Tower. Timmon was right. I had missed Marsea's Towers more that I admitted to myself. I walked among the withered grasses of Cortan's Towers, filling the hems of my skirt with dried seed and burrs. I sat with my friends and colleagues in the easy, equal company of men and women. We discussed Lir, rice, the differences between Liri and Marsean art, and the outcomes of this last campaing in Niev, led again by Cortan, which finally made significant gains. We mingled freely and greeted each other warmly, with no false pretense of inadequacy applied to my face for my male colleagues. 

Timmon spent his day in the Duke's court and inspecting the rice paddies. We dined at the castle with the small Liri community that had already settled in Cortan. Even that was a pleasure. I was a Marsean among my own. I greeted the Liri men respectfully, according to their fashion, but did not shy from giving my hand to soldiers of lower rank in Cortan's army. Nor did I shy from taking my proper part in the conversation at the Duke's table. I was the gifted wife of the Marsean ambassador. The court valued my opinion. I did not have to don a false veil of modesty and silence to keep from bruising the egos of the ungifted men around me. Timmon looked as if he were spinning a fantasy in his head during dinner. I wondered if he were thinking of a life in Cortan after his ambassadorial duties ended, where we could both find happiness in this land. I hoped I was right. It would give us peace again. After dinner, I suggested to the Liri men that they take my husband to the small temple they had built. Timmon hesitated, but I insisted. I wanted our future to satisfy both of us. It was my peace offering to him.

"It turns out Commander Dielo and several other units of Black Riders across the country trained over the last two years against Niev's ambush techniques.In Cortan's hand's the black rider's are without parallel," Timmon said late the next morning. We had been on the road for some hours. He did not mention the moisture on my cheeks when he pulled his horse up to match my gait. Leaving Cortan had been harder than I had expected.

"Not bad advice from an old general gone soft on ambassadorial duties then?" I asked with a smirk.

Timmon gave me a beautiful broad smile, handing me a handkerchief. "No, Sophia. Not at all." 

"The Liri community is causing less trouble than I feared, Timmon. I underestimated you." 
 
My husband waived away the even the hint of appology. "Not entirely. It is still a very unpopular in most duchies, but Cortan, Deyalorn and Selvand are welcoming them. The rest will come around with time."

"Selvand?" I asked.

"Boats, Sophia. The very thing that brought Duke Griswold back from exile."

I looked at my husband. We had been travelling in Marsea for three weeks now. He had shaved for the journey, to keep toungues from wagging unnecessarily, but that was the only change he had undergone since leaving home. He had not returned to the dark man teetering on the brink of his own distruction I had married and lived with for a year in Cortan. He was confident, proud to see he fruits of his service so far from home. I told myself that it must mean that there was hope for my family's future in Marsea. "I think I would like to stay here a while." I informed my husband. "Perhaps spend the Day of Unions in Deyalorn, then winter in Cortan. I will start the improvements to the house while I am here." With some time apart to ease my loneliness and a pause in our hostilities, perhaps we could even be friends again.

\vpsace{.5cm}

We slipped, of course, in our precariously balanced marriage. It seemed the closer we rode to Deyalorn, the harder it became to keep our fragile peace. We were both happy be returning to the lands of our birth. We both seemed happy to spend time in Marsea. We both feared the approaching meeting with the duchess. With little else to distract us, we made peace offerings to each other through our children. As we entered the duchy of Deyalorn, Timmon scoured orchards for the ripest peaches to delight Firi. Neither of us could believe that our daughter had grown up on mangoes and lychees but had never tasted a perfect pink and orange peach. To ensure that the mistake was not repeated with our son, my husband pealed the soft skin off an overripe fruit, and gave it to Desmond to suck at messily, earning a reprimand from the child's nurse. For my part, I insisted that Eugenio continue his training throughout the journey, as his step-father demanded, even when the road wore the boy out in the evenings. I comforted my son when he claimed he felt inadequate and weak compared to other Marsean boys his age, and sent him back religiously to work with his step-father. I agreed that Eugenio would return to Buzen with Timmon, while I would keep Firi and Desmond in Marsea for as long as I wished.

The first sight of the high bluffs above the captal city was as fullfilling as I had dreamed it would be. We arrived in the middle of summer, with the gardens in full bloom, the smell of roses and lavender everywhere I turned. My gardeners had tired. They made a true effort to capture a sense of home in Buzen. That small patch of land in front of my house could never hope to compare to the colorful glory of flowers that hang off the palisade overlooking the Tulsi River. The city had already started adorning itself in anticipation of the Coronation to come. Houses all along the main road up the cliffs were freshly whitewashed, their gardens manicured and perfect. Red and gold banners hung at every turn in the winding road. In short, my city gleamed. There would be long garlands of gold chrysanthemums strung along the street on the day of the coronation. My husband and I would certainly be a part of the procession following the new king down from the palace to the ampitheatre where he would address the city. I was proud to be home.

This time, I was not visiting Deyalorn as the wife of a rising commander. The very successful ambassador and his wife would stay at the palace. Timmon and I agreed that we would see my father first. The house was on our way up to the palace. I wanted my father to see what his son-in-law had made of himself.

I gasped when I entered the parlour where my father sat in the sun. I had left a sturdy, upright man waving at his window nearly four years ago. I returned to a gaunt, frail looking old man dozing in his chair. When he stood, he stooped slightly, leaning lightly on a cane for support. There had been an illness I had not been informed of, certainly. My father wrote so regularly, why had he not said anything?

"Do not stare, Sophia." He said mildly. "Old age happens to everyone. I am nearing the completion of my seventh decade."

"I am sorry, father. I forget myself." I tore my eyes of his frail form and greeted him as appropriate. 

He turned from Timmon, to me, to my children. "You one must be Firi," he said. My daughter shyly greeted her grandfather. "Come here, child. I have not seen you since you were younger than your brother. You look like your mother. You will be very pretty some day, and hopefully gifted too." My three year old lost her nerve and ran back to her nurse. The old man chuckled and picked up his grandson. "This must be Desmond." He looked at Timmon "The Barron was a good man, Timmon. Deyalorn is poorer for his loss." 

"Thank you, father," Timmon replied, as my father inspected our son.

"This one, I think, does not look like either of you." 

"He is young, yet, father," I said, nervously taking the infant away from the shrewd old man. Minoru's betrayal had shaken my confidence in my decietful life. "It is not alway easy to tell yet."

With the children chased off to the garden, we three sat and filled in the details of our lives left out in our correspondences. My father had been ill early this spring. My heart clutched at the hope that what I saw was only the aftermath of a bad illness, and not the beginning of his slipping from the Preserver's guard. My brothers were all well. They had quarreled with my father recently, and each had found excuses not be present in the house to greet me. My father assured me they would all come to the palace to see me, however. I had missed the birth of a niece two months ago. The family had not written, as I was on the road.

Eventually, I left to help with dinner. My father's situation in Deyalorn was better than where Timmon and I had started off our marriage. What I remembered as comfort and luxury during my last visit now seemed a hardship compared to the life I had become accustomed to. None the less, I was temporarily the woman of this house. I reaccquainted myself with the kitchen and set to work.

When my father and husband came to eat, the old man was explaining to the younger the intricacies of the politics surrounding the debate over allowing healers to boost their power. Listening to him talk, it sounded like Nisrita had some valid concerns, mostly surrounding the poor long term health effects that the mushrooms have had on Healer Ezaro's health. The kingdom's Tower would not wait to study the long term health effects of the fungus. They wished to reap the benefits now. In fact, given the success of the last campaign in Niev, which the Towers attribute in part to the mushrooms, the Towers were now focused on a way to cure the wasting sickness. In the meanwhile, many Towers, including Deyalorn, have set up elaborate systems to allow healer's access to the power boosting doses on a fortnightly basis.

"What is the Duchess of Cortan's position on this, father?" Timmon asked to my dismay. I had hoped to hold off discussing her for longer. 

"Healer Nisrita thinks she knows how to cure the wasting disease, or at least, prevent it. Neither I, nor Healer Malia, can get her to reveal her secret. All I know is that she stumbled upon it while caring for Healer Ezaro."

"How is Healer Ezaro, father?" I asked, wanting to change the subject.

My father shook his head darkly. "The boy is not well. There is something to Healer Nisrita's fears on the mushrooms. Mark my words. I pray I do not live to see the day when the Tower discovers they have been wrong." I squeezed my father's hand. I did not wish to think of his not living at the moment. "You will visit Healer Nisrita, of course?" he continued.

"Of course, father," I stumbled. "How could we not?"

"She is not at the Tower, the Duke and Duchess have a house on the bluffs. She is not well, Carlotta. You have heard that she is pregnant again?" I nodded. I had heard that rumour in Cortan. "This pregnancy has sapped what little of her will she had left after Duke Emile was born. Her physical health is also failing her. Your visit may do her good, but go gentle on her. Do not mention her husband or the Tower, if you can. Healer Malia lives with her now, to see to her needs ... and safety."

The news took some swallowing. We had suspected that Nisrita was not well, but to hear my father speak of it filled me with a fear that letters could not convey. Still it was better to be informed ahead of time, than to be shocked at the first visit. I glanced at my husband. To my relief, he showed no outward sign of disturbance. "Enough of dark tidings," my father said. "Did you know that Healer Malia is now apprenticed with the triumverate, learning to be a scribe?"

"A scribe, father, but that is fantastic. How did she convince the clergy to accept her?"

My father winked. "The dutchess is a generous friend. She posted bond that Healer Malia would not marry until she had finished her apprenticeship. Do not look sour, Sophia. You do not know the girl. She is not Healer Carlotta. This has been her dream for years."

We talked and gossipped until Timmon feared we would not reach the palace before the gates closed for the night. He left on some pretence to give me a quiet moment with my father. "What is bothering you, child?" my father asked as soon as my husband left.

"Nothing, father. Why do you ask?"

"I have seen that look in your eyes before, during the first year of Eugenio's life. You would not lie to an old man?"

I laughed. My father had been using that line since my mother first noticed white hairs on his head. "Timmon is not Julio, father. I will manage."

"Is there another woman?"

I released much of the tension that had been growing inside me since this visit began. My father had not guessed. The secret of my marriage was safe. "No father. There is no other woman. If I need your help, I will come. Rest easy on that." I kissed his cheek and left to gather my sleeping children.

\vspace{.5cm}

We agreed to see the duchess the next morning, to remove the dread hanging over us. We argued over which children should accompany us and finally settled on bringing only Eugenio to play with the twins, as of old. Thus the Mersean ambassador and his wife approached the small manor house where the Duke and Duchess of Cortan currently lived, with their son riding cautiously between them. The house itself was about the same size as our house in Buzen. A beautiful, old stone and brick building, with stone work decorated in a style popular a century ago. "It looks ugly, somehow, without balconies," I said lightly to my husband, a small peace offering.

He smiled. "You have forgotten your roots, Sophia," he teased, a peace offering in return.

Healer Malia met us in the library. She was a slight willowy woman with large eyes and a bright smile. Her modest healer's robes flowed fluildly around her, the belt revealing the narrowness of her waste. She was the type of woman that would strike fear in the heart of and wife in a conventionaly marriage already worrying about her husband's fidelity. Timmon did not give her a second glance. No, the woman I feared lay upstairs, too unwell to leave her bed today. Healer Malia took Eugenio away to find the duchess's children. She appeared shortly to bring us to her grace's chamber.

I thought I had reason to pause when I saw my father. That shock was nothing compared to what I saw in Nisrita. It was not only the matter of her physical body, she was disturbingly bloated and unhealthily large. A Liri sea man might have compared her to a bloated beached whale slouching against her pillows, what I could see of her face and arms puffed beyond human proportion. It was more than that. She took no care of her presentation. It looked like Healer Malia had barely managed to convince Nisrita to don her healer's robes and pull back her hair. Her head was uncombed, her nails were unkempt, her robe wrinkled, she had given up maintaining her face, hair grew on her lip and between her brows. Her eyes were dark, as if she had trouble sleeping. I could not believe that this woman had made a positive impression on the large network for Barroness Paulis's friends in Deyalorn. It appeared that Nisrita had given up caring about life.

While I gawked, my husband walked smoothly in and took Nisrita's hand. "You will forgive me for not standing, ambassador." she said, her face brightening at his touch.

"Not at all, duchess." My husband bowed with a disgustingly elaborate flourish.

When I went to kiss her hand as well, embraced me and chided, "Since when have you been so formal, Sophia?"

The duchess's hands were puffy and waterlogged, her face swollen like a bladder. I could see tiny veins bulging, bursting and bruising under her dark skin. "Forgive me, Nisrita, I am just shocked to see your condition."

Her puffy eyes crinkled into a smile. "Everyone who sees me says that. I may have to miss the coronation for the sake of the children." When Timmon started to protest, she changed the subject smoothly. "But I have not yet congradulated you, ambassador. You have achieved Makala's dream. He would have been extremely proud, had he been here to see this day."

The look that passed over Timmon's face at that moment would break the heart of even the most trusting naive wife. It was not only cruel, it was indecent of him to publicly show that intesity of love and longing with which he looked at Nisrita. "I could not have done it without your your support," my husband mooned. "Had you not been in Deyalorn, perhaps I could have convinced Duke Ergino, but..."

"Yet I was in Deyalorn, Timmon," Nisrita corrected with the grace and firmness of a long time lover. "This is a time for celebration, not doubt and regret. He would be proud, ambassador. I know he would."

It is not right, for an honorable Mersean man, to discuss, in front of his gifted wife, the treasonous plans of his lover's ex husband. Whatever plans Nisrita and Timmon spoke of, I was certain it did not involve rice. It irritated me that Timmon saw fit to conspire with Nisrita regarding his work while they lived months apart, where he placed conditions on me before informing me of my son's welfare when I shared a house with him. Living in Lir had changed him, it had drained him of his honourable Marsean values. I had agreed to protect his miscast nature, not sit neglected while he flirted with another woman.

"Why so quiet, Sophia?" Nisrita asked suddenly.

I smiled for the duchess. "I saw my father yesterday. He was disturbingly frail. After seeing you, I cannot get my mind off him." The conversation turned to my family and details of my father's health, then Malia's work as a scribe. When it wandered towards court gossip, I left with the excuse of visiting the young dukes and duchesses. I could stomach this meeting of lovers no longer.

My husband stayed alone with Nisrita for two hours. I do not know what they spoke of. I did not ask. We argued that night, being as we were, forced, for the sake of propriety, to share one miserable chamber. My husband wanted to bring young Duke Griswold to Lir. A long standing promise, he said, to the duchess. I would not hear of it. He said that he did not wish to take this action without my permission, but he would not break his word. If that was the case, I retorted, it did not matter if I gave my consent or not, did it?

The next day I put my domestic woes aside and attended the coronation. Every one of my five sense reminded me that I walked in my home city was at its best. The clean wide roads, the flowers, the musicians, the velvet gowns, the crisp dry wine the color of summer rain, the brilliant crowds lining the walk down to the ampitheatre, this was the glory of Deyalorn's court. The coronation hall sparkled with smooth white marble floors and gold gilt pillars. Every step and whisper echoed off the high vaulted ceilings. The colored glass window cast colorful shadows across the white marble floor, a reds and greens and yellows played on the magnificent faces of the spectators. King Gustav walked, with his long red and gold cloak sweeping behind him, silently, like a lion, on padded shoes, followed by a representative of the Triumverate and Head Ionus, of Deyalorn's tower, the triumverate made flesh: the Maker, in the form of the clergy, the preserver from the tower, and the Destroyer, our new king and leader of our armies. The gong sounded as the new king took the throne. The sound echoed like thunder through the hall. It is said that the sound of the coronation gong, heard by a false king when sitting on his throne will kill him instantly. A fantastical myth, to be certain, yet standing in that hall with the deep metalic sound echoing in my ears, I had to wonder. Nothing in Lir, no ceremony to their countless gods, no mosaic tiling, no warship, no printed silk covered wall could ever eclipse the glory of Deyalorn's court. 

\vspace{.5cm}

The following day, bolstered by my country's greatness, I resumed the habit of bringing my children to visit with Nisrita's. Whatever her relationship with my husband, my family owed her much. I had to look to my children's future. I did not wish them to say that they suffered in life because their mother could not swallow her jealousy. 

"You are angry with me, Sophia," Nisrita said suddenly as we watched our children play along the bluffs. She had her chair brought out and set in the oak grove behind the house she now lived in. Even that movement had made Nisrita uncomfortable. She had not failed to appear at the coronation for the sake of frightening children. Head Isadora of Deyalorn's Women's Tower monitored her condition closely. There was a chance that her child would have to be cut out of her early to save both lives. 

I tried to brush off her comment. "Of course not, why would I be angry with you?" Nisrita had pulled herself together today. She had cleaned the hair off her face and wore a becoming red and gold gown. She gloved her hands to hide the thrombosed veins. While her health kept her from walking, she looked her station. The duchess could donn a regal bearing, but she still had not polished the gauche directness out of her conversation. It was a pity, really, that no one had taught her better.

"I could not tell you why, my dear," she said without reproach, "simply that you are. Why else would you refuse to take young Griswold to Lir with you. Furthermore," she smiled sadly, "you did not used to be in the habit of lying to me."

She spoke with the authority of  a Master in a Tower, not the grace of a duchess. All the same, I had to answer her. "It is not worth mentioning, Nisrita. The accumulations of little shocks that must arise when two friends have lived apart for so long. It will pass."

Nisrita's laughter tinkled through the grove. Her tone suddenly changed to the familiar smoothness of Deyalorn's court. She stretched out a gloved hand delicately to squeeze my arm. "I have forgotten what a pleasure it is to talk to you, Sophia. You have been away so long, I have missed you. I am sorry for being indelicate. Let us change the subject. How are you finding you life as an ambassador's wife? We did not get a chance to talk much last time you visited."

The change in her manner startled me. This was not Nisrita as I had left her, though a woman I was not surprised to find, given what I had heard of her doings. As she said, I had been away so long. "It took some adjustment, but I have learned like the Liri for their art. I have even taken up drawing."

"You are a brave woman, Sophia, much braver than I ever could be." She lowered her voice conspiratorily, "Sama Araki has simply been scandalous in the way he has treated the women of Deyalorn. He simply does not realize that gifted women are different. It must be so hard living in a country without the gift."

My heart fluttered. She knew. It did not surprise me that she had pieced together what had happened between Minoru and myself. Rumours of the ambassador's public duel certainly made it back to Marsea. Nisrita knew the arrangements of my marriage. It scared me that she should bring it up in this way. "The embassy has two healers. Between their services and my skills, we are safe enough. As for the rest, it is the job of a diplomat's wife to learn new cultures."

"Ah yes, and I had forgotten that you brought healers with you. And Carlotta came to help you with Desmond, did she not?" She paused to watch our children play for a moment. "You and Timmon have two beautiful children. They are the same ages as my Lyca and Emile. Watching them play together, I would say they are as different."

The threat was clear. I had lost her favour. I had no choice but the accept her son. "Our children do play well together. If you think young Griswold will not miss his siblings too much, I believe his company would do Eugenio good."

Nisrita laughed lightly again. "You misunderstand me, Sophia. This conversation is not about my son."

I took a deep frightened breath. "What is it you wish then, your grace."

Nisrita raised her eyebrows slightly with a hint of criticism on her face. I realized suddenly how much she looked like Consort Dario. "To see how far apart we have grown, and how much you mistrusted me." When I looked suprised, she added. "Did you really think, Sophia, that I had convinced my brother to allow for Liri settlements in Marsea by playing an ill bred healer?"

Her performance troubled me deeply. With one word, she could destroy everything Timmon and I had built for ourselves. "I should go, Nisrita."

"So soon?" Nisrita cooed. "The children will be disappointed. Also, you have not yet answered my question." There was an authority in the last satement that kept  me from rising. Disturbed as I was by the sudden shift in our relationship, I could not bring myself to have the indecent conversation Nisrita wished to have. She waited some time for me to respond. When I did not say anything, she mused "Men have it easier, I think. When they argue, they get into a fight. They walk away bruised and battered in search of a healer, but with their anger spent. When we fight, we must be so delicate, and yet so much more brutal. We do not even have the soldier's custom of allowing our opponent to yield." She paused and waited for me to respond. I did not know what she wanted of me in this observation on the habits of the two genders. "Did you know," she continued, "Timmon used to call me a soldier every time I forgot how to act like a duchess. I think he was trying to train grace into me. It never worked, I'm afraid."

She knew more than she let on, I thought. I had angered her. She toyed with my emotions in her rage. I no choice but to endure these insults for the sake of my family. "I did not know, your grace," I said icily.

"So that is it, is it?" Nisrita asked, running a finger lightly over the arm of her chair. The gesture would have been inviting to a man. To a woman, it threatened. "Believe me, Sophia, I did not know until you told me just now." She paused to contemplate me coolly, her hand moving in its mezmerizing snakelike fashion until she came to a conclusion. "I do not suppose you know who your husband's first lover was."

"We do not speak of such thing in our marriage, your grace." I said tightly. 

"More's the pity. You might have saved youselves some trouble." Her gloved hands steepled themselves in front of her. "Duke Makala bereaved both of us when he died."

I gasped, still terrified by the duchess, my world spinning at the revelation. "Duke Makala was --"

"I warn you, Sophia, do not use that word in my presence," the duchess interrupted with an angry authority.

"My apologies, your grace. I meant no disrespect." I replied humbly. I no longer knew where I stood in this suddenly shifted world. 

"You do not deserve to know this," she continued calmly. "I will tell you anyway, as we have both known what it is like to live under truly cruel men. Ambassador Romino loved me once, the last time he was in Deyalorn. I sent him away to Lir just for the look he gave me. If you want my advice, take my son to Lir. I release the ambassador from his promise. I want nothing from you. Take my son solely because it will someday be useful to Eugenio to be foster brothers with the Duke of Cortan."

"Thank you, your grace." I had known Nisrita to be short tempered. She would get angry and kill an entire set of test animals in her passion. I had not known her capable of this cold calculating rage. 

She gave me her hand to kiss. As I bowed, she added "I need not remind you that the content of this conversation can only be repeated by the trees in this grove?"

She did not. I gathered my children and left.


\end{document} 
 
