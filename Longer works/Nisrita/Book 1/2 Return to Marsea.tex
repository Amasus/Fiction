\documentclass{article}
\usepackage{fullpage, verbatim}
%*****************
% Annotations
\usepackage{soul}
\usepackage[colorinlistoftodos,textsize=footnotesize]{todonotes}
\newcommand{\hlfix}[2]{\texthl{#1}\todo{#2}}
\newcommand{\hlnew}[2]{\texthl{#1}\todo[color=green!40]{#2}}
\newcommand{\sanote}{\todo[color=green!30]}
\newcommand{\egnote}{\todo[color=violet!30]}
\newcommand{\newstart}{\note{The inserted text starts here}}
\newcommand{\newfinish}{\note{The inserted text finishes here}}
\setstcolor{red}
%***************************

\begin{document}

I awoke from the growing intensity of the pain on my right side. I was tired beyond reason or comprehension. My limbs sank into the hard surface underneath me with a heavy exhaustion. I must worked them too hard last night. I sighed and flexed my legs. My calves warned me of aching muscles for the day to come. I decided better of it, and lay still to open myself to the Preserver's power. My chest stabbed me with every breath, and below that an intense pain threatened my ability to think clearly. It made it hard to concentrate on drawing the energy from the world around me. I was badly injured. I had to check my wounds. I mouthed the words I had learned to use as a crutch to focus my mind and ready my muscles. Most students outgrow the need to use them by their fifth or sixth year, but they are useful in difficult situation. I braced for the blast of heat that I called through my talisman into my chest. Nothing came. I must still be exhausted, I thought groggily, and tried to recall what I had done last night. If I could not check my injuries by the gift, I should do so physically. With an immense force of will, I convinced my left arm to reach across my body to explore my ribs. I could not.

My arm lifted half a finger's width to encounter a barrier. I was bound down.

This was enough to cut through my drowsiness. I forced my eyes open and looked up. I was in a tent. It was light outside, but the sun did not shine on the tent. I did not recognize my surroundings in the dispersed light. It was not the women's tent of the healing corp. It was smaller, meant for two or three people, not a score. I did not know where I was, how I had come here, or why I was bound to a cot. My heart and my breath quickened sharply. I whimpered in pain as my chest protested against my panic.

I closed my eyes, and tried to steady my breath. My body needed attention, where ever I was, I had to be healed. I moved my limbs gingerly and found that all four were bound. Panicking would get me nowhere. As I tried to get my pain under control, I listened to the sounds outside the tents. I was alone in a tent, in the middle of an army camp. I could hear the  sounds of men and horses coming from outside. The were familiar, comforting sounds, though the air smelled much more strongly of horses than it should. I had lived with these sounds and smells for the last six weeks. 

No, I recalled, I had been travelling for the last few days. With someone else. Sol, his name was. A friend of Timmon's. I turned and arched my head to  better see my surroundings. Where was Sol? Why was I in Makala's tent with the Black Riders? Why was I injured? I could not remember. Had the Black Riders recued me? Had Sol called them? I was on a raised flat hard surface. Was I already on a healer's table? The effort of the exertion proved to be too much. I lay back on the table, panting and exhausted from pain and fear. My eyes drifted shut, and sleep overtook me.

I remember what I dreamt that time. It was the first of a family of dreams that would trouble me frequently during those hard months, though the one that afternoon, in retrospect, was so much warmer and more tender than the dreams to come. 
 
I was in the Tower's garden, sifting among the long yellow grasses that still had ears of green seeds on the end. The older girls could weave the grasses deftly together to make mail for our toy soldiers. Once the seeds dried, the mail looked vaguely like chain. Only officers got to wear chain. No one wanted to play healer to an army of only footmen. 

I heard the head girl enter the garden and call for the young dutchess, and I turned to see Chamila leave the pageant she was taking part in to walk towards the White Tower. I felt very afraid suddenly. The head girl looked worried. I knew that Tonyo had been in the infirmary for the past two weeks. Chamila had been so sad. She went to see her younger brother every day. I did not understand what was going to happen, but I knew that something terrible was taking place. Fear froze me to the ground. 

In the way of dreams, I found my self suddenly and seamlessly in Duchess Chamila's room. It was not her room really. It was the room that Chamila shared with all the other second year girls, when she stayed at the tower, and not at the Castle in Cortan. The room was empty. I was waiting for Chamila. I ran my fingers along the stonework of the windowsill of the room shared by a dozen girls. It felt so intimately familiar. It was exactly like the room my class had grown up in. I knew Chamilla should not have ever been promoted to second year in the first place. She was charming and sweet and studious. Cortan needed a child with the gift. She showed promise in effort where she did not show promise in gift. The head mistress thought she was a late bloomer. She had, after all, shown sporadic, unrepeatable ability to weild the gift. I did not know much of this, of course, when the actual events had occured. I had been a first year student, a child of five. I learned this over the years. My dream self knew it all.

What I did know at that age, that the head mistress did not, was that the previous year, when Chamila was a first year, and I had not yet formally enrolled in the school, I had seen Chamila enter the tower with a badly wounded rabbit. The poor creature was bleeding all over her skirts. She said she had found it in one of her brothers' traps, and wanted to keep it for her own. Makala had said that she could keep it if she could heal it. I had not formally trained in healing yet, but I had lived in the tower since before I could talk. Stopping obvious bleeding was the first thing the Tower teaches. I had learned early on that by placing myself where my elders would trip over me, they would either punish me, or let me watch what they were doing. I had picked up a bit of healing before my time. Chamila took the now not bleeding rabbit to her teachers, and asked them for help healing its leg. Knowing Chamila, she told them that she had stopped the bleeding herself. I did not mind. She let me play her page in the pagents she was constantly in, and taught me songs she had learned at court. 

Her brother, Duke Tonyo, had been my friend. We were exactly the same age. Unlike Chamila, he had shown a good deal of ability to weild the gift from his first day at school. He was not as clever as I was, of course, but he had not been in the Tower for all of his life. He and I spent all our free time together. His friends did not seem to mind that I was a girl. Tonyo's brothers had taught him to climb. He would scamper up trees to gather hatchlings, and carefully putting their quivering unfeathered forms into his lined pocket. Back on the ground, he would pull them out gently and put them in my hand, blind, pink, squawking for food. Then we would pretend to practise on them as the older students did. We never hurt them much, and ususally managed to return the chicks as we found them to their nest. We were not breaking any rules by disturbing the birds that nested in the Tower's garden, but we never told anyone what we did. It was our private game. We felt so grown up healing the chicks. 

In my dream, as I sat in the second year's dorm, I knew that Tonyo was gone. The head girl had told me that I would not see my friend anymore. I knew he was dead, but I did not fully understand why everyone was sad. The Destroyer gave souls back to the Maker to recast. That meant that he would come back again soon. I had decided that the head girl was wrong. I would see him again, I just had to look hard enough. 

Chamila entered the room, her eyes red from crying. I went to embrace her, suddenly crying myself at the sight of her grief. She took me to her bunk and said that she had to pack. She had to leave the tower too. We could not play anymore. The Headmistress had decided that she was not gifted, and the Tower would no longer teach her. In reality, Chamila did not leave until the following year, but the memories merged and tangled themselves in my dream. Chamila was devastated. I helped her put her things carefully into her trunk while she changed into a black widow's veil. 

"Will you come down with me?" she asked when a man from the castle came to collect her belongings.

I nodded and clasped her hand as I had in life. When we emerged under the vaulted white ceiling of the Tower's entrance hall, I was a woman of fourteen, while Chamila still a girl of seven. Makala stood before us, not the boy that must have come to meet us that day, but the man I had married, waited for us. Chamila let go of my hand and ran to her older brother's open arms. I heard her sobbing into his chest. He whispered soothing words to her, and held her until she had the courage to pull away, as he did for all his siblings, as he had done for me when I miscarried. I stood awkwardly, not knowing what I should do in this moment of intimate grief. The scene blurred with my own tears. Makala dried Chamila's eyes and extended his hand towards me. 

"Tonyo had declaired his intension to marry you. Come walk with us. You may as well put earth on his face." Makala told me years later that his youngest brother Tonyo had, in fact, announced to his father on that Day of Unions that he wanted to marry me. It was a childish declaration that had amused Duke Ergino enough to make known to the family. That was by no means reason for Makala to take me to witness his brother's burial. In life, that was, in my mind, the first of Makala's attentions towards me. He always had a way of making me feel a part of his family. I never questioned why he had adopted me into his fold of siblings.

I tried to move to accept his hand, but I found I could not move my feet. They had become part of the Tower itself. When I did not come to him, he looked at me sadly and led Chamila out of the hall. "Makala" I cried after him. "Take me with you." But I could not speak. 

Emptiness welled up inside me. I knew that I had lost Makala, Chamila, Tonyo, and all of Cortan in that one moment. I only had the Tower left. That was the family I had started with. That was what I had left now. I had to leave the Tower. If I could only reach them, I would not lose everything. I tried again to  struggle against the stone my feet had turned into. My body ached and throbbed with pain. If I could just reach him in time, I thought, I could save him. I could save myself. I felt one of the hall guards restrain me gently. I tried to pull away, but my side hurt too much. I was not strong enough. "Makala, come back." I cried. Only a whimper emerged. 

With the whimper, I reentered the present. An soldier showing a black horse on his armor held me down firmly. I did not recognize the emblem. It did not belong to Cortan, Firvona or Gissal. Nor was it any emblem of Niev that I have seen. His hands and arms showed a network of scars so common among the fighters of the ungifted hordes. Natural healing is far inferior to the gift. He had a thick stubble on his face, the hair atop his head was tied back tightly in the many tiered fashion of gifted fighters. Not a single strand hung loose. I wondered what army had captured me, and what they would do with a gifted healer. If I was lucky, they would sell me. If I was unlucky, they would imprison me for my ability, or to force me to bear children. We had all heard the stories. We all knew the dangers of being captured by the ungifted hordes.

I stopped struggling. My body ached, and this man did not seem to have any intension of hurting me further. When I relaxed, the tension left my attendant's face. It was not a cruel face, I decided. He lifted up my shirt and exposed the wound on my side. I felt his hands work gently in the area of my wound. "Are you a healer?" I asked.

He did not respond. I did not know if he understood me. I did not know what language he spoke. It was an irrelevant question. Only a healer's hands moved as deftly and gently as his at a wound. I could not see his hands from where I lay, but I watched his face as he examined my wound. He looked confident, like he had done this for many years. He did not make a sound. Silently, he produced a bowl of clear liquid and put a drop to my lips. It was brine. I braced myself for the sting. When it came, I gasped sharply, then thrashed and whimpered from the stab from my ribs. He stopped washing my wound until I had recovered.

"I am sorry." I said, when I could. "Please continue."

He washed more carefully, or I controlled myself better from then on. Either way, I lay still on the narrow table while he washed my side. He remained silent. It would be easier to endure the brine wash if I did not look at his movements. I closed my eyes and tried to figure out where I was. I had been on the road to Bayam for three nights? No, four. Timmon had sent me away from Turina. My companion, Sergent Sol Nicose, had proved to be plesant company. He knew of my loss, but he did not bring it up. Nor did he say anything to fuel my fears about Cortan's safety. If Timmon was right, and every night I prayed to the Preserver that he was not, I may have lost everything that night. Only by sheer luck had I kept my life. Luck and a friend's needless death. I could not think about that day. I would lose my mind if I did. Sol talked steadily, about the terrain, about our route, about previous battles he had seen, about Timmon as a commanding officer, about his family. He kept up a cheerful and easy chatter. It went a good ways towards keeping me distracted. The strains of travel did the rest. Remembering him made remembering the rest easier.

Sol invented a story where we were two travelers, I his page. We had been chased off our path by the Black Riders that had ridden through these hills, and were now lost. I would keep quiet. Sol knew some of the local language. He would speak if we enountered trouble. He named me Nesto, and encouraged me to call him Sol. In the end, we did not need the ruse. We had avoided people successfully for three days. It had not been hard. Mountain hamlets were few and far between, travelers off the roads rare. We camped in the edge of the grasslands. That must have been last night. Though I was not certain. How long had I been in this tent? I remembered waking to Sol's voice calling me urgently, telling me to run. I saw a half dozen shadows just beyond the circle of fire. There was no place to run. Even if I had escaped, without Sol, I could not find my way to our forces in Bayam. I remembered picking up my bow and watching one shadow crumple to the ground, cursing. Then I remembered hearing a bowstring sing and a surge of pain in my side. I dropped my weapon to see to my wound. I had been so stupid, fighting in a circle of light while my foes had the cover of darkness. I became aware of three men surrounding me. It was like that other day while training in Cortan, when I had been found unarmed and unawares. That was a day I swore would never repeat itself again. No one would catch me unawares again. I pinned a man at five feet. I left him gasping for air and clawing at his burning throat. But I could only hurt one man at a a time with my gift. At this range, a bow and arrow was nearly useless. I threw my only knife at the man in front of me and missed. I heard his footsteps stumble and continue towards me. I burned the man behind me, but I was tired, I could not hurt him much. I felt something hard hit my ribs. Then the wound at my side went white with pain. 

I stiffened at the memory. We had been attacked. I had no idea by whom. Was I still in their hands? Was the man before me my rescuer? Where was Sol? A feeling of dread entered my conciousness. I pushed it away. I could not entertain the possibility that he too had died. It was too much.

"Where is Sergent Nicose?" I asked, knowing it possibly to be in vain. "There was a man with me where I fell. Where is he?" I got no responce. "Do you understand what I am saying?" I asked more urgently. 

As an answer, my healer put his hand over my eyes. A moment later, I felt the unmistakeable warmth of the gift coursing through my wounded flesh. I stopped questioning, and focused on directing the heat. It would burn with incredible intensity throughout my side if I did not channel and keep it contained in the area of the wound properly. I did not understand where I was, or who was healing me, or how there was a gifted healer outside Marsea. It did not matter. An untreated open wound could be deadly. Quick healing was the key to my survival. 

It was over quickly. The fire was never very hot. My healer was not a particularly strong one. He took his hand off my eyes. Hanging six inches from my face was my talisman. I thrashed against my bonds in a surge of fury. "That's mine. Give it back. You have no idea what the Tower will do to you for stealing a healer's talisman."

It was childish. He must have been trained by a tower. He knew exactly what the Tower would do to him. He did not care. I was in his power, days or weeks from the nearest Tower. My ribs reminded me of my compromised situation before anyone of those thoughts came to mind. By then, my companion had pocketed my talisman and left the tent. I let out a stream of curses that would have shocked Timmon. If the gifted fighter was anywhere near the tent, he would have heard. How dare he steal from me. I pulled at my bonds with my arms, twisting my body as much as I could. I made no progress. I paused for breath and tried again. Healing is almost as tiring to endure as it is to practise. I fell asleep during one of these pauses.

When I next awoke, three candles lit the area around my head. My healer sat silenly before me with a bowl of broth and a few pieces of hard biscuit. The smell made my stomach tighten and ache with hunger. How long had it been since I had last eaten? I watched him silently dip the biscuit into the broth to soften it. I did not know who this man was, or what his intensions were. I did not know if he meant me harm. I did know that he had stolen from me. I also knew that he understood me, but refused to speak. He had been trained by the Tower, but lived outside Marsea. He may be a traitor or a criminal, or anything. I had not reason to trust him. I did not accept the food put to my mouth. "Who are you?" I asked in an attempt at bravery. 

The man put the wet bread down, picked up the broth and the bread and turned slowly towards the exit. I was painfully hungry. It was dark outside, I had not eaten all day. Who knows how long I had been uncouncious before that. This was not the time to be headstrong. If he meant to poison me, I justified, he would not have gone through the trouble of healing me. As for the theft, I could not get my talisman back wounded and starving.

"Wait," I protested. "Please. I am sorry. I will do as you ask."

My silent companion sat beside me again. He fed me slowly and carefully. Much more slowly than I would have liked. I ate in silence. I did not want to do anything to chase my food away. After I had finished, he brought out my talisman and held it inches above my chest. I tried to guess at what he meant. 

"Do you want me to heal myself?" I asked. My caregiver nodded curtly but did not give me the talisman. It took me a long time to understand his meaning. "I promise I will only use it to heal" I said. 

I felt the weight of the crystal and silver fall onto my chest where it usually hung. I opened myself to the gift and let it flow through me. Healing oneself is hard. It is better for two wounded healers of similar ability to heal each other than for them to try to heal themselves. Focusing the gift at the wound hurts and required concentration. Doing anything at all complicated to a wound hurts and requires concentration. Keeping the gift and thus the pain focused at the wound requires concentration. The pain of the wound itself is enough to keep one from doing anything but concentrate before one piles the pain of the healing on top of it. Healing oneself is akin to embroidering one's initials into the skin over a wound instead of simply sewing the gash closed, possibly while holding the wounded limb inside a working kiln. Without a strong healer, I had no other option. I channeled the heat through my side and my rib until I can endure no more. I paused for breath and resumed again, working with less fire this time, for a shorter spell. I paused again, to recover before picking up a third time. I needed to recover. I did not have the luxury of healing myself at my leasure, when my strength permitted. My captor had my talisman. When I stopped for the third time, the weight of the talisman lifted from my chest. I was asleep in an instant.

The next morning, the silent ritual repeated itself. I was awakened gently by my healer. He held a bowl of thin gruel in his hands. He fed me, had me heal myself, then did what little healing his gift allowed him to perform. He was not an old man, perhaps Timmon's age, perhaps a little older. If he was ever a member of Marsea's armies, he was a strongly gifted fighter. Otherwise, he may have practised at a minor Tower in the land. I did not dare ask him who he was. This man before me did not speak. Neither did he seem to meet conversation or questioning with kindness. 

After he finished his healing, he tied my hands together in front of me, tied my feet with a length of rope that allowed me to walk but not run, blindfolded me, then unfastened the rest of my bindings. He helped me stand. He handled me gently, inspite of the coldness of his demeanour. He offered his arm as a crutch and a guided me firmly and slowly through the darkness. After a few steps, I felt the wind on my face. After a few more paces, he firmly tugged downward on my arm to indicate that I should sit. I heard him leave me. I seemed to be alone.

I listened to the sounds around me for clues of where I was, who my captors where, and what they intended to do with me. It would seem the the camp was moving. I heard tents being collapsed, and the hiss of cook fires being dowsed. Men and horses walked past me in all directions carrying on their conversations. I learned little other than the fact that they spoke Mer. I seemed to be among a group of my own country men. If so, why was I being treated thus? Why did they not tell me where I was, or what had happened to Sergent Nicose. Was it possible that these were Firvona's men, and they intended to take me captive to their Tower? Why would they spare my life after killing the dukes and Lysene? 

Suddenly, I became aware that a group of men and at least one horse had stopped near me. I stopped my speculation to pay attention to their movements.

"Is this the filly that bites?" a man asked. His voice came from well above my head. He must be mounted I guessed. 

"I believe the bitch has been defanged" another answered. 

"She does not look much like a prize to me," the first said.

"That is a good thing. She is not for you, Reno," a third voice said.

I heard the man named Reno dismount and walk towards me. I felt the hairs on my back stand on end and my skin tingle. It was happening again. I had said I would not be caught off guard and unarmed in the company of soldiers ever again, and here I was, alone and unarmed. I was completely aware of my surroundings, but what good did it do me. My heart was beating hard and my stomach knotted painfully when I felt Reno's finger's under my chin. He turned my face to examine it. "She must be a work horse," Reno declaired. "There is not much to look at." 

What I could have done if I had my talisman. These men were not stupid. They would not give a healer her talisman. I bit my lip to keep from spitting in the direction of the man's face. I was wounded, bound and unarmed. There was nothing I could do to defend myself.

"We could have some fun with her tonight," the second voice suggested. The others laughed. I held my breath at the words. I had no means of defending myself. I had to remedying that if I was to survive in this company.

"You could Cyrel, but every penny less than two thousand we get for her comes out of your share first." This was spoken by a fourth voice, coming from well behind me. It was a deep smooth voice, one that reminded me of Duke Ergino's.

I heard his footsteps approach, and the other men mumble "Yes, commander," then disappear to their own buisiness. 

The commander of this army sat down beside me. I sat stiffly at attention. They intended to sell me. That did not make any sense. Why would one of Marsea's armies sell a White healer? Who would they sell me to? The Tower would not pay to take back one of its own. Did they intend to sell me to the lands beyond our borders? That was foolish. That would only aid our enemies. What type of men were these?

"Stand up," the commander ordered. I obeyed gingerly.

I heard him stand and walk around me. I wondered if I was being inspected.

"What is your name?"

"Lysene of Cortan, commander." I said without hesitation. If he did not know who I was, it may be for the better. I did not know much of Lysene's story, our paths had not crossed much before the campaign. I decided I would state what I knew, make up the rest, and pray to the Preserver.

"Do you have a house, Lysene of Cortan?" Duke Ergino's voice asked.

"No, commander."

My interrogator circled me once more. I stiffened as I felt his gaze on me, but he did not touch me. "What is a married healer of Cortan doing deserting her post with an unmarried sergent?"

I balked at the question. My mouth opened and closed stupidly, unable to come up with an answer. The duke's voice chuckled harshly. "You've been caught, Lysene." He sneered my adopted name. "Lucky for you, our men found your talisman on you. It saved your life. It was too late for your lover by then."

A half strangled cry emerged from my lips. Sol, Sergent Nicose, had been a good man. He had done nothing to deserve his death. He had only followed Timmon's orders to keep me safe. He was a young man, a courteous man, devoted to his family and country. Why did the Destroyer insist on taking these good men from the earth. Timmon should not have sent him with me. Everyone I touched seemed to be marked for death. It was not right.

"Is your husband still alive?" the commander's voice asked from the darkness before me.

"No. Yes," I corrected. Lysene's husband was well when I left Cortan's healing corps.

"Which is it healer?" the voice asked sharply. "You are not eloping anymore. I will not be lied to."

"He was serving with the corps at Turina when I left, commander."

The voice grunted. "Can you ride?"

"Yes, commander." He led me to a horse, then had me mount sideways. He had not untied my feet. When I felt for the bridle, I found it held by another man.

"You do not get to ride off into the grasslands by yourself, Sylva." Said Reno's voice. He mounted behind me.

I was not a home wrecker, nor a whore as Queen Lucretia's daughter Sylva was. I held my tongue. The story fit the facts. I decided not to clarify the misperception for the moment. 

Reno was a gaunt man, I guessed. His arms around me felt thin, hard and long. He smelled of tobacco and horse shit. I was not riding with an officer of this strange army. I wondered if I could get any information from him, then thought better of it. He was a rough man. He scared me. As he sat on his horse, he pressed his body far closer to mine than was necessary. I was in trouble and defenseless.My years of training with Black Riders and the White Tower notwithstanding, I was no different than any ungifted girl in Reno's hands. 

Reno spurred the horse into motion, and my ribs crackled with pain. What little healing time had offerd my bones disolved with the rhythm of the horses gait, and my bones ground their raw unhealed edges against each other. I bit my cheek to keep from whimpering.

I tried to think. I did not understand who these men were. I wanted to comb through my memory for any clue that could help me understand. Blindfolded, I had no idea of what direction we traveled in. The wind was at my back, but that meant little. The pain distracted me. Within an hour, it was all I could do to recite the words I used as a child to clear my mind to allow the gift to flow through me. They were rhythmic and calming. It was all I had to endure this agony. In half an hour more, I could not even recite the words.

"Please," I gasped. "Slow down."

"What is your offer, Sylva?" Reno's body was, if anything, pressed closer to mine that it had been when we started. He was breathing heavily, though we were not riding fast. I recoiled at the thought, my hands went instinctively to the chain around my neck. It is a gesture of modesty among married women I had disdained until that moment. I had the Tower to protect me. I did not need to hide behind a man. The chain was gone.

Reno laughed. "A bit late for that, don't you think. Perhaps you can beg your husband to forge you a new one. If he'll still have you." 

My heart broke. I did not have Makala. I did not have his chain. I had looked forward to the moment that he put that chain around my neck for months. "I will make you one of Cortan's children," he had promised. With that chain, he brought me into his family, not as the world saw in Deyalorn, as his wife, but several weeks later in Cortan, as his adopted sister. That chain symbolized everything I once had in this world. With it gone, I had nothing. Tears burned down my face. My nasal passaged blocked, making it harder to breath. Pain won the battle for my self control. 

In response, Reno spurred his horse faster. "Make your offer, Sylva." I swore I would not give in to him. Grieving and wounded as I was, I would not give him what he wanted. 

My wounds screamed with each impact. I did not make an offer, but it did not take long before I found myself clinging first to the horse's saddle, then to my captor to stay seated. Eventually, he had to put his arms around me to keep me from slipping off the mount. I needed his support. I could not sit up myself. His hands found them selves in places they had no right to be. 

I do not know how long I sat slumped against Reno's hard chest, the sun baking my head, alternately gasping and sobbing. I do not even know that I stayed concious the entire time. If the pain had left any room for thought in me, I would have found a way to fight the man. I hated him. At that moment I was certain that he had robbed me of my chain, that he had killed Sol, that he had poisoned Makala, Lukos and Lysene. If there was one person in the world I needed to kill to avenge the good people I had lost, it was Reno. But pain had not left any room for thought in me. Grief and hatred and agony swirled and mixed in my head until I had no sense of time or position or self left. I could only whimper to the rhythm of the horse's gait and cling to my captor for my survival.

Eventually, the nightmare ended. The horse stopped, I slid off and lay where I landed. Someone came for me and lifted me into the shade. My healer took off my blindfold and examined my wounds. He did what he could do for my flesh wound, then showed me my talisman. I shook my head. I could not heal myself in my current condition. He left me in my tent.

I took the moment to gather my bearings. We were still in the grasslands. The sun was still high in the sky. I was unbound and on a soft rug this time. I prodded at my side. My bandages were showing spots of red already. I would not be able to ride again if I could not heal myself. I felt my chest. I had determined yesterday that I had fractures in two ribs. I could not tell what state they were in from feeling. The flesh was too tender and swollen. 

I lay back and cleared my thoughts. The men I rode with intended to sell me. They had gifted in their company, but they did not want me. From Reno's behaviour this morning, I gathered that they did not have any female gifted in their company. They were probably outlaws. I could not see any reason why one of Marsea's armies would sell a gifted to the ungifted hordes. Furthermore, the price of two thousand gold was too low. A gifted woman is worth much more than that to those that do not have the gift. I guessed that they intended to sell me back to Marsea. It was a guess. There was much I could be missing, but it was what I had. I closed my eyes and contemplated my next actions. The pain had settled into a dull throbbing by now. As long as I did not move, it was bearable. 

My healer entered the tent a short while later. I found myself glad to see a familiar face. Whoever these men were, this man did not wish to hurt me. As the tent opened, I saw two pairs of feet standing at attention just outside. They had posted guards on me apparently. My healer brought me food. There was a tepid salty broth with some tough meat in it, and some more of the hard bread. When he tried to help me up, I shook my head. I did not want to jostle my ribs. He fed me slowly and silently again.

"May I ask you a question?" I asked, when he rose after my meal. He paused. "Are you taking me to Bayam?" His body did not convey any information. I hesitated. I did not know who these men were. I did not want to give too much information into the wrong hands. I did not see that I had much of a choice. "If that is your intension, you may wish to tell your commander that Cortan cannot pay. The Duke of Cortan is in trouble. If someone could come to his aid, they would be rewarded handsomely." My healer nodded curtly, then left.

I lay alone in my tent wondering if I had just made matters worse. What could be worse I argued. If the house had fallen, what further harm could come of a group of outlaws raiding the castle? The dukes were dead. I was as good as lost. What was left? If on the other hand, the house had not fallen, these men could get news to Bayam, much faster than I could. It was possible that they would turn out to help Firvona against my house. I hoped that a group of horsemen would not sway the battle significantly. 

My healer entered the tent again and put on my blindfold. Duke Ergino's voice followed him shortly. "What news do you have of Cortan?" it asked shortly.
I told him what I knew of Firvona's actions, of the dukes' deaths, of Timmon's theories. I left out the fact that one heir of Cortan still lived, and my involvement in the matter. The commander sighed deeply when I finished. "Do you have any proof?"

Proof? Only that I had risked my life, and spent that of a good man to ferry this message. I shook my head. I had no proof. I heard the commander rise and leave. My healer undid my blindfold and indicated that I should sleep.

I dreamt of pain, the type that came in waves and shuddered my body. Mother, Duchess Cybeline, stood over my birthing bed and accused me of being a traitor. She was hysterical. She said that I was more loyal to the White Tower than I was to Cortan. She called me the poison that had crept into her house. She called upon the Preserver to turn his back on me, as I had turned my back on the one that bled its way out of my body. She called me a bastard and a child already claimed by the Taker, prophecied that everyone I ever cared for would be claimed by the Destroyer. She accused me of wanting to destroy her house. She swore that she would not let me. I closed my eyes and plugged my ears to her fury, but on and on the hatred spewed, her words growing harsher with my pains. I found myself sobbing and screaming for her to stop when two firm hands gently placed themselves over my own. "Hush little one," Makala said, "She is gone now. She cannot hurt you." His voice was always so soft, even in anger I have never heard him raise it. He was furious with his mother the morning I lost his seed. He sat by the head of my bed and stroked my forehead until the healers asked him to leave. I tried to look up, to thank him for his kindness as I had not done that day. Try as I might, I could not position myself to see his face. Makala left, the child emerged. I sat up to see the misformed placenta that had emerged from my body. I reached for a knife to cut it open. I wanted to see the tiny creature that had failed to be born. I found myself calling for my hand to stop as I moved. I knew what I would find, and I did not want to see it. I had no control over the actions of my body. They layers of flesh fell away to reveal a small sleeping form, smaller than my thumb. When I wiped away the blood, I nearly dropped it in horror. It was a miniature Makala, black and pus streaked, poisoned. A thread thin umbilical chord still throbbed. It was red near the plancenta, and night black near the dead creature. I had poisoned him, I realized, as a wail of agony escaped my lips. 

I cried myself awake, and found myself already half sitting up, my chest throbbing at my sudden movement. My healer was at my side in an instant, helping me ease back to bed. It was dark outside, I realized. He lit some candles then helped me drink some water. "How long have I been sleeping?" I asked. 

He held out five fingers. Five hours. "Why did we not ride earlier?" He handed me a note.

On it, written in a neat hand were the words "Riders have gone to Bayam to investigate your claim." I recognized the handwriting, but I could not place it. I felt like I had seen regularly once, and that the text it wrote was significant. It was impossible of course, but I could not shake the feeling. 

"Did the commander send this message?" I asked. My healer nodded. Then he left to bring me some food.

He deemed me well enough to eat what the rest of the men ate. It was not particularly good fare. On the other hand, it was solid. I appreciated having something to chew. 

"Do you have instructions not to speak to me?" I asked my silent companion when the edge had come off my hunger. The man beside me had kept a silent vigil during my meal. I found the silence stiffling. I wanted some information, or some companionship. My healer shook his head. "Are you mute?" I asked. He opened his mouth and moved the candle so I could see. There was nothing where his tongue should have been. Instead, the back of his throat held a shriveled and scarred red stump. I gasped in horror. He shut his mouth calmly.

The punishment for leaking military secrets is having one's tongue removed. Depending on the secret, it may or may not be accompanied by exile. I had read the criminal code of Marsea once, many years ago, out of curiosity. Perhaps Makala or Lukos had been in the process of learning it. Some of the punishments were so harsh that they burned themselves in my memory. "You are a traitor!" I exclaimed. I felt the burning sensation start at my sternum then work its way slowly to my wounded ribs. It was a warning. "I take it back" I said. "I won't ever call you that again. Please don't hurt me." The burning stopped.

I had suffered enough pain today. I could not take any more. I finished my meal in silence. I was among Marsea's enemies. I rode with a band of outlaws. At the very worst, they actively worked against Marsea, and had taken this man in as a spy. At best, I was among a company of Mersea's outcasts: Traitors, cowards, criminals, miscasts. I despaired for Cortan. What had I done?

My healer cleared the plates then brought me my talisman. I started immediately working on my ribs, while he examined my progress. Suddenly, he plucked the talisman from my chest. I gasped and choked at the sudden change in the energy flowing through me. When I recovered he looked apologetic. He shook his head at me and touched my side. "When do we ride next?" I asked, irritated that someone else would tell me which of my wounds I should tend to. He made the motion of a page being flipped in a book. Tomorrow.

What he suggested was unreasonable. My side was still bleeding yes, but all our efforts had focused on that so far. Without attending to my chest, I could not ride. "I will ride tomorrow like I did today if I do not tend to my ribs." My healer pursed his lips. He looked like he would have said something if he could have. He put the talisman back on my chest and stood back. I returned my attentions to my chest. He stubbornly healed my flesh.

The next morning saw the same routine. This time, he silentely insisted that I work on my flesh. He examined the wound as I healed. When I exhausted myself, he pressed on it gently with one finger until I cried out. He looked at me with aproval, then spent his gift on my ribs. I closed my eyes and let him work. When I had lain exposed on the ground for a long time after he had finished, I opened my eyes out of curiosity. Then I cried out in panick and immediately tried to get to my feet. He left off working at his belt, and clapped a hand over my mouth. His other hand held me down. He shook his head and smiled at me. It was a kind smile. I did not understand what he was saying. I knew nothing of my companions. For every question I found answered, led to three new questions. I had thought this man to be an ally, or at least one that wished me no harm. Why then had he been unbuckling his belt while I lay half naked before him? 

He waited for me to calm myself. When I did not, he put a finger to his lips and let me go. He backed into a corner of the tent, leaving my path to the tent opening clear. I watched him unbuckle his belt then toss it within my reach. I reached for it and understood. His belt bore his talisman. He was arming me in secret. Whether he saw what had happened yesterday, or heard from others, I did not know. He wanted me to be able to defend myself, but he had instructions not to give me my talisman. 

In the end, I needed his help to fasten the belt, and hide it under my shirt. It rubbed against my wounded flesh, but it would not cause me too much more pain, I hoped.

"This is why you wanted me to heal my flesh?" He nodded. "This leaves you defenseless." If I had need to use this, no, when I used this weapon, and his companions saw the belt, they would turn on him. My healer raised his eyebrows and pointed to the various places where he kept weapons on his body. I laughed. It was I who was weaponless. He then pulled a leather thong from around his neck. Attached to it was a pouch from which he extracted my talisman. "May the Preserver keep you." I said solemnly, a shorthand for the prayer said by the Tower's master when he puts a talisman on a healer. He nodded curtly, put it away. I felt ashamed for doubting him. "You are a good man, healer." I said. "I am sorry for not trusting you." He grinned. In that moment, it could have been Timmon standing before me. Some people have brilliant grins that brighten the world and chase away all fear and sorrow. He and Timmon both share that quality. 

I let him blindfold me, gingerly took a deep breath, and let him guide me to my horse. Come what may, I felt confident that I had made an ally.

I rode with Reno again. The belt chafed, but it did not bother me too much. My ribs ached, but not so much that I could not find the strength and the focus to heal myself during the short pauses when the horse stopped to rest. Reno wanted a repeat of yesterday's performance. He had dubbed me Sylva, and had been emboldened by my desperation and weakness of the previous day. When it became clear that I had healed enough that he could not cause me to suffer by simply riding hard, he slowed the animal to a trot it could sustain. Sporadically and unpredictably, he either drew the animal short or spurred it forward. Blindfolded and sitting sidesaddle as I was, I nearly fell off the first few times he changed the creature's gait. To a lesser horseman, the experience would have been terrifying. I found myself clinging to the horse and saddle for stability, giving its movements my full attention, trying to anticipate the next change. It was not always possible, but I did not lose my seat.

With my hands occupied with the task of keeping me mounted, Reno's wandered where they pleased. I could do nothing to stop him. He sat behind me, heavily armed, and in control of our mount. I sat injured, and armed with a weapon I could only use once, and at the expense of being able to heal myself. I hated Reno. I spent the day powerless and terrified, caught between the whims of the cruel gaunt man and his obedient steed. I focused my attentions away from the terror assaulting me from the movements of Reno's hands and the motion of his steed to my hatred of the man behind me. I needed to hurt him, both from a need for vengance, and from a need for safety. If Timmon's training had taught me anything, it was that men respect and fear those that can hurt them. As a healer, even wounded, I could earn that fear. I did not need the respect.

I endured Reno until I felt the day start to cool. Evening was setting in. If I did not want to face this torment the next day, I would have to act soon. I reached forward, grabbed the bridle, surprised my mount by burning its front hooves, and prayed that I was a better horseman than Reno. The horse reared. Reno fell, and I held on for dear life. When the mount calmed, I steadied it, and waited for someone still permitted to see  to guide the creature back to the place where Reno lay cursing. 

He swore that he had broken his wrist. He swore that he would kill me. He called me a whore and a witch. He swore that I had made his horse throw him. 

"Really." I said, when I thought I could be heard over him. "It might be simpler to admit that a girl rides better than you."

His companions laughed at him. It would have been nice if I could have seen his face at that moment. Some victories are not meant to be perfect.

"What is going on here?" I heard the commander ride up beside me. 

"She threw me, sir." Reno cried.

"She?" The commander questioned. Apparently Reno rode a stallion.

"The girl is a witch. She grabbed my bridle and told the horse to throw me," Reno continued. 

I heard the commander sigh. "Help her down," he barked. I felt two pairs of arms help me dismount. "Heal him, girl," he said.

I stood still. Even if I had wanted to, to obey would be to admit that I was wearing a talisman. "I cannot, commander."

I heard him dismount and approach me. "Do not lie to me, girl. That horse has never thrown its rider. That is why you were sitting on him. Heal him."

I swallowed hard. "I cannot." I repeated stubbornly, not knowing what else to do.

The people around me hushed. I felt a finger, probably the commander's, swiftly run around my neck, searching for my necklace. Then I heard a ripping sound, and felt the wind on my bare chest. My shirt hung around my waist, revealing my talisman and so much more. "Heal him" the commander barked once more, and I tried desperately to cover myself. The circle of men roared in laughter again, this time at me. I knelt before Reno and prodded at his wrist. Tears of humiliation stung my eyes. I had not won the day. I had lost the war. Any hope I had of safety among these riders had disapeared with that rip in my shirt. There was nothing I could do but kneel blindfolded before my captor and heal him. Reno had not broken his wrist, as he claimed. It was sprained. He would not know the difference by the morning. Someone wrapped me in a blanket as I set to work. 

I sat alone in my tent well into the evening until my healer came to blindfold me. He brought me a new shirt and healed my wounds. His silence seemed deeper. His face was somber. 

"Are you in trouble?" He shrugged. "I am sorry." He shook his head and offered me the blindfold. "Where are you taking me?" He did not answer me. It was a hard question to answer, I suppose. He tied the blindfold as I expected. Then I felt my talisman fall onto my chest and his hands fasten it behind my neck. What should have been a moment of relief at having my own tool back in my possession turned into a moment of fear. "Will I see you again?" Would I now be expected to heal myself? He put a finger on top of my head and another under my chin and nodded my head for me. I smiled with relief. I did not know this man's name, but I did not want to lose his friendship in the company of strangers.

He led me carefully through a maze of catcalls and cook fires across the camp. I felt my face darken with shame. How could I possibly face these men again. They had seen everything. They thought owned me now. Even with my talisman, I was just a woman to them. They did not fear me as a healer. I could not imagine being safe in their ranks at this point. 

A tent flap brushed against my face as I entered my destination. "This is why I do not keep female healers. Do you have any idea how much trouble you have caused?" the commander's voice asked.

I bowed my head. "I am sorry, commander."

"The tower's women are atrocious. Take Ernesto for instance. I leave you together for four days, and you are friends already. You have no sense of what is decent behaviour. It is the gifted against the ungifted, is it? Or is there something more?"

I had not thought of it that way. I did not know why Ernesto had given me his talisman. I had thought him to simply be a generous man. "No, commander." I said. I certainly felt nothing beyond the casual friendship so common among healers for Ernesto. It was not a type of bond that gifted women formed easily with ungifted men. I did not know why. I hoped I did not hurt Ernesto by my answer.

"What should I do with you, Lysene?" When I did not answer him, he continued. "I suppose I should have done this from the beginning." I felt someone's hands behind my head. The blindfold fell from my eyes. Makala's ghost stood before me. No, not a ghost. He was real enough. Neither was he Makala, but the resemblance was eerie. He was taller and broader. That is not saying much. Makala had been a slight man. That is what made him such a good rider. The commander had the same high pronounced cheek bones, the same dimpled chin and his hair framed his face in the same manner, also falling to just below his shoulders. There the resemblance ended. Where Makala's eyes were warm and laughing, the commander's eyes were hard and cold. The same held for the ways they carried themselves. The were of similar ages. They could have been brothers. I wondered if Duke Ergino had fathered any sons the court did not know about.

"What are you staring at, healer?" the commander asked sharply.

"Nothing, sir." I averted my eyes.

"You will stay with me until we reach Bayam. We will negotiate a price for you then, and part ways. Is that understood?"

"Yes, commander."

"We will have to endure each other's company for five days. If you use that talisman of yours against me, I will not hesitate to have you cut up and personally feed you to the vultures. Do you understand?"

"Yes, commander." I did not dare look up. He had stripped me in front of all his men. I believed that he would present me trussed and stuffed to a flock of the Destroyer's falcons.

"Good. Then eat." I startled at the change in subject matter. 

He indicated a place for me to sit. Plates of food were brought in. The fare was no better than the previous night's meal, but there was more food, and more meat. Ernesto served silently, then left. The commander, named Griso, told me that I rode with the Hundred Horsemen. They were a group of mercenaries, primarily exiles, cast out from the armies or Towers of the various border duchies of Marsea. They hired themselves to ungifted kingdoms in need of extra men. They had numbered one hundred seventy four until four days ago when Sol and I each killed one of their members, commander Griso told me pointedly. They took any contract that did not have them fighting Marsea. Commander Griso, like the commander before him considered himself a son of Marsea. His men would not do anything to harm her, in spite of the crimes that they had been expelled for. It seemed to me that they fashioned themselves much like the hundred sons of Valir, with the black horse as their emblem. Commander Griso seemed to approve of my observation.

The night grew late. The Commander Griso indicated a corner for me to sleep in. I obeyed. I became known as Griso's whore among the men the next morning, though the commander had not touched me. Everyone knew him as Commander Griso. No one called him by his last name. The epiteth had its advantages. I no longer wore a blindfold. I gained access to all to the Horsemen's healers, not just the mute one. No one other than Ernesto spoke to me as we rode, in as much as Ernesto can be said to speak. I got my own horse. Commander Griso watched me like a hawk. He was present when I healed myself. He watched me as I rode. I had every meal with him. Though I could see the world around me, the confines of my prison seemed to have shrunk. It was a safer prison, yet I found myself longing to be bartered over and sold to the healer's at Bayam. Some would know me, I was certain. That would only help Commander Griso. He would be able to get a better price for me. Whatever the circumstances, I wanted to be rid of his company. 

Three days into my reputation as Griso's whore, we received news from the messengers from Bayam. They came to us in the heat of the day, riding hard. Commander Griso called a stop and met with his captains. I served the refreshments in my new role, then sat with Ernesto with a stick and a patch of dirt, asking him questions about his life. 

I asked. He wrote. We had taken to this ritual in the evenings after he healed me, sitting within view of the commander just outside his tent.  Ernesto had been falsely accused of slander. There had been some turmoil over who would become the next Head of the capital tower in the duchy of Allepo. Ernesto, then a new member of Allepo's army, and many of his colleagues, backed the losing candidate. He was strongly gifted for a gifted fighter. It did not surprise me that he had a foot in the tower's business. In a minor tower, he could have been a tutor. According to Ernesto, the new head spread rumors that he practised wild magic, then put the blame for the rumors on his opponents backers. Ernesto had his tongue cut out for slander. It was a severe punishment, but with the military involved, the new head wanted to make his point. Allepo would have kept him at his post, but Ernesto could not face his comrades as a mute. He had heard about the Hundred Horsemen, resigned his post, and went out to seek them. I admired his courage to venture out as a lone gifted in the lands of the ungifted hordes. But he was a gifted fighter. He was probably better able to take care of himself than I. Ernesto told me that Commander Griso took the company's oath not to attack Marsea seriously. He questioned every man in the company about their loyalty to Marsea. For every Marsean he took in, he turned away two. As a result, the company consisted of a lot more soldiers from ungifted lands than it had in his predecessor's time.

Commander Griso emerged from his meeting as the day started to cool. The crowd turned as one to face him. "I have elected to step down as Commander of the Company of the Hundred Horsemen." He announced. "You will ride to Cortan's aid tomorrow. The captains will choose a new commander tonight." Then he simply walked away to the edge of the encampment. The company burst into a buzz of excitement and questions, but no one went to disturb the Commander. The various captains emmerged, and each man flocked to his officer for instructions, answers, or to lobby for his favourite leader. 

Ernesto sat grinning beside me. "You get to go home," I suggested. He nodded. "Cortan is not the same as Allepo," I said. He shrugged, still grinning. I understood the sentiment. Life in a White Tower is so different from life anywhere else in the land. To a certain extent, it does not matter which Tower one is in. It is being in a tower that matters. 

Some time later, a horseman informed me that Commader Griso wished to see me. "You are Griso's whore," he said as I approached. "I announce my retirement, and you remain seated and laughing with your kind. This is not how my whore should behave."

"I--" I paused. I had no idea how to answer this accusation. I was not in actuality his whore. At least, not until that point. Was that about to change? "I thought you wished to be alone, commander."

Commander Griso's lips twitched in disapproval. "I want your company now. Come."

I followed him inside his tent, terrified of what would come next. He ordered a soldier to guard his privacy. I swallowed hard as I entered. I could feel myself shaking with fear. "Who are you?" he asked after I had seated myself.

"Commander?" I asked. I did not know if this was some new trick.

"What is your name, girl. You are no more Lysene of Cortan than I was born of the ungifted hordes. Who are you?"

It was over. I did not know what this man wanted of me, rather, I feared I did know what he wanted. I did not know if my true identity would save me or doom me. I licked my lips and hesitated. 

"I have watched you these last three days. You ride like you were born in a saddle. You heal yourself tirelessly and effortlessly. That is a strange combination for a healer. Then I hear that there was a third heir of Cortan, a widow adopted into the family by marriage. She is exceptionally gifted, and by the whim of her husband, trained for some years with the Black Riders. The girl is missing. I ask you again, who are you?"

There was nothing left to say. "I am she." I mumbled.

"Prove it." He rose abruptly and led me out of his tent to a cluster of tethered horses. "Down a horse," he ordered.

Horses are big, and they are strong. A well trained war horse can withstand significant injury and keep charging into battle. As a child in the Tower, we had competitions to see who could down the most goats. Horses are sturdier than goats, or men. Downing one would not be easy. I walked within range of a grey mare and burned her with all my strength. The horse reared and tore at its teather in surprise, then bolted across the field. Commander Griso looked at me with disgust. "I said down it, not chase it off. Bring her back."

I jumped onto the back of the first horse I found that was not panicked by the mare's performance, and rode after her. I returned as the sky turned dusky, exhausted. I had only just retethered the two beasts when Commander Griso said again "Down a horse."

I looked at him in disbelief. He wordlessly led me to another cluster of horses. "Try the guts."

I did not question his knowledge, but obeyed. It took every bit of strength I had left in me to inflict sufficient pain before the steed could bolt, but the horse went down. I sat on the ground next to it, panting. 

Commander Griso stood over me. "How old are you, Nisrita?"

"Fourteen."

"Impressive. Drink." He handed me a horn of sweet liquid. I felt better for having drunk it. When I could stand, he took me back to his tent, handed me a portion of dry rations and told me to rest. I did not need telling twice. 

The next morning, The Company of the Hundred Horsemen left for Cortan. Commander Griso and I remained behind with three horses, a week's supply of rations, and the tent we had slept in.

"Where are you taking me?" I asked after we had packed up our camp, with some trepidation. The plan of selling me to Bayam had disappeared. I did not know what had replaced it.

"Home," Commander Griso responded. 

"And where is that?" I asked, genuinely confused. 

"Where you lived until recently," Commander Griso said with annoyance. "Cortan, I presume, unless there is more I should know."

"No. But." I hesitated. I truly did not understand my situation. "Why am I not going with the rest of the Horsemen?"

"They are going to take the castle. You will not be safe with them. Firvona is looking for you. I will deliver you to a trusted family nearby. They will watch over you until it is safe for you to be known in Cortan."

"You will not enter the castle, or claim the reward for my return?" Commander Griso did not respond. He simply spurred his horse forward. I followed.

We traveled at a liesurely pace for two weeks, avoiding houses, hunting hares and small game among the grasslands. Commander Griso knew the lands well. He proved to be a worse conversationalist than Ernesto, speaking only to me in the evenings, when he showed an intense curiosity for my abilities with the gift. He reminded me of Makala in that respect. Makala had always shown an interest in my studies. He did not have the gift, but he was ever willing to let me demonstrate the newest skill I had learned at school. Where Makala's curiosity was passive and benign, Commander Griso's was aggressive and cruel. One day, he brought me a rabbit he had hunted. He had me heal it, then kept it over night. The next morning, he skinned it alive and asked me to heal it. When I could not do so without killing the creature, he looked at me with disgust. I tried to explain that rabbits were delicate creatures. They could endure more suffering than men, given their size, but they were small and hard to handle. He did not accept my excuses, and yelled at me for being ham fisted in my art. I was furious at him. Who was he, and ungifted fighter, to speak to me this way? I said nothing, however. How could I? He knew the lands, he was my hope for safety. He was a cruel man, but he looked after me. Every evening after that, he asked me to do the impossible, then blamed my skill or my trade for not being able to perform. He did not believe me when I told him that even the Masters of Cortan's Tower could not do what he asked. He simply became surly. Evenings with him put me in a foul mood. 

Days with Commander Griso left me with nothing but time to brood. I was returning to Cortan. I had left it because I thought I wanted to escape from Duchess Cybeline. That had been less than three months ago. I had been so innocent then. I lived a protected life that I had imagined to be hard, and wanted and escape. I knew nothing of the world. Now I returned and the Duchess's blood did not. How could I face her? She would blame me for her sons' deaths, I knew. She would find a way, though it was not my fault. Just like I found a way to blame myself for my widowhood. If I had only tried to convince Timmon harder, if I had found a horse and fought my way to the Black Riders, would Makala still be alive? If I had been a more dutiful wife, and prayed to the Preserver for his life as I ought to have done, would he still be here with me? Makala was the first family I knew. I had been friends with his siblings as a child, but it was he who called me into the structure of his family, inviting me to court for every holiday, asking me about my education, then teaching me what he had learned as a fighter. He found more time for me than any of my tutors at the Tower ever did. He adopted me as his own long before his father did. I was his little one, by virtue of being two weeks younger than Tonyo. I loved him and adored him as the older brother he had made himself into. His door was always open to me. He had not even thought to shut it when lying with Timmon. No wonder the poor man had been jealous of my hold on him. How could I have let Makala down? He would never have let me fall. I am a healer. My place on this earth is to save lives. How could I have let the one I cared for the most slip through my hands?

My dreams as we travelled were torturous. I avoided thinking of that day during my waking hours, only to find myself trapped in that battle every night. The mine would swallow me. Sometimes the long journey out of it would go on forever. I would pass soldier after soldier, wounded, buried at the bottom of the pit, extending their arms out for help. I would run past them. I had to get out, to warn Makala. The end of the maw never came. It always loomed, at the edge of my vision, around the corner, only a short, but never decreasing distance away. On other nights, the wounded soldiers would grab at my feet and tear at my robes. I was a healer of Cortan, they accused. I was in the army to serve them. I had no right to leave them behind to chase a man already dead. "Help me, heal me" they would cry. I could hear the Destroyer's drums in the distance, as he came in to claim these men for his own. "You are a healer," they reprimanded. "It is your job to keep us from His claws. You are deserting us." One night, I gave up on the idea of traversing the endless length of the rift. and decided to climb up. The earth turned to shear marble half way up the wall. The ground gave way below me, and I swung hanging from one lone remaining hand hold, looking down into an endless abyss. In the worst dreams, I sat by Lukos's side while Master Adele asked him who had wonded him with the poisoned arrow. Lukos was weak, fading in and out of conciousness. His head rolled back in Master Adele''s arms, he raised and arm to point to me. "Traitor" he cried in his mother's voice. Master Adele turned to me, her face now that of duchess Cybeline. "Traitor" she hissed. "You poisoned my family."

My nightmares disturbed Commander Griso's sleep. I tried not to, but I could not control my dreams. He would grumble accusatorily, as if I wanted to relive that day every night. 

We arrived on the fifteenth evening at a goatherd's cottage, a day away from the castle in Cortan. "We stop here." Commander Griso said.

"This is the family that will keep me safe?" They did not look like they could defend themselves against a large pack of wolves, let alone Firvona's soldiers. 

"You will be close enough to hear news of the castle. No one will think to look for you here. They will not sell you out. As long as you do not announce your gift, you will be safe here."

I shook my head and bit my tongue. It was better not to argue with Commander Griso. I would have been safer in the wilderness. At least there I would not be endangering another family.

"Chella," Commander Griso called out. "An old friend has come to visit. Open your gate."

A man a few years younger than Duke Ergino came to the gate. "Good evening, uncle. My mother has not lived here these past five years. But you are always welcome here."

Commander Griso cleared his throat harshly, and entered the gate held open by his host. A woman appeared from behind the cottage, carrying a basket full of eggs. Our host introduced himself as Groto, and his wife as Maya. When they asked for my accquaintance, Commander Griso answered for me. "Her name is Lysene. I need you to see to her needs for a while. Keep her away from Frivona."

Maya smiled and asked me to follow her into her house. Commander Griso called afer us. "The girl has been suffering from night terrors. See if you can do something about them." Maya laughed and tutted at me in a matronly fashion. I liked her instantly.

Griswold did not leave me at Groto's house, as he said he would. He was still with his friends when I left them. Groto only referred to this man half his age as uncle that once. He seemed to forget that form of address as soon as Commander Griso entered his gate, addressing him simply by his first name. Maya was a lovely, warm, solid woman. I spent my days in her company and that of her four children. She taught me to milk and feed goats, collect eggs and turn hay. I watched her infant son while she cooked. Her solution to my night terrors was a cup of warm goat's milk with honey at night after a day's work in her shadow, followed by a warm bed sleeping between her daughters aged ten and seven. As a cure for a malady the Tower cannot touch, it was quite effective. 

Griso did not spend a single night at the house. A different woman called for him every evening. Maya rolled her eyes and muttered "Griso's whores." I wondered at her courage, to mutter disparagingly in the commander's hearing. Maya would send me off to keep her daughters from seeing the spectacle Griso made of himself before he followed the nightly caller off the farm. I did not mind my duty of sheltering the girls' innocence. Commander Griso could have his whores. I considered myself lucky not to be among their number. 

***

I had stayed with Groto and Maya for four days when I heard that Duke Ergino held Cortan. I bid farewell to Maya and Griso. Groto accompanied me home on the fifth day. I stopped my horse when I saw the east tower loom darkly over the horizon. The setting sun put it in shaddow. Cortan's flag hung limply atop it, in the still evening. I wondered if it too was uncertain of my return. The White Tower lay directly behind it. It would be glittering orange and gold in the sunset, if I could see it. I could not from this angle. I did not want to be home, I realized. I did not know where I wanted to be, but I did not want to be confined in those dark stones.

Groto watched me hestitate. "Is anything the matter, your grace?"

"No." I said. I returned to Cortan without Makala. Everything was wrong. How could I face my parents? Groto still watched me. I rode forward.

By the Preserver's grace, the castle expected me. A guardsman led me to a private chamber where the Marshall verified that I was who I claimed. He paid Groto for my delivery, then questioned me about my doings for the past three weeks. I told him of my experiences, leaving Commander Griso's role out of the story, as he had requested. The Marshall could not permit me to go to my chamber. I bathed and changed in the guard tower, and readied myself to be presented at court. 

If I had feared the shock and discomfort of facing a thousand eyes peering at me expectantly from above their starched and embroidered collars after a month of travelling in solitude or in the company of rough men, I need not have. Less than two hundred people waited me in the great hall. The room was less than half full. There were no signs of joy or festivity on the walls. The mood was somber. Cortan had just lost its two beloved sons. I understood my duchy's grief. I too entered in a widow's veil and a simple dress. The sobriety of the room was more than just grief, however. Cortan did not wish to welcome me. I was the adopted heir, forced upon them to replace Duke Antonio, Cortan's only gifted son, who had died too young. It was Makala who wanted me to be a part of this family, not Duke Ergino, certainly not Duchess Cybeline. Makala was gone. Who would want me now?

"Welcome home, daughter," Duke Ergino said when I had walked the long path past the rows of empty tables then past the guests and up the steps to the dias. 

"May the Preserver bless you, father. I grieve for our loss. I hope I may serve you well in their stead." I said, bowing low. He extended his hand stiffly for me to kiss. When I went to receive my mother's welcome, she did not say a word. She left me bowing awkwardly before her a long time before she extended her hand to be kissed. The priest welcomed me warmly, as was his duty, and thanked the Preserver for bringing me safely back to Cortan. 

We sat and ate. I ate slowly and spoke little. I had been coming to this hall to dine with the Duke's family since I was a small child. Years of practise have done little to make me feel more comfortable in this setting. Makala had forced me to agree to meet with a tutor to polish my rough edges. I had not wanted to, but Makala could be very insitent. The lessons were to start when we returned from the campaign. It was too late for that now. I sat in an unpolished silence before the Duke, as was my habit. Even before marrying Makala, the Duke terrified me. It is not that he was a cruel man. Makala certainly did not think ill of his father. Rather, it was because I was a child, and he my Lord. I knew that the Tower had not bred me for this table. I did not wish to displease him by my comporture. I had always kept myself silent in court, never at ease in that setting. I was a guest, it was not my place to speak unless spoken to. That day, few spoke to me. Mother shot daggers at me with her eyes. Father never spoke much to me, usually preferring the company of his sons. Not having them by his side, he ate in silence. 

A few bravely curious advisors, officers and ministers at court asked me about my time away. I avoided speaking about my time since Turina as well as I could. I did not think my adventures riding with the Hundred Horsemen would do anything to soften my mother's feelings towards me. I answered questions about the campaign as best as I could. I knew little of military tactics, but I could describe the terrain and the events of the march to Turina well enough to please the officers at the table. 

I learned that the Hundred Horsemen had arrived in Cortan a week before I met Groto and Maya. Their first step had been to dig a well to the undergroun stream that watered the castle and pour a week's work of horse manure into the hole. When evidence of this contamination started appearing in the castle's well, three days later, Firvona's command had a hard time contolling her troupes. No man wants to fight where he risks death by diarrhea. It is a dirty lingering death. Two days later, when Cortan's forces from Bayam and beyond returned to the castle, they broke and fled. 

I put down my food, with a feeling of dread growing in my stomach. "The castle's water is soiled?" I exclaimed weakly. What had I done? Commander Griso had sent aid to Cortan, but he had also destroyed her. 

General Galderan cleared his throat and shifted his weight in his seat. "The duchess does not know the story of this castle's founding," said my father, curtly.

General Galderan explained. The Duke's father had chosen this site for his stronghold because it sat on a junction of a network of underwater streams that made this part of the dry grasslands arigable. The castle drank from one stream, the army and the White Tower from another. The castle would now draw its water from the Tower's well until its own cleared. It was a well guarded secret until the Horsemen had forced Cortan to reveal it. Until last week, General Balderan knew only of six men living who knew of this strategic strength of Cortan's castle. He did not think any of had spread the information. It was still a matter of debate at the table whether the Horsemen had known this strength of Cortan's water, or if they had wrecklessly poisoned the castle, weakening Firvona, but possibly killing Cortan in the process.

I had nothing to add to the debate, though the question troubled me. The Hundred Horsemen claimed they were sons of Marsea, that they would never hurt her. That had been under Commander Griso. Now Commander Belkame led them. How did he feel about Cortan? Did he have the same loyalty towards Marsea? The horsemen had just poisoned the castle I had set them upon. My home. I looked at Duchess Cybeline. My mother stared at me with accusing eyes. Did she think I had poisoned Cortan? It was by sheer luck that the Horsemen had not destroyed everything.

"How many have fallen ill, General?"

General Galderan smiled patronizingly at my concern. "Only a half dozen people from the kitchen. We have had to kill a few livestock and horses as well. You have nothing to worry about, your grace. We are now drawing water from the Tower. You are safe."

It was not my safety that concerned me, but the great harm I had nearly caused my people. "And the Horsemen?" What would Cortan do with these traitors I had sent back to her?

"Do you see now, Ismelda?" I heard my mother's voice say to the General's wife. "I am cursed with a daughter who cares more for a band of brigands than the fate of her own family. I fear for Cortan when my good husband is called to be made afresh, though that is many years away."

This was why I disliked conversing before the court. Conversing with nobles felt like playing chess. I had to be constantly aware of all the other pieces on the table, of the moves they could make, always looking ahead to guess how they might react to a statement or which way a conversation might turn. It was not like the conversations of the Tower, where one could get lost in the excitement and detail of a philosophical or physiological discussion. I turned to my mother before the general's wife could respond. "Forgive me, mother. I am overjoyed to see you and father safe and healthy. I had not meant to cause offense." Duchess Cybeline snorted and returned to her conversation with the general's wife. My stomach knotted and the blood drain from my face. There was nothing I could do to please the woman. I was completely and totally unworthy of my postion in her eyes. I had gone away to be away from this woman. Now I had returned, but I still did not know how to live with her.

General Galderan chose to ignore my discomfort and told the company the story of the Duke and Duchess's rescue. The entire table entered the converstion, and I returned to my position as listener. After poisoning the castle's well, the Hundred Horsemen spread out to search for the Duke and Duchess. Firvona could take the castle at Cortan, but they did not dare assassinate a Duke of Marsea. It did not take long for eighty five pairs of horsemen, many of which containing one man who knew the terrain well to scour Cortan and Firvona for my parents. The Duke and Dutchess were found safe, if uncomfortable in a small keep just north of Firvona's border with Cortan. The Horsemen notified Cortan's troops who took the tower with little difficulty. Most of Cortan's army that was not in Turina was engaged with Firvona. The neighboring border duchy of Allepo, and the interior duchy of Silet that shared a border with Firvona, had sent troops to aid our cause. Makala had sisters married to the dukes of both those families. They were coming to their brothers' aid. More troops from Bayam and beyond arrived nearly daily. They would make the difference in this fight against Firvona. The General Galderan could not praise General Madriano's brilliance enough for having guessed at the castle's troubles. The Preserver must have granted him a second sight, he claimed. I refrained from informing the General of Timmon's role in Cortan's rescue. I had told the Marshall of Timmon's role. Word would reach the General's ears eventually. The fewer words I said at the moment, the better. 

The White Tower was empty, all but a handful of healers supporting Cortan's troops to the north. It was being used to imprison the Firvonese men who had taken the castle. Gissal would see to securing the lands Cortan had just fought so hard to win. The Duke did not sound pleased with the situation. He made a mysterious comment about Gissal stealing Lukos's lands. 

Many of the Hundred Horsemen were still filtering back from their search to collect their reward. If the Duke deemed that the crime for which a man was exiled to be forgivable, he would lobby for his pardon from the Queen Regent, so that he may return to Marsea. Those from without who wanted to settle here would be granted a position in an army of an interior duchy. Everyone else would be sent away with a payment for their services. 

Firvona, according to Baron Farone, still held a grudge against Cortan over a piece of land that we claimed from the ungifted hordes. I did not understand most of the details he gave regarding the politics involving borders and border duchies versus becoming interior duchies, though most of the men around the Duke's table seemed to have an opinion on the matter. I think the main argument went as follows, though there was much that I missed: Marsea needed to expand. Firvona wanted a share of the glory showered upon the duchies that push Marsea's boundaries outward. Moreover, the Duke of Firvona had taken the manner in which his last stretch of border was turned into an interior border of Marsea as a personal insult, though everyone was a bit vague about the events that had taken place when I was a child of eight. I did not ask. 

It was surprisingly easy for Firvona's armies to take Cortan. When Makala had arranged a marriage between his brother Lukos, and the Firvonese duchess Rochilda, Cortan could not but treat Firvona's envoy as honoured guests. Timmon's theory, it seemed had been exactly correct. When the young duchess came to visit, her honor guard learned the weaknesses of the stronghold. General Galderan's words suddenly became short and cold at the discussion of Makala's negotiations with Firvona. I learned much later that he thought the decision to arrange a marriage between Firvona and Cortan to have been ill condisidered on the part of the late duke. He never said anything within the hearing of anyone at the Duke's table.

As part of Firovona's pledge of support, the duchy sent supply wagons and soldiers to Cortan's barrack from Firvona, and back from Bayam. No one noticed when a few extra wagons came in from Firvona, or failed to return from Bayam. These things happen in a campaign. Two exceptionally large groups of soldiers and ammunition wagons appeared from Bayam and Firvona on the same night that a Firvonese locksmith and a squadron of men entered the castle through the narrow tunnel that connects the cellar of the White Tower to the cellar of the castle. They opened the castle gates from the inside. Men poured out of the wagons of wounded soldiers from Bayam into Cortan's castle and barracks. It was over by dawn.

It frightened me, how easily it all seemed to have been done. I had always heard Cortan called the strongest of the border duchies, yet she had nearly been destroyed. If the men were frightened by this fact, they did not show it. I listened to the conversation continue to speculate about negotiations with Gissal, and how the Queen Regent would decide in this conflict between Cortan and Firvona when the matter came to her court. 

Dinner ended, I bowed to wish my father and mother goodnight. Duke Ergino returned my wishes with a formal coldness. Duchess Cybeline did not. She had not said a word to me since I had arrived. How would I possibly live with this woman?

This time, I was allowed into my chamber. My heart lifted a fraction on an inch to see that the guards at the door to the part of the castle holding Makala's an my rooms had not changed. I could not take refuge from my family in the White Tower. It housed Firvonese prisoners. Everyone friend I had was either dead or at war. The halls were empty, most of the castle's residents engaged elsewhere. The best I could do for company and comfort was to seek solace in the few familiar faces still left in the empty castle. 

My room looked sterile. Most of my belongings sat in crated and trunks, ready to be cleared out in the hopes that I too had died with Makala and Lukos. I was certain that this was my mother's doing. I had not been allowed in when I had arrived because the room was not fit for habitation then. I looked through the trunks and started arranging my possessions in my room. I found two trunks of my books and my instruments from the Tower, one crate of clothing, including my white robes, and a smaller one of other possessions. It seemed that the contents of my chamber in the Tower had also been cleared out and brought here. 

I put away what I could, working to keep my mind busy against the encroaching walls of guilt, shame, grief and desolation that threatened to smother me. How long would they wait for me to return, I wondered, before they purged my possessions and my memory from this castle and its tower, and rid themselves of an unwanted heir? Sergent Sol Nicose and six kitchen boys appeared behind my eyes. I had caused a good man to die. I had made six innocent boys ill. Perhaps mother was correct. I did not deserve to be an heir of Cortan. I did not know how to lead or rule. What could I do to this great duchy but bring it to ruin? I was just a woman and a healer. I knew nothing of politics or war. It was too late to run away to the Tower and dream of heading the Woman's section one day. I had shouldered this duty when I married Makala, thinking only of how nice it would be to officially be part of his family. I had not dreamt of this. I had been so naive then.

I beat the dust off my robes more vigorously to keep the flood of tears from my eyes. One of my guards announced that there was a mute soldier to see me. I left off my beating. Did I wish to see Ernesto? He had helped poison this castle. Did I really wish to see that type of man? The Hundred Horsemen were exiles and outlaws. They were cruel and brutal people. I had unleashed them upon Cortan. Ernesto was different, someone said inside my head. He had deserted from shame, I had called him a friend once. I grabbed a sheaf of paper and ink and left. I would meet him publicly. I would not have my loyalty to my dead husband questioned.

"What do I do with you?" I asked, after I had offered my hand to his bowing form. He looked at me questioningly. "You and your men poisoned Cortan's well. You could have killed every single man in this castle. Is nothing sacred to you?" Ernesto looked shocked at my words. He shook his head against my accusation, then saluted. It was not he who had dug the well, it was the commander who had ordered it. I was furious at him for passing the blame like that. "And your commander only gave the order because I told the Horsemen that Cortan needed help. You may as well blame me, then, for poisoning this castle's water. I am sure Duchess ..." 

He made a curious gutteral sound that stopped my words. He took ink and paper from my hand and wrote three words in a large clear hand: \emph{Not your fault.}

I stared at the words and then at the writer. Someone had poisoned the castle. I could not blame Firvona for this as I could Makala's death. The commanders would only the the glory for weakening Firvona. The soldiers would hide behind their duty. Who then was left but the inadequate heir that had poisoned her own people? Ernesto returned to the page as if he had read my thoughts. He wrote the soldier's order. I knew the words. Everyone who worked with the army did. \emph{A soldier obeys// What officers order // On the King's behalf// According to the rythm of the Destroyer's dance.} He pointed and me, and then at the word soldier.

"No" I said, and pointed at the word King. I was Cortan's last heir. I was responsible for her wellbeing. Ernesto sighed deeply then wrote the word: \emph{woman}. Then he pointed at the three words at the top of the page: \emph{Not your fault}.  

It was kind of him to absolve me of my responsibility. If this had been the house of my father by blood, it would have been easier for me to accept. I was an outsider here, my actions were judged harshly. It did not matter if Ernesto did not hold me responsible. There were plenty of Cortan's men that did. I changed the subject. "Will the Duke lobby for your pardon?"

Ernesto shook his head and grinned. Then he pointed at Cortan's insignia on his shirt. He had not been exiled. He was a deserter. The Cortan had the right to take him into his own army. "Congradulations." I said. His grin widened. "Will you stay here?" I asked. He pointed to the south east. "Turina?" He nodded. Newly conquored lands are an opportunity for men of ambition to make a name and a fortune for themselves. I wished him luck. We spoke for a short time longer about the three weeks since we had not seen each other, then parted company. I retired in the small comfort that I had one friend here.

I awoke with a start before dawn to the chorus of one hundred accusing fingers pointing at me from the bottom of a dark abyss, shouting "poisoner." I lit a candle when my hands stopped shaking and took it to the window. I could see the Tower and barracks from here. The Tower was uncharacteristically dark. The infirmary was empty, the children had been sent home. No one rose before dawn to attend morning prayers before breakfast, or to relieve the night shift at the infirmary. Still, the Tower shone a grey-white against the deep blue sky behind it. It was a beautiful sight. It was more my home than these rooms. Makala's room on the other side of the wall held the same view. I wondered if that was why Makala had chosen to live there. 

Of course not, I realized. He must have valued the view of the barracks. I wondered if he stood at his window at night looking for Timmon among the antlike people walking the walls of the complex. They were so in love. It did not matter that the priests called them miscast. There was nothing wrong with Makala. That much was certain. I justified that this meant there was nothing wrong with Timmon. Theirs was a love story out of song. By the time I realized what they had, I found myself wishing for the same. Not from them, of course. I could not imagine loving Makala any differently than I did, and to love Timmon was laughable. He had eyes for no one but my husband. I did not know who, but I wanted someone to care for me as they cared for each other.

I sighed at the memories of the three of us walking among the winter grasses, wanting to return to that world where I was protected and watched over by the friendship of those two men, and the Tower. The candle flickered at my breath. I would not get them back. The best I could do was to salvage a few keepsakes of Makala's belongings as momentos of that time. I did not want to enter Makala's room. I did not want to face its emptiness. The Duchess would not grant me the widow's curtesy of permitting my to take a few items of my husband's possessions for my own. It was best to go in now and claim the items before the castle awoke in the morning. 

If I had found my room to be sterile, Malaka's room spoke of death. All the hangings and curtains had been removed, as well as the matress and cushions of his chairs. The bed frame sat naked and awkward in the middle of his room. It seemed to be grieving its lost matress and the man who slept on it. Everything lay empty. His table did not hold the collection of small trinkets he collected from all the lands he had traveled. His walls were not adorned  with his hunting trophies. His shelf did not hold his marble carving of the dance of the gods. He had been a man of war. The Destroyer's figure in that had become black and sticky from years of annointing when he prayed for strength, success and victory. It made the god look fiercer than ever. I remember it scaring me as a child. 
 
Instead of his belongings, a fine layer of dust lay on every surface. No one had been in here in days. Makala had always been an impecably neat person. The dust seemed a final declaration of his passing. My heart ached for him. An emptiness filled the room, claiming the chamber for itself, barring me from crossing the threshold. Makala was gone. I was alone. Nothing in the world could change that fact. 

I sat in the doorway and wept for the first time since inspecting Makala's body on the battle field a month ago. There had not been time to cry before. Now it seemed there was nothing left to do. 

A predawn twilight streamed through Makala's windows by the time I had recovered myself enough to look through his belongings. I found his ceremonial armour, and took a richly decorated knife from it. The hilt was made of a dark wood, with ivory and pearl inlaw. I had always admired it. The blade was amazingly sharp and finely balanced. It would be perfect for throwing if I did not fear losing the weapon. I combed through his other possesions, and found nothing else that I wanted. I contemplated taking the stone figure of the dancing gods, then decided against it. I had my own more staid depiction of the gods I prayed to. I annointed the Preserver as most people do. After all that had passed this last month, I did not think I could take put an item favoring the Destroyer on my altar. 

I found the set of beautiful chess pieces and a ring that Timmon had given Makala. I added them to my collection. I would give them to him when I saw him next. Timmon deserved this curtesy as much as I. I did not know what Timmon would have wanted for himself, so I guessed at those two items and left. 

I had dressed and readied myself for breakfast when Dutchess Cybeline stormed into my room. "You must think you are so clever," she said. The first words spoken to me since my arrival.

My heart beat furiously, and lodged itself in my throat. I bowed to await her hand, greeting her formally "Good morning, mother. May the preserver grant you health." I did not know what I had done.

"I do not care for your prayers, girl. My son's body is hardly cold, and I find you blushing and laughing with one of your kind already. You love your Tower before your house. You will never be one of us. You made a spectacle of yourself and shamed this house last night. I should have you disowned." 

I stood bowing before her, and let her unleash her anger. What else could I do? I could not rise or speak until she had given me her hand. I wondered what she would do to me. Last winter, she had barred me for going to the Tower as a punishment for miscarrying. What would be her punshment for unbecoming behaviour? My vision blurred with tears when she brought up the subject of poison. She accused me of poisoning the castle, of being a poison to the house. I was shaking when she accused me of killing my husband and Lukos. I would not sob before her, but I could not keep my tears from rolling down my downturned face. She finished her tirade with the declaration that I would no longer stay in this tower, but my belongings would be moved to a room in the suite she controlled, so she may better keep an eye on me. Then she left without offering me her hand, or accpting my prayer.

I sat down heavily at my table and dried my eyes. How was I supposed to survive this? When the battle for Cortan finished, I would have to find a way to seek refuge in the Tower. How could I survive living under her nose? My simple existance seemed to be enough to set her off. I could not turn myself into a mouse and hide from her view. Only the old magic could possibly grant me that luxury.

The paper from last night's problematic encouter lay on my table. Across the top in a large legible hand lay the words \emph{Not your fault}. I shook my head in disbelief. I could not believe the words. It gave me courage to know that someone did. 

I reached over to my book shelf to pull out a book to pass away the morning. I would not join the Duke and the Duchess for breakfast. I was not welcome there. I had lost my apetite. On the top shelf was the original set of Griswold's ten tretises. I had admired them as a child, and looked at them when I could in the Tower's library. Griswold had founded Cortan's tower. He had been an accomplished and wise healer. I admired the man immensely. By he time he had disappeared at just over thirty years of age, he had already written ten treatises on subjects ranging from warfare to healing delicate bones. I could not imagine ever having such a prodigious and varied intellect nor ever being such a prolific scholar. I had become interested in the human skeleton from reading his writings. When I married into Griswold's house, the Tower had given me the ten original books as a wedding present. I valued them above all other possessions. 

I pulled a book from random and read. Rather, I tried to read. My eyes passed over the words, my mind unable to string them together into meaningful sentences. If flipped through the pages and studied the illustrations. Something tugged at the corner of my mind. I had seen this hand in a different context.

Suddenly, the world shifted. I understood how the Horsemen had known that they could safely poison the castle's water. I understood why Groto had escorted me to Cortan. I knew why I had seen Makala's ghost among the Horsemen. Why I had to endure those infuriating tests of my skill on the road. There were still a lot of things that did not make sense, the age, for instance. I knew that old magic was involved somehow. Details like age are unreliable. I knew what I had to do to save Cortan. It was a much better use of my time than sitting unwanted in this castle.

I packed Timmon's chess pieces and ring in the box with my wedding jewelry. I put the key in a letter explaining its purpose and sent it to the barracks for his return. I did not know when I would be back or when he would return. I tied Makala's knife to my belt, found the horse I had carried me for the last month, along with the armour I had left Turina with, and a some food and water for the journey. I left the castle to find Griswold.

***

It was much easier than I expected. He was cutting hay in the lot behind Maya's hen house. The sun was an hour from setting when I arrived at the goatheard's house. Groto came to the gate to meet me. When I asked him if he knew were Duke Griswold was, he pointed to a sickle weilding man in the fields beyond his house. He did not blink at the name. He took my horse and asked if anything were the matter. I begged permission to speak to his guest and to impose upon their hospitality for one more night. Groto welcomed me warmly. 

Commander Griso was less pleased. "What are you doing here?" he asked curtly, standing up from his labors.

I bowed before him. "I have come to ask you come home, your grace."

Duke Griswold threw the sickle at my feet irritably. When I did not rise, he quickly gave me his hand and pulled me up. "Stop that nonsense. You will get us both killed. Come inside, girl."

Maya set me to work as if I had never left. I watched her children while she worked, and ran what errands she needed running in the hour until dinner. Groto and Duke Griswold sat at the table talking quietly. It was as if the last day had not occured. When the children, and Duke Griswold's nightly caller had been sent away, he turned to me and said, "You have come in vain. I cannot return."

"Is it because of the prophecy?" I asked stupidly. 

Duke Griswold laughed derisively. "That twaddle about helping Cortan and destroying myself. Is that still circulating?"
 
"It was rather a good story," Groto said with admiration. 

"Thank you, Groto. You and the Tower seem to have done a thorough job keeping it alive." Groto looked pleased. 

I did not understand what I was hearing. "You spread the prophecy?"

"I joined the Hundred Horsemen because I was exiled, girl. I had to leave Marsea. I cannot return if I value my life. I was a young man with a flair for the dramatic. The prophecy seemed a nice veil of mystery to exit under."

"Do they still tell the story of the Duke becoming a wraithform?" Groto asked.

"Some do." I admitted.

The Duke laughed again. "I suppose you believe that too." I flushed. I had believed it once. I had not known what to believe of the great mysterious Griswold. The Duke turned to me sternly. "Listen, girl. You have promise as a healer. You cannot wander through life believing in children's ghost stories. The gift only effects living matter. It cannot bring dead matter to life, or change flesh to stone, or air, or spirit, or anything else. Do not join the crowd of fools that believe they can use the gift to bring back the dead." 

I took his words seriously. Duke Griswold was a stern, commanding man. It was hard not to fear him. Yet a part of me thrilled a the idea that the great Griswold had just lectured me on the gift. I could learn so much from him.

"The Duke, your nephew, will lobby for pardons for those who helped Cortan. He still loves you. He would push for a pardon of their crimes, whatever they are."

Duke Griswold smiled grimly. "Ergino will have little sway on the Queen Regent for my crimes."

I could not imagine what the good Duke could have done to earn the crowns wrath. I stated at him blankly. Maya relieved my confusion. "When the king's aunt was around your age, your grace, the duke, in his lust filled wisdom, took her for a lover. She was gifted, and did not realized she was pregnant. She lost the child while returning to Deyalorn from Cortan. The royal healer alone could not save her." I shuddered. Every woman knows the dangers of practising while pregnant. Every woman who might be pregnant is taught how to check. The test is not reliable. This is why many women give up practising altogether when they marry. The risk to her life and that of her child is too great. 

"Do you see now why I cannot return?" Duke Griswold asked.

"No." I said stubbornly. "Cortan needs you now. She needs and heir who knows how to rule. Given the trouble between Cortan and Firvona, the crown will overlook your past. I cannot help my duchy. At best I can hope to marry a man who can rule, leaving Cortan in the hands of an outsider."

"Look at me, girl. I am over sixty years old. I do not age. Ergino would be a fool to name me his heir."

"Come as an advisor then. You cannot let Cortan down now," I persisted stubbornly.

"No." Duke Griswold sighed and looked at his hosts. "It is clear that I cannot stay here any longer. I need one day to settle my affairs. Then I will leave you. It may be some time before I can visit you again." He looked at Maya with more tenderness than I had thought him capable of. "Keep the children well."

"I will go with you." I blurted out after Groto and Maya had accepted this news. Now that I had found Duke Griswold, I could not lose him again. Even if the prophecy was a hoax, Cortan need Griswold. I would not return to the castle again without the duke's blood.

"As what, girl? Go home."

"You cannot heal yourself." It was a guess. I had not seen him practise his art on the road. He had asked me to tend to both our minor injuries.

Duke Griswold looked at me as if considering the proposal. "You belong in a classroom in your tower. The road is no place for a young woman."

"You could teach me," I insisted. It was an alluring idea. I had never dreamed of being in the tutelage of the great Griswold himself. There was so much I could learn from him. 

Griswold looked at me long and hard. "Come with me to the dell tomorrow morning. If you still wish to travel with me after what I show you, I will not stop you."

"Griswold, she is young enough to be your grandchild." Groto protested. "What you do to yourself is your business. Do not involve her in your habits."

I looked around the table trying to figure out what I would be involved in. The Duke answered Groto slowly and coldly. "It is her choice, Groto. Let her make it." He rose and left the cottage. Groto stormed off after him. Maya pursed her lips resolutely. Then she rose to clear the table.

Not knowing what else to do, I helped Maya with the dishes. When we finished, she said "You do not have to go tomorrow, your grace."

"What will he show me?"

"Griswold has been collecting mushrooms while he has been here, your grace." I began to see. There is a certain type of mushroom that is said to unlock the gate to the old magic. Maya continued. "Do you know how he came to us?" I shook my head. "My mother-in-law, Chella ran a brothel not far from here that Griswold frequented. One day, when Groto was a boy of ten, on of Chella's girls found his body lying in the grasses. The girl took him for dead, but when they brought him back, Chella found him to still be breathing. She nursed him back to health for six months before he could return to his Tower. He was in a terrible state, nearly mad, when he recovered enough to speak or move. Groto can tell you stories of that time that will give you nightmares for weeks. You have enough trouble with your dreams. I will spare you the details. It was then that he stopped aging. He also lost his gift. He has been returning to the mushrooms every few years since, trying to get back his power. I think he would trade his long life for his ability to weild the gift if he had the choice. I think he is ambitious enough to try to have both. Every time he toys with the mushrooms, he ends up desperately ill at our door."

Old magic is powerful. It is dangerous. It is banned by one of the first laws of Marsea. I had heard the rumors that Griswold had practised the old magic. I had never believed the man I respected so much to be capable of such an act. "What does he want with me," I asked. Commander Griso scared me. What I justed learned about Griswold made me fear the man even more. I was starting to hesitate in my conviction that I would travel with him the day after tomorrow.

"I cannot say, your grace. It involves the mushrooms, I promise you that. Be careful with them, child. They are very dangerous. It is not just ambition that has kept Griswold returning to their power over and over again. They hold him in their grip for months, until he needs to consume so much to control his madness that he nearly kills himself." Maya changed her tone suddenly. "Now, enough of frightening stories before bed. You are not yet cured of your dreams, I gather." She handed me a cup of warm milk.

"Are you one of Griswold's women?" I asked, sipping my drink.

Maya laughed softy. "You are a perceptive girl, your grace. I am his daughter, by one of Chella's women. I have many half blood siblings by him. He looks after the ones he knows about as well as he can. For all his faults, he is devoted to his blood. That is why his prophecy is not entirely a myth. Griswold cannot let Cortan go as long as his kin rule. Nor can he bring himself to return to her. He hovers dangerously around her edges, wanting to help, desperate to hear news. If you do travel with him, keep talking to him about Cortan. He wants to go home. He needs someone to clear the way for him."

I thanked Maya and went to bed. I had made my decision.

The next morning found me in a shallow cave at the bottom of a valley, a hour's walk from Maya and Groto's home. Duke Griswold was boiling water in a small bowl. He produced a thin black leathery disk from a pouch, tore of a quarter of it, contemplated it, then tore it in half again. He dropped the wedge into the bowl, then lifted it off the fire. He waited a while, then removed the wedge from the tea, studied it, then threw it reluctantly into the fire. He poured the tea from the bowl to cup. I watched, mesmerized, as a mouse watches a cobra. 

"Do you wish to come with me?" the duke asked. I nodded, speechless in fear. "Then drink this." He pushed the cup towards me.

I drank the hot liquid. It did not taste foul. Slightly earthy, but nothing extraordinary. I did not know what I had expected. Nothing happened when I finished. I looked at the duke questioningly. "What is supposed to happen?"

"You tell me."

I waited. At first, the changes were imperceptible. I thought I was imagining them. I thought I heard the music of mountain birds in the distance, which was impossible. We were days away from the southern highlands. Then my vision started to blur when I moved my head. "I cannot see straight." I said.

"Then sit still and do not move" I heard Griswold's voice say. I turned to look at him. There was a supply wagon where he had sat. I blinked and rubbed my eyes. My sight did not change. Beyond it, I saw the high valley of the battle before me. The battle had not yet started. I jumped to me feet. I still had time I realized urgently. If I could reach him now, he could be saved. The battle had not yet started. I could still get to Makala. "Do not move." I heard Master Alerio behind me. He put a hand on my shoulder. 

"I must," I cried. "I have to reach my husband. Do not deny me this second chance. I have to save him this time." 

"Is that really what you want?" Master Alerio asked.

"Yes!" I cried. "What else is there to want."

"Then go forward, but do not move." I shook my head. I did not understand what he told me to do. "Go forward, but do not move. Quickly, before you run out of time."

I did as he said, holding my body stiff. The landscape flew by me, as if I were running, but I did not move. I ran until I could make out individual voices from the ranks of our men. The Black Riders were in the front. I had to push through the men to get to him. I started shoving and elbowing my way through the ranks of infantry. The turned to me and cursed me for haste. "Makala!" I cried out. "The Duke, please call the duke for me." Some soldiers heard me, and passed the message forward. Then the General's horn blew. People started streaming past me. I was too late. Take it all, I had been given a second chance, and I was too late. I started running with the line. 

Suddenly, the earth shook under my feet, I heard a deafening noise, and saw the earth crumble underneath me. I screamed and fell, scrambling for solid ground. Timmon pulled me across to the other side. "Sit" he said. "You will do no one any good if you fall to your death." I obeyed.

Timmon spoke again. "What is it you want?"

I looked at the cluster of Black Riders. They were the only mounted men on the field. "To save Makala. I have to get to the Black Riders."

"No." Timmon said firmly. "You cannot join the battle. You are Cortan's only heir. You cannot die."

I looked at him in desperation. How could he not want me to save his lover. Makala was still alive. I was not yet the only heir. Timmon's face grew black and pusy, as if it had been poisoned already. It grew to fill the field of my vision. It smelled of death and rotten meat. His eyes rolled wildly in his head, as if they were marbles swiveling in a cup, and not eyes in a socket. I cried and shrank back from him. He stepped back, "The fear is setting in. You have to act now. You do not have much time. What do you want?"

"To save Cortan," I said. If I could not save Makala, I had to do something.

"Then look," Timmon pointed. "There falls Lukos. Save him."

I followed his finger to see Lukos' body twisted at the bottom of the ravine. I held myself stiffly and climbed down the ravine. He was not badly injured. I turned him over and found the healed wound in his back, black as pus filled. I cut it with his knife to let the blood out. He would not bleed. "I cannot do it," I wailed. 

"Yes you can." Master Adele aproached from behind my shoulder. "You are the only one who can now." 

"Tell me how" I said desperately. Lukos ellapsed into a fit in my hands. 

"Cut the wound with this knife." I took the knife she offered me, and cut Lukos deeply. His seizure worsened. "Now use your gift to draw the blood out. Tap your power slowly. It is too easy to draw down too much fire in your current state. There is a door where the gift enters you. Can you open it?"

I focused on the gift. I felt the door. There was a weak spot in my skin. Something pushed against it to enter. "I feel it." I tugged at it. Nothing happened. "I cannot open it." 

"Try again," Master Adele said.

"No. I must save Lukos. I will not get this chance again."

"Forget Lukos, girl. Open the door."

I ignored Master Adele and focused my gift on my brother. The gift came through me as I expected, then something slipped. The fire surged through me, uncontrollably, burning me, burning Lukos, spraying blood everywhere. I could not stop it. I put my hand over the wound to stop the jet of blood spraying from his body. I could feel the stream hit my hand and ooze around my fingers, like water being squeezed from a skin. Lukos's body went limp. "No." I cried "I am failing him. I am killing him. I am not a murderer."

"The open the door, girl." Master Adele ordered. "Use your gift." I turned my head to see a physical door beside me. I used my free hand to pull. It budged a fraction of an inch. "Put the blood back." Master Adele continued. 

"That is impossible." 

"Put the blood back," she insisted "before Lukos dies." I looked down at Lukos's pale form at my bloody feet. I called the gift down gingerly this time. I was tired. I had used too much today. To my astonishment, the blood stopped flowing out of him. Mistress Adele poured water down the duke's throat. I put my hand over his chest. His heart beat was very faint, then slowly, very slowly, it started beating regularly. I exhaled in relief. 

"Stay focused," Master Adele barked. "Do not let the gift run lose." 

I tried to gather my strength to focus my mind again. I was so tired. My mind was wrapped in a thick fog. Duke Lukos's eyes opened, two black holes in his ashen face. Maggots crawled out and walked across his face. "I am dead, you fool. Your gift can only effect living matter. You failed me." 

I screamed and pulled my hand from his chest. All around me, I saw the corpses of the soldiers that died in the rift that day rise up, their rotten broken limbs hanging off their bodies, as they limped and crawled towards me. "You failed us," they chorused in unison. "You did not stop to heal us. You ran away."

"No," I cried. "I was sent away. I would have died if I had not." I tried to scramble up the wall of the ravine to escape the throng of walking dead. I made it up a few feet when I felt someone tear my talisman from my neck. I gasped in pain and fell to the floor. The dead soldiers were nearly upon me. I could see the dried blood that had matted in their hair and smell their putrified flesh. "Go away," I yelled, as I scrabbled from the floor. "I cannot help you by dying." 

I found the knife I had used to cut Lukos and slashed at the army of my dead country men with it. "Bitch." I heard someone exclaim. Then a pair of putrid hands grabbed me from behind and bound me fast. I struggled and trashed to no avail. The army of dead men came inexorably on, grabbing me, pinning me down, holding me fast. The ground gave way, and I swung from a harness of a large dead tree, looking down into the depths below. 

The nightmare grew incoherent then. I only remember snatches of horrors: nests of poisonous snakes, vultures pecking at my living flesh, black towers made of rotting poisoned flesh. Eventually, I slept. 
 
I awoke several times to find myself in the Tower's infirmary with Makala or Carlotta or Ernesto giving me drinks of water. My skin burned when they touched me, the water tasted of ash. I could do nothing by whimper.

When I found myself able to speak, Groto sat beside me in a small shadowy room, lit by the setting sun. Unknown threatening creatures lurked in the corners. "Where am I?" I asked.

The goatherd gave me a worn and worried smile. "Do you know who I am?" I answered him, and a series of other simple questions. The worry left Groto's face, and I realized that I was in his cottage, lying in the bed reserved for Duke Griswold. He handed me a plate of cold food. The creatures in the shadows growled and moved threateningly when I sat up to eat. 

"What happened?" I asked. 

Groto's pursed his lips to a bitter grimmace. "Griswold brought you back here before noon, your grace, then left on personal business. You have been here, recovering from your visions since. As for what you did this morning, you can ask the expert when he returns." I returned to my meal. The food tasted bland. However,the creatures in the shadows retreated with every bite. 

"Could I have some light in this room?" I asked when my stomach had restored a bubble of reality around me. I still did not trust the shadows. I watched the darkening corners suspiciously while Groto brought me candles. "The Duke volunteers for that madness time after time?" I asked. I could not see anything that would induce me to want to drink that brew again.

Groto nodded slowly. "He is looking for something. Only the Maker and he can tell you what that is. The mushrooms fullfill this need, when nothing else pleases him. Leading the Hundred Horsemen seemed to help. He did not come here to collect his mushrooms for the seven years he led the band of mercenaries. You were lucky. The madness left you in a few hours." I shuddered at the thought of weeks of the nightmares I had just had. "Will you go with him?" Groto asked. 

"I am not sure." The visions had shaken me. It was hard for me to be certain of anything. "He could teach me so much."

Groto looked disappointed. "Ambition does not suit a young lady well, your grace. It is your decision. I do not presume otherwise. Be careful of him, your grace. You are young and innocent. He is a cynical old man, despite his looks."

"Thank you, Groto." Even before today's events, I did not need reminding of Commander Griso's cruelty.

He left me to rest. I slept without nightmares until I heard the braying of goats being milked. Duke Griswold sat vigil at my bed. His right palm was bandaged. "How are you feeling?"

I sat up and looked around the room. The shadows did not hold shapes that scuttled in the corners of my eyes. My vision did not blur when I moved my head. "Better. Did I do that?" I asked, indicating his hand. Duke Griswold nodded. He did not hold the wound against me. "Would you like me to heal it?"

He pulled my talisman from around his neck. "It is better that you do not practise until we are certain all the side effects have worn of. Can you stand? Come see your handiwork."

I followed him out of the cottage to a large box lined with cloth and moss sitting in the shade. Inside it lay a large wounded tabby cat. "You bled that creature dry, then convinced its body to regrow its own blood. It is not an impressive act of wild magic, but considering that most people cannot even open the door the first time they try, I would say ..."

"That is Lukos?" I shrieked. I felt my body quivering with rage. The nightmares of the previous day appeared in the corners of my mind, threatening to break down my self control and wreck havoc again.

"Lukos was a dream, Nisrita."

"I endured everything yesterday to save Lukos," I said, a part of me knowing that to be false. The dream had been so real. It was hard, even now, to believe that I had not actually been given a second chance. "I did not heal a cat."

Duke Griswold raised his voice over my histeria. "The gift cannot bring back the dead. The sooner you learn that the less you will suffer."

"You lied to me. You told me I was healing Lukos." I turned and ran from him blindly past the braying goats until I could not see where I was going for the tears. I sat down and sobbed into my knees. Tears surged inside me at the thought that I had failed Cortan again. I knew Duke Griswold was correct, that I never had a chance of bringing him back. There was no room for logic in my broken heart. The dream had been too real. I had been so close to Makala. I had failed to save him. I had Lukos breathing in my arms before the nightmare snatched him away. 

I heard an argument coming from the direction of the cottage, and then silence. After a few moment, I heard the rustle of footsteps in the tall grass.

"The difficulty with the old magic in knowing how to control the dream." Griswold said as he sat down beside me. "Your dream was predictable enough. I could guide you through it to a productive end. Opening the door for the old magic, however, was all due to your strength and control. You did well yesterday."

I dried my eyes. It is not often that one is paid a complement by a healer such as Griswold. "Thank you."

"The brew leaves the nerves raw for a while, even after the visions disappear. You will recover shortly. Do not put this on until you feel stronger." He handed me my talisman. I nodded. The duke continued, his voice sounding more distant. "You are still young. If I could teach you to control the dreams while your gift peaks, I wonder what you could do with the wild magic. The possibilities boggle the mind."

I balked. I did not want to undergo yesterday's experience at any price. I had hoped to travel with Duke Griswold in the hopes of learning from him and bringing him back to Cortan. A series of experiments like yesterday was too high a price. It would not serve to learn from him at the cost of my sanity. "Is that your price for my traveling with you?"

He laughed bitterly. "No. I have business beyond the northern sea. I cannot be your nursemaid on the journey. I was idly speculating." I sighed in relief. "Do you still wish to come with me? We will travel by road to the river port of Reyena in Gissal, then find passage on a boat to Skalsbad. I will have you back in Deyalorn by the Day of Unions. I will not add absonding with Cortan's heirs to my list of crimes."

The north sea. I was born and raised in the arid plains of Cortan. I had never seen a body of water as large as the Tulsi River until I had married last year. The idea of sailing on a boat and crossing the ocean was irresistable. "I do. When do we leave?"

"After breakfast. Maya will have my head if I do not feed you well before leaving. Put this on."

I opened the small cloth bundle he held out for me. Inside it lay a thin gold wedding chain, shorter than the one I had wed with, the type used by freemen and small merchants. I gaped at the duke. 

"Would you rather travel incognito as the missing duchess and the exiled dead duke? We will be Lyete and Andre, newlyweds on a pilgimage to the holy lands beyond the North sea. If the story does not suit you, I will ask Groto to escort you back to the castle again."

The story would have to do. I followed the Duke to Maya's table as Griswold's wife.


\end{document}
