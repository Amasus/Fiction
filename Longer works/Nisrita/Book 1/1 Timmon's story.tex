\documentclass{article}
\usepackage{fullpage}
%*****************
% Annotations
\usepackage{soul}
\usepackage[colorinlistoftodos,textsize=footnotesize]{todonotes}
\newcommand{\hlfix}[2]{\texthl{#1}\todo{#2}}
\newcommand{\hlnew}[2]{\texthl{#1}\todo[color=green!40]{#2}}
\newcommand{\sanote}{\todo[color=green!30]}
\newcommand{\egnote}{\todo[color=violet!30]}
\newcommand{\newstart}{\note{The inserted text starts here}}
\newcommand{\newfinish}{\note{The inserted text finishes here}}
\setstcolor{red}
%***************************
\begin{document}

I sit on the stern blank face of Mount Turin and look down. I would be cold soon, I remind myself from afar. I had just spent the day climbing up this mountain to this spot on the south face of Turin, the highest peak in the Pensid mountains. A formidable peak, they said when I had questioned the Mievian traitors who would sell us information. Turin was a holy mountain, not to be taken lightly. Nothing is formidable when one has nothing to lose. My gods had forsaken me twice, I no longer hold much with gods. I am still hot from the difficult climb. The mountain air blows cool and welcoming on my back. My body would cool, night would fall, the temperature would drop. If I did not do something soon, I would be shivering on this narrow ledge in a few hours, the distant voice insists. I consider this rational suggestion, then ignore it.

I look down. The cliff whose edge I sit on ends in a steep slope of rocks about thirty feet down. The rocks continue until they formed the banks of a lake. The water is reflects the blues and the reds of the setting sun, as if dyed by the blood of the miriad of men we lost in that Destroyer Taken high plain. I wonder how deep the lake is. I do not know how to swim. 

The reds deepens the fades, the blues turn to to purple then black. The sweat of my back cools, leaving a lite salty crust. I do not move at my perch until my teeth start involuntarily chattering in the darkness. I have prayed for death every day for the last ten days. The Destroyer has not shown mercy. I came here to take matters into my own hands, given my gods' callousness. They have taken everything from me. 

Behind me, sits a castle and the remains of my duty. Beyond that, a wide open field, and a crack in the earth at the bottom of which lies most of my life. Before me is the black eye of a lake that would swallow the rest. I simply have to find the courage. Barring that, I have to wait for sleep and cold to settle in.

The crescent moon rises. It reflects itself like a white bow in the dark waters below. He had been an archer, the first to fall; the brother of the one I lost completely, and of the one I have only misplaced. As for my duchy, I did not know if Cortan yet stood. If my duke fell, what did the harsh mountain castle behind me matter? I do not know why I yet lived. This morning, I decided it was because I lacked the courage to die. I would fix that tonight.

A thin cloud shades the invisible archer's radiant bow. Then it passes. Behind me lies a world of terror loneliness and unsurmountable grief, the barrier between my rank as a commander and exile thinner than a widow's veil. The army I have given my life to will betray me with more joy than it would kill a Nievian archer. These mountains have taken from me my only hope of a life free of grief and terror. The choice should be easy. I sit shivering on my ledge. It is difficult not to reflect on the life I would end. 

I came to Cortan barely a man, a nineteen year old sergeant with little chance of inheriting my father's lands, a promising military carreer and a terrifying secret. My uncles knew Duke Ergino to be a powerful and cunning military leader three decades ago when they served under him conquoring what is new the most prominent and powerful border duchy of Marsea. It seemed obvious that I come to Cortan to test my skill and luck. Border duchies need ambitious talented men to fight their never ending wars, and settle their ever expanding borders. I had done well for myself. I was promoted to captain by twenty five. For the last eight years, I had the  pleasure of watching scrawny adolescent boys come into my training, to emmerge on the walls of Cortan's Tower or barracks a few years later. I had made a name for myself as being skilled at gathering truths from secrets and rumors, until Duke Ergino himself asked me to dine at his table for predicting the food riots two years ago. 

Who would trust me now, after that catastrophic battle on that high plain? Whether my commanders and colleagues voice it, or not, I know that I should have caught the signs that led to that devastating day. It was my duty, as sworn to my duchy, and given to the duke I loved. 

The young Duke Makala was a boy of fifteen when I first came to Cortan. Even then, he was as smooth as silk and  as confident as a king. He would inherit the lands his father had carved out for Marsea, he would add to Cortan's glory and wealth, if not by his military strategic abilities, then by his ability to inspire men to greatness. He was a small man. Duke Lukos, his younger by two years, and a sterner, more dogmatic man, was almost of a size with him.

I tried no harder than any other sergeant in Cortan's army for either of the Duke's attention, though by the end of my second year in Cortan, I found myself inexplicably graced with unexpected moments of Duke Makala's time. It was not unusual for the duke to draw me away from a dicing table to sit across a chess board from him. He had the habit of leaving me in the middle of a game, extracting from me the promise to meet him again to finish what we had started. His game was always unconventional, often surprising, rarely successful, but almost always challenging. He left me looking forward to our next encouter. Duke Makala had a habit of making queer friendships. He had grown attached to a young orphan from Cortan's White Tower, a girl of five when she started first appearing at the Duke's table. I did not question the Duke's habit, or his favours towards me. Those first years in Cortan were difficult for me. I found myself alone in a harsh foreign duchy, far from the comforts of home, yet with no with abatement from the risk of ruining my father with my secret. My position in charge of a tent full of ambitious boys presented me with dangers more often than I would like. I responded by forcing myself to be scrupulously fair to those in my care. I had no room for personal favourites if I wished my true nature to remain concealed. I found middle aged bell ringer from the temple to share my secret with. Dim witted and large, he is called miscast for his shortcomings. Yet Fidelo was loyal and warm in his own way. We shared our outcast status when I returned from missions to the border or when loneliness and fear threatened to overwhelm me. I made certain to pay the man well for his time and cursed the Maker for seeing fit to make me miscast as he did.

I remember being pleased, though not entirely surprised, when, as my second summer in Cortan drew to a close, the young duke invited me to join him on a small lion hunt. I was more surprised when, still smelling of lion's blood, the duke asked me to ride with him, leaving the rest of our company to celebrate his kill. He wanted to spar. With the thrill of the hunt still surging in his blood, and the growl of the great beast thrashing in the grasses still echoing in my skull, we dropped our spears and drew swords in the hot dusty midmorning sun. The taste of victory over death ran rampant through our bodies. Anything was possible in this world where our physical strengths reigned supreme. Makala moved like a dancer more than a fighter, as the Black Riders he trained to fight with did. The young duke may not have been a member of that elite group of mounted archers yet, but he was fast, agile,and fearless, even for a member of Marsea's army. In comparison, I was slow and stoic. A larger man, I would press an advantage when I saw one, but I would not tire myself uselessly on a spry opponent. The dry grasses and crunched underfoot, the sun beat on my thick black hair, the dry heat stung my nostrils, Makala glistened with sweat and the exuberance of his first kill, prowling just beyond my reach until he chose to close in to exchange a few blows. I blocked and parried and circled and waited, enjoying flexing my muscles in this excersize in warfare and the false sense of danger that accompanied it. Makala danced against me, drawing me forward at times, pushing me back at others. He was playing in the late summer heat, and so was I. I found myself at one point backed against a thorn tree. When I moved to step around it, I snagged my shirt on a bramble. The young duke swept out my legs, landing me firmly on my back on a bed of thorns. It had been foolish of me to get so close to the offending tree. I had focused too much on the delight of the fight, and not enough on my surroundings. I prepared my self for the demand that I yeild. Instead the duke tossed aside his sword, lept upon me as if to wrestle me and kissed me full upon my lips. The exuberance drained from my body as I lay under him, suddenly feeling more endangered than I would  had the lion we hunted sat atop my chest. My duke, grinning, counted down the seconds to my defeat. The ambitious third son of an ungifted barron does not decry his duke as miscast, especially when he fears the same label upon himself. Duke Makala's hold on me was weak at best. I swallowed my immediate terror and threw my laughing duke onto the dusty yellow ground and wrestled him as he desired. I did not dare defy him in this isolated compromised position. If he knew, if he would tell, not only would I be exiled or stoned, but my father would be ruined. I had spent the last six years of my life training myself not to feel desire when wrestling and sparring and drinking with men and boys I would later share tents with. I had trained myself not to notice the ripple of shoulders, or breathe the acrid smell of sweat, or linger on the thought of dark lips meeting mine. But my duke was laughing, drunk on the success of his hunt that the pleasure of our bodies grinding and struggling against each other. We rolled and grappled each other among the brambly twigs, yellow dust stinging as it caked itself onto our torn skin. I would never had dreamt of letting my desires lose if my lips did not still throb with the memory of the kiss they had just met. Duke Makala was nearly a head smaller than me, wrestling not his strength. I told myself that my duke wanted from me only what I wished from my bell puller, wrapped my legs around his narrow waist and pushed his heaving chest to the yellow earth and called for him to yeild. The young duke's body went limp. He was still laughing, his chest and shoulders rippling silently with delight under my thighs, a motion that would become to familiar to me in the years to come. The extacy of his joy and the ferocity of my terror became one and the same. I was only aware of our shared solitude. The dark smell of lion's blood mingled with the sharper aroma of my duke's sweat to awaken an urgent need I had made a long habit of leaving unacknowledged. My duke still laughed as I helped him to his feet. His fearless reveling in our illicit sparring captivated every last drop of my attention. I stepped out of the cautious shadow I had learned to live my life in. We were alone in this grassy clearing. No one would know if I returned to those dark parted lips the kiss my duke had taken from me. I still remember the relief I felt when he returned my embrace. I gave myself to the pleasure of the moment and whimpered in his arms. We fell to the ground again, our tumbling becoming more urgent, our clothes scattered across the clearing as we danced to a different tune. We moved quietly, secretly, united in our delight and our shame. The sun beat down upon our bare prespiring skin, caked with yellow mud and straw where flesh did not desperately stick to flesh bound by desire and  sweat. I licked at the salt crusted on his neck, felt his hot breath drying the moisture beading on my chest, dug my nails into his supple narrow back and brought my mouth down hard onto his, in my urgent need to be closer to him than the physical barriers of our bodies would allow. My duke still tasted of the spice of the dried meat we had consumed before dawn. My shame sat with me, as it always did, but further than usual, dwarfed by the delight of a consenting conspirer. When I could control my desires no more, I rolled the young duke over and mounted him. He moaned when I pushed myself inside his small muscled buttox. "You have done this before," he gasped as a I moved, breaking our silent passion with his surprise. I could not respond in my shame. I had not realized that I introduced my duke to the world of fear and shame our corruputed natures forced us to live in. I rubbed his engorged member to the rythm of my need until we both released the malcreated tension the thrill of the lion's hunt that planted in us. 

I did not realize then, as I lay on the prickly grass rubbing the yellow mud from my duke's brown shoulders, that I had bound myself to this arid harsh border land with that act. I listened to Duke Makala spin his ambitions for Cortan's Black Riders, as I am certain Fidelo patiently listened to me speak of my father's lands and the royal court at Deyalorn. I did not realize what I had entered into with the next Duke of Cortan until I returned next spring from the year's campaign. The young duke had arranged to move to a suite of rooms he did not share with his brother in the castle. His new chamber had two entrances, one of which was not guarded for all the night. He called for me in the privacy of this new setting the evening I returned before I could seek out Fidelo. I was to meet with my duke, and only him, in the privacy of that room, he stated. He had paid my bell puller well to stay away. Makala and I had met in private during the short autumn and the wet winter, stealing kisses in the terrifying public setting of an unobserved corner of the barracks, or humping hastily when nothing else would do in an unoccupied shed, entering unseen, emerging separately, terrified that we may be observed. To ease my fears of dsicovery, Duke Makala suggested that spring that we end our public friendship completely. He was my duke and my protector. It was my duty to swear to him anything he demanded. 

Sometime, over the next months, in the privacy and safety of his bedroom, my love for my duke and his station changed into a love for Makala the man. The duke spun me stories of the glories we two would acheive for Cortan until I began to believe that miscasts like us could find comfort in another's company as readily as the uncorrupted could. By the time the Day of Unions came that autumn, and the gifted and the powerful descended on Cortan's court to beg from the duke their right to marry, Makala led me away from the crowds watching the dancers or inspecting the whores, to draw me a world where the duke sat above being called a miscast, and we could wear each other's wedding chains. I admired Makala's brave irreverance. The only difference between his dreams and the world he lived in was Makala's ability to find a means of bending the world to his will. Where I lived in fear of my nature, Makala sought to make his irreperable flaws less of an inconvenience in his life. I came to believe that there was nothing that could stop Makala, not even the Destroyer himself could take him.

By the time Makala turned twenty one, and Duke Ergino wished his eldest son to marry, I refused to give my consent. I loved him, but it was more than that. Makala's dreams had given me the courage to to think of the duke as my own, as certainly as I would had he been my spouse. We had planned a future for ourselves. In those fantasies, I would lead his armies, and Cortan's borders would expand until the duchy would be split in two to create lands for his brother Lukos. When Cortan became an interior duchy, then Cortan's expertise would help Lukos' lands become great. I would be Makala's eyes and ears in this process. I would help him rule. There was no place for another in our future. 

I was aware, of course, of the need for heirs. That did nothing to alleviate my jealousy of any intruder in our relationship. Any woman, with her sons and the changes she would bring to Cortan's court would inevitably take Makala away from me, no matter what Makala promised to the contrary. We argued over this issue every summer, when Duke Ergino wished to take his heir to Deyalorn for the Day of Unions. I won the battle for two years. The first year, Makala did not wish comply with his father's sudden wishes. The second, we argued until my duke found it easier to resist his father that to argue with me. The third year, Makala put an end to my arguments before they began by announcing his intentions to marry the orphan from Cortan's White Tower he had taken to treating like a younger brother since she had barely passed her infancy. 

"It is decided, Timmon," Makala said quietly picking up the marble and rhinestone chess pieces I had upset when I heard the news. My father had gifted me the set, knowing my love of the game, not knowing the reason I had developed an increased fondness for it during my years in Cortan. I had lost my temper when Makala told me who he would marry. I paced my anger across his room. Any woman would have been difficult to swallow, one he was already fond of was insufferable.

"You have been having an affair with her." I accused irrationally. His betrothed was a girl of thirteen. She frequented Duke Ergino's table, was close to Makala's sister Chamila, she even trained with Makala's Black Riders. She was a White Healer of Cortan's Tower. She would never serve with Cortan's army, and many thought Makala's indulgence in letting the girl train a scandal. Her training allowed her to spend an hour with the young duke on a daily basis. On the frequent days that she dined at the castle, she spent more time with the duke than the piteously scant two hours I was allowed with my lover. 
Makala, to his credit did not suffer my outrage with an answer. He put away the interrupted game, then took his goblet of wine to his window. His room had a view of the barracks. On mornings when I lay awake, suffering from nightmares of what would happen to us if the duchy learned our secrets, I could see his windows light in the hour that the sky turned from black to blue, when Makala, along with most of Marsea readied themselves for morning prayers. It gave me comfort those terrible nights to know that my lover and duke was safe in his castle, not drowning in a pit of tar, or bloodied by the rocks of his outraged subjects. I would not let an adolescent girl intrude upon this fragile oasis of joy I had finally found. 

"I cannot allow you to marry her. She will destroy everything we have," I continued, presenting the same arguments I had for the past two years against Makala's marriage.

"She has a name, Timmon," Makala said from his window. "Nisrita Sentaveri, the neice of a barron in the duchy of Voltain, the half sister to the Regent Consort, and the most powerful healer Cortan's White Tower has seen since its founding four decades ago. My family has not had a gifted child survive to adulthood since my father's uncle Duke Griswold. The duchy needs this union."

Duty and duchy be Taken. I needed Makala. "She will ruin us," I insisted.

Makala looked at me with the sad patience afforded to an old but senile loyal servant. "She is a child. She wants a family and to study. I promise you, she wants little of what we have."

"Have you learned to read the future, your grace," I sneered. "You cannot promise that." Even if he could promise me that the girl would not impinge on his feelings for me, a fact which I doubted under the circumstances, he could not promise our safety. "What will you do when she reveals our secret in a fit of feminine pique or jealousy?"

"There was a time," Makala mused, "in the early days of Marsea, when great kings would marry great healers, to lead the country to greatness. The queens would fight alongside the armies. They were politically powerful heroines, revered as mother protectors of the land. It has been a long time since Marsea has had such a figure. The memory of the kings of old would make Cortan a powerful duchy."

I stared at my duke in disbelief. "That is your vision? To return Cortan to the days of the first kings, when we first settled this land and fought the Deyanesi? You have not answered me. What will you do when she betrays us? I will not be sacrificed for the good of Cortan."

This was not a lover's quarrel I would win. My duke had not only come to his conclusion, he had a grander vision. He could not swayed. "My guard will arrive soon. You should leave," he said quietly, sadly, not cruelly. "I cannot give you better than this union, Timmon."

I stormed out, my anger refusing to be soothed by Makala's calm. I knew his affianced as well as any man in Cortan's court. She was an orphan born to a once great man who had fallen out of favour in Deyalorn's court. The Black Riders Makala rode with reported her to be a cheerful, helpful girl who kept to herself while training, and doted on the Duke Makala. I had heard Duke Lukos refer to her once as hoax taken too far by his brother. I could not extract further details from him. Cortan's children were close. While not the oldest, Makala was the one they all turned to in their troubles with each other or otherwise. Whatever Duke Lukos's feelings towards his affianced, he would not speak publicly against his brother or his decisions. The child, when she came to dine at the Duke's table, kept to herself. She was a short girl, squarely and muscularly built, plain to look upon. She seemed uncomfortable in the court, or at least painfully shy. Perhaps a member of Cortan's White Tower could say more about her personality. From the point of view of anyone outside her teachers and classmates, she did not seem to be a glittering addition to Duke Ergino's table. Were it not for her promise as a gifted healer, it would seem she had little to offer Cortan, unless one took into account Makala's inexplicable desire to teach a girl to kill with poisoned darts. 

It would be unfair to say that I argued with Makala for the next month. I stayed away. Makala granted me my distance. I saw him only in the mornings at the training yards, where, by long standing agreement and habit, we did not speak to each other. I told myself that I could not partake in a relationship that endangered me as much as this upcoming wedding promised to. Makala had been scheming, I learned much later. At the time I knew nothing of his plans. I only knew, or thought I knew, the loss of losing him. My duke bore his sufferring more gracefully than I, priding himself on wearing a public face that never showed what he endured. The boys in my training suffered as I did. Several times that month, the White Tower reprimanded me for the number of trainees I sent into their care. Our recruits train hard. It is part of what makes Marsea's armies so great. Selection into the army is based in part on whether one can maim and injure more of one's comrades than the others in one's cohort. The White Tower serves our army in peace time by healing our boys and returning them to the practise yard ready to fight again. It serves the civilian communty as well. Sending more than one or two boys to them a day with severe injuries is generally frowned upon. Eventually, Sergent Jaquim Lorzo, the man in charge of the trainees tent started intervening in my boys favour. He came to claim them when I kept them well past the time that others in the barracks had stopped their daily training excersizes. I could never be accused of cruelty to my boys. However, Sergent Lorzo hinted at the possibility of excessive zeal on my part when he found my charges running laps in the blistering hot, dusty late summer mornings. I had not overstepped my bounds. A stronger class of recruits would only benefit Cortan's army. I had nothing else to unleash my grief and loneliness upon. I burried myself in menial tasks around the barracks, below both my rank and my station. I let Sergent Lorzo check my harshness and drove my boys, and myself to the edges of our abilities. 

I watched Cortan's delegation leave for Deyalorn a month later. I calculated that it would take Makala three weeks to get to the royal court. He would stay there a week or two surrounding The Day of Unions, then return home to days with midday heat made bearable in the name of autumn. I had seven weeks, maybe eight of solitude. The Duke's court would celebrate The Day of Unions without the presence of the Lord's family. Makala's brother, Duke Lukos would oversee the proceedings. Makala, his parents, and his remaining unmarried sister Chamila left for the more temperate royal court,  along with anyone of any importance wishing to marry this year, especially if they wished to marry into a family known for creating Gifted. I knew the path they would take well. I had grown up near Deyalorn. I travelled the road often. 

In four weeks' time, Makala and Chamila would present themselves before the others wishing to get married this year. Duke Ergino would meet with the Regent Consort, Dario, Nisrita's half-brother, in the hopes of squeezing some dowry out of the crown. Dutchess Cybeline would do what she could to find a better match for her daughter than whatever it was she had arranged. I usually had no interest in this matter, but my Makala would be up for display and barter in front of the entire kingdom this year.

It was possible, I comforted myself, as I watched my duke ride, that someone else would take the girl Nisrita away from him, or that the Regent Consort Dario would not give his consent. The Day of Unions was a old tradition, from when the Gifted were few in Marsea, and increasing their number in a balanced manner among the great families was crucial for the kingdom's survival. Now almost every dutchy had a tower, and the tradition of marrying before the king is respected more because no king wants to be known as the one to outlaw the day of hedonistic pleasure that is The Day of Unions, than for any practical political purpose. Well over half of all matches are decided by the time any family arrives of importance for the festivities. In most of these cases, oversight by the crown is a formality. There is always a potential for an upset. A better positioned husband, or an larger offer of a dowry may come forward the day before, or even the morning before the union is sealed, forcing vows to be rewritten. Every year, one or two such sudden changes in marriage plans feeds the kingdom's greedy apetite for gossip and intrigue. Makala had set his sights on one of the more powerful Gifted in the kingdom. Someone from a more comfortable dutchy in the interior of Marsea may claim Nisrita for their own. Her half-bother, as Regent Consort, may deny Cortan his consent. He may decide to place her somewhere safer than this land of sunbaked yellow clay my love for Makala has bound me to.  There was hope, I told myself that if Makala would return to me a single man, and we would put off the inevitable for another year. Makala had a greater plan to enact. He likely would not marry if he could not marry Nisrita.

\begin{comment} He had asked me, the night before he left, to marry in the local ceremonies. I would have thought a different man to be mad for making such a request. My duke was scheming. He would not tell me his plans, though it was clear that he would not marry if he could not marry Nisrita. If I had not stayed away from him during the last days of his life as a single man, I could have wheedled his plans from him. As I had stayed away, I let my duke leave without promising to follow his wishes. I would leave, I had decided, if Makala returned a married man. If, by some miracle or the Preserver's mercy, he returned to me single, I would stay with Duke Ergino another year. The miscast third son of a house barren of the gift did not need to marry.\end{comment}




\begin{comment}Where had it gone wrong? I knew the answer to that. But before it had all gone wrong, there had been a brilliant moment when everything was right. The Preserver is a cruel and fickle god. The paths he leads us down are twisted and unreliable. He dances with the Destroyer, playing not only with our lives, but everything we struggle to build for ourselves.

It had only been nine months ago when Makala met me in Cortan's barracks with his world changing announcement.

"I think I will marry her."

He looked across the yellow dusty fields at someone. I followed his gaze to learn his intended's identity. It landed on a cluster of boys wrestling in the dust. They were beautiful young creatures, strong, limber, beardless, and totally lacking in discipline. They were my responcibility. Those that I taught to cripple more boys than crippled them stood before the Duke and the White Tower to start their life as a member of Cortan's standing army. They rest would go home to be called upon in time of need. The honor came with a gelding. This detail was part of the White Tower's cruelty. They had learned that castrating boys under the age of sixteen sometimes caused a few ungifted men to bloom into gifted eunichs. In Marsea, the gift is all important. The backbone of our army is formed by eunichs.

Makala would not be marrying one of my boys. I was not fortunate enough to live in a world where I could unleash the storm of jealousy building inside me upon my recruits. I knew this day would come. He had badgered me about it for over a year. We had fought and argued over this day. By some miracle, I had convinced him last year to put it off. I had thought I would do the same this year.

I opened my mouth to protest. Not this year. I would not let him leave me yet. I thought I knew about loss. I thought I knew what lay ahead. I knew nothing of either. At that moment, I only knew a blinding, crippling, jealous anger. "Who?" came out of my mouth where I meant to say "No."

Makala's gaze did not move. "Nisrita," he said. He had not raised his voice, but I could hear his anger. Then he rode off to meet her training with his Black Riders. I saw him meet the group of his archers on the far side of the field. Off their horses, they looked more like dancers than warriors. They moved like he did, with a snakelike and mesmerizing deadly grace. Nisrita, walked a taught leather strip a foot off the ground, shooting at a series of targets positioned twenty yards away. She was short for an archer, but she was still a girl, squarely built, plain of face  and lean. Makala had invited her to train with him every morning. He had taught a plowhorse to prance. 

Makala rode through my group of wrestling boys to meet his riders and betrothed. Some of my boys saw him coming and rolled out of the way. Others, the slower ones, got a hand, or a limb trampled by Makala's steed, as a lesson against their tardiness. I vowed not to watch him train today. He had no right to desert me like this and trample over my students. He had no right to be angry with me. It was he who should be apologizing. I turned to my wounded boys. Their agony was more urgent than mine. I called over the white robed healer from the Tower to tend to those who lay whimpering in the dust, and organized the rest into groups for races. 

I should have seen this coming. For as long as I have known Makala, she followed him around like a puppy. Everyone knew Makala's affection for the orphan. He made no secret of it, not even from me. He snuck her out of the White Tower to teach her to shoot and ride. He sent his whipping boy to take her lashes for her when he could. She worshiped him. He treated her like a little brother. It was indecent what he taught her. But he was Cortan's heir. No one dared speak against his whim. 

I had not thought I had anything to fear from her. To Makala, she was the sixth of his five siblings. Makala was the oldest of Duke Ergino's surviving children. He fashioned himself as the keystone for the family. His siblings may have spats with each other, but I have never known one to have a serious quarrel with him. He was devoted to their happiness, and they repaid him with loyalty and trust. He intervened in their arguments with each other, and they came to him for advice. Living as far from my family as I did, I envied him his closeness with his family. I had never thought to begrudge him his relationship with his siblings. I had not questioned him when he claimed his affection for the Tower's orphan was filial. 

I silently raged at Makala for lying to me. I raged at myself for letting him slip through my hands into a relationship I knew of and had condoned and tolerated. If he had to marry, why could he not have done me the courtesy of chosing a stranger?

I was furious, and jealous, and hurt. I made sure my boys trained hard that day. They stayed in the field long after the riders lead their horses to drink in the shelter and shade of the stables, well after the infantry stopped kicking up dust and haze in the distance with their maneuvers and games. I kept them fighting in pairs and lines until the weakest started to stumble and falter under the baking heat of the midmorning summer sun. Their sergeant came to me then to point out that I was causing their morning duties to be neglected. Jaquim was soft on them. The giftless would not be considerate enough to only fight during the early morning and dusk. I let it go. If I sent too many more to the Tower today, I would be reprimanded for overtaxing their resources.

I did not see Makala until that night. I had not intended to see him at all. He would be the next Duke of Cortan. I was a captain of his father's army. We had different duties during the day. It was perfectly reasonable that our paths should not cross again until the evening. I did not have to try to avoid him. I simply did not seek excuses to find myself within his sight.

I found a scrap of paper on my bed when I retired after dinner. It had one word written on it. I crumpled it in my fist and threw it into the fire. I would not go, I decided. He had no right to call me. 

Twenty minutes later, I saw that a chess board filled Makala's table. It was a good sign. He wanted to talk. There was little else I could take from him tonight. Makala stood at his window, his back to me, silent. I decided to wait for him to speak. He had set the board with the marble and rhinestone pieces I gave him two years when I became a captain. My father had gifted me the pieces in his pride at my acheivement, knowing my love of the game. Twenty-five and a captain in Marsea's most successful duchy. He prayed to the Maker that he would live to see me a general in the royal court, living near him again. I did not tell the old man on whose board his present arranged itself. 

I upturned a wooden cup to reveal a white pawn. I placed our pieces, poured our drinks, and made my move. Makala did not break his silent vigil at the window. He tested my patience. "Black's move," I announced, realizing it immediately to be a fruitless gesture. He was angry with me. He would not answer. I drummed my fingers on the table, and picked at the bits of dirt beneath my nails. His silence droned on. Eventually, I grew tired of waiting. An archer lurks behind the lines for an opportune moment. A swordsman charges ahead. "What do you want of me?"

"You have not spoken to me all day," he responded. His voice was soft. Not apologetic. Just infuriatingly soft.

"I have nothing to say." That was a lie. There was much I had to say, but did not want to revisit the old fruitless arguments. If he was going to throw five years away in favor of an ugly adolescent child, he did not deserve my protestations, I thought. What a fool I had been. I did not know that our time together was truly, irreperably coming to an end. I should have embraced him and begged for an explanation instead of spending those precious moments in a futile jealous rage.

Makala's dark slim shoulders rippled in a sigh. When he turned to me, his eyes may have been two spheres of onyx set in a face carved of teak, for all the warmth they held. When he spoke, he cut his words with steel. "You know I need to marry. I am twenty-three. I have delayed this long enough. We have been through this before."

"And I do not wish to go through it again," I protested. 

He ignored me, stepping towards me as he spoke. "Nisrita is powerfully gifted. The Head Cortan's Women's Tower says she has not seen anyone as strong as her in her twenty years here. My father will name her heir after Lukos. My generation does not have anyone with the Gift. Barring Antonyo, we have not had a Gifted in the family since Griswold died thirty years ago. He was my father's uncle. In spite of Antonyo's brief life, people accuse Cortan of being barren in the gift. My family needs her."

"You told me she was nothing more than your sister!" I exclaim, as loudly as I dare. The other doors to Makala's rooms were guarded. Under no circumstance could I risk being overheard. No one knew. I was not supposed to be here.

Makala did not raise his voice. He never raised his voice, no matter how angry I made him. He continued icily. "I cannot deny Cortan this union. Do not shame yourself by sulking before me. You are a Captain in Cortan's army, not my mistress."

Would that I were his mistress. I could appear with him in public. We would not have to meet in solitude and secret, jumping at every shaddow. This life we led now was a fate worse than gelding. At least Cortan honored their eunichs for being only half a man. The exiled cowards and miscasts like us. "I am less than your mistress, Makala." I spat. "No woman would stand by your side as you take pleasure from your wife every night."

Makala laughed, a soft bitter burst of sound. Moisture sprayed from his lips onto my arm. It stung where it landed. "Is that what you think will happen?" He paused to listen to his inner thoughts, then smiled coquettishly and slipped his hand under my arm. "Timmon, let me show you what will change between us."

Fool that I was, I pushed him away. If I had felt any differently about him, I would have hit him. I was bigger than Makala. He never beat me in hand to hand combat. But I could not bring myself to injure him outside training. He was my lord's son. He was more than that. He was then, and always will be, my Makala. I stormed out of his chamber, a performance fit for the adolescent bitch that would take my place in his bed.

Gift and family be Taken, I raged to myself as I returned to the barracks. Makala is the one I chose for life, privately, away from the auction house of The Day of  Unions, hidden from the sight and blessing of our Maker. I wanted him to chose me in the same manner. I would not share him with another. If he could not honour our relationship the same way. I would not have him at all.

I watched him leave a month later at the head of the caravan for the court at Deyalorn. I had foolishly squandered the intervening time arguing and sulking. Watching him leave, I felt the first sting of true loss break through my anger. I calculated that it would take him three weeks to get to the royal court. He would stay there a week or two surrounding The Day of Unions, then return home to days with midday heat made bearable in the name of autumn. I had seven weeks, maybe eight of solitude. The Duke's court would celebrate The Day of Unions without the presence of the Lord's family. Makala's brother, Duke Lukos would oversee the proceedings. Makala, his parents, and his remaining unmarried sister Chamila left for the more temperate royal court,  along with anyone of any importance wishing to marry this year, especially if they wished to marry into a family known for creating Gifted. I knew the path they would take well. I had grown up near Deyalorn. I travelled the road often. 

In four weeks' time, Makala and Chamila would present themselves before the others wishing to get married this year. Duke Ergino would meet with the Regent Consort, Dario, Nisrita's half-brother, in the hopes of squeezing some dowry out of the crown. Dutchess Cybeline would do what she could to find a better match for her daughter than whatever it was she had arranged. I usually had no interest in this matter, but my Makala would be up for display and barter in front of the entire kingdom this year.

It was possible, I comforted myself, as I watched him ride, that someone else would take the girl Nisrita away from him, or that the Regent Consort Dario would not give his consent. The Day of Unions was a old tradition, from when the Gifted were few in Marsea, and increasing their number in a balanced manner among the great families was crucial for the kingdom's survival. Now almost every dutchy had a tower, and the tradition of marrying before the king is respected more because no king wants to be known as the one to outlaw the day of hedonistic pleasure that is The Day of Unions, than for any practical political purpose. Well over half of all matches are decided by the time any family arrives of importance for the festivities. In most of these cases, oversight by the crown is a formality. There is always a potential for an upset. A better positioned husband, or an larger offer of a dowry may come forward the day before, or even the morning before the union is sealed, forcing vows to be rewritten. Every year, one or two such sudden changes in marriage plans feeds the kingdom's greedy apetite for gossip and intrigue. Makala had set his sights on one of the more powerful Gifted in the kingdom. Someone from a more comfortable dutchy in the interior of Marsea may claim Nisrita for their own. Her half-bother, as Regent Consort, may deny Cortan his consent. He may decide to place her somewhere safer than this land of sunbaked yellow clay my love for Makala has bound me to. There was hope, I told myself. Makala may return a single man, or at the very least, he may return bound to a stranger. With any luck, she would be delicate enough and disagreeable enough that Makala's chivalry would keep him from taking his displeasure out on her, and he will be driven back to me by the barbs of her tongue.

There was still hope, I told myself, as I watched him ride. The brown of his boots and gloves were just a shade darker than his skin. I gave him that set at the beginning of the summer. It made him appear to ride barefooted and bare handed to a casual observer. It was an exotic effect that pleased him. Nisrita rode beside him, talking to him, I am sure, in that adoring and winsome way of hers. There was hope, I told myself, but I did not see how Makala could possibly escape the snares she lay.

The night before he left, he told me I should attend the local festivities under Lukos. I had argued and sulked and stayed away from him for as long as I could, until the thought of leaving him proved too great a burden. I begged permission to see him again. Makala, in his wisdom took me back.

"You should attend Cortan's celebrations this year, Timmon," Makala whispered as he ran his finger along my sternum.

For the last five years, the Day of Unions was the only day in the year that Makala and I dared appear together in court. On that day, we could sit, two amdist a dozen soldiers, drunk and jocular, watching the performers. Our companions were always either to inebriated or absorbed to notice that our four eyes lingered on different acrobats and dancers than theirs did. An army of whores descend upon the court for that one night every year. However, there are alwas enough young men who are either pious enough, or dislike the idea of sharing their possibly Gifted blood with an unfitting household that our abstinence went unnoticed. We would retire early in the privacy offered by the roar of jubilation gripping the court. I hid my longing with a jest. "Are you offering me your brother in lieu of yourself?"

I felt his chest spasm in silent laughter. "You lack two crucial traits to arouse my brother, Timmon: a cunt and my father's approval," he said into my chest.

I admired Makala's playful irreverence for the danger we both lived in. He was careful and secretive about our affection, he had planned and slowly rearranged his life for six months before he satisfied himself that he had created a situation where we could meet safely. He had moved out of the barracks of the men he trained with, placed himself in a room with two exits, repositioned his guards to leave one door unwatched half the night, all with legitimate and believable reasons. He was cautious and vigilant,  but he did not live in fear and shame. He could jest about his brother's dry love of duty, the man of whom I feared discovery the most. I had no answer to his quip. 

I let him lie on my right side, his thick black waves of hair blanketing my shoulder, thinking it would be the last time I would hold him like this. How wrong I had been. How paltry my grief compared to that which consumes me now.

It was that sense of finality that forced me to ask the question that had been eating at me the last month. I would have to leave him soon, perhaps for the last time, before the door I entered through gained a guard. "Why her Makala?"

"Nisrita?" he responded, not moving. He did not sigh in impatience, but his voice sounded weary. "Cortan needs a Gifted. This dutchy with fall if I cannot provide."

It was an old answer. It did not adress my jealousy. "There are other Gifted from powerful families from the north. Of all the wives you could have chosen, why did you chose one with whom you share a fondness? I could have taken you leaving me for a stranger."

I heard the smile in his responce. "Or better yet, a woman I despise. Would it make you happy to see me as miserable in my marriage as you are in daily life?"

Yes, I thought silently, though I did not mean it. I would never wish him ill, even when I could not stand to see him. I said nothing. 

He changed the subject. "I do think you should attend the celebrations here. You could marry."

I broke free from him in my surprise. For Makala to marry was cruelty and madness. He had a sound, legitimate reason why he had to undergo it, but it would drive us both out of our heads. For me to undergo the same, for no reason at all, was senseless violence upon my person. I stared at him and composed myself. It must be grief that caused him to suggest this. There was no other explanation. I tried to make light of the comment. "What woman would I be able to hoodwink into uniting with a miscast like me?"

My attempt at humor failed. Makala's dark lips curled sadly upwards on the right side of his face. "You did not have to hoodwink the duke's son. His young captain can find a wife." He kissed me. "Dress now. My guard will be at the door soon." 

I felt ashamed. I had not wanted our last night to end like this. "Consider it, as a favour to me," he said as I positioned my weapons on my person. I nodded. I would not, but neither would I voice that lie. "Will you see me off tomorrow morning?" he asked I put my hand on the door.

I turned to face him, forcing myself to smile. "I will be in the honor guard, my liege." I said, and bowed. "I will watch from the east tower until I can no longer see the dust of your trail."

That promise, at least, I kept. 
***
\end{comment}

I do not know how the next weeks passed. The Day of Unions found me far away from court, leading a small group of infantry in the western hills. Makala and I had celebrated the holiday together for past five years. It was the only day in the year that we dared appear together in court. On that day, we could sit, two amdist a dozen bachelors, drunk and jocular, watching the performers. Our companions were always either to inebriated or absorbed to notice that our four eyes lingered on different acrobats and dancers than theirs did. An army of whores descend upon the court for that one night every year. However, there are alwas enough young men who are either pious enough, or dislike the idea of sharing their possibly Gifted blood with an unfitting household that our abstinence went unnoticed. We would retire early in the privacy offered by the roar of jubilation gripping the court. I could not spend the day in the Duke Lukos's court without Makala.

There had been a series of raids in the western farms. There is always a threat of small groups of ungifted brigands kidnapping  our children. It provided a convenient excuse to be out of the court for the offending four days. The raiders turned out to be a small group of home grown brigands interested in the farmer's sheep and a barron's trout, but that did not matter. It had been a convenient enough excuse to slip away. I spent the offending evening drunk with my men outside a small hillfort, raiders safely imprisoned inside. It seemed possible then, when my duchy was still safe, to bury myself in my work if I could not have my duke. 

With the aid of Sergent Lorzo, I avoided being cruel to the boys I trained during the next interminable eight weeks. When I met them on the dawn of the first day of the ninth, it required a concerted effort to control my personal fears and keep from shaming my boys individually for their failings. Tempting as it had been to vent my fears on my charges, humiliating adolescent boys would only make Sergent Lorzo's task of maintaining peace in the barracks more difficult. It would not hasten news of Makala to me. I waited for news of the Duke's party as anxiously as I would await news of Makala when he fought on a different front than I did, though I knew him to be on a well traveled safe road. How could I leave Cortan, I wondered, if I could not endure these days of not knowing my duke's whereabouts.

They returned five days into the ninth week. Dutchess Cybeline wore an expression of intense distaste, her face puckered and drawn as if someone had forced a lemon between her teeth. For a woman of forty, she was remarkably well preserved. She had a streak of white in her otherwise jet black hair that granted her an air of elegance, rather than a preluding her withering. If she kept that visage up for long, however, she would be irreperably wrinkled by the month's end. Duke Ergino, likewise, looked tense, his brows drawn and close, and his chin thrust out in angry defiance to the world in general. He looked like an older version of Makala. Silver Makala's hair, add furrows to his brows and cause the skin to start drooping off his regally high cheekbones, and there would stand Duke Ergino. Except, I cannot see Makala ever developing deep furrows above the brow. Crows feet, yes, lines around his mouth, possibly, but lines put there by years of worry I cannot imagine. Not that I will ever get a chance to find out now. The few other families from court who accompanied them also looked somber. Something had gone amiss. 

I scanned the returning party until I saw Nisrita riding beside my duke. The failure of their union was not the cause of  the party's tension. Both Makala and his wife displayed a single gold chain around their necks. Everyone but the guardsmen in the party did. The chain signified marriage. Some men were sentimental enough to wear the chain into battle, where it would be picked off their dead bodies if they fell. Newlyweds wore no other adornments for the first few months, as a meditation of the importance of the act they have committed. The glint of gold on Makala's brilliant white shirt sealed my fate.
I turned my attention to my men. This is what my life would be until I could find a positition elsewhere. I had been cast aside for the good of the duchy. As a servant of Cortan, there was little I could say to argue. 

The afternoon had worn to evening by the time the five new wives were announced and the happy families congratulated. Duke Ergino adopted the gifted healer into the family. Father Anglius blessed the unions in the Preserver's name, and the girl's adoption in the Maker's. The court presented an endless row of gifts and adulations upon the new dutchess and the blushing brides. I lay my presents at the feet of my duke and his bride. The girl seemed more frightened of the court than she normally did. She made little eye contact with no one, least of all her newly adopted parents. It was not a sign of modesty, but fear. She stayed by her new husband's side as if she were his shadow. Makala positioned himself protectively over her. The lies I had used to comfort myself that Makala would return to Cortan still faithful to me in his heart vanished at the sight of his affection for the new duchess. There was no life for me in Cortan, I understood from their manner. I was glad that my gift to the duke contained my letter of resignation. Jealousy replaced any respect I might have held for my duchess. I kissed the girl's cold fingers, nodded to my duke and returned to my position among Cortan's officers.

The returning guardsmen told many tales over the celebrations. Cortan had been humiliated in Deyalorn. The Dutchess Chamila, Malaka's sister, had not been married to a son of the reigning Duke of Belisale. He had chosen to wed a gifted widow of a northern minor baron who had already given birth to two children with the gift. Chamila was no longer a dutchess, but a baron's wife near the royal court. Her grandfather's brother, Griswold, had been strongly gifted, but none of her six siblings to survive long enough had the gift, neither did any of her half dozen neices and nephews. The Duke of Belisal had suggested that Duke Ergino try  to extend his line to prove that his family was not barren in this respect. Cortan needed a gifted marriage. That could not be denied. Even with the duchy's military might, we lost favour in the royal court because of the lack of gifted members of the Duke's line. Admitting these facts, and putting my jealousies aside, I could not accept this marriage. The small struggling barrony of Romino did not need the shame of its son being exposed as a miscast because a gifted adolescent girl could not control her jealousies. Cortan's northern neighbors, in Firvona suggested that the crown increase Cortan's taxes for the drain on Marsea's Gifted we constituted. It was a maddening request, and easy for Firvona to make, given that Cortan is all that stands between them and the barbarians of the south. If Cortan fell, Firvona's White Tower would be a drain on the nation's resources. We could not force the healers in our Tower to do nothing but breed. Even if we did, we could not produce enough new members to support our armies without help from the interior dutchies. Our Duke had responded with a proposal for war to the south east. The Pensid Mountains were a good source of granite. Their foothills held silver and tin. The Regent and her generals had accepted the proposal. It would give Duke Lukos lands of his own. It had seemed such a prosperous proposal.

I discussed politics and gossip with my fellow officers when a page brought me a message with one word on it. I looked at the duke's table to see Makala and the new duchess missing. The doting husband had gone to see his wife safely retired, I thought, knowing it to be untrue. The note in my hand spoke otherwise. Over our years of secrecy, Makala had devised what we had hoped was an relatively safe means of attracting the other's attention. Unsigned messages sent with two words signified a request for the other's company, one word, an urgent need. My anger and jealousy towards Makala faltered. The relief I felt at the sight of his safe return won.

I became aware of the danger I was in as I approached my duke's rooms. My rival occupied the adjascent room. Frequent visitations would not be possible. The risk of a wife wishing to see her husband was too great. I had no faith in Makala's desire to prohibit the girl he doted on from visiting his chamber. This visit, I promised myself, would be the farewell I had not had the peace of mind to give my duke when I first learned of this engagement. I would do this for the sake of my father's safety, if not for the sake of my safety or dignity.

"Catch,"  I heard a voice say as I entered my duke's room. 

An apple appeared in my outstreached hand. Makala appeared from behind a shaddow to remove the dirk that that sprung in my surprise in my other hand. He kissed me. I let the embrace linger, wrapping my arms tightly around the narrow waist I had not held in three months. For five terrible days, I had feared I may never see it again. I had missed him. "It is good to see you, again," Makala said at long last, releasing me from his kiss. "Your father sends his regards." 

I looked at the apple in my hand. I could not accept it, even if it came from my father's orchards. Makala had filled his table with enticements, a bowl of apples from my father's orchards, pears and walnuts from Deyalorn, the chessboard sat folded to the side, a quiet possibility. I disentangled myself from Makala's arms and returned the fruit to the table. "I cannot accept this."

"Of course you can," my duke said, oblivious to my attempts to separate myself from him. "I dislike hearing you complain about missing your father's orchards every autumn. I have burned your letter. My father has not seen it. I heard the sad news this evening that Commander Tierno died while we were away." 

The sudden change of subject took me by surprise. I nodded. The summer fever season had been hard this year. I lost a man in my command to the delirium, and another lost his unborn son when his pregnant wife fell ill. 

"My father will consider his replacement tomorrow. Convince him it should be you."

I was not surprised that my duke would promote me for the position. I knew the heights of my duke's ambition. It flattered me that he held my carreer within its scope. I grinned. If I got the position, it would make me the youngest commander in Cortan's army. My father, when he had heard of my promotion to Captain had been so proud. He had prayed that he would live to see me as a general in Deyalorn. This promotion would serve my father's house well.
 
It would be a continue the dreams Makala and I had spun over the years. I would lead an infantry unit in this year's campaign to conquor lands for my duke and duchy. I would gladly, unquestioningly, risk my life and my men's lives to add to lands he would inherit. I would gift him as much territory as he wished to rule. Makala, or Cortan, may publicly reward me with rank, but that would never be the true reason behind why I fought. 

"After you have done that," Makala continued, recalling me from my fantasy, "I believe that the sergents in charge of your trainees knows the Pensid foothills well. Inform him that the will lead the second scout troop when we march this spring."

Jaquim. "Yes. Sergeant Lorzo." 

"Good, I am glad that is settled." Makala replied, smiling, returning an apple to my hand. 

The fruit reminded me that I had walked headfirst into one of Makala's brilliant schemes. "No Makala, I cannot." I could not afford to be wooed by the man I loved again. 

"Please, Timmon." It was not like my duke to plead. He had my attention. "She is pregnant. I will not touch her again."

I recoiled at the thought the my duke had spent nights with my rival as I would never do. His humility was not enough to overcome my jealousy. "She is also in the next room," I bit off.

My duke permitted himself a hint of exasperation. "You are not listening. Nisrita will give me a child. Then she will go her way to study and let her gift bloom. She will bear me more children once she is past her gift's peak. I have made her my sister as my father's heir. The orphan has a family. It is a better arrangement than most women can hope for. We have talked it over. She has agreed that nothing would change between us." 

I could not believe what I heard. Makala had built a castle on a cloud based on his faith in a girl. "She has bewitched you into betraying me," I snapped before I caught my anger. I had come here to say a civilized farewell. What Duke Makala did with his wife after that was not my concern. "I should leave you now." I continued more gently. "I will write a letter to your father tomorrow."

Makala planted himself between myself and the door. He spoke in ambitious, seductive tones. "We could rule together, the three of us. You at the head of my army, she leading my Tower. You wanted this once." 

I had wanted, still wanted, more than that once. It seemed my duke could look only towards his duchy. My place was elsewhere. I moved past him.

"Eight years, Timmon," Makala persisted. "My wife and I will live as brother and sister for eight years." I paused at the door to hear him out. "That is a long time, if you will trust me."

It was a long time. A gifted woman does not leave the prime of her healing until she is in her early twenties. If the duchess would keep her word, my duke had given us eight years without interferance, if one could count the word of a girl. I wanted to believe him. My fear of discovery was too great. "I will consider it," I said and left my newly married duke to his solitude. 

I did not know then that our discussion that night contained both the seed of what would become the the brightest months of my life, as well as the seeds of the destruction of everything I held dear. I wonder now what would have happened if I had not have accepted the challenge so readily, or grinning like a dog at the sight of his master holding a bone. If I had stayed where I belonged and done the job I knew how to do well, could I have saved my duchy and the rest? But I could not see the future, I could not even see the present clearly. Makala had filled my head with exquisite dreams that questioned the truth that miscasts were only fated for ill in this world. 

\begin{comment}
"Put that away, Timmon," Makala said emmerging from a shadow to gather the rolling fruit. Annoyance and mirth played in his tone. "You've bruised it," he reproached, placing it on his table.  He had set it with bowls of apples, pears and walnuts, along with a thin wooden box. The chess board lay folded and unset to the side, a quiet possibility. 

This was how so many evenings had begun during our last five years together. I would enter and read his mood and his desires from the arrangement of his table. From that, I would choose how the evening was to proceed. That day he was a suplicant. He presented me with objects he knew would please me, fruit from my father's home I cannot find in the south, the game I loved. I resolved not to take the bait. "Why did you call me here?"

"The apples are from your father's orchard," he answered as if he had never deserted me. "The pears and walnuts I got at the market, I'm afraid. I dislike hearing you lament the loss of your father's fruit trees every autumn."

What little dignity I had would not endure this glibness. "I have no desire to discuss fruit with you. I hope you found my father well. Goodnight Makala."

"Wait." His voice barred my way to the door. "I called you here for the same reason I always call you here, Timmon. I have missed you."

The emptiness I carried with me for the last two months heaved and thrashed inside my chest. I could not relent in spite of its protestations. He had made a choice. He may have had valid reasons, but he had made it all the same. I could no longer come to him as before. "You are a married man," I reminded him.

He looked at my chest. "And you are not."

"No." I looked away. Why should he make me feel guilty for that. I am the third son of an ungifted baron. Marriage for me would serve no one, unless my Duke suddenly found himself with a glut of gifted women he could not pair off. I had not promised Makala that I would marry in his absense. I had simply told him I would consider the proposition. 

"Open this, Timmon." He pressed the thin wooden box in my hands, then stood back. 

I opened it, and started at the contents. I lifted my eyes to Makala, then looked again at the contents of the box. Understanding waded into my head, fighting a tidalwave of disbelief. "You have gone mad." I gasped.

"I wed with a simple chain in the hopes that you would have done the same," he replied softly. "I had imagined a world in which you could wear this gift publicly without shame. As is, I hope that you will at least keep this in your possession. The chain you hold is identical to mine."

I dropped the box and the wedding chain on the table. My head swam with disgust. He had wanted me to take a public oath, just so I could turn around a break it in a month. I have no lands or status. The Maker did not even see fit to form me properly. I am miscast. My flaw is irreperable, like a dwarf or an idiot. My word is the only good thing I have. A wedding chain is not a simple gift. The ceremony joins man and wife with identical chains. "Your wife still lives. You cannot marry again."

He looked at me, uncomprehending. "I thought you wanted this." When I did not respond, he retreated to his window. "Keep the chain Timmon. If you ever do marry, reforge it as a chain for your wife."

If he had been angry, it would have been over. His griefstricken generosity snared me. I could not leave him like this. Moreover, he was right. I wanted this. What I did not want was to be an afterthought to his wife. How could I wear the same chain he had wed her with. Yet every autumn, I dreamt I had chosen him for The Day of Unions. The dream always ended either with the two of us drowning in a pool of pitch or being stoned by my family. I would awaken in a cold sweat and wait for dawn to bring his silouette to his window, or any other sign that he was unharmed. In reality, I reminded myself during those dark vigils, I could no more chose him before the Preserver than I could choose a horse. In reality, if the world learned of our acts, we would simply be exiled and given over to the giftless farmers from whose lords we have so systematically conquered lands. Lakes of pitch and trips to Deyalorn to organize stonings were too dramatic and expensive, -- unnecessary when foreign clubs and shovels would prove sufficient. 

Dreams and nightmares be Taken. I still wanted this. I wanted this from him when it was his to give.

I joined him at the window. Beyond it, the grasslands continued as far as the eye could see. Somewhere to the southeast lay the Pensid mountains and our next conquest. I would lead men to victory there. Makala loved his home, and I loved him. I would risk my life and my men's lives to add to lands he would inherit. I would do so gladly, unquestioningly. I would gift him as much territory as he wished to rule. He, or his father, may publicly reward me with rank, but that would not be why I fought. 

On this side of the glass, I could not be his lover. "I am sorry. I cannot do this. If word would get out that the heir to Cortan was a bigamist, it would ruin--"

Makala emmitted a soft mirthless syllable of laughter. "It cannot be bigamy if I cannot seal it before the gods and king."

His stubborn twisting of my words irritated me. How could he possibly think I could accept this. I snapped at him. "You bound yourself to one person, Makala, before gods and king. You cannot promise yourself to me as well." 

Makala did not respond for a long time. He gazed out the window, thinking thoughts I could not read. I did not move from his side. His hand lay barely a foot from my own, its fit as familiar to my palm as my hilt. I watched the fingers twitch to Makala's private thoughts. They beckoned and revolted me in turn.

I would lose my resolve if I kept my thoughts on him. I considered the news I had learned during the celebrations. Cortan needed a gifted marriage. The dutchy would not last long without it. Our White Tower was the second best at training and employing gifted in the country, only bowing to the one in Deyalorn. Its fame and power did not come from our women birthing legions of gifted. The border towers naturally drew healers of the gifted and ungifted varieties because of the armies they supported. Cortan's army was the strongest. It stood to reason that our Tower be strong. "Where is she?"

Makala nodded to the wall behind him. "The next room." I startled. "Relax, Timmon. She cannot see or hear through a foot of stone wall." 

I did not relax. He had been so careful about our meetings until now. Calling me here tonight could ruin us both, if the little girl wished to see her husband. What had he been thinking? "What is she doing?"

"Studying, I believe. She is interested in little else. You asked me why I chose her. I knew she would not impose herself on," he hesitated, "my schedule."

I wanted to plug my ears with wax to keep from hearing his mad plans. In his own perverse, gods defying manner, he had intended to serve his house and myself, sacrificing his integritity and his wife to the Destroyer. He made it sound so tempting and easy; life as it had been, protected by a wife who did not care to know. It was a brilliant plan, if only it were not impossible.

Makala turned from the window with a lighter tone of voice. "What do you make of her?"

I had met her before. It is hard to be close to Makala and not run into her. She visited court frequently, even as a young child, accompanying more prominent members of the Tower. As Makala's favorite, she was always welcome, and frequently brought along to please the young duke. As she grew older, it became more common for her to appear with one or two older, husband seeking girls, from the tower at the Duke's table. She was always a quiet presence at those dinners. Never pretty, but well presented. Makala's fondness for the girl first manifested by teaching her to ride. Then he taught her some of the acrobatic excersizes his horsemen train with. Three years ago, for reasons only he can explain, he decided to take her training seriously. They would meet in the early morning before the rest of the army. She was allowed to continue with the riders in training as long as she was not a drain on Makala's or anyone else's time. She was dogged in her practise schedule, atheletic for someone her size and age of average strength and skill. She was a cheerful girl, if shy. Upon reflection, I recalled a time, not too long ago, when I held a fondness for the girl. She was unobtrusive, and disciplined, her comrades liked her. She was pleasant to have around in the mornings. She did not much resemble the girl who rode in this morning. "She looked ashen and disheveled. Her hands were cold." 

"The journey home was hard on her. She has not been well. She believes she may be pregnant."

"Makala!" I cried out in alarm. A gifted woman, practising while pregnant could have disasterous consequences for mother and child.

"She is only studying, not practising. She will do neither herself nor the child any harm." 

"Are you certain?" I asked, eyeing the door, wondering how I would explain myself if I rushed into her rooms.

"She knows the risks better than I do, and she is not a stupid girl." He watched me reluctantly accept his verdict, then smiled sarcastically. "Though I am pleased to see you do not wish her harm."

"I am surprised at the revelation as well." That smile and that playful voice had haunted me during the months of their absence. Now that they, and their owner had returned, what stupidity kept me from throwing myself at them? I cleared my throat and answered the question at hand. "She has comported herself better before the court. Why did she appear so harried today?"

Makala's mouth hardened. "Because she has been harried. My mother has given her no ground since leaving Deyalorn. I believes she had developed some convoluted chain of events that casts the  blame on Nisrita for Casmira's poor marriage. The only joy she gets is when she is allowed to retreat into her books, which causes her to neglect her appearance, which enrages my mother further. I pity the girl."

"But you do not protect her?" I was amazed. I had watched Makala scare the ghost out of boys who had been rumoured to bully Nisrita. The strength of his devotion to her was almost as strong as his devotion to Casmira. No, it was stronger. He had chosen to marry her. He must love her more than he loved his siblings, more than he loved me. What had changed his attitude towards her?

He mistook my surprise for a reprimand. "Do not think poorly of me, Timmon. It is better this way. Nisrita will give me a child. Then she will go her way to study and let her gift bloom. She will bear me more children once she is past her gift's peak. I have made her my sister as my father's heir. The orphan has a family. It is a better arrangement than most women can hope for. We have talked it over. She understands that if my mother dislikes her, it will only make it easier for her to retreat to the Tower. I had hoped this would leave me with time to,  ... well."

The depth and ambition of his impossible dream amazed me. He would save his house, remain by my side, and protect the girl he had long since adopted in his heart as a sibling. He had every inch of it planned. It would be perhaps another eight years before her gift passed its prime. Eight years is a long time for a man who lives by the sword. It was more security than I would ever have hoped for. I do not know what private arrangement Makala had made with the Destroyer to pull off this fantastic scheme, or how he found the courage to lie about his intentions to the gods and king. I have wondered since if what came next was caused by the audacity of his will to change the nature of Nisrita's and my relations to him. On that day, I simply wanted to believe that all the pieces would fit together smoothly. I only needed to follow his lead.

A part of me resisted. It was too bold and dangerous, I wanted to protest. I cannot wear your wear your chain when you are already married. We cannot meet here with your wife in the next room. "You deserve better than me," is what I said.

"Quite possibly," he replied in that familiar jesting tone. He walked to his table and tossed me an apple. This time, I caught it. He watched me eat half of it before changing the subject. "One of the sergents in charge of your trainees knows the Pensid foothills well, I believe?"

Jaquim. "Yes. Sergeant Lorzo."

"I want him leading the second scout troop." I frowned. They were part of my command. I liked working recconisance with them. Makala waited for me to finish frowning. "I learned that Commander Tierno of the fourth infantry died while we were away." 

I nodded. The summer fever season had been hard this year. I lost a man in my command to the delirium, and another lost his unborn son when his pregnant wife fell ill. 

"My father will consider his replacement tomorrow. Convince him it should be you."

I grinned. If I got the position, it would make me the youngest commander in Cortan's army. If I had know then what that promotion would cost me, I would not have accepted the challenge so readily, or grinned like a dog at the sight of his master holding a bone. I would have stayed where I belonged and done the job I knew how to do well. But I could not see the future, I could not even see the present clearly. Makala had filled my head with exquisite dreams that questioned the truth that miscasts were only fated for ill in this world. 

He had transformed me like this before. I have known Makala for eight year. When I had met him, I had thought myself the only soul unfortunate enough to be deformed and tormented as I was. Three years later, he showed me that I was not alone, that my true nature was not to be hated by everyone. He reshaped our lives to create a space for warmth and passion. Sitting in his room that day, I thought he could perform that miracle again. I believed there was no end to wonders he could create. I wanted to help him. 

Makala returned my grin and wished me goodnight. As I passed the table on my way to the door, he nudged the box with the chain towards me with a beseeching look. I took it, and told him I would let him know my decision soon. I knew, but I wanted to show him. We did not meet again in private for a few days. He gave me time to make up my mind. When I came to him next, it was as commander, with his chain dsiplayed as a line of gold on my scabbard. Some men are sentimental enough to wear their wedding chains into battle.
\end{comment}
\vspace{.5 cm}

It only took a month before Makala's plans started to fray. I returned to visiting Makala at night, but I found my jealousies could not be controlled. My duke showered his wife with attentions in full view of the court, while I lurked in shadows and saw him in secret. Whether he touched her or not, that he shared his affections I found difficult to swallow.

One morning in early winter, my boys had been running through the yellow mud for nearly an hour when Makala appeared to  join his horsemen shooting through the rain and mist at distant targets. The short square figure of his wife was neither with the archers, nor by his side. When I asked him if all was well, he pursed his lips, shook his head and met his men. Coming from a man who returned home serenely after watching his father insulted before the king, I prepared for the worst. By evening, the entire army knew that the young dutchess would lose her unborn child. My jealousy flared. Women in Marsea with access to a Tower's healer rarely die in childbirth. It is not that Marsea's women are stronger. The women of the Tower are very good at saving women's and livestock's lives from a variety of ailments. Yet Makala paced with a drawn face for most of that night, as if he lived in a distant village, days travel from the nearest healer, while the Head Rosaria, the leader of Cortan's Women's Towe was not attended his wife in the next room.

Makala's concern eventually passed, as Nisrita's health slowly improved. I did not understand or approve of his preoccupation.I told myself that I did not rank below his wife in his life. I did not believe my arguments. I stood beside Makala inspite my misgivings. If I gave them voice, Makala would only remind me that I am a grown man, and she is a still a child, and that he expected me to act according to my age. I did not need that humiliation, and Makala did not need to be burdened with my jealousy. I did what I could to make the night pass peacefully.

The week passed with rumors that the miscarriage had happened because the young dutchess had been practising her gift, and whispers that she had stronger loyalties to the White Tower than to Cortan. The latter, I gather, was spread by Dutchess Cybeline. It sounded scandalous, but in the end did not matter. The White Tower exsisted to support Cortan's army. Whether she was loyal to the tower or Cortan, seemed a matter of splitting hairs to me. Makala denied all these rumours, and I believed him, though I never defended his wife in public. The situation upset him, and he claimed, threatened his wife's recovery. I patiently let him share his worries with me, though I found it hard to empathize. I wondered how many mistresses willing let their lovers speak to them of their wives.

Several days after the miscarriage, I lay drifting dangerously towards sleep with Makala blanketing himself over my right side when the door to his room opened. Makala expended considerable effort to make it widely known that he retired early, and did not like being disturbed in his sleep. Short of an emergency, we should not have been disturbed. In case of an emergency, there should have been a knock. This should not have happened. I bolted upright. We were ruined.

My heart raced as I rushed to snuff the candles by the bed. I hoped for safety in the cover of darkness. She was in the room gawking at us before I had reached the second taper. It would have done me no good had I succeeded. She held a light in her hands. Makala rose, to me it seemed he crept to a sitting position, and found something to cover his body and mine.

"What is it Nisrita?" he asked calmly. I hated him for his calm, for his negligence, for his arrogance. How could he have been so careless as to let that girl slip through into his chamber without warning? His priorities had changed after marriage. Whatever dreams of a life together he had spun for me, I shared him with her now. As a married man he was less sensitive to the risks we lived by. How could he possibly not be alarmed?

The girl closed her gaping mouth at his voice. "I am sorry, my liege. I did not mean to interrupt you."

I watched him smile at her with too much kindness. "If you will give me a moment, I will meet you in your chamber."

I stared at the door as she closed it behind her. It was finished. I was ruined. Everything we had was over, sacrificed for the love of a nosy adolescent child. Under Cortan's current politically weak position, the duke would not hesitate to cast his son out, to say nothing of a soldier in his army. Images of my future uncertain, desolate life swam before me. I would be sent away at the southern border, behind the news that a Marsean commander had been exiled and a list of my crimes. It was the same punishment reserved for deserters and cowards. If the gods were kinds, I would be given a horse. It would be up to me to survive alone off this inhospitable land against my country's enemies. I would be alone. They would not permit me exile with Makala.

Makala heaved a sigh behind me. "Since when has she addressed me as her liege in private?" He rose and started dressing.

It was not calm he eminated, I decided. It was callousness, or madness. Or perhaps it was simply that he had stopped caring for me. "You are worried about what she called you?" I hissed at him. Habit kept me from raising my voice. It did not matter any more if the guards learned of my presence, if they did not know already.

"Yes." He continued dressing.

Panic gave way to disbelief. I watched his steady composed movements. Could it be that he had wanted to be discovered? It was not like him to embark on such a self destructive path. Did he simply not realize the danger we had just found ourselves in? He had always been so cautious. That too was against his character. Had I misjudged his strength? Had his wife's miscarriage and his duchy's political problems caused him to lose his senses? 

I sat and stared at him, pinned by my confusion. If he had asked me to flee to escape the inevitable punishment, I would have lept to my feet in an instant. But he stood before me calmly lacing his shirt. It drained my composure and my ablity to make sense of my surroundings.

He looked at my still form and urged "Dress Timmon. I told her I would be meeting her soon. You should not be in this room alone."

His voice was a quiet, and warm as ever. His words were rational. His bearing and behaviour did not make sense. Disbelief gave way to anger. 

He had ruined us by allowing her access to his room so easily. He broke his promise to keep our secret. If he had wanted to be discovered, he should not have involved me. He owed me that much. I would never have betrayed him. In spite of his words to the contrary, as a married man, he no longer cared for me as he used to. His concern was for his wife. He felt no shame, no alarm, for what he had done to me. "You did this on purpose." I blurted. "You love her. You wish to be rid of me. This was a disgusting way to send me off."

"Timmon." The word was sharp and cold. "Enough. Dress."

"No. Your wife has destroyed my life. You have conspired with her. At the very least I want an explanation."

He sighed and started gathering the pieces of clothing by the bed. I watched him and let my anger bubble. When he finished, he handed me the bundle. He waited until I had started covering myself before he spoke. Looking back at that day, I must have seemed an infant throwing a trantrum before him. "Your life has not been destroyed, and I have no desire to be rid of you. But I will explain. Nothing has happened yet. Nisrita has almost as much to lose by revealing what she saw as we do. My father may have made her heir, but that will hold little weight if he disinherits me. Even if she were to remain in this house, she would have to live with my mother. I do not think she would agree to that arrangment absence my ability to offer her some protection. You are frightened for no reason."

"Dog's bollocks, Malaka. Anyone knowing about us is a threat. When did you lose sight of that?"

"I am fully aware of the potential consequences." His voice was patient and exasperated, he spoke as if to a dull child. "I am also aware that she is a thirteen year old girl."

"Find a way to silence her." I spat.

"Silence her, Timmon?" He slightly raised his quiet voice. It did not matter that I had provoked him against me. I only cared that he was no longer showing that maddening calm. "You want me to threaten a child? Even you are a better man than that. She has had a difficult week. She deserves my compassion. You would see that if you were not blindly obsessed by the fiction that I no longer care for you." 

How could I be obsessed with a fiction, when the evidence of his favoring his wife surrounded me? Even now he was going to comfort her after she had put my existence at risk. How could I not believe he no longer cared for our safety. He used my brooding silence to compose himself. When he next spoke, he had returned to his maddening calm. "Her coldness worries me. I can trust her as long as she has my best interests at heart. All three of us are safer if we see each other as allies."

I did not like the emphasis he put on that last statement. As if I were somehow at fault in this situation. I was not the one arguing that a thirteen year old could wish a miscast like myself well. I made one last attempt to get him to explain his actions. "Why did you call me tonight? If you knew she might come, why did you take that risk?"

Makala smiled sadly. "Because I have also had a difficult week, Timmon. And because I leave for Firvona in the morning. I will not see you for at least ten days. At least try not to antagonize her in my absence."

He left for his wife's room. As I had dressed, I left for my own.  He had no right to expose me to the risk of discovery. Whatever the circumstances, however long his upcoming trip, or however dangerous, he should not have given some indication of the new dangers I faced by meeting him this week. Our lives were intertwined, whether he respected that or not. 

I considered leaving that night. I considered taking a horse and deserting my post. If I were caught, my punishment would be no worse than what I faced if the girl talked. I walked within sight of the stable, telling myself that it was he who did not deserve me after tonight's performance. And then I stopped. The thought of spending the rest of my life without him frightened me. I could not face the possibility of being alone again. If I had to secretly live in fear of being an outcast, it was a thousand times better to live that life with someone than to live in fear alone.

I mulled the night over alone in my bed. I hated Makala for not caring about how I felt about exposure. I prayed that he would succeed in making the young dutchess hold her tongue. I told myself that he would not leave for Firvona without settling the issue to his satisfaction. If he could not, he would send word to me. As the hours passes with news, I forced myself to sleep.

The morning brought a messenger boy with the simple message that "The duke has taken care of his part of yesterday's arrangements." The implication was understood. Makala had asked me not to antagonize his wife. My part of the arrangements was easily enough to manage. I had no intention of speaking to her.

****

That intention was more easily made than followed through with. Six days into Makala's trip, Nisrita sent for me. I had no grounds to deny my dutchess an audience, so I went.

"You wished to see me, dutchess?"

"I did, Commander Ramino. I must ask a favor of you."

The girl stood in the middle of the room, before a caged songbird on the floor. Her table, and bed were piled with books, the floor littered with ink wells and vellum. She had carefully left a circle several feet wide centered on the birdcage uncluttered. She was still dressed in the trousers and shirt she must have worn under her armor in the field this morning. As she could not be seen by Dutchess Cybeline attired as she was, she must not have left this room since returning from her morning's excersizes. Her wedding chain and the Tower's talisman the only two pieces of ornamentation visible on her. She was still a newlywed. She was practising again. The brusqueness of her manner explained her shyness and quiet demeanour in court. She knew better than to speak when her tongue could do her harm. 

"What may I do for you, dutchess?" Not offerred a chair, I remained standing before her now seated form.

"I wish to serve as a healer in the upcoming war."

I stiffled a gasp of laughter. "That is not for me to decide, your grace. Have you spoken to the Duke?"

"My husband will not listen. He obtained the position of commander for you. I want you to secure me a position with the healers."

I paused to survey her. No one seemed to have taught her that insulting the people she sought favors from would not encourage them to serve her. The girl balancing herself on the leather strip every morning could be a threat to an unarmed man twice her size, if caught unawares. The girl who used to visit the duke's table would be useless at espionage and intrigue. But that was before she became a dutchess. More importantly, that was before last week.

"I cannot do that, your grace."

"You are wrong. You are a commander. You have some control over who serves under you."

"This is true, your grace. But I will not go against the Duke's wishes in this matter."

She looked up at me sharply as I spoke. "You care deeply for my husband."

She purposely mistook my meaning. I had meant Duke Ergino. "You are threatening me." 

"I would like not to be." she said lightly, and turned to her books. "What is your answer?"

"No, your grace." There were a thousand reasons why letting a thirteen year old girl onto the battle field, even in the healers corpse was a bad idea. There were a thousand more reasons against putting all the Cortan heirs onto the field at once. If she spoke to Makala, he had presented her with all the arguments. 

"That is a pity," she said and started reading.

The audience was over. I recalled Makala's reluctance to scare this girl, and inwardly gaped at his conviction of her innocence. "You will gain nothing by marking your husband as a miscast, your grace."

"Makala!" She turned to me in genuine surprise. "My husband is not a miscast. He is a good man. It is impossible that the Maker forged him from a faulty mould. He is flawless." She paused to take a breath from the vehement praise for Makala. "You on the other hand," she continued, "are a different matter."

That seemed to sum it up then. Our feelings for each other were mutual. Makala's trust in her was completely misplaced. And I was trapped. I had one question left, out of genuine curiosity, rather than any hope of it saving me. "Who do you intend to bring down with me?"

She returned to her reading. "If I search around among the bell boys and stable hands and pages for long enough, are you certain I will not find someone?" This time, the audience truly was over. 

"I will speak to the Duke, your grace."

"Thank you, commander." She rose and saw me to the door. 

Preserver help me. Makala would not forgive me for this.

I met him on the road  five days later. I had decided to forgive him for his negligence on the night before he left. I did not see any other choice. I could not leave him. I had to tell him what had passed between myself and his wife. I was not in the habit of escorting him home, but the circumstances seemed exceptional. He rode ahead of his party with me for privacy. In general, he looked pleased with himself. "It is good to see you, Timmon."

"And you Makala." There was more to say than that, but we have gotten in the habit of leaving things unsaid. "How did the visit go?"

"Firvona has pledged an acceptable number of troops. I have talked the Duke out of the idea that as my father suggested this military manuever as a counter measure to increased taxes, that it should be on Cortan's burden alone. The Duke wanted the lands we took in the southeast to be given to one of his sons in return for his support."

"That is an exorbitant price. Lukos will not be pleased." Tradition dictated that new terrirories were settled by the conquoring duchy. Cortan would lead this war, expend the most men, arms and money for the cause, but with the support of neighboring dutchies. The support of the interior duchies was generally given as a gesture of good will and thanks for the protection the border dutchies provided against Marsea's enemies. Duke Ergino had promised Lukos the new lands as a place to start a duchy of his own.

"Actually, he will. I convinced the Duke to settle one of his daughters in the new land, by marriage. She is moderately pretty, and well tempered enough to please my mother."

I laughed in amazement. Makala could talk a dog out of its bone. The word amongst the other commanders was that Firvona did not like losing its entire frontier border to Cortan's steady conquests. As an interior dutchy, it was not adding to Marsea's military glory. A good marriage on a new border dutchy could go far to ease tensions. I wondered how Makala had managed to talk the Firvonese Duke into accepting those terms.

Makala interrupted my speculation. "You did not come out all this way to ask about Firvona."

I sighed. Makala was about to talk me out of my bone. "I spoke to the young dutchess in your absence."

"Ah." His tone darkened. I hated ruining his mood.

"She wants to be positioned with the healing corpse."

"You refused, of course."

I took in a deep breath. "I told her I would talk to you."

Makala looked at me sharply and saw through the half truth immediately. "You did more than that Timmon, otherwise you would have waited until I reached home."

"I positioned her in my batalion," I admitted sheepishly.

"Timmon!" His exclamation contained more terror than anger or disappointment. "Remove her from the position."

"I can't."

"Horse eggs. You put her there, you can take her out. The war has not started."

"She threatened me, Makala."

He laughed in my face. Why could I not get this man to understand the threat his wife posed. "The newest commander in Cortan's army comes to me to complain of being threatened by a thirteen year old girl. If word gets out to out enemies, we will be beseiged by children."

His glibness, his callousness, his blindness, whatever it was, it was infuriating. "Stop that. In the Maker's name. Why will you not treat this situation seriously."

"I am, Timmon. I had hoped you could manage a child for a fortnight." He sighed his disappointment. "Give me one full day after I get home. If I cannot resolve this by then, you do not have to recall her."

He had an answer by the next morning. He approached me as I dragged my boys sleepily out of their barracks into the early winter rain. "I cannot talk my wife out of serving under your command. However, she has apologized for her rudeness."

"I see, your grace. Thank you." I said, with a significant, but hidden, wave of relief.

"Since you have gotten her into this position," Makala continued sharply, "I expect you to train her. She will be with your boys within a quarter hour. My father has no objections to her joining your group."

I stared at Makala, uncomprehending. "She will not be serving on the front lines as a gifted fighter, your grace. The dutchess will be in the ranks of healers."

Makala laughed boisterously at my comment. Perhaps a different man would have clapped me on the back, but we did not touch in public. "You are a talented trainer, Commander Romino, but you are not miracle worker. I do not expect you to ready her for the front lines. However, if all three heirs of Cortan are to be in the Pensid hills this spring, they should all at least know how to defend themselves until help arrives." 

He wanted me to teach her basic hand to hand combat in the next three months. I could not make her a proficient infantryman, but I could give her a chance at defending herself. His reasoning did not make sense. Up until the day he married her, Makala had always elected to ignore that she was female. People thought it unusual that he taught her to ride and shoot, but the complaint was the manner of his teaching, that is, with other soldiers, than the subject matter of the lessions. It is not unusual for women to ride, and a few even prefer hunting with a bow to strictly using a falcon. When he handed her other weapons and started her agility training, people saw Nisrita, then just an exceptionally gifted orphan, as something of a deviant. No one went so far as to call her a miscast woman, however, and most people allowed the young duke to have his peculiar quirks. I feared her engagement in hand to hand physical combat with a group of boys was taking this strange hobby of Makala's a bit too far. "People will talk, your grace."

"They will. But with the duke's consent, they will not do so in her hearing. Do you have any personal objections?"

I had no objections in the world to being given permission to teach that meddlesome little trollop to keep her eyes to herself and her nose in her own buisiness. "She will get hurt, your grace."

He smiled and said "I fully expect her to." 

There was mischeif in his eyes. He was scheming. I did not know what his game was this time, but he was using me for something. Makala was always scheming. He was very good at it. He would study a situation until he identified the people who wanted to help him enact his plan, even if his plan eventually went against their own self interest. It was never safe being caught up in one of his schemes without knowing the full details. The problem was, he always made you want to help.

And thus it was that I found myself responcible for the training of the young dutches Nisrita of Cortan. One among a group of thirty seven trainees I had in my tutelage at the moment.

The boys I trained were between the ages of eleven and fifteen. For better or for worse, the priests liked to cut young boys. After being selected for the army, they are sorted into different branches. Most end up in the infantry, though some become scouts, cavalry or unhorsed archers. A few show enough promise to be selected for the elite group of mounted archers. Every year, one or two newly formed eunichs will suddenly develop the gift. The gift is most common and far more powerful in women. The gift is the strongest amongst adolescents. Something about losing one's balls at that age must make it easier to manifest. The White Tower claims those, returning them in a year or two as a gifted fighter, an incredibly valuable man to any brigade. He is capable of performing moderate amounts of healing in the heat of battle. Every veteran has a friend who has been saved by a gifted fighter. Marsea's army could not be half as efficient as it is without them on the front lines.

The young dutchess appeared and raced with my boys. She was older than every two out of three boys, but smaller than more than half of them. She kept up with the pack when they ran, and was faster than most through the hurdles. With stones tied to her back, she quickly fell to the bottom of the pack. Fortunately for her, she would not have to defend herself wearing heavy armour. On the other hand, she climbed like a cat, but I had expected that. One of Makala's odd whims had her climbing up poles since she was eight years old. When I set the boys to wrestling, I sent her to move a pile of boulders. Makala had given me permission to see her wounded. I would not be responsible for anything else happening to her. Furthermore, the boulders would help her develop her deplorable strength. 

I contemplated the strategy for the rest of the morning as I corrected and advised my boys. I did not like to humiliate my trainees individually, but no boy like being beaten by a girl. Nisrita had been training in one form or another for longer than most of the boys under my care. She would certainly beat some of them. I would have to find a way to keep from insulting them. Then there was a question of weaponry. We did not use bladed weapons often. The injuries they caused took too long to heal. A broken bone from a mace could heal in a little over a week, bruised or dislocated joint took a day or two before a soldier was fit enough to train again. A serious injury from a sword, however, could take weeks to heal, a moderate flesh wound several days. A lost limb or digit could not be replaced. We brought out blades only once a week. I had hoped to have them working with axes today. Makala had given me permission to injure his wife. Until I knew more about his intensions, I did not dare damage her too much. He would forgive me if I returned her bruised more easily than if I returned her bloodied. I had the armory bring out mauls. She had no strength. It would be satisfying to see her beaten black and blue.

When the boys finished wrestling, I called her from her pile of rocks. She had moved precious few. She came cheerfully and obediently. The girl who had threatened me a little over a week ago had disappeared. Perhaps she truly was sorry. Perhaps Makala had taught her to how to carry herself on the training grounds. It was hard to tell. At any rate, she gave me no trouble. Neither did she indicate that she feared what I could do to her in the field. It did not matter. She would soon learn.

I paired the group for sparring before addressing them.

"We have a new member in our ranks today." I announced.

"Yes Commander Ramino," they chorused back.

"She will be sparrig with you. You will not be afraid of hitting her. She certainly will not hesitate to hit you." It was better not to intimidate the boys by reminding them of her rank. Most of them came from landless houses if they came from houses at all.

"Yes Commander Ramino," the chorus responded less certainly this time. Barring tales of the warrior queens of old, were all in uncharted territory. The young duchess gave no sign of emotion at my statement. 

"She has trained alongside the Black Riders, though she is not one of them." I saw boys look at each other uncertainly Every one of them dreamed of riding with the riders one day. "She will be better than some of you. Any boy found teasing another for losing to her will have to answer to Sergent Lorzo." Jaquim oversaw the boys' barrack. He coddled them after I finished filling their bodies with bruises, but he was firm in his own demands of them. He was a good person to put in charge of them, young enough that most of them aspired to be him, generous enough that most loved him, and harsh enough when needed that every last one of them thought twice before crossing him.

"Yes Commander Ramino," the chorus shouted back. Sergeant Lorzo was firm known ground.

I paired her with the weakest of my boys. I did not know what her abilities were. She beat him easily. He had been with me for less than a month. He knew how to use the maul, but he was both slower and weaker than her. He brought the head up, or to the side, but swung it too slowly. She dodged and bobbed around his blows, using the weapon more as a staff with the weight close to the ground than a hammer. She eventually drew him into the open, away from the other boys, dropped her weapon, wrestled him for his, and forced him to yield using her knees and fists. It was a creative solution, though it failed to help her learn the days' lession. Her form and strength were attrocious, as I expected. She depended on her agility and footing to fight. This too I expected. That she fought intelligently was new information, as was the fact that she liked to fight in the open. That would have to be fixed. She would not always get to chose where she encountered her enemies. The boy yeilded too easily, and still had little control of his heavy weapon. He either swung too slow, or too wildly. He had gotten better at running with stones today, but that was not enough. I decided to set them both to moving boulders tomorrow.

I reprimanded the young dutchess for leaving her weapon behind while sparring, and paired her with the winner of the next strongest pair. She beat him easily. He was as strong as she was, but slower. She followed her instructions not to discard her weapon, leaving the maul head close to the ground, taking swipes at her opponent's legs. The tactic would only serve her against weaker foes. 

Her third foe was dimwitted and clumsy, but as strong as a bull. He came to me a shepherd's son, used to wrestling sheep during shearing season. At least that is the excuse he gave for why he could pin men much larger than him to the ground so easily. She used her weapon as before, but he refused to be drawn out to give her confidence. Apparently he had been watching her. I had not given the boy's wits enough credit. She changed tactics, keeping her distance, hoping to tire him out. They continued for a long time. They attracted a circle of watchers from the other pairs who had finished their matches. The watchers clearly supported Pinnoch, the shepherd. I watched the crowd of her opponents friends sap Nisrita's will. She slowed slightly, and took her eyes off her opponent. It was disgraceful. She should know better than to do that, he should have known how to take advantage of her hesitation. As he did not, it was another minute until she knocked his feet out from under him. He threw himself into her. Even the slightest skill at defense would have allowed her to escape his grasp as he lunged. Lack of skill at defense was precisely why she was in my care at the moment. Once Pinnoch got his arms around her, she had no chance. The circle of boys errupted in a vicious chant of Pinnoch's name. I saw, rather than heard her yeild over the uproar. Pinnoch did not
care to see or hear. He twisted her arm behind her back until I could hear her cry of pain, then used her long braid of hair as a convenient handle to bash her head against the ground. Something in his expression told me he had found a rock in the yellow mud to use to crack her skull open. 

I was disappointed to learn how unsatisfying the sight was. Earlier this morning, I had wanted to do something very similar to the girl. The display before me only stank of a lack of discipline. It did not hold even the slightest hint of the sweetness of revenge. 

The boys parted as I stepped into the circle. "Off her Pinnoch! She has yeilded." I brought the flat of my sword down across his back. Some trainers carry whips for occaisions such as this, but I dislike treating my boys like cattle. The sight of my steel called some of the watchers out of their frenzy for the girl's blood. It took three of them to separate the pair. 

Nisrita face was swollen and bloody. I thanked the Maker for the gift's ability to prevent scarring. Her shoulder was at least sprained, but she could limp off the field with the Tower's healers. Pinnoch could not put weight on his right ankle. I had him stay back with the rest of the crowd for a moment. He needed this lession as much as the rest of them. 

Every once in a while an intellegent boy passes through my hands. He fights with his brain as well as his body, and invariably, he starts off using unconventional techniques, as Nisrita did. He frustrates the boys in my care, until he is made to understand that he must learn to fight with the weapons I give him. The more experienced boys then proceed to show him his place in the pack, until the next time I allow them to spar freestyle, when the newcomer shows his claws again. Such a boy causes some tension in the pack, but never for long. After I finish with them in the mornings, they all return to the barracks together. They eat, sleep and work together. He quickly becomes one of them. Nisrita will never be one of them, and they will not accept a girl into their ranks. How could I expect them to? I could not favor her, and I coud not let them kill her. I realized that bladed weapons would have to wait for a day when she was to injured to return to the field.

"Who beats a soldier that has yeilded?" I asked my boys.

"The giftless, Commander Ramino." The chorus answered. A gifted healer can work miracles, but she cannot bring back the dead. Our enemies did not fight to wound us, they fought to kill.

"Are you part of the giftless hordes?"

"No, Commander Ramino."

"Then I expect better of you. If I see this morning's performance repeated, the next time you face her will be in a free form fight. She will bring her talisman. Is that understood?"

"Yes, Commander Ramino." They boys shuffled anxiously. They feared her now, but that did not guaruntee her safety. A trained gifted in possession of her talisman was not completely helpless. She could inflict pain as well as heal. Using the gift, I am told, requires a good deal of concentration, so causing pain is not something a healer can do when pressed, unless she has been specifically trained for the act. Most healers to follow armies have this training, but I did not know if Nisrita did or not. Healers do not attend armies until they have finished their training. Technically, Nisrita was too young to be in my command. Few of my boys knew any of this. All they knew was that the White Torturers were known for causing excruciating pain without leaving a mark. They were a boogie man every child feared.

I paired the boys off again and considered my options. Nisrita had to be taught to defend against an oncoming attacker. I did not trust my boys to work with her. Makala had made her my problem. I could not pass her off onto any of the other sergeants or captains. I certainly was not going to wrestle her myself. What choice did that leave me? I could either ask one of the eunichs who had passed through my hands, or one of the elite riders who knew her to work with her in his spare time. I chose the latter. Her training was similar to theirs. A rider would know how to best use her skills against my boys.

Somewhere in the middle of my contemplation, I found myself cursing Makala. He had me in mate. He put his wife in my training. He knew that would make me feel responsible for her. He knew I would not endanger her life. He had cornered me into watching out for her. If I practised that feeling every morning for three months, it would become habit. He would rid me of my jealousy. I was in love with a man as stubborn as a mule and as devious as a snake. What had I done? It would be pointless to drag this game on. I would let him win.

I was surprised that evening to find Nisrita dodging sandbags swung at her by a pair of riders. Makala stood a short distance away, eying her critically. Her hand was bandaged, but she did not favour her shoulder. Her wounds must have been lighter than I had imagined. I had expected her to be recovering for another day, at least. I reconsidered my plans for the next morning. Blades would have to wait.

Makala thanked me for my morning's service as I passed, but said nothing else. We stood in clear view of two of his comrades. It was too public a setting to talk. 

The second morning with Nisrita was much like the first. She showed a marginal improvement in her ability to defend herself. The pack showed a unified and vicious desire to kill her. I was disappointed in my boys, but not surprised. She beat three boys before she had to be carried off to the Tower. I reminded them of my promise to let her bring her talisman, and hoped her injuries would buy me a week's time to calm their nerves.

Two days later, Nisrita found me in the officers' room. She was dressed in the Tower's formal robes, and beautifully veiled. She stood outside the door, leaning on a pair of crutches. "Is there a place we may speak in private, commander?" 

I led her down the narrow balcony to a small room that quickly emptied itself of infantrymen at her presence. After we had both seated, I asked "How may I serve, your grace?"

"I wanted to let you know that I will be fit to return to practise tomorrow morning." I could not see her face, but her voice sounded immensely exhausted.

"You need not have come to me personally for that, your grace," I said, eying her crutches. I was training for Marsea's military before Nisrita was born. I had not misjudged the injuries she incurred two days ago. It was impossible that she would be fit to fight again tomorrow.

She responded to the disbelief on my face. "Have you noticed that gifted fighters return to the field faster than ungifted soldiers?" she asked sharply. I had. "There are two main barriers to how fast we can heal a wound: the pain the act of healing causes the wounded, and the exhaustion the act of healing causes the healer. The latter is easily overcome in peacetime. A dozen healers could make a man who is holding his stomach in his hands whole again in two days, if the pain would not kill him first. As a result we have to work so slowly on the injury that infection often claims him before we can force his body to stitch itself back together." The graphic imagery made its point. Healing hurt, it had a reputation for being unpleasant, but so did getting injured. I had thought the pain to be a result of the injury, not the act of healing. "A broken leg takes eight or nine days to heal because we do not wish to torture our patients. Any gifted, even a gifted fighter, learns to channel the healing energies when we heal others. Thus we can control it better when being healed, feeling less pain. My leg will be better tomorrow."

Her tone was supercilious and didactic. She did not posess the finess required for political or even military speech. That superiour attitude, from a child my boys had just beaten bloody, made it difficult not to laugh at her impudence. It  hard to resist the temptation to let her self important gifted aloofness be humbled by my boys' ungifted but talented weapons. However, encouraging that behaviour would only make my mornings miserable for the next three months. They boys had to learn to respect her, or at least fear her for her own sake. Otherwise, I would spend my days punishing them and defending her. I needed time to help them settle into this new order. "Take the next five days to learn to defend yourself, your grace. Stay away from my boys in the mornings. I promised them they would be allowed to fight you with weapons of their choice when you return. Have you learned to defend yourself with your gift?"

The young duchess hesitated. When she spoke, there was fear in her voice. "I am learning."

"In five days, your grace will have the opportunity to practise your skill. Bring your talisman."

The veiled head nodded. "Yes, commander."

I looked at her sitting before me. Nisrita was exhausting herself healing the wounds that kept her from fighting. The scars and bruises that did not hinder her, she left untouched. I assumed that was why she wore the veil. The last time I saw her, Pinoch has left her face a patchwork of red and purple. She could not be seen outside the field and the infirmary like that. She did not need to put herself through this gauntlet. Makala was playing her as well. What else could force her to endure this beating. I wondered what he had said to her. Then I realized that I did not know the real reason she had come to see me. I sat back in my chair and asked her. "Why did you wish to see me, your grace?"

Her embarrasment showed, though I could not see her face. She looked down and made a motion that would have been shuffling her feet if one leg were not splinted. "I wanted to apologize, commander. For threatening you the other day."

"The duke has already sent me your apologies, your grace." I said curtly, though it was satisfying to hear it from her personally.

"I wanted to do so personally, ... and," she hesitated, "I wished to speak to you in private."

I groaned inside. When I had served as a sergant in charge of a group of newly gelded infantry, it fell upon me to hear about their youthful problems and help them ease their troubles with the other boys. I hated it. I was nineteen, lonely, and new to Cortan. I had enough problems of my own to keep me up at night. Eight years later, I like the task no better. "There are confessors at the temple if you need to unburden yourself, your grace. They are more practised at listening than a military commander."

"I cannot confide in a confessor about this," she said hurriedly. "No one else knows about..." She let the rest of the sentence hang like an executioner's blade in the air between us. 

I sighed. "Continue."

"When the duke wed me, he promised me that nothing would change between us. He made me promise the same to him. I made that promise without knowing what I was saying." I was certain the dutchess could see me through her veil, though I could not see her. I hid the distaste I felt for this confession from her. I did not want to know how they had wed. "Lying in the infirmary for the past three days has given me a lot of time to think. I understand better what I promised that day. I do not wish for anything to change for you."

When she finished, she waited nervously for me to absorb her words and respond. Was I ashamed that an adolescent child had come to this magnamous conclusion before I had? I do not think I had seriously considered that question until now. It certainly did not cross my mind to be ashamed then. Perhaps I should have been. It does not matter. That emotion could not have changed the future. That day, I felt satisfaction at the sound her words, but still little kindnes towards her. "Do you wish me to thank you for this, your grace?"

"No, commander. I simply wished you to know. I hope that you will not hate me for what I have done." 

She spoke haltingly. It was either hard for her to find the right words, or to humble herself before me. Without being able to see her face, it was impossible for me to learn her full intentions or desires. I ventured a guess. "You do not have to continue with this training, your grace. If this was the lession the duke wished to teach you, I will consider us to have no ill will between us if you stop now." I would be willing to make that trade for the hassle having her train with my boys was proving to be.

"Please don't send me away!" The vehemence of her plea took me aback. "The duke is displeased with my actions that day. He told me that the only way I will be allowed to serve in spring is by completing this training. If you send me away, I will be forced to remain here when you march." She paused. For a moment I was afraid I would be stuck with a weeping child. She composed herself and continued. "I cannot endure several months alone with my mother."

Pieces shifted in my head. I had not known why the girl had wanted to come to war. I had assumed it was out of some misplaced desire for glory or to be close to her husband. I had not guessed that she was running away from someone. Makala was not close to his mother, I knew her to be a hard woman to love. The entire court knew the Dutchess Cybeline's dislike for her newly adopted daughter. Even still, going to war to avoid a mother-in-law seemed extreme.  "I am sure the Tower would be willing to send you to another for the days of the engagement, your grace."

The veiled head shook in disagreement. "My husband suggested the same, but my mother will not allow it. In his absense, she kept me from the Tower completely, she will do so again in spring. Please let me stay."

I was surprised that she feared my boys less than she feared her mother, but I did not question her judgement further. Mother-in-laws are not pleasant relations in general. I could not imagine that forcing one to adopt their son's wives as their own for political expediency would improve the situation. Whatever her personal conflict with the Duchess, she would have to manuever that territory herself. "You are welcome to endure my boys' beatings for as long as you wish, your grace."

"Thank you, commander."

I stood to end the meeting. "If there is nothing else, I will call for a chair to see you back to the Tower." The White Tower was conveniently placed near the army base, both a mile from the Duke's castle. There were however several flights of stairs between her current position and the infirmiry. I appreciated her discretion for climbing them herself for the sake of our privacy. I saw no need to punish her further by asking her to traverse them alone on her return. I cursed Makala for softening my feelings towards this girl.

"Actually, commander, there is one more thing," she corrected, slowly rising with the aid of her crutches. 

"Yes, your grace," I asked, dreading what new confession I would have to hear.

"Will you teach me to play chess?"

I blinked in surprise. "Your grace?"

She spoke quickly to allay any jealous thoughts she imagined I entertained. "I do not ask because I wish to come between you and my husband. I have never learned the game. I would like to."

It was possible that the veil hid a scheming face that spoke with an innocent voice. If so, the girl would have showed a greater degree of duplicity than she had ever done before. The simplicity of her request disarmed me. It seemed she actually wanted to be friends with her husband's lover. "You could ask your husband to teach you, your grace."

I heard her smile as she spoke. "I did, commander. He tells me you are the better player."

I gave up. I was being openly conspired against. If I was to teach my boys that there is no shame in being beaten by a girl, I told myself, I should learn the same lesson. "I will see you in two hours in the palace, if it please you, your grace?"

"I would prefer the White Tower, commander. One of the advantages of my morning beatings is that I now have a convenient excuse to avoid the castle."

"As you wish, your grace." As I left to find transportation for the dutchess, I shook my head in disbelief. Whether the disbelief was directed more strongly at 
her, Makala, or myself, I still cannot say for certain.

I met her as planned in the garden of the White Tower in the cool winter evening sunlight. She still looked exhausted, but leaned on her crutches less. I wondered if she had spent the intervening time under a healer's care. We had been playing for perhaps half an hour when Makala appeared behind her shoulder. Nisrita, absorbed by the board, did not notice. I, trained by years of experience and fear, did little to acknowledge his presence. She jumped when he moved her rook from over her shoulder. 

"You should be more aware of your surroundings, Nisrita," he chided quietly.

"I am not in the field now," she snapped at him.

He sighed. "Assassins do not kill Gifted only in the field. May I join you?"

The young duchess looked at me. I nodded my consent. Makala found a seat for himself, and we put aside the game. We talked. Or at least, they talked. Five years of practise had drilled into me the dangers of talking too freely with the duke in public. It was not that Makala and his wife did not try to draw me into their conversation. So many years of habit, cemented by fear was hard to break. I saw what Makala was doing, certainly. I understood what the world saw. The duke was chaperoning his wife's time with his father's commander. It was a completely natural and exteremly generous gesture for a husband to make. Nisrita was a lonely child. She wanted a kind word, and to learn chess. Makala was tired of the secrecy imposed upon us. It did not matter who chaperoned whom, only that the world saw what it wanted to see. In his wild impossible dream, all three of us would be safer if we worked as allies. He had moved the pieces into place. I could see check from here.

How do I describe the next three months of my life? Makala's dream briefly and brilliantly flared into a reality. Under Nisrita's supervision, we met in public. We spoke, never intimately, but with definite openness and warmth. Our meeting place varied, depending on the weather or the state of Nisrita's health. When my boys left Nisrita whole, which happened more and more as time wore on, and the skies stayed clear, we three ventured past the castle walls to walk among the blooming grasses. Nisrita walked between us, placed there as an obvious insistent barrier daring anyone to believe that there was anything wrong with this painfully domestic scene. Otherwise, I taught the duchess chess, or let her pepper Makala and I with questions as we played. Nisrita and I spent enough time alone together to plant the idea that the young dutchess had a fondness for an older man. No one, at least in my hearing, suggested that we had anything untoward beween us. For one impossibly daring day, Makala, by which I mean the young duchess, invited me to watch the day of games celebrating Duke Ergino's birthday from her box. I sat before the entire court, enjoying myself with the duke. I had never imagined that the sanctity of marriage and the blessings of a colluding wife could bring such freedom.

For her part, Nisrita did not seem to mind her role in this sharade. Makala continued in his filial affections towards her. I learned to release my jealousy. It was not so difficult when the rewards were so great. I do not know how she kept from being jealous of me. Perhaps Makala was right in his evaluation of her. Perhaps all she wanted from him was this filial affection. Perhaps she had honestly come to terms with the fact that marriage would change nothing in their relationship. I did not understand how she could want so little of Makala, but I was willing to take advantage of her lack of desire. When he left her company, I watched them link forefingers. It was a childish gesture of affection among many siblings. Few carry it into adulthood. Makala and his siblings did. His sisters linked fingers openly with each other and their eldest brother until each married. I have seen Duke Lukos do so discretely when he came to his brother for advice. Duke Ergino's children were close. Perhaps Nisrita considered herself lucky to be one of them.

In the meanwhile, Nisrita learned chess more slowly than she learned to defend herself. When she returned to train with me, she came prepared. The wore the black boiled leather of the riders. Their armor itself is indimidating to look upon. It is fitted with a miriad of pockets and slits in which they hide their throwing stars, daggers, caltrips, and other varried sharp acoutraments of their fighting style. I could count a dozen small sharp objects on her person. I am sure some were hidden from my view. To add to the effect, she had strapped a talisman to her belt. The distinctive oval of crystal and silver could not be mistaken for anything else. It glittered against the dull black of her armor, a shimmering warning of pain to come. I learned later that the talisman she displayed was not hers. Infact, the glass had cracked and it could not be used as a focus. A healer's talisman is most efficient close to the body. Most wear it over a layer of clothing rather than touching skin because practising the gift heats the crystal and metal object. Worn outside leather armor, its efficiency would be reduced enough to inconvenience most gifted. Her own talisman was safely tucked away where she could use it easily. She appeared before my boys as a black rider and a white torturer, a member of Cortan's elite forces, and a feared bestoyer of unidentifyable pain. In reality, she was not fully trained as either. Whether my boys knew that or not, her attire cast doubt on any preconcieved notions they had about her weakness. She had decided to play an intimidation game. Against an untrained group of adolescent boys, it worked. The boys shuffled and whispered when they saw her. As she lined up with the rest of the group, they moved to position themselves elsewhere.

I reminded the group of the two rules of a free spar day. If a boy stops fighting, he takes himself out of the match. Fights cannot be more than two against one. I added two more rules for the day, ones that normally go unsaid. If a boy yeilds, the fight is over and he takes himself out of the match. The aim is to disable and wound your opponents, not kill. The game is meant to mimic an actual battle field, as much as I could allow. It was also meant to let the boys have a bit of fun. It gave some of the smaller boys a chance to plan tactics and take on some of the larger, or more skilled trainees. It gave the older boys a chance to best each other in their favorite way. It generally served to release the tensions that invariably built up among those living together in any military setting. I stood back and let the match begin. I had my misgivings. There were many things that could go wrong in this excersize, but it was necessary.

One of the stronger, older boys picked Nisrita first. He fought with an axe and sheild, she with two daggers. He rushed her before she could get her bearings. She dodged and caused him to stumble. She had learned something in the last week. As the boy went past her, he sprouted a dagger between his shoulders. Then he stumbled again, crying out in pain. His right hand spasmed, and he dropped his axe. Nisrita kicked away the weapon, twisted his hurting arm behind his back, pointed the other dagger at his side and forced him to yeild. She extracted her dagger, looked around her and found another, smaller, boy to fight.

Thrown weapons against bladed is a fair fight. It is not one my boys have trained for, but it is not an unreasonable one for them to face. Once gelded, they would be formally trained for this scenario. It would not hurt them to see it now. Handheld weapons against an invisible ranged attack of pain seemed unfair. But that was the point. If they would not respect her as one of their own, they would respect her for her abilities, or at least fear them. I would let her fight them any way she chose until I had discipline in my group again.

Nisrita held her own until only the stronger third of my boys remained. When she could pick the fights, she chose the smaller boys. Her gift seemed to have a range of five feet. Since I had no boys who preferred to fight with spears, she could accurately hurt anyone before they reached her. At five feet, she was completely accurate with a dagger or throwing star. She conserved her gift well. She inflicted relatively little pain on her opponents. She used it as a tool of surprise and fear. By the time the boy had yeilded, he was more concerned with the flesh wound he had received than the burst of pain she had used to position him where she wanted him. 

Eventually, she came across an opponent she could not frighten so easily. He stumbled five feet away from her, as everyone did, but did not stop. He was upon her before she could turn him into a pincushion. His reach with his mace was longer than hers with her daggers. She tried several times to back away, to gain enough distance to throw something at him, but he stayed with her. After three failed attempts to pierce his leather with stars thrown from too close, she stopped fighting. The boy hesitated at this strange reaction long enough for her to focus. He doubled over in pain. When he yielded, Nisrita looked unsteady on her feet.

She heard someone approach behind her, turned, threw a dagger at his unarmoured thighs and fled before it hit its mark. She had not realized that her victim had an ally. She ran straight into the second boy's sword. If this had been an actual battle, or she not the dutchess, she would have been dead. There was nothing that prevented her new opponent from slicing through her torso. Makala was right. Nisrita needed to be more aware of her surroundings. Mercifully, the boy took advantage of the lack of resistance to swing with the flat of his blade. The blow took her off her feet and sent her scrambling for balance. He lept upon her, knocking her to the ground. Nirita struggled for long enough for her other assailant to limp up and tear the talisman from her belt. She went limp. I saw her signal her defeat. I got ready to step in to rescue her from the two boys, but found I did not need to. Her wounded opponent responded to her release by angrily shoving his partner into the dirt before limping off the field himself.

I was surprised and relieved. Nisrita walked off the field. She was battered and bruised, but no more than anyone else. My boys, or at least two of them, had learned self restraint. She had fought well for the setting she had prepared for. She still had a long way to go to defend herself on the field, but she was learning.

The boys gave her more respect on the field from then on. They let her yeild, for one. They fought her harder than they fought each other, but there was nothing I could do for that. In the end, it was probably better that way. The problem moved off the training grounds. Boys cornered her as she left the arena almost every day. Pulling them off her or chasing them away was more responcibility than I wanted to shoulder. Train her as I might, I could not change her essential femaleness. My boys were still uncut. I told her to bring her talisman to the field, and warned her that I would remove her from my training if she used it during the excersizes. I prayed it would be enough. My prayers were mostly answered. The incidents fell from almost daily to one or two a month. I had taught my boys to fear her. Over those three months, not a single boy decided that she could be a useful ally in their training games, inspite of the clear advantages of an alliance with a gifted fighter. They chose to fear and hate her instead. She would never be one of them.

The Duke and Duchess of Firvona came to visit, along with their daughter Duchess Rochilda, Lukos' bethrothed, and a full miltary retinue. Firvona was a thriving and powerful border duchy until a decade ago, when Cortan finally erradicated the Hynsa people that plagued our western border, as well as Firvona's southwestern. They were mounted fighters, and Cortan's cavalry and mounted archers know no equal. As per tradition, this Duke Ergino claimed the piece of land his armies had cleared. As long as he has held it, Firvona has been an interior duchy, left to grow fat on the blood and gold spent by the border duchies. Interior duchies are safe from raids and all out military assaults, their gifted women are not kidnapped, their farmlands are not burned. They benefit from trade with whatever resources the exterior duchies find in our incessant march outward. Every Duke of a border dutchy dreams of wedding their daughters, especially their gifted daughters, to families in the interior. There is no doubt that life is safer and for those on the interior. If an interior duke find himself with a number of ambitious and expendable sons, he may fund an expedition into the borderlands to add new territories for the glories of Marsea. Duke Ergino's grandfather hailed from the same province as my family, just north of Deyalorn. A powerful and wealthy baron, he sent three of his six sons south to carve out a piece of land for themselves. Duke Saro, Makala's grandfather, and his brother Griswold established Cortan as the lands supporting a small fort on the border with Domalin about forty years ago. They lost their brother Pierd to the conquest. Since then, Cortan has grown and propered under Duke Saro's and Duke Ergino's military wisdom. The duchy of Domalin no longer exists, it found itself beaten and abused by its giftless neighbors, then absorbed into Firvona, which had greater military strength at the time. Duke Griswold disappeared in a plume of mystery a few years later. Life in the border duchies is neither easy nor secure. 

I did not understand why Firvona begrudged Cortan its border duchy status, but politics and bickering between duchies is not a matter for me to judge. After all, I have chosen the riskier life of the sword to the more mundane comfort's offered by my father's lands. Perhaps the Duke of Firvona is as crazed as I, willing to give up comfort for a chance to court death in the name of glory instead of the shame of suicide. Whatever the reason for the tensions between Cortan and Firvona, it was impossible to deny them. I thought at the time that the large military retinue they brought with them was an attempt to intimidate us. Now I have my doubts, but no facts with which to support them. I had to reprimand several of my officers for mocking the Firvonese honor guard for carrying spears with beautifully carved shafts. They poor sod they killed would not care if the splinter of wood sticking out of his chest had been lovingly painted and carved. Of course, these were primarily ornamental weapons, the roads in the interior of Marsea were relatively safe for travel. Men from both camps were jumpy and hostile. Insults of various sorts flew in all directions. I was not the only commander to have to reprimand men. 

The Duke's family was nervous. Makala's mind was elsewhere, for reasons I will shortly reveal. Lukos was as jittery as the bride to be. Duke Ergino was a tense host, burdened with the upcoming wedding and desirous of more support for the campaign in a month's time. Dutchess Cybeline like her intended daughter-in-law. The Dutchess Rochilda was plump in a way that foreshadowed a large brood, well mannered, and pleasant to look at, if a bit tall. She was not gifted, but her sister was. There could be hope for Lukos's line. Dutchess Cybeline however made very public her distaste of Nisrita's absence from the proceedings. If Nisrita had been present, the Dutchess would made her disdain for her adopted daughter public instead. It is not the Dutchess' opinion that casts doubt into my mind now. It is Nisrita's absence. 

The morning Firvona's head family arrived, I had her sit out the more violent excersizes. She had spent the last two weeks sporting a full set of minor injuries. The Tower could heal her bruises and make her presentable for the afternoon's reception. Cortan could not risk her injured that day. 

Leaving the training grounds, I gather that she was ambushed. I do not know the details of what happened to her, she refuses to discuss it with anyone. Two of the men in charge of the armory found her unconcious and bleeding in a ditch in the open field beyond. Makala convinced her to identify her assailants, two of my boys and an infantry man, but she would say nothing else of that morning. I feared the worst, but had no basis for my fears. I punished the assailants as harshly as the military code of Cortan would allow, then asked Makala to end this game of his, or at least to let her train in private. He agreed, but she did not. It took Nisrita two days to recover from her wounds and another four to recover the courage to see anyone other than her husband and her companions in the Tower. She joined her fellow trainnees a week after the incident, two mornings after Firvona's head family had departed. I continued my protests until Makala shut me up. There were just over three weeks left before we marched. If Nisrita had finally learned the importance of being aware of her surroundings, then she needed practise to master that skill. Outnumbered, I acquiesced.

At the time, I gave no thought to the fact that Nisrita's injuries had sequestered her completely from our visitors from Firvona. I was more concerned with Firvona's visit and Nisrita's physical and mental well being. But now, with the gift of hindsight, and the curse of too much time on my hands, I begin to wonder. Had this been one of Makala's schemes as well? I cannot imagine him playing with the lives of those he loves. Unless of course, he knew what would happen to him. That is possible, though if he did, he never let me know. He would not have. His first instinct has always been to protect those close to him. Which is why I cannot see him planning for Nisrita's body to be mangled and found in a ditch. Did he fear that Nisrita would share his fate if he did not intervene so drastically? Had he planned a different course of action, and had his scheme go horrifically awry? I cannot tell. Is there anything I would not give to be able to ask him that and so many other questions. But this is the baseless idle speculation of a haunted man. It does not serve me, or Cortan, or Nisrita, where ever she is. 

*** 

As the winter rains slowed, we gathered our troops outside the castle. The bulk of our army would march directly to the Pensid Mountains. Makala, with the rest of the black riders, along wtih one cavalry brigade and several scout units would cut directly south to the Velta River valley, and follow it east to Castle Turina, where we would meet in six weeks' time. Cortan was mostly dry grasslands. The Duke had his eye on the fertile Velta valley. If we held The Pensid Mountains, we held the mouth of the Velta. His eye was already on the next conquest. 

By the time we were ready to march, it did not bother me that Makala's departing words were "Take care of them, Romino. I want to see them both in one piece at Turina." We stood below Duke Ergino's window with his brother Lukos, who would travel my route with the archers, and the captain of the castle's defenses, taking our respective leaves of each other. We had said our private goodbyes earlier. 

"I will do my utmost, your grace." I replied, and turned from him to my men. Duke Lukos marched with us, beside his archers. He was not under my command. While I had appointed Nisrita to my command, she answered to the Master of the healing corp. The company that guarded them was not under my command. None of this mattered. In Makala's eyes, I was as responcible for their safety as their commanding officers. 

The horn blew a second time, and the riders heading south left. Makala would move more quickly than we would. He would be less well protected, and move through more hostile ground. I would worry when my duties allowed.

It took us three days to cross Cortan's southern border, and enter Niev. It took another three to see our first organized opposition (6). The gift gives Marsea's armies several advantages. Our healers return soldiers to the field as if they had never been injured. A man can dislocate his shoulder once a year without fearing lingering and crippling pain to the joint. An arrow through the thigh does not lead to lameness. It is impossible to completely rid a man of the fear of injury the Preserver planted in us to keep us alive before we found the Gift. We train our soldiers as best we can to master that fear. 

Unlike our opponents, the youngest fighters on our field are eighteen. They have been training since their early adolescence. I started training in my father's house at twelve. Some of the boys I train come into my care at eleven. All ungifted boys in Cortan must train from the ages of sixteen to eighteen, and serve until they are twenty two. When they see their first battle, they may be as inexperienced as the beardless boys holding spears and swords they face, but they are larger, stronger and better trained. At twenty two, any ungelded soldier may choose to make a career of the army. Many of our veteran soldiers are over forty. Our veterans cunning and strong. They have lived by the sword for decades, and we take advantage of their wisdom.

Our men depend on the Tower to make their wounds disappear in a few weeks. It is that knowledge that lest them march into battle unafraid of wounds to their extremities, agile and lightly armoured. The bulk of our infantry move in groups of twenty, bypassing and flanking, rather than clashing swords head on. The squadrons have a fair ammount of autonomy, led by men who have proved their minds as well as their bodies agile and resourceful over decades of service. We do not have enough gifter fighters to place on in every squadron. We reward bravery with their placement. In uneven terrain, our infantry is unmatched. In open plains, it is our cavalry. 

This is not to say that we do not have our weakneses. As willing as our men are to risk limb for their country, they are much less willing to risk their lives. Marsea does not have a navy to speak of. Our men hate fighting near rivers. A man who falls wounded in the water cannot be dragged out to a healer when the fighting has subsided. More than likely, he will drown. Similarly, they do not like marching if they are not well supplied. The gift can do little for exhaustion,  starvation or illness. The most famous revolt in Marsea's military history occurred shorty after a latrine pit leaked into the the the army's drinking water. When the diarreah and vomitting started, the men turned on their officers, killed half the engineering corp, and walked off the field. They could not all be discharged, so Marsea took them back into service. Commanders have learned to keep our men's health near the forefront of our tactics.

Our enemies knew our weaknesses. Niev would not meet us in the rolling grassy borderland. Our first encouter was at a ford of the Sarena River. The bridge had been burnt. The Nievian army awated us, with its lines of arches, on the other side. Moving up river, the next ford lay two days away, below a dam, and Castle Bayan. Positioning our army downriver from a dam only encouraged the Destroyer to take our men through drowing. Moving down river to ford took us in the wrong direction. 

I suppose our enemies would have forded the river, they would have some soldiers willing to gamble their lives for the glory of this victory. We could force our men to do the same, but at the cost of being reviled as a leader. We decided not to waste the the entire force's time, and waited. We sent our mounted troops and half our healers to double back downstream and ford where they could. They would move quickly. We would cross here under their protection. It took two days for them to arrive, greatly reduced in number (8). Nievian pikes awaited them upstream, inflicting heavy casualties, but we expected that. We incur more casualties than our enemies would dare to. It gives us the advantage of intimidations. We appear reckless and unstoppable. But we are not stupid. We travel with replacements for the ones our enemies take down. Many of the horsemen, and horses that survived the encounter at the river would be with us again at Turina in a little over four weeks. 

Nievian archers distracted by a cavalry charge, we forded the river with an acceptably low number of losses. The battle on the other bank was hardly more than a skirmish. It is not uncommon for armies to draw us into a position of incurring one set of heavy losses, then retreating, hoping to wound us faster than we could heal. That battle, the choice was between cavalry and infantry. The cavalry were more expendable. We would be fighting in the mountains soon. 

They left us, predictably, in an undefendable position. We are at our weakest in the days between when a battle ends, a our wounded are gathered up by our supporting forces and carted back home to recover. In this game of whittling down our forces, Niev did not choose a defendable plot of ground they wished to defend. Their plan was to retreat, leave us in the awkward position of defending our wounded, the attack again before our support reached us. We set up a tight camp, dug ditches and waited. We were only a few days out of Cortan. Support would arrive soon. We were weaker, but by no means weak. The second wave came the next morning, and had already been beaten back by the time reinforcements arrived in the evening (9). Two battles in a day had exhausted our healers. If our support had not arrived, we would be little better than our giftless enemies that night.

We took Castle Bayam five days later (14). In theory, if we have surprise on our side, a castle's walls pose little obstacle to the skilled climbers of our men. Caught unawares, a handful of squadrons, supported by archers can penetrate the walls of a modest sized keep long enough to open the doors for an invading army. The problem is always getting an army to a castle's gates undetected. Especially when the enemy knows we approach. I cannot be done. 

I dislike taking castles. It is slow, and ponderous, a complicated process involving seige engines and catapults and tunnelers of warfare that our ungifted neighbors are forced to use in day to day warfare. It is dull and dangerous, and contrary to very nature of Marsea's army. We are lythe and adaptable, cats and lions in armour. We strike, we overtake, we heal our wounds, we move on to the next target. I could kill three men in the time it takes a catapult to wound a wall. 

But we needed a base in Niev. It would take over a week to march to Turina if we could walk through friendly territory. We were already five days out from Cortan. We could not hope to cart our wounded back to Cortan through nearly two week's of enemy terrain. Our supply trains are well protected, but they can be overrun. Our healers needed a defendible base. For that, we needed a castle.

It took over a day for Castle Bayam to fall. The healing corp took our wounded and established themselves in the stronghold, while the rest of the army made camp outside. We would rest for a few days before continuing. We had secured the castle by the afternoon. The slowness and frustration of the last day, and the several days of waiting before that as the war machinery reluctantly rose into existence under our engineer's careful hands, ate at me. As a tactitioner, I work best on the open field. Taking a castle is not a situation under which my men or colleagues consult me. It is not a situation under which I offer my opinion. I sat for days and waited to be told what to do. Frustration and boredom seized me, interrupted only by brief periods of battle scattered over the course of a day. That day, after I had settled into the stronghold, seen to my men's needs, and communicated with the patrols for the area, I settled over maps of Niev I had to plan the next stages in our war. The commanders would convene with General Madriano over the next few days. It was a good opportunity to plan. 

My mind would not focus. Frustration crept upon me again, this time bringing with it anxiety about the fate of the riders crept into my mind. If the map before me were accurate, they would be at the Velta valley by now. The valley was mostly farmland and wood covered hills. The possibilities for ambush, it seemed from the map, were endless. They were our best troops, and well fitted with healers. But they would be fighting near water. They river was Our men, Makala, grew up in the dry plains of Cortan. He did not know how to swim. Even if he had known how to swim, a badly wounded man could not hope to survive long in a strong current. I knew they had planned to travel due south as much as possible, until they entered the valley. They would meet the Velta just before it joined with the Swene, several weeks' travel from its start in the Pensid mountains. It was sure to be wide and strong by then. 

I rolled up my map and left my tent, irritated at myself for my weakness. My men would not respect me if they knew I sat pining after a soldier in different part of this campaign. I was acting like a little girl. I wandered the camp, seeking the company of other officers with too much time on their hands, then changed my mind. There was someone else who would have the same worries as I did. I would not be forgiven for neglecting her.

I found Nisrita walking gingerly towards the building that used to be the castle's temple to whatever gods the Nievian's worshipped. She favoured her right foot. 

"Are you injured, dutchess," I asked, stepping up beside her, offering her an arm. 

She ignored the offer and kept limping forward. "It is only a bruise. I hurt it removing rubble from that curtain wall." She indicated a piece of collapsed wall with her chin. "As long as you do not ask me to run a mile with stones tied to my back, it should recover." She smiled weakly at the comment. 

"You should get that treated." It always seemed wrong to me that healers ever walked with injuries on their body. One would think the gift elliminated that need. 

"You have kept us busy, commander," she snapped. "It will heal on its own."

"Nisrita," I said sharply. I wanted her attention, and I got it. She stopped and looked at me for the first time, but did not meet my gaze. There was something not right about her. There were dark circles under her eyes, her face was drawn and her shoulders sagged. She looked worn. Her brilliant white healer's robes had grown dingy with travel. Specks of brown showed where the apron she wore while working did not cover her garb. This in itself was not surprising. Most of the healers had a similar appearance. The battle of the last two days had exhausted them. They worked as hard as we did in the days following an encounter as we did during it. They worked and slept in shifts, with healers on duty at all times. The agonized cries of the wounded at all hours attested to the fact that on a campaign, gifted healers pushed our bodies to reform as fast as they could without killing their patients. There was something else about her bearing that concerned me. It had been four years since I have had been in charge of green soldiers on the field, but I recognized her attitude. She had seen wounded from three battles now. Even if she had not been in danger herself, seeing what lengths our men go to to destroy an enemy cannot be pleasant. Barely fourteen, she was by far the youngest member on the field. I cursed myself for my cowardice last autumn. No one who could call himself a man would have allowed a child, a girl, to expose herself to this madness.

"Are you busy now?" I continued. "I would like to speak with you." I had gotten her into this mess, I could hear Makala accuse. It was my job to see her through this.

She eyed me suspiciously. "You have barely spoken to me since we left. I have wounded to tend to. Whatever it is that could wait this long can wait a few more days." She left me and resumed her limp towards the battered temple.

There were no wounded in that building, but I did not pry into her affairs. It was true I had not spoken to her during the last two weeks. I had men to see to, she had wounded to tend. The healers and the fighters tend to stay apart outside the infirmiry. Many of the men think it bad luck to spend time with the healing corp except when needed. When I did see her in the aftermath of a battle, she was constantly busy. I saw older more experienced healers wear out before she did. Seeing her work, I understood why she was considered a powerful healer. Her stamina was impressive. 

I could sense Makala's disappointment from miles away. I was failing him. I train boys every day in Cortan. He had entrusted me with the care of one girl the same age, and I could not care for her. What good was I to him? I could have found her during her spare moments, as I had today, or visited her during the dull dreary days of marching. I had not thought to visit her out of habit and tradition. 

Fourteen is a hard age to be traumatized by war. I spoke to the Master of the healing corp about my concerns for her well being and retired to the the company of friends. I had too much time on my hands, and nothing but worry to occupy it. If I let myself continue like this, I would not be able to serve my men.

A few hours later, she found me in a much better mood. The company of my fellows, a few horns of ale, and a warm meal consumed in the protective shadow of a castle had worked wonders on my state of mind. For her part, Nisrita was no longer limping, and looked like she had found some time to sleep. "You wished to see me, commander?" She was, however, still sulking. 

"I wanted to see how you were doing."

"I would like to remove some more rubble from the breached wall." That was not an answer to my question, but I followed. 

As we moved rocks in silence, I wondered if this was a game to repay me for the piles of stones I had her move. Eventually, I ventured "Why are we doing this?"

"I am looking for cadavers, or trapped soldiers. It is better if they are near death though."

I stopped working. We should not have lost men in this pile of rock. Furthermore, the coldness of her voice made me uneasy. "Nisrita, why are you healing Nievian wounded?"

"I'm not. I am studying them."

"And the cadavers?"

"I want to plant flowers in their skulls," she said flatly without turning to look at me. When I balked at her comment she corrected "For study as well, Timmon. How can we possibly learn how to help a body put itself together if we do not know how it is put together in the first place."

I still did not understand. "Why not simply take them out of the burial pit?"

She stopped digging and gave me an exasperated look, as if I were a slow student. "We have. We have a dozen bodies with various types of injuries lined up in the temple. We lack someone with a shattered bone. Cut, yes. Broken, yes. Not shattered. We cannot fully heal a shattered bone. No matter what we do, we leave small shards in the flesh. Sometimes it is absorbed by the injured years after the injury, but not always. It is hugely painful to the patient, and the Tower should be able to do better. We have a Marsean man inside who has shattered his wrist while working the catapult, but it would not be right to toruture him.  Also there are too many bones in the wrist. A femur would be best to start with. If I can find a Nievian man, on the other hand, I think I can learn a lot from his injuries. I would not learn as much from him if here were dead, but it would still be a worthwhile excersize."

My stomach lurched at the coldness with which she spoke of our enemies. No, of everyone in her care. I have known Nisrita to become supercilious when talking about the gift, or completely incomprehensible when trying to explain why and how she was torturing the rat in her cage, but I had not heard her this cold about human life. 

NIsrita sighed. "It is getting dark now. I won't find a body tonight. I'll just have to wait until you take another castle. Good night Timmon, thank you for helping."

She picked her way down the pile of rubble and back inside, leaving me sick and feeling troubled. I asked some gifted fighters about this strange attitude. They admitted that it was extreme. Most healers learn to develop a level of distance between themselves and their patients. It is too difficult to inflict the necessary amount of pain otherwise. Using cadavers as learning tools, as dishonorable as it seemed to me,  was a necessary evil. I was asured that the practise is conducted respectfully, at least while in the hearing of the Masters and those outside the Tower. Jesting about using their skulls as pottery was macabre. At least that is what they admitted to an ungifted outsider. The White Tower makes strange men. The key to Marsea's army, they stay aloof from the rest of the men. Their powers make them hauty and aloof. They do not mix with the men they heal. Wounded, we trust them blindly, not knowing whose hands we put our lives in. The gifted fighters were a different matter. They came into our ranks with low opinions of their simple ungifted companions, but years of working with, fighting with and depending on the rest of us eventually wears the tower's influence off of them, making them friendly, likeable, and the only men one wants to see when bleeding rivers from a bad flesh wound.

Still, the thought that the sight of war, and not her time with the White Healers, had turned Nisrita into this heartless monster unsettled me. What would Makala do to me if I returned his wife battle scarred and heartless?

A week after that (21), the grasslands gave way to the woods that covered the foothills of the Pensid mountains. The road from Bayam to Turina passed through the forests. We would have to break formation and pass through the woods in a narrow line. It left us exposed. We doubled our guard on our healers and supplies and marched into the shadowy depths.

The attack we expected came a day into the forest (22). The Nievian archers had hid themselves in the trees, well out of sight. Their arrows injured a dozen healers and twice as many guards before our archers could shake them out our their trees with their arrows. They came down by the hundreds. We gave chase, but scattered more than we killed. 

I was near the front of the column, the healers near the back. I did not get a chance to see these events as they unfolded. Rumors filtered forward that a few of the younger healers, untempered by years of service had scattered. The men blamed the need to gather them up again like so many lost sheep as the reason our army killed so few Nievian ambushers. Two girls were still at large. We had, however successfully captured several hostages. The Master of the healing corp and General Madriano would extract information from them soon enough.

I went back along the column to check on Nisrita. None of the fighting men knew the names of the missing girls. How could they. It is rare enough for any of the healing corps to come forward into the ranks of fighters. It was nearly impossible that a young girl would willing the enter the ranks of fighting men. 

As I entered the clearing between the fighters and the white robed healers, I found Duke Lukos sitting on a moss covered rock sweating beads of pain. A tall thin man in a white robe with white streaked hair stood over him, calmly examining the duke's right forearm. The duke's left knuckles were white from the force with with he dug his fingers into his thigh. In his mouth he held a leather bit to help with the pain. In contrast, the White healer stood over him as a loving father would over his son. Most healers do not need to touch their patients while healing, but some chose to. The juxtaposition of the extreme agony, and the serene by intense focus in the pairs working in the healer's tent always startles me. If he was knowingly causing me so much pain, he could at least show some remorse. The distant apathy is infuriating. I prefer not to look at my healer's face while he works. The urge to lash out at someone causing me so much pain is too great otherwise.

Duke Lukos gasped, I saw his eyes roll back in his sockets, and then he fainted. His healer, Alerio, caught him, and gently lay him on the ground beside his rocky seat. When he saw me, I indicated that I would stay with the body until he brought the supplied he needed. He returned in a few minutes with a blanket, some bandages, a flask of brandy and a box of white paste. While he worked, Alerio told me that Nisrita was safe, and asleep. She had exhausted herself during the attack. He volunteered the information. Everyone knew why I visited the healing corps daily. The change in Nisrita's behaviour scared me. Not knowing what else to do, I started visiting her regularly. I do not know how I expected her to respond, or how I thought I would help. I went because I felt I should do something. I suppose I hoped that my regular visits would change her, or at least counterbalance the blame Makala would thrust upon my shoulders.

Alerio, told me about Nisrita's performance during the attack. The healers, by and large, had defended and healed each other as best they could. The coldness they had towards their fellow man did not hold within their own ranks. The members of the healing corps were as tightly bound to each other as the members of any brigade. 

When the first arrows came down from the trees, Nisrita had been walking with Ezar, a boy of eighteen, just entering his gift's prime. Engrossed in their conversation, they had fallen to the back of the crowd. Two arrows came towards them, one finding Ezar's shoulder, another catching the soldier behind them in his gullet. At least, this is the boy's story. He and Nisrita turned to help the wounded guard. The man was lucky. If they had run, as so many of the other younger healers had, he could have easily drowned in his own blood. As soon as other healer's had come to their aid, Nisrita took the man's shield and knife and slipped out of their company to find the tree the arrow had come from. 

Everyone saw her drop a Nievian from the tree. By the time he fell, she had two arrows in the sheild she hid behind and her calf bled. She stole the quiver and bow from the now still body and, positioned herself behind the line of fighters forming around the corps. Alerio sounded impressed that one in three arrows brought down a man from the trees. I was certain more hit their target. I had seen her fight. When she had emptied the quiver, she wandered the grounds staunching bleeding. She must have had to hurt her first oportunity very badly to get him out of his perch. She stumbled and fell after only tending to two men.

Alerio was impressed by how well she had handled herself at the first sign of trouble. In his opinion, she had comported herself like the veterans, rather than the other girls her age. I was fuming. I had not trained her in battle so she could break cover and become a target for enemy fire. Makala had wanted her trained to keep her safe. These acts of excessive bravery were commendable in a soldier, not in a dutchess. She had Cortan to think of. She was the only chance at a peaceful life I had when Makala and I returned from this campaign. She had to keep herself safe.

I decided not to visit her at that time, excusing my absence with her need for rest. Returning to my position, the Master healer passed me on his way to the prisoners. A young woman followed him. She was tall and full, her hair loose and forming a black mantle over her bright white robes. She was lean from the days of travel, but not thin. She wore an expression of innocent flirtation on her face. I wondered how she could do it. She looked so young and so sweet. How could she possibly take part in torturing our prisoners. But that was part of the act. After two weeks of travel, it would be more accurate to call the healing corps the grey or mottled healers. No one had a robe that brilliant any more. Her clothes had been set aside for her, to be pulled out specifically for an occaision such as this. I was certain that the open and welcoming expression on her face was as much a habit of hers as wearing her hair unbound was. This was an act. More fool I for being taken in by it. 

I went back to my post disgusted. General Madriano is known for his ability to extract information from people, on and off the battle field. With prisoners, his favourite technique involved positioning himself before the victim, a white torturer beyond the victim's line of vision, and a young female healer within it. The general would question. His ungifted torturer would maim, the gifted would hit him with blasts of pain from an unknown position at unpredictable times, and the girl would beg him to reveal his information so that General Madriano, usually her cruel father or uncle or possibly her husband, would stop tormenting her with the sight of his misery. She would promise to heal his wounds if he spoke, or do whatever else she needed to do to make the prisioner fall for her. The general and the white torturer would inflict pain and fear for a time, then leave. Then the female torture's game would begin. She extracts the required information by whatever means she deems necessary. If she is cruel, she leaves the prisioner with tearful promises of begging the General for mercy. Marsea's army does not let tortured prisoner's live.

The army made camp as best we could, and waited. Early that night, we learned that Nievian soldiers held the hostages in a village we knew to be half a days' march through the trees. The village we had planned to spend tonight in, held another ambush for us. When the general called a councel, the debate among the commander's was between attacking tonight, and waiting until the morning. No one suggested leaving them behind. A gifted healer, especially a female gifted was too precious a commodity. I advocated attacking in the dark. Our men were rested, we knew their current location. We could not know when or how they will be moved by morning. The general agreed. He also sent a sizeable portion of our forces to take the village before our ambush could be readied. Our healers needed time to recover. Getting them quickly into a defendable village would serve better than telling them to heal in the woods.

I had just released my day's tensions into the tender care of my bedroll when the boy sleeping outside my tent announced that I had a healer to see me. I sighed and pulled myself away from the allures of sleep to meet her outside. 

In the shadowy torchlight, she looked shrivelled. She stood with her shoulder's hunched, eyes to the ground, as if he were uncertain she should be here. Her robe was torn and stained with blood by her feet. She leant on a cane. When she moved, I could see that her leg was bandaged. One in four of our healers sported some sort of injury. There probably were not enough whole healers left to tend to the others. I had no idea how a wounded healer healed others. "Good evening dutchess, how may I help you?"

"Stop that Timmon," she snapped at me, more vehemently than her hesitant posture had indicated possible.

Preserver defend me from adolescent tempers. I had thought myself rid of these woes when I was promoted from my position of supervising the barracks that housed Cortan's training youths. I blinked the sleep from my eyes and asked as calmly as I could. "Stop what, dutchess?"

"You only call me dutchess when you are upset at me. I will leave if you do it again."

Leave then, I wanted to say, and let me return to my bed. But I did not. I was upset at her. She was right. It had been some time since had developed the habit of adressing each other by first names. Our titles we reserved for arguments. "Very well, how may I help you, Nisrita?"

The intial hesitation of her bearing returned, and entered her voice. "I was hoping... to talk to you, perhaps? About today?"

I had no one to blame for this. By visiting her daily, I had brought this visit upon my head. Sleep would have to wait. I pulled back the tent flap and followed her in. People would talk, I realized. There was nothing to be done. I could not let her sit in the night air, wounded as she was. She sat on the offered chair for a long time, leaning forward with her elbows on her knees, staring at her hand hanging listlessly from the ends of her arms. Eventually, I interupted her. "What did you wish to say, Nisrita?"

"I killed a man today," she said to her fingers.

"I heard."

I must have spoken more curtly than I intended. She looked up at me sharply from her fingers and asked "Is that why you are mad at me?"

Preserver defend me from observant adolescents. This was not the time for this discussion. I tried again more gently. "No, Nisrita. Is that what you wished to discuss?"

Nisrita's eyes returned to her fingers. "Do you hate me for it?"

It took me a moment to recover from the surprise of the question. I am a soldier. Why would I hate someone for killing in war? My anger towards her solely focused on her recklessness. "Why would I hate you?"

"Because it was so easy, Timmon."

Oh. I have heard men describe their first kill in many different ways. A soldier's manner of describing that rite of passage reveals a lot about his character. Every man I have heard describe it as easy either turned out to be inhumanly cruel or an insufferable braggart. I prayed that the Maker had given me the latter, though Nisrita's posture told me otherwise. I tried to imagine what I would say to Makala to explain this change in his wife. Perhaps I would remind him that the Tower breeds cruel men, and remind him that this would have happened someday, whether or not she marched in a campaign. I could remind him of whatever I wanted. I would have to do better than that if I wanted to be believed. What was I going to do?

"I hate healing, Timmon."

I left aside my self pity to pay attention to this strange girl in my tent. She had simply stopped making sense. The Tower was her life. She loved nothing more. At a loss for words, I asked "Why?"

"It is too cruel." She looked up to see the bafflement on my face and hurried to explain. "You wouldn't understand, Timmon. You aren't cursed with this gift. You do not know how lucky you are. I spend all day causing men to scream in agony. The men I see came to me because they were wounded trying to defend me, or everything I love in Cortan. Do you know how hard it is to hurt the people who are supposed to be your allies? You all hate us and fear us for what we do to you. Don't deny it. I see how the soldiers shuffle and move away from us when we camp. I hate healing.  It exhausting to practise, and torturous to look at the face of the man I am supposed to help. It isn't fair. Killing was so much easier. I shoot an arrow, he falls from the tree, he bleeds to death. It shouldn't be like that. If the Maker had created this world properly, he would have made taking a life much harder than saving it."

She was crying now, and I was baffled. She had not told me anything new, but what she said did not make sense. Healing hurt. The ungifted feared the White Healers for their gift. Marriage between gifted and non-gifted was a necessity for Cortan's survival. Most women left the tower when they married. When a wife did not, she socialized little outside the tower. The gifted and the ungifted do not mix much. At the moment of Nisrita's confession, I had no idea how a husband manages a woman still surrounded by the Tower's scorn for the ungifted. I still do not, but I am starting to understand that perhaps what we view as cruelty in the Tower's healers is simply an aloofness they put on to allow themselves to perform their cruicial role in society. Watching Nisrita I have realized that soldier's have the luxury of knowing that their enemies are different. They speak a different language, worship weaker gods, do not give birth to gifted children. They are inferior to us. It makes it easier to maim and kill. A gifted healer hurts their own. The cries of agony from a wounded soldier in the healing tent can be heard on the opposite side, often above the hubbub of camp life. If the healers allowed themselves to be friends with the ungifted soldiers, they would be forced to cause their friends such pain. 

I did not know what to say. I found a jug of strong wine and poured Nisrita a good deal more than I thought she should have. She accepted it and dried her eyes. "Drink that slowly, Nisrita, Or does the gift heal wine headaches as well?"

She gave me a look that only girls of a certain age can give. It clearly stated that I was a complete idiot. "Of course not, Timmon. That's not an injury."

I grinned. If she could claim her childish superiority again, she would recover from this shock. Before that day, I had never heard Nisrita speak of her gift in a manner that did not indicate that the ungifted were lacking somehow. "Of course not," I replied.

We drank in silence. I wondered what I would have done if one of the men in my squadron had come to me in such a state when I was still a sergeant. I quickly abandoned that attempt at finding a solution. Those responces would not help Nisrita. I could send her back on the next supply train. There was no point in suggesting it to her. She would not want to leave. That did not mean it was a bad option. However, I was not in a position to decide. I emptied my cup and asked "Have you spoken to the Master healer about this?" Nisrita shook her head. "I think you should."

"In the morning," she mumbled into her cup.

"Probably best. Not all men here look forward to surprise visits from women in the middle of the night." 

Nisrita looked at me quizzically, then at my rumpled bedroll, then stammered her apologies as a chaos of confusion and embarrasment passed across her face. I laughed. She joined in. When she finished, I said. "You need your sleep."

She gave voice to the rest of that thought. "Otherwise I won't be able to serve Mersea's army tomorrow." She handed me the remains of her drink. "May I come visit you tomorrow evening?"

I understood her dersire for company after a day of healing. However, my men would not take kindly to frequent visits from a white healer, and I did not like the idea of a young girl in their company. "I will find you when I am free."

I walked her back to the women's tent of the healing corp, she leaning heavily on my arm. Against all reason, it still surprised me that a healer should ever spend as much time wounded as Nisrita seemed to. 

"Good night, Timmon. Thank you." she said at her tent.

"Good night, Nisrita." As I watched her limp towards the entrance, I rethought my anger from earlier in the day. "I spoke to Aliero this afternoon. He said you comported yourself with courage and maturity." She paused and turned towards me when I spoke, but in the darkenss, I could not see if she smiled. 

I visited her, as promised, the next day(23). Our army had taken the village of Gorim during the night. We arrived in the late morning to a collection of 30 or so houses surrounded by a well fortified wall. Our healers poured themselves within and got to work. When I found time to enter the village, Nisrita lay with another girl on a patch of grass in the late evening sun. Her leg was still bandaged.

They sat up as I approached. My throat clenched when I saw Nisrita's companion. She had been yesterday's female torturer.

"Timmon!" Nisrita cried happily as I approached. She had not seen my anxiety or recognition. "You had time to come. I was just telling Carlotta about you." She indicated her friend, and introduced us.

"I am please to meet you, Commander Romino." There was nothing flirtatious about the girl's manner today. She wore her travel stained robes, had her hair up in a stern chignon, looked as tierd and worn as all the healers who were not in the traveling infirmiry. She spoke with the reserved formality I expect of a gifted healer talking to an ungifted soldier.

Nisrita lay down her back in the fading sunlight. Carlotta remained standing. "I heard the two kidnapped women have returned. Are they well?" I ventured.

"They have and they are, Commander Romino. Thank you for asking." Carlotta responded in that same reserved tone.

"They were stupid for have gotten themselves kidnapped. Their captors gave them a good scare, that is all. I don't understand what all this commotion is about." Nisrita said.

Before I could figure out how to tell her that things could have gone very differently, Carlotta corrected her gently. "Yes you do, Nisrita. Don't be jealous. Everyone is glad to have them back, and so are you."

Carlotta's words had a calming effect on the girl I had not seen from anyone other than Makala. She accepted the criticism gracefully. Instead of  increasing her pique, it calmed her down. I wondered what magic Makala and Carlotta weilded that the rest of the world did not. Nisrita rolled over to face me. "I spoke to Duke Lukos this morning. He says there has been no news from the black riders."

"There is nothing alarming in that. We do not expect any news." I replied. "You will be with your husband in less than three weeks." Nisrita responded by plucking at a piece of grass and nervously tearing it to bis in her hands. I understood the feeling. We are both men of arms. We often find ourselves engaged in different campaigns, fighting in different and distant fronts. Years of practise does not make the weeks pass more quickly or the anxiety easier to bear. "Be brave, Nisrita. You will see him soon enough." 

I left it at that. Nisrita did not look at me, but vented her frustration on another blade of grass. We fell into silence. Carlotta who had spent the exchange looking sympathetically at her friend broke it with stories about Nisrita's doings in Cortan's White Tower. We talked until the sky grew dark and the stewards called dinner. Nisrita showed none of her moral uncertainty from last night. After Carlotta had drawn her out, she laughed at my quips and told stories of the Tower or of training with the Black Riders. She ribbed me for the humiliations I made her suffer while in my training. Our conversation was easy, as it had been in Cortan's court. Carlotta was reserved but pleasant. She was three years older than Nisrita. She clearly cared for the girl. Nisrita returned her care with unbridled adoration, similar to that she showed Makala. The time passed pleasantly, echoing softly of other evenings spent in Nisrita's company. 

When the dinner call came, she asked me to eat with the healers. I obliged. People would talk, I knew. I justified that Makala would not be foolish enough to be jealous, and the rumors would only support Makala's need to chaperone his wife's time with me when we returned home.

The healers were courteous and reserved, as I expected. Their conversation ranged from the day to day gossip of people living in close company, to clinical discussions of how to treat patients in their care. Some seemed happy to have a healthy military officer in their midst to question about future movements. I answered questions as best as I could, and listened to their talk. Amongst themselves, the conversation did not seem so different than what could be heard around an officers' table. This surprised me. For one, nearly one in three healers in the corp are women. I could not imagine any other setting where a dinner conversation involving that many women would sound like an officer's table. For another, though I was starting to see the flaws in this logic, until I experienced it, I had a very hard time seeing the Tower's healer's as being anything but aloof and unfeeling.

It took us two and a half weeks to reach Turina. In those seventeen days, we saw one more feint, intended to injure as many of our archers as possible, and two all out battles. We took the field every time, but the intensity of the encounters increased as we reached Turina. The terrain became rockier. The road to Turina passed through a narrow valley guarded by a steep and rocky slope. A planned avalanche divided our army in two, leaving the rear, composed mostly of  archers healers and supply wagons, exposed to attack. When the bulk of our forces regrouped behind the rockfall, Niev's army retreated into the hills and away. It took us two days to clear the rockfall enough to pass our carts through. Just the rocks removed fifty soldier's from our ranks, either through death or injuries serious enough to ensure they would not rejoin this campaign.

The two battles occurred in and near mining towns. Not as well fortified as a castle, they were both easy to get into. Once in, however, they were harder to keep. Mining shafts can hide an endless stream of soldiers. It is difficult to plan a battle if one does not know the strength of the enemy's forces. We limped through the first battle, fighting awkwardly and ineffiently spending our resources, but we took the town. We left no one alive to spread news of our victory. 

When we approached the second town, I suggested we march past it. I laid my plan before General Madriano. When he accepted, we split our forces. I took a large number of our men into the hills. Our scouts could not tell us where the entrances to the mines lay, so we hid and waited. Three days after the bulk of the army marched past the gates of the town, a heavily armed and exceptionally long wounded train returned past the towns gates. The Nievian commanders took the bait. We watched a long line of men leave a hidden cave and sacrifice themselves to the carts full of healthy fighters. We set fire to the mouth of the cave, and picked off the men who tried to escape via other egresses. The town fell quickly after that.

Every evening during those two and a half weeks, I visited Nisrita. As the time wore on, the sense of guilt and responcibility I felt for her changed into an odd sort of enjoyment of the child's company. When battle and healing had not upset her, she was much as she had been in court. I hope our visits were a comfort for both of us. They were for me. 

The day after the avalanche, Nisrita took me by the hand to where a grey haired female healer worked on a man with a crushed leg. She pointed to various points of the mangled limb, explaining to me what the healing proceedure did to the flesh. I did not understand half of what she said. It frustrated her that I could not see the difference between bruises caused by flesh healing properly, and those caused by the flesh resisting, due to infection or marrow poisoning. She tried to explain to me the difficulties in healing a crushed limb. I understood that enough of the bone needs to be brought together for the body to join the seams, that tiny slivers of bone left in the flesh caused pain, that a badly mangled limb like the one before us would have to be amputated, fever would claim the patient before the healers could cure him. That was not enough to satisfy her. Nisrita used words such as beauty and intriguing while standing before a man about to lose his leg. When I did not share in her fascination, she dragged me to an amputated arm, dug with her knife until she found the fragments of elbow she sought and held the bloody shards for me to examine. She pointed at the bone with her blade, explained the different types of breaks, spoke about the different layers of bone and the different ways they responded to the gift. She as well have been pointing to a blind man. All I saw was that a good man had suffered a devastating loss. When I told her so, she threw down the pieces of flesh, and stomped off to join her own. I must admit, she scared me at that moment. Men, seasoned warriors, are impassive about the bodies we create. It does not suit a young woman to dig through a detached arm with such irreverence. The army she serves may be populated primarily by ungifted soldiers, but we are her countrymen. The man who gave his arm for her deserved more respect. Marsea cannot live without its White Towers. Her army cannot be successful without their support. The more I learned about Nisrita, the more I associated with her friends, the more I feared that we were supported by monsters. 

As the wounded returned to service, or were carried back to Bayam, and her duties as a healer lessened, Nisrita's conversation returned to normal. She told me about her teachers in the Tower, and her ambition to become a scholar, not a practitioner. Both scholars and practitioners healed, but to hear Nisrita describe the two, without the former to find new methods of healing, the latter would have nothing to do. She chided Carlotta for her desire to practise, begging her and wheedling her not to waste her skill. Carlotta bore the onslaught gracefully, as any elder sister might, then reminded Nisrita that her line of work during her gift's prime was between her and her tutor. Nisrita pouted, declairing that practising was for old age, when the gift weakened. Carlotta pointed out that she wanted to raise a family when her gift weakend. Nisrita scoffed at the idea, and trounced off, leaving Carlotta laughing musically in the evening sun. 

On another day, Nisrita introduced me to Ezaro, a gaunt young man, with fair skin and a pimpled face that had ambition written on every curve. He was Carlotta's age, a few years older than Nisrita. As I understand the gift, men enter their prime later than women do. Therefore, unlike Carlotta, he had not yet entered the prime of his power. His manner suggested that he was chomping at the bit to start. He was the youngest male among the healers. Nisrita held this impatience in disdain, calling it bravado. She teased him about his wild ideas. He ridiculed her for her single minded obsession with the skeleton. Part way through this slinging of insults, they would fall upon a topic that served a nuetral territory, and fall into a discussion on the merits of one Master's teachings, or another's, completely forgetting about my presence, or that of their friends. As no one else in the healer's corps found this unusual, I held my tongue. There was a passion in their discussions that would have reminded me of rivals, had Nisrita been a man. If she had been a woman not raised by the Tower, I would have feared for Makala in his marriage. But the Tower breeds a strange sort of creature. Their men and women speak in a way that does not happen elsewhere. 

I am not complaining of her conduct with Ezaro. On the contrary, as queer as it appears to my eyes, there was something clearly childish about it. On the days when her duties were easy, Nisrita returned to her natural state as a young girl. She spun dreams for her future, gossiped with and tormented her friends, fought with them and stormed off, made peace and wheedled stories from them. She sat in the sun and lay on soft grass when she could find it. She did not speak of punctured lungs or excavate livers out of dead men's bodies to study.

One evening, she found me as I left a week's salary behind at a game of dice. She must have been watching me, not wanting to mingle with the company of officers I keep, or not wishing to impose herself on my activites. A cloud of annoyance hung over her shoulders. "Timmon, if you have a moment, I would like your help on a matter." Her tone was not urgent, only insistent. Given two decades of honing, it could nag a man to his grave. At fourteen, it was hard not to laugh. I turned my steps from the direction of my tent to the encampment of healers. 

There I joined a group of ten or so arguing white robes, all in their late teens or early twenties. Nisrita sat down with an air of boredom at the edge of the  discussion, and motioned for me to sit. I heard Ezaro furiously try to defend some position. From his tone, I gathered that he knew he would lose soon. "Are you questioning what I saw, Carlotta. I tell you, it happened before my very eyes."

Carlotta gave him a sardonic smile. "I question nothing but your sanity, Ezaro. You spend too much time hunting for mushrooms and dreaming of things that should not be dreamt." A group of girls beside Carlotta tittered at this comment. A few young men frowned at her severely. I wondered why I had been called to witness this internal argument among the healers.

"Ask Master Alerio, if you do not believe me." Ezaro's eyes flit from face for face, pleading for ground. Nisrita avoided meeting his gaze, managing to look intensely bored. 

"What exactly do you want us to ask him Ezaro?" one of the older women in the group asked. "Whether or not you saw him bring a frog back to life? It cannot be done by conventional means. To suggest that it can would get you laughed out of the Tower. To suggest that Master Alerio used unconventional means would..."

Carlotta and several of the boys hushed the speaker sharply. The conversation dipped into an awkward silence for a moment. An untied tent flap fluttered in the wind, and I wondered what had been said. 

Ezaro gathered his courage and continued uncertainly. "No, Sophia. That is where you all are wrong. The frog was not dead. It did not come back to life. It never died in the first place."

"And this is why I question your sanity." Carlotta retorted. "You claim he cut out its heart. Every ten year old can tell you that if you destroy the heart, or the lower part of the brain, or both lungs, death is imminent and unstopable." Most of her female supporters laughed at the enormity of the mistake Ezaro had purportedly made. The woman name Sophia, however, did not. She studied him, or was lost in her own thoughts. It seemed Ezaro had planted doubt in one mind. 

"There may be some truth to what he says", a plump short youth said softly. Carlotta turned on him sharply, and Nisrita rolled her eyes. "The lower animals can endure more pain than humans. Maybe they can live without a heart. Perhaps they are not lower, as we think, but greater than us. Perhaps we should aspire to be like them."

Half the group guffawed. The other half fell into a shocked silence. I joined the latter. The Maker had made man the highest of all creatures. The Preserver helped us find the gift as protection against the Destroyers voracious apetite. This was the natural order of things. To speak otherwise was heresy. I wondered at the audacity of the Tower to question even the gods' given order. Carlotta looked as if she would burst in her fury.

"Enough." Nisrita stood beside me. "We have had this argument every day for the past week. We have a guest. I am sure we can find more interesting matters to discuss in his company?"

The crowd hushed. Laughter and tension ebbed from their faces, replaced by the calm reserve adopted by healers when faced with an outsider. Eyes turned to me as I wondered at the speed of the transformation. Every once in a while, one encounters a boy whose natural strength or skill at arms is so great that he not only earns the right to compete with boys and men older than him, but also gains their attention when he speaks. Wealth and nobility helps, but is not sufficient. I suddenly realized just how powerful Nisrita's gift must be to draw the attention of healers more than half again her age.

"Commander Romino, is it not?" The woman named Sophia addressed me. "Forgive our rudeness. Would you grant us the favour of a story to distract us from this tawdry argument?"

She spoke with warmth, and a hint of the elegance of court. Now that she faced me directly, it seemed that I ought to recognize her, though I could not place her. She was Makala's age, and married I saw. She must be the wife of someone I knew. Her hair came down to a sharp widow's peak. Across a wide forehead, she had plucked the hairs of her brow to create two cleanly separated lines over her eyes. Her fingers were long and her nails painstakingly shaped. Somehow she found time to care for her appearance in the midst of a campaing. I wondered why, and hoped I was not making a fool of myself before a nobleman's wife. "I would be happy to. But you have the advantage on me, I am afraid, my lady."

Someone laughed at my formal adress. The woman named Sophia smiled. "I hold no title, commander. My husband is Commander Lazaro." She went around the group and introduced everyone I did not know.

I felt like a pig's heel. I had served under Commander Lazaro when I had first come to Cortan. In my defense, he was not married when I left his command. He was not an easy man to work with, I recalled. He felt the need to know every detail of every one of his officers' doing, both on duty and off. I spent those first two years terrified of discovery. I wonder what he was like as a husband. He was a good two decades older than the woman before me. It was none of my business, I told myself. I started telling the first story that came to mind.

I told of King Aler's voyage with his two brothers, Valir and Dorino, across the great northern sea, how he came down the river Tulsi conquoring the lands on both sides until he came to the stronghold of Deyalorn. I am certain it had a different name then, but that has been lost to history. At Deyalorn, it is said, he discovered the the first stronghold worthy of his skill as a warrior. At Deyalorn, he also discovered the gift. Queen Lucretia held the stronghold at Deyalorn. She was both preistess and queen to the Deyanesi. It is said that men feared to look upon her terrible beauty, that those that did were hypnotized as a bird by a snake.  It is said that she turned those that displeased her to stone. The gift was stronger in the old days. There was no end to the power a gifted could weild.

The Preserver in his wisdom had led Aler to Deyalorn so that he may discover the gift and use it to protect his people. But the triumverate dance a convoluted dance, and the Preserver's mercy could not keep King Aler from the Destroyer's jaws. Aler had all but breached the stronghold's walls, when Lucretia emerged to challenge him to single combat for the fortress. The great king was a proud man, not a wise one. He could not reject a challenge to single combat, especially not one from a woman. They met on the bluffs behind the castle, he with his sword, she with her scythe and whip. Two men from each camp came to witness the battle and guard against wrong doing. The king's younger brother, Valir, and his general Carlo came to watch him fight. King Aler was the larger, the faster, the more experienced warrior. He could not fail to win. But he did not know of the gift. Within the first few moments of the duel, he had wounded her badly. It took only a few moments more for him to shatter her scythe. As the queen reached for her whip, Aler laughed at her. What good would a leather weapon do against his blade. He had shattered her scythe, he would destroy the whip. Then, he said, he would destroy her. That had been his fatal mistake. In the moments it took for for him to utter his disdain, Lucretia turned her whip into a snake that wrapped itself around Aler's neck and bit him deep in a vein and drank his blood. The great king struggled for a time against the serpent, trying to unwrap it from his neck. The more he struggled, the tighter its grip became, until he fell, poisoned and suffocated at Lucretia's feet. 

General Carlo declaired trickery, and swore that he would call his armies in to shatter the walls of Deyalorn, and wipe the Deyanesi from the earth. Lucretia, however, had cast a spell on Valir, the king's brother. In love with the witch queen, he swore that he would not allow his brother's armies to attack her stronghold. He went to the youngest of the three brothers, and gave him Aler's crown, on the condition that he not attack Deyalorn for as long as he remained in the stronghold. In return, Queen Lucretia granted Dorino the right to settle the lands on the other side of the Tulsi River. 

Valir disappeared into Deyalorn for thirty years. Some say that he was imprisioned by the witch queen to ensure that his brother would not attack. Others say the spell that she had cast on him were the only shackles he needed, that he ruled as her king for the next three decades. Deyalorn prospered during this time. King Dorino's lands did not. Queen Lucretia set fire and plague and war on his lands on the east side of the Tulsi, while blessing her own lands with rains and mild winters. 

After three decades of disease and famine and misfortune, King Dorino could not defend himself against attacks from his neighbors to the east. It was clear that his lands would fall. It was at this time, that Valir appeared outside the gates of Deyalorn. Behind him were one hundred black riders, each a son of Valir, each man as black as his steed, each man trained in the gift. They poured out of the gates like a river of death. They were the first black riders. They could not be wounded. Dorino's people fled before Valir's forces like leaves before the wind. The few that tried to resist, Valir cut down like so many stalks of wheat. On an on he rode, until he reached Dorino's stronghold. But when Valir met his brother, he could not kill his own blood. He knelt before him, and begged for forgiveness. Dorino showed him what the Deyanesi had done to his own people, and Valir wept. 

The next day, he returned with his warrior sons, a shriveled head of an old man, and Dorino's crown to Deyalorn, claiming a decisive victory. The stronghold let them in, rejoicing at the swiftness of his return, and sang songs of his glory. He laid the head and crown at his queen's feet, as she demanded of him. As he entered her grateful embrace, he ran her through with his blade. The sons of Valir took Deyalorn's guards by surprise, and opened the gates for Dorino's army. At Valir's suggestion, Dorino killed the men, and imprissioned the Deyanesi women, because he knew that the gift ran strongest among the women. If Dorino and his children were to succeed in this new land, they must have control over the gift.

When I finished my tale, I found Nisrita looking at me with a mischevious smirk on her face. Carlotta asked, "How much of that do you believe, Commander Romino?"

I had not seriously considered the question before. I had told the tale passed down from generation to generation of the founding of Marsea. "Some of it is bound to be exaggertation, I am sure. But on the whole, it must be true."

Carlotta's face remained impassive. General Madriano had trained her not to reveal her emotions well. The rest of the group did not hide their mirth so well. 

"Is that the story the soldier's tell of Mersea's founding, Commander Romino?" asked Mistress Lazaro's polite voice. 

"It is, Mistress Lazaro. Is it not the tale told in the tower?"

Someone giggled at my answer. She was sharply hushed by her companions. Mistress Lazaro continued when there was silence. "We tell a slightly different tale, Commander. One that we believe may be closer to the truth. For one, it is unlikely that Queen Lucretia's whip turned into a snake. No matter what any tales claim, the gift cannot work with dead matter. It is impossible to know for certain was the good queen could do. Only a handful of people in recorded history have ever had any real control over the old magic. Queen Lucretia was one of them. But it is highly unlikely that even with this control, she turned dead leather into a living snake. The power only effects living flesh."

Ezaro shifted uneasily at the reminder of the argument that had brought me into their company. I shifted uneasily at the reverence with which Sophia spoke of the witch Lucretia. She was Marsea's first formidible enemy. But for Valir's loyalty to his blood, she would have destroyed us before we began.

Nisrita offered an explanation. "The whip was probably lined with poisoned barbs. There are watersnakes in the Tulsi whose venom paralysis the flesh where it strikes. A smart warrior would chose where she struck with this weapon carefully. Wounding at the neck would cause suffocation." 

I believed her. The black riders have been known to fight with poison in the past. Who knows what Makala had trained his wife in.

Mistress Lazaro continued. "As you said, Commander Romino. There were probably some exaggerations. A whip lined with snake venom easly becomes the living snake itself in retelling. It is perfectly natural." She paused to smile, reassuring me that she was not trying to humiliate me. "The tales of the hardships Dorino's people faced were true. I have seen the records in Deyalorn written by his son, Dario. The first several years after settling the lands east of the Tulsi were riddled with disease and drought. Dario recorded one fire, following two years of drought. He only mentioned one plague. Less than ten years after Dorino settled the lands east of the Tulsi, Dario recorded a plague of blood red locusts that came from the west. No one has ever seen red locusts in these lands since. Do you wish to know the story we tell in the Tower?"

I admitted that I did. Even had I not, I could not refuse. The rest of my company had clearly settled in for a good tale. 

Mistress Lazarro picked up her tale. "After Queen Lucretia kills King Aler, his brother Valir convinces his people that they had lost the challenge, and that he would honor the agreement his brother that given his life for. As his armies are still stronger than Lucretia's, she cedes him the lands to the east of he Tulsi river to settle. This negotiation makes Valir an unpopular king, and within a year or two, he abdicates in favor of Dorino. The first years of settlement are hard for both Lucretia and Dorino. The river floods, and crops fail. People go hungry. Fighting breaks out at every opportunity. Valir, a man of honor, unwilling to see his brother's life given in vain, tries and tries again to keep the peace between the two kingdoms. But the land is too hard, and his watches his people suffer too much. After nearly a decade of hardship, he leads an attack on Deyalorn's gates. Lucretia captures him and chains him to her wall for all to see. She demands that Dorino surrender himself and his people to her. Otherwise she would call upon her gods to destroy them. When he refuses, she retires to her temple to pray. She emerges a full day later, her eyes red like the summer moon, her black hair tangled and streaming about her face, her fingers bloody from the agony and fear of standing before the gods, drinking from their well. No one has ever channelled that much wild magic since. When she emerged from her temple, she was already half mad from supplicating her gods for the strength. She stood on the walls of Deyalorn, some say she stood on Valir's head, as she called down the power of the gods into her body. You could see her start to spark and crackle with the energy as she worked. Then came the locusts. Some say they burst from Valir's body first, consuming his flesh, and dripping blood red towards the lands of his brother, before bursting first from Lucretia's orafaces, and then boring out of her flesh as well. Other's say that the Destroyer joined the battle at this time, striking down Lucretia's gods in a great battle in the sky, so that their spell went wrong, forcing her body to turn into a swarm of locusts so large that it clouded the sun. Both tellings are very clear. Lucretia died with that last spell. She tried to control more of the wild magic than she could, and paid with her life. She stands as a warning to us against the temptations of the old ways."

I found my stomach turning. The tale sounded impossible as it was told, but Sophia had claimed that she believed it to be true. She had backed up with historical record, and explained the falacies in my version. I could not erase the image of blood soaked locusts pouring out of a person's ears and mouth from my mind. 

"I liked the part where they are lovers," I heard a man say. I could not look away from my inner nightmare of insects to identify the speaker. 

Carlotta snickered. "I suppose you liked the part where Valir fathered a hundred sons as well?"

Ezaro spoke. "He must have known what he was doing. Imagine, one hundred gifted fighters in thirty years. I wish I knew his trick."

I found myself glancing at Nisrita, uncomfortable at the thought of Makala's wife partaking in this conversation. She was nonplussed. 

"Impossible," a young woman said. 

"No it's not" said Ezaro.

"It is," she insisted. "No woman at the height of her power, let alone a queen, would condescend to lie on her back and bear children. Who would protect her country?"

"Perhaps Valir was convincing," Ezaro continued. The men laughed at the young woman's discomfort.

"I still like the part where they are lovers." It was the plump heretical man who had spoken.

"And when would they have been lovers? Before she went to pray, or after she turned him into a mass of insects?" Nisrita sneered.

"Perhaps she became a wraithform," he offered weakly.

I heard Mistress Lazaro giggle. "A wraith form and a human. I would like to see how that works."

I felt terror creep into my chest. As if the story of Lucretia's fate were not enough, I found myself amoung companions that spoke as salaciously about ghosts as they gossiped about their neighbors. I wondered, again, what type of wife Makala had brought home, and what type of mess I had found myself in. 

Ezaro spoke. "It is possible. Not the lovers part. But I heard that Griswold became a wraithform when he ..." Several of his companions hushed him sharply. 
"Griswold is dead." Carlotta announced, in no uncertain terms. 

"Have you seen his body?" Ezaro asked smoothly. The healers mumbled and shuffled, but did not answer.

"Duke Ergino put earth on his face." I pointed out. The duke's evidence should be enough for any subject of Cortan.

"That is only a figure of speech," Ezaro insisted in his impudence. "The duke put earth on his coffin, which was sealed shut. No one saw the body. Griswold became a wraith form. He's dead now, of course. No good to anyone. A wraithform only lasts a year or two. But he is not buried beneath his headstone. Ask Master Alerio, if you do not believe me. He was Griswold's student when he disapeared. Or better yet," he said, turning to Nisrita, "Ask your husband to dig him up. He will not do it."

"Stop it." Nisrita was on her feet, towering over the seated Ezaro. 

"Careful, Ezaro," a mustached man warned.

Mistress Lazarro stood. "Being Alerio's favourite will only get you so far." She left the circle of friends, presumably to find Master Alerio.

Carlotta took Nisrita away towards the women's tent. There were many reasons for NIsrita to take offence at Ezaro's words. Griswold's death had always been a sensitive topic for Makala and his siblings. Duke Ergino loved his uncle, and will not have ill spoken of him. Back home, Ezaro's statements could easily have earned him a slanderer's muzzle for a week. On the road, his skills were too valuable for him to be punished too severely. Griswold founded the tower at Cortan. Nisrita describes him as having been very strongly gifted, and a visionary scholar. She clearly admired his work. To hear it insinuated that a man so revered on both familial and professional grounds would stoop to something as mean as practising wild magic, an act that goes against some of the oldest laws of Marsea, could not have been easy for her.

The gathering broke up. I found myself in the company of the mustached young man that had quieted Ezaro. "Do you know the story surrounding Griswold?" he asked. 

I had heard enough stories for the night. "I do not wish the Duke would wish us to discuss this further."

The lips curled under the mustache. "I do not think his grace will mind this one tale" their owner insisted. "It is said that Griswold will return. When he does, he will face a terrible choice. He can either return to save Cortan, and destroy himself, or ignore Cortan in its hour of need and watch his family, and the land he built, fall. That is Griswold's prophecy."

I stood transfixed by the statment for a moment. We were days away from a battle at Turina. The thought of Cortan, with her great army, falling gripped at my chest for a long moment. I shook the thought away, ignoring the possibility that it was a warning. I had heard too many stories tonight. "Prophecies only come true in the old tales, my friend."

"All old tales are histories retold. All histories happened once. I believe this is why the Duke wants Griswold to be dead. I make no statement about whether he is or not. I do not know, I will not speculate. I do know that the Duke does not wish for his uncle to have to make such a decision." I could not help staring at the man. Who was he to know the Duke's personal thoughts. "I trained with Duke Ergino's personal healer. A man says many things when he fears for his life."

I bid my mustached friend goodnight, and retired to the safety of my tent. Between dreams of locusts and ghosts and prophecies, I did not sleep well that night. 

***

We arrived at the meeting point a day's march outside or Turina one day ahead of schedule. Gissal's troops were waiting for us with fresh archers, horsemen and healers, and a well defended encampment. Firvona's troops did not meet us there until the morning after the scheduled date. We still had no word from the riders in the Velta Valley. 

I spent that day in a knot of tension. As is the habit of days like that, my possessions conspired against me. I broke strap to my saddle, and soiled my laundry in the only muddy puddle in the entire encampment. My bootlaces, after a month an a half of travel gave way to wear. When my boy saw me reach for my oilskin and whetstone, he chase me away to the company of commanders. It was probably better for my fingers that he did.

As all the Commanders had gathered, General Madriano called a strategic meeting. Sergent Lorzo, who had been delivering intellegence directly to the General, and Gissal's scouts both reported that Turina was massing her troops outside the castle. We weighed the wisdom of attacking now versus waiting for the Black Rider's to arrive. I did not want to move until I had seen Makala once more. Too much can happen in a battle. I argued that to attack now would be to attack with neither Cortan's most elite forces, nor a fair share of her cavalry. According to the reports, the troops amassed outside the city of Turina. The battle would be on open ground. We would need our horsemen. I was not the only one to hold his opinion, though now I regret my stance. It would seem that everything I had put my hand to that day, I would give to the Destroyer. If I had known that he had claimed me for the day, I would have held my tongue. The dance of the gods is unseen and complex. I had no way of knowing.

General Madriano sent out scouts east to follow the enemy's movements, and south to search for the riders. He gave Makala two days to arrive safely to me before we would march without him. I saw Jaquim ride off towards Turina. I considered briefly asking him to search for the riders instead, but the Destroyer had possessed me, and I thought better of it. It was not my place to meddle with the General's orders. Interference like that might get me noticed, I justified, fool and a coward that I am. Would that I had showed more courage then. Could it all have been different with a different pair of eyes searching to the east.

With nothing to do but wait, I found my feet wandering the boundary of the encampment. After circling the perimiter twice, I told myself that my time would pass easier in the company of other officers. Five minutes later, however, I found myself with the healers. When Mistress Lazaro saw me she greeted me with the warmth that I have come to expect from her and led me to the tent where a group of younger healers were practising and discussing. I could hear a cluster of raised voices from within. Among them, I recognized Ezaro's. Mistress Lazarro raised her carefully shaped eyebrows at the row and begged my forgiveness for what must seem like unruly behavior to an outsider. "The only way we can train and learn is by practising and challenging each other's ideas. Among the young, challenges and discussions become quite heated."

I put my head into the pavillion to witness an argument between Ezaro and a tall young woman barely older than Nisrita over what seemed to be a stoat struggling against its bindings on the table. I watched the creature struggle and squirm itself off the table, only to be replaced by a white robe before it could get much further. The stoat bit the young man for his courtesy. I did not follow the discussion, but watched only for a moment as the onlookers interjected comments and suggestions to the two at the table. An older matronly woman, showing the first wisps of grey in her hair, watched the proceedings without comment. Nisrita had introduced her to me as Master Adele. I recognized the expression on her face: observant and critical, but also alert and ready to step in should matters get out of hand. Her eyes glanced disaprovingly in my direction. I was a potential disruption to her students' studies. I saw Nisrita's back sitting uncharacteristically quietly near the back of the group. I withdrew my head from the tent. If she turned to see me, I ran the risk of a lesson in a stoat's anatomy, or an explanation of what cruel tests the two at the table had proposed performing on the poor creature. I did not have the patience for that that afternoon.

"You have no need to apologize, Mistress Lazaro," I said turning to my hostess. "I train recruits for the standing army in Cortan. It would seem that a group of ambitious competitive youngsters training together looks very similar across our professions."

"Would you like me to tell Dutchess Cortan that you are here?"

"No, thank you. I will not disrupt her studies. Tell her that I called when she is free."

I turned my feet towards my own kind. Mistress Lazaro saw me to the invisible boundary between the two camps. "Do you have any news of the black riders?"

The question startled me. "No. Why do you ask?"

"The Dutchess has been worried."

"Tell her that scouts have been sent south to search for them." I could have added that the black riders were Cortan's best trained and most agile troops. They have never been lost. Those words would have done nothing to calm my nerves. I elected not to insult Nisrita with those empty assurances.

I spent the evening at the dice board, losing more than I won, and the first part of the night patrolling the walls. No officer beyond a sergant stands watch. My walk along the walls was a personal vigil. Sleep for me was hopeless during those early hours of that night. It served to tire my feet, even if it could not touch the worries that preyed upon my mind. By the time the midnight change came, I had heard nothing of Makala. The men on duty, however, informed me that my presence presented a distraction. I retreated to my bed.

The scouts from the south returned a little after dawn, along with a handfull of outriders from the southern party. A small group of healers and mounted support left immediately. I was certain Nisrita rode among them. The thought gave me comfort. The scouts and riders told all who asked that the Nievian had discovered the Black Riders' route. They blocked the main routes to the meeting point with more troops than they could take on, forcing the Riders into the hills. The Riders are an extremely mobile unit. If Niev hoped to corner to trap them by blocking the roads, they failed. They only slowed them down. They did encounter pockets of resistance, and suffered some injuries. All told Cortan had lost three cavalry men, but no riders. 

I breathed easily for the first time since we stopped marching. When the troops from the east returned early that afternoon, General Madriano called another meeting. 

Niev's armies were marching. It would seem they intended to meet in the high valley half a day's travel from Turina. It was a trap. This much was certain. Cortan's armies with our strong mounted units took the field in open ground. Turina was built deep into the surrounding mountains. It would be an easy place to defend and difficult to take without a seige. It did not make sense for Niev to meet us ground favorable to us, less so when the stronghold lay so close. The sergants running intellegence had no inkling of what the planned trap may be. The commander's argued strategies. There were those that suggested that we spend more time in our defended camp, to investigate the terrain chosen by Niev. It was foolish, I argued. Niev had units scattered along the mountain roads for the purposes of defeating the Black Riders. Any delay would enable those troops to join the main mass of force before us. Commander Morel, who led the conquest of Castle Bayam, wanted to circumvent the valley, and march to Turina by the northern ridgeway. Under the best of circumstances, I would have disliked that idea. It would leave our flanks exposed to the south while marching, and then to the east while attacking the castle. Cortan's strength was not in taking castles, whatever Morel's personal tactical strengths may be. By attacing Turina directly, we would take the battle to a position of relative weakness. Unlike the Nievian army, we would do so without knowledge of the terrain or a trap to protect us. The debate continued for some time. The gift has made Cortan's army powerful, but it has made us conservative. In the choice between a quick heroic death and a lingering painful injury, we chose pain every time. In the choice between an unknown risk and a known one, we lean towards the known. We do not like to lose our men without good reason. 

The horn at the gate announced friendly troops. I did not need General Madriano to read the note delivered to him to know that Cortan's riders, that my Makala, had returned to safety. The debate paused for the announcement and a few moments of relief, then picked up again. I itched to leave the repetetive discussions to see the young duke safe in his tent, and stiffled the desire. We were both on duty. Our personal time would have to wait. He or I would find an excuse to linger within the other's field of vision, perhaps exchange a few words, but no more than that was possible. That was the reality of life on the campaign. I should have been used to it by then, but I was not. I found myself jealous of the intimacy that Commander Lazaro could share openly with his wife as she served with the healers. I forced my mind to the discussion at hand. I found I had another supporter for my plan to attack in the high valley. With a little more work, we convinced a third, and the decision was made.

The sun was low as I returned to my tent, confident that I had pushed the right course of action. I was as convinced of my wisdom then as I am now that I was wrong. I have no one but myself to blame for the fact that I lost everything by the time the next day came to a close.

When I entered my tent, my boy handed me a note. It was not on a small scrap of paper, as messages from Makala usually are, but on a neatly folded, sealed full sized piece of paper. It was the duke's seal. I paused for a moment before cracking it open. Makala would not write me like this. Who then? Duke Lukos? He had no need of me that could be settled by sending an army page to wait for me in the tent? Nisrita then? I had not received a formal missive from her since leaving Cortan. Everyone in the camp knew of our friendship. A good many made it out to be more than what it was. The speculation was pointless. I cracked the seal and read. "The Dutchess of Cortan requests the honor of Commander Romino's company in the Duke's tent, at his convenience. If he is unable to grant her his company..." etc. etc. I dropped the note onto the fire pit from habit, found what passed for a clean shirt on a campain and left. 

I must have been walking a foot of the ground. Gone were the days of mysterious one word messages to indicate that we wished the other to come, and two word messages to ask permission to pay the other a visit. A colluding wife, a happy marriage and unsubstantiated rumors of an affair sheilded us for now. An heir would complete the fascade. Nisrita could follow her studies. Makala and I could keep each other. I had been so foolish, I remember thinking, to ever have harboured any jealousies, or to have doubted Makala's devotion. I thought that there was no limit to the dreams he wove. 

I paused at the entrance to his tent with my hand at the flap. I had been invited here. I had nothing to fear. The years of necessary secrecy had ground caution into my bones like an instinct. I paused to listen. My hesitation was rewarded the muffled sound of Makala's soft laughter, and the sound of Nisrita's voice telling him of a contrary stoat that refused to be experimented on. I entered.

She stood to welcome me warmly, as a hostess should. He stood awkwardly, smiling shyly. This was uncharted territory for him as well. "It is good to see you Timmon," was all he said. 

"And you, Makala. Thank you, Nisrita." I said, turning to her. 

She looked at us with an expession of incredulity and disdain. "A fine pair of lovers you make. I recall having more courage on my wedding night." 

Makala turned to her as sharply as his nature allowed. "It is not so easy, Nisrita." He left it at that. She did not know. She had not spent years in hiding and secrecy. At least she had the wisdom to speak quietly.

She ignored Makala and turned to me. "Where have you been hiding this modesty, Timmon? And where was it when you told of Valir's one hundred sons? Kiss him. I won't watch." Makala's mouth fell open at the words. He looked from me to her in disbelief then shook his shoulders in laughter. Nisrita look at my frozen form impatiently. "Oh of all the idiotic things in this world." She turned her back to us.

I kissed his mouth still convulsing with silent laughter. He flinched when I put my arms too tightly around his familiar narrow waist. He had been injured. I losened my grip and tasted his toungue. He smelled of garlic and thyme. The days of worry and not knowing had made the time crawl by. It felt like years since he had burried his face in my neck.

When I released him, Makala called Nisrita from her charade of modesty. "At least this has clarified on matter for me," he said. When we both looked at him questioningly, he continued. "On my way here, several healers warned me that I may have had horns put on me." I felt my mouth widen into a grin and saw, with some satisfaction, Nisrita's face flush purple. "It is good to see the two of you close," Makala continued. "It is good to see you both well." 

There was nothing to say to that. I kept my grip on his hand and sat with him while Nisrita poured refreshments. 

"Now will you tell me?" she asked Makala impatienly when she had finished. "Timmon is here. You will not have to repeat yourself." 

"In a moment, little one," he answerd. "Timmon, when do we march?"

"Early tomorrow," I answered. "Tell her what?"

"She wants to know what happened on the road here. She saw my injuries. Naturally she is curious." Naturally. Now I was as well. "It will have to wait. I will not speak of danger I faced the day before I go to battle." Nisrita whined. "I will however, tell you of our journey up the Velta." Nisrita forced herself to be satisfied with this lesser tale, took a rug off of Makala's bed and settled herself on the ground by our feet. 

The evening passed like a dream. Makala lay on his bed while I sat with my arms entangled with his, knowing that neither healer nor soldier would interupt the Duke's reuinion with his wife. Nisrita lay on her rug next to us as if we were in the most natural position in the world. She twisted the tassles around her fingers and listened to Makala's stories. He told us of the wonders of the fertile land he had scouted. The Velta, he claimed, teemed with enough fish to keep an army from needing a supply train. The surrounding hills were always green, he claimed. The peasants brought in four crops a year. They had developed a complicated irrigation system that brought water from the river in the valley into the terraced hills surrounding it. 

He described temples built out of marble and silver, and idols with precious stones set into their stone faces. He described a palace twice the size as the palace at Deyalorn, with windows fit with etched and colored panes of glass and floors made out of rich mosaics. The garden had orchards of mangos and pomegranites, and flowers constantly in bloom. Every man owned a cow, and no meal was complete unless the food was smothered in cream. It was a rich land and prosperous. 

Makala, dreamer that he was, thought that it could belong to Mersea without a military conquest. When I balked at his statement he indicated his saddle bag. I retrieved it. From it, he extracted two large red fruit, handing one to Nisrita, the other to myself. 

"Pomegranites!" the girl exclaimed, tearing greedily into her fruit.

Makala smiled at her joy. I made the mistake of wishing her many healthy children. She returned my wish with a withering look that only women of a certain age can pull off and asked Makala "Did you give him one in the hopes that he would give you heirs as well?"

"Nisrita." Makala sobered and chided her.

"I am not back in Cortan yet. Mother is not here. She will not hear me speak like this." When Makala's face did not lighten, she added more meekly "I will do as we agreed Makala. I promise." She held out her forefinger for him. When he linked it with his, in that ubiquitous sign of filial friendship, she turned to me and apologized.

I waved away the thought that any insult had been given and turned to Makala. "Where did you get these?"

He resumed his story. The Velta Valley is prefecture governed by Prince Keno. The Prince hosted the Black Riders in his palace for a week as envoys from Marsea. His younger half brother is king of the lands extending from the Velta valley to the southern sea. They were currently engaged in an expensive sea battle to the south. Makala impressed the prince with stories of Cortan's and Marsea's military prowess. After a week, Prince Keno admitted that he may be willing to sell Marsea a portion of the Velta valley to resolve some of his personal financial difficulties. 

My jaw fell to the floor. "Your father sent you as a scout to survey Cortan's next conquest. You went as a diplomatic envoy."

He sank into his pillows with a satisfied expression on his face. "I do what is necessary for my country and my duchy. Be that fight, talk, or scheme. Look at us, Timmon. Between your strategic skill, Nisrita's gift, and my diplomacy, is there anything that Cortan could possibly want for?"

"You are mad, Makala. Marsea will want to expand south again. We cannot simply form an alliance to break it a decade later. And what about the people. We cannot simply absorb the ungifted masses into Marsea's population. That would leave the new dutchy bereft of the gift."

"All in good time, Timmon. There are many other directions to expand before we need to go south of the Velta. Cortan has not yet conquored the lands between our borders and the northern hills of the Velta. As for the people, we will work something out, I am sure. You have not seen their cities, Timmon. They may be ungifted, but they have machines that we do not. If we destroy the cities, and chase away all their craftsmen, we will lose their knowledge. The Velta dwellers are certainly the first people Cortan has met that has something to offer. Marsea will conquer them, eventually. But it would be a pity not to learn what we can from them." He paused for a moment, considering his words. "They are an intriguing people, Timmon, their gods are generous. He reached into his bag again and pulled out two leather pouches. "The Prince gave me these as tokens of his good will. To bless my marriage and my love life." 

He handed Nisrita and myself each a bag. Hers contained a beautifully carved, if rather explicit, locket of a voluptuous, half naked woman carved in a soft off white stone. Mine contained a locket of an extremely well endowed naked man, carved of the same. 

"This is for Nisrita," I whispered, not fulling grasping the situation. Makala shook his head almost imperceptibly.

"I cannot wear this," Nisrita declaired, blushing. She stared at the stone with curiosity and embarrasment on her face. I wondered for a moment how someone so intimately and often crassly familiar with the human anatomy, and so casually willing to discuss death in all its forms, could be so flummoxed by the female form. Her form. I am not fated to understand women.

"Then don't, little one. But keep it hidden. The gift was well intended."  In otherwords, do not let the dutchess see. I could not imagine Dutchess Cybeline reacting well to her adopted and hated daughter owning such leud object. Perhaps Makala could explain it away as a gift, but the pair had enough trouble with their mother to go seeking more.
 
I turned the male figure over in my hands. The female locket was to grant my duke a fertile marriage. That much was clear. But what was the man?  In Marsea, no man would give another such a figure. It would bring the label of miscast on both their heads. "Then this is for you," I suggested.

"I do not think so." Makala replied softly. There was awe and clarity in his subdued tone. He held my eyes with his gaze. 

I brushed my finger over the form again. A fine layer of sand came off onto my skin. What had the Prince said to make him so certain? Makala had said one would bless his marriage, the other his love life. It was impossible. "The prince knows?" I wondered, awestruck. 

"I told him nothing." Makala's voice was low, barely audible. He put his fingers between mine and held my hand to his chest. His eyes still held mine, but his mind was elsewhere. He was scheming again. 

Nisrita leant in to examine my male figurine with an expression of curiosity and disgust. I let her have it. "What are you suggesting, Makala," slightly terrified of what he might answer. A country that did not think men like us abominations. Gods that blessed unions such as ours rather than cursing our existence. It was impossible. Even Makala did not dare speak these thoughts outloud. It could not be. Makala must have been mistaken. But he was schemeing. I did not want to know what he had planned.

"I am suggesting nothing at the moment. In a few years, when we turn our attentions to the Velta, I will propose to my father that we not destroy her people. That is all." His face was distant and sad. As I blinked, I realized that my eyes had filled with tears. Where was this new unfamiliar feeling of longing and hope coming from? I sat with my hand on his chest wondering what he had seen in this strange new land.

Makala roused himself first. "Enough. Now I want to know what the two of you have done these last six weeks."

Nisrita and I took up the tales of our travels. Dinner was called shortly, and Nisrita brought our food to the tent. She ate in silence, and let me do most of the talking afterwards. 

Makala silenced me less than an hour after the meal and pointed to her sleeping form. "She spent the morning and much of the afternoon riding hard and healing. She is exhausted, poor girl," Makala explained as I covered her. The evening was still mild, but the predawn hours in these hills were cold.

"As are you, I imagine." I said, bending down to kiss him good night. He had spent the evening reclining on his bed, his body still recovering from the wounds and healing he had endured. I felt his fingers stroke the softer skin on the inside of my arms. I had missed him as well. I let my kiss linger, then decided against departing immediately. My duke's back was weak, and pressure against it caused him pain. I undressed him gently, then let him guide my mouth onto his nipples and navel. We moved carefully and quietly, for Makala's sake, for the sake of our privacy, and out of courtesy to the girl sleeping in the corner who had made this possible. 

When we had both spent ourselves, I curled myself around his back, letting my duke settled his wounded back as he wished against me. We were a tight fit on Makala's small bed. "I am worried, Timmon," he said, quiet and low, when he had found a comfortable position. 

"About the morning?" I asked, running my fingers over the bulging veins on his limp sword arm. I felt him nod in the darkness. Makala was not a coward, by any means. Every man needs a place to put his worries. With my greater experience, I served as this place when it came to warfare. "We have won every battle we have faced against the Niev. Tomorrow will be no different."

"This feels wrong. Only a fool would meet us in open ground like this."

"It may be the last stand of a desperate army." I felt his back tense as the protest rose to his throat. I stroked the muscles and quieted him. "I am not making light of the situation, Makala. Niev's army is much smaller than ours, but they have fought bravely, and they have fought cleverly. Their cunning and bravery is not enough. They cannot stand against us. Whatever happens tomorrow, you will return home to gift your father Turina." I was certain of our army's superiority. I had fought Niev for the last six weeks. Makala trusted my judgement. I felt his body sag and his breathing slow. "Sleep now," I whispered. "You will need your strength in the morning."

I rose and dressed. When I bent to kiss him again, he was asleep. I returned to my tent a happier man than I recall ever being since childhood. However Nisrita fit into this strange triangle we three formed, it felt, for the first time since I left my father's house at thirteen to train at Deyalorn, like I had a family I belonged to. Together, as Makala had said, what more could Cortan possibly need? What more could we three need. I whistled into the night air. It was not yet late, there were men milling around fires. Some that knew me well looked up in surprise. I smiled and wished them good night. I am not known to whistle, or sing in public. On the contrary, I am known to be the sole man to not partake in a drunken tavern song. I have not whistled in public since I was a boy swinging on my father's apple trees.

It pains me to think of my joy that night. It was so intense, and it was so fleeting. I thought I had everything that night. Perhaps I did. I lost it all now. So has Cortan. I wonder if the Duke has any hope but to pray for Griswold. Yet more painful than the memory of that joy, is the knowledge that I lied to Makala that night. I did not know that I had lied. That does not change the falseness of my statement. Makala trusted me, and I lied to him. There is nothing I can do now to change that fact. 

***

I did not see either Makala or Nisrita the next morning. I rose before dawn and left with the infantry. The riders and healers would leave a short while later, bringing up our rear. It was a beautiful morning. The skies were clear, and the air was cool. It was good weather for a battle. The sky had been blue for some hours by the time the sun first made its appearance over the hills. I remember feeling priviledged to see the first rays chase the shadows from our valley. The sense of ellation from the previous night had settled itself into a deep contentment. I felt like I was going for a ride in the hills, not marching to battle.

The sight of the enemy lines at noon did little to somber my mood. We outnumbered them four to one. They had few horsemen. Even if they had people hiding in the hills, and our intelligence thought this unlikely, they could not have enough men to make a difference. We would take this fight carefully, and win the day easily.

General Madriano called the charge. I took my infantry and cavalry units forward. The first of our cavalry had just reached the line of pikes when I heard the noise. The ground shook beneath my feet. My steed threw me and bolted in terror. So did most of our other horses. When I stood, the sky behind was clouded by a haze of dust. Underneath, a great jagged maw opened into the valley floor. I surveyed the battlefield. We had gotten perhaps one in eight of our infantry men, across before the earth opened to swallow our men. Destroyer only knows how many lay wounded in the bottom of that pit. We were outnumbered now. Most of our cavalry were foot soldiers now. Here and there a lone horseman fought without support.

The black horn blew. It is long wooden horn with a low loud sound designed to inspire fear in Marsea's enemies. It has always given me hope. Across the field, the black riders moved as a dark cloud of death, cutting through Nievian lines, working their way towards the northern hills. They do not have the reputation for being the best horsemen in the land for no reason. It seemed most of them had kept their seats. The horn called the remaining mounted men to ride with them.

Someone found me a horse. It was not mine. It did not matter. My role was not to ride, tempting as it was. If our troops could still be coordinated, then that was my purpose. I looked to the hills to see what had drawn the rider's attention. I saw nothing, but I was positioned on the northern end of the field. They had a better vantage point. I turned my attention to the rest of our army. Our archers had positioned themselves as close as they dared to the edge of the maw and were trying to give our infantry some support. The range was too great for them to be of significant use.

"Back!" I told the man handing me the horse. "Order the men back. Draw the Nievian closer to the archers." It would leave our men little room to manuever, with the enemy to their front and the hole to their back. We were cut off and trapped. That had to be fixed. But for now, my infantry needed cover. I saw the men retreating, and other commanders see the same positioning I saw. I turned my attention to the next problem. I spurred my horse to examine the hole in the ground.

It was a wide chasm, between fifteen and forty feet at points, but not especially deep, as least not where I stood. Perhaps twenty feet deep. A man falling in would not injure himself badly. Unless of course he was burried by earth. 

A thin line of men carefully climbed around the rocks and boulders that formed a boundary at the northern end of the maw. I assumed a similar phenomenon at the southern end. I saw other men tying together rope ladders and attempting to build makeshift bridges. It was too slow. The hills on either side of the valley had been cleared of any trees thick enough and strong enough to serve as a bridge. I saw no obivous tools. The men on the northern end of chasm did not either. I rode down the gap. It was shallowest at the edges, increasing to a depth of perhaps fifty feet towards the middle. At least that is what I saw of the northern quarter I traversed. I stopped exploring because I saw our men doing something ingenious. At one point, the maw in the earth had left a gently sloping ridge on both lips. Someone had placed a wide plank of wood where the two ridges came within five feet of each other. Men scrambled down one slope, across the plank and up the opposite. The gift gives Marsea's soldiers courage such as this. If a man slipped off the plank, he faced a thirty foot drop. He would at least break a bone or two. To our men, that did not matter. The soldier who laid the plank would be promoted for his ingenuity. His squadron may be granted a gifted healer if it did not have one already. It was a noble effort they had undertaken, but too slow. Men had to cross one at a time. But the idea was sound.

I went back to where the chasm was shallower and ordered men to drop down where they could and move earth to build stairs. It would take time, but one a good slope was established, men could march across it in fives and tens. It was faster than anything we currently had. Once I had conveyed the idea, I saw one man run to the back of the line, presumably to enlist our engineers. Another ferried the message to the southern end of the valley.

I stood and organized the digging. Most of my officers had crossed over. They were more than sufficient to commanded the movements of the few men I had engaged in the action. The archers gave them cover from the eastern lip of the chasm. There was nothing more I could do without more troops on this side. I heard the black horn call again. I turned to see our Black Riders retreating, pursued by a long line of cavalry. I wondered where the horses had come from. Then I realized that they had been hiding in the hills. The Black Riders had seen Niev's cavalry in the hills, and had cut through enemy lines to meet them before they could desimate our footmen.

There were too many. Our black riders must have been outnumbered nearly three to one. I wondered if we had lost the day. With our army cut off, our cavalry dismounted, and our best troops in trouble, there was little we could do. I watched the two tight masses of black fall back through gaps in the enemy line. Then I heard the unmistakable sound of five hundred bolts being loosed in unison. I understood.

The riders may have been outnumbered, but escape had not been the intension of their retreat. As soon as they had drawn Niev's infantry within range of our bows, our archers changed targets. I watched a second round of bolts fly through the air, and Niev's cavalry shrunk to half its original size. One or two more rounds like that, and our archers could return to covering our infantry. I turned to see them release the next round of equestrian death.

Then I saw it. It was the beginning of the fall, though I had not realized it then. Had I blinked, or turned a second later, I would have missed it. As it was, I did not wholely believe the evidence of my eyes. Time has verified what seemed so incredible at the moment. I saw Duke Lukos fall. The archers fired from across the chasm. The first line, containing the younger duke, stood at the edge, knowing that every foot of ground increased the chance of an arrow hitting its mark. He tripped as he moved the step forward from the ranks of those reloading to the ranks of those firing. I say tripped. That is what I wanted to believe had happened. What I saw was his body spasm, then crumple, then fall over the cliff. It was too incredible to believe, so I chose to believe that he had tripped, and lay injured at the bottom of the rift.

I signaled for healers beneath the archers and returned my attention to the effort at bridging the gap.  We had two good fords built. Men crossed twelve at a time. The third bridge was nearly complete when I heard an infantry horn from the south end of the field. I saw men fording the chasm by the scores. My heart lept with relief. The south end of the chasm either differed geographically from the north, or someone had a better idea than mine, or both. Whatever the circumstances, this new development may be enough to turn the tide of the battle in our favor. 

I learned later that the chasm had been formed by collapsing a mining tunnel that ran under this valley. Something about the quality of stone in that end of the valley made the rift narrower and the walls less sheer and cliff like as in the north. That was enough to work with. Commander Lazarro jammed dozens of supply carts into the the narrow gaps to build wide stable bridges for men to cross over. 

With men pouring onto the field, I turned my attention to the battle. Lazarro's luck had indeed turned the tide. Within an hour it was clear we had gained the upper hand. In another, the Niev line started to break and retreat. I cursed and ordered my men to pursue. Moments later, I heard General Madriano do the same. Niev had feinted and retreated too many times in this campaign. We would not let this battle be moved to Turina. It had to end in this valley. I heard an infantry horn from the northern hills, and saw our men pouring down the slopes at an angle to cut off the retreating Nievian lines. The bastard miscast Morel had taken men up to the ridgeway after all, inspite of the dawn instructions to stay in the valley. It must have been a desperate attempt to circumvent the rift and join the battle. I prayed that the Preserver grant him a long life for his stubborn disobedience.

I heard someone call my name while in the heat of battle. With his help, I dropped the Nievian horseman before me and turned to look where he pointed. A lone healer sprinted across the open field from the rift to Cortan's line. He had no armed escort. I cursed Master Alerio for his lack of control over his men and turned my horse to meet him. Our healers are visible targets in their gleaming white robes. Killing one of them is easily worth the lives of a score of footmen. Even behind our lines, it was not safe for them to move alone in the open. It would only take one man breaking through our barrier to kill one of our most valuable assets.

When I reached Nisrita, she grabbed my hand and jumped onto my moving steed. She had done this many times before, I am certain. "What in the Taker's seven names are you doing?" I yelled back at her. 

"Find Makala," she said. "Lukos has been poisoned."

"Is he dead?"

"He was not when I left him. He probably is now. Find Makala. Get him to the healers. It may not be too late for him yet."

My heart sank. Duke Lukos was dead. If she was right about the poison, and I had no reason to doubt her, it would seem that my eyes had not tricked me when I saw him spasm and fall. I spurred my horse towards the rift. If there was foul play involved, I needed to protect her life. All three of Cortan's heirs were on the field.

"You are going the wrong way!" She protested. "You have to find Makala. You have to save him."

"He is with his riders. They will keep him safe. You must get back to the healers tent for your own protection. Niev cannot get to you there." I shuddered at the thought that this battle for Turina had turned personal against Cortan. All of her heirs were on the field. 

"It is not Niev, Timmon. Firvona poisoned him." I felt my blood run cold at the words. "You must listen to me."

As I had reached the rift, I stopped my horse and did. "Tell me what happened from the beginning."

Her words tripped over themselves in their haste to get out. "We could not save him. His wounds were light. He had only fractured a rib and broken his femur in a few places. He had no signs of internal bleeding. He should have been stable. We should have been able to move him. But he kept lapsing into fits. We could not find any injury on him that could induce seizures. His helmet is unmarred. He had no head trauma. Finally Master Adele asked if he had suffered any other injuries before the fall. He told her he had  been hit by an stray arrow from one of the Firvonese archers. One of his gifted comrades had healed the wound almost immediately. When we turned him over and saw the injury, it was black and swollen. The poison had entered there. Master Adele cut open the wound and sent back for anti-venoms, but she said that our healing weakened him. He will not survive long enough for help to arrive. No one has healed Makala. If the Firvonese have done the same to him, there may still be him to help him. Please. Turn this horse around."

"No." What could I have said. Makala was on the other side of Niev's lines, blocking their retreat. I could not get through to him alone. I was not foolish enough to risk a healer's life by taking her with me. Makala needed to be warned, but she had not evidence that Firvona would poison him. The only hope I thought Cortan had, in fact, was to keep Nisrita with her healers and Makala with his riders. Firvona would not have access to them among their own. 

"Then give me the reins" she said, reaching over my arms.

"Cease, healer." I commanded. She stopped short. "You are in the middle of a battle. Return to your post with the corps. That is an order."

Nisrita looked at me in surprise, then dismounted. "Yes commander." He voice was surly, her shoulders sagged, but she obeyed. She crossed the rift, picked up an armed escort and walked back to the direction of the healing tent. I turned back to the battle, intending to take a knot of men through the Nievian line to Makala's position. 

Warfare is never simple. By the time I had gathered a squadron of horsemen to work through the line, I saw Commander Lazarro fall, and his line break in two. Nievian soldiers flooded through the breach. I was torn between my duty to Cortan, and my role as a commander. I let the mounted squadron carry my warning for me, and went back for men to patch the breach. I regret the decision, of course. But there are so many decisions that need to be made in war. Nothing I could have done that day would have changed what happened. The destroyer's dance had already started. I was not a part of his plans.

I held the line with the men I had gathered, and Niev surrendered shortly. The sun had just sunk below the western hills when General Madriano called the peace. I should have stayed on the field to take prisoners, but I saw the Black Riders taking their wounded towards the healing tent. I could not sit idle. I rode out to meet them.

Their captain, a man named Diel, saw me and rode out to meet me across the open field. He saluted somberly and said, "We got your message, Commander. It arrived too late."

I heard the words without comprehending. My voice emmerged from the approximate location of my mouth. I had not told it to speak. "What happened?"

"The duke suddenly slumped in his saddle while drawing Niev's cavalry towards our archers. He seems to have had a fit of some sort. We were fortunate enough to lead his horse away to quiet ground before he fell from his steed. Our healers tried to help him, but they could do nothing. He was not wounded. He died within the hour." 

"Poison."

Captain Diel took offense at the word. "No Black Rider has ever poisoned himself in battle. We know how to use our weapons."

"Firvona, not his own." I must have sounded irritated. I did not want to have this conversation anymore. I saw a lone white figure sprinting across the field to the knot of black horsemen. "I want to see him."  

Captain Diel followed me to the Makala's body. Nisrita had him down on the ground, and was struggling with the ropes that had bound him to the horse bearing him. Two black riders stood uncertainly, just beyond the range of her gift's weapon. As I dismounted and approached, I felt a burning sensation in my chest, a first warning of the pain that she would inflict. 

"It is I, duchess. Let me approach." The burning stopped. She had not raised her eyes from the knots on his arms. In the insane disassociation of grief, I remember aproving of her ability to gauge her surroundings even under distress. 

I handed her my knife and dropped to my knees beside Makala. His eyes were closed. He did not have a mark on him. He looked asleep, like the last time I had seen him. I put my hand on his shoulder and shook him gently. If he were asleep, surely he would wake up if I called him. "Makala" I mouthed. All my grief had welled up and lodged itself in my throat. I could not utter a sound. 

When Nisrita tried to turn him over, I helped mechanically. I did not know what she was doing. I did not pause to observe. My mind was elsewhere. Makala was four years younger than me. He was Cortan's favourite son, the key stone of Cortan's children. The Maker had gifted him with an iron will. He had given him charm, a gentle manner and guile. No one could stand between him and what he wanted. How was it possible that the gods had failed to be won over by his charisma, how could the Destroyer be so cruel as to take him from this earth so early. How was it possible that he was no longer in my life. 
I traced his jaw with my forefinger. A thin layer of stubble had formed there, gently poking at my skin. He was always meticulously clean shaven. He had never let me come to him otherwise.

"Commander Romino!" Nisrita's voice called me sharply, preventing me from making a display. I looked up to see her struggling with his leather jerkin. her face was marked with tears. "I cannot see. Help me find the wound in this leather. Be careful. I have not removed the weapons. Are your hands uncut?"

My only wound was in my right shoulder. I dried my eyes, took the jerkin from her and started my search. She turned to the body and told me to look behind the left shoulder blade. She had ripped open his shirt. I looked down to see Makala's left shoulder black and swollen. A layer of pus and blood made a thin line just below the shoulder blade. It was not a long cut. It did not look particularly deep. I found where the leather was broken. 

"Take this to the healers, if you would. They may be able to match the poison with Lukos. I would like to stay here with my husband."

Was I angry at her for those words? I was too grief striken to notice. Perhaps it was best that she sent me off. I would not have been able to speak for me actions in my grief. Perhaps she saved me from making a spectacle of myself, of revealing to the world what Makala had tried so hard for years to hide. He was my husband by any meaningful definition of the word. Whatever claim Nisrita had on Makala, it was not that word.

I left for the healer's tent and found it in chaos. I found Commander Lazaro's body laid out for washing on a table, and his wife working grimly and stoically across the tent in a widow's veil. I saw one or two other widow's veils in the compound. Niev had not surrendered easily. We had suffered many deaths. Just stabilizing the injuries from today's batle would exhaust our healers. We still had to take Turina with these weakened forces. The next day would be grim. 

I found Master Alerio in a quiet tent on the edge of the healer's encampment. He had Duke Lukos's body laid out on a table, and was examining a young healer on a different table. He accepted Makala's jerkin and my third hand description of his death wordlessly, and turned back to the woman.

"Who is this?" I ventured, pointing at the dead healer.

"Her name was Lysene. She was one and twenty. She came from peasant stock in Gissal. She joined Cortan's tower a year ago when she married one of our members. She showed promise as a practitioner." I looked at Lysene. She was a non-descript woman, a bit short and muscularly built. She looked younger than her age. I would have guessed her to be sixteen.

Her right hand was black and swollen. I could just make out a thin cut across the back of her hand. "Did she die as the dukes had?"

Master Alerio nodded. "Her seizures stopped about four hours ago, shortly after you called for help for the young duke." He took a slender blade and cut a piece of black flesh from the wound in the dead woman's hand. He placed it carefully in a dish. I studied the woman. I did not understand why she had been poisoned. How was a farm girl from Gissal connected to the dukes? "We could not have saved the dukes." Master Alerio continued apologetically. "We do not have the serum for this poison. It is a fickle serum, the plant it comes from does not grow in this region, and the black riders do not use it." He did not need to justify his failure to me. As I turned to leave, I saw Master Alerio cover the woman with a shroud out of the corner of my eye. Something in the vision reminded me of last night.

"Master Alerio, where was the duchess all day?" I asked.

Master Alerio looked at me in confusion. "She is not poisoned, Commander. We have checked."

"I trust your thoroughness, Master. Did the duchess see wounded today?"

The Master shook his head. "She saw no one until her brother fell. We wanted to hold her stamina in reserve until we knew what would happen in the battle. I do not know where she was. Probably in the reserve tent."

I studied the dead woman's form under the sheet. The idea taking form in my head clarified into a certainty. "Did Lysene see any Firvonese soldiers?"

"We had several Firvonese men here through the day, Commander. I cannot tell you who Lysene say. They are all being held for questioning. We cannot spare the torturers at the moment."

"Does anyone know of Lysene's death?"

"We have not made it a secret among the healers, but we have not told the rest of the army."

"I would thank you to keep this death quiet for as long as you can. Good day, Master Alerio. I will not take up any more of your time."

I bolted from the tent to search for Nisrita. I did not dare ask anyone where she was. It may already be too late. Even if Firvona's fighters did not know of the mistake, Firvona's healers did. Without knowing who or how many were involved in this conspiracy, it was impossible to guarantee her safety. Her anonymity was all that protected her.

I found her walking back slowly with Captain Diel and Makala's body, conspicuously the widow of Cortan's heir. "Duchess, I must speak to you." I pulled her reluctantly from her vigil, out of sight among a cluster of supply wagons. "Do you have your armor?"

"It is with my belongings on the supply cart."

"Wait here. Do not move." I found a message boy and ordered him to find the duchess's armor, a map of the area from here to Bayam, and a horse. Then I found Sergent Nicose and asked him to meet me armed and ready to travel covertly cross country in half an hour. He was a good man who had served under me for years. I trusted him to take Nisrita to safety.

When I returned to Nisrita, she was standing irritably where I had left her. "What is the meaning of this, Commander?"

"You have to leave, Nisrita. Lysene is dead."

"I am aware, Commander. There is not a healer here who is not mourning her." She was furious. Did she blame me for Makala's death? 

I ploughed on. I could not let her persish as he had. "She looks like you, Nisrita. The poison was meant for you."

"She looks nothing like me. I am not leaving my healers. Cortan's army needs healing. I have barely tapped my power today." 

Nisrita was a stubborn girl. She walked briskly past me towards her tent. When I grabbed her with my good arm, I felt my hand burn. I kept my grip. "Stop that. Listen to me. If you still wish to get yourself killed after I am done, I will not stop you."

Nisrita turned to face me. The pain in my hand flared until I let go of her. "Speak," she commanded.

I flexed my hand and looked at it. I expected to find my fingers blistered, or at least red and inflamed. My hand eerily looked as it always did. The burning sensation was slowly fading. "You were injured and in Cortan's infirmary when Duchess Roschilde came to visit before we marched. No one in Firvona's delegation saw you. You have no clearly identifying features on your face. If I were to describe you generally as short, squarely built, and muscled like a fighter, I could just as well describe Lysene. Firvona wanted to eliminate all the heirs of Cortan. Including you."

Nisrita steadied herself against one of the wagons. For a moment I thought she might fall. "What should I do?"

"You cannot stay here. It is sheer luck that Firvona's spies have not discovered your identity in these last three days. I do not trust that luck to hold out any longer. Leave now, before nightfall. Go to Bayam. Tell them that the castle at Cortan needs aid." 

She looked up at me in surprise. "How?"

"I know nothing for certain. I thought Firvona had brought a surprising number of men in their honor guard when they visited. I fear now that they were investigating the castle for weaknesses. Cortan is barely defended. All her troops are here."

The errand boy arrived with Nisrita's armour. I turned my back on her as changed between the wagons. She emerged a calvalry man. She handed me her white robes with a thick black rope coiled over the top. She had cut her braid.

"I will do my best. What else do I need to know?" I approved of her calm. I hope that it will serve her on the road.

"Remove the emblems of Cortan and the Tower from your belongings. Stay away from the road we came along. Firvonese troops hold most of the lands we have conquored. Bayam is held by our forces. Avoid people as much as possible. Sergent Nicose will go with you. He will keep you safe if you obey him. I will show you the best route when he arrives."

Nisrita obeyed, then knelt to pray for the few minutes before Sergent Nicose arrived. I stepped away to grant her some privacy, then contemplated my own sins. I was quite possibly delivering her from one death into the hands of another. I did not know what the Destroyer had planned for her. He had gorged on so much today. I hoped that he would let this one girl slip from his grasp. I could not protect her here. We would march again tomorrow. In the chaos of taking Turina, anything could happen, as it had happened today. 

I had to get her away. She is too precious to Cortan's Tower, she is too precious to Cortan. Duty and Duchy be Taken, she is too precious to me.  Even now, General Madriano and Master Alerio support my decision. Why does it still feel so wrong?

Sergent Nicose arrived, I showed them the best routes, and rode with them a mile to see them onto the right path. There was still light in the sky as they rode alone into the shadowy forests. It was less than an hour until sunset. 

I found myself unable to ride when I turned my horse around. My head spun and my throat tightened. I had done what I needed to. Nisrita was off to take care of herself, for better or for worse. There was nothing left for me to do. There was nothing left for me in this world. Makala was dead. I had just sent Nisrita off to an uncertain death, in favor of a nearly certain one. My world was shattered. I had no one left. I sat on the ground and gasped for breath against the grief crushing the life out of my chest. 

It was fully dark when I came to my senses. My horse had wandered off a short distance to graze. I returned to the brightly lit encampment of a celebrating army. It had no right to be celebrating at all. Cortan had not won the day. She had lost everything. I spit on the ground and took my anger out on my horses's flanks. 

I took my meal in private, and drank alone. I must have had enough drink to kill an ox. My sorrow would not drown. The more I drank, the faster it floated to the surface, bloated, red and angry. Makala had no right to be dead. Our army failed him. In spite of everything we did, we failed him. We should have known that there were mines underneath the battle ground. Mines cannot be dug in a day. How, by the Taker's poisoned teeth, did our intelligence not know that there were mines under the valley?

I staggered to where Jaquim drank to tomorrow's success. He was unwounded. He had no right to be whole. Not when Makala lay dead. Not one member of Cortan's army deserved his health when our duke lay dead. 

"Coward" I yelled at him. The sound of laughter stopped in his circle of companions. A dozen men turned to look at me. "This man before you is a coward and a failure." I yelled, pointing at Jaquim. 

"Go to bed, commander. You are not well," someone shouted. I saw a few men rise around me. 

I did not care. I was not interested in them. This was between myself and Jaquim. His blindness caused our forces to be cut off. If he had not been in charge of scouting that valley, if a better man had searched, who knows how many lives may have been saved. Perhaps Makala would have been among them. "Stand up, Sergant Lorzo." I continued. "Stand up and tell everyone how you were too blind to find the mines that claimed so many good men's lives today."

Jaquim looked at me calmly. He did not even have the dignity to be ashamed of his failure. "I will not, commander. Go to bed. We can discuss this in the morning."

I swore to myself that one of the two of us would not see morning come. I had no use for another day. Jaquim had no right to see the sun again. "Stand up, sergant. That is an order." Jaquim rose reluctantly. "Show me that you are man enough to deserve you place in Cortan's army."

"I will not strike a commanding officer, Commander Romino. We will discuss this in the morning." His passivity and repetitiveness only fueled my anger.

"Stand down, commander" I heard a few men say. I saw a circle of men close in around me. 

It was now or never. Morning would not come for Jaquim. I drew my sword and lunged for his unarmed stomach. My wounded right shoulder blade screamed pain as I did so. I saw blood spurt from Jaquim's body, then a second shock of pain from my shoulder. The world spun. I do not know who hit me. I must have blacked out. When I came to, I was lying on the ground twenty paces from the cluster of men surrounding Jaquim. It looked like someone was bandaging his leg.

I cursed my aim and staggered off into the darkness. My shoulder throbbed. I found my tent and fell onto my bedroll, praying to the Destroyer that I might join Makala in the night.
\end{document}
