\documentclass{article}

\usepackage{fullpage, verbatim}

%*****************
% Annotations
\usepackage{soul}
\usepackage[colorinlistoftodos,textsize=footnotesize]{todonotes}
\newcommand{\hlfix}[2]{\texthl{#1}\todo{#2}}
\newcommand{\hlnew}[2]{\texthl{#1}\todo[color=green!40]{#2}}
\newcommand{\sanote}{\todo[color=green!30]}
\newcommand{\egnote}{\todo[color=violet!30]}
\newcommand{\newstart}{\note{The inserted text starts here}}
\newcommand{\newfinish}{\note{The inserted text finishes here}}
\setstcolor{red}
%***************************


\begin{document}

Hello again, sweet child. Can I move you over to make room for your mother? I see you and your brother are sleeping in different beds now. It has been a long time since you have helped me with my midnight vigils, Makala. I slept better in my tiny little room in the tower. I wish I could have kept you there, Makala, but it was too small. I did not stay in the Tower all this time because I do not love you, darling. I hope no one has said that to you.

I noticed that you did not cry when Timmon left two days ago. Your brother did. He was heart broken, the poor boy. I suppose Timmon likes him better, doesn't he, because he is big and strong and boisterous. Don't be offended. You are my favourite, if I must admit it, my thoughtful gentle child. Do you know why Timmon left, my sweet? It is because your father came home yesterday. I know it must seem strange to you. Timmon is a married man, with a house and a child to raise. He has proved his worth to your grandfather, and, now that he is married, also reearned his trust. Oh, it is so complicated, Makala, these games we play. He may not have been able to do any of that with your father in Cortan. I don't know. Your father does not take much interest in military matters anymore, he is interested in the Tower, and the old magic.

Timmon left anyway. He's going to organize troops on the southern front. He said it would be better for him to get out of the way. I am so sad, Makala. I miss him already. I know I will see him on the southern front in less than two months. It just feels like those horrible days before you were born again, when I was alone with your father, and Timmon was nowhere to be seen.

Do you think Timmon is afraid of your father? It can't be. Timmon is not like that. I am afraid of your father, but you have nothing to fear. I will not be afraid if I have to protect you from him. Do you know what he did to me tonight, sweet child? He wants me to give you a brother or a sister. Well, he did not say as much, but that is what his actions amount to, isn't it. That's why I'm in here with you now. I could not go back to sleep next to him after that. 

There was once that I thought he cared about my gift, Makala. Otherwise, you would never have been born. You will be gentle with women when you grow up, won't you? Timmon will teach you the importance of respecting women, won't he? I don't think he'd ever dream of treating Sophia like your father treats me. Well, there are other reasons for that, but that's not the point. 

I think your father wants me to stop practising. He hasn't said anything yet, but he's only been here one day. He doesn't like that I've done what he could not do. I don't know if anyone in the Tower will be able to stop him, if he wishes to keep me from practising. I don't know that they will want to. They all revere him too much, even after what happened with Ezaro. He's still not back, you know, Ezaro. Its been nearly four months since the accident, and he still has not recovered. What will I do, Makala, if I am forced to leave the Tower?

What a bad mother you must think me. Spend more time with you, of course. Take care of you like you deserve. That is what your grandmother will tell me to do. I can't, sweet child. I'm sorry, but I can't.

I wish I could take you with me into the other room. Your father wouldn't like it. I don't like it in that room, and your face is so sweet and calm and beautiful. It would help me sleep if I could hear you breathing next to me.

\vspace{.5cm}

"Where are you going, girl."  Griswold asked the morning after he had returned.

"To the Tower, my liege, as is my habit" I replied, not looking at him, not pausing in my dressing. I would not let him intimidate me.

He watched me dress. Would other husbands give me this as a moment of privacy? I do not know. Makala had given me my own room. I have only known two husbands. "I do not like this habit," he said finally.

"I am sorry to displease you, my liege." I would not volunteer to stop my work to please him. My abilities were too imporant to the Tower. We would march soon. I can do the work of three healers. I am one of only a dozen people who can regrow a limb. If whe wanted me to stop practicing, he would have to fight the Tower for it.

"I am going out. I expect to find a more obedient wife when I return."

Griswold left. I finished my toilet, spent some time with my rowdy little boys and left for the Tower, trying to figure out a way to change my schedule to accomodate my duties in court, now that my husband wanted me residing in the Castle again. I would have to give up attending any evening lectures, that much was clear. I would probably have to cut back to only two patients as well. It was unfortunate, but the Tower had many healers helping our veterans walk now. It was the best solution.

Master Adele found me after my first patient of the day. I had started my day late. I asked her to walk with me as I went to my second charge of the day. I would have to skip my first break and shorten my other two today if I wanted to see everyone and be seated at the Duke's table for dinner. 

"You should be resting, Nisrita." 

"I know, Master Adele. I also know that I am acting irresponcibly with my resources," I added before she could give me her reprimand. "I wanted to talk to you about that. I think I will need to cut down to only two patients a day from now on, if that would be acceptable. I am sorry for the short notice."

Master Adele sighed. "Unfortunately, Nisrita, it will not be." I stopped. I had not expected this. I had thought that Tower would be able to find someone to cover for me. I started calculating how it would take me to heal two of my current charges. Over a week, unless... "I will have to ask you to stop healing altogether."

"Master Adele?" I asked. I need not have. I knew where this decision came from and why. 

"Starting now, I'm afraid, Nisrita." Master Adele looked as if she did not want to have to say those last words, but I could not feel sorry for her. Master Adele revered my husbad. She could have resisted him as she had stood up to the Duke to let me see my boys on their birthday. She chose not to. 

It would not do to get upset. If I was to be expelled from the ranks of Coran's Whilte Tower, I was left with nothing more than my role as a duchess of Cortan. I donned that persona. "Am I still to be in your tutelage, or is that over as well, Master?"

"I'm afraid it is, Nisrita." 

I gave her my hand to kiss. This audience was over. This suprised my old Headmistress. In the tower we go by the ranks granted us by our skills and acheivement, she was a Master, I not fully a healer yet. If I was no longer part of the Tower structure, I saw no need to hold to that convention. After a heartbeat, Master Adele recovered and bowed. "Send someone to gather my belongings in my room, Master." I said.

"Yes, your grace." 

By the time my room was packed, and my belongings sent back to the Castle, I had recovered an sense of calm and perspective. The Tower would not back me against my husband. There was another pillar of the Duke's court that had been willing to back my will over the desires of the Duke. I left the Tower and went to General Galderan's office.

\vspace{.5cm}


The next week crawled by. It is not that I did not love my children, but I could not live for them. No one can say that there has ever been a moment in my life when I did not wish to Cortan to the best of my abilities. The best of my abilities had to be more than entertaining the wives of barons and priests and commanders, listening to them drone about the Liri gods I had no reason to hate, or the Liri customs I had no reason to find scandalous, or the Liri fashions I had no reason to love. There would be a campaign soon, my service to Marsea lay in healing her wounded soldiers, not in gossiping about politics that did not concern me and that I could not effect. As for my precious sons, I quickly found that spending more than a few hours a day with them drained me far more than healing did. I grew testy and impatient, mistaking their childish demands to have the same crippling motivations as the more malicious ones put upon me by my husband or Duchess Cybeline. By my second day away from the Tower, their nurses sent me away. I filled my days as best I could with study and reading. Sophia visited with Eugenio, providing a buffer between myself and my sons. All of these were half measures against the crushing weight of boredom and defeat that I carried with me those days.

It was therefore a moment of great delight and relief when I found Ezaro walking slowly and unsteadily on the smooth gravel path of Cortan's castle garden, one afternoon. Makala sat intently digging up the bed of lillies, the damage he engendered unchecked by my lackluster supervision. Little Griswold poked at a tortise under his nurse's careful supervision. "Healer Ezaro," I called, "it is a pleasure to see you. How long have you been at the castle?" 

The skeleton of a man tottered towards me on his twig like legs. His face was drawn and shriveled, dark bags hung under his eyes. His clothes hung losely on his emaciated body. When he bowed, I could count the bones in his spine through his thin white cotton shirt. "I arrived this morning, your grace. Your husband has been kind enough to invite me to finish my recovery here." His voice was weak and hoarse. He kissed my hand and wobbled as he tried to stand again. I grabbed his elbow to steady him. I could feel his bones slide against each other under a layer or skin. I fear I must have bruised the frail man in my eagerness to help.

"You are most welcome, healer. Please join me." What had he done to himself? He had acted foolishly, recklessly and irresponsibly that day nearly four months ago. Had less time interevened since I had last seen him, I would have been very explicit in explaining my opinions of his actions. The months had worn down my anger. I could do nothing but pity my husband's skeletal guest. "Are you here to continue your studies with my husband?"

"I am, your grace, though I am here primarily to recover. Duke Griswold wants me fit to march in four weeks."

"I see." I said, and contemplated this news. It was good news for Ezaro, this was undeniable. Griswold would not have him partake in their mad experiments if he needed to put meat on his bones. I was undeniably jealous. I would not be going to the front in a month. Carlotta, Sophia, even Ezaro would go. I would be left behind again to be wife and mother and nothing more. General Galderan had not yet been able to pull off a miracle for me, though he did assure me that he would do his best. His best may not be enough, I knew. My impatience for news ate at me. 

"Have you had much opportunity to study during your recovery?" I asked in an attempt to find something to speak about.

"No, your grace, I have been too ill. My meeting with your husband this morning was the first I have seen anyone beyond my nurses since my collapse."

I told him what I knew of the world. I started with gossip from the Tower, and then about the success of his teaching methods. Over the course of less than nine months, Cortan had gone from not being able to regrow lost appendages to having a dozen healers in this Tower alone that had the skill, and possibly another dozen scattered throughout Marsea. Master Adele had coordinated the efforts of several healers in Cortan to try to replicate my original feat. Only Ezaro had succeeded. Whatever condemnation he deserved for his recklessness, he also deserved to know the gift he had given Marsea.

I told him what I knew of the campaign, details learned mostly from Timmon before he left. We were marching early this year, leaving while it still rained in winter, instead of waiting for it to cease as we normally do. The reason for this was simple. Sama Gorou, the crown's Liri guest wished to return home before the summer rains visited Lir. If we could secure a path to Lir, such that he did not have to pass through Niev, a Liri ambassador would walk on Marsea's soil by the Day of Unions. My husband was to accompany him at least to the border. There was no other reason the great Griswold would leave his legions of Deyalorn's healers for a minor military matter.

General Galderan had his eyes set on Niev's southernmost stronghold, Escasaine. If he could hold that this year, all of Niev would fall within two, perhaps less. After it fell, the newly accquired Nievian territories woud be held for Makala, when he reached majority. General Morel had would stay in Cortan's new protectorate region advising Duke Erfat. This created the need for two new generals, one for Cortan, another for the new territories. Timmon still had his eye set of commanding the armies south of Cortan. Whatever he felt towards the triumverate, I think he believed that Makala had returned in my son's body. There were too many similarities between the infant and the man. It would give him great pleasure to gift Makala's reborn soul with the lands he had originally hoped to gift his lover. 

Ezaro and I talked until he tired, that day, and on others. Sometime Sophia would join us, while Eugenio played with the boys. It broke the blankness and monotony of days spent doing nothing but entertaining wifes of Duke Ergino's courtiers and watching my sons toddle. 

After two weeks, Ezaro was strong enough to practise. He told me his new theories about controlling the gift. He had found a way to be gentler with the fire, though he had not had the opportunity to try it on anything but animals. I poured hot wax on my wrist for him to demonstrate, and then asked him to explain what he had done. He was infact gentler. It was not what Carlotta did instinctively, but the pain I experienced in his hands was a far cry from what most healers inflicted. His explanation was beautiful, and intricate, and far beyond what my poor skills could ever hope to accomplish. I wondered at his technical abililities. I felt like a donkey working next to horse trained to dance. Ezaro admitted that because of the difficulty of his discovery, it was not very practical. It was also inefficient. He wanted to explore it. Griswold, and their work with the mushrooms, had opened his eyes to a whole new world of possibilities. He had given him a whole new language with which to talk about our abilities. Watching Ezaro work his wonders and listening to him describe his lessons, I could see the man I had met on the way to Szarvis, the great brilliant Griswold of myth. Ezaro's devotion to his tutor almost made me glad to have brought him back from exile.

"I see you are practising again," Griswold said one morning about a week after Ezaro had healed the burn on my wrist. 

"I am, my liege." What did he expect me to do after cutting me off from the Tower then brining Ezaro to the castle, a guest in my own house. When he said nothing more, I offered what I hoped might be an olive branch. "I have enjoyed hearing the ideas you discuss with Healer Ezaro second hand from him. It would be a great honour if I could attend these discussions myself."

"Your duties as a wife and mother come before your desires to become a healer." 

I licked my lips and ignored the reference to the fact that I had not yet earned the title of a healer yet, not having completed my training with Master Adele.  "I beg your pardon, my liege. I do not think I am neglecting my duties either to your, or my sons." 

Griswold laughed at me."You have no shame, Nisrita, if you can still ask that after your devastating performances on our journey to and from Skalsbad. It is not that I will not teach, you. It is that you cannot learn. You are hamhanded and clumsy with your art. You rely too much on your power, and pay no attention to your need for skill. As I said then, I will not take a student who will become an unremarkable, forgotten healer once she is past her peak power."

I sat and took the criticism calmly. He had said as much to me before, as had others, though more gently. I had not hoped that my new acheivement would have done anything to dissuade him from this opinion, but that my ability to learn Healer Ezaro's form of regrowing a leg would had convinced my husband to at least reconsider his opinion of my abilities. 

"This came for you, today." He put an opened letter in my hand. "I will send you home the moment you carry my seed. I will not have you risking your life."

My husband left me. I opened the note to see that General Galderan had secured a position for me among the ranks of healers at the stronghold of Rialot, supporting the troops securing a passage to Lir. I took in a deep breath, kissed the paper with the General's promise on it, and ran to play with my boys. There was only a week left until I would march with Cortan's armies to serve in this campaign. I could endure this boredom for another week. I would be useful to Cortan again. I could endure a lot for that honor.

\vspace{.5cm}

It is not uncommon for women of the White Towers to marry men of the military. The Towers, especially in the border duchies, are dedicated to the service of the military before all else. It is beneficial to both institutions to forge close familial ties. At the same time, the Towers cannot afford to keep all wives of military men at home. More than twice as many women are born with the gift than men, most of whom chose to marry and serve by giving Marsea more gifted children. During a time of war, the Tower does everything in its power to convince trained married women to serve, including women who would serve with their husbands. Lysene served with her husband from Cortan's Tower. I served on the road to Turina, while Makala explored the Velta, Sophia served while Commander Lazarro on the same field.

The Tower takes certain precautions to protect the women who chose to serve in the healing corpse. Men and women, even those who are married sleep in different tents on the battlefield. Mixing between the soldiers and the healers is unusual, primarily because soldiers tend to fear us, but also because it is frowned upon by the military and the Tower alike. The healing corps does not have the resources to spare to save the life of a pregnant practising woman, nor the inclination to monitor the status of every woman's womb. As a result, the occaisional accident does happen. Generals and Heads may be able to keep lovers from meeting in private on the battle field, but they can do little to intervene in the sacred bond between husband and wife. However, a wife who can heal is a valuable enough resource for the country that most men stay separated from their wives while on a campaign. Even a man of Commander Lazarro's demanding needs let Sophia live amongst the Healer's Corps on the march to Turina, sparing her his hawklike inspection of her actions during the last six weeks of their marriage. 

Griswold does not consider himself among the ranks of most men. I spent my days travelling to Rialotte in the Healer's Corps, and my nights in the Duke's tent. No one dared deny the duke his pleasures, as no one dared deny Makala his last night with me. There must have been whispers amongst the army, but the whispers never reached my ears. It was not until I reached Rialotte that I realized how much of Griswold's actions were motivated by his desire to keep me separated from the rest of the army, beyond the more obvious desire to impregnate me and send me home.

Rialotte was a large village on the road between Bayam and Escasaine. Marsea gained it last year, and had built it up somewhat to provide a safe base for healers, supply trains and a large infirmary. To the east ran the road to Escasaine, along which a quarter of Cortan's forces departed the morning after we arrived in Rialotte, including Carlotta and Ezaro. To the south lay the proposed road to Lir. The rest of Cortan's army waited in Rialotte, awaiting orders and men from Turina. Two days later, these arrived, and the small base emptied, leaving me with a corps of healers and the requisite guard to protect the town. My job was to tend to the wounded sent back from the lines, to determine who could be healed quickly enough to send back to the lines, who was well enough to travel back to Cortan or Baya for treatment, and who needed to be seen to in the temporary infirmary I was to call my home for the next several weeks. It was not a glorious job, but it was where I had been assigned. It chafed, but I did not question it. I dared not while Griswold was stationed at Rialotte.

I was reminded, in the relative quiet of Rialotte after Cortan's and Turina's troops had left, of the fact that some still called me the Maker's daughter. I served under the title of Healer, though I was not entitled to that rank amongst the country's Towers. I had been asked to leave without completing my training, one of many wives who chose to marry and settle in their husband's houses early in their prime, unable to heal or earn for her family without the supervision of another healer. None of that mattered in Rialotte. Amongst the guard, I was still the Maker's daughter. I needed no more rank than that of the girl who had given their comrade back their arms and legs. They did not see the woman I had scorned during my childhood in the Tower: half a healer, simply a mother and a wife. They saw a duchess and a miracle worker walking among their ranks.

Mostly men of low birth, Rialotte's guard respected me for my station, but did not fear me for my skills, as they did others of my kind. In exchange for their respect, I gave freely of my time. Let the other healers maintain their aloofness. The maker's daughter served the army. She would mingle with them as well. How could I feel otherwise, young as I was, giddy with the absense of my husband? I found myself drunk on the praises of these men who had all heard of my skill. Their captain, a eunich named Scuda, aged about forty, a tanner's son from the western hills of Cortan, took it upon himself to chaperone me amongst his men. I spent my days with my kind, and an hour or two after dinner with the ones I served. I drank with them, I listened to them speak of their lives. They were sons of goat herds and publicans, a few claimed ancestors who had come to Cortan with Griswold, sons of families that had long made their living, if not their fortune from Cortan's armies.  I listened to them boast about their deeds in past battles, I did not correct them when they exaggerated my abilities. When I beat them at darts, I told them my brief training with the black riders. 

The next morning, I trained with them. I picked up a weapon for the the first time in over two years. I was rusty, undeniably so. I was smaller, faster, and smarter. They had trained their entire lives. I left the field bruised, and bloddied every day, but never seriously wounded. These were not Timmon's boys training with me, but men of Cortan's army, disciplined by nature, more so under the watchful eye of their captain. They fought for the honor or sparring with the duchess, and the pleasure of engaging with a woman with my curious patchwork of skills. I fought for the sheer pleasure of feeling my body move and dance in the company of others more skilled, more powerful, more able than I. For three weeks in Rialotte, I was a girl again, not a mother, not a wife: training, healing, mingling freely and comfortably in the company of men, dancing in the mornings to the flattery of strong young men, a chaperoned dance where nothing would happen, but which left me alert and burning with the knowledge that every possibility lay open. 

After three weeks at Rialotte, a second army of troops passed through the town, wearing the colors of interior duchies, along with the Liri guest, Sama Gorrou. The southern front of this campagn had all but accomplished its goals. The path to Lir had been established. It only remained for Duke Griswold to accompany our guest back to his homeland. It was time for Mersea to turn its attentions to Escasaine. It did not take much effort on my part to establish my transfer to the more active front. The crown expected the battle to be long and bloody, Niev's abilities to prey upon Marsea's weaknesses growing with each year we march on their lands. A strong healer would better serve Marsea's armies at the active front, where I can save lives that others cannot, than in a field base where I merely can heal more men than my companions. I left with Marsea's army, turning my back on Griswold's return, eager to mingle again with healers hailing from Towers across my great country.

\vspace{.5cm}

"What are you doing here?" Carlotta hissed into my ear soon after I had settled my posessions into a bunk of the main army's healer's corps.

"Serving, Carlotta, as you are."

"Duke Griswold does not want you serving in this campaign. Why are you continuing in your madness of thwarting your husband at every turn?"

"As far as I know," I replied with an air of calm authority I usually reserve for Duchess Cybeline's hearing, "Duke Griswold has little to do with the assignments of healers to armies. The generals and commanders have that authority. They saw need of my skills here, so I came."

"Stop that." Carlotta snapped, which gave me no small pleasure to hear. Within the confines of the Tower, she is known for her even head, while I am known for my impatience. "You are not the naive girl you pretend to be. This is not a game. You have no idea how difficult you have made life for Master Adele. Just the slightest effort on your part to smooth matters with Duke Griswold, and her program of healing amputees would not have had to stop. Now you wish to trouble the poor woman again by appear in her tutelage on the front. Do you think she likes facing your husband's anger?"

I doubted anyone liked facing my husband's anger. However, I had no sympathy for Master Adele. My old headmistress would protect me against the wishes of the non-gifted, but considered the non-gift weilding Griswold as one of her kind. If she willingly sought his company so much, then she did not need my actions to sheild her from his wrath. "This is a large corps, Carlotta. If Master Adele does not wish to take me, I will find someone who does."

"Who is it then? Who is it you are so desperate to see? Is it still Timmon? Do you wish to be with Ezaro? I saw you could not stop talking to him on the road to Rialotte, or do you have someone else in mind?"

I stared at Carlotta for a moment. Was that the breadth and depth of her imagination? Could there be no other reason for me to leave Rialotte? It did not matter. I did not need to show her what she did not understand. She was a skilled healer and a good friend, who was inexplicably angry at me again. "The Preserver, Carlotta. I came to serve him. Nothing more. Now if you will excuse me, I must find a Master to supervise my work." 

I left her for the groups of healers milling in the evening air. I immeresed myself as quickly and as deeply as I could into a discussion about internal bleeding in the gut. By the time dinner was called, the discussion had changed into one on the merits of Marsea's ban on the use of the old magic. The subject teetered on the edge of discussing my husband directly, but stayed away from him. It seemed that a group of Deyalorn's Masters had taken it upon themselves to pressure the crown for a change in policy. The masters of Deyalorn and the border duchies were eager for this ban to be removed. Those hailing from quieter interior duchies seemed opposed or ambivalent about the change. I put myself vocally against the old magic. I could not tell my audience that this was my husband's madness, a manifestation of his habit to play with other people's lives, as if they were no more than beasts to be learned from. My views did not make me popular with my colleagues in Cortan's tower. By the time dinner finished, I found myself, instead, surrounded by three or four healers from interior duchies. 

Carlotta eyed me darkly from a distance. She sat between Healer Ezaro and Master Adele, a position I had, until recently occupied in Cortan's halls. I realized then, that Master Adele had brought Carlotta into the numbered few that knew of Griswold's work with Ezaro. She supported it, as Master Adele did. What need did they have of the troublesome wife who spoke against the goals they wished to help Griswold attain? I understood the trouble I caused Master Adele by appearing. I could not have her supervise my healing, even if she wished to tutor me again. For everything she had done for me in Cortan, I could not work for someone who supported my husband so blindly.

Master Alerio found me the next morning. The beginnings of a winter cold that had snared me soon after leaving Rialotte still had not released me. He found me stealing a ride on the back of a supply wagon, instead of walking with the rest as I should have been. 

"I noted that you did not ask Master Adele to oversee your healing, Nisrita," he said. "You should not put the decision off for long. We may come across fighting at any moment."

"No, Master Alerio." I had not decided what to do yet. I had thought of speaking to the Masters from Gissal who agreed with my position on the old magic last night, but I could not bring myself to speak to anyone outside of Cortan's Tower. I had been raised there. I always thought I would earn my healer's rank from one of the masters that had boxed my ears for breaking into the animal pen, or had taught me to perform my first cesarian. To look elsewhere felt disloyal. 

"Do you have anyone in mind?" Master Alerio asked when I said nothing more.

"I had considered Master Renata of Gissal, but I have not spoken to her yet."

"What do you hope to learn from her, Nisrita?"

"I beg your pardon, Master?" The question of being trained had not entered my mind. I was too young to be a healer yet. I had finished my formal education, but I had not completed the years of apprenticeship necessary for the rank. I had not come here to be educated again, I had simply come to serve. For that, I needed a Master to supervise me.

"A student should pick a tutor they can benefit from. Otherwise you are only wasting your years of apprenticeship. Master Givandi's arrival from Deyalorn has relieved me of my duties heading this Healing Corps, leaving me with some time on my hands. See me tonight if you wish to pursue this further."

He left me then, to my decision, as if there were a decision to be made. I sat on my hands to keep them from forming some unladylike jesticulation in my excitement, bit my lips to keep the grin off my face, and kicked my heels on the back of the wagon in an extremely joyous, girlish and undignified fashion. Head Master Corino was an old man. He would leave his position in a year or two. He had picked Master Alerio to replace him. This decision could be revoked by the other Masters, of course, but Master Alerio was well liked among his colleagues, there was a good chance that he would become head. I had gone from a rankless gifted of Cortan to Master Alerio's student, with hints of remaining in his tutelage until I could attain a healer's rank. All bitterness I haroured at Cortan's Tower dissolved. I had not, it seemed, been categorically cast off by them at a word from my husband. I stopped bitting my lips and grinned for the world to see.

A small band of black riders joined the group of marching men that evening. I heard the thunder of their hooves behind me as they rode up to the army down the long trail of flattened grassland and mud we leave in our wake. It was a beautiful, uplifting sound. I had once childishly wished to ride with them. I turned to watch our scouts return. They were not lead by one of their own. It took me longer than it should have to recognize the mailed figure in a black cloak at their head. Timmon had always envied Makala's position with the Black Riders. I wished Makala could see Timmon riding at the head of a group of his former colleagues, even if on a reconaissance mission. I wondered if he could have been prouder than I at Timmon's recovery.

Within an hour, we had stopped, and I found my duties of helping pitch the women's tent interrupted by the presence of a gifted Black Rider. 

"May I help you, soldier?" I asked. A gifted fighter in the healer's corps at night usually meant a fight or accident among the men. 

The soldier grinned, then opened his mouth to waggle his tongue. "Ernesto!" I cried, pulling him into a circle of light. "Forgive me, I did not recognize you in the dusk."

"I wanted to thank you in person," his rasping voice replied.

"Thank me? For what?"

Ernesto stuck out his tongue again. "I asked someone in Turina to try your methods on my tongue. You see the results, though I would have appreciated if you could have found a less painful method." 

He continued grinning. I blushed. I had not thought I would see this man again, far less hear him speak. "Your voice. What happened to it?"

"I had not used my voice in nearly ten years. The healer said that my voice box has shriveled."

"Shall I cut it off and grow you a new one?"

I watched Ernesto's eyes widen in terror and his hands wrap themselves protectively over his throat before he registered my jest. Then he grinned again. "The White Torturers are reported to be gentler than you, Nisrita. Have you considered joining their ranks?"

"No," I said shortly, spoiling the joivial mood. There are some matters we may jest about among our own kind. Under different circumstances, or a different healer would have taken it better. I could not.

"What have you done since you left Cortan?" Ernesto rasped. "I have heard many reports of your adventures."

Adventures indeed. "Then you will understand if I do not wish to discuss them." This interview was going poorly for a joyous reunion of two friends. I turned to Ernesto with questions about himself. He had joined the Black Riders of Turina, a position he had easy secured given the strength of his gift and his years with the Hundred Horsemen. He was married now, with a daughter. His wife was weakly gifted. He had hopes for the child. We spoke of Timmon, and his uncanny ability to sight a trap from looking at a good map, or predict an ambush from knowing the lay of the ground. Ernesto told me of a half dozen instances on this campaign and the last where Timmon seemed to read the minds of Niev's generals. 

Eventually, Ernesto left, with the promise of visiting again, and I went to keep my appointment with Master Alerio. I found him in the company of Master Givandi, and other prominent masters from across the land. I approached, suddenly self concious, feeling very much an untrained girl in the company of these wisend and powerful healers. Master Alerio bade me sit, and introduced me to his company as his student before I had a chance to give him my acceptance. Four greying heads turned to me in unison and asked me about the inspiration of my regeneration work. I had not expected to give an impromptu lecture on Horatia Joris's writings, especially not to a company of men who had not studied the craft of child birth and complications to labour as women in the Tower do. I could not falter, with Master Alerio watching me. I had become his student. This was my first test. 

I ate in the company of the Masters, listening more than speaking when the conversation veered from healing to the battles the army had been facing, and keeping my self quiet when it turned to politics between the Tower and the crown. The present company seemed split on the question of old magic, Master Alerio seeming undecided on the issue altogether. I left the masters early, while they discussed differences in the philosophies of the founders of Deyalorn's Tower. I was tired, and there was one more person I wished to see before retiring.

"I was starting to wonder whether I needed to send a formal invitation, duchess." Timmon said when I entered his tent. He dismissed the men he had been consulting with over a set of crude maps. His face looked worn and thin, more than I would have expected from the strains of travel.

He offered me a seat, then rose to gather refreshments. I took the moment to survey my surroundings. I was in a well lit, well appointed tent, closer to what I would have imagined a general's quarters to be. Besides the small desk and shelves containing maps, papers, and other necessities, the tent had room for a bed, a small fire and space for several pages to sleep. The idol of the triumverate sat where it should by the entrance to the tent. I suspected it was there to please his generals and for the sake of his pages. Soon after he married, Timmon started wearing a red string around his neck. I never asked to see what lay on its end, it was some sign of devotion to Liri gods, I guessed. I had seen Nubo wear something similar. Even now, when he had left his wedding chain behind as is the custom of many military men, the red string remained, protecting him, I hoped, in battle. 

"What do you think?" Timmon asked. He had been watching me survey his accomplishments. 

Did he need me to tell him I was proud of him? "Your men think you possessed of a second sight that lets you read minds and speak to maps. Is this how Marsea rewards its wizards?"

My comment failed to make him grin. "Not a wizard, simply in charge of this army's eyes and ears. That requires space and a modicum of peace. The Hundred Horsemen do most of the legwork."

"Commander, what is wrong?"

"Nothing, duchess." He paused and changed his mind about lying to me. "The next few weeks will be difficult. I am glad you are here. We will have need of your services. What news do you have of home?"

I told him what little stale news I had. I doubted I had anything to say that others had not told him already. Sophia was pregnant, Eugenio had learned why not to bother a hornet's nest, and was recovering when I left. The Duke and the Duchess were well, as were the young dukes. The Liri guest Sama Gorou had arrived at Rialotte when I left, and the eastern branch of this campaign was coming to an easy conclusion. 

"Anything else?" He asked when I finished. I shook my head. "I am disappointed, duchess. You have not mentioned that you are no longer with the Tower, due to your husband's wishes, that you would not be here if it were not for General Galderan's, that you have taken up training with weapons again, a fact which gives me great pleasure, or that your husband would rather you were in Cortan being more of a wife and less of a soldier."

"You are having me watched!"

"Cortan has commissioned me to watch Niev's movements and keep Marsea's army safe. You will not begrudge me using a tiny sliver of those resources to ensure that my duchess is safe as well?"

"Timmon," I protested. He grinned. I could not be angry at him in the face of his impish joy. 

"No one else knows, duchess." He said more somberly. "Not the Duke, not your husband. I beg your indulgence on this."

I prentended to consider his words silently for a time. I could not be angry at him for this, but I could not let him think I would forgive so easily either. "You should reprimand your spies, commander. Your news is old. I am with the tower again. Master Alerio has taken me as his student over my husband's wishes."

"My heartfelt congradulations, dutchess." I beamed at his joy and the memory of Master Alerio's interview from the morning. "Anything else?"

He looked at me patiently. He would wear me down with that look, as he had countless times before. It was better to give in and be done with it. "He is much the same as ever, Timmon. I do not wish to discuss it," I snapped.

Timmon looked disappointed, whether at me or my husband I could not tell. "I am sorry, Nisrita. We do not have to discuss this. It is good to see you again."

We spoke for a time of his dealings with the old Hundred Horsemen, and the joys and travails of his new position, before I retired for the night. Timmon seemed happy. I had to admit, barring memories of my husband, I was happy as well.

\vspace{.5cm}

The next few days passed quietly. Master Alerio gave me a wooden box for my talisman, and a list of tasks to perform. The box would distance the talisman from my body, restricting the ammount of fire I had access to. The point of these excersizes was to force me to use less brute force and more skill in my art. It was exhausting and frustrating. It took many times the effort to work with the thin stream of fire the box permitted me access to. I would spend my day unable to heal a simple puncture wound on a rabbit, and go to sleep exhausted as if I had been healing all day. 

Carlotta and I made our peace. With Master Alerio's assurances that I would be given a healer's status as soon as I returned to Cortan, my jealousy of Carlotta's friendship with Master Adele ebbed. I spoke less vocally against the old magic, also upon Master Alerio's urging, and Carlotta's anger ebbed. There was, as Master Alerio said, nothing I could do as a lone voice against my husband, and the Masters of Deyalorn. It was better for me to continue my work quietly and effectively. There would come a time when I would have no choice but to take sides on this matter. Why waste my efforts and strength until that point came.

Then the trouble that Timmon had warned me about came. Our forward movement faltered and then stopped. We found a defensible position in some low hills, and made camp. Marsea's generals are not in the habit of telling the healer's corps much of the tactical decisions of the men we serve. Perhaps Master Givandi knew, as our head, but that information was not passed down through the ranks. We set up our infirmary, and set to work. 

"They have been harring our rear," Carlotta said one day, joining me by an unconcious page. I handed her the leather straps and she finished the job of securing the boy to the table. I only saw the worst patients. That was the curse of my power. I could save lives that others could not, therefore, I never saw the broken shoulders or sliced thighs that heal within a few days. 

"How do you know?" I asked, running a thin sheet of flame through the boy's innards to check for injury while Carlotta manually checked the boy's bones. 

"There are advantages to being promised to a general's son." I could hear the smile in her voice. I envied her ability to keep a distance from her patients. It made it easier to practise her art. 

"There isn't any internal bleeding." I announced, and turned my attention to the head. "What did Sergeant Madriano tell you?"

"Nothing much," Carlotta admitted, turning the face gently to examine the skull. "I understand as little of his art as he does of mine. We may have to turn around soon, but that is just a rumour." It was a hidden barb. Carlotta thought my attempts to engage Makala and Timmon in my art foolish, like teaching a fist to walk. I did not agree with her. They had, after all, both invested time and effort to teach me their art. "Look at this, Nisrita. Some idiot healed where he hit his head. He did a very thorough job of it, I nearly missed it in his hair."

I sighed. There was nothing we could do for this boy. For all our powers, we have not been able to unlock the mysteries of healing the brain. "There is no bleeding, we must leave him to the Preserver to decide."

Carlotta walked with me to the next patient. We worked in pairs when we the demands of the wounded did not dictate otherwise. For the more common injuries, this allowed two healers to work on a man at a time, returning him to the field faster. For the more demanding cases, it allowed me to evaluate a patient's needs faster. I would take the wounds that others could not touch. Carlotta, with her gentler touch would tend to what secondary wounds she could without overwhelming the patient with pain. 

"Distance, Nisrita. You cannot take each patient so personally." I shook my head at her chastisement, knowing her to be right. 

We turned to bloody mass on another table. There was a lot of small superficial damage that made the man look worse than he would have been if those were all that ailed him. However, he had severed an artery in his leg, which pooled blood slowly onto the table. The bleeding had been staunched, but if it was not healed quickly, he would lose his leg. The soldier was still concious, bleery eyed and muttering incoherently. "His gifted companion saved his life," Carlotta said as she checked his bindings.

"I'll mention him to Master Givandi when this is over." The man cried for his father when I checked for internal bleeding. He had been badly injured, but whoever had tended to him on the field had been thorough. Blood was leaking slowly from his spleen, it was badly bruised. This man would suffer a painful week, but he would live. I plugged my ears with wax and sat down on the chair next to the table. The infirmary had filled with sounds of anguish. I found it distracting. I closed my eyes to the world around me, and started on his leg. 

Carlotta tapped my shoulder when she felt his pulse weaken. He would not lose any more blood from his leg, but I had neither finished with his insides, nor sealed the wounds from infection. "Can I stablize his spleen?" I asked. She shook her head. I trusted her judgement. She had been monitoring his condition while I worked. He would have to risk a rupture in patched organ for few hours until I could  return to him.

I stood up abruptly to attend to my next patient, and had to steady myself against the table. Carlotta grabbed my arm. "Are you alright, Nisrita?" 

I had to be. "I've had a cold that I haven't been able to shake off these past few weeks. Its nothing serious." 

Carlotta looked at me gravely. "Turn in early tonight?"

"I promise," I said as we walked to the next table.

I kept that promise that night and for many subsequent nights. I found myself easily tiring. It worried Carlotta and Master Alerio. Carlotta even asked at one point if I thought I might have another nervous episode from the constant sight of death around me. I did not think so. It was not death that disturbed me, it was my inability to do anything in the face of it. That was not the case on these high plains. I lost patients, but I was constant busy. I was doing exactly what I ought to be. 

Over the course of the next few days, our infirmary filled beyond capacity. We could not send our wounded back, neither did we receive replacements for the injured we held. The decision came that the army would turn back. We set up our wagons and began the slow crawl back to Rialotte. A healer's wagon is a slow moving infirmary on wheels. Men are strapped to healing tables three high, six on either side. There is a narrow aisle for a handful of healers to stand and work. It is not very comfortable for either the patient or the healer, but it is the only way we have to get our wounded to safety. Under normal circumstances, only a few healers accompany our wounded back to a camp. They moved slowly, were heavily guarded and the healers returned with the supply train returning to the line. 

These were not ordinary circumstances. The army moved slowly to accomodate the needs of the wounded. The healers worked as we had before, except that we did not run a shift through the night. The wounded that could walk, did. We healed them when we camped. Those that could not walk we healed during the day. The journey was harder on the healers than it was on the non-gifted. We reduced our activities as little as necessary to accomodate the strains of travel. If we could not get fresh troops to the line, the wounded must be healed to serve again. 

A sense of fear permeated the army. What would happen to us if our supply lines failed. I had joined this branch of the campaign two weeks ago. We were not yet at the point where our food would run low. At this rate of travel it would take another two weeks to reach Rialotte. The men braced for that eventuality. Marsea's men were fearless before swords because they depended on the White Towers to save them from the Destroyer's claws. They did not like to play with death by illness, starvation, or other ways in which we could not help. Something in the men's mood must have effected me. I found those nights always too short, my bed always to hard. I awoke every morning exhausted and uncomfortable. The malaise passed as the day drew on. I took the evening shifts of healing, helping the men off the uncomfortble stacks in the healer's cabins to work on their wounds in the vaulted healers tents we errected every evening. 

It was almost sunset in the healer's tent when I heard the thunder of horses approach the camp. Another group of wounded men, I thought, and finished healing the bruised bone in my charge's elbow. He would be fit to weild a sword after a night's rest. Healer's behind me set up healing tables and cleared the way for the new comers. We had small groups leaving the main army every day and clearing the path for the march forward. So far, we had not faced any serious resistance to our retreat, though we had heard rumours of a need for another battle at or near Rialotte. Some groups came back worse than others. We did not lessen the burden on the wounded wagons, for every man I made able to walk, I found at least another in urgent need of care.

What I saw come through that door took my breath away. I had not realized that our Black Riders had left that morning. Even if I had, the sight of that endless stream of men covered in black and red made me lose courage. What had they faced? Each man I saw had several broken bones, and injuries that suggested that they had been in close, unmounted hand to hand combat. I stood, frozen by the sight of so many mangled bodies, unsure of where I should turn first. I saw Master Alerio enter the tent. I went to his side to examine the body he stood over. His face was pale and ashen, I could barely find his pulse. His body oozed blood in several places. Someone had tried to staunch his wounds, but they had not done enough. 

"Can you heal him, Nisrita?" Master Alerio asked.

"I will try, Master." I gathered myself together to stop his bleeding, everything else could wait.

"Stop, Nisrita." He said sharply. "That is not good enough." I looked at him. What could I do but try? "There are two dozen men in here that need strong healers. We do not have enough to save them all. Can you heal this man? If not, I will take you to another."

I pinched the skin of the man before me, and watched the bit of skin between my fingers slowly return to its original form. He had lost too much blood. I shook my head. I followed Master Alerio through the tables of dying and injured men, grateful for his experience in judging the severity of injuries quickly. I could not tell from among the dying who I should tend to first. I hesitated when I saw Captain Dielo moaning on the table, his right hand missing two fingers. His eyes caught mine, and pleaded for his life. "Leave him." Master Alerio ordered curtly. "He is still concious. Others need you more."

I turned away two more men before I found one I could save. "Stabilize him only. He does not need to be well enough to be moved tomorrow," Commander Alerio said. I did as I was told. Master Alerio stayed by my side, monitoring their conditions, making suggestions, and healing beside me as he saw fit. Working next to him was different than working next to Carlotta. He was an old man, his power was weak. He could not tend to a broken rib while I healed the punctured lung underneath. However, he could walk me through a man's body, and lend his strength to mine at the exact moments where it would do the most good. I would learn a lot from him when I returned to Cortan. When we finished, the Black Rider still breathed with difficulty, but he was no longer drowning in his own blood.

Then Master Alerio took me to three more men who would not survive the night without attention. During the third, he left my side, and told me to rest for a while. I gladly agreed. The evening's task had drained me physically and mentally. Without the need to listen to Master Alerio's instructions, I plugged my ears and continued to work. After I had sealed the flesh where his ear would have once been, to keep it from being infected, I sat by his bed, and took only a mouthful of the food brought to me by a page. I would eat when I was finished. I could still tend to two or three mor men tonight. I heard a commotion outside through my plugged ears. I did not care to know what it was about. The dying and wounded around me had sapped me of my curiosity to know what bothered the healthy. 

Master Alerio had not returned an hour later when I had bolstered my courage and regained some of my strength. I went in search of my next patient. At this point, the three men I could not help had been moved to a side of the tent. I saw a pair of healers move a fourth to their ranks. I made my way over to them, annoyed that I had not been led to this man earlier. When I saw his face, I could not forgive Master Alerio his decision. Timmon lay opened on the table. I could see the dark red flesh of his stomach glistening beneath his muscles, and the black of his liver, lying above it like a half discarded blanket. He was no longer bleeding. I could see the patched tear in his intestines where half digested food had once escaped. Someone had cleaned his insides, staunched his bleeding, then given up on him. Likely two or three people. I sat by his table and wept. 

I felt a shadow over my shoulder. I dried my eyes and looked up, expecting to see Master Alerio. Ezaro looked down on me instead. "Will you help me heal him?" I asked. 

Ezaro nodded. "Have you eaten?" he asked. 

When I shook my head, he put a small flat wooden box in my hand. "What is this?" I asked.

He put a finger to his lips and unlatched the box. Inside were a dozen small compartments, each containing several flakes of black dried mushroom. I snapped the box closed and gave it back to Ezaro. "That is not the help I wanted."

"Listen to me, Nisrita." He said quietly. "I have figured out a safe dose. You have not eaten. Take the contents of each compartment every hour. You will not be able to access the old magic. It will increase your strength, and you will not tire. You could heal all four of these men before you. No one need know."

I looked at the life giving box before me. I wanted to believe him. I thought of the small flakes of mushroom that I had pumped out of Ezaro's stomach six months ago. Is this what he had been doing? I thought of the skeletal man I saw months later in Cortan's garden. I looked at the men before me on the table, and made up my mind. Preserver forgive me for the calculation I did at that moment. I will never know if I was correct. "No, Ezaro. You killed a man with this method. I will either heal lucidly, or not at all."

"I must have consumed a bad mushroom. Occaisionally, one finds one whose effects are... unpredictable. I will be here to watch against that possibility."

I looked at the men again, at Timmon dying in front of me. I fingered the box. I could save them all. No one need die today. "Are you lucid?" I asked Ezaro. He nodded. "Let me see your eyes." Ezaro let me inspect him. He did appear free of the mushrooms. "Do you know how long it has been since Timmon was healed?"

"About two hours. You cannot work on him for at least another."

I took in a deep breath. I needed some fresh air, if I was to do this. I needed some time to gather my courage. "I will return in an hour then."

"I will wait for your return," Ezaro said. He took the seat I vacated to keep vigil over the dying. 

Outside the tent, I saw Master Alerio arguing with my husband. "And here is my wife, Master. This discussion is over," my husband said when he saw my leave the tent. "Duchess, make yourself ready to leave."

"I beg your pardon, my liege?"

"We ride tonight, your grace. Gather your belongings."

I did not understand. I looked at Master Alerio's grave face and the group of former Hundred Horsemen behind my husband. My tutor was outnumbered, an old scholarly man pitted against a half dozen young warriors. He had not returned to the healer's tent to keep my husband from taking me from the wounded. I gathered up what little courage I could find when faced with my husband's visage. "With all due respect, my liege, there are wounded I must attend to tonight."

"Rialotte has fallen." Duke Griswold explained with more patience that usual. "Niev's army is on the offensive. You are Cortan's heir. You cannot fall in battle. What will happen to your sons? We ride tonight."

I looked at Master Alerio. His face told me that my husband spoke the truth. I could not believe my ears. Neither could I leave Timmon behind to die. "We ride tomorrow, my liege."

My husband approached me, and put his arm protectively over my shoulder to lead me away from the crowd that would overhear our argument. I did not know how I would stand up to him without the aid of witnesses to curb his cruelty. "Do not question me, girl. We ride tonight. Get yourself ready."

"The only thing that will possibly ride from here tonight is my corpse." I thought of my sons at that moment. I could not raise them to be good men without Timmon. If they were to be turned into monsters by their father, it was better for all that I not be there to watch. 

I felt my husband's grip tighten on my shoulder. "You are mad, girl. Get your belongings together."

"You are right, my liege. I am the mad duchess. Do you doubt my ability to kill myself?"

He stopped then. He looked at me, long and hard, for the first time, it seemed, in many years, as not simply a woman. "One day then. We ride tomorrow."

I left him. I put as much distance and as many people as I could between myself and that man. I found myself near the kitchens. My stomach rumbled in protest. I had not eaten since morning. I was exhausted from healing. Ezaro had implied that the mushrooms would not work if I had eaten. I could not do it, I realized. I could not be another subject for my husband to study. He would not learn how to regain his gift by watching what the mushrooms did to me. I did not know if I could heal Timmon at all. His body may give out before I could do enough to fix him. It was not my strength that would fail him. I took a bowl of hot gruel with a few pieces of meat in it, and attacked it with the ferocity I wished I could apply to my husband.

I heard the sound of tired, shuffling feet walking towards me, and a familiar rasping voice ask "Are you eating alone by choice, healer, or may I join you?"

"Please, Ernesto. I welcome a friend after what I have seen today." Ernesto sat across from me with a steaming mug between his shaking hands. He looked as drained as I felt. He should be resting. "I am glad to see you back," I said. "Will you tell me what happened?"

Ernesto sighed and rolled the mug restlessly between his palms. "We heard last night that Rialotte had fallen. The scouts reported that the duke and the remains of our men from the base were headed here. Commander Romino took the five units of Black Riders out in the middle of the night. If I could tell you how he knew where Niev's men hid in ambush for the Duke, I would. We stayed off the road, taking a circuitous route through rocky terrain and copses to where we expected to find the remains of Rialotte's defenses. An hour before dawn, we destroyed an encampment of archers hiding in a copse overlooking the road. It was over swiftly, most of them dead before they realized we were upon them.

We saw the cluster of Niev's men waiting for the Duke just before noon. It was another ambush." Ernesto put his mug down in firmly, and coughed. He had spoken more than he was used to. When he recovered his voice, he continued angrily. "That is the only way Niev has fought us at all this campaign, ambush, after ambush in these dry dusty grasslands. We have not had an honest engagement all month. The prefects of Niev do not fight that way against each other. That is not how they hired the Hundred Horsemen to fight for them." When Ernesto resumed his coughing, I brought him another cup of hot broth. It was the drink for healers who had pushed themselves beyond their normal abilities. At that state of exhaustion a healer's body will not accept anything more substantial.

"What happened at the ambush, Ernesto?"

"We were at Tirasi dale, an oasis of green formed by a circle of rocks and caves scattered in the planes. Commander Romino waited until our scouts could see the line of Marsea's men from Rialotte on the road. He took a third of our company into the caves to provide cover for the duke's entourage. The rest stayed in the valley to see the Duke through. Those that went into the caves fared worse than those that stayed below. Men fought until they were cornered or too wounded, then jumped dozens of feet onto the rocks below, hoping to be picked up after the valley cleared. Those we sent to you are most of who remain of the men who took the high ground." Ernesto started coughing again.

The bastard, I thought. "The Duke shows his gratitude for Commander Romino's sacrifice by taking me away from here."

"It is not ingratitude, it is prudence. Niev's troops are scattered across these plains. The commander thought that they are collecting for an engagement soon. We cannot defend this patch of road. Commander Gri--, Duke Griswold knows these plains. You will be safer with him than here."

"Thinks, Ernesto." I corrected. "Not thought. Commander Romino was alive when I left him." I hesitated. Why, of all people, did this soldier's good opinion of my husband move me to doubt my certainty regarding his character? It did not change my resolve, I decided. "I cannot leave tonight."

Ernesto's eyes went wide. "I saw his wounds. There were many that wished to leave the commander behind. I-- I could not allow that. It was a miracle he survived the journey to this camp."

A miracle and Ernesto's patient tireless efforts, I was certain. The man had spent more of his energies saving a dead man than he had in him. Not letting Ernesto's efforts be wasted would have been yet another reason to do what I could for Timmon, if I needed reasons at all. "If you will excuse me, Ernesto. I have a patient to see."

I found Ezaro keeping his vigil by Timmon's unconcious body, as he had promised. "He is strong enough to work with, but you have lost one of the others in your absence." He opened the box of shredded mushrooms.

"I have eaten, Ezaro. I do this my way."

"But-" He looked disappointed.

"Can you help me?"

"No."  Ezaro closed the box and shook his head. "I am spent."

"Then call Master Alerio," I ordered. Ezaro hesitated. "Call Master Alerio, and I do not tell him what we discussed today."

Ezaro left. I turned my back to Timmon, closed my eyes against the sight of wounded men around me and set to work. I could not plug my ears if I wanted Master Alerio's instruction. Healing a wound like this is rarely successful. It requires too much skill, and too much power. Skill I did not have. It should be Carlotta at his side, with her gentler touch, or Ezaro, with his intricate elegance. It probably had been, I considered, under Master Adele's guidance. They had spent themselves on this man while Master Alerio kept me occupied with others. I felt anger at my tutor and fear at losing Timmon overwhelm my ability to heal. Distance, I reminded myself. I could not help Timmon live if I already greived his passing. I thought of Griswold. 

On the way to Szarvis, Griswold had gutted dozens of fish for me. He would do so callously, not caring whether he severed intestine or stomach in the process. Fish, being lower life forms, are much sturdier than humans, I had a chance of returning them live to the ocean. I only succeeded once. I recalled his stories about dancing, as if on hot coals, with my gift over the most sensitive parts of the body. I started with the liver. That was the easiest. I imagined a finger of flame leaping over the bruised tissue, and watched the bruise shrink. The liver only needs a nudge. I worked on it until it was stable. The liver is an easy, rewarding organ to fix. A damaged liver can quickly bleed a man to death, but is coaxed easily back to health. 

Having gained some confidence, I turned to the kidney. Two of the passages were blocked with clots. One was still pooling blood. I stopped the blood flow, then contemplated my options. I could either coax the bruised and lacerated tissue to reabsorb the clots, or I could strengthen the passages and hope the clots passed on their own. I did not have the finesse for the latter. I doubted I had the delicacy for the former. I had no choice. He would poison himself if I did not do this. I set to work, massaging the blood back into the damaged organ. I did not think of the intestines. That was where most of the damage had been done. If I could not stop them from bleeding and spilling their corrosive juices onto the rest of his body, all my efforts would be for nothing. 

"Let him rest, Nisrita." Master Alerio's voice spoke. I stopped tinkering with the kidney. "Get some sleep."

"But--" 

"You will kill him if you continue. I will wake you in four hours. We will start on the intestine."

"Yes, Master Alerio." I went to my bed. A page woke me as soon as I had closed my eyes. It was past midnight. Master Alerio wanted me at the Healer's tent. 

There was too much blood on the table when I arrived. "What happened?" I asked. Blood did nor frighten me. Timmon's blood did. I pushed the thought aside.

"Much of the new tissue did not hold." Master Alerio said academically. "This happens when the patient is too weak. Too many of his organs are unstable. I have stopped the bleeding, but his heartbeat is faint."

I did not wonder. There was not much blood for the organ to pump. Master Alerio's academic tone helped. "Then we must work quickly." 

"What did your husband teach you about intestines, Nisrita?" Master Alerio asked when I had settled myself with my back to Timmon.

"He told me to imagine a mesh from which the tissue would regrow. I could never do it. The details were too fine."

"You worked on fish, correct?"

"Yes, Master Alerio."

"Human intestines are larger. Try a larger mesh."

I did not learn until much later that Master Alerio had been guessing at what to do for much of that night. Timmon had been given up on as a lost man. To work on him at all at this point was considered foolish, to succeed would possibly be an unique accomplishment. I did not care for any of those overtones that night. I provided a mesh. He helped me adjust it. He healed as he saw appropriate. When he stopped me, I could feel myself shaking from exhaustion. 

"Get some rest, Nisrita." Master Alerio advised. 

"Does Commander Romino need rest?"

"No. You do."

"I beg to differ, Master Alerio. This is not a time to conserve my resources. I ride tomorrow. I will not be able to help you in the battle ahead. If I rest now, and Commander Romino bleeds, he dies. If I continue, and collapse, I bruise my head on the ground, and Commander Romino dies. If I do not, this man may yet live."

"You are too close to your patients, Nisrita. This is why I kept you from the commander when he came in."

"I am, Master. This is why siblings and lovers do not heal each other if there is any other possibility. Is there anyone else who can take my place?"

"He is only one man in an army, Nisrita."

I felt the tears spring to my eyes. Timmon was far more than any man to me. Whether Master Alerio believed in the lies about us or not, he must realize how important he was. 

Somewhere, from a calm recess of my mind, a voice suggested that my tutor was testing me. He was right. I could not help Timmon if I saw him other than an officer in need of healing, like the others I had tended to that evening. I swallwed the lump in my throat and waited until my voice would be steady. "He is, Master. He is the man in this army that knows where to look for ambushes and how to save Duke Griswold's life. The army will survive without him, as a man survives with one eye." I watched Master Alerio nervously and awaited my verdict. What I was asking for was irregular. Master Adele would not have let me proceed. I did not know then that the benefits of my success for Master Alerio would be so much greater than the could have been for Master Adele.

"Leave the clots in the kidney. He will pass them later." I did as I was told. 

I worked until the world spun. Then I rested and tried again. I could not stand when Master Alerio finally told me that Commaner Romino was stable. I felt like a horse that had been spurred on to ride until it collapsed under the weight of his rider. I was alive, however, and Timmon was as well. It did not matter that I lost conciousness on the way to my bed, or that I could not keep down any food I tried to consume that day. If fever did not claim him, Timmon would live. 

\vspace{.5cm}

We rode to Deyalorn, not to Cortan, as I had suspected. We traveled at night, hiding and seeking shelter from friends and collaborators of Commander Griso and his horsemen. My husband's company was, quite possibly the safest place I could have been in Niev. He and his men still knew these lands well, he still had allies, in spite of his known position as Cortan's duke. We rode swiftly and silently through the grasslands, encountering no one my husband did not intend to.

Healing Timmon had exhausted me. I spent much of my first week struggling in that saddle. There were several times my husband threatened to tie me to it. I did not need to endure that humiliation, though I never recovered my strength on the road to Deyalorn. I begged for relief when we entered Cortan's border to be allowed to go home and rest. I was ill, this much was obvious. My husband wished to be in Deyalorn. He wanted me where he could see me. I was his wife. I had no choice but to obey. I spent the ride short of breath, jittery and restless. I must have been more effected by my husband's presence than I thought. Try as I might, I could not calm myself. My heart refused to do anything but flutter ineffectively in my chest. 

We reached Deyalorn four days before my sons turned two. They were to mark another birthday without their mother, I thought, then chased the maudlin thought from my mind. They would be with me soon. My current situation was nothing compared to last year. I called for a bath in the quarters I shared with Duke Griswold in Deyalorn's Tower. The road had been long and hard. I had been ill. If I had to live with my husband again, I would have to approach life practically.

It was in the bath that I first realized the enormity of my mistake. I felt the first undeniable stirrings of life inside me. Perhaps they were not the first. If my womb had quickened on the road, would I have felt it? How long had it been since I had bled? I could not remember. I did not know how long I had carried Griswold's seed. Certainly during the entire campaign. I had healed or practised most days with a fetus growing inside me. Had I not felt the resistance when I worked? I was fairly certain I did not. I was also certain that I had not been as regular about checking for abnormalities or other signs of pregnancy as I should have been. 

I shuddered when I remembered the night I spent with Timmon's wounded body. I had channelled so much fire through my body. It was impossible that this fetus did not feel the damaging effects. Yet it still grew inside me. I had not practised at all on the road. Was it possible the this fetus was still whole? I closed my eyes to the possibility of death that hovered in the steaming air, and sank further into the water to hide from my mistakes. I did not know what to do. I did not know anyone at Deyalorn's Tower. My last visit to this great city had been short and eventful for other reasons. I had no one to turn to.

Panic got the better of me. I found myself gasping for air, in the hot steamy room. When I tried to rise, pain shot through my left shoulder. I stumbled and slipped on the wet tile of the room. My maid steadied me and helped me shakily into a gown. She asked me if she should call a healer. No, I told her. I needed to rest from the journey. I could not imagine what I would tell a healer, or how I would face my husband with this news. She helped me to bed, and gave me a cup of wine to help me to sleep.

I awoke in darkness, with a boulder on my chest and agony in my abdomen. I could not breath. I could hear my husband sleeping next to me. I lay still in the quiet night for a long time, struggling for breath, and wondering what to do. To not get help now was foolish. I was on the point of miscarrying. I struggled, as quietly as I could, to get out of bed. I did not wish to disturb Griswold. I forced myself into a sitting position, to find my hands trembling too much to light a candle. I steadied myself against the bed, and stood. I would walk to the door, I told myself. I would wake my maid and ask her to get help. It was as simple as that. Then this nightmare would be over.

I took two steps before my legs gave out, and I found myself sprawled on the floor. Griswold awoke. "What are you doing out of bed, Nisrita. You are ill."

"I am, my liege," I gasped from underneath the crushing pressure on my chest. "Please get help."

"You are being dramatic. I drove you hard on the road here. You are tired. Go back to sleep. You will feel better in the morning."

"No, my liege. This is more than that." I could bearly hear myself speak against the pounding of my heart. My breath came in great painful gasps. I could feel my left arm start to go numb. 

I heard Griswold grumble, get out of bed, light a candle and walk towards me. He stopped when the candle cast light upon me. I saw the blood staining the lower half of my gown. "You are unclean, woman." He sounded alarmed.

I shook my head. "Please, my liege" I begged. "I am miscarrying. I do not wish to die."

I watched Griswold slowly take in my words. He took forever to make up his mind. Eventually he turned and left the chamber. I did not remain concious long enough to see him return.

\vspace{.5cm}

I do not remember much of the next three weeks. That I remember anything at all is due completely to the efforts of two talented women of Deyalorn's Tower, Healers Sana and Isobela. They both married rich barons in northern duchies five months after they saved my life, and left the service, taking the secret of what happened away from the Tower's grasp.

My fever eventually ebbed, and I found my two healers to be pleasant company. They brought me some news of the world around me, fresh cuttings from the Tower's graden, and books from Deyalorn's library. I saw now visitors, which did not surprise me. I knew no one in this city, and I hardly expected my husband's presence at my bedside. Days passed and my strength and apetite slowly returned. Eventually, I grew strong enough to sit on a soft chair in warm spring afternoon sunlight. It would be hot in the afternoons already in Cortan, I thought, looking out the window. The grassland wild flowers would be withering for the year, where Deyalorn's garden was just starting its brilliant show of colors for the year. My husband intended to keep me here. I would not gain from Master Alerio's wisdom and guidance. I would not become a healer. I would have to struggle under Griswold's watchful gaze to make a new life for myself here. I would have to adjust, I told myself. I was not the first woman to leave her family behind to start a new life with her husband. A new life in this duchy seemed a much harder adjustment to make than it had seemed three years ago under far different circumstances. 

The next day, I recieved my first visitor. Healer Sana helped me dress and stand to receive the Regent-Consort. He raised me from my bow quickly, and asked me to make myself comfortable for this audience. My half-brother seemed much more approachable than the last time I had seen him. For the first time, I realized that we shared a hint of a family resemblance. He was not a tall man. He was squarely built. It suited him, granting him an air of authority to compensate for his average height, where it made me awkward, and caused my taylors much anguish and handwringing when designing my gowns. He too had a brow that barely thinned over the bridge of his nose. Not having to pluck it, as I did, I have seen him draw this bushy line of hair over his eyes together to terrifying effect. They were not terrifying today. The Regent-Consort waited for me patiently, as Healer Sana propped me into a comfortable sitting position on my bed, then bowed and left the room. 

"How are you feeling, sister?" the Regent-Cosort began, when we were alone.

"Stronger, your highness. I am honored by this visit. How may I serve you?"

He smiled benevolently. "I have come to perform my filial duty, and visiting my sister's sick room."

His words confused me. He had not spoken so kindly to me last time we met. "I am grateful for your consideration, your highness."

My half-brother tisked at me. "There is no need for such formality among family. You have made much of yourself since we last met, sister."

"I thank you for believing so, brother." I replied haltingly. The last word felt strange upon my lips. 

"I have several generals, preists and dukes who wish to make your accquaintance, if you would be willing." I gasped. This was too good to be real. I was addressing the Regent-Consort as my brother, the country's leaders sought audience with me, and my brother asked my preferences. "Do you have an objection, sister?" my half-brother continued when I could not find my toungue.

"No, your highness. I-- Only I am not certain that I am awake."

The Regent-Consort laughed. "I assure you, this is not a dream. I bear some bad news as well. While my generals and dukes may be clamouring at my door for an audience with you, I cannot permit it for several months. Has your husband spoken to you on this matter?"

"No." I hesitated. It was hard to break the habit of addressing my brother as his highness. "I have seen no one but my healers until now."

Dario's eyebrow knit itself together. "Very well. I leave it to him to speak to you as he sees fit. In the meanwhile, I must keep you here. Your healers will see to your comfort and needs, but they may not let you leave this room or receive visitors. I regret this inconvenience, but there are other matters at stake here."

"I understand," I replied. I did not. Neither did I wish to seem ungrateful of my brother's time or condescention in explaining my situation to me. 

"If there is anything you need, let your Healer's know. When this is over, the crown would be grateful if you would dine with us."

"I would be honoured, brother." Prisoner in the Tower or not, this must have been a dream. I could imagine no reality where I would be invited to dine with the crown. I dared to test my favoured position. "If I may make one request. May my sons be sent to me during these months?"

"Now that your recovery is certain, your children and your belongings have been sent for. It should be possible to arrange for your sons to see you from time to time. My nephews are two?"

"They are, brother. Thank you."

The Regent-Consort rose, and gave me his hand while I still reclined in my bed. "I look forward to meeting the future Duke's of Cortan."

I lay back in my bed and beamed at the ceiling after he left. Any disappointment I had even thought of feeling at not returning to Cortan, of not being under Master Alerio's tutelage, or not recieving my healer's rank disappeared like memories of an old dream. This was my new life in Deyalorn. Regent-Consort Dario has recognized me as his sister. His court wished to speak to the Maker's daughter. 

\vspace{.5cm}

Nearly uninterupted solitude and restlessness born of my body's recovery conspired to wear down the good humour and courage I found that day. About two weeks later, I awoke early one morning to the sound of the Tower's Bells announcing the birth of a noble child. I vaguely wondered who in the royal court had given birth. I had seen Dario a few times since that first interview. He always came in a good humour, his visits brief, but of a social in nature. I had started daring to believe that I had a family that cared for me. He had mentioned nothing of an expected birth in the court.

I rose from my bed and dressed. My healers served as my maids, and kept me in the habit of attending to my presentation. I saw no point in it, I had no visitors beyond my brother's occaisional visits. I saw less point in arguing with my healers. They were good natured women, and gentle prison guards. I had little other than books and conversations with them to while away my empty hours. Antagonizing them would serve no one. 

My husband entered my room as I brushed my hair. "Congratulations, your grace. You have given Cortan a son." The comb fell from my hand and clattered to the floor. My husband ignored my shock. "I will send him to you shortly. He was born, unfortunately, before his time. He may not live. You are too unwell to admit visitors at the time. It was not wise of you to practise while pregnant, and the delivery has been difficult. I forgive your carelessness with Cortan's bloodline, this once. I will not look upon your practising while pregnant so kindly again. Once your physical health recovers, and your confinement ends, you may leave this chamber."

I struggled to take in this new situation. I looked at Sana in the corner. She looked down at her feet, embarrassed to overhear this exchange, or her own conspiratorial guilt. "How, my liege?" I whispered.

My husband paid me no heed. "I will bring your son to you. Take a moment to compose yourself." My husband left. 

Sana picked up my comb and started combing my hair. "You knew about this?" I asked her.

"I beg your forgiveness, your grace." She replied softly.

I had been too stunned to be angry with Griswold. I was not too stunned to be angry with her. I snatched my head away from and spun to face her. "What else have you kept from me?" How dare my husband burden me with his bastard. Did his cruelty have not limits?

"I am sorry, your grace." Sana begged. "The Regent-Consort ordered that it was your husband's place to tell you of this. I had no choice."

"Of course not." I sneered. "No one has any choice but to bow to my husband's wished. You had every choice in the world. You chose to stand aside and watch my suffer this humiliation." Sana shrank from me. It occurred to me that I could easily overpower this woman and escape my prison. Where would I go? I could not make it out of Deyalorn's Tower without being caught, imprisoned this time in a dungeon as a madwoman, or killed for having embarrassed my husband by appearing decidedly not as woman who had given birth a few short hours ago, a fact he had gone through so much effort to try to hide. 

I threw myself down in a chair and brooded. The Regent-Consort knew about this. The mother must be important to him. There was no other reason for his involvement in this matter. Any fantasy I had of living a life in Dayalorn in the warm embrace of filial affection disappeared. I was useful to my brother, he was kind to me. That was all. I could not deny this child. My husband had announced this birth to the city already. The child existed. Any attempts to deny his parentage would only add to the rumours of my madness. My brother would not support me. I knew no one else. 

"May I start a fire to keep the infant warm?" Sana asked timidly.

"Yes." I glowered. I watched her work. She was a young woman, not yet twenty, I guessed, still in her prime. She did not wear a wedding chain. She probably had lived in the shelter of the Tower, and her father's home her entire life. I should not blame her for being intimidated by Griswold or the crown. It took a fool like me to decide to stand up against husbands and messiahs. I felt my anger drift into bitterness. I knew of my husband's affairs. I did not think flatter myself to think that I was any different from his other women in his eyes, unless it was that I carried the disadvantage of being shackled to him in the eyes of the gods and the crown. I knew that Griswold cared and provided for his bastards. It should not surprise me that he would want to raise a bastard of a noble woman in his own house. 

The fire crackled brightly and replaced the chill of the spring morning with a warm glow. When I began to sweat, I rose to remove an outer layer of clothing. My husband returned, followed by a petite still bundle in the arms of a wet nurse. My bitterness gave way to the demands of my art at the entrance of this tiny child in need of care. 

I took the infant from his nurse, unwrapped his swaddling, and examined him.  Healer Sana joined me. He was tiny, more than tiny, his bones were fine and brittle. His entire body fit into the palm of my hand. I looked at this boy and thought of my sons, born small and early, by virtue of both having to share the small space my body could create. They had looked like strapping warriors compared to this poor creature. His skin was wrinkled, pale and patchy, with a worrisome blueish tint. It was already bruising where his nurse had handled him, though he had only lived a few hours. I had extracted so many pups from their mother's wombs in this condition. I had hoped never to see a living human child alive in this half formed, suffering state, certainly not one I was to call my son. His breath halted at irregular intervals, as if the child occaisonally forgot the necessity for air. He did not suck my finger when I offered it to him. "Does he eat from you?" I asked his nurse.

"No, your grace."

"Sana, make arrangements for goat's milk and clean cloths for the child." Healer Sana stepped outside to make arrangement. I settled the wetnurse and child by the fire, then turned to my husband. "He should be seen by the head of the Women's Tower, my liege." 

"No. That would draw too many questions regarding the circumstances of his birth."

I looked at the babe by the fire. He looked like he was born two months early, perhaps a little less. That made sense, he must have been conceived just before Griswold left Deyalorn. My husband had joined me less than six months ago. Any healer of the woman's tower would know that I did not birth this child. "He will not live without help, my liege." He may not even live with the Tower's help.

"No." The word was sad. The grief surprised me. I saw my husband look at the child with a tenderness he reserved for my sons. "Are you well enough to heal?" he asked gruffly.

"I, my liege?" Did my husband expect me to help his bastard live as well as claim it as my own?

"I will send someone discrete to help you." He shrugged. "Do what you can. I leave his care in your hands."

My husband left the room. As was his habit, he had slipped the burden of this child's life onto my shoulders then left. It would be so easy to ignore his desires and let this child slip away. The child would die on its own in a week, perhaps less. Given a strong healer and skilled supervision the child would likely, but not certainly, be saved. I was neither at my full strength, nor did I have an experienced guide. 

It did not matter, a voice inside my head uttered. The world knew this child to be  born three months early. There are stories of desperate gifted mothers who try to save children born so young. They invariably have sad endings. The Tower will not aid such a case. I could fold my hands in my lap and let it go. No one would even accuse me of negligence. I would be free of my imprisonment and responcibility. 

I shuddered at the thought. Griswold walked away from the wounded and the dying. I did not. This child had done nothing wrong. I could not walk away from him. I did not want him. I sat at my table and pulled out one of the books from Deyalorn's library, flipping listlessly through its pages. I listened abstractly to Healer Sana and the wet nurse do what they could for the child behind me.

***

"What did you do, Nisrita!" Carlotta burst into my room late that afternoon. I sat by the fire rocking the ill fated infant against my chest, reading through some of Horatia Joris's works. Curiosity had gotten the better of me over the last eight hours, as well as my ever present boredom, and, I suppose, a mother's and a healer's instincts to protect the sick and weak. The boy's wet nurse, Rosa, startled at Carlotta's explosive entrance. The boy, I noticed, did not. 

"I made a mistake, Carlotta." I said, pulling at the shawl to better cover my bare chest for the sake of my friend's modesty. I did not know what she knew other than the lie told to the city. Until she entered my chamber, I had thought her at Cortan. I would be careful with what I revealed, I decided. As much as I loved her, Carlotta and I were no longer the childhood friends that could share everything with each other.

"You made more than make a mistake, Nisrita. You could have been killed. Then where would we all be. Did you not think of your child--" Carlotta froze. She was staring at my stomach. I met her gaze calmly. She was an intelligent woman, it would not take her long. "Stand up, Nisrita." Carlotta continued uncertainly. I obliged. I took a step forward to ease her inspection. "You did not give birth this morning." she concluded softly. She sank into the chair vacated by Healer Isobel.

"No, Carlotta. I did not."

"Whose child is that?" my friend asked haltingly.

"Mine Carlotta, as has been announced. The mother does not matter."

Carlotta considered the situation for a while, sitting before the fire in the stifflingly warm room, twisting her hands against each other, fingers locked in her lap. "You were not pregnant on the campaign?"

"No Carlotta," I lied. The words came easier than I had expected. Too much had passed between us. I did not need her judgement on things that were not her concern.

Then she asked the question I had been struggling with all day. "Do you want me to help the child? Duke Griswold asked me for my help. If you wish, I will tell him that I failed."

"I do Carlotta." Saving these lives is what women train for in the Tower. I could not turn my back on that, even if I had no desire to raise this child as my own. 

"May I see him?" I laid the boy onto a table near the fire, and let Carlotta work. I put away the books and papers I had been studying, and began to dress. If Carlotta was to do this, I would have to be the one to guide her. "I am sorry, Nisrita," Carlotta spoke to my back. "I had misjudged Griswold. No wife should have to endure this."

"I doesn't matter Carlotta, see what you can do for the child." I replied. I smiled to myself as I buttoned my sleeves. Those words meant so much. Perhaps there was hope for this friendship, I prayed. Time would tell. 

The smile dropped from my face when I turned to see a familiar looking flat wooden box, and several narrow strips of black leathery mushroom on the table next to the infant. I took the first hard object I found on my dressing table and threw it at the box, sending it and the mushrooms scattering across the floor, and my comb to be consumed by the fire. Carlotta and Rosa both screamed in alarm. The infant wailed softly. My heart filled with a moments hope at the weak sound. "Rosa, please take the boy away from Healer Carlotta. We will not be tending to him at the moment."

"Are you insane, Nisrita. You could have hit the child." 

"No," I corrected. "I would not have." I walked over briskly to toss the larger pieces of mushroom into the fire, hoping that Isobela had not regisetered what had happened. Then I turned to my guard and asked for a moment of privacy. When she reluctantly left, I turned to Carlotta. "You are not one to be questioning my sanity, Carlotta. What were you thinking to bring out that box in full view of other healers."

"Times are changing, Nisrita. Ezaro's and Griswold's discovery will soon become the new way of healing. Not everyone can naturally perform miracles like you can. Is your need to be superior to the rest of us greater than the vows you took to the Preserver?"

My need to be superior? I did not ask to be made so gifted. I have only ever wished to serve. "Is that what they say about me?" I asked, aghast.

"No one needs to say it, Nisrita. The evidence is there for anyone who cares to look. Three men died in Niev that day you saved Timmon. They would not have if you had listened to Ezaro. You turned you back on them. You oppose anyone who would try to do more than we traditionally have been able to." Carlotta paused to swallow her emotion. "You've changed Nisrita. I thought you took your dutites to the Preserver more seriously than this."

I did not like Carlotta's perspective on my actions. Had I needlessly let men die? I had acted as I had to spite my husband, yes, but also because I did not trust Ezaro's offer. His actions were rash, and unapproved by the Tower. That did not erase the deaths of those three black riders. "I have not forgotten my vows, Carlotta." I uttered softly.

Carlotta pressed me, gently and compassionately now. "Nisrita, I am sorry. I did mean what I said about misjudging Griswold. I had thought your hatred for Griswold was part of your madness. I had no idea that your domestic life was so hard." She took my hand in hers. "His cruelty as a husband does not invalidate his ideas as a healer. Imagine what we could do if we could all heal as well as you could." 

For a moment, for one brief moment, I wanted to believe, seeing the world from my friend's point of view, that there were benefits to Griswold's research. "Nisrita," Carlotta continued, "Duke Griswold is a horrible husband. Please don't stand in his way. I am asking for your sake. You know that he wants to remove the ban on the old magic. You are not even a healer yet. I would not forgive myself if I saw you hurt for your stubbornness."

The spell broke. Carlotta may not be able to forgive herself, but she would not support me against my husband either. She might have convinced me of the need for Ezaro's results, if only she had not asked me to bow to my husband's will. I lifted my hands from my friend's grasp. I needed a friend I could turn to against the trials of my married life. Carlotta, once my confidant, could no longer be that friend. "I am not unprotected, Carlotta. I will dine with my brother when I am set free."

Carlotta's mouth hardened. "You have chosen already." She sounded dismayed.

I had not until that moment. Master Alerio had advised me to stay silent until I had to chose. Until that moment, I obeyed, remaining silent and formally unaligned. Until that moment, I had simply hated Griswold. I would stand in his way personally, I would take advantage of what shelter the crown could offer, but I had not decided to stand against him publicly. Perhaps my husband had pushed me too far by asking me to acknowledge his bastard. I could not stand by an allow him to turn more of Marsea's healers into cripples like Ezaro. I could not allow him to succeed in weilding the gift, far less the old magic. Lucretia's locusts seemed tame compared to what my husband could do if he could weild that power. 

Perhaps he went too far when he drew Carlotta into his web. I looked at my school friend. We had shared everything together once. She would marry Sergent Madriano this year. She wanted a family and children. I could not imagine why she would want to enter that world of nightmare and horror. Griswold made a habit of finding innocent young women and corrupting them. I had no reason to believe that my husband had touched Carlotta. He had spoiled her all the same. I needed to know how deeply he had drawn her in.

"Why did my husband ask you to come to me tonight, Carlotta?" I asked.

My question confused her. "What do you mean, Nisrita. To heal the child, of course."

"He had all of Deyalorn's Tower at his disposal. Why did he chose you?"

"He told me I might find you in a fragile state. He said you might benefit from the company of a friend."

How considerate of him, I thought. "He lied to you, Carlotta." At least he did not send her to sway my views. Nor did she seem to be his spy. I recalled my husband's many alluring habits. He had been such a different man when when had wanted something of me once. "What did he offer you in exchange?"

I watched Carlotta bite her lip. "No one has helped an infant born after six months to survive. If I succeded, he would find a Tower that would appoint me as honorary Master."

Becoming an honorary Master was a great honor. It would mean that Carlotta would heal at that Tower until she became of an age when she was better fit for teaching rather than healing. Given the illicit nature of this deed, it would likely be a minor tower. I wondered what Sergent Madriano thought of the prospect of following his wife to a distant barony, or if she had even had time to discuss this career move with her afianced. Griswold had a way of turning girls' heads. "You will not be helping a child born after three months early survive. This child is barely two months early."

"Yes, I see that now." Carlotta said sheepishly. "Nisrita, I thought I was helping a distraught father save his son. I thought I was helping you. The position was not foremost in my mind." She paused, and her tone softened. "He looked so vulnerable when he came to me this morning. He looked like he would offer me the stars if I succeeded."

I contemplated the idea of Griswold looking vulnerable, and decided I could not imagine it. I found myself inexplicably jealous. He had never tried to court me with gentleness and vulnerability. He never needed to, I reminded myself bitterly. I had been so willing to be by his side inspite his gruffness and cruetly, that he never had the need to feign humanity. It did not matter now. The damage had been done. "Do you still want to help him?" I asked.

Carlotta gave me an exasperated look. "It is not about him, Nisrita."

Horse eggs. "Then who is it about?" I asked calmly. "It is Griswold's child you are about to heal, using his preferred tools."

My friend answered me tightly. "You have tossed his tools in the fire, Nisrita. As I said when I came here, I will only heal the boy if you wish me to."

"Have you eaten today?" I asked abrupty. If we were to do this, we may as well get started. Arguing over my husband was a waste of time. Carlotta shook her head. "Are you lucid now?" I continued.

Carlotta nodded. "I did not want to consume the mushrooms until I knew what you wanted to do," she said.

There was still hope for us, I thought, frail as it may be. "Come back after you have eaten. You will need your strength."

I watched Carlotta leave. I went to the child whose life we were about to play with. I did not know if i had just condemned him to death by denying Carlotta her prefered tools. I gave him a kiss, then I knelt before the idol of the Triumverate and prayed.

***

Sophia found me one evening swinging gently between two dwarf oaks behind my house. In the month since my return, Nubo had devoted himself completely to my physcial health. This hour of hanging on a Liri hammock between two short stubby trees, for him, was a non-negotiable aspect. It was a pleasant enough impossition. I granted him the desire.

I remember finding myself alive in Niev a weak and broken man. What part of my flesh had not been eated away by fever was starved by my inability to keep down my food during those first few weeks after Nisrita's parting gift. Master Alerio assured me that in spite of the fact that my insides burned with pain, and my body still ached with fever, all was as it should be, that I would likely survive. The Black Rider from Turina, Ernesto of Celona, eventually explained the events that led to my waking amid the charred ruins of Rialotte. I remembered nothing beyond being outnumbered in those high sandstone caves, and the urgent need to hold a the line so the men I had led into that disasterous battle could jump, bleeding, to the relative safety of the rocks below. I could not believe that only half the men I had led up into those caves survived. Nor could I believe the tremendous effort our White Healers had gone through to keep that number as high as it was. To contemplate the efforts of the two healers who had refused to allow Rishiki to cut the thread of my life I found overwhelming. I thanked the rider from Celona for my life, and recommended his promotion to General Morel. The next time I found myself in Turina, I promised myself I would visit his family. Nisrita, it would be harder to thank. 

General Galderan sent me home as soon as I was well enough to travel, in spite of my protests that I could listen to intelligence reports as easily from a bed as I could while standing. Marsea had lost all lands west of Rialotte. We struggled to keep the lands to the south leading to Lir. Sama Gorrou had returned safely to his country, but without a secure road running south from Cortan to Lir, King Ayum would not agree to sending an ambassador. This defeat against Niev could prove politically disasterous.  I needed to see the road to Lir established, for myself, to honour Makala and uphold his last vision for Marsea, as an offering to Rishiki of the four winds for offering me his fickle protection. By his favour, I had re-established my carreer, and kept my life. Under the current state of religious dogmatism, I could not thank him openly. My service to the country he guarded must serve instead. 

It was not to be. General Galderan would not bend the regulations dictating who stayed with the army, and who went home. I would be a liabity to protect, my skills were too valuable to risk losing again. So I returned home. Nubo wept, and Sophia met with a composed concern that spoke more of the polish of Deyalorn's court than any tension in our marriage. There is something, I have learned, to the Liri belief that there are emotional needs that only men can fulfill for men. Nubo fussed and worried at me enough to put most Marsean wives to shame. But I found his manner far less grating than I would have found the same care given by Sophia. Under his desires, I let myself lay idle, to be fed, coddled, prayed over, and gently swung in the cool evening breeze in full view of my house. When I felt strong enough to partake in my household's activities, I did. When I tired, I watched, content to let my domestic life flow around me. I had, as Nubo never ceased to remind me, been granted a life that should not have been mine. It would be disrespectful to the gods and my friends not to appreciate what they held. 

Sophia approached in a beautiful red gown, her long black hair pinned and curled so that the ringlets bounced just above her shoulders. She was starting to show. As with everything, Sophia bore her impending motherhood with grace and beauty. Her gown drew attention to her changing body, making her seem more alluring for the one she carried.
 
"Are you going out, Sophia?" I asked drowsily. The warm sun and evening breeze had acheived their intended effect.

"No, Timmon. I attended lectures at the Tower this afternoon. I have no engagements this evening."

This display was for my benefit then. I watched Sophia settle herself in the seat by my side. Since my return, my wife had started doing little things that were meant to please me. I believer it was an expression of her gratitude towards me. Sophia was pregnant, the child was not mine. That was within the terms of our original marriage. She had, however, concieved while I was in the field, which was not. Our marriage only provided a thin shelter for both our unconventional habits. It had been foolish of Sophia to bring scandal and possible ruin to our door by this carelessness. She, as Nisrita had once, offered to destroy the life inside her, if I so wished. I forbade it. The Tower's women were too casual with unborn life, always too ready to remove an unborn child from a mother's body as if it were an inconvenience, and not a life in itself. I would accept the improbably child as my own, we would weather the talk together. She had made a mistake. I had certainly made my share of mistakes in our marriage. It was not ill intentioned, and she was pentinent. I no reason to shame her. She, on the other hand, mistaking our union for a conventional marriage saw a need to please me with these little displays of femininity.

I felt my cradle rock slowly between the trees, rocked by Sophia's hand. "Did you enjoy the afternoon's lecture?" I asked, sinking comfortably into the pleasure of the motion and the peace of my domestic life.

"I did. Master Alerio spoke very eloquently about your recovery." I groaned, to Sophia's gentle amusement. I did not like my insides being discussed so publicly. "You cannot escape this fame, Timmon. What the duchess did is unprecedented."

"I have heard," I said tightly. While still in Niev, I had heard my healers explain that what Nisrita had done was possibly unique. Others said that she had nearly killed herself with exhaustion in bringing me back to life. How could I possibly repay her for that? Nisrita was a stubborn girl of phenomenal power. I knew she would move the earth and sky for me, given half a chance. She had the wrecklessness characteristic of her age. Men her age were among the youngest men in Marsea's army. They have trained with their comrades for years at that point, learning loyalty to each other, and to ignore certain dangers to themselves, but also the futility of exchanging one's own life for that of a comrade. Nisrita's training with both the army and the Tower has been patchy and often interrupted. She lived in the dangerous state of extreme power, good intentions and a tendency to be driven by her passions rather than thought to long term consequences. I feared for her in Deyalorn, alone with her husband and her tumultuous domestic life. Not only could I not repay her for my life, it would take weeks for me to even get news of her.

"Duke Ergino asked about the health of his new general," Sophia continued after a long lull.

"I have not been promoted yet," I replied modestly. It was a formalitly. The position at Turina was mine as soon as I was fit to serve. "Did his grace have anything else to say?"

"Not as such," Sophia gave a troubled pause, "he did not like the lecture."

I stopped Sophia's rocking hand and sat up slowly in my resting place. "Something is bothering you."

"The Tower is in disarray. Master Alerio is using Nisrita's feats as evidence of the strengths of the old methods of healing, in opposition to the new methods Duke Griswold espouses. I have heard people go so far as to say that Master Alerio will no longer be the next head of the Tower because of his opposition to the duke. If Duke Ergino takes sides on this matter..." I watched a tiny crack form on my wife's normally perfect composure. She took a moment to repair it. "The young duchess may join this debate soon. Life may get uncomfortable for us here."

My wife was frightened. I did not blame her. We lived our lives balanced on a ledge. It would only take one person asking the wrong questions to destroy our names. "You could come with me to Turina."

Sophia looked at me, pleading. "Do not take me to that desolate mountain court, Timmon. I  beg you."

If Sophia had come here to charm her husband with the power he had over her beautiful suplicating form, she would have succeeded, if she had a more conventional husband. I understood the insecure position we found ourselves in. The rest of the performance, I ignored. "Our family will be safe from intrigue there. I will return to my post in a few week. Make you arrangements for your confinement in Cortan's Tower, or with you family in Deyalorn, as you see fit. Meet me in the new lands when you and the child are fit to travel." 

Sophia understood her error and addressed me as she would address her colleagues. "Turina hardly has a Tower yet, Timmon. Do you expect me to uphold my end of our agreement on the with the company of gifted fighters and goat herds?"

As she spoke, I understood my error of expecting that my wife would follow me where I dictated. Ours was a strange marriage. We could choose to be each other's allies and companions, or we could prove to be the ruin of the other. We lived more like two barronies sharing a contentious border, than husband and wife. I saw no reason to engage in an unnecessary war. "What do you propose?"

"Come with me to Deyalorn. Much of the duchess's household leaves in a week to join her in the capital. I will travel with them. The matter of the Liri ambassador has yet to be settled. I am certain, whoever the ambassador is, he would benefit from a promising general on his staff."

I laughed at my wife's scheme. "You wish to live in Lir. There are no Towers there at all, Sophia."

"You description of the prince's court in Lesoko has intrigued me. I think I could be happy in Lir. Our family would be safe. I will find Marsean tutors for Eugenio. I think," she continued seductively when she saw me composing a counterargument, "that there would be advantages for you in this as well?"

\vspace{.5 cm}

I arrived in Deyalorn in time to enjoy the last few weeks of the strawberry harvest. I fed Nubo the plump red fruit until the juices dribbled from his lips into his beard when he laughed. The poor man spent his days every spring speaking of small white fleshy fruit with a prickly red skin that grew near his home at that time of year. He claimed it tasted like sweet rose water. Strawberries were a poor substitute for rose water, but they were better than anything I could offer him in Cortan. It pleased me to see him lick the red juice from his lips.

Nubo had lived away from his home for nearly two years. I understood his homesickness. After more than a decade in Cortan, my mind still turned to my father's orchards every autumn. Nubo behaved as if I had already decided to go to Lir, happily planning our southern journey and teaching Sophia his native tongue. He paid not heed to the fact that I did not wish to leave Turina behind, pouting and sulking whenever I mentioned Duke Erfat's name. I wanted to see Niev conquored. I had been there at the start of Marsea's expansion into the Nievian grasslands. I had lost Makala to the Nievian wars. Now, they beat back and shamed Marsea's armies. It could not be tolerated. I knew the lands, I knew the people. Marsea's armies must change to win Niev. My duty lay in helping my nation reforge its army against this extraordinary enemy, not growing soft as an advisor to a diplomat on foreign soil. My views, however, were outnumbered in my own house. A different man, in a different marriage, would have been well within his rights to order his wife and his servant to follow him to Turina. But Nubo was more than a servant, and Sophia was not quite a wife. 

News of Nisrita's third miracle reached us when we stopped for a night in the Barony of Cleyoren. Master Alerio hosted guests and healers every place we stopped, speaking and answering questions about his teaching methods. Sophia was very good about sheltering me from the prying eyes of Master Alerio's guests. I had no need to have my stomach bared or my intestined examined. However, the Baroness of Cleyoren was Makala's youngest sister. I could not deny her an interview. 

"Did it hurt, General Romino, to be brought back to life?" Baroness Chamila asked during a private dinner after I had given her news of her family.

I smiled inspite of the memory of the devastating pain of those first few days of conciousness. "I did not die, baroness. I had simply been given up for dead." 

"How horrid," she exclaimed with an excitement that revealed her true fascination with the subject. "It was cruel of the healing corps to give up on you while you still lived. I think I am glad I am not gifted after all. I could not live with that cruelty."

"We all do what we can, Baroness," Sophia replied. It was neither a reproach, nor a defense of her kind. It was simply an elegant statement of fact that could not be questioned. 

Baroness Chamila turned her attention elsewhere. "I must thank you for saving my brother-in-law and uncle, General Romino. I have heard it rumoured that you would be a general by thirty. My husband tells me that this is quite an accomplishment, even in the border duchies," 

I smiled politely at Baroness Chamila's praise. She still looked every bit the pretty girl of fifteen I saw ride out from Cortan's Tower the year Makala married, though she was a mother of two now, and no longer a child in her father's court. "I do what I can to serve, Baroness. I do not yet hold the title with which you please to address me."

The baroness turned to my wife. "Talented and modest, healer. You are a lucky woman to have such a husband by your side again." 

I watched Sophia blush as much as appropriate for such a compliment, and no more. "My husband," she explained, "is hoping for a position in the Liri embassy."

The baroness tittered. "You wish to raise your family abroad, healer." 

"I am not so fortunate as to have a home in as comfortable barony as Cleyoren, my lady. The Liri court has its appeals." The women continued their discussion about Liri fashion and stories of court in Lir and Deyalorn. I turned to Baron Cleyoren. I admired the baron for his patience with his wife. I would not have been able endure her childish flippancy. However, no one can accuse me of being a connosieur of women. I spoke to the baron about the battles I had just fought in, the troubling new Nievian tactics we faced, the difficulties in convincing the other interior duchies of the prudence of a peacful border with Lir, and the queer sudden fervor of the priests of the triumverate over foreign gods.

I was grateful to Sophia for taking the Baroness's attentions away from my person. The Baron harboured no such curiosity, and my evening proceeded pleasantly. When joined her later in the chamber we shared as the barron's guests, I found her sitting at her table, fidgiting with her ribbons. "What is the matter, Sophia?"

"Baroness Chamila gave me two pieces of news of her sister, the young duchess." She paused while I seated myself to give her my full attention. "It would seem that she and Carlotta have performed yet another unprecedented feat. They have found a way of coaxing a child born nearly three months too early to live."

"But that is wonderful" I exclaimed with great admiration for Nisrita. There seemed to be no end to the wonders she could produce.

"It would be, if not for two points." Sophia proceeded somberly. "This means of performing this miracle is being kept as quiet as possible. No one in the Towers are speaking about it. Master Alerio thinks there may have been old magic involved."

Nisrita hated and feared the old magic with the same fervour that she hated and feared her husband. I could imagine no way in which she would have willingly practised the forbidden art. "NIsrita is a powerful healer. Could she not simply have done this herself?" 

Sophia shook her head. "I do not think so, knowing what I know of childbirth and Nisrita's abilities. She is powerful, not subtle, and infants, especially those born early, are fragile. Carlotta, on the other hand, is close to Griswold. She followed him to Deyalorn as soon as she was allowed leave from the campaign. I could believe her capable of touching the old magic under his tutelage."

Sophia let me mull over her words. It seemed unlikely. Nisrita and Carlotta had drifted over the past few years, much to Nisrita's distress. Was it possible that Carlotta pathched their childnood friendship enough to coax Nisrita into partaking in an act she was so opposed to? It seemed unlikely. My wife studied my frown and said "This may be good news for us, Timmon. If the duchess is not going to fight her husband on his desires for the nation's towers, we may not need to chose sides."

I shook my head. I could not think only of my family's future if therer was a possible threat to the duckess. "There is a piece of the story missing in this. I am certain. Nisrita's actions do not make sense. You said there was a second piece of news?"

My wife rose from her table and crossed the room to sit next to me. She placed her hand on mine, seeking comfort from the frightening implications of the news she bore. "There is. The child she saved is her own."

I startled and stood, pacing the room, inspite of Sophia's attempts to calm me. I could think of nothing other than the fact that Nisrita could have killed herself by attempting to save me. If the cold calculation were to be done, her skills were more important to Marsea than mine could ever be. She had no right to threaten them as she did for our friendship. She had no right to play a game where she exchanged her life for mine. What would I have done in a world where I awoke without her living. I would have been lost, almost as surely as I was lost in a world without Makala. I had not given her my consent to thrust me into that madness.

"Timmon."  Sophia's voice beckoned from beyond the world in which my life had not narrowly avoided disaster.

"She had no right." I quietly raged. "I did not ask her to exchange her life for mine."

"Timmon." My wife's genteel authority restored me to the four walls of the barron's guest chamber. "Nisrita saved many lives that day, and many on the days before that. The problem is that she should not have been healing at all."

My wife was correct, but knowing that Nisrita wrecklessness extended beyond my person did little to console me. I could not leave Nisrita defenseless in Deyalorn, erradic and unstable as she seemed to be. "I will not go to Lir." I announced. I did not have a further plan, but that much was certain.

"You are upset, Timmon," my wife cooed uselessly by my side. "Do not decide now. You cannot do her any more good in Turina than you can in Lir."

"Then I will not go to Turina." I blurted. I did not know what to do. I could not lose Nisrita.

I saw anger flash in Sophia's eyes. In her mind, I had overstepped the bounds of our agreement. She feared that I would threaten our family for the duchess. She did not trust me to find a way to protect both. When she spoke, her eyes remain angry, but her voice emerged flawlessly composed. "As you said, Timmon, there is more to this story than we know. You will be in the capital tomorrow. Wait to hear what she says." 

I grumbled, but paid heed. I did not wish to relive those first months of our courtship where Sophia's eyes accused me of destroying our postential future with my drink and dice.
 
****

"Do you have any idea what you have done, sister?" Regent-Consort Dario asked me during one of his visits after Griswold had given Lucretious his name. 

I was certain he was not referring to the matter Carlotta sat investigating in the Tower's hatcheries as we spoke. "I am sorry to have inconvenienced you brother. I did not know how else I should have acted."

"How do you intend to speak against your husband when all his allies claim that you have used the old magic in saving your son."

I insisted the truth to him again, as I had already insisted numerous times. "I did not use the old magic, Dario. I swear by the triumverate I did not."

"If I bring you another child born three months early, can you save it?"

"Of course not, because that is not what I did in the first place." I lost control of my temper where I should not have. My brother knew of Lucretious's origins, that he was not as young as the city believed. In the nearly three weeks since the child had come into my care, no one had offered me any information surrounding him. I did not know his mother. I suspected that my husband was the father, but I had no proof. I did not know why the child was important, or what resources I would have to raise him, now that he stood a chance of surviving his infancy.

My half brother responded to my raised voice by raising his own. "Then it does not matter what or by whom you swear. It only matters what my dukes believe. You are useless to me in this debate Nisrita. I will find another way to resist your husband's allies."

Dario looked ready to end the interview. My anger slipped into desperation and depair. "I am sorry, brother. I did not know. I have been raised in Cortan's Tower, and trained only in healing. I know little of politics. Forgive this mistake, I beg you. What would you have had me do?"

My brother gave me a curious look. "Have you do, sister? I had expected you to do what most mothers faced with their husband's ill bastard would do."

The ungifted are monsters I decided. They do not value life as the gifted do. Most mothers faced with Lucretious would have let the child die, but most mothers did not possess the knowledge and ability to save such an infant. I could not let that happen as my brother could. Lucretious should not be a casualty to a political game. Even the name my husband gave him invoked the old magic and Queen Lucretia's power.

I responded meekly to my brother. "I am sorry for misjudging my role, Dario. I am still called the Maker's daughter. That must be of some use to you."

"I cannot parade the Maker's daughter in front of my dukes if she is to be called a hypcrite."

I felt desperate and trapped. I had done nothing but follow my vows to the best of my abilities. How had that simple act given my husband an advantage? There must be more I could do to be of service to my brother. I had to stand in the way of Griswold's mad desires. "Give me access to the mushrooms and animals from the kennels, Dario. Let me prove that the old magic is more dangerous than my husband claims."

My brother looked at me afresh. Had no one had proposed systematic study to him before? "Can you do that?"

I drew myself up with more confidence than I felt. "My husbands methods are are dangerous to healers. I have seen the side effects it has had on his students. What I can prove depends on the resources I have at my disposal."

\vspace{.5cm}

I received regular visitors after Carlotta and I finished with Lucretious. My healers kept the visits from court and the Tower confined to a few hours in the afternoon. I played the part of a tired and recovering mother who had foolishly served Marsea's army during her pregnancy, staying in bed, and feigning weakness. Masters of the Tower came to see Lucretious. I was certain my husband used the boy as a shining example of what could be attained with his new methods. I had played foolishly into his hands. What could I have done? Should I have let the child die? How would that make me different from the corrupt man I had married. We both once took vows to serve the preserver. Unlike my husband, I honored mine. I struggled daily with my hatred Duke Griswold. I had chosen the lesser of two evils I told myself, I had taken the higher ground,  even if it landed me firmly in his trap. I was a healer before all else. I could not have let the child die.

Members of the Royal court visited my birthing chamber. Many would come as husband and wife to bestow presents upon the new child of Cortan. The wife would play with the child and remark upon strong he looked already. Lucretious's bones had thickened. He had grown under the care Carlotta and I had provided. He did not struggle to breath any longer. He learned to suck within four days of our efforts, and he startled and cried at loud noises. He was a small skinny boy, he may be sickly all his life, but he had lived long enough to be transfered to the Preserver's domain. While the wives cooed, their husbands thanked me for my contributions to the military. Some would propound their opinions on restructuring the way the healers and the military worked together, on changing the ratio of healers to fighters, or advocating for less reliance on bases, and a greater reliance on the moving infirmary I had served in during my last days in Niev, or a number of other possible changes. A priest visited me one day to ask me about my feelings about Horatia Joris's heresies. I was fortunate that Carlotta joined the audience part way through. I had convinced the man that I held no heretical views, but it was Carlotta's greater knowledge of the Maker's designs that turned the man from a wary accquaintance to an ally. I revelled in those visits. I saw what role my brother had planned for me in the weeks and months ahead. It was a mantle I would happily don. 

Others visited as well. My brother, once every few days, my husband saw his son briefly every morning, as he had done for my twins when they were born. Carlotta and I would sit and discuss her work in the evenings after the courtiers had left my bedside and I was free to move about. When Lucretious failed to respond to the traditional methods of coaxing each of his organs to stabilize one by one, I suggested a different route. As the boy would likely die in our care, I justified that this was an appropriate time to test a new idea on a human before verifying it on lower animals first. Carlotta could not regrow limbs on grown men. She could perform the feat on infant animals, but as a body grows, it becomes more reluctant to respond to the healer's directions. When Carlotta worked with pups, she could regrow a limb in half the number of days it took any other healers. The sensitive fragile creatures responded to her gentler touch, they sustained her efforts for longer than anyone else I had seen. We discussed the idea and decided that it was worth trying her gentle touch on Lucretious. Carlotta coaxed Lucretious to grow, not stablize while I monitored his organs and directed her efforts. The short daily sessions exhausted Carlotta mentally. It required incredibly fine attention to detail, she claimed. It took a week, but it worked. 

Having saved one life, Carlotta retreated to the animal pens to codify her process and make it rigorous. We had found another way to save infants born too early. It was more difficult, and required  the attentions of a deft healer, but it seemed more reliable than the methods currently practised. Carlotta did not think we had changed the number of months a child had to be in the mother's womb before it could be saved. We had not perfomed a miracle, as the city claimed, simply improved existing methods. However, if Carlotta could make the process rigourous and teach it, would win her a Master's position in almost any Tower in Marsea when the time came.

I started establishing a workspace to research on the effects of the mushrooms. It was impossible for me to leave my confinement. Moreover, given the nature of my work, Deyalor's Tower would not give me access to its teaching pens. My brother arranged for my needs in the palace, and found an intelligent young student, capable of following instructions and caring for the animals. I could not meet with him for the same reasons I could not meet with most of the world outside my birthing room. I satisfied myself for the last two weeks of my confinement by giving instructions on the set up of my workspace to my brother's servant who returned with my student's reports.

It was a restless existence, but I was spared the agony of boredom. I worked, I met with the court, I paced when I grew restless, when I had nothing else to do, I learned to care for this child who had unexpectedly become part of my brood to keep my worries about the upcoming conflict with my husband at bay. I burst with delight on the morning when my two precious boys suddenly ran into my room. Their father was there, visiting Lucretious. They tore through my chamber, investigating the thick velvet curtains common to these northern lands, Lucretious's crib, their father's pockets, the feathery pillows of my bed. We played and sang and laughed. They jumped on my bed and tumbled into my lap. I did not like that they seemed as fond of their father as much as they enjoyed my company, but I put it aside for the moment. I had not seen them for six months. For one beautiful moment that morning, my life was complete. Thoughts of what would happen to my two sons, how I would raise them in the upcoming conflict with my husband, or worries of what I could do effectively against their father would come later. For that moment, I was happy.

\vspace{.5cm}

When my supposed confinement ended, I was released from my prison. I left the Tower with Lucretious to join my two sons in the palace as my brother's guest. My husband remained among his kind in the Tower. This would be a long and difficult battle ahead. I was grateful for the gifts of my son's company and the crown's support.

My days were mostly my own, to study, work or spend with my family. My evenings belonged to my brother. He filled them with private interviews with members of the Royal Court. Master Alerio, I found was the first to lay his claim on my time. 

"Good evening, Master Alerio. I hear you have been drawing large audiences with your talks on your teaching methods. I congradulate you on your success." I greeted the Master in the rooms I had been allowed to turn into my workspace. I had been given access to a suite of five rooms: three as living quarters for myself, my children and their nurses, one for my animals, and tests, and a small office. The past two weeks had not been entirely wasted. The two rooms I would conduct my investigations in were well appointed. Two walls stood lined with cages of rodents and birds that had already adjusted to their new homes, and sat relatively quietly in their spaces, another was lined with glassware, scales, dissecting knifes, and other instruments necessary for my work. In the center was a large stone table, only slightly larger than a healing table. I could eventually turn this into a room where I would see patients if circumstances provided that opportunity. It was almost as large as any Master's workspace in Cortan's Tower. At the moment, it sat pristine and unused. Master Alerio and I sat on wooden stools at the work table.

"Thank you duchess. I am, however, disappointed in you. I would not have taken you on as a student had I known."

"I am sorry, Master. It was irresponcible of me." There was no point in clarifying his mistake. The reason for his disappointment was valid, even if the details were incorrect.

"Now I hear that you have sided with your husband after all on the question of the old magic." 

This slander I could not take. I was an impossible position my husband had put me in, my honor as a woman, or my honor as a law abiding member of Marsea's Towers. "You have heard incorrectly, Master. I am as opposed to the old magic, and my husband's use of the mushrooms, as I ever was."

Master Alerio cocked his head to study me. "You brought back a child born three months early without the old magic?"

"No, Master Alerio, I did not."

"Then you can reproduce the feat?" 

"I cannot."

The old man gave a sigh of exasperation that echoed my frustration in this situation. "Would you care to explain yourself, your grace?"

I could not. Not publicly. My husband had snared me firmly. I could not bastardize Lucretious, for fear of being called mad again. The evidence stood against me. I had arrived in Deyalorn, and fallen ill. A month later, I give birth. My husband had established for me a reputaion of madness. Who but a mad woman would help her husband's bastard survive, only to deny him a name? I feared losing my sons as a punishment for my honesty. If that happened, I would in fact go mad. The public debate had to be focused on something that did not involve my family. "There are some matters of delicacy, Master, that are best left untouched by public debate."

I watched disapproval cast a shadow over the old man's face as he jumped to the obvious conclusion. My husband had snared me. I could not save my honour as a healer without sacrificing my honour as a woman. If that was the price to gain this influential Master for an ally, I had not choice but to pay it. I prayed that I had judged him correctly. Men of the White Tower are well aware of the discrete services the Women's Tower provides. Many do not wholly condemn a female healer for a single mistake made while working closely in the company of other healers. Our abilities are too valueable. 

The shadow cleared. "I have spoken to Healer Carlotta about the new methods you two have uncovered. She has attracted the attention of the head of Deyalorn's Women's Tower. Head Isadora is impressed."

"I hope Healer Carlotta is able to teach her results. It would be useful to towers throughout the land. She is uncertain that she can."

Master Alerio agreed with the sentiment then continued sternly "I cannot use that accomplishment to argue against your husband, your grace. Until she reproduces it legitimately, it is unclear whether the work is tainted by the old magic." 

I tapped my feet restlessly against the leg of my stool. It was frustrating, this trap. "I understand, Master Alerio."

"Is there anything you have to offer?"

I bristled at the question. I was the Maker's daughter. I had healed Timmon when he had been given up for dead. I had possibly found a new way of helping premature children. What else do I need to do to show the world that I was worth listening to over my husband's word. "I am still the Maker's daughter, Master Alerio. No one can question that accomplishment."

"Yes." Master Alerio paused to fidgit with the collar of his robe. "You have not heard yet, I suppose. Healer Ezaro will be promoted to Master in Deyalorn's Tower. The decision came down today.

"What!" I screamed. A Master's position is a teaching position, an honor given to a healer of great accomplishment and experience deemed too old to heal efficiently, and better suited to teaching. Ezaro was twenty one, still in his prime. He should be practising, not teaching.

"It is an honorary position only, given to him for his success in modifying and teaching the techniques you discovered. He is already being hailed as one of the greatest teachers Marsea has ever known. Deyalorn's Masters consider his modifications and teaching accomplishments more impressive than your original unreproducable act."

It was not fair. It was cruel and sinister. This was my husband's doing, another instance of his robbing me of my acheivements and credibility. "Ezaro would be nothing if it were not for me. Worse, he would be dead if not for me. He nearly killed himself making himself in full view of representatives of every tower in the country!"

Master Alerio remained insensitive to my anger. "I am aware of that, your grace. However, there are only five people outside this room who know what happened that day."

I could not stomach any more of this constant thwarting. "This is my husband's doing," I yelled. "He made Ezaro a Master before I could come out of my confinement to say anything embarrassing. It should have been me getting that honor."

The old man's voice hardened. "Duchess, your accomplishments are great, but you are not even a healer yet. No Tower can give you that honor while you are still in training."

"Cannot, or will not, Master?" I continued in a raised voice, oblivious to the disrespect I showed the Master of Cortan. "What more do I need to do to prove that I am capable of healing? Shall I find a way to bring back the dead? Any other student would be granted the rank of healer for just one of my accomplishments. Any Tower woud consider itself lucky to have one of my power among their number. Now, no Tower will touch me because of my husband's wishes."

Master Alerio sighed. "That is the point, is it not, duchess? You are engaging in a fight against your husband. There is no difference between a Tower's will and its ability in this setting." 

The cold, terrifying logic doused water on my rage.  Master Alerio was correct. Did I have no support? "And the crown? My brother has me as guest in his house. Why did he not speak for me?"

"The crown cannot directly dictate to the Towers who it promotes. His highness has made his preferences clear. To do more than that would aggravate relations with the Tower." I sagged and dug my feet angrily into the stone tiles of my new workspace. I feared that everything I had done been for naught. It seemed my husband won already. What was the point of performing impossible acts of healing Marsea's Tower's credited others with them, and cast me aside as a renegade. "You are not without allies, duchess." Master Alerio reminded me. "There are some in the Towers who do not agree with your husband, though few of those can be found in Deyalorn. The crown dislikes your husband for reasons of its own. The clergy fear that the old magic will make the Towers too powerful, and upset the balance currently held between the army, the clergy and the healers. The military tends to support the Tower, but you are still the Maker's daughter to them. You could still be a powerful figurehead against your husband, if that is still your wish."

It was. It was moreso now than it ever had been. "What do you advise me to do?"

"If you had come to Cortan, I would have immediately made you a healer. It is unfortunate that you could not. Find someone who will support your rank here. I have been asking among my associates to no avail. Perhaps the Queen Regent may have some ideas." I nodded. "If there is anything else you can do to prove your raw skills against your husband's augmented skills, it would help a good deal."

Master Alerio wanted me to perform another miracle. I was a farcical request. "This room, Master Alerio, is at my disposal to investigate the harmful effects of the mushrooms my husband espouses. Anything you could do to help me in that task would be greatly appreciated." 

Master Alerio smiled at me with a twinkle in his eyes. "Then we should begin," he said, "with a visit to Deyalorn's libraries." He rose and ushered me out of my room. For all practical purposes, I was a student following my teacher.

\vspace{.5 cm} 

I entered my office the next morning to read the biographies of certain old heads of powerful Towers who had resigned due to madness. Master Alerio had recommended that I search for a commonal themes between their symptoms. Many heads of major Towers and a few elite Masters toy with the old magic at some point in their career. It is a well kept secret, even among the ranks of the healers, one that I did not know until I married Griswold. I could only hope that the biographers had been accurate enough to unknowingly leave me clues to their symptoms of a madness whose cause could not be discussed in public.

My children played in the garden behind me. I could hear them through the open window. Given the dark object of my reading, it was pleasant to be surrounded by the sounds of their joy.

I read from morning until noon, then closed the book against the vividly depicted horrors, drew my knees up to my chest and dug my knees into my eyes. The garden was silent, my children had long since gone inside. I had learned that Head Auturno of Selvand's Tower, a man who coordinated the efforts of the Marsea's Towers during the last ten years of our constant wars against Szarvis before the Treaty of Cretius, had died of a wasting sickness. The picture painted by his biographer reminded me of Ezaro's appearance both times he had emerged from Griswold's quarters in the Tower, recovering from a poor experience with the mushrooms. I did not know what happened to Ezaro during his recovery. I did not even know for certain what caused his bouts of sickness. I only had a mysterious comment about certain fungus having unpredictable effects. The Head had been forced to step down after physically attacking a series of healers, and eventually a kiling a master by pushing him down the Tower stairs. For his defense, Head Auturno claimed that his victims were poisioning his wine and hiring wizards to interfere with his dreams. Starvation was a horrible way to die. I did not understand why Head Auturno could not be satisfied with simply being a famed for his valor and diplomacy in helping Marsea acheive the long standing peace with Szarvis brought by the Treaty of Cretius. 

My maid interrupted my brooding by announcing guests. I put the book away, spent from my morning's readings, glad to turn my mind to something else. I could not have dreamed of a more pleasant distraction that the three people who entered my office. I stumbled from behind my desk to embrace Sophia. At that moment, I could have embraced Timmon as well, I was so happy to see him alive, healthy, and standing before me. I could not believe my fortune. Propriety, as this man would once have said, be Taken. Timmon lived, what else could possibly matter. I controlled my explosive overwhelming joy and let him kiss my hand. Eugenio greeted me bashfully, then followed my maid into the rooms where my boys would be. 

"Come in, come in, please. It is so good to see you. How long have you been in Deyalorn?" I led them both by their hands to chairs and fussed over refreshments. I could neither stop babbling nor beaming.

I had last seen Timmon stable but weak in Niev, his organs still exposed to the air, death by fever still a distinct possibility. Sitting in my office, he was not as muscular as he normally was, but he looked strong and able. He had lived. By the Preserver's grace, my efforts had allowed him to live. Carlotta had told me, of course, that Timmon had recovered from his injuries and that he had been sent home. Knowing that he was well, and seeing him sitting in my rooms, a mere few feet from me were as different as night and day. I forced, with great effort,  my giddy exuberant mind to turn from the contemplation of my friend to the task of pouring the wine I would spill if I did not pay more heed.

Sophia, I had noted, was showing, not as much as she should be given the late stage that her pregnancy was supposedly in, but that was none of my concern. I had not expected this visit. I had heard of their arrival in Deyalorn. Until they appeared before me, I had told myself that I was grateful for the distance they had kept from me. There were many good reasons for this. I did not know what I would tell Timmon about the lies that had been spread about my family. More importantly, I did not know what I would do if he pressed me on the half truth that lay in the center of it all. I did not want his family involved in this battle against my husband. I knew Timmon could not help entangling himself if he knew my situation. My friends' household was a fragile construction, a house of cards that could topple at the slightest disturbance. Sophia had a child coming. Associating with me in my current state could only bring them trouble. With a great force of will, I had decided not to visit them upon my release, though I knew Eugenio had regularly visited my boys while I lay imprisoned in my false confinement.

Seeing Timmon before me, whole and hearty, every valid reason for keeping my distance seemed as insubstantial as the morning fog. How could I possibly think of turning my friend from my door?

"To what do I owe this vist?" I asked, seating myself before them. 

Sophia left off surveying my office to answer my question. "We have not yet paid your son our respects, Nisrita." 

Timmon simply stared at me intently. I ignored his gaze  and smiled at Sophia's flimsy excuse. "There was ample opportunity for that at the Tower."

"A first meeting among close friends should be more private, don't you agree?" Sophia answered easily. I beamed. I should not have been so pleased to have them as my guests, but I could not help myself. I could hear Eugenio's happy voice above the squeals of my two boys. They were outside again, playing. I wanted our families to grow up close. I wanted my boys to know Timmon's goodness. It wrong of me to want it under the circumstances, but I could not suppress my joy.

We sat for a time and talked. I heard of Timmon's offered promotion, and Sophia's dream of living in Lir. I recieved news of Timmon's family, and that of Sophia's. Timmon would leave Sophia to deliver her child in Deyalorn's Tower. She would winter in the Romino estate, then join Timmon to the south in spring. I explained how I obtained a Master's office and work space. I showed them my new living arrangements, and the small menagerie I had put together. I told them my hopes and ambitions for my new work, and laid bare my frustration at not having access to Deyalorn's library. I may have a Master's work space, but I was neither a healer, nor a member of any Tower. I could not visit the country's largest library. I spoke freely of my plans, my hopes, my fears. I was giddy with the pleasure of their company. If I could confide completely in anyone, it was with these two. 

Rosa brought Lucretious to me when he awoke, as instructed, interrupting the happy reunion. Sophia took him immediately into her lap and started cooing. Timmon sat beside her, looking on with definite interest, but also timidly, as men often are near infants. The domesticity of the scene pained me, though it had no right to. There had been a dream, in a past much less distant that in seemed, in which I sat in Sophia's place, pregnant with a child to carry Timmon's name. I had dreamt of splitting my time between Deyalorn's Tower and the Barony of Romino, of bringing the gift to that barren house, of having a husband who would look over my shoulder protectively at our child. It had been too much to ask, it seemed, that the Preserver guide me on a path free of bastards and lies. 

"Lyca." Timmon said. Sophia laughed. I shook the visions of paths not taken from my mind. That dream had vanished for me when I married. It was someone else's to enjoy now. I had no right to it.

"I am afraid your son has a nickname in our household." Sophia explained. "Your miracle has astounded Nubo. He refers to the child as nothing other than Lyca. It means impossible in his tongue, I am told. My good husband has fallen into the habit of calling the new duke by that name. It is a pretty word. Initially, I had thought of using it to name my own child." Her eyebrows danced in amusement, "but that seems to be impossible now."

She handed Lucretious back to me and left me alone with Timmon, offering some excuse regarding her son. "Lyca, Timmon?" I asked.

"Is Nubo wrong? This child cannot exist given the circumstances of its birth."

I rose to hand the boy back to his nurse, and sent her out of the room. "What are people saying, Timmon?" I asked urgently. Too much depended on the secrecy of this child's orgins. If Carlotta had told anyone...

"There are some," Timmon began slowly, "who know you well enough to know your dislike of the old magic. They think is unfair for you to be discredited as a hypocrite by your husband's friends. The grief and desperation of a mother giving birth to a child born too early to survive is a powerful idea. It touches the hearts of even the many hardened politicians. There are fewer who know the dark side of your personality. They surmise that you knew of this child, practised with the intent to destroy it, guessing that you no longer have access to the White Tower's discretion in Cortan. They say you failed, by the grace of your gods, to destroy it and yourself. They further speculate that your husband forced you to agree to Carlotta using the old magic to save it, or gave you no choice in the matter."

Timmon was not explaining an impossibility. He made the secret sound safe. "What do you think, Timmon?'

He shifted uncomfortably. "I do not know what to think. It is possible, I suppose, that you had no choice in the matter, but Sophia assures me that Carlotta's current work is inspired heavily by yours. Carlotta did not save your son alone. I know enough of your domestic life to know that there is little your husband could threaten you with to force you to the mushrooms. Your sons, perhaps, but I would think he would leave that for something more worthy of punishment in the public's eyes. I know you well enough to know that if you intended to destroy this child, you would not change your mind upon seeing him born. If you had intended to simply destroy yourself, however, without thought to the child, and failed, I could see you possibly changing your mind about the child's fate. Seeing you know, it does not appear that you had any intentions of causing yourself harm. But Niev was several months ago."

"You do read minds." I laughed nervously, trying to lighten the sudden solemn mood of this interview.

"No, I simply know how men think. Will you put my mind at ease?" Timmon persisted.

I had lied to Carlotta. The words came easily. They stuck in my throat now. I could not put Timmon's mind at ease. Whatever the stakes, it seemed I could not lie to him. "I was pregnant in Niev." I began without looking at him. "I did not know it then, though I was not as vigilant in checking as I should have been. It was reckless, and I have learned from my mistake. I lost the child, and nearly my life. Lucretious is not mine."

"I see," said Timmon, contemplating the wall behind my head. "This is far different from anything I would have guessed."

I had not found myself in a believable or probable set of circumstances. If I had, it would not have been so easy for my husband to spread his more plausible lies about my family. I waited for Timmon to voice his disappointment. It did not come. Neither did anger, or a judgement on my foolishness. Timmon sat still and silent, his eyes unfocused at the space behind my head. I wished I could read his mind as easily as he had read my motivations surrounding Lucretious. "I am sorry." I said when I could not take the silence any longer.

He shook his head abstractly. "I will talk to Duke Ergino about leaving his service for the crown's," he replied. "I nearly lost my life in Niev. I have a child coming. These things tend to effect a man's courage."

A thousand curses on Timmon's excuses. If I had the decency to not lie to this man, at the very least, he could spare me his false reasoning for his acts of foolish heroism. I lost my temper. "Absolutely not, Timmon. You and your family are at risk in Deyalorn. I am a danger to you."

"I made a mistake, Nisrita," Timmon conitued solemnly, remorsefully, "last time we were both here. I left you alone to struggle with the man you now call your husband. I will not make that mistake again."

No. No. No. We could not both be thinking of the same events. We were both separately married now. We both had children. That dream had vanished. "I am surrounded by cowards and fools," I declaired. "Those that can protect me from my husband chose not to. Those that can do nothing against him throw themselves uselessly in front of his blade. I will not tolerate this Timmon. You are going to Lir."

"Is that an order?" Timmon asked, amused by my outburst.

I was done with this interview. I was furious at Timmon's inability to see reason, and think of his family. "In as much as I am your duchess, commander. It is."

The laughter froze on Timmon's face then vanished, leaving a stony fascade. "I see, your grace." He rose slowly. "I wish you a good day." He took my hand, kissed it coldly, bowed formally and walked out of the room in search of his family. He left me at my desk, surrounded by my books, feeling more alone and frightened than I had since Lyca's recovery.


\begin{comment}
I approached Master Alyosus's house with some trepidation. I had last seen Timmon stable but weak in Niev, his organs still exposed to the air, death by fever still a distinct possibility. Carlotta had told me, of course, that Timmon had recovered from his injuries and that he had been sent home. Knowing that he was well, and seeing him face to face were as different as night and day. I knew little about his current state of mind or body. I guessed that he had not come to my confinement chamber for reasons concerning the politics I had embroiled myself in, but I could not erase the fear that he was angry with me for my negligence. Everyone else was. The lies I helped spread about me were not inaccurate in one crucial aspect. When my carriage reached the Joris residence, I had still not made up my mind as to what I would reveal to my friends.

Master Alyosus had a modest house, consisting of five small rooms above the servant's quarters and a quiet garden. I arrived on a pleasant summer afternoon. The steward ushered me into the garden where the household awaited me. My two boys immediately wrapped themselves around Eugenio's legs and the three tumbled into a heap of laughter in the grass. Nubo and Lucretious's nurse Rosa chased them away from the table around which Master Alyosus, his wife, Mistress Valira, Timmon, and Sophia stood to greet me. 

I tried not to stare at Timmon. He was not as muscular as he normally was, but he looked strong and healthy. He had lived. By the Preserver's grace, my efforts had allowed him to live. I greeted my company with more cheer than was my habit and seated myself before Sophia. I could not believe my good fortune. A month ago, I had not dared to dream that I would see him again any time soon. I had assumed that our lives had just moved to different parts of the country. Yet there he sat, just on the other side of his wife. He was somber, but I had expected that possibility. He was alive and in my company. For the moment, that seemed enough. 

I sat over light refreshments in this giddy state between Master Alyosus and Sophia, speaking about the battle with Niev, about Master Alerio's talks, about the possibility of Timmon's promotion to General, Sophia's plans for her new child, the quality of Eugenio's tutors. Timmon sat darkly throughout the conversation, speaking as little has he needed to be polite. Sophia glided the conversation away from her husband. She was good for him, I reflected. She could take better care of him that I ever could have. In spite of today's circumstances, I knew she made him happy.

Eventually, Master Alyosus and Mistress Valira retired, giving Sophia, Timmon and myself the privacy that had been the entire object of this visit. The children had long since been chased inside.

"I hear you have a Master's office in the palace, Nisrita." Sophia said.

"Who told you that?" 

Sophia extended her long fingers elegantly onto the table to arrange a stray petal on the flowers decorating the table. "I have my informants, as my husband does."

"Carlotta" I guessed. Sophia smiled her acknowledgement. "What else has she told you?"

"Surprisingly little, given how closely you two have been working. I must wait to hear from you."

I glanced at Timmon sitting quietly by his wife's side. He was alert, just silent. I decided to start with the easier pieces of news. "I am looking for someone to tutor me until I can attain my rank as healer." 

Sophia laughed. Timmon's scowl deepend. "That is ridiculous, Nisrita. There are few in the land who can teach you much, none who are more accomplished." 

I held my toungue about my discussion with Master Alerio. "All the same, I am not currently a member of any Tower. I can do little in my palace office if I cannot gain access to the books in Deyalorn's library." 

"I will ask father to take you on." Sophia said, as casually as she would ask her maid to help her dress. 

I blinked. I had not expected this to be so easy. "He will take me on?"

"The Regent was born a Joris. As long as the crown backs you, my father will have little to lose."

I beamed at her. "Thank you, Sophia. I am in your debt."

"No, your grace, it is we who are in yours." I would not have considered anything I have done for Timmon or his family a debt that needed paying off. Timmon's mood had darkened further. I did not press the issue. "What other news do you bear?" Sophia asked lightly.

"I am training again," I said to Timmon, in an effort to cheer him. "I have made arrangements for a private room where I can climb and pick up my weapons without causing a scandal."

"So soon?" Timmon asked sharply. 

"Yes, Timmon." I hesitated a moment before adding "I was never as ill as I was reported to be."

"You have just given birth, Nisrita, you should not -- " he stopped himself, and studied me. I thought he would tear through me with his gaze, like a flame through a piece of parchment. "Nisrita, why did your husband keep you in the Tower these last several months?"

I had not expected to be asked so directly. "As you point out Timmon, I have just given birth." 

In responce, Timmon excused himself abruptly from our company. I sat in stunned silence. Sophia sighed patiently. "Have I done something to offend him, Sophia?"

"No." Sophia pursed her lips. "He does not like feeling powerless. He knows something is wrong, and he cannot help you."

There was something in Sophia's expression I did not like. "I am sorry, Sophia. I never intended to interfere in your marriage."

Sophia's consternation dissipated into her polished smile. "All men have their faults, Nisrita. Timmon has fewer than most. Tell me, is the child yours?"

I balked at the abruptness of the question. "Who else knows?"

"Only those who know you as well as Timmon does," she said airly. She relieved my confusion by adding "Timmon refuses to believe that you would use the old magic. He refutes the argument that it was a desperate act of a mother's love for her child. He saw you want to destroy your two boys when they were concieved. You took your talisman off. I have convinced him that you and Carlotta could not have done what you purportely did without more preparation than you possibly had, or the old magic. Timmon does not know what to believe. Nubo, when he heard the young duke's name, nicknamed him Lyca. It is a Liri word that, I think, means impossible. I would have used the name for my own child," Sophia looked down at her stomach, too small for how long she should have been carrying Timmon's child, "but the name, in this household, seems to belong to the new duke now."

I did not know what to say. I had not expected to be the subject of such detailed household gossip. The confession stunned me.

"Whatever your husband threatened you with, this household will not betray you." Sophia took my by the elbow and led me inside. She left me in a room occupied by Timmon playing chess against himself.
\end{comment}


\begin{comment}
When I heard the door open near noon, I opened my mouth to tell Peno that the glassware I had sampled was of sufficient quality, and that my list of requirements were on my table. When I did not hear the old man's shuffling steps enter the room, I closed my mouth and looked up from my reading. "Timmon!" I gasped, putting the book roughly aside. At that moment, I could have embraced him as I would have embraced Carlotta once when she returned from her school vacations. He was alive, healthy, and standing before me. I could not believe my fortune. Propriety, as this man would once have said, be Taken. 

My guest was not smiling. "I had expected to see you recovering your strength, not in your healer's robes, reading at your window."

"Come in, Timmon." I took his hands and led him to a chair, ignoring his surprise. I had a lot to explain, and little I could say in the presence of my guards. "Please sit. How did you get permission to see me at this time?"

Timmon allowed himself to be fussed over. I had last seen him stable but weak in Niev, his organs still exposed to the air, death by fever still a distinct possibility. He was not as muscular as he normally was, but he looked strong and able. His long life was more certain thanLucretious's was. He had lived. By the Preserver's grace, my efforts had allowed him to live. "The Joris name opens many doors in this Tower, duchess. How are you?"

I did not know how to answer that question. I did not know how to speak in the presence of this living, breathing, healthy man before me. Carlotta had told me, of course, that Timmon had recovered from his injuries and that he had been sent home. Knowing that he was well, and seeing him sitting in my chamber, a mere few feet from me were as different as night and day. "I am well, Timmon. Recovering. I will be allowed to leave this room in another week. I have kept myself busy, which has helped. Carlotta and I are working together now. It has been good to have her company this past month."

Lucretious whimpered his lonely awake state, stopping my mindless babble. Timmon watched Rosa pick up and clean the child, then looked at Healer Isobel sitting in her customary corner by the door. She scowled at her sewing, no doubt upset at this irregular visit.. "What is the young duke's name?" he asked gravely.

"Lucretious." 

Timmon cleared his throat. The reference had not been lost on him. The took the child from his nurse and returned with him to his seat. "Lucretious is an ellaborate name for one as small as this." He put the child on his chest and watched it search for food on his chest. The image brought back a long forgotten memory of a dream I had once held. In my dream Timmon sat in this very Tower, holding my child on his chest. Our child. I had had my fill of bastards and lies. It did not matter. That dream had disappeared so long ago."There is a Liri word, Lyca, that means something akin to an impossiblity." I blinked away my tears, and met Timmon's intent gaze. He was trying to tell me something. "Given the young duke's circumstances, do you not think that would be more apt?"

I gulped. The intesity of his gaze did not refer to the child's survival. Was it possible that he knew? Ernesto had claimed him able to read minds. I had thought him to be exaggerating. "It is a pretty name, commander. I would not object to you calling him by it."

He nodded. Timmon had much more experience at holding conversations where the meaningful words were not said. I did not know if I had conveyed my meaning to him properly or not. I watched him softly kiss the child on his head. "It is hard to imagine that I will have one like Lyca of my own soon." He handed me the child. "For what it is worth, duchess, I admire what you have done. I do not think I would have advised Sophia to fight so hard for Lyca's life, had she been faced with a similar situation." 

"I pray she never finds herself in such a situation, commander." I struggled to grasp his meaning. He must know. Why else would he go through so much trouble to tell me that Sophia's child was not his own, a fact that I had guessed and did not care about?

"I doubt that she ever will, duchess." Timmon rose and kissed my hand. "I will leave you now. I have disturbed your rest enough for today." He left me with a silk padded Liri jacket for the child, far more elaborate than the embroidered shirts he had given my boys. 

It took me several minutes after his departure to realize that Timmon was not confident that Sophia would never deliver before her time. He knew, he wanted me to know that he understood the difficulty of my decision."


\end{comment}

\begin{comment}

My first firm memory of that time is of Healer Sana bringing in my sons. In my mind, it was an etherial day, with bright sunlight streaming through the windows, and the room smelling of wildflowers. My children played with their toys on my bed while I watched them weakly. I do not think I had the strength to roll over on my side at that point. I may still have been fevered. In my mind's eye, their nurses ushered them out of the room when I grew weary, though I know that to be false. No one other than my husband and my half-brother, the Regent-Consort, was allowed to see me at that point. 

The next memory that I have is of Griswold and Regent-Consort Dario arguing. Healers Sana and Isobela attended me almost as silently as Ernesto once had. The sound of the men talking disturbed my sleep. 

"You are making a habit of this, Duke Griswold. I do not like it." My half brother said.

"You cannot blame me for this one, your highness. She must have known she was pregnant. If she chose to practise inspite of it, I cannot be held accountable."

"I warn you, your grace." My brother said coldly. "This is the third woman in my family you have endangered with your seed. You live only because she does."

"She is my wife, your highness." I could hear the anger in my husband's voice, held in check only by his habits of attending the gravely ill. "You have no right..."

"Nisrita is a national treasure" the Regent-Consort bit off. "If you cannot keep her skills safe, then you may not have her for your wife."

I heard a man rise angrily, sending the chair thudding heavily to floor. I must have startled at the sound, or murmurred at that point. I heard my nurses usher both men out of my room. 

From that point forward, I become more certain of my memories, though there was not much to recollect during those days. I had a few more visits from my sons. My boys were always led in and led out by one of my healers. As my strength grew, I spoke to my healers about day to day trivialities. Eventually, I grew strong enough to spend part of my days sitting at my window. My husband never visited me again. Neither was anyone else allowed to see me. I learned that while my presence in the Tower was not unknown, my location, and the cause of my confinement was kept secret. Whether this was to save embarrassment to the crown or my husband, I did not know.

It surprised me, therefore, when I awoke to the sound of familiar footsteps in my room. My heart lept at the thought that he still lived. I did not dare open my eyes. I half feared that the footfalls would disappear into a dream if I did. The footsteps faltered as I stirred. I held myself still and prayed that I had not been dreaming. It had been barely two months since I had seen him last. I could not believe that he was here so soon. The footsteps resumed. I listened to them cross my room, once, twice, thrice, before speaking. "How did you find me, Timmon?" I still did not open my eyes. I was afraid to look at him. 

The footsteps stopped abruptly. "It took some doing," Timmon said eventually. "I know more people in this city than your husband does." I heard him walk to a chair and sit. I lay in my bed and smiled, comfortable in the knowledge of his proximity. I did not need to speak to Timmon. Words would only interfere with my joy. We were both alive. That seemed to be more than enough for the moment.

Timmon cut the happy silence shorter than I would have liked. He spoke slowly, as he did when he knew the destination of his line of questioning. "There is a rumour circulating, about a miscarriage. Is it true?" I took a deep breath and let it out slowly. Then I nodded." Did you know while on the field?" I shook my head. Timmon tisked impatiently. "Will you look me in the eyes and tell me you did not know when you healed me?"

I could not look at him. I was afraid of what I would see. I was afraid that he would see through me. "No, Timmon. I cannot. I am sorry."

He sighed. I heard him sink back in his chair. He sat in silence. This time, he let the silence hang between us for much longer that I would have liked. I hated knowing that I had disappointed him. "Duke Ergino," Timmon began at last, "has offered to make me a general in Turina. Duke Griswold has spoken in favour of my becoming the ambassador to Lir." He shifted his weight accusingly in his chair, stealing the congradulations from my lips. "I will not go to Lir."

"You must, Timmon" I protested. I opened my eyes to push myself to a sitting position, and froze. He sat rail thin before me. The skin on his face hung low to form two unseemly jowls. His clothes, which he chose with care and pride, had been hastily and poorly tailored to fit a man half their intended size. Gone were the proud broad shoulders and chest of an accomplished warrior. The man before me had aged before his time. Only his hair, which still fell in  abundant soft curls about his face and shoulders,  spoke of the man he used to be. The  rich mane only served to make the rest of him seem more shriveled and shrunken. 

I saw Healer Isobela rise sternly from her usual position at the foot of my bed. "I am sorry, Isobela. Please let him stay. I will not excite myself further." I sank back on my pillows. Why had this man come to Deyalorn so soon? He was still in need of recovery. I did not dare ask the question. I feared what I might hear. "Tell me about the rest of the campaign, Timmon."

"I did not see much first hand, after you left, duchess. Rialotte has been razed. Marsea could not keep its army supplied without the town as a base. Our army retreated as best it could to the lands we had won the previous year. By some miracle of --, by some miracle, we had not lost control of the lands adjascent to Lir, though that part of our army was cut off from all resources. It took another three weeks to reestablish the road to Lir. We have left more men to control that area than anyone is truly happy with." Timmon paused to shrug his shoulders. "The battle for Escasaine and Niev will take some years. Duke Ergino wants me to lead it, given General Galderan's failing. I am not a bad man for the job. I know the area well. We will have to retrain our troops and change our tactics if we wish to survive Niev's fighting style."

"And what of your gods, Timmon?" I asked. He looked at me sharply. We did not speak of his change of faith. As he did not wish to discuss it, I pretended I had not noticed, though the evidence lay subtly around him for anyone who cared to notice. "How will you lead the Destroyer's army if you pray to different gods?" I persisted. 

Timmon's eyes passed from me to Healer Isobela, who applied herself with even greater interest in her sewing, then back to me. At last he seemed to convince himself that if he was not supposed to know that I lay in this room, no one could know that this conversation occurred. "Is that why you wish me to go to Lir?"

"Among other reasons."

"Let me hear them."

I sighed. I would have to bring my husband into this conversation. "Duke Griswold wants the crown to permit the Tower to use the old magic. I intend to stand in his way. This will be easier without you here."

"You are doing a poor job of showing me why I should leave you defenseless."

I laughed. "You intend to fight the country's Towers?"

"You are seventeen years old Nisrita, barely a woman, not even a healer yet."

"I have resources," I said, hoping I sounded more convincing than I felt. My husband terrified me. I did not like the idea of fighting him, publicly or privately. I could not let him win on this issue. The idea of Griswold living forever and weilding the gift was too much to bear.

Timmon studied me silently until I looked away, ashamed of my boast. "Master Alerio says that what you did to me was unprecedented."

"Was it?" I asked casually. That aspect of my healing did not matter at all then. Now, it sat barely noticible in this new world where Timmon and I could still sit and speak to each other. 

Timmon mistook my indifference to the act for a lack of appreciation for its gravity. "I owe you my life, Nisrita. I cannot simply leave you here, not knowing..." he trailed off.

"You owe me nothing, Timmon. And you will not be simply leaving me. You will provide a home for my ungifted children away from this city. As for the rest, can you not plant spies among the Liri ambassador's men as easily as you did in Rialotte?"

Timmon rubbed his bony temple wearily with one finger as he listened to me speak. Then he rose stiffly to his feet. "I will consider it, your grace." I gave him my hand to kiss. He bowed formally as he took it. "I am ever at your service."  

"Come visit me tomorrow, then, if your health allows." I wondered if this interview had tired him as much as it tired me. He grinned at my request as he left the room. I would gladly chase that grin off to Lir if I knew that doing so would make the man wearing it truly happy.

\vspace{.5cm}
 ***
 
It took me two days to find the dutchess after I arrived in Deyalorn. After my initial attempts at locating her, I had expected the task to be much harder. The name Joris opens more doors in the capital's Tower than I had originally understood. I was lucky to have Sophia. Or perhaps it is simply that the original terms of our agreement were broad enough to cover life's unforeseen eventualities. I did not reprimand Sophia for her error in concieving a child while I was away in the field. In return, Sophia helped me visit a woman who meant more to me than my permitted lover. My unusual marriage faced its first test, stumbled slightly and survived.

Rishiki, it is said, is a dangerous god to seek patronage from. He can be devious, or he can be generous. Thus far, he had allowed me a wife to serve my kingdom, and a man to serve my higher needs. He had toyed with my life, but let me keep it. He had promised me a child, along with a flurry of whispers and questions to hush and attend to. On the whole, however, he had been kind. I had my life, a household, and child coming to carry my name. I considered myself fortunate.
 
Sophia generously allowed my first interview with Nisrita to be private. After I told her what I had learnt, she categorically refused to visit her at all. I gave my wife the courtesty of not mentioning the duchess unless asked from then on. "We owe her much, Timmon," she said, after I had recovered from my first visit, "as does Marsea. That does not mean I will give my company to a woman so careless with her life."

"I would not be here if not for her," I observed. I could not remove that needle from my mind. I had recommended Ernesto for promotion. There was nothing I could give Nisrita to show my gratitude.

"Neither would several other black riders. The White healers save Marsean lives, as surely as you destroy those of our enemies." She smiled an elegant patient smile that told me to keep my protests to myself for the moment. "I am not saying that your friendship did not cause her to perform phenomenal feats of endurance. That you are here is a testament to her loyalty. I am deeply grateful for that. You must understand, the problem is not that she brought you back from the Destroyer's clutches. The problem is that she healed at all."

I was lucky to have Sophia. She kept me from shouldering guilt that was not mine to bear. She could, however, do little to ease my fears about leaving the young duchess alone while I acted as diplomat to Lir. Nisrita was too erradic and unpredictable. With her living in Deyalorn, I could no longer stand by her side every day to guide her through her illness. Nor could I bear the idea of leaving her for such a distant land, where news of her condition would take months rather than weeks to reach me. Nisrita was not the only factor in my decision. I had been raised and trained as a man of war. Makala was Cortan's diplomat, not I. I did not doubt that Duke Griswold saw promoting me to ambassador as the easiest way to get rid of me without calling attention to himself. It was a more prestigeous post, but serving as General would be more rewarding. Conquoring Niev would not be as easy as we had once thought. Relying on the Hundred Horsemen's tactical instincts had not helped us at all this spring. I could lead my men to victory, or I could sit eating dates and mangos while the men I had trained fought and died. It should have been an easy decision. It was not.

For her part, Sophia held no strong opinions over which carreer I adopted next. It was my decision alone. She simply asked that I provide for Eugenio if I expected her to accompany me to Lir. Sophia was a reasonable wife. We kept our domestic life, for the most part, within the confines of our arrangement.

I visited the duchess daily, bringing her news of the Marsea, or reading to her from my two Liri texts. Her nurses, she assured me, were bound to secrecy as to the nature of her confinement, and the nature of her visitors. She was a prisoner to her husband, but her guards belonged to her brother. Nisrita told me she enjoyed the music of the words as the poets had strung them together on the page. It pleased me to tell her the stories. It pleased me to tell her of Lir. She was as curious about that aspect of my life as she always had been about anything I wished to share with her. There were things I could tell her that I could not share with anyone outside my servant Nubo. I would miss her when I left Deyalorn, no matter where I went next.

Sophia, for her part, took Eugenio to visit the young dukes daily. I attempted to visit them as well, but found I did not have the stamina to keep up with the two toddlers. The twins admired Sophia's boy, as was natural at this age. Their friendship would serve Eugenio well if I could keep them close. With the young dukes in Deyalorn, I wondered if I could better serve my family as General or Ambassador. I had promised the duchess to watch over her sons. I could do that better as General, where they could be in my care and in their father's lands. Nisrita wanted more than that. She wanted me to take them to Lir, away from her husband's reach. It was not right, what she wanted. The dukes should know Cortan. They would rule there eventually.

A week into our arrival at Deyalorn, Sophia invited Healer Carlotta to dine with us in her father's small house. She had accompanied Healer Ezaro from Cortan, arriving the previous day. Ezaro, presumably came to join his mentor Griswold. I did not know why Healer Carlotta accompanied him. My mother-in-law ushered Eugenio away after dinner, and I found myself surrounded by the company of gifted: Healer Carlotta, my wife, her only gifted brother, Velo, a captain in Deyalorn's army, and my father-in-law, Master Alyosus.

"Did you know, Sophia," Healer Carlotta began, when we had gathered in a sitting room, "that Master Alerio has started lecturing about Nisrita's accomplishments her last night in Niev?"

I fingered the rim of my cup and looked out the window into the dark of the night. I did not wish to be a part of this discussion that threatened to touch on my anatomy. I saw Sophia observing me out of the corner of my eye.

"I congradulate you, Timmon," Velo clapped me jovially on my back. "You will be known in every Tower in the land."

"I am blessed many times over, brother," I replied, "by a beautiful wife, favour from the crown, and the most famous intestines in Marsea." The company laughed. I returned to my view from the window. I had asked Sophia, once, to explain why there was such excitement over my survival. She was discrete, as ever. Even so, the details of the acts done to my body were more graphic than I needed. I was surrounded by a company of healers. The conversation promised to impose upon the privacy of my organs.

"Have you heard him talk, Carlotta?" my father-in-law asked.

"I have, Master Alyosus. I think he makes too much of the accomplishment. The duchess is a powerful woman in her prime. What she did cannot be reproduced. It was an anomally." My wife gently cleared her throat. "I mean no offense, commander. I am grateful that Duchess Nisrita succeeded where I failed. I mean this on a purely academic level."

I waived away the apology. "I appreciate your efforts, healer, and I take no offense. You are among your kind. Speak as you will."

Carlotta returned her attentions to my father-in-law. "Master Alerio is showing this as an example of the successes of the traditional ways of teaching and training healers. He is an old man, fighting a lost battle against the changes that must come."

"You cannot explain this with power alone, Healer Carlotta" Velo objected. "There must have been significant skill involved. I have seen stomach wounds. We do not even try to save a man who has been gutted severly enough." 

"I do not claim that Master Alerio's guidance, experience and skilled aid did not play a significant role, captain." Healer Carlotta explained. I let the conversation drone around me and contemplated what my life would be like in Lir, away from these callused discussions of patients within their hearing. During my months of courtship, I visited Cortan's Tower often, finding myself surrounded by Sophia's friends. While my wife always keeps her genteel composure, regardless of her surroundings, she did not curb her colleague's discussions to accomodate my presence. It took me time to adjust to this domestic life where I would sit quietly in a conversation led by my wife's companions. Thankfully, she did not impose this awkward interaction upon me after marriage, until now. She knew I did not like it, and not simply because it caused her to leave her tradional domestic role. I did not enjoy the flavor of these discussions.

A life in Lir would certainly save me from enduring the company of healers. It would also deprive me of their art to which I currently owed my life. An ambassador's rank would let me provide much more comfort for my family than a general's. Sophia would be the perfect ambassador's wife. She would enjoy the challenge of showing Deyalorn's elegance of velvets and brocaids to be superior to the fancies of silk and silver that adorned Lir's nobility. But we would live without the gift. Marsea's ambassador would have a healer in his staff, I was certain, but it would like life in my father's home again, where minor injuries would have to heal on their own, and the gifted would only serve men of sufficient rank. I laughed at my dismay. I had not thought that style of life a hardship once. Training with Deyalorn's armies, then living in the shadow of Cortan's Tower had spoiled me. I would not be leading armies in Lir. My life would be lived in court, surrounded by the great works of Lir's artists, and scholars. The risk to my person would be much less than it was here. I had never imagined that a sedentary life could hold such allure for me. The freedom that Lir's society allowed my kind: the expectation that every married man should have a close male companion at the very least to fulfill his emotional and spiritual needs that a wife could never hope to satisfy, if not his physical needs as well. Until visiting Lir, I had not believed that I could be anything other than miscast. Rishiki's price for welcoming me into his fold seemed to be that I give up my sword for a silk glove. Five years ago, I could not imagine making such an exchange. Five years ago, Makala still lived, and I knew nothing of Lir.

I heard Griswold's name mentioned a few times, and the Regent-Consort, as well as Nisrita's madness. The young duchess had mentioned that there would be a political battle between the Towers and the crown. I wondered if I was hearing the precursors to that storm in my father-in-law's house. Nisrita was right, I was a soldier from a barren house. I could do little to help her in this fight she would pick. I prayed that her gods would see her through it safely.

"Do you know anything of the duchess's new illness, commander?" Healer Carlotta's words brought me back to the conversation at hand. "Commander?" she asked again when it took me a moment to regain my bearings. 

"No, healer. I know only that she is in Deyalorn's Tower." Healer Carlotta looked at me expectantly. "From what I saw of her in Niev, I doubt it is her old malady." I lied authoratatively. I did not like to hear Nisrita's accomplishments whittled away by allegations of madness.

"Are you tiring, my husband?" Sophia asked sweetly.

I took my cue. "Yes, I believe I am. I beg your pardon, healer, father, Velo. My strength is still not what it used to be. I wish you goodnight."

Sophia walked with me, a model doting wife, to the chamber we awkwardly shared for the sake of propriety. "You are offended," she said.

"No, not for myself." I lay on the bed and let Sophia fuss at me, her long elegant fingers arranging covers, propping pillows and relieving me of my outer garments.

When she finished, she said "Carlotta is entranced by Griswold's teachings. He is a charming powerful man who has a way with women." I looked at her sharply. Sophia laughed beautifully at my mistrust. "You have nothing to fear, Timmon. Even if I wanted the power he has to offer, I would not threaten our household by becoming close to him."

"So Healer Carlotta has fallen out with her friend over the Duke?"

"They are certainly less close." Sophia gazed thoughtfully at the wall behind my head. "If the young duchess decides to speak against her husband, I will ask father to speak to his cousin."

I raised my eyebrows in surprise. "Will the Regent listen?" The Regent Elena was a gifted woman born to a powerful Joris line. My wife's family were only distant relations.

"My father does not have much sway with his cousin," my wife admitted. "However, it seems the Regent-Consort supports his sister. My father's words may not even be necessary under the circumstances."

I shook my head in disbelief. My wife was not normally a political woman, prefering to taste the pleasures of court while shunning the intrigue. "You would do this for her, Sophia?"

She smiled at me patiently and shook her head. "No, Timmon. I do this for us. It seems that we will soon have to choose between Duke Griswold and the duchess. You have decided which side we choose. I can only try to help her win."

I could not argue. I had chosen. Perhaps Nisrita was right. When this storm came, it would be easier for her, and my family, if I were not present. My marriage, and the child Sophia carried, sheilded us from the false scandal of my former relationship with the duchess. I would not win me her husband's favour, not if I saved Duke Griswold's life a hundred times.

"I should return to our guest, Timmon. Do you need anything?" Sophia asked gently.

"Will you be long?"

"Possibly. I have not seen Carlotta for several months."

"Then ask Nubo to attend me, if you would."

"Of course." Sophia glided benevolenty from the room. "Good night, Timmon." 

\vspace{.5cm}
***

I awoke early one morning to the sound of the Tower's Bells announcing the birth of a noble child. I vaguely wondered if my half-brother had a new child, or if this was to mark the birth of one of his courtier's children. I found my complete isolation frustrating. Even with the pieces of news that Timmon and my nurses brought me, I knew nothing of the world around me. 

I was a prisoner, not just a patient. My body had healed, not to its full strength, but certainly to the point where I would welcome some fresh air and activity. I did not know why I was held in secrecy. My healers could only tell me that I was to be held for some months longer. They were good women, and pleasant companions. They were also faithful guards. I tried for a day to gather information from them about the nature of my confinement and my husband's or the crown's intentions with me. I only succeeded in making them uncomfortable. They assured me that they would not tell my husband of my actions or my guests. They made no such assurances about my brother, and told me nothing else. Eventually I stopped pressing them. There was no point in makeing my gentle guards hostile. They brought me books from Deyalorn's library, and fresh cuttings from the garden. Conversation with them about the few subjects on which we were allowed to speak was one of the only way I had to pass my empty hours.

I rose from my bed and dressed. Timmon's regular visits forced me to pay attention to my presentation. If I had not had his visits to look forward to, it would have been easy to spend days on end without even bothering to sit at my window.

My husband entered my room as I brushed my hair. "Congradulations, your grace. You have given Cortan a son." The comb fell from my hand and clattered to the floor. My husband ignored my shock. "I will send him to you shortly. He was born, unfortunately, before his time. He may not live. You are too unwell to admit visitors at the time. It was not wise of you to practise while pregnant, and the delivery has been difficult. I forgive your carelessness with Cortan's heirs, this once. I will not look upon a repeat performance kindly again. Once your physical health recovers, and your confinement ends, you may leave this chamber."

I struggled to take in this new situation. I looked at Sana in the corner. She looked down at her feet, embarrassed to overhear this exchange. "How, my liege?" I whispered.

"I will bring him to you. Take a moment to compose yourself." My husband left. 

Sana picked up my comb and finished tying my hair. "Did you know about this?" I asked her.

"I beg your forgiveness, your grace." She replied. I was too stunned to be angry with her. I was too stunned to be angry even with my husband. I was simply numb. "May I start a fire to keep the infant warm?"

I startled, at the sound of her voice. "Yes." The sight of her tend to the medical needs of the child cut through my shock. "The morning's bells rang for Cortan?" I asked.

"Yes, your grace."

"Who is the mother?" I assumed the child was Griswold's bastard.

"I do not know, your grace."

"Does the Regent-Consort know about this?"

"Yes, your grace."

I took off an outer layer clothing and sat by the fire to think. I was to legitimize the child of one Griswold's innumerable lovers, with the crown's knowledge and consent. Who was the mother? She must be important to my brother. I could not fight this. My husband had announced this birth to the city already. The child existed. Any attempts to deny his parentage would only add to the rumours of my madness. I wondered how I would live with my husband's bastard in my house. I felt myself anger at my husband's actions, then a weary bitterness tamp it down. I knew of my husband's affairs. I did not think flatter myself to think that I was any different from his other women in his eyes, unless it was that I carried the disadvantage of being shackled to him in the eyes of the gods and the crown. I knew that Griswold cared and provided for his bastards. It should not surprise me that he would want to raise a bastard of a noble woman in his own house. 

My husband returned to my room before my bitterness could turn to despair. Behind him, a tiny still bundle entered shortly in the arms of his wet nurse. My bitterness gave way to the demands of my art at the entrance of this tiny child in need of care. I took him from his nurse, unwrapped his swaddling, and examined him.  Healer Sana joined me. He was tiny, more than tiny, his bones were fine and brittle. He had no layers of fat between his bone and skin. His skin was wrinkled, pink and patchy. It was already bruising where his nurse had handled him, though he had only lived a few hours. I had extracted so many pups from their mother's wombs in this condition. I had hoped never to see a living human child alive in this half formed, suffering state, certainly not one I was to call my son. His lips were tinted blue, his breath halted at irregular intervals, as if the child occaisonally forgot the necessity for air. He did not suck my finger when I offered it to him. "Does he eat from you?" I asked his nurse.

"No, your grace."

"Sana, make arrangements for goat's milk and clean cloths for the child." Healer Sana stepped outside to make arrangement. I settled the wetnurse and child by the fire, then turned to my husband. "He should be seen by the head of the Women's Tower, my liege." 

"No. That would draw too many questions regarding the circumstances of his birth."

I looked at the babe by the fire. He looked like he was born two months early, perhaps a little less. That made sense, he must have been conceived just before Griswold left Deyalorn. My husband had joined me less than six months ago. Any healer of the woman's tower would know that I did not birth this child. "He will not live without help, my liege." He may not even live with the Tower's help.

"No." The word was sad. My husband looked at the child with a tenderness he reserved for my sons. "Are you well enough to heal?" he asked gruffly.

"I, my liege?" Did my husband expect me to help his bastard live as well as claim it as my own?

"I will send someone discrete to help you." He shrugged. "Do what you can. I leave his care in your hands."

My husband left me with the burden of this dying infant. He wanted me to attempt to save it. It would be so easy to ignore his desires and let this child slip away. The child would die on its own in a few weeks. No one expected it to live. I would be free of my confinement and responcibility. No one would question ot blame me for my negligence.

I shuddered at the thought. Griswold walked away from the wounded and the dying. I did not. This child had done nothing wrong. I could not walk away from him. I did not want him. I sat at my table and pulled out one of the books from Deyalorn's library, flipping listlessly through its pages. Healer Sana did what she could for the child behind me.

Timmon came at his usual time and positioned himself silently beside me. He knew what the city knew, and had pieced together enough of the rest to not need to ask me questions. He waited until I was ready to speak. It took some time. There were too many questions in my head that begged for his wisdom. I had too many doubts about my duties to the Tower, the Preserver, to Cortan. Timmon no longer held with the triumverate, but he was a good man. I trusted his wisdom at that moment more than I trusted a confessor. 

"Three years ago, Timmon, if things had gone differently, would you have wished me to destroy the children I carried? Not knowing what you know now."

"No." The answer was firm and confident, spoken without hesitation. He had made up his mind on that point already, even if he had not on whether he wished to marry me. 

"You would have raised Griswold's children as your own, alongside your own blood?"

"I would have adjusted." I wondered at his courage that I did not have. He would have given years of his life raising my two boys, while I could not find the strength to give this child a chance to live. "That situation is different than this, Nisrita." Timmon said softly after a time. "Our marriage, I think, would have been of a gentler sort."

He handed me a cloth for the tears that had found their way silently to my cheeks. There was no point in thinking of what might have been, I told myself. Timmon had a child of his own coming soon, and I, it seemed, now had three sons to my name. Timmon seemed happy. Under different circumstances, I could have shared that happiness with him, even been the cause of it. No, too much had passed between us for that fantasy to bear reflection. There were too many witnesses in the room for more words on that subject to be said. 

"Did you bring your book, Timmon? Will you read?" He opened the Liri epic poem about the wind god Rishiki's battle against his brother, the sea god Sinto, and the ensuing great naval battle between the human kings Ayumu and Matsya. I listened to the music of the verses, and Timmon's rough retelling of the events. I knew these stories brought him peace. That day, Timmon's voicing of the tales brought me peace as well. By the time he left, I knew that if Timmon could be brave enough to raise Griswold's bastards, then so could I.


\vspace{.5cm}

"What exactly did you do, Nisrita!" Carlotta screeched as she entered my room late that afternoon. I sat by the fire, holding the infant to my chest for warmth, rocking, and reading over my notes from my months of dissecting. The infant did not even flinch at Carlotta's outburst. There was a significant chance that this was all for naught.

"I made a mistake, Carlotta," I said rising to meet her. I found myself relieved at her anger. I had feared that she could come cold and 

"You were trying to get yourself killed! How could you not have known?"
\end{comment}




\end{document}
