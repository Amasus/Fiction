\documentclass{article}

\usepackage{fullpage, verbatim}

%*****************
% Annotations
\usepackage{soul}
\usepackage[colorinlistoftodos,textsize=footnotesize]{todonotes}
\newcommand{\hlfix}[2]{\texthl{#1}\todo{#2}}
\newcommand{\hlnew}[2]{\texthl{#1}\todo[color=green!40]{#2}}
\newcommand{\sanote}{\todo[color=green!30]}
\newcommand{\egnote}{\todo[color=violet!30]}
\newcommand{\newstart}{\note{The inserted text starts here}}
\newcommand{\newfinish}{\note{The inserted text finishes here}}
\setstcolor{red}
%***************************


\begin{document}

I filled my days in Deyalorn to keep from thinking about the terrifying task at hand. It was difficult at first, but there was so much to do. My mornings I spent with my three children. Woldino and Makala were the pride and joy of my days, as they ever had been. Their nurses brought them into me dawn. We joined the crown for morning prayers then played extatically for an hour in the dewy garden, tromping through the grass, looking under rocks for bugs, teaching Woldino to climb. Makala was less adventurous, preferring to swing on the plank hung from the large elm tree that shaded my bedchamber while Woldino and I climbed the branches above him. When their nurses took them away, I escaped into my world of books and animals and mushroom brews at different dilutions. Sometimes, Makala would escape his nurses attentions, and find me with my rabbits and mice. He was an inquisitive boy, full of questions about why the cats could not be kept with the rabits, watching how the dogs lapped their water wondering why he could not. 

My husband visited my children in the mornings. I could do nothing to keep him from my sons, as he had once kept me from them. I could only put a wall of work between myself and his visits, and not dwell on the negative effect. He did not trouble me further with his company. My brother made it clear to me that he would house me in the palace only because of my research. Inspite of his presonal dislike of my husband, he could not appear to take sides in my domestic dispute. I had no choice but to let my husband win my sons' love. I shuttered my windows when he visited, dulling the sound of his voice mingling with their laughter.

When I tired in the afternoons of the work, I trained in a quiet grove near the palace, away from any eyes of the court, tower, army or clergy that would be shocked by the sight of a seventeen year old mother of three swinging from a series of ropes strung between the branches of trees five times her age. Woldino's nurse brought him to me during this time. He was too interested in playing on the ropes high above his head and my throwing stars. I dulled a small dirk for him to play with. When Timmon went to go to Lir, I would have to find a way to involve myself in the education their father would plan for them. Teaching my two year old son how to throw and climb would do him no harm.

Eugenio still came to play with his two friends, though for the first few weeks of my life in the palace, his parents did not. The Regent, Queen Anyesa frequently visited my children in the afternoons. I spoke to her highness on one of these occaisions, asking for her aid in obtaining a Mastor to take me into his tutelage. She gave me a non-committal answer. It would seem the crown's support for me was very limited.

My evenings I devoted to my brother's court. These were the most taxing hours of my day. During the other times, I could ignore the dangers of this journey I had embarked on, lose myself in the innocence of my sons' world, or in the deliberate and painstaking details of caring for and observing my menagerie. My evenings robbed me of everything I had created to distract me from this political battle, and thrust me into a world I manuevered in uncomfortably at best. While the Tower's desire to seek more power was far from only topic at my brother's table, it was almost the only subject anyone discussed with me. Dario had officially not chosen sides on this debate. It was true that he did not hold an opinion on the merits of the mushrooms or the old magic, though the Queen Regent, I guessed, stood in favour of it. In theory, Dario was open to the opinions of his generals, his priests, and the Heads of Towers throughout his land. In truth, is hospitality and protection for me at this time was not a reflection of his opinion that the old kings of Marsea had been correct, and that the old magic was too dangerous to be allowed to run loose, only loosely controlled by the ambitious men of the white Towers. He balanced the appearance of taking sides on the debate that my presence created by giving freely of his time to Masters of Deyalorn's Tower who wished propound their arguments before him. 

In actuality, Dario had his own reasons for not liking my husband. Supporting my desire to fight my husband on the old magic was, for him, simply a way to cripple an enemy without dirtying his hands. When Griswold returned from exile, my brother originally with to allow him return only on the condition that the Duke not be allowed to take any part in ruling Cortan. The idea that the powerful duchy would be controlled, if not officially ruled, by a man who could not age, terrified him, as it should have. Duke Ergino could not accept this condition attached to his beloved uncle's return. He respected my husband's wisdom, and did not trust mine. The long negotiations on the day of Duke Griswold's return ended eventually with Cortan accepting a man of the crown's chosing to sit on her counsel for as long as Duke Griswold remained in residence. To save his duchy from the crown's interference, my husband spent much of his time elsewhere, testing his ambitions in the nation's Towers. 

In the interst of displaying he crown's nuetrality, my brother held a debate between Duke Griswold and Master Alerio several days after my imprisonment ended. Members of the court filled one half of the palace hall, members of the Tower, the other. In between them, lay an aisle. The court was divided, visually as well as in fact. The colorful displays of Deyalorn's powerful, inflential and rich on one side, the gleaming white simplicity of the gifted they depend on on the other. Not being a member of any Tower, I appeared in blue and white lace, the Duchess of Cortan, relegatd to sit among the ungifted lords. 

At the appointed time, Master Alerio and Duke Griswold both appeared at the bottom of the hall. They walked up the center ailse in their ceremonial robes, dressed in dazzling white linen, heavy with ivory coloured silk embroidery on the arms and chest, pearls adorning the collar and wide sleeves thay hung nearly to the floor, their trains sweeping the ground several feet behind them. We of the White Tower wear no color while serving the Preserver. It is a reminder of the burden we bear, having been chosen to weild the gift. That does not mean we do not show off our greatness in various hues of white. It made me proud to watch Master Alerio walk up the the aisle. His head was streaked with grey, tied back tightly behind his head as it would be in the infirmary, his hands clean and unadorned, the symbol of the Preserver hung prominently from his neck. He represented the ideal we of the Tower are taught to strive to, the son of a long dead duke of Gissal, he appeared a man of noble birth who had given up his political ambitions to serve his country and the Preserver. He had the power of the country's Towers behind him, but he appeared before us unadorned, a sage and a healer, ready to guide Marsea with his experience and save its countrymen with his gift. In comparison to his simple nobility, my husband appeared gaudy. He wore a talisman on his neck, though he could not use it, and a heavy gold and jeweled glinted colors against his white chest, on his hands he wore the signet of Cortan, his black hair flowing behind him as a ruler, not a healer. He appeared young, proud, broad shouldered, a warrior and a Duke in healer's clothing. He walked with no humility, he was approaching his throne. The hall whispered at the disparate images the two men presented. Many had heard Master Alerio speak. In the two weeks he had been here, he had given three different public speeches and several private audiences. Everyone knew Duke Griswold's views. He and his allies had been propogating them to anyone who would listen for the two and a half months since our return. 

Head Ionus of Deyalorn's Tower hit a round brass disk three times with his ivory staff. The sound reverberated off the high valuted ceilings. The crowd hushed. He introduced the two speakers who the entire gifted world had heard of by now, and took his seat at the side of the dias.

Master Alerio spoke first. He reminded the audience of the events recorded in Dario's history of the first few years of Marsea foundation, the deaths of dozens of Deyanesi healer in feuding houses trying to bring down the destruction of each other and King Dorino's men. He reminded us of a time when great feats of magic were done, but none were repeatable. He spoke breifly of King Beldoro's wisdom in forming the Towers as centers of knowledge and discipline, bowed in service to the military and the Preserver, that allowed Marsea to harness the gift rather than let it run itself out wrecklessly on the madness of its uncontrolled strength. It was a history lesson, but it was passionately and powerfully given. I sat spellbound by the story of my country he wove for me. Then Master Alerio moved on to remind the audience of the accomplishments of the Tower. We did not worry about our leader's dying of an accidental and tragic fall off a horse. Our mothers no longer died in childbirth. We had an aggressive army of strong dexterous soldiers who could survive nearly any injury on the field. What did the old magic have to offer in over the benefits reaped by centuries of study and discipline and service? Duke Griswold had performed possibly the greatest feat of old magic since Queen Lucretia set locusts upon King Dorino's lands. He had attained eternal youth. That was an impressive act that every healer of every Tower would like to duplicate. No one could. The old magic could not be controlled. As such, feats performed by it were inherently unteachable and unrepeatable. The old magic was selfish. To practise it went against the very philosophy on which the White Towers were established. The White Tower's of Marsea faced a choice, Master Alerio concluded. They could spend the time and engergies of its Masters as teachers guiding young intelligent students to find ways of regenerating limbs, healing men left for dead on the battle field, searching for cures to crippling childhood diseases. Or, they could take the guidance of the great Masters away from these promising young healers, leaving them to the chaos of their own unguided youthful passions while they spent their time attempting feats to promote their greatness that could neither be repeated nor taught. Some Masters, he admitted, would choose to serve Marsea, and attempt to perform great feats of healing for our soldiers or destruction for our enemies. Many, like Queen Lucretia, woud die in the attempt, leaving the Towers bereft of their wisdom, experience and guidance. The White Tower's Healers were all human, he reminded the court. Who in the audience would not be sorely tempted to seek eternal youth for himself, if he knew the possibility lay out there to be discovered?

It was a powerful speech, well delivered. Master Alerio never stated that the Heads of Towers already played with the old magic in an attempt to perform great selfish powerful feats. He nimbly danced around the idea of the White Tower obtaining power that could not be rivaled by the nobility, the clergy or the army. He sketched a world where ambitious men could do what they wanted, at will, without check to the audience of ungifted around me in broad suggestive strokes, without revealing any of the Tower's dark secrets I was just starting to learn. The colorful half of the court shifted uneasily in their seats. 

Master Alerio sat, and my husband took the stand. He looked so spry and strong compared to the old man's thin wiry stand. It was impossible to believe that my husband was the older of the two. 

"Master Alerio calls the old magic selfish," he began. "He does not know the ways in which it has allowed me to serve my country better." My husband spoke of the new insights into the workings of the gift that he had acquired by tinkering with the old magic. His student, Master Ezaro, the great teacher could not have taught the nation's towers to regrow limbs without the insight Griswold had given him, insight he gained from the old magic. My husband claimed that what he proposed did not mean the choice that Master Alerio put before the court. There was no dicotomy between serving and teaching on one hand, and performing untold miracles on the other. Who would the miracles serve, but Marsea? He was not asking for the mandate of the White Tower to be removed, He simply wanted to harness the power of the old magic as well as the gift to serve Marsea's clergy, its military, its nobility. The Tower would be as the Tower always had been, aiding Marsea's armies, exalting the Preserver's power to its true greatness. It was a tragedy to cripple and stunt the greatness of the Preserver's will out of fear of what can be done with his power. Even under these cramped and resourceless conditions, in this world where he had to work alone and in secret,  he had found a way to boost a healer's power and endurance. 

The room gasped and murmured when he announced this. His research with Ezaro had been a tightly held secret, know only to a dozen or so healers. My brother frowned darkly. I looked at Master Alerio. He sat calmly in his chair, his twitching fingertips the only sign of his surprise and unease. Few had expected Duke Griswold to admit to the illicit activities that occur behind the closed doors of Master's offices. My husband risked not only himself, but the entire Tower structure by his admission. He offered to the ungifted a tempting fruit of the old magic, if only they would allow him to give it. He balanced that against the crown's rage on his head, and the Masters of Cortan and Deyalorn who supported his work for breaking one of the oldest laws of the land.

My husband commanded his audience's attention again and continued. He painted a vivid picture of a world where strong healers in their prime could heal powerfully and tirelessly. Broken limbs could heal in a day. Any soldier still found breathing on the battlefield could be saved. The Destroyer would only be able to collect the souls of the very ill. In my husband's picture, gifted fighters could heal in battle as well as those in the healing corps. Masters need not only observe and guide the younger healers, they could heal as they had when young. The powerful healers of the land could save dozens of lives at a time. What other miracles like limb regeneration lay within a strong healer's power to acheive with the insight that the old magic gave. Could we heal head injuries? Could we heal the old? Could we perhaps slow the process of aging itself? The possibilities were vast and endless. They were impossible without the freedom to access the old magic. He begged the crown to think of the good of Marsea, and not hide behind fear and tradition. The gifted were of Marsea and for Marsea. They would not betray their own country. 

The crowd errupted in a buzz of excitement. Griswold had charmed them, as only he could when he so desired. He stood before them, Duke and healer, promising to bring the entire nation to a better place. He lied to them, of course. He lied cleverly, and subtly so many did not even suspect a lie. Deyalorn's court did not understand my husband's ambitions or his cruelty. I sat dismayed among the excited hopeful ungifted lords chattering about me like so many children. 

Head Ionus took the floor and opened the debate to questions. The Queen Regent stood first. "Master Alerio," she said, "You have been practising for much longer that the Duke. The crown trusts your experience and it grateful for your service. What do you make of these boasts of the Duke. Could the Tower slow the process of aging?"

The Queen Regent is gifted born of a powerful Joris line, though like me, she did not stay with any Tower for long enough to attain the rank of healer. Master Alerio adressed her as a student, his answer scattered with techinical details that were lost on many, including my brother. The jist of his answer was clear. He did not think so, unless it was in one or two instances as Griswold had done for himself.

An white haired barron leaned on his cane to rise from his seat. He asked the Master if he thought that the old magic could extend the benefit of the gift to the old. Master Alerio had to answer that it probably could. The ungifted around me murmurred in excitement. It took some time for Head Ionus to quiet the audience again. He took more questions for the speakers, almost all on the powers that the old magic could grant the country. I listened to Master Alerio answer cautiously and conservatively, and my husband make bold daring promises of what he could give the country if only they let him lead them to victory. My husband had transformed Master Alerio from the noble healer to the the conservative doddering old man clinging to tradition and afraid of change, unwilling to take risks. He himself remained a strong bold visionary holding out the possibility of long health and glory to all of Marsea. It was disgusting. The fools around me all misssed the point.

I waited for the questions to thin, then rose from my seat. "My, liege," I began when given permission to speak, "you make great promises about the powers of the old magic. Our history tells us what it can do. Queen Lucretious died attacking King Dorino's forces, and the Deyanesi fell. What risk do the healers of Marsea undertake if we follow your example?"

Master Alerio gave me a disapproving look. I ignored him. He need not worry, I would not reveal our secret. "There is a risk to the old magic, duchess, as there is to any great endeavour. I do not advocate forcing any healer to serve in any way against his will. Healing is our duty to Marsea. Those who will be brave will reap the glory for their accomplishments, as it is among our warriors." 

I remained standing against the decorum of the debate, and asked my next question before Master Ionus could recognize another speaker. "The duty of a female healer to Marsae, my liege, is two fold. She serves by healing, and she serves by bearing this country the next generation of giftes. The healers of the Tower are not soldiers. If they die performing these great feats of bravery, who will replace them? One cannot train an ungifted to take their place. Will you force the Tower's women to chose between saving lives and birthing new ones?"

I heard the female students in the back of the hall stir uneasily in their seats. On the colorful half of the hall, several women did the same, gifted wives themselves, or mothers to gifted daughters, I was certain. They would bear the burden of Griswold lust for power.

"The extra power that the old magic promises will reduce the need for as many healers on the battlefield," my husband continued of the hushed discontent that filled the hall. "Our gifted women will be able to serve as they always have."

I took my seat. My husband had no way of supporting his claim. I had not evidence to argue it. I had to satify myself with planting a seed of doubt in many minds, and possibly sewn discontent in many households. Gifted women are not as powerless as the ungifted women are. We are educated, we have friends in the Women's Tower, we earn for our families, our husband's marry us in the hopes of the glory a gifted child will bring. We do not shirk any of our duties, but we should know all the effects of our husbands' alliances for and against Duke Griswold will have upon us and our families.

Head Ionus took one final question from a man wishing to know if my husband thought the old magic could touch ailments like blindness. I had not expected the Head to give me the final word. He was of Deyalorn's Tower.

\vspace{.5cm}

The discussion heated over the next weeks. The possibilities and the dangers offered and posed by the old magic became a subject discussed in nearly every household. Everyone had an opinion, impromptu debates broke out in the garrisons and squares. There was much the old magic could offer. There was much it could take away. 

My brother was furious at me for speaking so freely on matters that should be left between husband and wife. I was indecent and unreliable, he claimed. His dukes would not be swayed by a woman who spoke so boldly out of her place. I asked him how many black riders would have died in Niev had I stayed in my place as my husband had wished me to do. Whatever my husband's claims about the old magic, and there was truth to its power to boost a healer's power and endurance, too many would be tempted to do too much. Someone had to point out the problem of replacing the healers we lost due to the reckless risks my husband tempted them to take. My brother accused me of being naive and childish. My points were not wrong. They were simply ill put. I could not be trusted to visit the houses of his court with my brash manner and uncontrolled tongue. If I did not mend my ways, I would be of no use to him.

I let public opinion do what it would. My brother was right. My name and character was dragged through the muddy streets. Some called me shameless, and pitied Duke Griswold for having such a outspoken wife. Others questioned my breeding and my dead mother's name. Yet there were some, mostly among the army, that stayed devoted to me as the Maker's daughter. I met soldiers on the street who thanked me for a finger or a brother's arm or a son's leg. They would be thanking Ezaro, I thought, as the Tower did, but for the fact that Ezaro was foolish enough to remain aloof from the ungifted, like many of my kind. I decided to take my training to the barracks. I found an enclosed guarded, private place to work while Woldino played with his knife below me. I made myself visible to the men as I came and left. I made it clear that I served the army, not the Tower. 

Master Alyosus Joris found me struggling with an extremely aggressive cat one afternoon soon after the debate. My maid's sudden enterance provided enough of a distraction for the tom to pierce my leather glove with its teeth, escape my weakened grip and scramble deep beneath the cabinets housing my glassware. I took off my torn glove, and turned to my maid to admonish her for not knocking first, when I found Sophia's father standing in the open doorway. 

"Master Alyosus," I said, surprised and rubbing my wounded flesh, "I apologize for the state you find me in. I was not expecting visitors today. If you would give me a moment to tend to my wound."

The old man took that as an invitation to enter the room. He looked around while I made preparations to cauderize the punctures in my hand. He apologized smoothly. "It is my fault entirely, duchess. I am disturbing you at your work. Is the creature rabid?" 

"No, master. It is suffering form an expected side effect of my tests. He wants he dark and quiet. He will be asleep in a few hours. I will return him to his cage then."

"Ah yes. I have heard that you are working with the old magic." He turned from his inspection of my caged animals to the table on which I had lit a flame. "The crown has reprimanded Deyalorn's Tower severly for its secret practise of the art. Has it made an exception for your grace?" He took the red hot needle from my left hand while holding my right hand firmly in his own. He thrust the hot metal deep into a puncture wound before I could protest.

I whimpered in the unexpected pain. "This is entirely unnecessary, master. It is a small wound, I can tend to it myself." I was not entirely comfortable with a Master of Deyalorn's Tower questioning me about my research. I did not know where his alliances lay, or how many details of my work would return to my husband.

Master Alyosus smiled apologeticaly as he returned the needle to the flame. "You are a very capable healer, your grace. My daughter is very fond of you. I would not hear the end of it at home if anything were to happen to you due to my surprise visit."

I stopped resisting and let him work. There was much of his daughter in his manner. I had never seen Sophia drop her genteel manner, whether marching, or healing, or attending a lecture or appearing in Cortan's court. I had thought the elegance to be an unique to herself. It seemed it ran in her family.

"I have come purely to satisfy my curiosity, your grace. I have heard much about you from Sophia. May I ask why you are testing the old magic on animals? You may rest assured that I will not betray my daughter's trust." Master Alyosus prompted gently when he had finished cauderising my hand. 

It annoyed me that Sophia had spoken to her father about my work. I did not know her father, nor his political leanings. At the same time, I found it difficult to refuse this refined old man healing me and risk offending him. "The crown's restrictions on the mushrooms only concerns their use by humans. It says nothing about animals. My husband wishes to test different doses for their effects on healers. If he gets his way, it would be the first time in living memory that a systematic study has been conducted on humans before testing on animals firstl" 

Master Alyosus laughed. "Have you taught your menagerie to heal, your grace? What do you possibly hope to gain by working with animals first?" 

"No, master. I am testing for side effects. I believe that there are dangers to my husband's proposed methods that he may overlook in his zeal."

The old man tisked. "It will not be an easy set of findings to publish your grace. Deyalorn's library will be bound to keep a copy, but I do not see many other libraries wishing for a catalog of the ill effects caused by the mushrooms that every Head dreams of tasting." That depends on how the rest of this public debate proceeds, I thought, but kept to myself. "There now. Your hand seems to be whole again." He picked up my still stinging, but whole, hand and brought it to his lips with a bow. "Good afternoon your grace. My apologies again for disturbing your work. May I at the very least replace your gloves as a token of my regret."

I smiled and accepted the offer, feeling slightly browbeaten and cornered by his manner. There was an authority in his politeness that was impossible to disobey. I drew the curtains of the work room, and led Master Alyosus out to my office, leaving the cowering tom to the quiet and dark he so desperately wanted. 

"Why do you work with cats, your grace? Given the aggressive nature of the side effect I you just observed, would it not be better to test the mice or rabbits?" Master Alyosus asked when we had seated ourselves.

I shook my head. "I cannot get the dosage right on lower animals. Either they do not respond, or they die. Cats, dogs and ravens are the only creatures I have had any success with. It would be interesting if I could get permission to work on a horse. It seems only the higher creatures respond to the mushrooms."

"That is interesting" Master Alyosus crooned. "Perhaps the clergy would be interested in your findings, even if the towers are not. If the higher animals have similar nightmares to humans upon consuming the mushrooms, then they must be capable of dreaming, which may mean they have a soul. It is an interesting question, do you not think?"

I stopped short, fearing I had already fallen into a trap. This man knew more than he said, and had already lead me into admitting more than I wished. "You know of the nightmares?"

Master Alyosus nodded with a flourish that suggested that the answer to that question was obvious. "I am a Joris. There are many in our family who try to rise to Queen Lucretia's power. Several need tending to in the days and months after they fail. Our elders' illnesses and their causes are mostly kept within the family to avoid scandal." He sat back and sighed. "Times seem to be changing now. The day will come when many will know the cause of the madness that seems to run in the Joris family line."

I stared a him, thrilled by the implications of what he was suggesting. "Did Master Alerio speak to you about my work?"

"No. Most of my kin in this city is intrigued by the possibilities your husband has to offer. Master Alerio has not sought audience with any branch of the Joris line." He paused to examine a set drawings of mushrooms on my desk. "My daughter spoke on your behalf. She is very fond of you. And I am, I must admit, curious to know more about the woman who gleaned from my grandfather's sister's hugely unpopular meandering works the secret to regenerating limbs."

"You flatter me, master, and are unjust to your kin." 

"Horatia Joris had many virtues, I do not question the good she did the Women's Tower in her time as head. She was an unpopular teacher, her writings reportedly nonsensical at times. That you have learned anything at all, duchess, is a feat in and of itself. Therefore, as my daughter has mentioned that you are in need of a Master, I thought to make myself available to you. If it would please your grace to satisfy my curiosity about your skills."

"You would be my tutor, Master Alyosus?" I asked critically. Since leaving my imprisonment, I had entered a world where gifts where not given without price, and joy could not exist without suspicion. The man before me was loyal to his kin, he obliquely referred to but would not mention Horatia Joris's herecies. Most of his  kin,had sided with my husband. Since leaving my imprisonment, I had taken up a role in a game I did not know all the rules of. "What would you want from me in return?"

"I am only a Master of Deyalorn's Tower, I have little interest in politics or war. I love my children as you do, your grace. It would seem the fate of my daughter's family is tied closely to your success. I will help you, if you are interested. I do not wish to see my children in peril."

I relaxed. He wished for me to protect Sophia's family. I did not know that I could do that without forcing Timmon to go to Lir. It was a smaller price than he could have demanded. "You do not ask for anything I would not give simply for the asking."

"I am glad, your grace. Then the bargain is even." His manner subtly changed to become more businesslike, though his casual, polite conversational tone did not. "I have spoken to Head Isadora, of our Women's Tower. She is interested in your notes on various stages of fetal development in the lower animals. Upon completing such a manuscript, she believes she can get enough support in the Tower to have you named a healer." 

"A manuscript" I said. That was an extreme requirement for the rank. Most students become healers after a period of a few years of study with a tutor, at the culmination of which, the student performs a series of prescribed tasks before a circle of masters and other healers to prove himself capable of attending patients independently. My ability to write and study should not factor into the question of promotion. "That is a bit unusual, Master. Putting together those notes could take some time. I have other work that needs my attention."

"You have placed yourself in an unusual circumstance, your grace. What resources would you find useful in this task?"

I considered the offer, then gave Master Alyosus an ambitious list, consisting of students, scribes, boys from the kennels and other odds and ends that would help my work move smoothly. As my tutor, Master Alyosus would claim much of the credit for any findings I opbtained under him. It would serve us both if I was well equipped.

\vspace{.5cm}

It was not many days after that interview that the song started circulating in the taverns of Deyalorn. It was disgraceful, exactly the type of shaming my brother had predicted, and I had ignored. It was a simple tune, no more than a child's rhyme, really, about a healer who liked to talk. Wherever she walked, turtle doves fled, and swords sprouted from the ground. The meaning was clear, though it did not name me. I was a home wrecker. I loved the army, and the army loved me passionately back. It was infuriating this game I had entered. Master Alerio had left Deyalorn many days ago. Master Alyosus had arranged for a thirteen year old girl Malia to help me for several hours a day with my work, and a boy for the animals, but would not give any input on my work or the directions it should take. I had no one to guide or advise me. I knew I was not a key player in this game. If I was a pawn on a chessboard against my husband, someone was doing a very poor job of moving me into position. I was frustrated with my work, frustrated with the need to produce a manuscript, frustrated and only seeing such a small piece of all that happened around me, blundering blinding from one decision to another. I reduced my maid to tears when she brought me word of the song. I felt bad for the woman afterwards. It was not her fault these words were spoken about me. What should I have done at the debate? Let it close without mentioning the danger that the old magic put us in?

I worried about the stain on my name, and what it did to the credibility of my work. I wondered if I should stop training in the barracks. The army was my main support, but the song implied such hideous things about my habits. I could not leave the palace grounds without hearing it hummed in the street. Going to the barracks would only fuel the rumors. My work still served to keep the thoughts and fears about the game I had entered at bay, but I found myself dwelling on the dangers my actions posed to my sons more and more frequently. 

Timmon appeared in my office one morning as I watching Malia redraw my pictures of dissected fetuses for the manuscript. She had a skilled hand. My work would go much faster with her help. I stared at Timmon's outline in the door, torn between my desire for a friend's company and my fear of the rumours that followed me where ever I went. Eventually, my duties as a host won the battle, and I asked Timmon to enter. "What can I do for you, Commander Romino?"

"It is general now, duchess," he said gruffly as he sat across my desk in my office. He was still mad at me for ordering him to Lir, I thought. Sophia had started visiting with Eugenio again, after my interview with her father, but Timmon had stayed stubbornly away. 

"My congradulations, General Romino, where will you serve?" This should have been a happier moment, Timmon achieving the ambition of his life. We were close friends. Instead, we argued like fools. 

"I have not decided yet, duchess."

I wanted to burn the man. With everything else going on around me, could he not do me this one favor of taking his family away to safety before songs about my character turned into songs about us as lovers? How long would it take for my husband to discover the true nature of Nubo's service, or the reason for Timmon's absence from morning prayers if he decided to start investigating his life? I did not burn him, however. We had fought enough already. "How may I help you, General?" I repeated.

Timmon glanced at Malia in the corner. "What I have to say is for your ears alone, duchess." 

I sent Malia and my maid away, then sat myself at my desk. "Well?" I asked more impatiently than I had meant to.

"Do you know anything of the young duke's mother?"

"I do not wish to know of Lucretious's origins. It will only make raising him as my own more difficult."

"No, Nisrita, I think you do." He did not state an opinion or a fact, he commanded. 

"Very well then."

"Your husband was close to Leila Selvand, neice to the Duke of Selvand on her father's side and the Queen regent on her mother's. She studied in Deyalorn's Tower until late winter, at which point she was suddenly sent to her uncle's house in Selvand. She has visited her uncle several times in the past, but never without the company of at least her brothers. Her father is very protective of her. He does not even let her leave the Tower unescorted. She is his only daughter. She is a young woman, about your age, pretty, moderately gifted. She was one of your husband's favourite students, though I am made to understand  that she was by far weaker and less skilled that those he normally takes on. I do not know if your husband has harboured any mysterious guests in the Tower as he used to in Cortan. I do know that the crown has until recently. Yesterday, Leila Selvand reappeared in Deyalorn, looking significantly worse for wear for her trip to see her uncle. She has withdrawn from the Tower, giving up her ambitions of becoming a healer. She does not leave her father's house. Her father is putting a lot of effort into finding a husband to take her from him in two months. There are other women your husband was close to over the course of his year apart from you. Few of them are of potential interest to the crown, none have disappeared from the public eye for the months leading up to Lyca's birth."

I did not lift my hands from my desk when Timmon finished his revelation. I knew them to be shaking. "Why are you telling me this, Timmon?"

"Because, duchess, I wish to see you armed in this battle you have entered."

I closed my eyes and sank back into my chair, overwhelmed by the information I had been delivered and the motives of the messenger. I had spent my days so tired and alone, wandering blind through a political maze where I knew little and could depned on no one. Timmon and Sophia insisted on helping me where I wished they would leave for safety. To everyone else, I was a means to an end. I pushed away the fantasies where Timmon did not need to leave Deyalorn, and our families could intertwine without fear. He could sit in my office and tell me about his Liri gods, instead of about my husband's lover. There was no point in fantasizing. The path ahead was difficult enough without distracting regrets. 

"Thank you," I said at last. "How did you learn all this?"

Timmon gave a self satisfied grin. "I trained here as a boy for six years before going to Cortan. Many of my training officers and friends now have high positions in the palace and city guard. For the rest? Sophia is a brilliant conversationalist. She has a way of attracting gossip and rumour that I wish I could teach my men."

I laughed at the image of Timmon's sergents sitting over needlepoint and cakes with Niev's army. Timmon's grin broadened. I was lucky to have Sophia and Timmon as friends. "I am sorry for my rudeness last time we met," I said.

He waived away the apology. "It may surprise you to know that there is one matter on which you agree with your husband. Duke Griswold had suggested that I be sent as ambassador to Lir. I suspect he wishes to be rid of me."

"That is marvelous, Timmon" I beamed. "You will pursue this, of course."

Timmon stopped smiling and looked at me, skewering my joy with his gaze. "As I said, I have not yet decided where I will serve."

I bit my lip to hold the protests from escaping. His mood, for whatever reason, was fragile. I did not want to fight again. "Thank you visiting, Timmon. We should keep these private audiences brief. You have heard what they are saying in the streets. Your unchaperoned presence can only make matters worse."

Timmon smiled tightly and stood to leave. I gave him my hand. He gave me a painfully familiar look as he rose from his bow. It was a look I had seen on his face countless times, many years ago. It had never been meant for me. I walked him across the small room to the door. I suddenly did not want him to leave. Then I closed my door on him and leant against it for support against my loneliness. That look could not be meant for me. We were both married. We could not both stay in Deyalorn. It would ruin our families. This accursed city made us both revive a dream that should have long been dead.

\vspace{.5cm}

I learned, with time to play a smaller, simpler, version of the game my elders and betters at the Royal Court played around me. My days continued to be eventful, but they grew no less lonely. I had few social visitors during the day other than brief interactions with Sophia as she came and left with her son. Malia and I worked diligently and quietly. Master Alyosus came by every few days to check on our progress. I decided that I would continue training as I had before. I had learned from Timmon and Makala the value of not showing fear in the face of intimidation. The lesson had served me well against Niev, with the Hundred Horsemen, on my journeys alone with Griswold. The setting was different in Deyalorn, my body was not in danger. I decided that the lesson applied all the same.

Carlotta asked me to visit her in the White Tower. She wanted to discuss her progress on making our methods with Lyca rigourous and repeatable. She showed me what she had done. I gave a few suggestions. It was clear to me that in my isolation from the Tower, this had become her project, her achievement, and not ours.  The credit would be hers alone. I found myself regretting again the path that I had embarked on. Surrounded by the walls of the White Tower, I realized how much I longed to be with my own kind again, healing, attending lectures, walking in the garden, being surrounded by others in simple white robes all spending our days in servince to the Preserver. 

Carlotta asked me, again, to give up my work at the palace and return to the Tower. She pointed out the amount of good I could do as a strong healer. She reminded me that practising was my first duty as a member of the Tower, not study. I did not need to fight my husband. There were many in the Tower who recognized my skill. They would welcome me among their ranks if I would give up this struggle. 

It was not so simple. Even if I could give up my investigations into the effects of the mushrooms, to return to the Tower meant a return to my husband, who did not want me practising. He had made that clear in Cortan before I marched to Niev. He had hinted at it again in Deyalorn. In his mind, my position was that of wife, mother and duchess. Whatever my brother said about not interfering in my domestic life, he not only kept me away from my husband, but provided me the tools to serve Marsea in a solid, tangible way. Without my brother's support, I would have nothing other than an empty domestic life. With his support, I would become a healer, and have a public voice. 

I put my regrets aside, made some suggestions to Carlotta's methods, and corrected her recollection of certain steps we had taken. I learned that she and Ezaro had been working together to find a way to teach her gentle touch, they had not had any success yet. Even with the aid of the mushrooms, Ezaro could only do a fraction of what he could unaided if he tried to be as gentle as Carlotta. Without finding a way to cause less pain, or a way to make the process more efficient, the method I had devised to save Lyca's life was nearly useless. I left them to their work. Skill and finesse were not my strong points. Furthermore, it was clear they did not want me. 

A few days later, Lyca fell ill. His nurse went to wake him from a nap one morning and found him not breathing. She panicked and shook the child, at which point he awoke and wailed in protest. When I found her, Rosa was sobbing uncontrollably next to the still wailing infant. I scolded her, terrified myself, at the prospect of what might have been, and sent her off to compose herself while I checked Lyca's heart and lungs. The child had no injury. Lucretious had been born so small and so early, he lived within reach of the Destroyer's claws. No ammount of gift could keep him securely within the Preserver's guard. I told Rosa to keep Lyca within her sight at all times. Between my children's three nurses and myself, we devised a rotation of night watches by the child's bedside. I could not allow another incident such as this to happen while he slept unobserved. I spent my midnight vigils working on my manuscript by candle light. There was no point in wasting two hours of time I could do nothing else with. It kept my mind off the possiblity of losing this child that I had worked so hard to coax into life, that I was now inexplicably growing attached to.

I learned that there would be a convocation of clergy in the city during the weeks preceeding the Day of Unions. The Tower's lust for power made the House of the Triumverate deeply uneasy. Representatives of every duke in the kingdom would arrive in the city as well to make their views on Griswold's new promises heard. The dukes were nervous and split in their opinions. Most armies supported the idea of stronger healers marching with their men, inspite the Maker's daughter's opposition. I alone could not save as many men as Griswold promised his method's would.  Some, like Duke Ergino, with the powerful Tower in his land, hoped that as stronger Tower would only make their lands more powerful. Others feared an upset in the delicate balance they maintained betwen the Tower, the military and the church. This debate was no longer simply about the old magic. The subject had escaped the Tower's control. My husband had made it one of national interest. 

I had become  a minor player in the debate, my brother no longer seeing fit to display me at his table, with the fall of my name. He left me continue my work, in the hopes I would find something to display before his clergy and nobility. I spent my day in solitude, working furiously with my cats and dogs to prove my husband lying about the safety and efficacy of his proposals. My evenings I spent alone with my manuscript. Rarely did anyone wish to consult me on my opinions anymore. I told myself this only left me more time to work. My findings would appear for inspection before the dukes of the land, even if I did not. The thought exhillerated and drove me, until my brother asked to hear of my progress over the last month. He was not angry with my report.

"Do I understand correctly that you have spent a nearly a month terrifying dogs and ravens. I did not give you the resources of a Tower's master so you could turn my rookery into the Destroyer's winged messengers."

"You do not understand, brother." I tried to explain. "These animals are responding to the mushrooms as healers do. Those who try to open the gate to the old magic suffer for days if not weeks of untold nightmares."

"How do you know that?" he snapped.

I stiffled my disdain at his naivete. "You married the most powerful gifted in the land to my husband. Do you doubt that he tried his theories on me first?"

Dario tapped the arm of his chair impatiently with his finger. He could not blame me for my illicit use as long as he thought it occurred after my marriage. "How long has the Tower been defying the crown on this?"

I swallowed to wet my suddenly dry throat and furiously thought for a way to escape this line of questioning. I did not know the full extent of the Tower's flirtation with the old magic, though my reading of the biographies of various heads was starting to give me an idea. I did not wish to betray my kind with secrets they kept that I was not officially privy to. My brother, as if reading my thoughts, grew impatient with my silence. "How long, sister. You cannot protect the Tower and fight your husband. You have chosen to stand against your kind. Tell me what you know about the depth of their treachery."

My brother spoke falsely. He must have. I had not chosen to stand against the Tower. That have never been my intension. The Tower had raised me. I had chosen to stand against my husband only. Had I really erred in judgement so far? Was is possible that the two were essentially the same? I licked my lips and gave as vague an answer as possible. "Up until a few years ago, most healers knew and feared the old magic. We knew that there were mushrooms that opened the gate to the power, but no which ones, nor how to access it. We heard rumors, occaisionally that a Head of a Tower or a prominent Master had successfully accessed the old magic. The stories we heard were fantastical, unconfirmable and sometimes false. For instance, many in Cortan's Tower believed that my husband had turned himself into a living ghost through the use of the old magic. Ther was no end to the fantastic nature of the feats performed by ambitious masters because no one knew the power that could be unlocked by the mushrooms. Many of us did not pay the stories much heed, knowing the illegality of the act. Students and healers asking too many questions about the old magic were reprimanded, or even pushed out of a Tower. I do not know how it was amongst the Masters, but I suspect that most did not pay it much attention in that late stage of their lives."

"But not all."

"No, not all. I think there have always been a few people throughout the land toying with the old magic. Its power is too tempting."

"Can you give names?"

"Only if you wish to execute the dead, brother." I prayed that he did not ask for a list of my contemporaries who spent their days in the grip of the mushrooms. I did not wish to betray Carlotta.

"What is the situation now?"

My brother hammered me with his questions, demanding that I betray my own. "Times are changing," I tried. "It has been a long time since I have been privy to the doings inside the Towers."

Dario saw easily through the attempted evasion. "Who defied the edict in Cortan's Tower last year?"

"I knew that my husband conducted experiments with Healer Ezaro. I do not think he lured any other students to taste the mushrooms," I lied.

"What of the Masters in Cortan?" I balked. My brother's single brow gathered like a dark storm cloud over his face. "I asked you a question, sister."

"Cortan's Tower raised me, brother. I had no other family until recently. Do not ask me to betray them."

"Then you may take your children and return to your kind, sister. I have less need for an informant who will not talk than I have for terrified cats."

I put my head in my hands and prayed. My brother waited. Cortan's Tower had raised me. On the other hand, what it was doing was wrong. Speaking alone, I could not keep it from embarking on this disasterous course. As a voice to counsel the rulers of this land, I stood a chance of being heard. Yet the Tower had raised me. I could not escape that reality. Master Adele had called me the closest thing she had to a daughter. She guided me through my original breakthrough. She let me see my sons when I was alone. I owed her much. 

I thought of my children and what would happen to me if I left the palace. I would have nothing. I had no rank, no Tower would give me a home. I was a nominal duchess in a duchy that longed for a way to be rid of me. I would be completely at my husband's mercy. Master Adele's love for me was not the love of a mother for a child. When I needed her protection, she left me for my husband's favour. I was on my third master now, struggling towards the rank of healer by producing a manuscript, an extremely stringent requirement. The Tower may have raised me, but it had also betrayed me, honoring Ezaro for my achievements, leaving me powerless and unrecognized inspite of everything I had done. I could chose between my brother's shelter or life with my husband. To protect my children I had to betray the Masters that had raised me. It was hardly a choice. 

"Master Adele knew of my husband's work," I said in a small voice. "As did Head Corino. They both took an active interest in my husband's work, though I do not know the details of their involvement. Master Alerio knew as well, though he did not approve of the work. Before my husband returned from exile, he refused Ezaro as a student because of his too keen interest in the old magic."

"Thank you. That is useful to know." Dario smiled and lightened his tone. "Any solid proof you can gather of the Joris family's involvement in the old magic would be of value to us. If we can tame the voices in that house, we could do a great deal to silence the incessant appeals and arguements of this city's gifted."

My stomach churned and tied itself into a painful knot. Master Alyosus was my only hope of becoming a healer. Once I gained that rank, my husband could not expel me from the Towers of the land. I would not need supervision to heal. My ability to practise would be simply a matter of a domestic fight, not a political one as well. "The Queen Regent was born a Joris. Would she not be better equipped to tell?"

"I cannot ask my wife to speak against her blood. You, on the other hand, have the attentions of a master from that line."

This I would not do. Sophia's father had come to me out of love for her, against the general leanings of the rest of his family. Betraying his loyalty would also betray Sophia and Timmon. This I would not do. "Master Alyosus does not stand in your way."

"I am not questioning your Master's loyalty. I only want information about his family. My dukes and my priests are converging upon my court. I need something more than trembling animals to show them."

Dario dismissed me. When I left my brother's audience chamber, my stomach could not take the strain of my betrayal any further. It relieve me of its contents in a busy passage leading to my quarters. I stumbled to my room, taking the arm of a young barron's son for the sake of minimizing the scene I had already created. I wanted to be alone with the guilt churning inside me. I had betrayed Master Adele and Cortan's Tower. I was now being asked to betray Master Alyosus's trust. Would Sophia forgive me if I did as my brother asked? Would she understand that I had to do it for my children? We women are not fortunate enough to die for our family's honour in battle. We cower in the shadows of brothers and husbands, betraying our friends to save our children. 

There had to be another way. My husband had not won so far as to reduce me to betraying my friends. This I would not do.

\vspace{vspace .5cm}

\begin{comment}
He would not reach for the position in Lir. My brother eaned towards giving the position to a nephew of his, ours, I suppose, in the Duke of Belisal's court. Belisal lay to the west of Deyalorn, a quiet interior duchy with a gifted Duke. Duke Rafail was the only interior duke to stand in favor of the old magic. My brother wished to bring him in line with this appointment.

 "He is not a man who does well with nothing to do," Sophia observed with a hint of sadness. "He has had nothing to do since he came back from Niev."

She was quite likely right about her husband, though I knew there to be another reason for his sudden lack of ambition. Whatever Sophia's, or my, plans for Timmon's future, the Destroyer, or whatever god Timmon now held with, seemed to have plans of his own. 
\end{comment}

I did not learned about Timmon's foolishness until I entered the barracks to train the afternoon after it occurred. No matter how much I wished Timmon to leave for Lir, he made no movements hinting at preparations. He would not reach for the position in Lir. Given the general's lack of ambition, my brother leaned towards appointing a nephew of his, ours, I suppose, in the Duke of Belisal's court. Belisal lay to the west of Deyalorn, a quiet interior duchy with a gifted Duke. Duke Rafail was the only interior duke to stand in favor of the old magic. I guessed that my brother wished to bring him in line with this appointment. None of this troubled Timmon. He spent most of his evenings in the company of other generals and commander's either in the barracks, or as guest in their houses, seemingly without a care as to his future or that of his family.

I was not terribly surprised when I heard that Timmon, in a drunken attempt to defend my honor, wounded both a barron's son and a commander. In my mind, it had been a matter of when, not whether he would heroically ruin his prospects. As everyone involved was still alive, and would recover their healths, there was nothing immediate I could do. I did not face Timmon directly. I would only rage at him for his stupidity. That would not move him any closer to a position in Lir.

I picked at the problem throughout the afternoon, and into my solitary dinner. It distracted me from my manuscript as I kept my midnight vigil to listen to Lyca breath in his sleep. His breath came unevenly too frequently. Every night, there would come a time when my heart would beat nearly one hunred times between two of the infant's breaths. It may have taken him longer to take his next breath on some of those occaisons. I never had the courage to count beyond one hundred. A touch or a gentle shaking would always remind the little boy of his need for air. That night, I left my writing aside. I fretted and counted until I knew what I needed to do. 

\begin{comment}
"Explain to me your relationship with General Romino." Dario asked when I asked him to name Timmon ambassador to Lir. "He must mean a lot to you for you to advocate for something your husband wishes for as well."

"My opinion on this matter has nothing to do with my husband's," I replied haughtily. "I simply want him gone from this city."

My brother knew the rumour about Timmon and myself. He gave me a knowing smile. "You are young, Nisrita. You will learn that these passions come and go like the wind. A lover's quarrel is hardly a reason to make a man ambassador."

"I am not his lover," I corrected. Somewhere between the rumour and the truth lay the key to what I wanted.

"Ex lover, then, since you have argued so passionately. That still does not qualify him for the post." 

"Oh?" he asked, incredulous, but interested. "I have heard that you had admitted to as much once. He wished to marry you, I believe, the same year you married Duke Griswold."

"That?" I asked, doing my best to imitate Sophia's casual grace. "I was young, foolish, and angry that he had left me. It was a childish attempt to get back at a man I had once loved."

"It was very foolish, sister," Dario reprimanded. "You left your name and honor to be smeared by all."

I bowed my head, giving my brother every reason to believe that he was simply counselling his misguided, pentinent younger sister on the mistakes of her youth. "It was, and I suffered much for it. I was not completely well at the time. A sort of madness and grief had set upon me during the year after my first husband's death. I loved my first husband. I had not had enough time to grieve before marrying again. Please, brother. Have pity on me for indiscretions conducted in grief and passion. I have not dishonoured Cortan. I have never known when to hold my tongue and when to speak."

"You poor foolish girl," he said, taking in my countenance. "You must learn to control yourself. Your tongue will be your ruin." I nodded miserably. "What has been done, cannot be undone. However, I will discourage any word against your honor in my court that may arise from General Romino's disgraceful conduct."

"Thank you, brother," I whimpered.

"If you are not his lover," Dario continued, "why did he wound Baron Calesi's son with his sword and break Commander Turnis's arm? By his own admission, he did so to defend your honor against that nasty little rhyme that has been circulating." 

"General Romino was drunk, brother, he would never have done such an act sober."

"You are not helping your case for making him an ambassador, sister. And you are not answering my question." 

I trembled as I realized the mistake I had made. I could not defend Timmon directly. I returned to my play of the errant sister. "General Romino is a married man, brother. He is too honorable to touch any woman he is not married to. We are not, and never have been, lovers. We are, however, two people who have served together and saved each other's lives in combat. There is friendship that comes from that. We share, perhaps a bit more than that, but there is nothing dishonorable in his feelings towards me, uncontrolled as his actions were. I beg you to be linient with him for the sake of his service and friendship to me. Send him away to Lir. His presence here, and behavior like last night's threatens the weight of my testimony. If people were to think that we are what we are not, I would be of little use to you." I bit my lip delicately and beseeched him with my eyes. 

"That is a predicament, sister. I feel for you. I can keep his name and honor intact inspite of his dusgraceful conduct. However, I will not send a man who cannot control himself on a diplomatic post." He paused to watch my face melt into a misery that was not entirely false. "I will order him to serve in Turina, which is where Duke Ergino orginally wished to station him, I believe. I am sorry. I can do no more for you."

I thanked my brother, and left. Given the story I had spun, there was little more I could do. At the very least, I had assured that Timmon would not be punished for his violence. That was not enough. 
\end{comment}

My maid informed me when Queen Anyesa came to her now daily visits to Lyca. I instructed my student Malia on the feeding of the animals I was in the middle of, and went into the nursery. The queen regent sat with Lyca craddled in her arms, looking down at him with a good deal of motherly affection. Queen Anyesa had six children of her own, three by her first marriage, and three by my brother. That did not explain the devotion to the child she currently displayed. I was playing with signs and auguries. I hoped I read them correctly.

"Good morning, your highness, I pray I find you well."

The queen rose to give me her hand. She was a fair, regal woman, still beautiful, inspite being more than three decades old. She stood a few inches taller than myself, and carried herself with the authority of a general on the field, inspite of her soft, curvy form, reflecting her lifetime spent in court. If I thought about it rationally, I had nothing to fear from her, I was stronger, younger, and armed with my talisman. Audiences with her intimidated all the same. "I am well, duchess, as I see, by the Preserver's grace, is your son. Is he still ill?"

"He is, your highness, as children born so early often are. In the Women's Tower, as I am sure you know, it is said that such children benefit from their mother's touch. It pains me to think that Lucretious will never know it."

"Some children are born unfortunate. Duke Lucretious should be grateful for his life. I am certain it is a comfort to his mother to know that he recovers."
 
"I cannot imaging, your highness, speaking as a mother, giving birth to a living child, only to have him snatched away from me. What my husband did was a great cruelty. He does not understand how we mothers need our sons."

"Your husband is a hard man, duchess. Even as a queen, I cannot interfere in matters between husband and wife."

"I would not dream of asking that, your highness. But we are not ungifted women, bound only to our husband and children. Could we not exchange favours for others we care about?"

Queen Anyesa put Duke Lucretious down in his cot. "I am sorry, your grace, but your lover is a drunkard and a disgrace," she said with disdain. "I cannot speak for him."

"That is a shame, your highness. I had wished to protect your cousin Sophia from what is to come by sending her family to Lir. You see, the Regent Consort has given me the choice between investigating her ancestor's use of the old magic and continuing to enjoy your generous hospitality."

"You will get no where by threatening my family." Queen Anyesa's face turned cold. She may be an unarmed woman I could overpower. I still feared her. Every bone in her body spoke of the great power of the crown she would bring down upon my head if I touched the Joris name.

I gave her a pleasant smile. "I love Sophia, your grace. I have no desire to threaten your family. But I must protect my children and my name. I only wished you to know my predicament, and how I came to be here."

The Queen Regent contemplated my offer. "I will do my best. I cannot keep you as a guest against my husband's will."

"Thank you, your highness. I understand. I will do my best for your family as well."


\begin{comment}
Would your neice like to visit him?"

I watched the Queen Regent stiffen, then put Lyca down carefully on his cot. "Why do you ask?" 

I did my best to smile before her cold eyes. The Queen Regent did not involve herself in matters of Marsea's policy, but she held the court together at her will. I understood why. I would not easily wish to endure her anger or mistrust again. "We are both mothers, your highness. News of Lucretious's fragile health cannot be easy for her."

When she passed her eyes over my body, I felt that she stripped me naked, not only of clothes, but of my flesh as well, to examine the soul the Maker had given me. "What is the price of your mother's generosity, your grace?"

"I wish for General Romino to become the ambassador to Lir, your highness."

"The drunken fighter?" Queen Anyesa laughed. "My husband has decided to send him to Turina."

"I am aware of his decision, your highness, which is why I ask you. I am doing my best for Lucretious, but he is fragile, as you see. It would be a pity for him to never see his mother."

The queen unleashed a cold contained fury upon my small uncertain person. "You are threatening me?"

I do not know what inner courage prevented me from quivering at her question. "You misunderstand me, your highness. I have grown fond of Lucretious. I will give him no reason to doubt that he is my own. I have also experienced the pain of being separated from my children. I would not wish it upon any mother."

The Queen's eyes softened a fraction as she realized that I did not intent to play with the life of this child. I may be rumoured to be many unwomanly things, but I was not that type of monster. "I can make no promises, your grace." 

"I understand, your highness. This is a request for your neice as much as it is for me." Her highness left me relieved to have made it through the interview intact. I did not return immediately to my animals. I joined my boys wrestling in the garden, under the noon sun, staining my healer's robes green and brown with grass and mud. I released my day's tensions to the tumble of their laughter and tears as they swung on ropes and bruised their knees. I loved them so much. They were the hope and joy of my days. I had spent the only shred of security I had for them on a man I loved, if impossibly, as much as I loved them. I hoped I had not deprived them of their safety in the future. This was a treacherous game I played. They had not asked to be ensnared in it.

\end{comment}

It took me two days to learn how devoted gifted Joris women are to the Joris line. Timmon burst loudly into my room ruining my observations of a mad raven. That he did not get his eyes pecked out for his interruption is only due to my ablity to kill the bird quickly. The shock did nothing to intefere with Timmon's rage. He accused me of tossing away the only sheild he had been able to give me against my husband. I told him that only lions eat their young. I would not use Lyca as a weapon, no matter what the cost. He called me a fool and a mad woman to oppose my husband without thought to my safety. I called him a drunkard not worthy of Sophia's shaddow. We argued more fiercely than I had argued with anyone. He refused to leave me undefended in Deyalorn. I could not see how he could defend me by tossing his carreer away to drink and shame again. He stormed out of the room without letting me touch the wounds on his face and shoulders. I sat solitary and  shaking at my stone table, my bird killed, the day's work ruined and wept for the loss that I knew was about to come. I did not expect to be forgiven for half the things I had said to Timmon that day. Whether he left Deyalorn or not, I feared I had lost the truest friend I would ever make in life, the only man since Makala who had taken an interest in my happiness. If Timmon did not leave for Lir, I would have tossed it all away for nothing. 

\vspace{.5cm}

My meeting with the Queen Regent was not completely wasted. She reminded me of a lesson I must have known during the self assured days of my youth, but had forgotten during the years of terror and the tension of living with my husband. Where grace, good will and reason will not serve to rule, terror will always subjgate one's audience. 

Reason was a dulled weapon in this debated over the mushrooms. I had seen the effects of good will already, winning much of the army to my side with it, only to lose many due to my bad name. The men who would converge upon this city, the Dukes and the priests of the kingdom had far more courtly grace and many more years of practise with it than I. What could a woman, not even a healer, do to impress them on this point? That left me one weapon. I changed, for the moment, the direction of my research.

By the time the clergy had started arriving in Deyalorn, my brother had lost his patience with me. It had been three weeks since he had asked me for information about the Joris family. I had given him nothing. Timmon had left to settle his affairs in Cortan before continuing to Lir. His enterouge for the embassy would follow him in a few weeks. I would not betray his family in his absense. I started to despair that I would have to return to my husband, and possibly lose my chance at becoming a healer. It was for a worthy cause I told myself. I had resigned myself for the worst when the boy who saw to my menagerie every evening informed me of a dead dog.

I lept for joy, the practise of stayed courtly manners still a learned behavior when it came to matters surrounding my art. Timmon would certainly call me a soldier. I did not care. He could call me any name he pleased from Lir. I had a dead dog on my hands. By the Preserver's grace and the Destroyer's will, I had something to work with. I stayed awake late into the night dissecting and analysing the creature's organs. Finally, I had something to show my brother. I had found a weapon. I was certain it would work.

My brother was less certain, but he was pleased. He promised to speak to Head Ionus of Deyalorn's Tower to arrange another public debate, pitting me against Healer Ezaro. The Tower had been asking for an opportunity to demonstrate the mushroom's power since the debate between my husband and Master Alerio. With my result, Dario granted them their wish. He arranged the public demonstraton in three weeks time, just over a fortnight before the Day of Unions. By then, representatives from all the duchies of Marsea would have gathered in Deyalorn. To say that I was nervous would be to call a lion a cat. I trusted my findings. Just to make certain, I returned to my quarters and killed a cat. The results would hold. It was my composure that would not.

When Master Alyosus heard of my discovery, he was less pleased. He had never made his position in this debate clear, and Sophia would not, or could not ellucidate. He sat in my office, his head bowed in contemplation for a long while, then said, "This will not effect the Masters of Deyalorn's Towers. Those who had not guessed of these dangers have learned of them from your husband." 

Then how can they possibly condone this plan of my husband's, I wanted to cry out. Had the Tower gone mad? I controlled my outburst in the Master's stately presence. "I do not intend to sway the Tower with this, Master Alyosus. My audience are the ungifted who will have gathered." 

Master Alyosus reluctantly gave his consent. "If you wish for your results to be taken seriously, your grace, have the tools you plan to use sealed and kept safe by the Tower. I will send observers to testify to your storing the evidence propoerly. Use the royal seal, if you can on this. I believe your husband has a copy of Cortan's seal as well?"

A stone grew inside my stomach, filling me with nervous dread, at the thought of sealed testimony and the porspect of tampering with evidence. This was not how the Tower conducted its lectures, but how political games were played. It was another sharp reminder that I knew little of the game I played.
"And then, may I suggest you stop spending so much time with your animals?" Master Alyosus continued? "You will serve no one by appearing worn and nervous before the court. Spend some more time with your children. Perhaps a little more time on the manuscript?"

I granted Master Alyosus's request. It would slow my investigations into the effects of the mushrooms, but he was right in observing that I had put more energy into my investigations over the past few weeks than I could sustain for much longer. I instructed Malia on the rudimentaries of observing and administering to my menagerie, slowed down my work, and spent more time with Makala investigating the differences between the sedated and agressive animals, and teaching Griswold to throw a knife.

It was during this time that I met Lyca's mother. I came upon her one morning as I led my sobbing Makala from the menagerie, frightened by a snarling dog nearly as tall as he was. I held my son until he calmed himself, and reminded him of his duties to be a brave duke before handing him to his nurse, who gave me a look containing more disaproval than I thought necessary. Then she nodded to a corner behind me where a woman awkwardly stood, torn between holding her son, and ridding me of her unwanted presence.

I had not given Leila Selvand much attention after my interview with Queen Anyessa. I had heard that she visited her child regularly. As I had been given what I wanted in exchange for her presence, and she did not disturb my schedule, I did not give her a second thought. There was something in her posture that was hauntingly familiar. Had I once looked so distraught at the realization that my husband did not value me as I thought he had, when he had taken my two boys from me? I pitied Leila's pretty face, made drab by the dark circles surrounding her remorseful eyes. With a little care, and a bright smile, I could see why a man would be taken with her. Had we met under different circumstances, my pity for her condition would not have overcome my anger at her imposition in my marriage. With the debate with Ezaro approaching, I needed allies more than I needed enemies. My husband had chosen her out of a multitude of Deyalorn's young women for a reason. I wanted to know why.

"I- I beg your pardon, your grace. I had not expected you to return." Leila mumbled to the floor. "I will leave at once."

She was my age, a dark girl with long lashes, generous breasts, and a waist that would have been thin a year ago. She was certainly pretty. Could it be that my husband had taken to her just on that? He was known to have his whores. I dismissed the idea. He could find prettier women outside the Tower. He could find more powerful ones within. "Not at all, Miss Selvand," I began, giving her my hand and doing my best to immitate Sophia's confident easy grace. "I will call you Leila, if you do not mind. I will be in the garden with my sons. It will be time for Lyca to sleep soon. I would enjoy a moment of your company at that time."

Leila's frightened face looked a fraction less scared and a deal more confused. "Lyca, your grace?"

I smiled benevolently. "It is a nickname given to the duke by some friends. It means impossible in the Liri toungue. Fitting, wouldn't you agree?"

Leila smiled weakly as I gave her privacy with her blood. In the garden, Woldino proved more difficult to console than Miss Selvand had been. Nothing short of my full attention would comfort him. When Leila came to join me, she found me sitting on the swing with my first born, distracting him from his irrational desire to kick his younger brother to death, while his nurse kept the dejected Makala occupied elsewhere. "Leila," I crooned, indicating a shaded bench. "Welcome. I hope you do not mind talking over the screams of my sons. I am a woman of the south. I fear the oncoming winter and wish to take as much advantage of the pleasant weather as I can." I let Woldino wriggle off my lap to do what damage he could.

"No, your grace. This is a pretty garden."

"You are welcome to bring Lyca into it any time that Rosa is willing."

"Thank you, your grace." Leila answered with the attitude of one who would never take up the offer.

"But you will not come."

"It is a lovely garden, your grace, but... No one knows I visit" she paused, struggling for a word "the young duke."

I allowed myself an internal self satisfied smile at this willing confession.  I had finally found someone less able to keep her thoughts to herself than I. Perhaps learning her secrets would not be as difficult as learning Master Alyosus's motivations, or sending Timmon to Lir.

"Not even his father?" 

Leila blushed, and fumbled though I had asked the question warmly. "No, your grace. Especially not the duke." I watched her teeter on the edge of tears. "He- he wanted me to visit the Women's Tower. He was furious with me when he learned that I had not. He has hardly spoken to me since he returned."

I wondered how many people Leila had access to in her current state of shame. Timmon had said that she was not allowed outside her house. I did not imagine her treatment inside her father's house was very gentle. Her father had gone to great lengths to protect his daughter and his house from shame. Her aunt, certainly, still spoke to her. I doubted there was anyone else. Why else would she confide so openly to her previous lover's wife, if she were not completely isolated and bereft of friends.

"If I may ask, why did you not visit the Women's Tower? You must have known you could not keep the child."

"I am gifted, your grace. The chance of bearing a gifted child was too great. And..." she trailed off, still teerering on that teary cliff. What fools we women are. How different my life would have been if I had destroyed my fetuses and taken a life of service. I would not have my two wonders, but I would also be free of the terror and anxiety that had permeated my life for the last three years. It was a hard decision, one that I, and Leila, made for the sake of Marsea's need for gifted, rather than out of consideration for our future.

My next words were cruel, but they were no where near as cruel as my husband's desertion of this woman he had promised the sun and the sky to. "You loved him very much."  Tears streamed down Leila's face as she nodded. I put my arm around her shoulder and held her as I would hold my dearest friend. If my husband had seen potential in this young weepy woman, if she had something that he thought could help him rise to power, and he had not robbed her of it by planting Lyca in her womb, then she would be a potentially useful ally.

\vspace{.5cm}

Master Alerio arrived in Deyalorn a few days before my debate. I sent for him as soon as Sophia notified me of his arrival. In spite of Master Alyosus's instructions to rest and calm my nerves, I was jumpy and ill tempered with my boys and my servants. Master Alyosus had not given me any further instructions on how to conduct myself against Ezaro. It was as if he did not care if I won or lost. His only concern during our meetings was on the progress of the manuscripts. I wanted some guidance and wisdom from a more experienced mind. I hoped that Master Alerio would serve me in this regard.

In responce to my missive, Master Alerio invited me to dine at the Master's Table in Deyalorn's Tower. I had dined with Masters before. Every spring, when the army and most of Cortan's Tower leaves for a campaign, when the students go home from school, the remaining Masters, heaIers, and the occaisonal orphan all eat at the same table. The Tower is empty, and the need for comraderie overcomes the need for maintaining ranks in the large echoing hall. I have never dined at the Master's table. I found my palms sweating as I walked down the side of the Tower's great hall to the raised table at its head. It was silly, I told myself. I have a rightful place at Duke Ergino's table in Cortan. I was once a regular guest at the royal table in Deyalorn. Certainly Deyalorn's Tower did not hold more power than the crown. I had dined in more priviledged seats than this. I should not be intimidated. My logic did nothing to reassure me. I had dreamed of being a Master all my life. I had wanted to dine at the Master's table in Cortan for longer than I had ever wished to dine at the Duke's or the crown's table. I knew that all eyes watched me. I would be on display before this audience and so many more in a few short days.

Master Alerio helped me to my seat possessively. He kept the conversation around me focused on his travels and the talks he had given, not giving the Masters nearest me much room to question me about my work or my findings. I watched him work. Master Alerio was also a skilled politician. He had not taken sides on his debate in Niev, because he did not think he could gain anything by making his views clear, though he must have held his current views even then. Why else had he offered to take me on as his student? Now, he sheilded and protected me, appearing to talk about his personal views on the issue of the old magic, but always quickly sliding the conversation quickly to the subject of how his lectures were recieved, rather than stating what he had said, or what his opinions were on statements made my various clergy or dukes currently in residence in the city. Like Master Alyosus, he kept his opinions on many matters secret, though he did a better job of hiding the fact that he did so. He spoke with ease and gentility, but he gave the impression that the subjects at hand touched him. Speaking to him felt more natural, unlike the slightly dizzying effect that Master Alyosus's nonchalant manner on all subjects had on me. I studied Master Alerio. If I was to survive this fight, I decided, I would need to learn to display less of my mind, and pick a manner of deflecting my political opponents, either as the Master from Cortan did, or as the Master from Deyalorn.

After dinner, Master Alerio took my arm and led me paternally to his chamber. He put a crystal of a deliciously rich, sweet young wine in my hand before seating himself and stating "You wished to speak to me, duchess." 

I must admit, having easily survived a dinner in the company of my future interrogators, I felt much more comfortable about the upcoming event. Being welcomed into the private quarters of a prominent Master of Marsea's White Tower went a ways further to boost my confidence. "I wish to seek your advice on the upcoming debate with Healer Ezaro, Master." I said.

"Of course. I presume you have some plan that instigated this event?"

I told him about the dog.

Master Alerio frowned. "It will not be enough, I think." 

My stomach tightened. "The Tower may expect these findings, but the ungifted do not, Master." I protested in vain against my fear.

"You are correct, your grace. However, the Tower will tear apart your every argument with the simple fact that a dog is different than a human." He paused to consider. "Have you had the chance to read Dimetre Joris's biography yet?"

"No, Master." I explained to him my schedule, and Master Alyosus's tutelage, and my hesitation to betray Sophia's kin.

"Very well." Master Alerio pursed his lips. "That leaves us with few options." He rose to refill his glass, and poured more into mine. "I hear you have spoken against Head Corino already." 

I sank into my chair. "Yes, master." It felt wrong that I had so easily betrayed my tower while I stubbornly refused to betray the Joris family.

"Do not be ashamed of the decisions you have made, your grace. You chose as you did for reasons that seemed valid at the time. We can only play from the positon we find ourselves in, not from the positon we wished we had. Drink your wine. It will calm your nerves. We have much to discuss in the days ahead." I was relieved by his forgiveness for betraying my tower. I was shocked by his proposals. I was grateful for the time and the patience he gave to me that night and the days afterwards. In his skilled hands, I felt I could do anything. I found myself longing to be back in Cortan, away from these politics. With him as a guide and mentor, there was no end to the feats I could attain. I left the Tower that night still feeling nervous, but for very different reasons than I had been upon entering. The one thing I was certain of was that I had the tools to win against Ezaro's arguments. 

\vspace{.5cm}

The long awaited morning came. I spent nearly as long on my attire as I had for my wedding day. By virtue of Master Alyosus's tutelage, I was a member, however shunned, of Deyalorn's Tower. I had the right to the Tower's with ceremonial robes. By Master Alerio's suggestion, I wore the maker's symbol prominently on a chain around my neck, to remind the army of the name they had given me, and to remind the clergy that I did not prescribe to master Hortia's heretical views. I wore shoes that made me appear taller than I was, hidden under the flowing train of my ceremonial gown. I wore my hair as I would at my brother's table, pinned elaborately up, held in place by a net of pearls. As was my right as heir to Cortan, I wore a thin silver band on my head, something I had only done a handful of times in the past, for formal court occaisions. My relationship with Cortan was more complex than the process for healing a kidney. I was not the favoured daughter I claimed to be in public. I let the ruse stand. Half of political battles were won in how one presented oneself, Master Alerio told me. My husband had come before the court a prince in healer's robes, let me come before them a queen.

Ezaro appeared as a simple healer, completely unadorned, in contrast to the rich glory that my husband had presented. There was something in his humble presentation that drew the eye. He seemed not as gaunt as I remembered him, but it had been many months since I had seen him. There was none of his usual eagerness about him. Ezaro was calm and composed, he had an air of authority about him that I had never seen him bear. He carried him self  with the confidence of a king, no, like Marsea's Great Teacher. My husband had groomed him to wear the title on his person, and Ezaro wore it as easily as I wore my silver band. I hoped that winning this debate would not come down to simple showmanship. I took my eyes off of Ezaro and thought about the cat awaiting my attentions at the end of the hall. 

We walked to the dias, side by side, younger and opposite images of the men who had passed before us two months ago, whose causes we represented. I passed Duke Ergino, and briefly nodded my greeting. I had heard that he had come to Deyalorn. he had not bothered to send me word of his arrival. I ignored the insult. I was not Cortan's favourite daughter. There was no reason for him to seek me out when he had his uncles' company. I saw Sophia sitting beside her father on the opposite side of the aisle. I noted their position in the crowd so I could keep my eyes off of them. I did not know what I would do in my nervous state if I caught their eyes. I had been nervous when I gave my first lecture in front of Cortan's Towers, that fear and that performance seemed like childsplay now in front of the gathered leaders of the land. I could not think of that, I reminded myself and kept my feet walking steadily forward, and my bearing straight at proud. At the very front, of course were Duke Griswold and Master Alerio. I could not avoid looking at their ees, or feeling the force of their stares upon me. I took a deep breath. I would have to try to ignore it. 

This was a day of demonstrations, a lecture to the general public similar to lectures one gives in the Tower when displaying a newly developed skill or propounding a new theory. As such, my brother Dario had allowed Ezaro permission to consume a dose of the mushrooms that would not allow him to use the old magic. After Head Ionus resounded the gong three times, Ezaro spoke first. 

He stepped forward and spoke of the endless need for forward progress if Marsea is to keep up its great army and record of conquest. He spoke of the dangers presented by Niev, and the lives lost by the shortcomings of the gift as we currently practised it. He spoke less eloquently than my husband had, and for less time, but in a similar vein. He concluded by explaining that even under the stringent controls against the old magic, he had, for the good of Marsea's Towers taken the risk of punishment upon himself to develop a means of healing almost painlessly. The room gasped and murmurred. He then explained to the general public what the military and the Tower knew all too well: that there are numerous proceedures that can only be performed by the strongest healers, or not at all because of the pain the gift inflicts upon our patients. Every generation, there are a handful of healers who can heal naturally with little pain. By strengthening a healer's power using the gateway mushrooms, Ezaro promised that he could teach this skill to anyone who cared to become skilled enough to practise the delicate art. With discoveries like this readily available, Ezaro clamed, the crown's ban on the old magic was a hideous act of tyranny driven by fear. The ban may have been necessary at one point, when Marsea was young and unstable, and our gifted population barely alive. The law protected the country then. That was generations ago, a different time. Marsea had no need for the law any more. Marsea now needed the gift to fully support its population to the best of its ability. Ezaro called for a volunteer to demonstrate the reality of his claim. 

On the of three representatives who had traveled from the House of the Triumverate in Voltain stepped forward. Ezaro drew a dagger and cut a long shallow gash in his forearm, carefully, I noticed, avoiding any tendons or major arteries that are harder to heal. He held the priest's arm up for the hall to see, letting the blood drip slowly down the forearm and collect on the dark sleeve rolled at the elbow for effect. Then he spent fifteen minutes healing the arm while his patient answered questions about the sensations he felt. By the time Ezaro finished, beads of sweat had collected on his forehead. Ezaro, it seemed, had found a way to mimick Carlotta's skill, but at the cost of his full power. He would not be able to heal a more elaborate wound without the aid of the mushrooms. 

Ezaro stood basking in the cries of adulation from the audience. The men sitting around my husband took his hand and congradulated him for his acheivements. Many young white robes in the back of the hall stood for gain a better view of Marsea's Great Teacher, eager for a chance at learning the skills he knew and promised to teach. My stomach tightened. Against this wave of praise for Ezaro, a show of killing a cat paled. I looked at Master Alerio. He sat as calmly and confidently as ever, unperturbed the hushed uproar around him. He met my gaze and smiled his certainty in my skill as a speaker, my finding, and the solidity of our plan. 

After what seemed like too long a period of praise fo Ezaro, Head Ionus took control of the audience and passed the debate to me. I kept my speech short. I had no grand dreams to weave. Just a sharp revelation of the dangers Marsea was about to walk into blindly. "Gathered Dukes, honored clergy, generals, Masters and Heads of Marsea's Towers, I am humbled by this opportunity to speak in your august presence. I thank you for the time you have given me to present my case. I will not impose myself on much of it. We have seen Healer Ezaro's accomplishments. It is a great feat that he has performed. For that I congradulate him." I paused for a small adulatory whisper for Ezaro to die down. "However, the good Healer has not mentioned to you the cost of his accomplishments. I offer you a small demonstration of the potential risks. I must warn you that the following demonstration will be unpleasant. It may not be suitable for sensitive members of gathered audience." A few women left with their children. I signaled for the cats, the mushrooms and the two small flat wooden boxes to be brought out. The healers bearing the testified to the appropriate sealing and storage of the evidence. I invited a member of the audience to check that the cats were healthy and calm. I had chosen two affectionate creatures. They rubbed her head against to the inspectors hand, wishing to be scratched behind her ears. The audience laughed. I took the moment to verify that my seals had not been tampered with. I opened the boxes, and fed a poor cat a piece of bad mushroom. "As many in the Tower suspect, there are occaisonal gateway mushrooms that have unpredicatble effects. I feed this creature a slice of one of those." I placed the animal's cage on a high pedestal so the full hall could see its convulsions. I had taken care to give the cat more than what it needed to die. It thrashed and howled in its cage for less than a hundred heart beats, before suddenly spasming and going limp. I waited for the echoes of the clatter of the metal cage against the stone pedestal to vanish fully before stkicking the dead creature with a knife to show that it was indeed dead. The white half of the court held stern faces and cold eyes. The silence there was complete. As Master Alerio and Master Alyosus predicted, I did not frighten them. I simply threatened to take away their access to the power of their dreams. The other side of the ailse shuffled its feet and murmured nervously. I saw seasoned generals avert their eyes. I hoped they imagined similar violent deaths of their gifted sons and cousins, the possibility of a healer convulsing while saving their life. "This is not simply a death from consuming more than the creature should." I continued after an appropriate pause. I fed the other cat a similar sized piece of the untainted mushroom, explaining my actions as I did. It only took a moment for the creature to start baring her teeth and clawing at her cage, as the expected fear and aggression took hold of its mind. I briefly explained to my ungifted audience the terrors and nightmares that accompany doses of the mushrooms large enough to allow access to the old magic. I spoke for long enough to plant the seed of uncertainty and fear into the ungifted minds without distracting the audience to a full contemplation of the greater gifts promised the old magic. A gloved healer removed the terrified cat from the hall. I held out my two sealed boxes before my audience. "I have here two boxes, each containing one dose each according to my husband's prescription. Each box also contains a piece of the mushroom the dose has been cut from, for your inspection. I fed the first cat from the mushroom in one box. The second ate from the other box. Healer Ezaro, would you care to break the seals, inspect the mushrooms and take one of your safe doses?"

I watched Ezaro pale. "No, your grace. They cannot be distinguished."

I smiled and turned to the gathered dukes. "He is correct. A tainted mushroom cannot be distinguished from a good one. I leave the cases here for inspection at your pleasure. I beg you to consider the risks of the proposed program. My husband offers with one hand great promises for Marsea. With the other, he hides the risks. I would be breaking my vows to serve Marsea if I did not lay bare all aspects of the matter you have gathered here to consider. I remind you that only one in every several hundred children are born gifted in Marsea. If you, my gathered lords, demand that we increase our power using my husband's devices, my generation will obey. Many will kill ourselves in the process. I ask you, who will you find to replace us? It will take years to train the next class of healers. In the meanwhile, who will heal our warriors, who will tend to our nation's wounded?"

The room filled with silence. I could hear Ezaro breathing next to me. I waited patiently for a responce. I was not assured of my victory yet. Master Alerio had warned me of the questions to come. Eventually a Master rose to his feet. "This is unfair, your grace, the tainted mushrooms are rare. The risk is much smaller than the one you presented." 

"No one knows what the risks are, Master." I corrected. "No one has studied it. I will grant you, far fewer than every other mushroom is tainted. Let us suppose for argument's sake that one out of every hundred mushrooms is tainted. A healer accesseing the old magic once a month may never encounter a bad one in three decades. That is not what my husband proposes. He wants to use his small dose to boost a gifted fighter's power. How many gifted fighters does Cortan's army have, father? Around three hundred?" Duke Ergino nodded.  "Very well. Even if one were to take the time and effort of preparing doses for our gifted fighters on the field the day before the battle, serving each fighter from one mushroom only, to prevent contamination, that would kill one of our gifted fighters every engagement. Our men depend on each other during battle, especially their gifted comrades. Is this gamble the promise Marsea's Tower have made our troops? As a child of Cortan, I do not favour using poison against our own men."

I held my peace while the crowd murmured its outrage and sympathy. Makala and Lukos and Firvona's betrayal had not been forgotten. 

When the crowd quieted, a general from Balisal's contingent asked Ezaro "What is the effect of the small doses on humans?"

"There are no ill effects, general. I have taken the prescribed doses since this morning. I am healthy, I suffer no hallucinations or terror. I cannot access the old magic. I freely submit myself to any test you desire." The general asked him to perform a few simple tests demonstrating his coordination and strength and let the matter stand. 

Then the question that would force my hand came. Master Alerio had told me to expect it from a young ambitious healer. It came from a general of Allepo. It was easier to take from him. If I had made the generals of the border duchies nervous, I had succeeded in my task. "You have demonstrated on animals, your grace. What is your guess as to the effect of the tainted mushrooms on soldiers?"

"I do not have to guess, General." I answered, not looking at Master Alerio. I waited for silence.  "I have seen a healer lose a patient in Cortan's Tower because he suffered from siezures while healing on his boosted power." The white half of the hall exploded in anger at my betrayal of my own kind. The colorful half of the room burst into outrage at the corruption of Cortan's Tower, at its overreach in using the mushrooms against the crown's will. 

It took a long time, and several bangs of the brass gong before Head Ionus could restore order to the hall. He then immediately recognized the angry figure of Master Givandi. "This is the word of one woman, known to suffer from madness, not even a healer yet. Her testimony cannot be relied on."

Master Alerio stood in his place, he spoke without permission, and without waiting for the crowd to hush. His clear voice rang out over teh dull roar of the audience echoing in the hall. "As a Master of Cortan, I verify the event the alludes to. I applaud her for her restraint. There is much more she could tell about the events surrounding that incident that she has chosen not to." 

Master Givandi blustered insults at Master Alerio, and Head Ionus hit the gong for quiet again. A representative from the House of the Triumverate rose and waited paitently to be recognized before addressing the crown. "Your highness, this has been a most informative and shocking debate. May I suggest that we conclude for the day and convene a private counsel to discuss how the Tower may be sanctioned for its blatant disrespect for the laws of the land, and its overreach of power?"

"Thank you, Revered Octav. I agree with your suggestion that this gathering be drawn to a close. However, I do not believe the right course of action at present to be sanctions on the nation's Towers. I have heard much about Duke Griswold that may concern the clergy and the military as well. I wish to convene a different counsel. I ask for three representatives from the House of the Triumverate, to speak for the clergy, I would ask the Duke of Selvand and the Duke of Allepo to accompany me to represent the military. I leave it to the White Tower to chose two of its three representatives. I wish for Master Alerio's presence. We will meet in the green chamber in two hours." Reverend Octav paled. He knew of what my brother spoke. He bowed his head and accepted the proposal. The men around the Duke of Selvand murmurred and shifted in their seats and whispered to each other, outraged at being singled out, or aware of their guilt, I could not tell.

The hall stood as the Regent Consort and the Queen Regent left the audience. In the chaos that ensued, I could not tell you who was more stunned, the members of the audience at large, or the two young pawns who had been set before them to allow for this dramatic set of revelations to unfold. My husband stormed out of the Great Hall of the palace. I saw Duke Ergino follow him. Masters and Heads clustered together in heated discussions of who to send to the king's counsel, and what was to be done. Dukes and Barrons and priests mingled, gossiped and speculated as to what my husband's other activities could have been to so upset the crown. Ezaro and I sat stunned at stupefied, trying to comprehend what had just happened. We had been puppets put on a stage by my brother and Master Alerio. We were no longer presenting our study and our findings as we had thought. We did not know what we were representing.

About half the people had left the Great Hall when Ezaro found his toungue. "You betrayed Cortan's Tower!" he spat.

"Yes" I said with a calm that emerged from somewhere deep beneath my shock. "You took vows to serve the Preserver before all else. You broke those vows. What did you expect me to do?" I left the hall.

By the time I had walked to my quarters, the shock had worn away to anger. Master Alerio had lied to me. This debate had not been about the mushrooms. I had not idea what it had been about, but I had been played and used for ends I did not know. I changed into simple healers robes, leaving jewely and clothing scattered on the floor. I went into my workspace and systematically killed every animal in my menagerie. I would not conduct this research for my brother's benefit. I would not do this for Master Alerio's unknown cause, or under Master Alyosus's mysterious guidance. I would not waste my efforts for people who would not tell me what they wanted my results for. It was maddening and demeaning. I was the most powerful healer living. I deserved more respect than I was granted by the rulers of the Towers and the country. I would not be an unknowing servant to somebody else's cause.

I washed the blood off my hands, shuttered the window to my office and sat at my desk in darkness, fuming at myself as much as at my husband and Master Alerio's betrayal. Even after these long months in my brother's palace, I remained naive and trusting enough to think that I had a true ally in the Master. How would I ever protect my children if I did not learn to expect betrayal from everyone around me? I do not know how long I sat there. I do not know how long I would have sat there if Makala had not softly opened the door at some point and ask "Did mommy lose?"

I wiped my eyes and looked at the small frightened face of my two year old son. I smiled in spite of my self. "No, sweet child." I threw back the shutters to let in some light. Makala crawled into my lap and put his little arms around my neck. "I'm not sure what happened today, Makala." I told him, returning the embrace. "I don't think I can lose if I was not playing the game."

\vspace{.5cm}

That evening Master Alyosus called. I told my maid that I would entertain no company. Not Master Alyosus, not the Queen or her neice. Not even Eugenio or Sophia. I would see no one until I could figure out what I had done. 

The Counsel of Nine, as it would later come to be called, lasted for eight days. I spent the first night in a brooding solitude with only my sons for company and comfort. The second night, my curiosity got the better of me, and I emerged from my self imposed seclusion to dine with my brother's court. My brother was not present, but rumours were more numerous than mosquitos in spring. I could determine for certain that my husband was under guard in the Tower at least the end of this counsel, and that one of the subjects under discussion in the green room were various undetermined actions Duke Griswold had taken part in. No one seemed to hold me responsible for any of my husband's actions yet, but as husband and wife we were bound together. I feared for my sons. 

By the fourth night, I learned what my husband had done in Skalsbad. Marsea does not have much of a navy. Navy battles come with a high cost in lives. Putting gifted fighters and healers on ships carries with it too great a risk of losing the gifted blood. Our men will not fight without gifted support. Therefore, in spite of having a coast line to the north, we do not have a navy. Much of our need for a Navy was solved by the Treaty of Cretius about a century ago, forming a peace between Marsea and Szarvis. It allowed Marsea to defend its northern land less well, though they are still very well guarded, and Selvand's Tower is the greatest Tower outside Deyalorn's and the three border duchies', and turn its expansion southward. It has been an extremely successful treaty. However, the Duchies of Selvand and Voltain, the two with coastlines to defend, constantly wish for money from the crown to develop what little navy it does have. Szarvis, while willing to trade with Marsea on many matters, refuses to share any naval technology. My brave husband took a fourteen year old girl to Szarvis, under the guise of his wife, left her on the coastline with a pair of smugglers, traveled to the shipyards of Skalsbad and stole several sets of plans for warships. A little over a year later, when no one was looking for the plans anymore, he appeared in Deyalorn and either sold or gave them to the Duke of Selvand. When the Duke of Selvand refused to train his Tower's healers as sailors, the two men fell out. The Duke's brother, Romero Selvand, Lyca's grandfather, runs a trading business up and down the Tulsi River. His daughter, Leila often traveled with him on the river. I guessed then, and confirmed much later, that my husband was using Leila as means of convincing her father and her uncle towards allowing healers onto ships. I also learned later that my husband had encouraged Leila to learn to swim and sail. It would seem that he wished to turn her into his first model healing sailor.

All this may lead one to wonder why my husband was so eager to set healers on ships. The answer to that question lay to the south. My husband had his heart set on conquoring Lir. He was aware of the difficulties Marsea's armies would have in conquoring and keeping those lands without naval skills. He was not pleased with Commander Dielo and Ambassador Romino's report of naval shortcomings against Lir. He sold the plans to Selvand in the hopes of gaining the Duke's support for the military operation. When that failed, he paid the three of the Speakers for the Triumverate in Voltain to drum up a religious fervour against the Liri gods. The fervour was not enough to sway the Crown, only enough to make Timmon's life in Marsea even more precarious than it had been. The Crown was against a costly war when there was other easier lands to conquor. Marseans are not used to hating other gods. We simply do not need to. That ours are stronger and better is proven by the success of our armies. We have no more need to denounce weaker gods than we have to kick a passing dog. 

Learning what I had, I awoke on the fifth day of the Counsel fearful for my sons. I could not believe the corruption in the man I had married. His attempt to access the old magic with the Towers seemed the least of his faults when compared to bribing the church and international theft, possibly treason. I lay awake at night thinking about the horrors of our trip to Szarvis, and the fact that I was party to what he had done. If anyone recalled, or figured out that I had traveled back with those stolen plans, there would be nothing left for me. Duke Ergino knew, of course. As did my brother. My husband may have enough political allegances to withstand the accusations made of him, but I did not. If the crown chose to punish the favoured Duke of Cortan, how long would it take Duke Ergino to renounce me for my part in his uncles downfall and send me to prison for my crimes? Would my brother protect me now that I had finished my part in his little game? 

I rose in the morning, having slept little, frightened of the days to come. I wanted to cling to my two children, but they were two years of age, and resented being clung to. I settled myself onto their swing in my garden and watched them play with their balls. 

It was a foolish place to be. The garden was open to any guest of the palace. It had no gate to lock to keep out visitors. I could not protect my need for solitude and contemplation. "Good morning, Nisrita. I thought I might find you here." Sophia said, as if she had been looking for me because I was late to appear for an outing.

I scowled at her. I did not want to hear her nonchallance. "I believe I have made it clear that I do not wish to see anyone."

"You did," Sophia replied as if I had asked if I had asked if I had gifted her the hat she wore.

"Then do me the favour of leaving."

Sophia tisked. "You are just upset by the rumours from the counsel. I wanted to see how you were doing."

"Now you have seen me, Sophia. Please leave." I heard my voice rising. This woman was more stubborn than her husband. Why could they not leave me alone?

"Master Alerio is in your office. He does not have long before the counsel meeting start today. You should meet with him."

"No." I yelled. I saw my boys turn to me in surprise. I lowered my voice enough to keep from alarming them. "I have made it clear, I will see no one. I will not be a pawn in someone elses political game. It is bad enough to have a husband trying to force me to act against my will. I do not need this treatment from others."

Sophia smiled and put her arm around my waist. "Which is exactly why you need to see Master Alerio." Her tone was inches away from being condescending. She led me gently towards my rooms. I wanted to resist. I hated her intrusion upon my privacy. How does one fight a determined woman eight months pregnant? One simply does not. 

Master Alerio took my offered hand and bowed. "Good afternoon, your grace." He spoke quickly and formally. "Thank you for making time for me. I will not pretend that I still stand in your favour." Of course he did not, he had heard every word of my argument in the garden. "I have come, however, out of a sense of duty to advise you in this time ahead. What you do with my advice, is, of course, your business." 

"Proceed."

"The counsel's negotiations are still underway, your grace. I cannot disclose any information about them. From the information already made public, you must realize the your husband's actions and guilt are one of the main poins under consideration. If the counsel finds against him, I encourage you to denounce him publicly immediately, offer to give any information you have about his actions to the crown. Your views against your husband are well known, but as his wife, that may not be enough to save your name or protect you from prosecution." He paused. Master Alerio's calm concerned face showed signs of unease. "If the counsel find in his favor, then I will leave for Cortan immediately after the meetings conclude. If you and your children would like to accompany me, I will find you a place of refuge."

"Why will you do this, master?" I no longer trusted the man to work in my interest.

"Why?" the old man broke into a sad smile. "You are a child of Cortan's Tower, your grace. How many times have I had you whipped for entering the Master's workspaces or libraries without permission? If you do not believe a teacher's affection for a strong pupil, then believe that I do not wish to lose the most powerful healer living."

I knew the value of his teacher's affection. I had hear Master Adele say the same. He needed me to serve his purposes, my current state stood in the way of his goal. "Very well, master. If the worst happens. I will be ready with Griswold and Makala." Lyca would not sustain a long journey. I would have to leave him behind, as painful as that thought was. It was better he live in his father's care than die in mine.

Master Alerio nodded and moved to rise. "Do not be angry with him, Nisrita. He means well." Sophia interjected. Her composure had cracked enough to let a sense of urgency through. I looked at her sharply, then at Master Alerio.

The old man took this as permission to speak. "I did not, as you suppose, your grace, wish to force you to do anything against your will. My intentions have always been to cripple your husband's influence in Marsea. As you seemed to share a similar interest, I directed your efforts to support my own. It grieves me that you feel misled."

"I would not have felt misled if you had told me what your plans for the debate were."

Master Alerio considered my person. "You are passionate, as befits the young, and inexperienced, as proves their downfall. If I had told you, you would have been too nervous to give that rousing performance. The Towers of this land will now all be eager to systematically study the effects of the mushrooms. If anything, your experience in this area will be in high demand."

"The old magic? After all this, the crown will lift the ban? My husband will have won?"

Master Alerio shook his head. "I have said too much. Nothing is for certain. Everything is much more complicated that you imagine. By your leave, your grace, I must attend the counsel."

The old man left, and I sank back into my chair, my forehead supported by my knuckes. I had been played. Surely and certainly, I had been played. I had made alliances without fully understanding my allies, and now the cause I had fought for, as well as the safety of my sons hung in the balance. I had known that I risked everything when I had started this battle, I reminded myself. That did not help me accept my fate now.

"My father sends his apologies," Sophia said gently.

I looked up in surprise. "What does Master Alyosus have to apologize for?"

"He did not want you involved in this matter in the first place. He regrets having listened to my advice that you would not have accepted him as a tutor if he tried."

I gaped rudely at Sophia. Father and daughter had argued about this recently. I could not imagine an argument in that composed Joris household. "You were right. I would not have."

Sophia's light manner returned. "Father also suggests that given your husband's current state of shame, it may be possible for you to attain your healer's rank before you finish your manuscript, if that is still your wish. He does, however, encourage you to finish that work. He also says that if you have any need to move your menagerie from here, you are welcome to share his space in the Tower."

"I no longer have a menagerie," I admitted sheepishly. "I destroyed it in anger the day of the debate."

Sophia threw back her head slightly and laughed a silvery laugh. "Young and passionate, Nisrita. You are the only woman in Marsea who cannot be refined by company of Deyalorn's court."

\vspace{.5cm}

The Counsel of Nine emmerged from their negotiations a week before the Day of Unions. Heads of several Major Towers were replaced, including that of Cortan. Master Alerio was the new head of Cortan's Tower. Others were also replaced my men more reputedly loyal to the crown, not necessarily not desirous of working with the mushrooms. Similarly, the leaders of several temples stepped down to be replaced my men equally devout, and percieved as less corruptable. The leadership at House of the Triumverate in Voltaine was almost completely replaced. Several dukes, including Ergino, Erfat, and the Duke of Selvand, had representatives from the Crown pushed onto their counsels. My husband, it would seem, in his overreach, had toppled the very systemt of corrupt rulers he had wished to use for his own purposes.

The new religious leaders promised to undo the false religious fervour that had been created by their predecessors. The crown promised resources towards the building of a small fleet of ships, though the money went to Voltain, not Selvand. The ban on old magic still stood, but the Towers would be allowed to systematically test the mushrooms until all branches were convinced that a safe method of healing with them had been found. Any Tower that wished to undertake suc research had to submit to monitoring by the crown. Duke Griswold's name had been destroyed. He would not be tried for treason, but he would remain, for the time being, a prisoner of Deyalorn's White Tower. He would be allowed vistors, but not freedom of movement.

I did not know what to think. Both Master Alyosus and Master Alerio congradulated me on my victory. I accepted their well wishes with half a heart. It would seem that so many of my husband's corrupt desires had been acheived after all, that this could hardly be called a victory. Master Alerio had won a resounding victory, certainly. He had crippled my husband's influence, rid the Towers of his corrupting power, won the crown's favour, gained the opportunity systematic study of the mushroom's power. For Master Alyosus, his children were safe from my husband's retribution, I had helped protect his family's name. He had no further stake in this debate. I, on the other hand, had put too much at risk. I had won the right to be made a healer by the same standards as applied to all others of my age, if that could be called a victory. Beyond that, I had only won a pause in the never ending fight against my husband. I had achieved my ambition for now on the old magic. I had not won on the lareger issue of mushrooms. I trusted my husband to find a way to push for his ambitions again soon. The threat of his weilding the old magic again still loomed over Marsea, though few saw it. I now faced a choice of denouncing my husband, or sharing in his shame. In denouncing him I would be unpopular in Deyalorn's Tower, and I would lose favour with Griswold himself, to say nothing of Duke Ergino. Unlike the Masters who congradulated me, I could not leave my husband. While he remained in Deyalorn, I must practise in her Tower. My colleagues would not love me for my denounciation. My brother could not keep my husband in Deyalorn's Tower for long. He would eventually be set free to do as he pleased with the wife that had turned against him.

The other option was worse. I denounced him, opening myself to the vengance of his allies. I publicly offered my testimony to the crown should they wish it. In return, my brother promised his protection against those who would wish me or my sons physical harm. I kept my sons and my freedom for the time being. I felt the eyes of the court bore through me as I walked through the hall. From this moment forth, they seemed to say, every action of mine would be scrutinized. I had no where to hide. Sophia came to walk me from my rooms to the great hall then back again after I had finally made public what my associates knew to be my feelings towards Duke Griswold. The implications of my actions had subjugated even her eloquence to silence. As she left me at my door, she said only, "I would have done the same." It did not matter how skillfully she did not speak of it, or how happy she seemed with Timmon, Sophia had not forgotten the violent and controlling man that had given her Eugenio. I embraced her in thanks for her support, and we parted. There are actions after which one simply needs to rest in solitude, that no strength of friendship can comfort.

Three days later, I became a healer in Deyalorn's Tower. Officially, my husband's allies could not collude to keep me from my art. That did not mean that they supported me. Carlotta flatly refused to speak to me. I had betrayed Cortan, I had denounced my husband. She saw no reason to associated with a traitor such as I. Many of Deyalorn's Tower felt the same. Some went so far as to comment that I had acheived my rank because of my brother's wishes, and no more. I chose not to fight them. I was tired of fighting. The day I became a healer should have been one of the more joyous days in my life. Instead, I felt that if denouncing my husband ensured that I would not rot in prison, becoming a healer meant I would not starve. I found myself retreating from the Tower to my garden, watching my boys play, reminding myself that if it should come to where I should have to accept Master Alerio's offer of protection from my husband of Duke Ergino, I would be able to provide for myself and my boys by virtue of my art. At least I had won that.

If i found myself isolated before the Counsel of Nine, I actively sought solitude now. I ferreted myself away from the animosity I met on the streets around me and worked on my manuscript. I did not need to hear whispers that I was an unfaithul and wicked wife by the nobility, or a treacherous healer by the Tower. If every my every public action was to be scrutinized, it was better not to appear outside at all.

I scowled when Malia bounced into my office and asked "Are you not going to have any fun today?"

"No, Malia," I replied, not looking up from my transcription. "I have no intention of being in the streets watching people make drunken lusty fools of themselves." I had no intention of being on the streets at all.

"But it's the Maker's celebration," Malia whined.

The Destroyer could eat the Maker for all I cared, I thought. I had had enough of marriage at the moment. "Then enjoy it with your friends. Don't bother me." 

"That's just it," Malia explained timidly. "We want to walk the maze of veils, but ..."

But they were all thirteen year old unmarried girls. "You are too young," I barked, knowing that to be false. I was the same age when I had walked the maze with Makala, shocked and mesmerized at the shadowy movements I saw behind the thin white screens. The happy memory stung.

"Please" Malia insisted. "we promise we will behave as young women, and not giggle like girls. And I will sit with you for longer tomorrow to help you with your drawings."  

The primary purpose of the maze is for the enjoyment or edification of newly married couples, as a prelude for the night to come. Unmarried girls are allowed to walk the maze chaperoned by a married woman, preferrably a mother or an older sister, who will use the opportunity to teach them about the marriage duties. Carlotta always found a married friend to chaperone her to walk the maze of veils for the pleasure of the experience, but she had always been more interested in these matters than I. 

"Can't you find someone else?" I did not want to accompany Malia and her friends. The last, and only, time I had walked the maze had been so long ago, the experience so pleasant, my life so different from where I found myself now. I had no desire to remember what I had lost. 

"We could." Malia paused to form her statement properly. "But they would tell us about married life and duties as if creating gifted children were all we could do. You..." she trailed off.

"I am jaded and cynical enough not to feed you that nonsense, so you can enjoy yourselves?"

"No, oof, I'm making a hash of this." Malia spluttered. "I, we, my friends and I, that is, know you want more than that. I've seen how hard you have fought to become a healer. I would trust what you had to say."

I put down my papers and looked at her. I had been her age when I first married. We were four years apart in age, had I ever been that ernest and innocent? I must have been. What had happened to me in these last four years? It did not bear thinking about. Malia looked back at me ernestly. Could I protect her innocence, I wondered. I could barely protect my own. She was a smart, diligent girl. I stood up. "Clean up in here. I'll get dressed."

Malia introduced me  to her two friends, Lyta and Uma, girls in her class, who seemed not to have the slightest interest in talking about marriage or mating. We walked slowly in the direction of the maze, pausing to watch suggestive performers, or buy morsels of food and pretty trinkets as we went. They told me about their teachers and their friends, and the tricks they played on each other in the dormitories. They were so young and hopeful, they still had over two years of formal education left in the Tower. They reminded me of those heady days when my greatest fear was being caught stealing oranges from the Tower's orange grove after curfew. I had lived such a joyful hectic life then. I would position myself near the first of the line of healers placing our offerings to the triumverate in the predawn prayers, not because of any intended disrespect to the gods, but because I had to sprint from the temple to the practise yards to meet Makala and the black riders the the hour when my fellow students breakfasted, then sprint back in time to swallow a dry biscuit and duck in, never more than a few minutes late to my morning classes. The thrill of maintaining this routine and escaping Master Adele's notice was what passed for danger and excitement in those distant innocent days. How did they disappear?

"Healer, what did you want to do with yourself when you were young?" Uma asked 

I returned myself to Deyalorn from the dusty dawns of Cortan. "I wanted to be the head of Cortan's Women's Tower."

Lyta laughed nervously at the audacity of my youthful ambition. Malia put her arm in mine. "Wanted, Nisrita? Do you not want that any more?"

The girl had a charming directness about her. She reminded me of someone I knew once. "No. You are right. I do still want it."

As we walked to the maze, I told the girls about traditions and rituals of Cortan's school, comparing them with their experiences in Deyalorn. The maze itself was much as I had remembered it, a winding path formed by transluscent white screens that took the traveler past several stages blocked off by billowing cloth screens where shadowy dancers moving erodically in the diffused light. The posts from which the screens hung were carved and painted with crimson scenes of husband and wife in all stages of marital life, beautiful women supplicating their husbands, strong men protecting their families. husband and wife praying together, eating together, lying together. We passed newly wed couples whispering affectionately to each other, and others where the man looked on in lusty fascinations while the wife stood awkwardly terrified, or embarrased at his side. There were a few older women who educated their daughters or neices on the art and importance of pleasing men and bearing children. Invariably the daughters watched in fascination or discomfort at the suggestions surrounding them of the duties they would have to perform. 

The maze wove its spell on me. I found myself thinking of wedding nights, as it intended. I remembered the laughter of the first one, Makala's silent shaking shoulders, and my girlish giggles when his kisses tickled my skin. My first husband had spent my wedding evening bending my mind to the task at hand. He had started by introducing me to sweet wines stronger than I had tasted, wrapped me in his warm cloak as we walked the maze of veils to arouse my curiosty, followed by syrupy cherries and whispered stories that explained what the maze left unclear. When he had entered, I felt as if he fed a hunger in a part of me I had not hitherto known had existed. When he rose from me, I ached with pleasure, unaware of a world beyond my own skin. Makala lifted my half naked from the bed, taking me to the idol of triumverate in the room while that first taste of desire and the delight of satisfaction still surged through my body. He asked me to exchange private vows with him, before I completed the marriage by publicly asking the Maker to bless our union. Makala wanted me to promise that nothing would change between us with the act we had just performed. I was drunk on wine, delight, and the excitement of the new experience. I promised him anything he wanted. In return he promised me that he would see me head of Cortan's Women's Tower. We hardly slept that night, but built a thousand castles on clouds. I told him my dream of riding with his riders, not necessarily into battle, but with them. He had laughed at my impossible ambition and told me he would see what he could do. There seemed nothing I could dream of that Makala would not encourage me to find a way to make real. He spun me fantasies for Cortan's greatness that we would create together. He told me of the gifted children I would give Cortan. Hand in hand with his new gifted wife, he promised to bring the Tower and castle together as the old kings of Marsea had, in the gift was young an no one new the limits of our power. Makala was full of dreams. One would never have guessed from his optimism that he lived his life guarding such dangerous secrets.

"Is she naked?" Lyta gasped, returning me to my duties as a chaperone.

I inspected the shadows chasing each other gracefully and coyly around a stage. They wore bells on their bodies, the silvery tinkling suggesting the sound of lover's laughter. "I doubt it," I said. "She is wearing a tight fitting bodice."

"But her legs," Lyta whispered.

"You have them too," I smirked, "under your skirts somewhere."

Lyta blushed purple, and Malia leaned forward to search for signs of the rumoured bodice. I left the two girls to their fascination, and joined Uma uneasily standing back several paces. "This can be an overwhelming experience. That is nothing to be ashamed of."

The girl shook her head. "My sister wed today. My brother in law is the head of the Ironmonger's guild. He is over sixty with many children, some older than my sister. She is his second wife." I put my arm around Uma's shoulder. Her sister's marital life would be a far cry from the fantasy depicted here. "My father is a tailor." Uma continued. "My sister helped him with his work. She is quite good. I doubt she will be able to practise her art anymore."

I wondered what Makala would have said. Even if he did not have a scheme, he would have swept away her gloom. Nothing was impossible to him. "You never know," I comforted, "Give her the comfort of your company. Some ungifted women find ways to have a life outside their marriage. It is rare, but not impossible."

I led Uma through the rest of the maze, her friends following reluctantly behind. I was in the company of extraordinarily ambitious women. Lyta came from a distant barrony. She transfered to Deyalorn when it became clear that she had a gentle touch, as Carlotta had. She missed rural life, and wished to serve in the Women's Tower near her family, possibly giving herself up to the Preserver's service if that was the only way to satisfy her ambition. Uma dreamt of seeing the world by marching with Marsea's armies. Malia, inspired by her work with me, wanted to become a scribe in Deyalorn's library. There are very few female scribes in the Tower. One had to become a healer first, then apply to the Temple for further training for several years. It was so heartening to hear the them talk. They were still inexperienced enough to defy the barriers that life would put up in front of them. If I were Makala, I would have used that innocent defiance to help them each forge a path forward. In the process, I realized, I would create a community in the Tower that supported me. These girls were young and powerless, but they could do great things if they had someone like Makala to guide them.

Malia shuddered audibly. The night had grown chilly. I bought the company hot fresh ciders, then watched them giggle and cavort down the street to see a bawdy play that Malia thought she could talk her way into without my company. They were as confident as boys. It had been nurishing to be in their company. I walked slowly up the hill back towards the palace, letting my feet lead, while my mind wandered in a fantasy where slander did not circle me like a vulture, waiting for my first mistake, and young ambitious girls could frequent my company without fear of retribution from the Tower.

I found myself not in the palace, but in the great temple just below it on the bluffs. I knelt before the Destroyer and lit an oil lamp and three candles, one for each year Makala had not walked this earth. I prayed for his spirit, whether still adrift, or in the body of my son, or elsewhere, I did not know. I thanked him. 
I had thought that he had left me to drift near Turina, but I was wrong. He had guided me tonight. I did not know how I would come out of the isolation I now faced in Deyalorn, but if he were here, he would have a plan. I his spirit for guidance. I knelt and prayed until the tapers guttered, then lit three more. There was an intimacy in the solitude and silence of the temple, a hope that my husband's spirit may touch me with his kindness. I did not wish to leave.

Eventually, the new wives who had just consumated their marriages started trickling in to lay their offerings at the Maker's feet. I moved from my position before the Destroyer to a quiet position in the back of the temple. I did not wish to appear a widow casting grief on their new unions. I watched women enter one at a time, or in clusters of two or three. "May the Maker bless their unions," I prayed. "May they be filled with laughter as my first had been, not desolate, as my second." The red wax balls laid at the Maker's feet filled like pincushions with the sticks of incense offered by newly formed women trembling with excitement or fear at their new experience, and older, more staid women on their second or third marriages. Guards outside the temple waited to escort the new wives home to their sleeping husbands. 

Leila, nee Selvand, saw me in my corner. "I did not expect to see you here, your grace," she said sittin next to me.

"May I congradulate you on your union, Mistress..." 

"Baroness Paulis" she supplied.

"You have done well for youself, baroness," I smiled, unable to completely hide my surprise. "I presume you mean the treasurer's son?" 

Leila blushed modestly "My father did not have the heart to marry me to a distant barrony." Her father must love her dearly, I thought. Selvand's name had been shamed by the Counsel of Nine as well. This union must have put Romero Selvand in debt to the crown. "I would like to ask you to visit my new household, as a token of thanks,..."

"You may visit my sons as you see fit," I interrupted, "there is no need for thanks." If I could send my twins to Lir when the time came, I schemed, and Lyca to the Paulis household, then I would have much less to fear from my husband.

"That is most gracious of you, your grace, but that is not what I meant. I wish to thank you for speaking against your husband. You spoke for me as well. I am more ashamed than I can say to have been drawn into Duke Griswold's plans so easily. I do not want you to suffer alone for doing what was the only honorable thing to do. I would be honored if you visited my husband's table." 

I thanked the now confident woman, and followed her out of the temple, clutching to this seed of a plan that Makala, in his undying love, had given me. 

\begin{comment}
\vspace{.5cm}

Over the course of the next few weeks, I found my courage again. Leila introduced me to a host of potential allies in Deyalorn's court. Malia worked tirelessly with me on my manuscript. With her help, I could finish the work by the solstice. I learned that Master Alyosus's son, Velo, a gifted captain in Deyalorn's guard was a devoted uncle, a relationship he eagerly shared with my twins. Master Alyosus encouraged me to study the effects of the mushrooms again. He showed a gentle amusement to my confession of my angry mass slaughter of my subjects.

"I do not believe young women should involve themselves in politics." He said when I asked him why he had a sudden change of heart. "Now it is not a matter for the politicians to decide based on power and influence, but for the Towers to decided based on the merits of  the case." 

Sophia cast me a brief warning glance from her needlepoint. I obeyed and kept my laughter to myself. There were plently of examples of Joris women involved in politics, the Regent Consort being the first example that came to mind. Furthermore, while the debate had passed to the confines of the Towers, and the merits of proper study would certainly factor more strongly in the final outcomes than it had in the hands of the dukes and the priests, power and influence would still play a large role. The support of a Joris of Deyalorn's Tower would help me get my views heard. If I could get Master Alerio's support as well, they would certainly be listened to. The montors for Deyalorn's Tower had not been appointed yet. I would have to make friends with the general overseeing Deyalorn.

Sophia gave birth to a healthy daughter as the elm tree in my garden changed its leaves to brilliant shades of yellow and orange. She named her child Nephriti, an unusual name, that she claimed belonged to the girl's parternal grandmother. Barroness Romino was named Ismain. I marveled at her boldness, jealous of the liberties Timmon allowed her. Makala, Woldino and I took equal delight in the changing colors of the trees in my garden. I brought Sophia elaborate bouquets of the prettiest leaves from my garden for her confinement chamber, while she laughed at my childish amusement. I did not care. I reminded her that I had not grown up in these lush fertile lands, and that after the events of the summer, I should be allowed my pleasures where I could find them. By the time she was released from her confinement,  I had not seen my husband in nearly three months. I had recieved news of Master Alerio's promotion to the Head of Cortan's Tower, Head Corino and Master Adele having been disgraced and forced to step down by the Counsel of Nine. The ambassador from Lir, Sama Araki, had arrived in Deyalorn, bringing with him news of Timmon's journeys, as they had passed each other on the road. The greatest anxiety that haunted my days was enduring the oncoming winter, with its periodic snowfalls. 

It was not that I had resolved anything with regards to my husband or my children, I was simply less isolated. With Leila's help, I became a known face in the households that chose to align themselves against my now unpopular husband. My husband's downfall left many people scrambling to change sides to regain the crown's favour. Baroness Paulis did not find my ill controlled tongue to be an embarrassment. As I was momentarily not in direct conflict with my husband, I said nothing to make her regret her hospitality. I met the court and established connections in Deyalorn. 

Master Alyosus's support, and my growing popularity outside the Tower had its ripples inside the Tower as well. I developed a new menagerie in Master Alyosus's workspace, and started testing the effects of tiny doses on the animals. There were others performing the same set of tests, as it had been deemed necessary by the Counsel of Nine. With my greater experience on the matter, I was begrudgingly given opportunities to lecture on my work. Head Ionus made it clear that I would only be allowed this priviledge as long as I kept my lectures based on my findings, and not ideological in nature. He did not place such restriction on others studying the mushrooms. It was humiliating and aggravating, but I held my tongue. I was certain that barring a deliberate misreporting of findings, all groups would find the mushrooms too dangerous to use. How could they not? I was young. I had been given a chance to lecture alongside Healers and Masters of far greater experience. I had access to Master Alyosus's students and resources, and Head Alerio's ear. I would find a way to show the Towers the error of my ways. I would not toss away these hard earned gifts because my ego had been bruised. This promised to be a longer battle that the one I had just engaged in with my husband. I knew this fight better that the one I had just participated in. I would move cautiously, and carefully. The mushrooms would show themselves to be too dangerous.

I found a kindred spirit in the Liri ambassador, Sama Araki, at least with regards to my fear of winter. After his formal introduction to the royal court, I invited him to join me for warm spiced wine and a fire that made Lyca's Deyalorn born nurse uncomfortably hot. He was a tall dark, bearded man, of about forty, with wide features, and a bulbous nose. He was not what I would call handsome, but he captivated my imagination. He dressed in embroidered silk tunics and tight fitting trousers that had extra fabric that gathered about his ankles. To protect against the cold, he donned a padded silk vest, which was barely sufficient in autumn, and would do little for him in winter. He smelled of sandalwood, and dyed his eyebrows a deep brown-ish red. He spoke Mer with a heavy accent, and traveled at all times with a trader who knew our language better. He carried himself as a man of refined tastes, though his tastes were decidedly different from that of Deyalorn's court. For instance, in Lir, plays are not performed by the spoken word, but solely through dance or song. He found Marsean drama to be dull and uninteresting. I could not see how a Liri play could contain half the intricate plot twists of a good Marsean comedy. He deplored the custom of kissing the hand of a queen or a duchess to show respect, though he performed it gracefully. To him, the thought of touching a woman not one's blood or one's wife was scandalous. One certainly paid respects to the queen, but from a distance. Even the scenario for our interview would be unthinkable in Lir. It would be impossible for a man to meet with a married woman without the presence of her husband. I despaired for Sophia and asked him what he thought of our Day of Unions. Sama Araki did not think the day strange, the Liri had various fertility rituals when women were allowed certain licenses, though none allowed as many. He thought the idea of sending women to their husband's homes alone to be cruel. In Lir, women always went to their husband's homes with a servant from her own, or women were married in pairs, so that they had someone's company to help make the transition easier. Otherwise, Liri husbands would have to spend the first year of their marriage consoling their wives and helping them adjust to their new lives. I admitted that our women must be braver, though I had enjoyed the good fortune of never being married far from home. 

Our meetings became a regular occurrance, in spite of the scandal that Sama Araki insisted they would cause in his homeland. I coaxed him into telling me about the life that Timmon would lead. I knew little about the LIri gods. Timmon could speak little about them, with the false fervour against them that my husband had helped drum up. Sama Araki informed me that it is assumed that all Liri men have and maintain close male relationships throughout their lives. There are many needs, spiritual, intellectual, emotional, and when prodded discretely, the ambassador admitted, physical, that a wife cannot hope to fulfill. I feigned my disgust and told him about our concept of miscast people. Sama Araki laughed. His gods never erred. How could one trust gods so prone to mistakes?

We never spoke of anything in our meetings than I would with any man of Deyalorn's court.  though he never said or did anything inappropriate, the ambassador's manner was so intimiate, so intense, his leaving always left me with the feeling that more had passed between us than actually had. 

I learned much of Liri culture from Sama Araki. With time he grew to be a friend and confidant. That autumn, I found myself dreaming of him, and waking with a desire that was not appropriate. He would be a possessive and domineering lover, not an entanglement I needed, I reminded myself upon waking from these dreams. He would not be able to protect me against my husband when that need arose. An affair with him would simply put Marsea's relationship with Lir at risk. More importantly, and this I had to say out loud to myself in my dark room at night, in spite of his deep knowledge of the culture and land that I hoped would make Timmon happy, in spite of his ability to help me imagine the life that Timmon would lead soon, he was not Timmon.



\end{comment}

\end{document}
 
