\documentclass{article}
\usepackage{fullpage, verbatim}
%*****************
% Annotations
\usepackage{soul}
\usepackage[colorinlistoftodos,textsize=footnotesize]{todonotes}
\newcommand{\hlfix}[2]{\texthl{#1}\todo{#2}}
\newcommand{\hlnew}[2]{\texthl{#1}\todo[color=green!40]{#2}}
\newcommand{\sanote}{\todo[color=green!30]}
\newcommand{\egnote}{\todo[color=violet!30]}
\newcommand{\newstart}{\note{The inserted text starts here}}
\newcommand{\newfinish}{\note{The inserted text finishes here}}
\setstcolor{red}
%***************************

\begin{document}
I stood as arranged, in a copse two hours outside Deyalorn, ten days before the Day of Unions. "I trust you can see yourself to the gate from here?" Griswold's cowled figure asked.

"Yes." I would be traveling armed through the civilized country surronding the heart of Marsea. This was the safest journey I could possibly take. Four months of living with me had not taught Griswold that there are many circumstances under which I can take care of myself.

"Then we part ways here." He handed me his horse.

I hesitated. Four months of living with him had taught me the futility of questioning his will. This was too important an occaison to heed that lesson. "When will I see you again?"

He frowned his familiar frown, and gave me that dreaded look. It was the look he spared for me when he gave me a test he expected me to fail. "That depends on what you do next."

I threw up my hands in exasperation. "Give me some proof, then."

"You have the letters," he responded flatly.

He was asking me to do the impossible. I needed more than just two letters. I took one last foolish chance in my desperation. "I have lived with you as your wife for these last four months. Does that count for nothing? If you had any honor, you would come with me now."

His face twisted in derisive laughter. "I did not find you a virgin, girl. I have done nothing to hurt your chances at a successful match. Do as you must. I will keep my word."

He turned and made his way through the trees to who ever had agreed to next protect his identity and provide him shelter. He did not see fit to even tell me where he would be. I stood and stared at where he feet had stood before me, furious at myself, furious of him, terrified of appearing at Deyalorn alone. What had I been thinking? That if I gave myself over to Griswold completely, when this day came, he would follow me into his nephew's arms out of a sense of honor? Griswold did not care for honor, neither his, nor mine. He would not return with me. Where did that leave me but as yet another of Griswold's whores? 

In spite of the terrible things I had seen these past few months, was I still nothing more than a foolish romantic child, I told myself. Was my head still caught in the stories of battles and lovers told after dinner rather than the realities of quests and warfare. Griswold was, as Groto had aptly described him, a cynical old man. The sooner I learned that cynicism, the better off I would be in this world.

My horse snorted at my anger impatiently. I mounted and dug my heels into her sides. I had to perform a miracle at Deyalorn. In all likelyhood, I would fail, and be laughed out of the castle, my duchy and my tower for my efforts. I had learned from the potter and fisherman we had stayed with the past two nights traveling to Deyalorn that Cortan had given me up for dead. Duke Ergino and several representatives from other duchies had collected already at the city to settle the conflict between Cortan and Firvona. Who would believe a dead duchess if she claimed to have traveled with a duke who had been dead for longer. Especially when she had no proof, nor knowledge of his location. Duchess Cybeline sought any excuse to discredit me and disown me as heir. Madness would give her the perfect excuse.

\vspace{.5cm}

It took me less than an hour to reach the tall spires of Deyalorns east gate sparkling in the afternoon autumn light. I paused my stream of self hatred and sense of futility to take them in. They were a brilliant monument to the greatness of Marsea. Last year, when I had first came to this city, a child who had never been outside the harsh border dutchy of Cortan, they had impressed me with their size and radiance. This year, I have traveled in Niev and Szavis. The spires remind me of the greatness of my home. Deyalorn lay behind, a vast, rich city. At this time of year, it teemed with people for the festivities of the days ahead. I recalled last year's childlike wonder with which I had gaped at the tall ornate walls and the white arched gate through which I had to pass into this city. Life had been so simple then, I thought only of Makala and my wedding night, dreaming of the wonder and attention I would earn from my classmates at the Tower when I returned to show off my wedding chain. I hardly recognized that child. Was it only a year ago? I found it hard to believe. So much had changed.

I joined the travelors crossing the wide bridge over the Tulsi river to the impressive east gate of Deyalorn. I could see the palace and the White Tower, high above on the bluffs overlooking the city. The river seemed tame and domesticated compared to the ocean I had just traveled. It as wide and deep. Quiet river boats ferried goods up and down its length. It held fish for the town. Washerwomen came to its banks with baskets of laundry. Children played at its banks. It did not rear and up on its hind legs with waves threatening to swallow houses whole. It did not contain great beasts larger than ships that could leap from the water and dive to unimaginable depths. Herds of ferocious predators did not sun themselves along her shores. I laughed at my memory of the child that had stood on this very bridge and gawked at the water flowing gently below. I admired Makala even more for not having laughed at my ignorance. 

I entered the city and made my way up along the half familiar path I had traveled a year ago to the palace a Deyalorn. I had not had the opportunity to walk the city streets then. I wondered what fate awaited me at the palace. Would I be allowed an escort to walk these streets, or would I be forced onto them, rejected by my house and Tower. I had done so much these last four months that I should not have. I had learned and seen so much. What would happen to me if anyone found out the truth of all that had had transpired? Would they call me insane, or would they exile me for my crimes?

I approached the palace gates and announced myself and my intension to speak to Duke Ergino, and my half-brother Regent Consort Dario. The commotion I caused was about what I had expected. The crown did not have reason to hate me as Cortan did. Unlike my return to Cortan, I was treated with suspicion, but respect. The guards escorted me into a comfortable office in the guard tower, given refreshments, a page to see to my needs and a set of guards at the door. I helped myself to some fruit, bread, and a book pulled at random from the office's shelves, and waited. What was I doing? I had no proof of Griswold's exsistance. The world would think me mad.

I had met my half-brother, the Regent Consort only once, at last year's Day of Unions. He was twenty years my senior. I did not know him. My father, I have gathered, had been a man of some importance in Deyalorn's court in King Romen IV's day. His grandson now held the country's scepter. Something happened, I do not know what, that caused my father to loose King Romen IV's favour. He moved to Cortan and settled with a gifted woman far below his ranks in an order that is unclear to me. The Regent Consort is the only of my father's children from his first marriage that I have ever met. My father died of diarrhea and vomitting from drinking bad water while traveling soon after I was born. I believe he had taken on a diplomat's mantle in the new Duke Ergino's court. My mother died of the summer fever when I was less than two. That is all I knew of my blood.

I did not hold any sway over the crown beyond this faint blood tie. I did not think an audience with my half-brother would help Griswold's cause in the slightest. Yet that is what he wished me to do. Griswold had given me two letters. One for the Regent Consort, one for Duke Ergino. Armed with these two pieces of paper, I had to convince Marsea that Griswold was not dead, that he should be pardoned and allowed to live in Cortan. Only then would he return, he had promised. It was an impossible task. I may as well admit now that I had failed in my mission to bring Griswold back to Cortan.

The door opened, and I rose for inspection. Chamila, now Baroness of Cleyoren, entered. We both gasped in surprise when we saw each other. She looked, if anything, prettier than she had a year ago when I had seen her last, married to the Baron of Cleyoren. She had a child growing in her. Other than that, she looked just as innocent as she had been that day. War, death and grief had not played darts with her mind as they had for me. I found myself consumed with an irrational jealously of her quiet life. The past year may have been kinder to her than it had been to me, I tried to tell myself, but I did not want a life where I spent my days serving my husband's guests and bearing his children. That was why I had married Makala.

"What have you done to your hair?" Baroness Chamila asked after she had walked around me once. I had cut my waistlength hair to fall to my chin when I left Turina. It now reached just beyond my shoulders. When posing as Griswold's wife, I kept it up, hiding its unusual length as best I could. When traveling, I kept it down. It would be safer if I appeared male at first glance. I had not thought to put it up again for this inspection.

"I have had to travel in hiding for a time, Chamila. Posing as a man seemed the easiest."

"How exciting!" my childhood friend said, taking my hands. I smiled weakly at her. These past five months had been exciting, without the positive connotation she gave the word. "You have grown so skinny, you look different."

"The road has been hard," I admitted. The journey to Skalsbad had been easy, we returned in hiding with rough company. Food and shelter had not always been readily available. 

Chamila bent towards me and whispered conspiratorially "I suppose I should ask you some questions that only you can answer?"  Was this a joke to her? I sighed and endured a series of questions that anyone with a reasonable knowledge of Cortan's Tower and Castle would be able answer correctly. This seemed to satisfy Baroness Chamila's sense of adventure.

"It is she," she announced importantly to the guards behind her. Then she turned to me and led me out of the guard tower towards the palace. 

She put her arm in mine and giggled. "I am glad you are alive, though I do not think mother will be." I gathered that our father was not pleased at the news either. That would explain why he sent his youngest daughter to identify me, rather than coming himself. Chamila leaned in close to me and spoke in hushed excited tones. "She is trying again you know. She wants an heir of her own blood for Cortan. It is simply scandalous at her age." 

I took in her news carefully. It should not have surprised me. I should not have been saddened by Duchess Cybeline's desire to replace me. I felt unwanted all the same. How was it possible that an infant could lead Cortan's armies better than I? If only Griswold would arrive. I could prove my ability to serve Cortan, show that I was of some use to her. I controlled my disappointment and spoke. "Mother is only just over forty. The Tower will be able to deliver her safely of her child. It is not so strange a dream on her part. I wish her good health."

"Pooh, Nisrita. You haven't changed. It is not her safety or the Tower's skill I am worried about. I am concerned about the talk mother's new pregnancy will generate. It has been so long since Antonyo."

I held my tongue and let Chamila prattle on, vowing that I would be patient. It was not her fault that I had chosen a life of hardship and grief instead of living comfortably and protected in Cortan's Tower. Each of her three sisters, and more importantly, their husbands had arrived in Deyalorn a week ago for the Duke of Firvona's trial. Cortan's children were close. The sisters could not sit idle while rumors circulated that the Duke had ordered their brothers assasinated. Master Adele had testified before the crown as to the manner of Lukos's death. It was the only solid piece of evidence that Cortan had. Firvona had hoped that the deaths would be thought of as a tragedy of the battle field, evidence scattered in the chaos of war. It had almost succeeded. With the heads of two duchies and an heir to a third lobbying the crown against Firvona, it now seemed that the crown would be forced to side with Cortan. 
I was relieved to hear the news. Makala would be avenged by the crown. His sisters had not let him down as I had. They deserved to call him brother. I did not. I barely deserved his widow's veil. I had wasted the last four months trying to return Griswold to Cortan, only to fail at the last step in this impossible mission.

We entered Chamila's rooms and let her maids fuss at my presentation. I had not brought anything suitable to wear before the royal court. I had not had the time to sit for tailors while traveling with smugglers. My concern had been survival. I had planned on borrowing ceremonial robes from the Tower, if I could get an audience at all. Chamila poohed at my dowdy choice of dress and put me in one of her gowns. It was too big for me, given her current state. Her maids pinned and tied it to my body until Chamila approved. 

Then she sat me down and asked "Are you here to marry again?"

"I suppose I am." That had been, in Griswold's opinion, how I could best serve Cortan now. 

Chamila looked at me with approval. "My brother must have tamed you," she said. "I have never known you to be a girl eager to settle down to domestic life."

I have never had such ambitions. So much had changed. I had not been tamed. If I now wished to settle down, it was because I had been frightened by the wilder parts of life. I no longer wanted to venture out beyond the safety of a White Tower. I also felt the burden of an heir's responcibility lie heavily on my shoulders. I did not see that I had a choice but marriage. My dreams of being the head of a women's Tower would have to wait, as long as I could still serve Makala's family. I did not say any of this to the radiant young Baroness Cleyoren. Her head was still filled with songs of glorious conquests and valiant men bravely fighting for home and country, only to return safely to the arms of their lovers. I could not rob her of that innocence as I had been robbed. I simply smiled.

"Tell me what you have been doing with yourself these last four months." Chamila continued. "Everyone thought you had run away and gotten yourself killed."

I took in a deep breath and exhaled slowly. What could I say? I did not think I could tell her of Griswold. Nor did I wish to disturb her by telling her of my journey home. As I reviewed my past several months, I realized that so much had occurred that I could not share with anyone. A woman could not admit to adventuring on the high seas with an unrelated man in polite company. I could not tell the Tower of my one encounter with the old magic. Who was there who would not condemn me for walking away from Sasha and Anya's burning bodies, or understand the cruelty aboard a smuggler's ship? 

Chamila looked at me expectantly. I had to tell her something. "I sailed beyond the north sea. I had intended to perform a pilgrimage." I began uncertainly. Chamila gasped at my words. Her eyes shone with excitement. I continued, knowing how I would edit my story. "I never made it to the Holy Lands beyond Skalsbad. I never made it to Skalsbad for many complicated reasons. The things I saw on the way there Chamila, there are unimaginable creatures living in the oceans." I told her about the whales, the great frightening beasts that swam along our craft, with tails as wide as our sails. I told her of their songs at night, and the way they would arc above the water to breathe and play. I told her about the terrifying storms when the boat rocked like a child's craddle and waves rose higher than houses. I told her about the need for sailors to tie themselves to their posts while Griswold and I hid below decks, I praying, Griswold doing only the Maker knows what. I told her about the sea lions, and the awkward way they flopped and crawled over the rocky beaches to sun themselves in the pale northern light. I told her about their beautiful cubs, and the raucous clamour they made when I tried to approach to see them better. They were deadly hunters. Their sleek brown bodies shot through the water faster than an arrow through air. I have never seen a pride of lions hunt. I cannot cannot compare the sea lion's swift sleek deadly dive with a lion's deadly prowl and pounce. Sasha took me to see the bulls bring back food for the herd. I have never seen anything move so gracefully. I do not think I ever will.

Baron Rafel of Cleyoren arrived, an older man, easily twenty years older than Chamila, and the conversation turned to tales from our childhood, as girls in the Tower or as a guest in Cortan's castle. Chamila had been one of my first friends in the Tower. With Makala's indulgence inviting me frequently to the Duke's court, she had remained one of my oldest friends, even as life in the tower and court led us in different directions. I found it pleasant to reminisce. Conversation with the Baron of Cleyoren was easier than conversation at Cortan's court. Chamila was a skilled hostess. In her hands, I found myself gently lead away from topics and statements that may embarrass me or my host. I admired her skill. Makala had wished to train me to be more like her in this regard. Without his guidance, if I wished to be of any use to Cortan, I suposed I would have to learn this art on my own.

Memories of childhood, and the pleasant parts of my travels north erased much of the sense of impossibility of my task. When the Regent Consort granted me an audience in the hour before dinner, I felt a glimmer of hope that I would manage to convince either the crown of Duke Ergino to allow Griswold to return home.

\vspace{.5cm}

I met the Regent Consort in a small private chamber where men and pages scurried about him readying him for dinner with the gathered Dukes. I was disappointed. I did not know how I would plead Griswold's case to him if he did not see fit to give me a moment of his full attention. 

"You have arrived at a very inconvenient time, sister," my half-brother said after I had kissed his hand. "This is a critical moment in determining the fates of two great duchies. The existence of a living heir of Cortan changes the conditions of all the agreements we have worked so hard to arrange. The crown is not pleased."

"I apologize for displeasing the crown, your highness." What was I to do? Even Marsea's crown would rather I had been swallowed by the northern sea, or killed by smugglers. How could I prove my right to yet live when better men had died?

The Regent Consort left me considering my unworthy condition for some time before stating "The Duke, your father, has washed his hands of you. He has not yet declaired his intension to disinherit you. He will consider his actions depending on what match I can make for you over the next few days. I have put aside the entirity of tomorrow afternoon for this purpose. Make yourself ready and presentable for viewing."

I gulped down my panic. This was not how I had expected this audience or this visit to Deyalorn to proceed. Last year I had seen women, mostly daughters of poor households who suddenly manifested the gift, or widows and orphans with no one to speak for them, come to the Day of Unions to be viewed. Men who had considered marrying but had not settled upon a bride lined up to inspect them like livestock at market. It had been humiliating for me to simply see them lined up for viewing. I had thanked the Preserver for Makala's choice. As a gifted orphan, if he had not chosen me, I could very well have landed in a similar situation. Now it seemed I was to have a private viewing. I had thought, as Cortan's heir, that I would weild some control over who I married. That possibility, it would seem, had vanished. Any hope I had of continuing to study and practise after marriage would live solely by the whim of my new husband. It may as well die now.

As the Regent Consort dismissed me, I realized that I had not yet fulfilled my intention of my audience. If my appearing at this crucial hour hurt the crown's negotiations, the news I bore would damage them even more. I licked my lips and spoke over the shoulder of the man leading me away. "I beg your pardon, your highness. There is one matter more."

"What is it?" My half-brother looked infinitely impatient. 

I spoke quickly. "I have a letter for you, your highness, from my father's uncle, Duke Griswold. I have met him. He wishes pardon for his crimes and to return to Marsea."

"I do not have time for childish fancies and lies, sister." Regent Consort Dario's face darkened with anger. 

"I would not dare lie to the holy crown, your highness. I am only an emissary." My voice emerged small and trembling.

"Duke Griswold's crimes are unpardonable. Your errand is in vain."

"I beg that your highness accept the letter he has entrusted me, that I may fulfill my duty to my kinsman. Nothing more." My escort took the letter from me, handing it back towards the Regent Consort. I left, trembling, before I could see it reach his hands. 

What was I to do? I had entered the audience intending to plead Griswold's case to help Cortan. I left begging only that the letter be recieved, and giving myself up for auction for the sake of my duchy. Even if the task of returning Griswold to Cortan was not impossible, it needed a better midwife than myself to see it come to fruition. I was not skilled in matters of diplomacy. Griswold needed Makala's silver tongue for this task, not my wooden one.

\vspace{.5cm}

My escort took me to Deyalorn's White Tower, and showed me to a room facing the inner yard. I entered to find that I would share it with another healer. Then I suddenly found myself enveloped in Carlotta's weeping arms. I embraced her long and hard, beyond relieved to find myself in the company of a good friend for the first time since leaving Turina over five months ago. I found myself crying as well, tears of joy and relief and gratitude streaming down my face for this mercy.

"Where have you been, Nisrita?" she scolded when we had finally disentangled ourselves from each other's arms. "When I came back to Cortan to find that you had run away, I was furious at you. When I came back from the northern front to hear that you had died, I ... Do not ever do that again, Nisrita. Whatever the reason, you cannot just leave. We all thought that our fight against Firvona had been in vain, with the last heir lost. We all thought..." Carlotta started sobbing again. 

"I'm sorry," I said, wiping her eyes and mine. "I had not thought. With Makala gone, the Tower empty, and only mother for company, I must have gone mad with grief and loneliness. I was not thinking clearly."

Carlotta calmed herself and forced a smile on her face. She cupped my face in her two hands and whispered "Where did you go? I was so worried." Before I could think of how to answer, she dropped my face, stood and went to the door. "They just called dinner, but I will not share you tonight. I will ask for plates of cold food to be brought here, will that do?" I nodded, still overwhelmed by my change in luck at finding myself in the company of my dear friend. 

When she returned, she handed me one of her robes. "Change out of your court dress, Nisrita. Make yourself comfortable, then tell me what happened. I cannot tell you how glad I am to see you again. I was so furious at the Duke for giving up on you so quickly. We all were. Headmaster Corino argued with the Duke personally for weeks, trying to impress upon him your value to Cortan and its tower. His pleas fell on deaf ears."

I did as I was told, relieved to find myself in comfortable surroundings, in the company of a friend I loved and respected. I returned dressed as I would be for classes at the Tower: in a tunic and simple cotton robe, tied at the waist by a belt, my hair tied back simply to keep it from interfering with my work. I wore no decorations, no jewelry but my talisman, no pretense of being anything other than a healer and a student. I was not dressed to inspire or impress. My task was to learn and cure. Many find the simplicity of the Tower's garb to be harsh and unflattering. I found it to be the only arraingment I was truly comfortable in. I was a White Healer. I stood within a Tower. This was where I belonged. I had been away too long.

"Your hair," Carlotta said, picking up a strand of my woefully short locks. 

"It will grow back," I replied, smiling. I curled myself up on her bed while she sat upright beside me. I felt like I was back home. I might as easily have  just snuck into the senior girls' room after dinner to gossip with Carlotta when I should have been in my own studying. 

"Now tell me," said Carlotta, running her palms along my arms. "Where have you been?" She could not take her hands or eyes off of me, as if she feared I might disappear again if she let go. I did not mind the attention. I had been so lonely recently.

I took a deep breath and started my tale. If I could not trust Carlotta, who could I trust? "I went to find Griswold."

"Nisrita," Carlotta sighed. She sounded disappointed. "Of all the childish legends you could have chased."

"I met him," I interupted. Carlotta looked at me in surprise. "I have been travelling with him these last four months. I tried to convince him to return in this hour of Cortan's need, as in the prophecy, but he has refused. He will not return until he has been pardoned for his past crimes."

"How do you know it is Griswold?"

I explained my journeys with the Hundred Horsemen, the critical questoning of my skill, the handwriting, the poisoning of Cortan's water. Then I ventured one further piece of evidence. "He has shown me how to open the gate to the old magic."

I saw Carlotta's eyes grow large. I thought for a moment she would scold me for this forbidden act. "What was it like?" she whispered.

"Frightening and horrible. I will never do it again. I do not know why people try it."

Carlotta laughed at my vehemence. "Dear Nisrita. Did you hurt yourself?" I shook my head and curled myself tightly around her waist. The nightmares I had lived through that day still frightened me. It was not a pleasant memory. Carlotta returned my embrace and stroked my forehead silently. I had missed her. She was a good friend, what I imagine a sister would be like. Having her near me was a solid reminder that I had more in this life than Cortan's malice, and Griswold's cool distant desire to study me. Once, I would have been able to tell her everything. 

A platter of cold food arrived, breaking the intimate cocoon. "What is Griswold like?" Carlotta asked while eating.

That was a complicated question. I knew many things now about Griswold that I did not want to know. Where would I start? I decided it was easiest to start with the unsurprising. "Stories of his brilliance do not do him justice." I told Carlotta of my months in his tutelage. His understanding of the gift was subtle and deep. Even unable to wield the power, he would tell me how to do things by metaphor and story. It took me a long time to understand what he meant by statements like \emph{branch a stream of fire under the wound} but once I did, his explanations were clearer than that of any tutor I had studied with. I think his language came from his constant use of the mushrooms. He spoke as if the gift were a real tangible stream flowing through my body, as it had seemed that nightmarish morning. Nightmares aside, it was a completely new language to communicate in.

Carlotta was surprised when I told her that I had spent so much of my time practising on fish pulled out of the ocean. "Fish," she exclaimed. "What good is that? Their anatomy is so different from ours. What could you possibly learn from them?"

Why not fish? That had been Griswold's argument when I had uttered similar words. Just because something is not directly useful, does not mean it should not be tried. I admired him for that type of thinking, while I feared and loathed his callousness. During the first weeks of our travel, when the full horror of the man had not been revealed to me, it was partly that type of  iconoclastic thought that had made his advances so irresistable. We were traveling as man and wife he had argued, as if that would have persuaded me. What did that matter, compared to the fact that the great healer Griswold saw fit to give me his attentions. He valued me as a healer. He taught me to weild my power in new unimaginable ways. I did not have to fight him in my desire not to become pregnant. He wanted to teach and study me. He could not do that if I carried his seed. I had thought, before I understood the rest of him, that I could want nothing else in a lover. "We ran out of rats in the ship," I said. Carlotta laughed long and hard. She had such a beautiful feminine musical laugh. I had missed her laughter, I decided. 

I told her about the blooms of jellyfish we encountered as we approached the southern coast of Szarvis. "I awoke one morning to find that the sea around us had turned yellow and green with hundreds of transluscent aquatic flowers, their thin tentacles flowing in the waters behind them, like petals in the breeze. I had thought I was dreaming when I first saw them. I had wanted to lower a boat amongst them to trail my fingers in the beautiful cold water they occupied. One of the sailors corrected my fantasy by explaining to me how painful their sting could be. 

When we disembarked at the port town of Groscht, Griswold took me to the house of Shasha and Anya, a fisherman and smuggler who lived on the sea with his young wife a little ways away from the town. There Griswold and I had our first major argument. Griswold is a brilliant man, but he is not a patient one. He is demanding and harsh with his criticism and sparse with his praise. He will never admit that he is wrong when it comes to matters of the gift, even when the evidence is as clear as the nose on his face. It was hard working with him. I learned quickly not to question his will, but I would not call the friction we had as master and student arguments. Our first argument happened at Sasha and Anya's house. He had a few days before he needed to leave for Skalsbad, on business he would not share with me, that I still do not know the details of. He wanted me to experiment with the old magic again. I refused, of course. I had tried once while in Cortan, it had been his price for letting me accompany him. I had promised myself I would never touch the mushrooms again. I have no desire to be burdened with those maddening nightmares. Griswold suddenly grew furious. He called me all sorts of names, told me that I would never be a successful healer if I did not show courage and curiosity. He threatened to stop teaching me. For a moment, I thought the argument would come to blows. It did not. He finally stormed out of our hosts' house, and left for Skalsbad before I awoke the next dawn. 

I spent three weeks with Sasha and Anya. They spoke some Mer, though we mostly made do with jestures and laughter. They were a lovely young couple. Sasha would leave for a day or two or longer at a time to fish, or conduct his business. I never asked. He took me with him once or twice to show me the silvery schools of fish he chased with his wide white nets. It was beautiful and exhausting work. They fell like piles of silver coins onto his boat, and flopped and gasped for air where they lay. Anya showed me how to clean the fish and prepare them for market. I told Sasha of my fascination with the jelly fish. I think he and his wife found the interest amusing. A week into my visit, Sasha woke me at night and told me to join him in his boat. He took me to where a bloom floated, an irridescent patch of silvery white in the black shimmering waters. He lowered his nets carefully, and we brought back dozens of the beautiful gelatenous creatures. For the next two weeks, I spent my days sitting on the beach in front of their house carefully cutting their tentacles and extracting their venom. It was painstaking and slow work, but worth it. I can cause excruciating, almost paralysing pain now, and take it away with a simple application of strong vinegar."

Carlotta's lips had curled into a wide smile and her eyes danced with amusment at my words. It was not unusual for healers to experiment with poison, and the Black Riders line their throwing stars with the venoms of instantaneously lethal snakes. "I can just see you kneeling in the salt encrusted sand, knife in hand, dissecting the petals of these flowery creatures. You must have had fun."

"I did" I said, smiling at the memory of those beautiful sunny days, and the sound of the ocean by my feet. I decided to end my story there. To speak further would come dangerously close to telling the story of the fire and our flight from Szarvis. There were things, I found, I could not tell even Carlotta. It was with the smugglers that picked us up from Groscht that I had decided to change my choice of poisons. I spent a day carefully washing each of my poisoned weapons I I had picked up with the Black Riders, then lining them with jellyfish venom. I had had enough of death. I swore that I would give up life as a fighter as soon as I returned to civilization. I was a not a tool of the Destroyer. I was a healer. I worked for the Preserver. Nothing would change that ever again.

"Tell me of Cortan," I said, changing the subject. Carlotta obliged. I learned that Ezaro had suddenly entered his prime, doubling his endurance almost over night. He was no longer under Master Alerio's tutelage, but working with Master Cheppe, an older man who would never rise above his position as the boy's Head in the tower. Ezaro had been too openly curious about the wild magic for Master Alerio to be willing to take him on. Sophia Lazarro had reverted to her maiden name Joris in her widowhood. She was here at Deyalorn, but staying with her family, not at the Tower. She was here to grieve, not to remarry. She would return to Cortan Tower with the rest, after the Day of Unions. Master Adele was in Deyalorn, as I had heard. Carlotta said she had testified bravely against Firvona. Carlotta had watched Firvona's representative cross question Master Adele about what she had seen and done that day. The questioned had seemed almost as harst to Carlotta as General Madriano's inquisitors, working without the aid of pain. I shuddered at the thought, and was grateful to Master Adele for showing such courage. Carlotta had met Ernesto, she said. She described him as the uncannily expressive mute soldier. He had left for Turina already. She had not heard from him since. I gathered news about other classmates and friends before asking her "Why are you here?"

Carlotta giggled. "I turn eighteen this winter. I would like to get married."

"Why, Carlotta," I whined. If anyone could be said to be gentle with the gift, it was Carlotta. She had tried to teach me, and so many other people, how she avoided causing excrutiating pain while she healed, but she could not. It was such a shame for Cortan's Tower to lose her gift. 

Carlotta feigned a grave face. "I am a priest's daughter, not a duchess. I do not have as many options as you do. I must marry while I am still desireable if I ever wish to leave the confines of the Tower. My brother is at the Temple, to arrange my match."

I rolled my eyes at her modesty. Carlotta was a priests daughter. Her father served Baron Farone directly. Her mother was gifted, as was her paternal grandfather. She came from a good family. She was a beautiful woman. She had little to worry about with regards to her desireability. The truth was, she wanted children, domesticity and all the burdens that came with a family. I did not understand this side of her. 

"You have forgotten someone, Nisrita," Carlotta said when I did not respond to her excuses.

"Who?"

She gave me a wicked smile. "Commander Romino is here. The duke has decorated him for his insight that saved Cortan. The Crown has done the same. I've heard talk that he would have been made a general if he were not so young." 

She raised her eyebrows meaningfully. I ignored the implication. "That is wonderful news," I said simply. I was glad that Timmon's life had been rewarded since we had parted company, where I had only met hardship. He was a good man, deserving of Makala's love. 

"Oh, stop that," Carlotta said, poking my side hard with her forefinger. 

"What?"

"The Day of Unions is in ten days. Marry him and be done with it. He has been a drunken mess since you left Turina."

"What!" I exclaimed, sitting up in Carlotta's bed. Poor Timmon. He had lost so much when he lost Makala. I had hoped that he would suffer less.

It was Carlotta's turn to roll her eyes. "You both care deeply for each other. Marry and be done with it. The Duke is about to disinherit you anyhow. Being happy as a generals' wife is not so much worse than being miserable a duchess, is it?"

"It is not so simple, Carlotta." I said somberly. This was not a game of love and romance. We had lost too much. There were secrets in my friendship with Timmon that were not mine to tell. How could I make her understand? "Timmon will not have me. Don't ask me why. He simply will not." She did not seem to believe me. I plotted a different course, and brought up the unpleasant news I had just learned. "Besides, I do not have a choice in the matter. The Regent Consort wants me to be viewed tomorrow."

"Oh, Nisrita, why?"

"Something about terms for the settlement between the duchies of Firvona and Cortan. I do not know the details. I am to be married off to produce heirs for Cortan." I paused to fully take in what my situation would be tomorrow. "My life in the Tower is finished." I looked down at my lap and twisted and pulled at the fabric of my robes. Carlotta said nothing.

After a while, she put her hands on my own. "Don't marry then," she whispered.

"What?" I could not go against the Regent-Consort's will.

Carlotta explained. "Any woman can apply to the Maker, or to the Tower, if she is gifted, for a life of service. You will lose all status outside Cortan's Tower, but when have you ever wanted anything beyond the Tower's walls? Master Adele has lived a life of service. She is happy with her life. With your gift, you could be Headmaster of the Women's Tower in Cortan. You used to dream of that."

"But Cortan," I protested. "I am her heir. I cannot just desert her."

"Poor Nisrita. You loved Duke Makala. You are a good widow. You love his family because of him. But they do not want you. Why fight it? Entering service is a hardship for most women, but it will not be for you. You were practically born in the Tower. It is your life."

She presented a valid argument. I could enter into service. It was not how I had imagined my life going, however. I had been born and raised by the Tower. Even if I did not dream of children, I dreamt of a life beyond its white stones. Beyond Makala's invitations to Cortan's castle for festivities, I never had a home to go to when school was not in session. I wanted more than just the Tower. "I will think about it," I said. "I do not think I will have a decision by tomorrow afternoon, but there may still be time to decide afterwards."

We continued talking well into the night. She told me of the taking of the stronghold of Turina, and what she had seen of the war with Firvona. Eventually, as I had started to doze, she kicked me out of her bed and into my own.

\vspace{.5cm}

I dreamt that night. I had not dreamt since the fire. Sleep had come in short snatches in the smuggler's boat. I had been either too exhausted or too busy with the fearful quality of my waking life to have nightmares at night. The soft bed, the warm fire, the security of a friend's company made it safe for the dreams to return. I wish they had not. 

It was my wedding night, the night of last year's Day of Unions. The maids had led me to my chamber to ready myself for my new husband. I was trembling and afraid but excited and happy as well, as I had been in life. As the maid's left me, I drew back the curtains of the bed to find Makala, face down, already lying there. I could hear his deep even breath, and saw the sheets rise and fall gently over his bare shoulders and back. He was sleeping, the oaf. I shook his shoulder gently. "Makala," I laughed. "Wake up. You aren't supposed to be here yet." He rolled over onto his side, and I saw his burnt face and back. "Verst." He said, as Sasha had. "Verst. Help." I backed away from him, my chest pounding at my throat. "Help me, Nisrita. Help me and my family." I felt someone strangling me. I could not breath. I started awake to find myself burried under heavy blankets in the cold autumn moring air of Deyalorn's tower. 

I forced the dream away. I did not have time for it now. I had returned to Marsea, and resumed my life as a healer. Everything I did these past four months were behind me, I insisted. There was no point in dwelling on them. I rose and readied myself for the ritual of pre-dawn prayers at the Tower. \begin{comment} How many days of my life could I force myself not to think of, I wondered as I wound down the stairs to the entrance hall where people would gather. If I refused to remember the fire, could I make that memory go away? I had not succeeded in chasing memores of that other day from my dreams, even if I refused to think about it while awake.  \end{comment}

Half an hour before the the dawn horn, the congregated healers walked into the Tower's temple. We took our sticks of burning incense and said the words that reminded us of the task given us by the Preserver. We lined up before the triumverate to annoint the Preserver for our daily strength and to give what other personal prayers we had need of. It was a comforting familiar ritual, it served as a balm against the strange disturbing experiences of my travels. It was yet another reminder that I was finally home.

Breakfast was a joyful affair. Master Adele and the handful of others who had come to Deyalorn from Cortan's Tower had heard of my arrival the previous night and were only too glad to welcome me back to their ranks. They peppered me with questions about my travels and told me stories of their own. They  listened with wrapt attention as I described the jellyfish's anatomy, and laughed at my misbegotten attempt at helping a sea lion give birth. I considered Carlotta's advice. I had thought that Makala had made Cortan my family. It had only been he who had loved me, no one else. My place in the Tower was much less intimate than my place with Makala had been. I was one piece among many that made the great institution function. However, I was a valued piece. I had a defined role there, I was wanted. No one was inconvinienced by the fact that I still lived. I had friends here and colleagues, As I grew older, I would have students. Could I live with just that?

\vspace{.5cm}

My father granted me an audience after breakfast. I met him in his chamber. Duchess Cybeline, I was relieved to see, was not there. "I am glad to see you safely with us, daughter," he said coldly after I had kissed his hand. I doubted his statement very much. This man had been all to eager to declair me dead. "What is it you wished to discuss?"

I put the letter down on the table between us. "I have a letter from your uncle, Duke Griswold, father. I have been traveling with him these last four months. He wished me to give this to you."

Duke Ergino looked down at the letter, then up at me. Very slowly, he put his hand on the table and drew the paper towards him. He cracked open the seal and read the first few lines. Then he looked up at me again. "Is that all you wished to say, daughter?"

Was it my imagination, or was his voice hoarse at that moment? I had not planned to speak the next words. It surprised me that I spoke them at all. "There is one more thing, father. If there had been any way in which I could have saved either of the dukes, I would have. I was with Duke Lukos when he fell. I could not reach my husband in time. However long my list of shortcomings may be, I am loyal to your house."

Duke Ergino nodded curtly, then dismissed me. He seemed to have heard my words, even if they were not ones he could easily accept. The second of my planned audiences had been shorter than my first. Yet it seemed it may have been more successful. However, I did not know that it was enough to bring about the miracle I had set out to perform.

I did not have much time to contemplate the success of my mission. When I returned to my room, I found it filled with maids and seamstresses. They immediately descended upon me with their fabrics, their pins, their combs, perfumes, tweezers and powders. The work continued for hours. I have never been an attractive woman. My face has always been plain, my shoulders too broad, my torso too square. Learning to ride and climb an shoot only exaggerated my stocky build. My only possibly redeeming feature, my thick braid of hair, I had cut off and given to Timmon that day near Turina. I did not think that any ammount of pruning and powdering could possibly transform me into an attractive woman. This horde of groomers were determined to try. I tried on dresses, got poked with pins, took off dresses, had my hair tied a dozen different ways, all to a chorus of dismay from the women surrounding me. "You are too thin." "No, she is built like a boy. A woman needs a layer of softness over her muscles. We will have to hide her arms." "What do we do with with your masculine hair?" "Look at your eyebrows, you barely have two." "How can you look yourself in the mirror with this hair on your lip." "Oh, these pimples. If his highness had told me you had pimples, I would have tended to them last night. It is too late now." I sat as still as I could and let them have their way with me. I ginored their insults. They did not tell me much that I did not already know. I had been traveling hard, and sleeping in damp corners for weeks. I had not had time to tend to my nails. When they finished, they offered me a mirror to see if I was pleased with what their handiwork. I refused. If they were pleased, that was enough. At best, I was a poor judge of such things. Currently, I was too nervous to care.

I entered the royal audience chamber in a rustle of tight fitting silks and a pair of ill fitting shoes. Two well armed guards opened two heavy wooden doors for me. If they were meant to intimidate, they did. Far from being armed, I could barely walk, let alone run, in my current state. War axes gleamed menacingly over their shoulders. They could behead me before I could duck. When had I last been so helpless? 

The room beyond was, if anything, more frightening. The doors opened to fifty feet of empty vaulted columned hall. This was was the formal audience chamber. My half-brother sat on the dias on the opposite end. Above us was a high vaulted cieling. Below me, a smooth stone floor. The hall echoed and magnified my every foot step. The afternoon sun shone through a series of windows, highlighting every flaw in my face or my carriage as I walked to meet the Regent Consort. The hall was empty but for a couple dozen men standing below my half brother. All eyes turned to me as I entered the room. All eyes watched me slowly make my tortured way to the front.

Upon the raised dias, my half-brother introduced me to my potential husbands. He turned to each man in turn, asking him a series of questions regarding their intension and vision towards Cortan, their loyalty to Marsea, their ties to Firvona, and various questions about their family members. Each man was asked a different set of questions. Each man answered the Regent Consort privately, beyond the easy hearing of his competitors. Then each man had his turn to inspect me. Some simply asked me some questions about my gift. One asked if I had reason to believe I was fertile. I few went so far as to examine my face, teeth and nails, as if I were a horse. One asked me if I were likely to run off with a strange man for a third time. My group of potential husbands laughed. I did not know where to hide. My half-brother, mercifully, cut their merriment short. 

The audience lasted a better part of the afternoon. I returned to my chamber with an aching back and feet, and an ego bruised worse than it had ever been by Griswold's constant criticisms. My interviews made the prospect of a life of service seem a kind alternative. I would have no family in the Tower. What good was a family if meant suffering this constant humilation? 

\vspace{.5cm}

I made haste transforming myself from a woman of the royal court into a simple healer. Carlotta found me at sunset fretting with a mouse I had caught lurking in the corner of my room.

She walked over to the table and rescued the poor beast from my hands. "It did not humiliate you today, Nisrita. Leave the creature be." Then, in her wisdom and kindness, she told me of her day, and did not question me further about mine. She did not even comment on my appearance.

When dinner was called, I begged off. My grooming and inspection had left me with little apetite. I wanted to consider my options. I walked the grounds around the tower lost in thought. I had no idea if Griswold would come. At this point, I did not know what his appearance would change. I had never truly had a clear plan for how Griswold and marriage and saving Cortan would fit together once I arrived in Deyalorn. At most I had imagined a vague world in which I was still welcome as Cortan's heir, even after I had known that to be false. I would have some negotiating power with Duke Ergino for my husband, and would chose one who would let me study, and would be good for Cortan. Griswold would be an advisor and guide for the duchy. 

Had I considered the possibility of marrying Griswold himself? It must have lain hidden and unspoken in the back of my mind. The other reality I had concocted for myself was so improbable, that making Griswold my husband must have been, in actuality, my solution to Cortan's woes. Griswold was no where to be seen. I asked myself if I truly wished to marry him, now that I had spent so much time living intimately with him. I did not know the answer. He was a cruel and violent man. As a legendary intellect, I still revered him, but the scholar and the man were very different. In spite of the deceiptful family resemblance, he was not Makala. It made no difference what I thought of him as a man. He was not an option. I turned my mind to the reality that lay before me.

I did not like any of the men that had seen me, though I did not have a chance to know them. I would get my list of offers tomorrow. I would technically have until the Day of Unions to decide, but given the political situation, I would in reality have a day, perhaps two. That would gain me little knowledge of my spouse. Griswold, if he appeared, at least, was a known evil. He wanted me to study. That went a long way against his other sins. My other option was to reject all my half-brother's choices out of hand. Where would that leave me? In service to the Tower? I would have no man to fear, and I would have the freedom to study. The idea sounded more and more appealing. 

I walked around the grounds one more time, contemplating my options until I encountered the smell of roasting meat wafting across the grounds. I was hungry after all, I realized. I found myself near the officer's hall. I decided to pay Timmon a visit. The doormen let me in when I gave Commander Romino's name. A few faces turned up to look at me when I entered, then returned politely to their own business. I was mostly likely an officer's wife or love interest. It was no business of theirs as long as I conducted myself appropriately. The hall was full of men mostly eating, talking or singing to the tunes of their collective drunken merriment. Here or there I could hear a heated argument brewing. It was the most civilized company of men I had been in since Turina. Military men are violent, but they are a civilized orderly type of men as well.

I scanned the hall for Timmon. My heart sank when I found him. He was at the center of the hottest of the brewing arguments, growing hotter by the moment. I hurried over, wondering what I would do if weapons were drawn. 

Timmon and a captain were arguing over a percieved insult against Timmon's father's, the Baron of Romino's, character. Timmon would not accept the captain's explanations without the man's abject apology, which the man was too proud to give. I saw the captain's hands clench to fists and move dangerously close to his dagger. He was not drawing yet, but he certainly considered it. Timmon, it seemed, had drunk more than a fish. He would not stand a chance against his more sober opponent. "Timmon," I called. "I apologize for not finding you sooner." Both men turned to me. Timmon went as pale as a ghost.

"I beg your pardon, captain, for interrupting your discussion. I am Duchess Nisrita of Cortan. I have not seen Commander Romino in over five months. May I please intrude?"

The captain bowed, grumbled, then shuffled off. He could do little against my rank. I breathed a sigh of relief and occupied his empty seat. "How are you Timmon?"

"You are dead," he slurred, and returned to a position hunched over his cup.

He broke my heart, that man before me. Timmon had always had a dark streak to him, but I had never seen him slouch. He took his work and his rank seriously. He would never be a bad example to those who served under him. Cortan's army was his life. He loved nothing more, outside Makala. Today I watched Timmon's broad shoulder's stoop. His hair, normally bound so tightly when he was on duty that one could not tell that it fell in ringlets over his shoulders, was losely bound, and tangled, with whisps clinging to his damp cheek and forehead. He had at least a day's worth of stubble on his face. Dark circles hung under his brooding eyes. I could not imagine this face ever grinning. I could not imagine this man ever inspiring respect out of a group of teenage recruits. What had happened to him?

"I beg to differ, Timmon. Accounts of my death were unsubstantiated.I am not a ghost." I forced a smile on my face in spite of his surly looks. When he did not respond, I pressed on, trying to keep my voice light. "I heard of your decoration. Congradulations." Timmon turned away from me. "Let me drink to your acheivement," I ventured. 

I raised my hand to call for a cup when Timmon said "I do not drink with the dead."

I lowered my hand slowly. What could I say to him? "Look, Timmon. I am sorry I did not come to you earlier. Please forgive me. I am happy to see you."

He turned on me with a startling ferocity. "What does it take for you to understand that you are not wanted here?" he roared.

I blinked to find tears in my eyes. "Nothing more, Commander Romino. I will leave you now."

I returned to my room and collapsed weeping on my bed. I could not believe what I had just seen. What had this man Makala had loved turned into? He had been my friend and trainer as well. I could not bear to see him so broken. He frightened me. He seemed to be a personification of all the grief and horror that I had felt over the past few months. Where I had tried to keep it tightly controlled inside me, he wore it on his person, raw and drunk and visible. He reminded me of everything I did not wish to think about. The sobs came without stopping, recklessly released by this vision of grief and anger. I could not stop them. I had known despair as well this summer. The world had ended for Timmon, it seemed, when Makala died. It had ended for me as well, though I had run away to keep from knowing it. I was only realizing the depths of my trouble now. Grief, uncertainty and humiliation poured from my chest onto my pillow, exhausting and draining me. I cried myself empty. I was asleep before Carlotta returned to our room.

\vspace{.5cm}

Anya visited my dreams that night. She indicated that she was pregnant. We embraced and celebrated the news, giggling with anticipation. She asked me to help with the fish that day. The smell was too strong for her. I left off my work with the jelly fish and obliged. Griswold came home that evening. It was the first time I had seen him in three weeks. He looked harried and tired, but was unwounded. He told me that we would leave early in the morning, before dawn. He had arranged transport back to Marsea. I knew better than to ask what the problem was. I made sure our belongings were packed. When he asked me to keep myself ready for trouble, I donned my weapons. Sasha was away on a job that day. We expected him late at night. Normally, Anya would wait up for him, but she was tired. I sent her to bed, promising to keep the vigil. Griswold was asleep in his own time too tired to keep watch for his host. I heard Sasha whistle after midnight. I went out to help him dock his boat. He was wounded. I had promised Griswold that I would not announce my gift. I had kept that promise easily, there had been no need to heal anyone. We were leaving in a few hours. I did not see the harm in helping our host this once. I sat him down on the beach and warned him of the pain. It was a simple gash in his shoulder, the type that heals in three days on the battle field, a week during peace time. I washed the wound and healed it enough to keep it for getting infected. I did not have time for more that night. He thrashed and jerked in pain, but did not cry out. When I finished, Sasha looked at me with fear and distrust written on his face. Then he put his fingers to the wound and looked me in amazement. We did not have the words to explain what had happened. I put my finger to his lips. He nodded his promise of secrecy. I must have been gone from their cottage for less than an hour. The arsons, kings men, other smugglers, or enemies of Griswold, I do not know, had come and gone by the time we climbed the dunes to see the cottage in flames. Sasha ran like a madman to save his wife. I ran to see to Griswold. He was awake and coughing inside the smoke filled cottage. I led him out to safetly. Anya was unconcious in her corner. Sasha could not lift her with his wounded arm. I could hear Sasha calling for help inside the smoking cottage. I checked that Griswold's wounds could wait, then turned to help Sasha with his wife. Griswold is a larger and stronger man than I. He pinned me down, putting the full force of his body on top of me. When I screamed for him to stop, I felt his hands press down on my neck, strangling me. 

I awoke with a start, clawing at my neck, gasping for breath. The dream had been too real. It had been too accurate a memory. My sheets were soaked with my sweat. I saw the smouldering fire in the hearth, and threw the bucket of sand over it to douse it. Not knowing what I was doing, I donned my armour and weapons over my robes and left into the pre dawn air. I found where Deyalorn's black riders trained, found the straps of leather they tied between posts to practise their balance and made a many tired leather structure for myself. I relaxed as I climbed and paced the thin strips of leather. It reminded me of working the rigging on the various smuggling vessels we returned on. High above the deck and the cruel men on it, holding on to the ropes of the swaying boats was the only peace I found during those long hard weeks. It was there, swaying to the rythm of the sea,  that I decided I did not wish to be part of the Destroyer's dance any more. If I had to dance, it would be differently. 

The wind blew across my face, and I sat atop a pole and looked down the ten feet to the ground. I could smell the Tulsi River from here. It was different than the sea, but somehow familiar. I inhaled the watery smell deeply and watched the sky lighten in the east. I heard the pre dawn prayers from inside the Tower. I said the words I knew my colleagues to be saying inside. Then I stood and stretched and paced the top strap once more. I would have to end this solitary vigil soon, I knew. Carlotta would wonder where I was. I reluctantly paced one last time, then dropped down to the next level and started taking off the top strap. I heard footsteps below me and froze. 

"Rico, I found your ghost," I heard a man call. I looked down to see a man wearing Deyalorn's livery looking up at me. 

Rico walked up to him and said, "That's not a ghost. That's a cat. A cat in healer's clothing." The first man laughed. "Come down healer." said Rico. "You will hurt yourself up there."

"I am fine, soldier." I said taughtly. I could not bring myself to trust these men.

"Eight feet up is no place for a woman" Rico said. He started climbing the leather straps towards me. My stomach tightened. It was happening again, I thought. I was trapped, soon to be cornered. I could not let this scenario play out again. I am no longer a fighter, I promised myself, and saw Makala's dagger sprout from Rico's thigh. He started cursing me, and jumped down to the ground. He companion called for help. My mind moved as if through tar. I could not make a decision or analyse my situation. With an enormous force of will, I had just decided that I could climb down and flee in the opposite side of the makeshift fence from the wounded Rico when three men answered the calls for aid. 

I was surrounded again. I panicked. I could not make out the faces of my attackers in the half light. I heard a familiar voice shout. "She can hurt you at five feet, but only one at a time." I cursed my overly knowledgeable opponents and threw a poisoned star at the speaker without looking. I howl of pain told me I had not missed. I climbed up to the top of my post, wishing there was more up that I could go. 

"Has the Destroyer taken your soul, Nisrita? What are you doing?"

Timmon. What was he doing here? I looked down to see him doubled over in pain clutching his right shoulder. "Call them off Timmon." I cried. "Just call them off, please." I sat on my post shaking. What had I just done?

I climbed down unsteadily when the three unwounded men had given me a wide berth and walked over to Timmon. I had grazed his shoulder. It was bleeding, but not a deep cut. The poison had seeped in all the same. This is the secret to the Black Rider's style of fighting. If the weapon had been lined with their standard venom, Timmon would be unconcious and dying by now. I reached for the small bladder of distilled vinegar I had taken to packing with my weapons. "This will hurt, Timmon."

"More than it does already?" he said through gritted teeth. I could still smell the alcohol on him. His eyes were red and bloodshot. He was still drunk.

"Briefly, then it will stop." I applied the antidote, then went to attend to Rico.

The soldier shrank from me. I took off my weapon coated vest and approached again. This time, he let me work on him. The wound was deep. I apologized. I did not understand what had come over me. I was a healer, not a fighter, I reminded myself. What had I done? When the vinegar had worked its cure, Timmon approached and introduced the Duchess Nisrita of Cortan to guardsman Rico of the town of Gelvand. The guardsman went pale until Commander Romino explained that this had all been a misunderstanding. We both agreed heartily in relief. I was on such tentative grounds already. I did not know what would happen to me word of this skirmish got out. The soldier limped off, and I gathered my tools of destruction. 

Timmon looked at me warily. "You used a poisoned star against me?"

"Not a lethal one, Timmon." I said undoing the straps strung between the poles. I could not bring myself to face him. "I do not use lethal poisons any more. I am a healer not a fighter."

"You could have fooled me," he grumbled. He continued watching me. 

I finished wrapping and putting away the leather straps. He was still there when I returned. Why wouldn't he leave? "Let me tend to your shoulder, Timmon," I said. I had to say something.

"It's just a graze. It will heal on its own."

"It will scar, Timmon," I insisted. He gave me an odd look, then sat down so I could see his wound. Sight is not necessary for healing, but the ungifted seem to think it is. I did not correct him.

"What happened this morning, Nisrita?" Timmon asked when I had finished.

"Nothing." I did not wish to discuss my disturbing and unforgivable behaviour. I changed the subject. "You did not sleep last night, did you?"

He refused to let me evade the issue. "No. Are you certain it was nothing?"

"Yes Timmon. It will not happen again."

He looked at me queerly again for a long moment. I shifted under his gaze. Finally he said "I was rude to you last night. Let me get you breakfast to make up for it."

"And I have wounded you with a poisoned weapon."  I replied flatly. He had been more than rude to me last night. "I think we are even."

"It was not a lethal poison." Timmon grinned. "Let me get you breakfast anyway." 

How many times over the last lonely five months had I longed for that grin? How many times had I yearned for it without knowing that it was what I needed. The world shifted as he smiled. I was back in Cortan. He beat me at chess while Makala watched. We three sat drinking and telling stories in Makala's box for the Duke's birthday games. Timon sat beside me at the healing corps telling my colleagues stories of battles he had fought in. It was a smile that chased away all the dark shadows from the world. "Let me stow my vest" I said. I could follow that smile anywhere.

We stopped by a small shop selling hot doughy dumplings covered in a mysterious salty gravy. We ate holding the warm bowls in our hands, standing by the fire in the cool morning air, licking the runny brown sauce off our fingers. Timmon watched me devour my food, then ordered me another bowlful. When I protested, he asked me when I had last had a decent hot meal. I stopped protesting. 

After two bowls, he paid the vendor and led me down the street to another selling fist sized pastries with a hot sweet soggy fruit in the middle. "Baked apples," he said when I had bitten into it and burned my tongue. "A good apple is something that can only be found in this duchy at this time of year. I miss them in Cortan. Did Makala show you around the city last year?" I shook my head, my mouth full of hot sticky fruit. "Then may I have the honor of doing so?"

"That would be lovely," I said. "I had forgotten that you grew up near here." Timmon grinned. It was infectious. I grinned back.

"The best apples," he continued, after I licked my fingers clean, "grow in my father's orchards, a half day's ride from here. I will be visiting him tomorrow. You would be welcome at his table, if you have nothing to keep you here."

"I cannot think of a better use of my time," I said. I truly could not. The idea of escaping the confines of this wretched court and the thoughts of marriage was more delicious that the apple I had just consumed.

He fed me for three hours. He gave me the gastronomical tour of Deyalorn, taking me from fruit vendor to baker to sweetshop, explaining the delicacies of the season. I protested against his generosity at one point. He answered that returning the meat to my bones was a better use of his money than losing it to his colleagues by gambling. I could not argue.

He passed the time by explaining Deyalorn to me. I did not ask him about how he had kept himself. I had seen enough of that last night to know the answer. He did not ask me about my travels. I suppose he had learned enough of my state of mind from the morning's display not to need further details. 

He returned me at mid-morning to the palace full to the point of bursting, and with a working map of Deylorn forming in my head. Having run out of landmarks to point out, Timmon fell into an easy silence as  we walked the mile to the Tower and barracks complex. 

"I hear you are to marry," he said finally, as the curtain wall loomed before us.

I should not have been surprised that news of my doings had spread to all interested parties. I licked my lips nervously. "I have not yet decided."

"I see," he said, and followed me through the gate.

The idea came to me then, in a sudden flash of brilliance, or desperation. I turned to him and stopped walking. "Timmon?" He stopped his gait to wait for me. "Duke Ergino wishes to disinherit me. I have half a mind to let him. That leaves me with the option of entering the Tower's service." I hesitated. I did not know how to proceed delicately. "If instead, we were to marry, nothing would need to change. If you would permit me to continue my studies, I mean."

Timmon's face darkened. "Makala's dream."

His answer surprised me. "Was it not your dream as well?"

Timmon sighed and looked to the horizon. I waited nervously for him to compose his thoughts. At long last he looked at me again and said coldly "You would gain a name and the ability to study. What would I get out of this proposal?"

I did not understand his business like manner. I tried to explain "I think I have shown myself capable of not being jealous. You would be free to have your lovers as you..."

"No." He cut me off flatly. "You may have forgotten Makala in these five short months. But I have not."

I opened and closed my mouth idiotically for a moment then stormed off to the White Tower. I had done many horrible unforgivable things in the last five months. Forgetting Makala was not one of them.

\vspace{.5cm}

I found my rooms empty but for a parcel and a letter. I sat on my bed until I had calmed my rage, then turned to the parcel. It contained two of Chamila's dresses, with the note "You will not marry in your ceremonial robes." I shook my head at Chamila's kindness and put the dresses away. There was a good possibility that I would not be marrying at all. The letter, I was certain, would adress that. It bore the royal seal. It informed me that the Crown would be happy for me to select one of three listed candidates for marriage. If I chose none of them, the Duke of Cortan may be content to chose  a man to take my hand during the three days preceeding the Day of Unions. The message confused me. Did this mean that I was not to be chased from Cortan's house? The Duke had said nothing to me to this effect. The letter seemed to imply that I may still be free to change the game. I would reject the men chosen by my half-brother. I would take my chances during the matchmaking process preceeding the Day of Unions. If I did not like my options then, I could still enter the Tower's service. I informed the crown of my decision then left for Deyalorn's libraries.

Deyalorn's White Tower housed the largest collection of books in Marsea. I would be here for eight more days. If I married a man that would not let me study, I would never get the chance to explore these books again. I found a copy of King Doran's history of the settling of Marsea, and settled down to read. I had heard Sophia mention these books on the campaign the day we had exchanged stories about Lucretia and the sons of Valir. She had aroused my curiosity. I wanted to read them for myself. 

I had nearly finished the first book when I learnt that I had a visitor wishing to speak to me. 

"Who?" I asked nervously. Was the Duke looking for me?

"A soldier, your grace," the page answered. Timmon, I thought. I wondered what he wanted. As the library is only for the gifted, I stepped outside to meet him. 

His appearance stunned me. In the last several hours, he had found the time to sleep, it seemed. He had also bathed, shaved, found a brilliant white shirt tied with a bright red belt the color of summer roses. His hair fell in soft curls about his face and below his shoulders, tied back only to keep it from covering his eyes. He was not dressed for court. He appeared more casual than that. I had not seen him appear this way since last winter, when Makala would play the blind and loving husband chaperoning his young flirtatious wife. I had forgotten how well he dressed when with friends.

"I came to apologize again for my rudeness." Timmon smiled awkwardly. "A group of players have arrived in the city today. They are giving their first performance in the square beneath the bluffs. Would you like to see it?"

It was not a bad peace offering, as such things are measured. I suddenly became very keenly aware of the disparity in our appearances. I was still in m simple healer's robes. "That is very kind of you, Timmon. If you would give me a moment?"

"Of course."

When I saw him next, I wore one of Chamila's borrowed dresses. Her seamstress had done a decent job of tailoring it to fit me. Timmon looked at me with approval. I felt myself blushing as his eyes took me in. "I had to be groomed for my viewing," I offered weakly.

"It becomes you," he said simply and led me out of the Tower into the late afternoon light.

The Day of Unions is a peculiar holiday. The day itself is one of grotesque and excessive hedonism. During the days leading up to it, all rules regarding decorum between men and women seem to be suspended. Almost all marriages happen on this one day every year. Some couples, under certain circumstances, may chose to live together for a time before the day, but they do not seal their union until this day in autumn. In the days leading up to the holiday, people of either gender who have not chosen their spouse are permitted to walk the streets openly in the company of those the are courting without causing scandal. It is the one time of year a woman may be seen with a different man each of five consecutive nights without having her character called to question, provided she marries one of them. 

I had been seen publicly with Timmon twice today. This would have its implications, I realized. I decided I did not care. If being courted by one below my rank was enough to be disinherited, then so be it. The Tower would not reject me for my reputation.

There were many courting couples in the streets. There were as many couples attending this performance as there seemed to be local city dwellers. I saw Carlotta make her way down to the square. She raised and eyebrow and smiled mischeviously at me as she passed me on the arm of a young priest, but we did not speak.

The drama itself was not worth recalling. It was a story of war and love, the sort that I had recently grown too cynical to enjoy. Timmon also seemed to lose interest in the performance. We left early, which Timmon used as an excuse to stuff me full of savoury tarts full of nuts I had never tasted. He returned me home a little after dark.

"I would like to leave early tomorrow. Should I wait for you?" he asked as he left me at the White Tower. 

"I will meet you as the gates open." I replied. I returned to my rooms feeling much better about my situation. I had thought the offer to escape this court to have been rescinded. 

Carlotta found me as I finished sending off letters to my half-brother and Duke Ergino, informing them of my absence for the next few days. My belongings lay packed for my journey. "Are you running away again?" she asked, only half joking.

Her eyebrows arched as I explained. "It is not what you think, Carlotta. He will not marry me. I asked today. He refused."

Her face fell to one of deep concern. She met my gaze and said "Be careful, Nisrita. You are a young unmarried woman still."

"I trust him, Carlotta. He will not hurt me."

"Perhaps not. I leave that to your judgement. But you have a reputation to guard. He may not be as careful with that as you need to be."

I chose to ignore Carlotta then. She did not understand the bond that Timmon and I shared. She did not understand why I needed to follow him from this place. I would see to my reputation later, I decided.

\vspace{.5cm}

We met at dawn and traveled briskly for six hours to reach his father's lands. Travelling alone with him was different than travelling with Griswold, or the Hundred Horsemen, or even what travelling with him on the campaign had been like. He expected me to carry myself like a duchess, not a soldier, or even a healer. If I spat, or urged my horse forward to freely run down a gentle slope, he would draw his horse along side me and say "That is enough, soldier." The first time he said this, I protested that I had given up fighting. "Perhaps," he replied. "But you have not stopped behaving like a soldier." The reprimand stung, though he had said the words gently. I was what I had trained to be. He had never minded before. However, I was to be his guest in his father's home. I decided that I would try to behave as he desired.

He told me of his time in Tiruna. He had been too wounded in the battle in the high valley to take an active role in the taking of the castle. He stayed at the stronghold, however, for a couple months afterwards, securing the area and helping to set up an army base. With Lukos dead, Cortan's new lands would be held temporarily by Duke Erfat, Duke Ergino's brother. Timmon did not like the idea. Duke Erfat was not the military commander that Duke Ergino was. He had lost three territories his brother had won for him, and had been living in Gissal up until Duke Lukos's death. General Madriano would stay in Turina to keep the territory safe until Duke Ergino could come up with a more permanent solution. Timmon was strongly considering the possibility of settling in the new territory. General Madriano would wish to return to Cortan in a few years. There was a good possibility that Timmon could take his place. 

When I asked him if he was still training recruits, he laughed. A commander's skills are generally needed elsewhere. He had only stayed with his boys after his promotion in Cortan out of a sense of loyalty to them. He wanted to complete their training without the disruption of changing officers. I got the sense that he was not being entirely honest. Timmon enjoyed training boys, for all his protests to the contrary. He was good at it. I wondered if he would miss that part of his life.

I asked him where his old scabbard was. He was not wearing the one he had decorated with Makala's chain. Timmon said he had left it in Cortan. Then he thanked me for the chess pieces and the ring. His father had given him the chess set, he told me. "My father does not know," he added soberly. He need not have said anything. I did not think the Baron of Romino knew about his son's peculiarity. He would not learn from me.

He asked me where my wedding chain was. "It was stolen," I said. Then I told him about the Hundred Horsemen, about Ernesto and Griswold, about Maya and Groto, about my mad quest to return Griswold to Cortan. I told him the same broad story I had told Carlotta, focusing on the military aspects of Griswold's genius, rather than the gift. It felt good to know there was someone else I could talk to without fear of judgement. I had been so alone and so scared for so much of these past five months. I had forgotten what the company of friends felt like. This was, after all, Timmon by my side. We had shared and lost too much to judge each other harshly. Unlike Carlotta, Timmon dealt with death. If anyone could understand what I had to do to return home, it would be him. I started speaking about my journey to Szarvis, and faltered. I could not bring myself to tell him about Sasha and Anya, or about the return journey. My voice simply would not obey. 

Timmon did not question the sudden ending of my tale. He matched my description of the North Sea with his own about the mountains that cradled the stronghold of Turina. He described a waterfall a day's walk from the castle, and high mountain meddows in full bloom where shepherds took their flock. He had climbed Mount Turin, the highest point in the Pensid mountains. It took him three days to go up and back. From there the world seemed to stretch forever at his feet. He thought he could make out the Velta River in the distance.There would be snow on the peak soon, he said. I was mesmerized. I had never seen snow before. He smiled at my wonder. It snowed every winter in his father's lands. 

We talked easily of pleasant matters until the last time we stopped to rest our horse. "We are in my father's lands" Timmon announced. "His house is an hour away at the pace we have kept."

I took a look around me. This was the Barony of Romino. These were the lands where Timmon had grown up. Everywhere I looked, I saw beautiful rolling hills or soft grassy plain. I was eager to ask him for stories about the ground I stood on. Timmon intruded on my inspection with a somber note. "Did anyone on your many journeys these past few months hurt you, Nisrita?"

I looked at him in surprise, the easy joy of travel suddenly and painfully knocked from me like wind from my chest. The way he emphasized the the word \emph{hurt} could only mean one thing. I wondered at his lack of creativity. Timmon is a warrior. Violence and death are his meat and bread. I had faced a miriad of different forms of violence that past year, many of which had occurred under his supervision. He of all people should be able to imagine more than one way to hurt a woman. Yet he only asked about that, the one way I had never been touched. "No Timmon, no one has ever hurt me." I saw a faint layer of tension leave his face. We mounted our horses and silently carried on. 

After a time, Timmon apologized for having brought a gloom on me, and told me of his family.

His father, Baron Desmond was a widower, Baroness Ismain having died ten years go. I would meet the young Barron, Guiseppe, a boy of thirteen who went by the name Cheppe amongst the family, his mother, Barroness Miri, and his three siblings Paulo, Anete and Julia, aged ten, seven and five. Timmon was the youngest of Baron Desmond's three sons. Timmon's oldest brother, also named Guiseppe had died two years ago of tetany and fever, a few weeks after he had been wounded by a rusty pitchfork while settling a dispute amongst some men on his father's lands. Timmon's next brother had become a priest and served in the neighboring barony of Cleyoren. Baroness Chamila probably knew him, he admitted. Timmon, it seemed, was the only member of the family to have cast his lot far from home.

Timmon led my horse off the road at a sharp curve and up a small hill. A beautiful black and yellow wooden manor house stood before us. "Welcome to House Romino, duchess" he said sofly. I stared at the many gabeled roof and the large curtained windows that looked out from the top floor. Two large stone griffins that sat on either side of the gate. It was more elaborately decorated, and looked more comfortable than house Farone. Farone was the only barony I have visited in Cortan. Baron Romino's house was certainly at least twice as big. I looked at Timmon. He was grinning again. He is proud of this place, I realized. He had every right to be. I followed him down to the house to meet his family. 

\vspace{.5cm}

Timmon introduced me to his family as Nisrita of Cortan, breifly and delicately explaining my precarious situation with the duchy. Baron Desmond, a plump man of sixty with a ring of thinning white hair, and Baroness Miri, a tall, fair woman of thirty, both gave Timmon meaningful looks, which he seemed to ignore. 

I bowed before Timmon's father. I would not insult my friend's family after all the kindness he had shown me these past two days. Baron Desmond gave me his hand hurried and raised me to my feet with embarrasment. "This is not right, duchess. I am a mere baron." 

I smiled as graciously as I could. "I will not be duchess for much longer, baron. I may not be one as we speak." That left me a nameless healer.

The Baron's eyes crinkled into a smile. "You are the Duchess of Cortan in this house until we hear otherwise, your grace." Having declaired my station, he greeted me accordingly, as did the rest of his family. He only bore a passing resemblance to Timmon, I decided. I could not see the commander aging into this portly old man. Cheppe, on the other hand, looked like what I would have imagined Timmon to be as a child. I studied the boy, looking for hints of who my friend may have been at my age.

The children took to me immediately. They led me away from Timmon, the baron and baroness to what would be my room. Without giving me much time to refresh myself, they then led me on a tour of the house, starting from the bedrooms, proceeding through the halls and sitting room, the servant areas and finally to the yard outside. 

"Is it true that you are a powerful healer, your grace?" Cheppe asked me between the stables and the kennels.

"Who told you that?"

"Everyone hears gossip about who the Dukes marry, your grace," Anete explained. "Last year, a Duke of Cortan married a powerful healer. This year, he is dead, and his father and the king are trying to divide up Cortan and Firvona. We had heard that the young widow was dead as well. But now you are here. Are you she?"

I laughed at the girl's frankness. That was an acurate summary of my situation. "Yes, young baroness, I am she."

Timmon and his father found us near the kennels as Cheppe explained that they did not have a White Tower of their own, but shared one with the neighboring barony. The siblings wanted to see how the gift worked. "Do you have a wounded creature in your yard?" I asked.

"No, but use this puppy." He stepped among the dogs and pulled out a black pup that looked less than two months old.

I looked uncertainly to Timmon.I did not want to harm this creature for entertainment. But he nodded his approval. I bound and muzzled the creature and knelt by its side, pinning it to the ground. It was still small and fragile. I would have to work carefully. I cut a deep gash in a hind leg then gently and slowly stopped the bleeding  before forcing the skin to join with itself again. Cheppe was wild eyed in amazement. Julia had started crying at the sound of the puppy's whines. "That is cruel." Anete whispered. 

"It is," I said, craddling the quivering pup in my arms. "Which is why we do not do it for pleasure or entertainment, but only to practise and learn." I heard Timmon clear his throat at my words.

Anete took the dog from my arms and turned on her brother for being a cruel monstrous boy. Baron Desmond chased his grandchildren out of the yard, leaving me alone with Timmon.

"You have changed since I saw you last, Nisrita."

I had, but I did not see anything in the display I had just given to indicate that. "What do you mean?" I asked, rising.

"The girl I knew walking to Turina never showed remorse at hurting a passing creature."

I sighed. I had not been able to make him understand then. He still did not understand now. No matter what I did or said, he seemed to think my art cruel and heartless, not built around the core principal of saving lives. I tried again. "That was to learn, Timmon. I do not enjoy hurting creatures because I can." Warriors and ruffians did that. Healers did not.

I would not win this battle today. Timmon changed the subject. "Follow me. I believe you wished to see the orchards?"

He took me to the orchards, pointing out various features of his father's lands along the way. He seemed to hold himself differently here. He seemed more boyish, more at ease than I had seen him in Cortan. His brilliant smile had settled in just below the surface of his face. It would come out with the slightest provocation. It was a pleasant change in his bearing. I wondered if I was seeing something of what drew Makala to this man so many years ago. 

The orchards were huge. Timmon explained that there were five different types of apples growing in the hills of fruit trees. He handed me two differently sweet fruit, then left me to gather samples of the other three. I cleared a patch of grass of bees and rotting fruit and lay back. The sunlight landed gently on my face filtered through a canopy of green and red. There is so little soft green grass in Cortan. There was so much of the luxurious vegetation here. Timmon's nieces and nephews did not know how lucky they were. I dug my fingers into the velvety green blades and closed my eyes. 

I must have dozed off. At any rate, I did not hear Timmon's return. The next thing I was aware of was the sound of someone whistling. I opened my eyes to see Commander Romino perched like a boy in his father's apple tree, whistling. I pinched myself to see if I was awake. Timmon never whistled. He never sang. I have heard Makala tease him countless time about this fact.

It was a simple tune. When it repeated, I sat up and whistled along. Timmon stopped suddenly. "Whistling does not suit you, soldier." The tune froze on my lips and I plucked nervously at the grass. I felt completely and abjectly humiliated. I did not understand how Timmon's disapproval had this effect on me. I hated when he called me soldier, though I knew he meant the word as a jest to ease the burden of his criticism. I hated the idea that I had failed to meet his expectations. No one, not Makala, not Carlotta, not even Griswold's criticism had made me feel this way. It was more than wanting to be polite to his family. I felt an irrational need to please him. I heard the branches rustle above me, and Timmon land lythely on the ground followed by a small shower of fruit. "Nor does swinging boyishly from apple trees suit a commander." I looked up. He was grinning. I felt my face forming one of its own, then bit my lip to tame the grin into a ladylike smile. I did not want to be called soldier again. This was Commander Romino, the trainer of recruits. He was critical and demanding, but he was scrupulously fair. He made one want to rise to his standards.

"Are you tired?" he asked, handing me three different apples.

"I must have been. I am fine now. What else would you like to show me?" We walked the afternoon away. Timmon showed me his childhood haunts, introduced me to his father's men, showed me what small points of interest there were in this tame green land. At some point, I realized that Timmon was happy. The despair I had seen on his face less than two days ago had completely disappeared. In its place was a proud, relaxed joy. He was almost as happy as I had ever seen him with Makala. It gave me a giddy painful pleasure to see him this way. If there was anything I could do to keep him happy, I would do it. I had been so alone since Turina. I had been so deeply engulfed by grief, then fear. What could I not sacrifice for the satisfaction of of seeing a friend happy.

If coming home to his family could do this to him, I wished suddenly and vehemently that he would stay here to settle, leaving Turina and promotion far behind. Carlotta's words returned to my ears, modified for Timmon. "Is being happy as a commander so much worse than being miserable as a general?" She was wise for her age. I decided I should speak to Timmon about this later. For the moment, I basked in his joy, and let the memories of my travels and worries for my future melt away. I learnt about Timmon's childhood until he led me back to his father's house.

\vspace{.5cm}

Dinner at the Baron's table was far more pleasant than at Duke Ergino's table. The Baron's men came to greet Timmon and congradulate him on his successful career. Baroness Miri introduced me to each, and guided me through the conversational niceties. It was like conversing with Chamila. I was not a lone pawn on a chess board trying to survive to the end of the meal. If I paid attention, and followed my hostess's lead, I could sail through the meal easily, even enjoying myself in the process. 

I watched Timmon quietly during the meal. There was an ample supply of alcohol at the table. Carlotta had described him as a drunken mess. I had not seen him drunk since yesterday morning, but neither had I seen him have the opportunity. The current situation made me nervous, though I could not think of what to do if he made a spectacle of himself in front of his family. I need not have worried. He did not drink more than I did, his spirits seemed to stay naturally buoyant. 

I speak of Timmon's joy that day. I should mention my own. When Baroness Miri left me alone in my chamber, I feared that I was too happy to sleep. I did not think I had ever felt quite so complete. I had wanted a life beyond the Tower. Timmon had done me the favour of letting me see his. It was a pleasant, welcoming life, I was grateful to share in it, even if it could not be mine. Come what may when I returned to Deyalorn, I decided, I would keep this vision Timmon had given me as a reminder that there is more to life that bearing children for Cortan or the solitude of the Tower.

I awoke in the middle of the night for no reason. A memory from that day tugged urgently at the corner of my mind. I tried to push it away and return to sleep, but it returned, insistent every time I relaxed my grip on the world. I sighed, sat up and fed the fire. I could not fight that day forever. This was a pleasant memory, at least. I sat on the warm hearthstones and relived the moment. 

I had awoken in Makala's tent before dawn. He seemed to be still asleep. I gathered my things and rolled up the rug I had lain upon to join my healers. I had seen him together with Timmon the previous night. They must have thought me asleep. I saw their shadows move in the darkness of the tent. They were so tender towards each other. I could not hear what they said, but their whispers seemed to caress each other as well as their bodies. I found myself filled with longing and jealousy. Not for them. I had learned my lesson. I had no desire to come between them anymore. I did, however, want that tenderness for my own. 

Makala stirred as I got ready to leave. "Will you not say good morning before you leave, little one?"

I grinned and hooked his outstretched finger with my own, sitting on his bed where he had moved to make room. "Good morning, my liege." I teased. "How is your back?"

Makala flexed the muscles that I had found bloody the previous day. "It will serve."

"Let me fix it." Makala started to protest, then stopped. His survival of this battle was more important than that of almost anyone else. Riding into enemy lines wounded was not a risk he should take if he could help it. His pride or desire to husband my resources would have to bow to that reality. Most of the lacerated skin an muscle I had found last night had knit itself back together. I patched some of what remained, then stopped. I did not want to exhaust him too much before battle by healing him either. "Timmon is a good man," I said when I finished. I did not know why the words came to my lips.

Makala's resting form suddenly tensed as he looked at me. "You were awake last night?" I nodded. "I am sorry." 

He truly looked remorseful. He had not reason to be. I was not the jealous childish girl he had married. "I am happy for you, Makala. Don't apologize."

His responce surprised me."Then why bring it up, little one?" 

Not only was Makala like a brother to me, he could read my mind. It was uncanny how he knew what I was not saying. I folded my hands in my lap and thought about what I had seen. "I wonder if anyone will love me as you two love each other."

Makala laughed his soft gentle laugh. "You are young, Nisrita. Time will provide. In the mean while, you are right. Timmon is a good and loyal man. He protects those he cares for. He does care for you." I had paid attention to the first part of Makala's claim about time providing then. I sat sulking on his bed, not hearing his message about Timmon. "Off with you to your healers," he said eventually, chasing me off like a little girl. "Let me dress."

I stopped thinking about that day. To recall anymore would be foolish. My face was wet, my chest filled with grief. The grief did not overwhelm me that night. I missed Makala. I missed Makala's friendship, wisdom and warmth. I missed the security and the sense of belonging he had provided for me. But I could go on. I had carried on. Tomorrow would be no different. Timmon cared for me. I had an ally in him. Perhaps that would be enough.

I found the idol of the triumverate every chamber should contain, and for the first time in five months I prayed for Makala's soul. I had not neglected my gods in this intervening time, but praying for Makala had always seemed to difficult. The memories of that day always threatened to break through my self control if I thought about him too much. I annointed the Destroyer, asking him to watch of Makala, praying for his comfort and ease in the lands inbetween. Then I annointed the Maker praying that he be returned to this world soon, and given a good new life. His was a kind soul. He deserved a good life. Then I thanked Makala for everything he had given me. There was a future he had wanted for me. I would not fail him by failing to realize it. I left the oillamp and incense burning for the gods, and returned to sleep.

\vspace{.5cm}

I joined the baron's family for early morning prayers. This pleased Baroness Miri greatly. One in the list of Timmon's many faults was that he was a much less religious man than he ought to have been. I spent the morning with the Baroness and her children. She asked me about my life in Cortan's Tower, we talked about the similarites and differences between my life and the one she led as the Baroness. She showed me the workings of the manor house she ran in her mother-in-law's stead, and took me to meet the wifes of the Baron's men, introducing me always as Timmon's friend, the duchess. There were implications in these introductios that made me uncomfortable. I wondered what Timmon's family thought of me, and why he did not correct them.

That afternoon, Timmon took me to his father's woods to check on the state of his traps. I watched him repair and reset the traps as needed, collecting and tying the small game they had snared to his horse. I did not offer to help. He would have found it unbecoming behaviour. He worked silently, letting me enjoy the smells and the sounds of the forest, speaking only when I asked him questions.

Half way through the rounds, he turned to me and said "Father mentioned that you were up in the middle of the night, last night. Is this true?"

I stammered in surprise. How had the old man known? "I had not meant to disturb him, Timmon," I said meekly. "I am very sorry."

Timmon smiled and waived away the apology. "He is an old man and a light sleeper. He did not sleep worse than he normally does for your movements. Something has been troubling you, Nisrita. Will you tell me what it is?"

I thought about Sasha and Anya, my guilt for not helping them and my anger at Griswold for the cruel way he kept me away. I thought about the smugglers and the horrible ways in which they treated their prisoners, and the things I had done in return for passage. I opened my mouth to speak, but found a hard lump blocking my voice. I stroked the mane of the horse I rode and licked my lips nervously. Timmon was waiting for me. I could feel his eyes on me. I had to say something. He would understand, I told myself. I could not bring myself to speak.

Timmon broke the long awkward silence carefully. "Perhaps another time, soldier?" 

I looked at him. There was no criticism on his face. I released a breath I did not know I had been holding and tried to smile. "Yes, perhaps another time. Thank you."

"So," he said, changing the subject, "What do you think of this barony?"

I looked around at the verdant green forest, took in a deep breath or fragrant air,  and chased the unwanted memories from my mind. "It must have been a very pleasant place to grow up, Timmon. I am jealous."

"Do you think you could live here?" he asked.

Timmon spoke in riddles. His family had been hinting at the impossible. I did not like it. People in Cortan's court spoke this way, cornering me into statements I did not want to make. "What do you mean Timmon? When did you pick up this habit of speaking obliquely?" I was certain to be called a soldier for my outburst, I realized too late.

"Do you need me to write a formal letter to the crown seeking permission, Nisrita?"

"But," I gaped. He must have been joking. His tone was completely sober, but he must have been joking. "You said you did not wish to marry, Timmon."

"I have reconsidered." He watched me stutter and stammer my surprise for a while. I had not expected this. I had thought my only options were to serve Cortan as a wife to a tolerable duke, or live in solitude in the White Tower. This changed everything. I watched my lonely future disolve before my eyes. Timmon sobered my giddy line of thought. "I make no promises, Nisrita." He spoke slowly and carefully. "If Duke Ergino still claims you when we return, I have little to offer to compete with other noble men he would want to tie his house to. But the duke values my services. I have some sway with him. I will do what I can." 

I found myself laughing and stammering and thanking Timmon stupidly. I could not control my joy and relief. I could have a home and a name and my art. I could live with a kind man, a friend, not as stranger I would have to be wary of. I could share in this warm family I had been introduced to. I could not believe that there was still a way for it all to happen. "If the Duke has disinherited you," Timmon continued when I had quieted, "my father has agreed to let you live here for a time. It is a sheltered barony, far from whatever still haunts your dreams. I presume you will be able to find a place for yourself at Deyalorn's Tower. I cannot help you there. At the moment, I do not have a life in Turina that can support a wife. When I have built that, and you are better, I will bring you to the border, if that is what you wish. Will that suit?"

What an idiotic question. Would that suit? The blessed man was handing me everything I could possibly want and then a list of things I did not know I wanted, asking if it pleased me. I could not believe my ears or luck. "Yes, Timmon. It will suit."

"Then I need you to promise me one thing," Timmon continued in that same solemn tone.

"Anything." He could name the sky and the stars as his price for his offer. I would find a way to deliver.

"You cannot run away again." he said gravely. "I cannot serve in Turina without knowing that you are safely where I have put you."

I looked at his solemn face. His eyes shone, not with the excitement I had just felt. I had hurt him. I had not realized that I would be hurting him when I had left in search of Griswold. I had not wanted to hurt him. As I told Carlotta, I had not been thinking. I would never hurt him again, I told myself. "I promise, Timmon. I will not run away again." Timmon seemed satisfied with my answer, but I was not. He deserved a reassurance as a friend, even if he could not bring about this miracle about and be my husband. "No matter what happens in a few days, I promise. I will not run away."

Timmon smiled for the first time since he broached the topic of marriage. "Thank you." Then he grinned. "There is clearing near here where bears search for blueberries. Would you like to see a black bear's print?" Of course I would, Timmon. I would follow him anywhere.

\vspace{.5cm}

Timmon told his family of my acceptance of his conditional proposition. The news was met with laughter and celebration. I spent the rest of the day, and most of the next engulfed in a fog of joy. I understood why the baron and his family allowed Timmon and I so much time together. House Romino had not had a gifted member in the family for many generations. This branch was likely barren in that respect. I would likely not have any gifted children, for I would have to have children, I could not escape that. But with a good match, my grandchildren may be gifted, returning prestige to the small house. Even without my title, I had connections in the Towers of the land. I was a good investment for the house. It was natural that they give Timmon as much time as he wanted to court me. 

I was no longer so naive as to believe that the kindness Baron Desmond and his family showed me was not at least in part due to this desire for a match. That did not diminish my pleasure at their generosity. I tried to imagine myself living in this house, tending to these grounds. It was an easy fantasy. The lands were rich and beautiful. The winters would be cold, I knew, but soft emerald green grass covered the ground for much of the year. Deyalorn's tower was at least half a day's travel from here. I could not live both with my family and and the Tower as I had in Cortan. I would have to chose when to be healer, and when to be wife. On the other hand, I would have access to Deyalorn's legendary library. I could study among the books when I could not practise. I would be expected to provide healing for the Baron's men and beasts without compensation when I lived here, but that was hardly an imposition. In many ways, life here could be better than what I had in Cortan. 

I redoubled my efforts to make myself pleasant to my hosts. If I was to live here, I wanted to be more to this house than just a good match. I played chess against Cheppe. I found I could beat him. Under Timmon's watchful and critical eye, I learned how not to humiliate him in the process. I listened to Annete sing, and sat with Julia as she learned to sew. I let Baron Desmond regale me with stories of this part of Marsea from his youth. There was nothing but Cortan's rejection I could possibly want to complete my happiness. 

I sat comfortably reading by the fire on the evening of my third and last night with Timmon's family when I heard a commotion outside. It had been raining off and on all day. The baron had asked Timmon to meet with a local merchant. When I elected not to travel in the rain, he took his nephew with him. The stablemaster's wife called me urgently to the courtyard. Cheppe sat wincing and apologizing to his uncle on a pile of dry hay in the stable. Timmon stood sternly over him. I wondered what the boy had done to injure himself.

"Is he badly hurt?" I asked, making my way across the muddy yard.

"He has hurt his ankle." Timmon answered crisply. "Thank you, Nisrita."

I examined the joint. It had just been sprained. Then I considered my patient. Hours away from the nearest Tower, Cheppe had probably never been healed before. I did not look forward to this task. A soldier knows how much pain my art will inflict. They often do not appreciate the Tower's efforts for the extra pain we cause, but they understand that it is necessary. Other ungifted do not know what to expect. No amount of warning them of the pain ever seems to be enough. I would appear to be torturing the young baron. It was not a pleasant prospect under any circumstance, even less so in front of people I wished to please and live with soon. "This will hurt, baron. It will hurt more than you expect, but you will be fine tomorrow." I tried to sound comforting.

Cheppe nodded. I took a deep breath and opened myself to the fire. Before proceeding, I ran a tentative flow through my own body as a precautionary test, as I had done with the puppy. I had left Griswold's company too recently. It was not a failproof test, but it was better than nothing. I froze when felt the irregular resistance. Then I tried to calculate how long it had been since I had last bled. Timmon, and much of his father's household, was watching me. I could not admit why I could not heal the young Baron. I had do what I had started. I decided quickly that it had not been long enough that I had much to fear. This would be the last time I would practise, until I could sort out this new development. I released the flow of fire from my body, imagining it curl around the joint, and Griswold had shown me. I made Cheppe howl with pain. Timmon held the boy down so he did not thrash. I do not need to see the wound I heal. But I do need it to be stationary. Otherwise, I heal inefficiently, exhausting myself and causing the patient more agony than necessary.

"Keep it warm tonight. You should be able to walk on it by the morning." I told Cheppe when I finished. I left for my room before he could rise to thank me. 

I sat on the bed, shaking like a leaf. I could not believe what I had learned. I did not know what to do with this news. I ran one more gentle flow of fire  through my body, willing the discovery away. I still felt the irregularity. It was impossible I told myself, against all evidence. I took off my talisman and reluctantly put it away. I would not be needing that for a time. I would not deny this harsh reality and risk my life. Then I sat on my shaking hands to still them and tried to think. I had to tell Timmon. That was certain. If, by some miracle, Griswold came to Deyalorn, I would have to tell him. Would I? Would he care? I could simply destroy this fetus instead. How long had it been since I had last bled? Five weeks I decided. It was still early. I could destroy this, and no one would be the wiser. I would not have to tell anyone. Nothing would have to change.

I heard Timmon stop in the open doorway I sat with my back to. "What happened, Nisrita?" he asked, still stern.

I cursed silently like a soldier. I had not had enough time to think this through. I did not yet know what to tell him. "What do you mean, Timmon?" I fumbled, deciding that I would have to lie to him. It would not be easy. I did not see a choice. 

"It took you a long time to start healing Cheppe. You looked like you did not wish to. Have you two fought?"

I hated his over observant eyes. Why did he have to know my habits so well? "No, Timmon, we have not fought." I could not, however, come up with an excuse to explain this behaviour.

"Then what happened?" he demanded, entering the room to face me. He would not leave without an answer. 

With my back to him, I could conceive of misleading this man. Facing him, it was impossible. I grasped for a way to reveal this news gently. "I will not be healing for a time, Timmon."

I watched sternness change to alarm then concern. "You are giving up on the Tower?"

No, of course not. Why did the man not understand? Would he force me to say the words? I tried again. "No, Timmon. I will only stop practising for a few months."

He eased himself into a chair as understanding filtered into his mind. "I see."

"I will get rid of this as soon as I can, Timmon," I spoked hurredly in my desperation. "No one need know. The women's tower is discrete in these matters. Your family will not lose face." I sounded like an idiot. My concern was not for House Romino's honor as much as it was for Timmon's opinion of me. I could not bear to lose his respect and friendship. Why did he have to come upon me immediately, before I had had time to absorb this news myself?

"Wait, Nisrita." I stopped my babbling and waited for him to continue. "I thought you said no one had forced you."

So it would come to this. I looked down at my hands. I could not face Timmon and tell him what I had done with Griswold. "No one had." I whispered. 

I heard him sigh. I did not dare look up. I imagined I could hear all hope of a future with Timmon escape through that breath. "I see." He said again.

I could not take it. I would not be held in the company of  Griswold's many whores. "I am not a whore, Timmon. It is not what you think. You do not understand."

"Wait," he interrupted me again. He was looking through me, at the wall beyond my head. I watched him contemplate his options and collect his thoughts. He held his body very still, the only motion in his eyes, as he flitted from one thought to another. I fidgited and twitched as I waited for him to make up his mind. I could not sit still in my anxiety. I valued Timmon's opinion of me more than I valued that of anyone else. I needed him not to despise me. It was terrible waiting for his verdict. I was completely in this man's control. Whether I had wanted to be or not, he would now decide my fate. How could I convince him to judge me kindly. After an eternity, he shifted his weight, looked at me and spoke. "No one is questioning your character, Nisrita. Who is the father?"

"Griswold" I mumbled. It was impossible. We had been so careful, but there was no other. Timmon let out a long slow breath at the word. His face darkened as he sat and contemplated the news.

"If, as you wish," he began slowly, "Griswold appears in Deyalorn and asks for your hand, I will not challenge him. Otherwise, I will proceed as before. We will discuss what to do with the child, if I gain your hand."

"Thank you, Timmon" I choked out. Marrying Griswold was not what I wished. I had perhaps once, but that had changed. I did not think I could explain that to Timmon under the circumstances. He had been more than generous.

Timmon rose. "Cheppe is downstairs with the rest of the family. Come join us. Father is worried about you." I nodded. I would come down in a moment. He paused on his way to the door. "You are more of a soldier than you think, Nisrita. You traveled with Griswold under the guise of his wife. You made a mistake. We will fix it."

I sank back onto my bed when he left and wiped the tears of relief and gratitude streaming down my face. How had I possibly thought that giving myself to Griswold was a good idea. How had I possibly thought that marrying him would help anything, least of all me? He was a cynical, cruel, selfish man. His intellect not withstanding, he did not hold a candle to Timmon.

\vspace{.5cm}

We arrived in Deyalorn late the next day, on the eve of the three days of matchmaking that preceed the Day of Unions. Everywhere we turned, people spoke in excited unbelieving voices about the man Duke Ergino had returned with that morning; a man young enough to be his son, who claimed to be his lost uncle. I turned fearfully to Timmon. What would he do with me now? He held himself impassive, as if he had not registered or recognized the news. He left me at the White Tower, saying only "I will see you tomorrow." He gave me no sign of his thoughts or intentions. I wished I could read his mind. I needed some omen, some insight into what I should expect from him the next day. I got nothing but his retreating back.

The Tower was bursting with excitement over the news that the great healer Griswold had returned from exile. The entry hall was full of people drinking to his health, or speculating on what he had done for the past thirty years. Every Master in the Tower wanted to get their hands on the healer to study him or study from him, to learn how he had traded eternal youth for his ability to work with the gift. I could have answered many of their questions, and destroyed their exalted opinion to the mysterious arrogant Griswold. I did not want to draw attention to myself. I was tired. I had gained and lost too much in these brief four days. The last person I wished to think about was Griswold. I found Carlotta, informed her of my safe return, and retired to the room we shared. 

I had barely put away my belongings and changed into healer's robes when she stuck her head through our door. "What is the matter with you Nisrita? You look like the world has ended. Griswold had returned. You succeeded in your impossible task. Is this not what you wanted?"

"No," I said and sat on my bed to face her. I explained the events of the past several days. Eventually, she sat down beside me, wrapt her arms around me and held me while I wept. 

The next afternoon, Carlotta ran panting into Deyalorn's library where I had been reading. She told me to go to the stables quickly. She did not need to tell me who I needed to make haste for. I ran. I saw Timmon tying the last of several packs onto his horse. The beast looked like it was prepared to go on a long journey. To Cortan, perhaps? Or Turina. "Were you going to leave without saying good bye, Timmon?" I asked.

He looked up briefly from his hands. "Of course not, Nisrita. I have not left yet." 

I looked at the burdened creature. It was all but ready to leave. His answer was a technicality. "Why are you leaving, Timmon? He has not asked me yet." I had heard nothing from Duke Ergino, Duke Griswold or the Regent-Consort. As far as I knew, they were still involved in negotiations regarding the terms of Griswold's return. This confused me. If I knew anything about Griswold, he would not have returned to Deyalorn without some gurantee for his safety and comfort. I did not understand what took so long. 

"Makala's shadow looms too large between us."

"Timmon," I protested, then stopped. No, I did not understand what he meant. "What do you mean, Timmon? Please?"

"I saw him, Nisrita. I went this morning to ask Duke Ergino for your hand. As I waited for an audience, I heard Duke Griswold be announced and saw him enter before me. I thought at first that I had seen a ghost. I understand, now, why you did what you have done. I do not hold it against you. Perhaps in your place I would have done the same. But I cannot stand by and see you married to that man." There were tears in his voice, and more in his eyes, pouring down his face. I stood by dumbly as I watched him check his straps and harness. I did not know what to say to him. I had hurt him. I had not meant to. Timmon was a good, generous man, and I had wounded him badly. Nothing in my power could heal that wound. I could not take back what I had done.

"I'm sorry," I managed weakly as he mounted. How could it be that he was riding out of my life?

"Goodbye Nisrita. I wish you well, whatever you decide." With that, he left. I could do nothing but watch him disappear beyond the barracks gate. I turned and ran up the steps of Deyalorn's White Tower. I could see the bridge leaving the city from there. I did not want to lose Timmon. I did not want to be alone again. He had made up his mind. There was nothing I could do but watch. I saw a lone horseman appear on the bridge. Was that Timmon? I could not tell from this distance. 

I sank to my knees at the top of the tower and prayed, begged the Preserver to keep him safe, to show him a path out of the pain I had caused. He had shown me kindness to the very end. He wished me well, whatever I decided. What choices remained? As the blood of two strongly gifted, the fetus I held inside me would likely grow to be a gifted child. I could not lightly get rid of it. It was too valuable. I could be duchess and wife to the cruel Griswold and give my child a name. Or I could live free of fear as Makala's widow to Cortan's tower, and raise this child a bastard, praying that the gift manifested early so the Tower would help me raise it. I did not know which would be the harder path. I did not know how I would decide.


\end{document}
