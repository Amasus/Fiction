\documentclass{amsart}
\usepackage{fullpage, verbatim}

%*****************
% Annotations
\usepackage{soul}
\usepackage[colorinlistoftodos,textsize=footnotesize]{todonotes}
\newcommand{\hlfix}[2]{\texthl{#1}\todo{#2}}
\newcommand{\hlnew}[2]{\texthl{#1}\todo[color=green!40]{#2}}
\newcommand{\sanote}{\todo[color=violet!30]}
\newcommand{\esnote}{\todo[color=orange!40]}
\newcommand{\note}{\todo[color=green!40]}
\newcommand{\newstart}{\note{The inserted text starts here}}
\newcommand{\newfinish}{\note{The inserted text finishes here}}
\setstcolor{red}
%***************************


\begin{document}
	Kalpana awoke in an unfamiliar room. She lay on a reed mat, not unlike the one
	she slept on, but rougher, not jute. The walls of the room she was in were
	rounded. No, the entire room she was in was a circle, with a woven screen, as
	tall as a man running through the middle, dividing it into two. There was a
	window on her side of the room, with wooden bars, not bamboo, as she had in her
	house, and, she could tell from the shadows cast by the screen, a similar window
	on the other side of the circular room. The walls were earthen here as in her
	home, and, she could tell from the window, about as thick. The sun must have
	been low in the sky to cast the shadow through the window as it did. Outside,
	birds sang unrecognizable songs and goats bleat for the attention of their
	herders. Early morning, then. But that was wrong too. There was the faint smell
	of something being burnt in the distance to bless the new day, but it was far
	too sweet to be any incense she knew, let alone the dhun she used.
	
	She sat up to look out the window. She caught the briefest glimpse of a
	carpenter's workshop outside and piles of logs of unfamiliar trees before being
	overtaken by a sudden dizziness. She was not well, she recalled, as she placed
	her head against the cool clay wall. It was drier and rougher than she expected,
	smelling more of dung and hay than her walls at home. She last remembered
	burning with fever in her own bed. It had been a bad year for the rice harvest
	everywhere, and none of them had found enough work harvesting other people's
	fields to make ends meet. She had taken to harvesting the small fish left over
	in the nooks and crannys of fishermen's nets to supplement their meals, but the
	waters were leech infected and dirty. Her left leg became infected, and she
	could not put weight on it to help care for the animals at the house she
	normally worked at. She vaguely remembered her second son finally convincing his
	older brother to let him go for a doctor. 
	
	Why had her sons moved her, then? That made no sense either.
	
	"You are awake." A man's slightly embarrassed voice appeared behind her. "I'm
	sorry I didn't catch you earlier. You have been through a lot. You should not
	have tried to stand up. "
	
	Kalpana turned to see a young man, in his early thirties of wiry build and dusky
	complexion. Not dark enough to be spending his days working in the fields all
	day, but not fair enough to be born of a wealthy household, either. His thick
	black hair clung in soft curls around his head and he was dressed in what looked
	like a pale lungi tied in a most unusual fashion. His bare chest and arms told a
	story of a man who ate well off the fruits of hard physical labor, and she
	imagined that the carpenter's workshop outside belonged to him. And ugly long
	scar protruded above the cloth on the right side of his abdomen that told a
	story of a bad accident many years ago. 
	
	"Are you hungry? There is food in the other room." The man smiled the kindest
	smile that she had ever seen. For that matter, there was something about his
	bearing that compelled one to trust him. Kalpana put aside her questions of
	where she was, and who he was, and let him guide her to the other half of the
	circular room, where, indeed two places had been laid out, with tea, and sweets
	and piles of hot savories in between. Moa, she noted, and fried jute leaves,
	fried balls made of palm fruit and small savory dumplings made of rice flour,
	and stuffed with, she hoped, spicy peas. All favorites of hers, either as a
	child or an adult, Kalpana realized, and smiled. She'd never had the luxury of
	having them all at the same time. There is a reason for that, she immediately
	realized. The ingredients for these dishes are never in season at the same time.
	
	"Have some lemon water before tea, sister," her host interrupted her disturbing
	realization with the offer of a tall, full wooden cup. "It will ease your
	stomach after your journey."
	
	So many questions awoke in her head, begging for attention. What journey had she
	undertaken, and where was she? Why was her host, younger than her eldest son,
	and impeccably polite in all other aspects, calling her sister?  But even if she
	could open her mouth to question her host's hospitality, she was too busy not
	recoiling from the sight of her hosts hands. Through the center of both palms
	was the largest, ugliest set of scars that Kalpana had ever seen. The skin on
	both sides of both hands was pale and puckered an knotted and warped; it was a
	wonder that the man could hold anything his hands, let alone dress himself or do
	any wood work. Kalpana imagined that only a bullet shattering the bones of both
	hands, or both hands being horribly impaled in some childhood accident would
	cause something that looked so ugly and so crippling. She could not help
	herself. She stared.
	
	Her host tolerated her gaze for a long moment, before prodding her again. "Lemon
	water, sister? It will make you feel better." 
	
	Kalpana quickly moved her eyes from his hands to his face, and smiled
	graciously. This man had gone to all the trouble of preparing all of her
	favorite dishes. It would be rude not to accept his hospitality due to a
	physical deformity. She reached out both her hands to accept the cup. Then
	recoiled again. This time at seeing her own hands. They were ... so young.
	
	She pulled her hands back and touched her face. The flesh felt firm and full.
	She looked down. Her breasts did not droop under her sari, her torso was young
	and plump and strong. This was the body of a young mother. \emph{Her} body when
	she was a young mother. Not her body as she remembered it, a mother of four and
	a grandmother of six. Her host pressed the cup into her hand and insisted,
	"Drink, sister."
	
	She looked at him in astonishment, at the gentle sadness and immense patience in
	his eyes. Where was she? And then it suddenly made sense. Her eldest son, Jagat,
	falling to his knees, begging for her forgiveness after his brother Ajit had
	raced off to find a doctor, the constant look of worry on his wife, Supriya's
	face as she lifted up and cleaned her painful, foul smelling leg every day. The
	odd overheard line in the argument over brothers "a priest is more expensive
	than a doctor." "I'm dead," she whispered.
	
	Her host nodded. Kalpana took a sip. What else could she do? The water was
	fragrant with the floral scent of the fresh lemons of her home, and something
	else. An herb she could not name. It was salty, and sweet and invigorating, if
	that was an appropriate word anymore. She drained the cup.
	
	"Feeling better?" her host asked, starting to pile food onto a plate.
	
	Kalpana nodded. "Where am I?"
	
	"Your soul is awaiting evaluation. It has fallen upon me to decide how you will
	be reborn." 
	
	"But." Kalpana took the offered food. She did not want to be rude, but this was
	very confusing. She was not a learned woman, but she had attended neighborhood
	pujas her entire life, and as she had become aged, spent more and more time
	listening to the priests. There were many things she might have expected of the
	afterlife from the stories she had heard. But being greeted by a simple
	carpenter was not one of them. "You are not ... You are not an incarnation of
	Krishna." 
	
	Her host smiled ruefully and shook his head. "No. That I am not." He started
	arranging a place for himself and motioned for her to start eating. "All the
	same, it has fallen to me decide upon your afterlife." He poured two cups of
	tea. "So, tell me, sister. Tell me about yourself."
	
	The question took her aback. She couldn't remember the last time someone had
	asked her about herself. Oh, her sons asked her how she had slept or what she'd eaten the
	previous day, and her daughter-in-laws asked about her health. Her grandchildren
	always wanted stories of her childhood. But that was different. This question
	was about her, as a person apart from her domestic duties. She realized suddenly
	that she didn't know what to say.
	
	"I'm a simple rural wife," she stammered. "I'm not very important. There's not
	much to tell." 
	
	"Of course there is," her host prompted, with an earnestness that made her want
	to find something to say just to please him. "There are seventy two years worth
	of stories to tell."
	
	Kalpana cast about for something to say to lessen the intensity of this man's
	attention from her. She was not used to people paying this much attention to
	\emph{her}.
	
	"Tell me about your childhood," he prompted.
	
	Kalpana relaxed. That was easy. Her grandchildren had asked for stories of her
	childhood more times than she could count. She spread out her too plump fingers
	and imagined herself sitting on the veranda of the house she had raised her
	family in. 
	
	"I was born in a village called Hatibari, in the Nadia District, less than a decade after
	independence" she began. Her
	father owned only a small plot of land, enough for vegetables and a few
	chickens, but not enough for a rice harvest. He fed the family by doing odd jobs
	in the off season and working other people's paddies with her brother during the
	harvest time. Her mother, her two sister and herself worked at different
	people's houses throughout the year. She only worked at one house for most of
	her childhood. The mistress of the house had taken her under her wing from the
	first day, watching over her as she learned to slice cabbage as thin as paper,
	or gut and clean fish perfectly, without cutting herself with the sharp blade.
	The cook, apparently, was quarrelsome, and the mistress did not think it
	appropriate to give her a young child to train. 
	
	The mistress, on the other hand, was perfect. She was fair and plump and soft
	spoken, and never appeared in a house dress or an unpleated sari in front of the
	servants. Whenever there was a puja in the household, the mistress made certain
	that Kalpana had an extra spoonful of ghee in her khichuri; she always gave her
	a dress for the holidays, and when her father came to collect her for three days
	every month, she always sent her home with a pot of ghee, or a jar of pickles,
	or a pound of jaggery, or whatever else was on hand, just as if she were a real
	servant, and not a child in training. 
	
	Her favorite evenings where when she had finished with her chores and the
	children with their studies, and they could run around in the field across the
	way from the house. There were six children in the household, ranging in age
	from 10 years older than her to a year younger than her. And when the youngest
	three played with her, she never wanted to belong to any other family. The
	youngest child, Preethi, never wanted to share anything with her, but her elder
	sister Jhinuk, was fair and level headed and always let Kalpana comb her dolls'
	hair, or prepare meals out of flowers and leaves whenever the girls had their
	dolls over for formal luncheons. Between the two girls was Tapas, the youngest
	son of the household, and, at a year older than her, her best friend. He always
	had a treat for her, even during exams when he could not get away to play:
	pithes in January, moas in February, lychees in April, mango slices in June and
	July, palm fritters in August, and sweets of all sorts during the holidays. As
	he grew older, he would read her stories from his reader, and show off the
	prizes he won at school. He got angry if she ever didn't understand why a particular
	competition was special, but he never pulled her hair or tattled on her, or any of
	the other things that little boys do to little girls. And they always made up in a 
	few days. Never, not even in the three short days every month that
	she got to spend at home with her parents, did she enjoy the attention she got
	from Tapas that she had when they were able to spend time together. 
	
	Oh, they got
	in trouble too. Tapas had taught her how to scale the wall and swing herself
	onto the smaller branches of the neighbor's guava tree to pick the fruit that he
	was too big to get at. That kept them apart for an entire month, and had Kalpana
	given to the care of the cook to be taught a lesson. They were more circumspect
	after that, with Tapas taking the blame for any missing food or damaged
	property. 
	
	Then one day, soon after Tapas's voice had change, he found her alone churning butter. 
	He kissed her and rubbed himself against her until he was gasping for air.
	
	Kalpana stopped suddenly. What had compelled her to say that? She had been
	dismissed two days later, and married before the year was out. But she had told
	no one, not even her parents of what Tapas had done. She looked up tentatively
	at her host's face to see if she had offended him. 
	
	He gazed back at her, compassionate and serene as ever. "Were you ever angry
	about what happened to you?"
	
	Kalpana was taken aback. It had never occurred to her to be angry. Who should she
	be angry at? Tapas? His family? Her parents? Even now, the questions unsettled her
	so she put them aside. They were so far above her. She didn't want to consider that 
	they had done anything wrong. 
	
	What purpose would anger possibly serve? The guilt had weighed heavily enough on her.
	It had made the first few years of her marriage difficult, unquestionably, but that was
	to be endured. Her husband had been patient with her, attributing her reluctance
	to her youth and an age difference of over ten years. But her mother-in-law was
	less patient. Why should they feed an extra mouth if she could not produce a
	child. Her husband had advised her to not talk back and promised that he would
	handle her, though she never saw the fruits of his labors. Instead, she
	redoubled her efforts to make herself indispensable to the running of the
	household as a means of creating peace at home. After all, since it was her
	misfortune that she was not able provide in one manner, then she should try to
	do better in other ways. "Angry?" she asked out loud. "How could I be angry at
	what had been written for me by fate?"
	
	"You can be honest here, sister. You have nothing to hide from me" her host prodded.
	When Kalpana continued to look confused, he added "What Tapas did was wrong.
	Surely you've come to realize that, sister?" 
	
	Kalpana shook her head, suddenly and inexplicably fighting off tears. There was that
	question of blame again. She'd never considered blaming
	Tapas for what had happened in life. Indeed, she did her best not to
	think about that incident at all. What good could possibly come of that? Surely it was
	better to accept ones fate, forgive and move on. There
	was no one she could talk to about it, and even if there were, she could not
	imagine said person being able to touch Tapas or his family. It was better to
	leave people like Tapas alone. "That is not my place to say. When his time
	comes, I trust you will judge what he has done."
	
	Her host sighed and shifted his weight. Then he poured them both some more tea.
	"Very well. And then you got married. Did you continue your trade in your new
	home?"
	
	Kalpana blew on her tea to regain her composure. Her life had not been easy, by
	any means. She was an uneducated rural wife. How could anyone think otherwise?
	But her in-laws had not been bad people. They had only wanted the best for her
	and the family. "My father-in-law was not opposed to my working. He even helped
	me find several jobs when I was new to his household."
	
	"And did you want to be working as a domestic worker?"
	
	Kalpana took a large gulp of tea to swallow the lump that had suddenly welled in
	her throat. Her host had a way of asking questions that skinned her and laid her 
	bare. No. She had not wanted to work. In fact, she had cried and clasped her father-in-laws feet,
	begging not to be sent off to a stranger's house in a new village where she did
	not know anyone or even the road home. But she had been so young and foolish. Surely
	it had just been childish willfulness, and not worth admitting. "My husband's family 
	was poor. If I could work and earn money for us, then I would do it."
	
	Her host did not say a word, but something about his silence made her look up to
	meet his eyes. He look immeasurably sad. Disappointed, even. Kalpana looked down
	and rethought her words. She was being judged for placement in the afterlife.
	Whoever this man before her was, surely he could see into her thoughts and know
	when she was not being honest. Shame would be a bad reason to be reincarnated as
	a dalit or an animal when honesty could grant her a better position in the next
	life.
	
	"I did not want to return to domestic work," she tried again, hesitantly. "I did
	not feel comfortable working in other people's homes. Especially," she paused to
	gather courage to state what should have been obvious for any modest woman.
	"Especially if the mistress would not always be around." Her host handed her
	another moa. Kalpana took it, grateful for the distraction. She did not
	understand why it was suddenly so hard to talk about the simple facts of her
	life. "I had a series of positions, but none of them lasted more than six months
	or a year. Honestly, I was relieved when I became pregnant with Jagat and could
	stop."
	
	Her host gave a small grunt that sounded to Kalpana like approval. "And did your
	husband take care of you then?"
	
	"My husband was a kind, gentle man," Kalpana said, thinking about his pride when
	she finally was able to give him a child. They had named him Jagat, because that
	is what the child was to his father, his entire world. There wasn't a day that
	passed that his father didn't find time to spend with Jagat or his brothers, no
	matter how tired he was, or how far he had traveled to find work that day.
	
	Her host smiled. "You grew to love him."
	
	"I did," Kalpana admitted proudly. Then hesitated. She had not answered her
	host's question. "And he did his best to care for me for as long as he lived.
	But he was not good with money." Always eager to join early in a get rich quick
	scheme, there were many times when, towards the end of the month, she found
	herself staring at an empty rice sack, and turned to finding creative sources of
	food to feed her family. It was not as if her husband gambled, or spent all his
	money on tea with his friends. He just had bad luck in investments.
	
	"And did you never try to persuade him to put his money elsewhere?"
	
	"Me?" Kalpana was genuinely surprised. "I know nothing about investments. If I
	got involved, I would only have made things worse. No, it was better that he
	make all the financial decisions." She had certainly wished, while he was alive,
	that she had been married to a man with better financial sense. And their
	marriage certainly had frequent arguments over money. But at the end of the day,
	she was married to the man she was married to, and she had to make the best of
	it.
	
	"So you forgave him his financial mistakes." There seemed to Kalpana that there
	was the slightest hint of criticism in her host's question. It made her
	defensive.
	
	"There was nothing to forgive. It was his place to decide how to spend the
	household funds. He did as he saw best."
	
	A look of surprise flickered across her host's face at her declaration, and he
	quickly changed the subject. "Tell me about your daughter."
	
	"Tulsi," Kalpana smiled at the thought of her only girl. If Jagat, Ajit and
	Karthik were the apples of her husband's eye, Tulsi had been the apple of hers.
	A quiet, cheerful child, she was always at her mother's side, gardening,
	cooking, sweeping, gathering greens and roots to supplement meals, whatever work
	was needed around the house. Kalpana's mother-in-law had never fully overcome
	her initial dislike of her, so it was a great comfort to her to have a companion
	throughout the day. Tulsi was a quick learner, and by the time she was ten, she
	brought in enough for the household that her grandmother started calling her
	their family's Lakshmi. It was at about this age that Kalpana's sister had 
	found an apprenticeship for her. She had neighbor who ran a successful sewing 
	business. She was looking for intelligent
	dependable girls to train and help her business grow. The training and the board
	would be free, they would just have to pay a small amount of money for food and
	incidentals. Best of all, Tulsi's aunt would be right there to look in on her regularly.
	She'd give her all the care a little girl needs when living in a stranger's home. 
	
	Kalpana had cried at first, not wanting to loose the her only companion. 
	Tulsi was so young and her sister's village was almost a day's travel
	away. If she were fully honest with herself, she was scared. She knew what it was
	like for a young girl to live away from her parents, even with an aunt in the same
	village. But her sister convinced her that this was a chance for her daughter to
	learn a trade that was easier and potentially better paying than any trade they
	had been able to learn as children, and so after much soul searching and innumerable 
	promises of regular check ins, she brought up the prospect to her husband.
	
	Absolutely not, he had declared. They could not pay to educate their three sons,
	how could she even think of educating their daughter? They had stayed up late into
	the night arguing that day. The seamstress was asking for so much less than any other
	apprenticeship program they would ever find for any of their children. A chance like
	this would only fall in their laps once a lifetime, why should they pass it up
	just because it came for their daughter? A girl trained in a real skill would
	make a valuable bride. They would be able to marry her off to a much better
	family than they could ever dream of otherwise. That, surely, was worth
	something. The amount of money was little enough, that she could make it up if
	she went to work the harvests with him every year. The rest, they could borrow
	from his father, and pay it back when Tulsi started earning. Eventually, she
	wore her husband down. "We won't have to borrow money from Father. My friend,
	Bikash's cousin is investing in a private banking pool. If we give them 100
	rupees now, he will give us 120 next month. Bikash has been making 50 rupees a
	month from this for the past three months. Let me put into this pool. We can
	easily pay for her board out of the profits." 
	
	So it was settled. But her husband did not put in 100 rupees. He put in 500
	rupees, which included all of her savings for the lean months before the
	harvest. Bikash's cousin payed out for three months before he had a cow fall ill and pulled out of the endeavor. They never saw the money again. 
	
	Tulsi came home
	during the threshing period, and two months later, during the lean dry months
	between harvests, her husband approached her with an offer from a man he had met
	in the market. He was looking for girls to train as hostesses for large
	households in Delhi. And, unlike the position that \emph{she} had secured for
	Tulsi, this one did not require paying for board for the girl. Instead, this man
	would pay them 700 rupees to take her off their hands. 
	
	Kalpana had not wanted to agree. She knew, even if she could not tell anyone,
	what hostessing in a large household would likely involve, and she did not want
	to expose her pride and joy to \emph{that}. But she went home and sent her
	daughter to bed with an empty belly, because, in spite of their best efforts,
	they could not gather enough food to keep the family fed. "At least she will go
	to bed with a full belly," her husband had argued, and after a long sleepless
	night, she agreed to her husband's plan.
	
	"Do you miss her?" her host asked, and Kalpana jumped. Lost in the memories of her favorite child, she had forgotten where she was. How old would her daughter be now, 45? 50? Old enough to be a mother of grown children, if not a grandmother herself. She missed her daughter so much that she had carefully arranged her days in life not thinking about her. Otherwise, she would have been lost, and nothing could have lead her back to her responsibilities for her remaining children. 
	
	"If I have sinned, ever, it was in sending her away like that," she said, and took some
	time to dry her face with a corner for her sari. She regretted her decision
	every day of her life, but she still did not see how she could have made a
	different choice.
	
	Her host waited patiently for Kalpana to pull herself together, then asked, "And your husband? Did this ever come between you?"
	
	Kalpana stared at her hands for a very long time. They had both grieved for the
	loss of their daughter, but in very different ways. She had missed her constant companionship, and feared for her safety. Her husband had been pleased with the
	money and the security it brought their sons while she had taught herself to put away her self recrimination and her grief. When the holidays came, six months later, Kalpana had relearned to love her remaining children, her sons. Her husband, on the other hand, seemed to be suddenly swallowed by the lack of his daughter's footsteps in the house. She never blamed him, at least not to his face. But it had been \emph{very} difficult to
	bear his grief. If only he had been more responsible with money. If only he had listened to her more. But there was no point in blaming the man. What was done was done. Her husband had died years ago, and Tulsi, well, she hadn't seen Tulsi since. "The loss of a child is hard. Surely, you know that. The grief came to us both in different ways. It did not break us, though."
	
	Her host let out a long resigned sigh, then stood, stretching his limb. "Very
	well, sister. Thank you for your honesty. If you will follow me, I will show you	the road where you may meet Krishna and be reborn."
	
	Kalpana froze. It was not that she did not wish to meet Lord Krishna and gaze
	upon his beautiful face. She was afraid of what it meant for her incarnation
	that her interview stopped at such a low point in her life. Surely, she had done so much more afterwards to redeem herself from this one mistake.
	
	"Are you certain? You had said that I had seventy two years worth of stories. Do
	you not want to hear of how I had taken care of my husband and his parents in
	their times of illness?"
	
	Her host smiled at her warmly. "I am certain, my dear sister, that with
	everything you have told me thus far, that you put aside any thoughts of wrongs they may have done you in their lifetimes, and given them all the care and respect that was in your power to give."
	
	It was meant kindly, Kalpana could tell, but there was something in his words
	that made her worry. 
	
	"And do you not want to hear about my sons? Or my grandchildren?"
	
	"Your sons are healthy and strong young men. You divided what little lands and	gold you had equally among all three of them, which is more than what many families do. I have seen them grieve at your pyre. I know genuine grief from that put on for the sake of propriety, or the type of grief that is mixed with the remorse for a source of labor now gone. That alone speaks more of how you sacrificed for them, even when they
	did not deserve it, than any words of yours could convey."
	
	Again, there was something about his choice of words that made her uneasy. He
	was praising her for doing exactly what she had been taught all her life that
	she should have done. Surely, he would not do that if he meant her ill. And yet. She was afraid.
	
	At a loss for anything else to offer from her life, she rose, and followed her	host to the door. He pointed out the road that she should take by turning left past his gate. Kalpana thanked him, unsure of how to take leave of a man she suspected of being a deity, but also not a member of her pantheon. She quickly bent down and touched his feet in parting. His feet, as well, she noticed, bore the same marks of impalement as his hands. If this was a god, why had he allowed this to happen to him?
	
	"Travel safely, sister," her host said, in blessing, as if he knew the ritual. "I wish you a pleasant
	next life."
	
	Kalpana stood and pulled her sari to cover her head, suddenly afraid to leave the safety and familiarity of this strange earthen hut to venture into the afterlife. Her host, however, was ushering her to his gate. 
	
	She put her hand on the latch, but found that she couldn't find the courage to open it. "Prabhu," she said, suddenly turning around. Whether or not this man was
	actually a diety, the honorific would not hurt. "Will you send me back as a
	lower caste? I can take being poor again. I'm used to that. But please. Leave me my dignity."
	
	Her host smiled his sad, patient smile. "Do not worry, sister. You will keep your dignity."
	
	\begin{comment}
	************************************
	
	Ben Yosef watched the young woman take the path beyond the gate, turn left as instructed and disappear behind a copse of trees. Then he let out a long slow breath and rubbed his face with his hands. Evaluation days were always hard. But then again, that was the point. This was his punishment for meddling where he did not belong, and he would have to take his medicine until he could figure out exactly what atonement They wanted from him.
	
	He shrugged on an air of composure and walked into the back room to check if there was another soul awaiting him. The worst part about these evaluations were their sheer randomness of their arrival. Sometimes, he'd go days without a soul visiting him. Sometimes, he'd get multiple a day. Once, he'd gone forty seven days without a visit. He'd almost convinced himself that he was free of this purgatory, when he got sent eleven souls in three days. He could not drag himself out of bed for days after that.
	
	There was no one behind the partition. Ben Yosef lent on the flimsy screen and let his shoulders sag in relief. He let out a long slow breath, counted to ten, then returned to the front room to begin cleaning up the remains of the interview. 
	
	He moved briskly and precisely, focusing on the next step in the ritual to let his emotions sort themselves out in the background. The thin carpets they had sat and eaten one were dusted, then rolled, then carefully tied and put on a high shelf. The food wrappers and dishes were neatly piled outside the door. The entire floor swept of crumbs to keep away the menace of ants. Then he picked up the pile of detritus and sorted it as he walked. Food wrappers went to the chickens. Dishes he took to the well. He liked to have the water level low after interviews. The extra exercise of pulling up a heavy bucket was good to clear one's head. Untie the knot in the thick rope holding the empty bucket. One, two, three turns to loosen the rope, then lower the bucket slowly until he heard the distant slap of wood against water. Fill then pull with long slow pulls, letting the strong muscles of his back work through the tension of the morning. These interview mornings made him feel so guilty. The suffering of these innocent souls, all due to some bastardization of words that he had once foolishly said. Pull. But that was the point, wasn't it. To punish him for having interfered with humanity in the first place. Pull. If the lesson was that he should not have interfered, and that was his interpretation of Their will, then how better to teach it than to rub his face in the harm he'd caused. He hadn't meant for any of these effects to come of his brief period guiding the footsteps of men. But that had been the argument of a younger him. Now he accepted the unintended harms and struggled to find a suitable means to atone to Them. Pull. But the interview were vicious. All the human suffering, the very human suffering her had wanted to ease. To think at all that suffering was due to him. It was too much to ... Pull. There was no point in self pity. He'd tried that route centuries ago. The only thing to do now was to figure out how to fix the damage that he'd done and then, how to convince Them that They should trust him. With a final pull, the bucket emerged from the well. He reached over and eased it to the ground next to his pile of dishes. How did he know that asking to return to fix his errors wouldn't just enrage Them even more. He didn't. But he didn't have a better idea.
	
	This had become the routine of his days now. Check if he had a visitor, bring them food and drink, question them, find a way to pass the time until the next visitor. He didn't have to prepare the food, or physically clean up after anyone. He didn't have to \emph{do} anything here other than evaluate the souls sent to him, and pass them off to the afterlife appropriate to their beliefs. So, for the first twelve or fifteen hundred years, that's all he did. He whiled away the hours between visitation however he wished, drifting from one fancy to the next, hoping to find comfort in something, anything, else. Recently, he'd stumbled up on the idea of structured days and physical labor bringing relief to his torment. Ironically, he'd learned about the idea from one of his visiting souls. He'd played with his routine for the last few centuries, experimenting and slowly improving it with time. It seem, at least so far, to work.
	
	The pan where the jute leaves had been fried had been put over an overly smoky fire. He rose and dried his hands, then walked over to the hearth to get some ashy dirt and hay to work at the oily soot coating the bottom. He had also learned from a series of his interviews that the enlightened of the current age no longer considered this women's work. And therefore, he resolved that he would not either. Perhaps if he could show Them that he had grown, that he was no longer the being that had made those mistakes, They would release him. 
	
	The steady scrubbing motion reminded him of the life his morning's interview had lived. He had made some real mistake, hadn't he, his conscience, aided by Kalpana's memory, derided. "The meek shall inherit the earth?" He'd said that. And now that was a weapon used by the powerful to keep the very social orders he'd spoken out against in place. He hadn't meant for it to be used this way, of course he hadn't, but he had said it. He'd also said "Turn the other cheek," of course he had. He had been hurt and young and searching for answers himself. And now, in response, Kalpana turns and turns and turns until she cannot find herself enough to protect the very things she loves the most. Forgiveness. Yes, he had taught forgiveness. It had become, he had come to learn many centuries into his imprisonment, synonymous with his legacy. He had taught forgiveness in the face of the most horrific crimes against your person and community. And so, the meek, Kalpana and her peers, forgive without question. Forgive without thinking that the act to be forgiven may have been wrong. It was foolish, and extreme, and a reprehensible lesson to teach. He wished he could say that in his youth, he had not meant for people to overlook the crimes themselves in order to forgive. He truly did want to claim that his younger self had that wisdom. But he wasn't certain that it was true.
	
	Ben Yosef splashed a last handful of water on the clean dishes and rose in disgust. He'd been so young when he'd come down as the manifestation of love, the type that holds communities, families, strangers together. He'd been so hurt, and so naive. He would never have done things that way now. He was older, wiser, chastened.
	
	He carefully dried the cookware and arranged it neatly on the shelves in the front room. He hadn't always lived in a thatch and earth hut. When They had first put him in this prison, when They had put them all in this prison, They had not stripped them completely of their powers. For instance, they could still somewhat mold reality to their liking. So, for a long time, he eschewed the aestheticism he has adopted as a Teacher of Humanity, and spent his days living with whatever comforts he could imagine. Sulking, really, surrounding himself by the comforts he learned of from his visitors. 
	
	Within the first few weeks of arriving at this prison, he'd figured out that he had neighbors scattered around him, other celestial beings who had chosen to interfere against Their wished. Someone, Mercury, perhaps, he didn't remember, had started a mapping project, just to understand where he'd been imprisoned. Celestial beings were scattered across a vast river valley, each house within a kilometer to two of their nearest neighbor. Try as they might, the inmates could find no pattern to who was neighbored with whom. No pantheonic, regional, or theological relationship was respected. Just a random scattering of every celestial being anyone had ever heard of, and several pantheons that had never encountered any other belief system, in a river valley that seemed to stretch on forever.   
	
	He'd tried to escape, of course. Who wouldn't want to return to the realm where they could thrive off the prayers and offerings of their followers, where one had the power to perform miracles, or give blessings, or grant boons. Who would want to be trapped in a never changing world, powerless to reach one's believers, unable, even, to distinctly hear their prayers. For the first ten thousand souls or so, he had joined forces with his neighbors to try to find creative ways to escape this prison. They'd quickly figured out that no matter how far he walked in any direction, as soon as he had a visitor, they'd each find themselves at home. He'd watched companions disappear from before him mid sentence, mid stride, mid breath. And he'd learned from watching neighbors suffer, what happened to those who did not conduct the interviews as They desired. It didn't take long for him to decide that it was better to just comply. It took decades for them to learn that any method they could come up with would inevitably end with pain. Or illness. Or a new means of torture that made Hades' creativity feel like the the imaginings of a school boy. So one by one, avatars left the bands of rebels, he left too, choosing instead to live quiet solitary live and figure out how to get by.  
	
	For the next thirty or forty thousand souls, he had railed, alone, against Them for Their injustice. Granted, the reach of the Roman empire was vast, and vaster was the reach of the churches missionaries. But why were They sending him souls from across oceans unknown to any of his followers? Denizens of entire continents hitherto undiscovered? He would perform the interview, to avoid punishment, but he absolutely would not be held responsible for their actions. Go blame their own gods for their failing. He wanted nothing to do with their flaws. The guilt built up in him though, and breached the walls of anger he had built around himself. Their will was just too strong, and eternity an awfully long time to spend in a sulk.
	
	 So for the next sixty thousand or so souls, he wallowed in self pity. He had never seen a being leave this place. He'd seen many loose their followers to conquest, plague, conversion or natural disaster. He'd watched those beings had shrivel up and waste away, but he'd never seen anyone leave. Eventually, he started to pray. Yes, he prayed to Them, for who else was there to pray to? He prayed that his followers too would die out so that he too could enjoy the sweet relief of oblivion. But They were not a kind god who granted the wished of Their devout. He prayed, yes. But the more he prayed, the more numerous his followers became.
	 
	 For the next fifty thousand souls, when it became clear that his followers were not going to dwindle any time soon, he became belligerent. Cursing himself, his followers, Them for their fickle sense of justice, the very stars in the sky for his fate. For the first and only time in his life, he broke from his nature and launched head first into hedonism. Women, men, feasts, wine, gambling, mind altering substances, anything to take his mind off the reality of the next inevitable visit. He drew the line at joining the group of inmates to entertained themselves by finding new and creative ways to die, knowing full well that they would arrive, hale and hearty in their abodes as soon as the next visitor arrived. That level of discomfort, he had decided, was just not appealing.
	
	Ben Yosef finished with his dishes, dusted out the bedding where visitors awoke, and walked slowly over to his workshop. He searched until he found a half log of cedar with a particularly large knot, reached for his hand plane, and got to work. For about the next sixty thousand souls, he tried to make friends with his other inmates. He'd known many of them from before of course. When they had first arrived in this purgatory, being sought out the other members of their pantheon. Perhaps mixing with other who occupied similar theological positions as themselves. By the time he chose to come out of his centuries long period of solitude, the old social structures were all askew. No one remembered who had been a minor god and who had ruled  the heavens. Animist and ancestral spirits went on walks, totemist and polytheist attended each other's feast days. Minor spirits held court and adjudicated differences between the leaders of different pantheons. All of the rules of the old social order were gone; replaced with bonds of proximity, and barring that, the ability to confide one's misgivings and make friends. During this time, Ben Yosef played cards with Ahura Mazda, Olorun and Pele. He met weekly with Indra for chess, until it was made too awkward due to his friendship with Shiva. Bula had started an amateur theater troupe, and he attended these performances, more as an opportunity to meet new faces, rather than the quality of the acting or the timber of the discourse. It was nice for a while to enjoy the company of his peers. It was pleasant to debate the metaphysics of Their will, to test his theories against other beings faced with the same conundrum. When it became popular to believe that They might be appeased if the prisoners collectively cast off their previous personas and adopted one more pleasing in Their eye, Ben Yosef eagerly cast aside his name from life and adopted one that referenced his mortal father on earth. It was far more comforting to mourn the passing of a being who had lost their followers with mutual friends than to mourn alone. When Bula passed into oblivion, everyone who had ever come to one of her plays came to her wake. It was a send of suited for a true queen. 
	
	But even the companionship of his peers grew stale after some time. With an eternity to pass in captivity, eventually, there were no new faces to be met, no new ideas to be discussed. 
	
	So now? Now he went back to his roots and meditated. He recreated his house to be a version of the house he had occupied when he walked among men. He went back to wood working. He chose the life of a peasant, leaning into the aestheticism and humility that had come to naturally to him his entire life. He used physical labor to structure to his days and waited for a solution to present itself.
	
	Ben Yosef paused to wipe sweat from his forehead, and looked up at the sun to gauge the time. The day was already quite warm, and it was still mid morning. He could hear the muted chanting of Terce in the distance, but not the joyous muffled song of Protestant worship or the faint chatter of penance after confession. It was not yet Sunday. 
	
	He went back to work. The cedar felt good beneath his hands. The knot was large and difficult, and required all his attention. That was exactly the point. As so many of his peers had once taught, if he could focus his entire attention on something, anything, enlightenment would come.
	
	...
	
	"Ben Yosef," The plane skipped at the sudden intrusion on this meditations. The familiar voice was not enlightenment, but it was welcome all the same. 
	
	"Ben Yosef, are you with a vistor?" It called again, urgently. "Please tell me you have time for me. I come bearing gifts." The last line was said in a sing songy coquetish manner that completely belied the tall fierce buxom four armed woman that spoke it. She stood on tiptoe at his gate peering around looking for him, as if the extra inches would make it possible to look over his roof to find him in his workshop. In two of her blue black arms, she held an amphora large enough to keep a family in oil for a year. One rested on the latch of his gate and the last shielded her eyes from the late morning sun. Her thick tightly curled hair normally splayed like a fan behind her, but today, she came to him unkempt, and in the morning heat it wrapped itself wetly around her arms and shoulders like a nest of snakes.
	
	As ridiculous as she looked in that posture, his breath caught in his throat. How could it not. She was Nyaya, once the only love in his life and arguably, the start of all his troubles. For the first several hundred years of imprisonment, they had stayed away from each other, both reeling from the disastrous end of their time together. Eventually, they formed a cautious friendship, trusting in the comfort of the other's company, based entirely on the memory of the intimacy they had once enjoyed. At least that was why he sought her out on his darkest days. He'd rather be mocked for his stupidity by someone he had once loved than face the sympathy of a stranger.
		
	She stood a head taller than him, and was naked, as always, from the waist up, but for a garland of one hundred and eight skulls. It wasn't an adornment she'd worn when he had first met her. Rather it was a change to her appearance she had adopted when she had followed the trend among the inmates and adopted an atonement persona. They clattered when she danced, he recalled. He'd seen her perform many times at Bula's shows. He also recalled many pleasant evenings, when they were both free, taking the original garland off of her chest, singing her praises at each heads like rosary beads while she ... Well, that was a long time ago, and things were different now.
	
	By the time he had walked up to the gate, he put away all thoughts of past affairs. They were simply friends now, and had been for over a millenia. "Nyaya," he said warmly, "I'm free for now. What's this all of a sudden?"
	
	His guest patted the amphora resting against her hip. "I found this buried in a river bank while taking a walk. I thought we could open it together?" Before he could answer, Nyaya let herself in, handed him the clay container and ducked past him into the house to find serving utensils. 
	
	Ben Yosef worked at the wax seal. It was too clean and fresh to have been discovered in an old forgotten ruin. Nyaya was a terrible liar, but he wouldn't call her on it. They had ... different senses of what was right or wrong. Sometimes, it was best to just let things be. Instead, he wiped the soot off the cooking pit, placed the amphora in it, and set two places in the shadow of the house. "Do you know if this is ambrosia or amrita?" he called out, playing along.
	
	"I'll take either," Nyaya replied cheerfully, coming out with two brass cups and a serving bowl. "But I hope it is ambrosia. Amrita is too sweet." She sat down and handed him the equipment with a smile. "Will you do the honors?"
	
	He reached in with the bowl and lifted the dark liquid to his nose. It was, indeed, ambrosia. He poured her a full glass and himself only half. Aside from that single dark period in his life, he had never been one for excessive drinking. Certainly after that dark period of his life, he tried not to judge others for their choices. He just preferred not to over indulge.  Nyaya tisked in irritation, reached over him to grab the bowl and topped up his glass.
		
	"You know you don't actually need to drink this to become intoxicated, right?" he asked irritably. She had always been ... more ... than him. Larger, stronger, more passionate. Quicker to anger, faster to act. But she had never imposed it on him. They were such different personalities. They could never have shared what they had shared with each other if they didn't know how to meet the other where they stood.
	
	"Absolutely," she said with cheerful sarcasm, touching her glass to his. "Just as you know that you can drink this entire amphora by yourself and stay sober. Just drink with me." 
	
	She was manic, he realized. Something had upset her, that much was clear. The best way to understand her was to wait until she was ready to talk. So he drank. The ambrosia was a good vintage, a pleasure to drink. He just made certain that he did not get drunk.
	
	They talked of her garden for the first glass -- her cucumbers and tomatoes were growing wonderfully, but she was constantly picking eggs off her cabbages, so she expected a poor winter harvest. For the second glass, they discussed his goats -- three of his does had given birth to kids so he had more milk than he could possibly go through and had turned to cheese making. Nyaya promised to come by with some bread an honey to help him consume the extra when it was fully cured. For most of the third glass, she talked about the walk she had been on when she had supposedly discovered the amphora, slurring her words more and more, as she struggled to keep up with her web of fiction. "You had a visitor," he interrupted. It was a guess, but he had to try something. It was painful to see her so upset.
	
	Nyaya finished off the rest of her cup in a single large swallow, placed it firmly on the ground between them and said without making eye contact, "I most certainly did." She reached for the bowl and poured them both a fourth cup. "And I most certainly do not want to talk about it." 
	
	He watched her take a large swallow, but kept his own in his hands. "I had a visitor," he began, then realized that it felt good to talk. "She was one of yours." He briefly outlined what had transpired in the morning, trying to paint the whole woman, and not only on the stumbling blocks she face, and then paused, realizing how her difficulties just dominated her story. "I don't understand how this works." he said finally. "You are the embodiment of protection. A mother's wrath that her children have been hurt, magnified to celestial proportions. My guest grew up with your songs on her lips. And yet," he trailed off, and ran his finger around the rim of his cup. "And yet she allowed her beloved daughter to be sold off for a mouthful of food." Ben Yosef burried his head in his hands, fighting off the waves of guilt. If Their message truly was that this was his fault, and it seemed to be, then he had to find a way to fix this. He couldn't take an eternity of these earthly visitors.
	
	"Sweet Ben Yosef" his companion said after some time, rubbing his back. "How are you still so naive? This has nothing to do with what we did during our brief time on earth. The humans were Their creation, we interfered, and They got mad. That's all there is too it. This has nothing to do with justice or teaching us a lesson.  They are arbitrary and capricious. It has everything to do with punishment and cruelty." She hawked up a large wad of phlegm and spat.
	
	Ben Yosef startled at the vehemence with which the spittle hit the dirt. He struggled with the impulse to correct his guest, warn her that They might be watching. Not because his fear wasn't real. It was. It absolutely stomach churningly was. More because she would taunt him for still being afraid. "You're so vain, little man," she had said once, centuries ago, when he'd tried to argue with her. "You think They care enough about you to watch you suffer? I bet They've thrown away the key and forgotten about us eons ago." Her perspective had stunned him. It was so bleakly cynical and hopeless. Over the decades, he's had no indication that she'd changed her mind. So rather than row with her, he stayed silent and waited for her to speak.
	
	His companion glowered silently at the space between her knees for an uncomfortably long time. "She was one of yours, I suppose, though by her own admission, she wasn't a believer. She was so ... angry. So. Justifiably. Angry," she pounded her fist against her knees at each of the words, then paused, searching for how to proceed. "Angry about the sexual abuse in her childhood, angry about her sister's controlling and manipulative husband, angry about being type cast and trapped in a demeaning and stagnant career, angry about not being able to provide her children and grandchildren with the education that could release them from the type of life she was forced to live. She was so angry." Nyaya threw up her hands in sympathetic rage. "And so ... impotent."
	
	When it was clear that she was done, he placed one of her coffee colored hand in both of his. She let him take it, but made no other sign of emerging from her frustrated reverie. Their hands felt comfortable and familiar together, both the exact same size in spite of their difference in stature. They'd been good together once. Even now, her simple presence made it easier to tease apart the details of what he had seen this morning, and the pain of the story she had just shared. A new possibility started forming in his mind. "Our guests were so similar today," he began "The struggles they had to overcome were different, I'm sure, in the details and minutia of their lives. But fundamentally, they are the same. I think there is a lesson here for us."
	
	But Nyaya flinched and pulled her hand sharply out of his walnut colored clasp. "Stop it, Ben Yosef," she said, and glared at him. "Stop acting like these people are interchangeable pawns in some game put together for some greater purpose." She got up and started pacing across the yard. "She has a name, my visitor. Her name is Susan. Susan Brown. She was born just late enough to be allowed to attend an integrated school, and had lived for nearly seventy years outside of Jackson, Mississippi. She had an entire life that is outside of the short story that we are forced to extract from them. She liked crocheting, and taking her grandkids to amusement parks. The ladies at her church placed bets among each other for what color she would dye her snow white hair the next month. The first thing she did when she got a smart phone was to figure out how to play scrabble with her daughter who lived in New York. She is so much more than just her anger. As I am certain your visitor, who has a name."
	
	When it became obvious that she was waiting for his input, he supplied "Kalpana Maji."
	
	"Just as Kalpana Maji is more than her submissive nature."
	
	Ben Yosef waited for Nyaya's rage to abate. As an avatar of retribution, these bursts of outrage did not surprise him. And when she cooled, more often than not, there was wisdom to be gleaned from her outburst. He had not, for example, extracted these details from his guest today. He was not in the habit of finding out more about his souls than he strictly had to. The interview were hard enough without granting his guests depth and humanity. Is that what Nyaya did? Why would she do that. She was simply torturing herself. "You are right," he said at last. "I robbed Kalpana of her humanity. But I can't ask for more detail about their lives. It would tear me apart." He let his words sink in, and braced himself for her indignation.
	
	Instead, Nyaya took in a deep breath and let it out in a low shuddering sob. "Do you know what she said to me? She asked if, regardless of where I sent her, Heaven or Hell, I would accept her prayers and protect her grandbabies. ... The woman had not heard my name until today, she had not believed in you for who knows how long, and there she was, on her knees, begging for my permission to pray to me! Begging. ... As if I have any power to do anything to protect her family." 
	
	Nyaya broke down completely in the middle of his yard. Ben Yosef understood. As different and varied the inmates of this prison were in size, shape, occupation or species, they all had one thing in common. They were all beings that fed off of human prayer and belief. Prayer, belief and the granting of those prayers were at the core of who they had Become when they first set foot on earth. That was why, when followers died, the beings shriveled and died as well. When They had constructed this prison, it must have been absolutely clear to Them that They could not cut them off from their followers entirely. That would shorten all of their lives, and thus, presumably, defeat the point of the exercise. Instead, They had allowed each of them to be aware of, but not hear, the prayers and beliefs of their followers, while robbing them all of their power to \emph{do} anything about it. It was like seating a hungry man at a feast and not allowing him to eat.
	
	Everyone coped with the struggle in their own way, just as each being modified and experimented with their coping mechanisms over time. For now, he had accepted the constant gnawing pain as part of the process, and did his best to background it while he tried to find a way forward. Nyaya, he knew, relied on her physical strength and endurance. She saw it as a challenge not to give in to the pressure. But a direct request to hear a prayer? That would break the strongest resolve.
	
	Ben Yosef rose and put a hand on her shoulder. Nyaya grabbed him with all four arms and buried her face in his hair and wept.
		
	When she had exhausted her grief enough that her breathing grew more steady, he wiped her tears and mucus off her face with the heel of his hand. Then he allowed for just enough of the ambrosia's intoxication to affect him to find his courage. They were both reeling from this morning's interviews. They both drew strength from the other's company. There was no reason for them to suffer alone. He put both his hands on her cheeks and guided her face down to kiss her.
	
	When she didn't resist, he kissed her again, slipping his tongue into her open lips. She shifted her position to give him better access. 
	
	"Nyaya," he said, when they had finished. "We don't have to do this alone. We would make a good team."
	
	She stiffened and moved out from under him. "Stop. I can't take this now. It didn't work last time, and it won't work now."
	
	Her words stung, and they weren't fair. They had been apart for two millenia, and when they had been together, it had been better than anything he could have dreamed of. The reason for the failure of their relationship had nothing to do with the two of them. He checked that he could respond evenly before saying, "It didn't work last time because your husband found out. Who cares if he knows now. It's not as if he can do anything about it."
	
	Nyaya lifted her head to the sky and laughed. A loud sharp bitter syllable. "It's not him I'm worried about." She looked down at him, and he steeled himself for her criticism. Her face softened. "You are a sweet, kind, gentle creature. What will \emph{you} do this time when we can't make this work?"
	
	His heart shattered at the precision of the attack. She was right. When Shiva had found them together last time he had flown into a rage perfectly befitting a destroyer of universes. It had been the one and only time he had felt fear that visceral. Not even when They had manifest Themselves before him in a storm of fire to condemn him for interfering with his creation, had Ben Yosef wanted the earth to swallow him whole, as he had when Shiva had found him reading a note from his wife in a grove near their house. Shiva had danced. Ben Yosef shuddered to remember the dance. The beat of the damaru, the sway of the cobra, the three points of the trident coming straight for his eyes. He was convinced that he would be impaled, poisoned and have his life force shattered by the destructive beat of the drum, simultaneous, and repeatedly until the cuckolded husband had satisfied his desire for revenge.
	
	But Shiva had let him live. Instead, he locked his wife away from the entire world. Not only would Ben Yosef not be able to see her, even if he had the courage to come within a mile of Shiva's home, but no one would be able to see her. For the next months and years, Ben Yosef imagined the untold tortures that Shiva was inflicting upon his wife, a woman he did not love, but who Ben Yosef revered. It was enough to drive anyone mad.
	
	Ben Yosef's heart had broken. He was not eating, he was not sleeping, he had stopped seeing his friends. He needed an antidote to the pain and the humiliation, and yes, the anger that he felt. And so, he Became. He visited humanity as the avatar of forgiveness, and taught the people to love and forgive each other, just as he needed to teach himself to love and forgive the being that had ripped his heart from his chest. It had been a stupid move. There was no question about it. He saw that now. He had been hurt, he hadn't been thinking straight, and he had been oh so very young. But he had grown, hadn't he? He was older now, older than she had been when they had first met. And he had seen more of the world now, not just of the earth and humanity, but also of this place. He wouldn't shatter so magnificently at the first heartbreak anymore. 
	
	"It'll be different now," he began, "we've both grown." He thought of the souls he had seen over the centuries and the lessons he had learned from them "I'll treat you differently this time. I'll respect you." Nyaya glared at him, both eyebrows raised skeptically. One pair of arms crossed over the chest, the other on her hips. It was an uncomfortable gaze, but at least she was listening. He took a deep breath and began again. "I've been thinking about what They may want from us. Why They keep sending us disfigured interpretations of who were were on earth. The cruelty cannot be Their only purpose." He paused, and when she gave him no reaction, continued. "I think I have a way through this. Hear me out. What if They want us to figure out a way to fix it? You and I, we are two sides of the same coin. If we were to join forces ..." 
	
	He was cut off by her raucous derisive laughter. He was used to her mockery and cynicism. Sarcasm, he had come to learn was how she expressed affection. But truly, there was a time, and there was a place. He threw up his hands in frustration. "Fine. Tell me how you really feel."
	
	Nyaya bit a knuckle of her forefinger and turned away to pull herself together. It took longer than he liked. "Sorry. I shouldn't have laughed at you. I'm not upset at you." Her eyebrows quirked ruefully and looked down "I'm not mad at Shiva either, I supposed." Then she met his gaze. "But seriously? Two sides of the same coin? You want me to do THAT again? Shiva and Sakti? Really?"
	
	"No. Justice and Forgiveness. This will be different." 
	
	She laughed again. A single sharp bitter syllable, but softer now. "You just called me a mother's rage. That's a far cry from justice."
	
	He hadn't said that, had he? Oh goodness. He most certainly had. He turned away in shame. 
	
	"Ben Yosef," Nyaya whispered. He didn't respond. He couldn't face her. He had just sworn to respect her and yet he couldn't see beyond the gender of her manifestation. What was wrong with him. "Ben Yosef," she said again, wrapping a pair of arms around his waist and running a third hand through his hair. "You're not wrong," she said softly. He still couldn't look at her. "As our best selves, we complement each other. But I can't be intertwined like that again." She kissed him on the top of his head. "I love you. I know that you have grown, and I know that it would be different this time. But I can't join with someone like that again. Not even with you."
	
	He bit his lips and fought back tears. He didn't want her forgiveness. That just made it worse. Her anger had been perfectly reasonable. He should have thought before he asked her to join forces with him like that. No one who had lived with Shiva for untold millenia should be asked to loose their identity like that again. He should have seen. He should have understood ... But the cruelty of this place. It was unbearable, and unending, and unpredictable and ... He sighed. It was just too much. It would rob anyone of their senses. He wasn't ready for a relationship. At least not one with her. What had he been thinking?
	
	Naya pulled away and called his name again, businesslike this time. "Can I ask you a favor?"
	
	He wiped his eyes, turned around to look at her, and startled. She had grown several feet taller and gained several arms. The skulls in her garland were no longer white and pristine like beads on a rosary, but fleshy and bloody severed heads as they had been when he had first met her. His heart jumped to his throat. She was terrifying. And magnificent. And ... beautiful.
	
	"I've been thinking about our internment as well," she began cautiously. "I know you and I don't see eye to eye on this. But, truly, I don't think They have a larger purpose. I don't think They ever did. He didn't like what we did, he got angry, he put us in here. That's it. The cruelty is all there is." She shrugged with all ten of her shoulders. "And it doesn't matter. Even if you are right, and They are waiting for us to discover the right form of atonement, They aren't being very responsive. The cruelty is still the point. So I've decided not to care." She clasped six of her arms demurely in front of her. "I haven't told anyone else yet. I wanted to run this by you first. I, I don't want to be Nyaya anymore. No one knew me as Nyaya in life, and adopting that name won't help me in here. I'm done being scared, and I'm done being angry. So I'm going back to Mahakali. Let Them do what They will, at least I can be myself."  He must have been staring, because she added, uncertainly, "Do you disapprove?"
	
	Ben Yosef shook his head abruptly. He disapproved. Of course he disapproved. Enormously and vehemently. Her decision, her presentation, her defiance terrified him. Surely it would bring retribution down upon her head and everyone who surrounded her. That was how They worked. But his feelings were  beside the point. "Of course not," he said out loud. To his surprise, Nyaya, no Mahakali, gave him a girlish blush and bent down to peck him on the cheek. 
	
	"Thank you," she said, and then "You've always been my better self. I could not have done this without you."
	
	...
	
	Late that night, alone with the dregs of the amphora, Ben Yosef thought about the events of the day. They had sat together drinking. Talking about the time before their imprisonment of their exploits on earth, and the period of their lives they had shared together. They'd drunk well into the afternoon, until Mahakali had lain down in the yard and gone to sleep. 
	
	Ben Yosef had left her there, and gone back to his workshop. The cedar log had lost its appeal, so he turned to the other tasks; curing planks, polishing handles, fitting together the forest marquetry for an end table. It was fine, skilled work that required care and delicacy. These were tasks he left for his moments of calm, or even, when he was happy.
	
	When the sun sank low in the sky, he went out to bring the goats home. As dusk fell, he lit a torch, and went to wake his guest up. She lay there, sweat and blood and dirt mixing to stain her body with mud. With her hair matted and snarled, still stinking of alcohol, she could have passed for a female version of Shiva. The thought gave him pause. How different the world could have been if they had met at a different point in time. It would have been nice, he concluded, but it didn't matter. She had staked her position about their relationship, and that was all there was. He shook her gently awake, and told her to sober up and go home, which she did.
	
	Her decision to revert to her old name still made him uncomfortable. Her defiance of Their superiority terrified him. He was convinced that it would hurt her, and he feared that it would harm him, by association. But he would stand by her side. At some point in their afternoon discussion, she had drunkenly declared that she did not think that any of their actions had ever had any influence at all in the course of mankind. Humanity, she had denounce, would have gotten to where they were without their help. He'd flinched at the pronouncement, hoping that They were not listening, but also wondered whether or not she had come to that conclusion while she'd been stone cold sober. She wasn't afraid anymore, yet he still was. He didn't know what to do with that.
	
	Yet she'd called him her better self, and that comforted him. They were friends, only, and that disappointed him, but they had been friends now for longer than she and Shiva had been intertwined. Well \emph{that} was an interesting realization, he thought, and took another gulp of ambrosia, spitting out the dregs that had drifted into his mouth. And she, in her own way, had forgiven Shiva for his torments, and perhaps even, she had forgiven Them. If an all powerful force could be forgiven. Maybe there was something of himself that had rubbed off onto her, just as a bit of Shiva had. He hoped so, and he hoped that he was right. That somewhere in the wide gap between their two personas was the way out of this hell.
	
	\end{comment}

\begin{center} ***** \end{center}

Nyaya hated her morning guest. But that was the point, wasn't it? To be reminded of everything she had been once, everything she should still be, but had set aside in order to get by in this wretched place. 

She raised an arm and beamed brightly at the poor woman as she looked back at her uncertainly at the end of the garden path. This was not a kind experience for them either, to be assigned an unknown deity at the end of their days to pass judgment on the most sensitive parts of their short lives. "Left at the end of the path," she called out, "Then just follow the main road. You cannot miss it." Susan paused with her hand on the wrought iron door. "Go," Nyaya whispered impatiently to the head full of grey braids that fell past the woman's shoulders. "You moved mountains when you were alive. Surely you can open my door." But Susan took her own sweet time finding the courage to meet her eternity.  She had raised her children, and her children's children, and their children through sheer force of will, but the afterlife, it seemed, was more than this woman could take. 

Nyaya waited in her foyer. It would be rude to close the door on her guest and turn away. That was the type of inhospitality that could set tongues wagging for years. Nyaya was not foolish enough to risk that sanction for something so small as a moment's frustration. But, mother's eye, she was taking her time. 

\begin{center} ***** \end{center}

Ben Yosef sat in the dark and prayed. He knew of nothing else that would ease the tourniquet of guilt tighting around his temples. 

He had stood at his gate while his morning's guest walked the two hundred yards or so to where the path took her behind a juniper hedge. When the pain threatened to darken his vision, he dug his hands into the fence post and reminded himself that it was his duty to bear her suffering. He would not fail her, her told himself. There was grace in suffering. Isn't that what they said in his name? 

But as soon as Kalpana had turned the corner, he doubled over in pain, voiding the undigested meal he had just shared near the chicken feed, and crawled into the cool darkness of his one room hut. He lay there breathing in the dark must of the floor, listening to the pounding of his heart and trying to control the wave of nausea that threatened to overtake him again. His head hurt. Someone was twisting a screw into his temples and he could not catch his breath.

Expecting this attack every day only made it worse. He knew, he knew that every morning would bring a guest to his house. A newly deceased soul to pass judgment on. He would meet another good, decent, moral human being who had been tricked, robbed, tortured, and taken advantage of for all of their life.  Saints, by any reasonable definition of the word, never fighting back or even getting angry at the horrible hand of cards life had dealt them. Most of them just kept their heads down and accepted their circumstances. Forgiving those who had wronged them, turning the other cheek. 

Ben Yosef swallowed a mouthful of bile. That wasn't what he had meant. He would deny that from now to the end of eternity. This was not what he had meant to teach.

Too tired to find his rosary, he summoned a new one and fingered the cool familiar, heavy beads. Something in the back of his mind awoke, a faint glimmer of hope or the shadow of the strength to endure. No, expecting the guest and the subsequent pain only made the anguish worse. And there was no way to avoid the guest. He had tried, oh he had tried over the centuries to be creative enough to escape their stories. But they would always find him. Every dawn. Without fail. Whatever he had done, or wherever he had spent the night. And he had promised to bear their suffering. 

No, there was nothing to do but to endure this punishment. This too would pass, he knew, it would pass. All he had to do was wait and pray. He fumbled the beads until he found the heavy wooden cross at the end and began the ritual.

Why had Kalpana never paused to question her husband. The man had made poor decision after bad. If she had pressed him, even once, how different would her life have been. This was not, surely, what he had meant when he had praised meekness over pride. He had also, he was certain, praised perseverance in the face of injustice. Surely, he had taught perseverance and persistence, and not just acceptance. Ben Yosef's chest ached and his breath caught in his throat. This would not do, he told himself. He could not let himself get swallowed by the maelstrom of his guilt. He fingered the first bead and prayed. 

``Our Ruler, who art in heaven, hallowed be Thy name; Thy kingdom come; Thy will be done. Here, as it is in heaven."

He did not know why They had thrown him into this prison, him and every other deity and spirit that humanity had ever worshiped. They were upset at them all, that much was clear, but no one knew what They wanted. 

"Give us this day our daily bread; and forgive us our trespasses as we forgive those who trespass against us."

They had tried everything in their power to appease Them. Prayers. Sacrifices. Self-abnegation. Glorious feast and magnificent monuments. Nothing got Their attention. Beings died in the millenia they had spent in this strange isolation. Others had manifest themselves and been sent to join their ranks. And still the visitations from the dead continued. 

"and lead us not into temptation, but deliver us from evil. Amen." ...

By the time he had finished his first decade of beads, he could breath normally again. The pain in his forehead had dulled to a low throb. His mind stopped racing frantically and the tidal wave of guilt that had threatened to drown him merely glowered at him, certain that he just had done something suspicious, convinced merely by the overwhelming evidence that he was in prison, not by anything more concrete.

They had been wrong, Al Jahil preached, and Ben Yosef agreed with him. Each and every one of them had all interfered with a race that They had created, taken credit, even, for Their creation. And They were not pleased. If the community wanted redemption, they had to show Them their repentance. Al Jahil's plan was simple. They had to meet the visitors without resentment, put aside their differences and come together as a community. It was simple and sensible, and Ben Yosef could not find fault with it. 

Al Jahil had pulled them out of the chaos and the confusion of their first few centuries in this place. He had organized the warring bands into a civilized village, centered around a central square that served initially as a tribunal in the early days, a place to settle differences without resorting to arms. Eventually a market place sprung out around the perimeter, then the court dissipated, leaving only a place for the community to exchange greetings and goods. 

Ben Yosef should go, he knew. Al Jahil exhorted everyone to attend the market on a daily basis: a show of good faith, evidence that they had turned away from the old ways. But the last vestiges of pain only subsided after five rounds of the rosary, and Ben Yosef was exhausted. He curled up on the reed mat in the eastern partition of his house and closed his eyes. Just a little nap, he vowed. He would clean up the dishes in the other room after he rested, he told himself. He would wake up in time to bring the goats in.


\begin{comment}
 There was nothing else to do in the sight of such inhuman human suffering. It was his duty to bear witness to the never ending stream of good, decent, moral people who had been tricked, robbed, tortured, and taken advantage of, who did not have the ability, each for their own very good reason, to protest. Each person he ever saw had spent their entire lives living solid, wholesome, humble obedient lives. Taking what was given to them with gratitude, never being unreasonably jealous, accepting their appalling circumstances with the dogged determination to live the next day. Kalpana was no different. Not once did he see a person who stole, or picked a fight, or even, got angry at the horrible hand of cards life had dealt them. They were all the types of people who smiled, and put their heads down, and simply did what needed to be done, no matter what conditions they had to work in. Saints, by any reasonable definition of the word. Every many woman and child of them. 

But it was not enough to know that a Kingdom of Heaven awaited many of them after their suffering on earth. Imprisoned as he was, forced to watch their suffering day in and day out, he wanted desperately to help these poor innocent souls. These people, who embodied everything that he had stood for, who seemed capable of forgiving anything. Whose capacity for forgiveness was so great that it seemed, almost, to be a fault. 

His forehead throbbed as if someone were piercing his temples with thorn. If he could only reach out and touch one life. The smallest token of gratitude for a job well done. The thinnest veil of protection for the people who walked in his footsteps. But that was out of the question, and it tore him apart. He would never be allowed to bless or aid or visit anyone again. Ever. Not anymore. Not after what he had so foolishly done. The sheer helplessness of his situation, the guilt, the humiliation of seeing the very thing he had once tried to do twisted into an instrument of torture. Day in, day out. It was too much, and it would drive him mad. He could not endure this torment any longer. 

Ben Yosef gasped for breath against his growing pain in his temples and stumbled around in the dark until he found his rosary. Immediately, he fell on his knees and prayed. ``Our Ruler, who art in heaven, hallowed be Thy name; Thy kingdom come; Thy will be done. Here, as it is in heaven." He had been wrong. He knew that now. He had been young, and foolish, and hurt and shortsighted. He should not have interfered. If Their point was to show him the harm that he had done in a moment of youthful indiscretion, he saw that now. He would never, ever, ever, not in a million millenia, teach like that again. If only he could be forgiven and released from this purgatory.

"Give us this day our daily bread; and forgive us our trespasses as we forgive those who trespass against us." Ben Yosef was weeping now. He could feel the hot tears of self-pity roll down his cheeks to fall onto the earthen floor where he knelt. It was too much to bear, to watch soul after suffering good soul pass before him, each with a capacity for forgiveness he had never dared imagine. And each, he now saw, had suffered, in part, because of it. No one would ever say to him, "This was your fault. I suffered because of you." That was not in their natures. But he knew it to be true. He knew that their suffering was the entire reason that he was in this prison. It was torture to have to witness their suffering, unable to do anything to affect it, unable to atone for his mistake. He wanted to fall at their feet and beg their forgiveness for what he had done. But that would only frighten them, he knew, and each one of them had already endured enough. So instead, he listened to them account for themselves and passed judgement, determining where they would go next. 

"and lead us not into temptation, but deliver us from evil. Amen." ...

By the time he had finished his first decade of beads, he could breath normally again. The pain in his forehead had dulled to a low throb. His mind stopped racing frantically and the tidal wave of guilt that had threatened to drown him merely glowered at him like a goaler certain that he just had done something suspicious, convinced merely by the overwhelming evidence that he was in prison, not by anything more concrete.

Slowly, tentatively, logic crept into the quietude under cover of the familiar repeated prayers. This suffering was his punishment, that much he knew for certain. They had come to him in a storm of fire and raged at him for having the temerity to interfere in Their work. He was in here because he has made a terrible mistake. He'd had the typical self assurance of youth, he'd thought himself worldly because of the breadth of his travels, never imagining the vast array experiences he had never had. He'd seen a people suffer under the tyranny of foreign rulers. More poignantly, he had seen her suffer, the only being he had ever loved, and he had not been able to do anything to help her. He'd needed an antidote, a salve for his own pain. In his youthful hubris, he'd walked among humans, teaching the very thing he needed to learn himself: forgiveness.

He'd been so naive, so unnuanced, so foolish, so wrong. He hadn't seen it then, but that was a long time ago. He'd been a different person then. He clearly saw it now. Their punishment had taught him that. He'd learnt. He'd changed. As They had demanded, he had watched, with both his eyes and his heart, the devastation his mistakes had wrought. Or at least, that was the common consensus among the rest of the prisoners, that They demanded each prisoner to bear witness to the horrific outcomes of their interference. The community agreed. The better they all watched, and the more they all showed their remorse and contrition, the sooner They would forgive them, and they could all be released.

But Ben Yosef, no, all the prisoners, had been collectively watching for so long, it seemed impossible that They had not noticed yet. Perhaps there had been too many defectors. It was hard, after all, to constantly bear witness to human suffering without being able to do anything about it. It was too tempting to harden one's heart and turn away. He'd fallen for that temptation once. About a millenium ago, he'd decided that being callous was the easier way out. Souls would come to him, and he would pretend to listen to their troubled lives. There wasn't much that he was missing, he'd heard so many stories over so many centuries. There was nothing new to hear. Then, after a couple hours of nodding along and asking a few meaningless questions, he'd arbitrarily assign his visitor to what ever afterlife suited his fancy that day. It had been so much easier to turn his face away and not submerge himself neck deep in a misery he could do absolutely nothing about. It had been liberating. His headaches had gone away. He could leave his home after the visits and mingle with his neighbors. Attend public lectures and play chess in the main square. Have friends.

But the company of prisoners had found out. Ahura Mazda was the most vocal, pasting posters full of fiery rhetoric about him at every street corner and shop window. \emph{Look at this creature. The self proclaimed avatar of societal love, turning his back on his community. The avatar of selfishness, rather, of greed. We as a community have worked too hard to earn our place in Their eyes. If he will not walk with us, let him walk alone.} 

It did not take long for everyone to find out. Some, like Freya and Eoto, started an intervention. Weekly meetings with Freya and her friends, filled with tea and cakes and sympathetic but pitying looks. \emph{Why}, they asked. \emph{Why had he turned from the pathway to freedom? Did he want to stay imprisoned forever? Sure, it was hard, what they had to do, and everyone stumbled, they all understood that. But that's why they had gathered here, wasn't it? To help him back to his feet. Didn't he want to be helped?} 

But most just turned their backs on him. After all, if he was going to shoot himself in the foot with this blatantly display of impenitence, then there was no reason for them to be splattered by his blood. Houses became closed to him, others did not speak to him in public gatherings. Some even crossed the street when they saw him coming. Eventually, the social pressure got to him. If he could not have friends, he would rather it be because he felt too ill too meet them than it be because they thought ill of him. He made a public apology and  returned to his posture of contrition. He resumed his practice of bearing witness to the daily suffering that passed through his home. 

The only comfort in this entire prison was in the company of the other prisoners. Most of his neighbors had been imprisoned for going on two millennia with nothing but each other. None of them had access to their followers or their prayers. No one could punish or perform miracles, grant boons or rain devastation on any earthly being. In short, no one could not do any of the things that made them happy. When They had found out that each of them had made themselves known to Their human creations, They had hunted each of them down and placed them in this purgatory. None of them had any access to anything from their old lives. And none of them had ever heard from Them again. Instead, they all found themselves in a strange land, with nothing to do all day, visited by souls that were twisted bastardizations of the very work that they had tried to do in earth. A teacher ripping into your best work on a daily basis, not because of any fault of the work, but because They were upset.

It was one thing to endure this punishment for an untold number of centuries to come in the company of ones peers. To endure it in solitude was something else altogether. 
\end{comment}
\begin{center} ***** \end{center}

Nyaya's guest reminded her of a very dark point in her marriage. There had been once, only once, that anyone had so completely robbed her of her power and left her feeling so helpless, and she had never forgiven him for it. Naya had always prided herself in her ability to protect the ones she loved. Wise men and kings had prayed to her once to help them do the same. It had filled her with pride, a sense of purpose, a steadying knowledge that she was doing her part in the making the universe thrum. To see a visitor live a life completely antithetical to those principles... it shook her.

Well, there was nothing to do about it. She tied her hair and sneakers tight, found a soft belt to keep her garland of skulls from rattling, and went for a run.

There had only been once in her life that she had not protected the ones who came to her, she reminded herself as she found her rhythm. Whether it was a king who needed her strength to win a war against an aggressor or a poor mendicant who had been wronged by a local land owner, they had all come to her. Once. She could always be relied upon to lend her strength to her faithful when they had been wronged. Most famously, perhaps, she had slaughtered the hundred and eight worst criminals and petty warlords of the age, and worn their heads as a garland: a warning for those who would think to cross her in the future. She had been the avatar of protection. Retribution. Revenge. That was who she was. She needed to remember that.

She let her feet choose the path, confining her mind to think only about her pacing and breathing, not allowing it to wander to the interview she had just concluded. Her feet took her by Gautama's house. The plump jolly old man sat on a rocking chair on his front porch, sipping his morning tea, watching the world go by. He reminder her of someone's grandfather, recently retired from a lifetime of productive labor, whittling away the days in anticipation of his grandchildrens' visits. A far cry from the being who had once started a movement that still had a half a billion followers in that distant land they had all been exiled from. Gautama, like the rest of the community, had given up the names humans had used to worship them. Modesty and repentance, Al Jahil, once Ahura Mazda, had preached, was the only path to freedom. Nyaya thought Al Jahil to be a fool, but he had been a very influential fool. After all, even she had relented, shedding her own persona for the mantle of justice. 

Gautama put down his tea when he saw her approaching and quickly stepped inside the house. He immediately emerged with several bags full of sweet smells. 

"Nyaya!" he exclaimed, interrupting her run. "I was just thinking of you." He held out his armload of gifts. "Figs and lemons from my garden, and," he beamed proudly, "I've been experimenting with a new biscotti recipe."

"Good morning, Gautama," Nyaya greeted cheerfully, hiding her irritation at needed to pause her run. Gautama was a sweet man, but sometimes, a bit slow. She didn't blame him. They'd all changed during this imprisonment, some for the better, some less so. It wasn't fair, or right by any means. But there wasn't anything she could do about it. This place was just hard.

When Nyaya jogged in place but did not reach for his offerings, the plump bald man grew flustered. "How silly of me, you aren't going home, are you?" He looked down at her clean white shoes "You've just started your run. Foolish man." He flushed and started to stammer, "I'll ju-ust ta-ake these o-over later, sha-all I?"

Nyaya smiled gently at him, she did not want to humiliate him. "Are the biscotti wrapped?" If they were small, she could take them with her, and he could safe face.

Gautama looked relieved and rummaged around for a bright green and yellow tin. Many several hundreds of years ago, the Buddha had accrued a large pile of gambling debts. When Wata and Khawandagar threatened violence if he did not pay, the Buddha had come to her for protection. It had pleased her to be called upon in this prison as she had been called upon on earth.  She had marshalled what resources she had in their new home and settled the matter. 

Gautama gave her the tin with a bow. Nyaya palmed it and continued on her run. Ever since that incident, Gautama acted as if he were in her debt. Nyaya had tried to convince him that this wasn't necessary. He had once been the very essence of self knowledge and calm. It saddened her to see such a great being grovel. She picked up her pace. It didn't just sadden her. It made her mad. Ad there was nothing she could do about it. 

This place. It was just hard.

Nyaya turned left at the fork in the road to avoid heading into town. She wanted to give the rythmic motion of her feet the time it needed to clear her head. She needed the time to let go.

Gautama's condition swirled with the memories of her interview. For almost every moment of her life, Nyaya had infallibly protected the ones who depended on her. Just as her guest, an eightyish year old woman from Detroit, could not. For all her strength and determination and rage, her guest had suffered. More critically, she had suffered the humiliation of watching her loved ones suffer. Susan Brown. Her guest had a name; skin the color of milk coffee, and eyes the deep blue grey of a stormy dawn. She'd never been beautiful, at least not by the standards of the community she lived with, but she had spirit, and that made Nyaya take to her immediately. No, Susan had more than spirit. Susan had a powerful rage that is born of knowing, clearly and unambiguously, that her people had been wronged. There was nothing selfish or personal about that rage. It wasn't directed at the lovers, the grandmothers, the managers or school teachers that had abused, overlooked or insulted her loved ones. Her rage was pure. It was generational. It called for the wrongs that had been done to her and hers to be made right. It was ... well, her rage was Nyaya's.

And that was why it had left her feeling so unsettled. Susan was, in every way that counted, one of hers. But she had been so impotent, so unable to do anything that made a substantial difference in the lives of those she loved. Nyaya altered her pace to match her breathing. That wasn't right. Moreover, that was unfair. Certainly, Susan had raised her sister's children when her sister found herself in prison for several years. In a very incontrovertible way, Susan had been there for them, cared for them, made them into good, strong, loved human beings. But that wasn't the change Nyaya had meant. Even if she still had power to influence human lives, even if Susan had know to pray to her, Nyaya could not have loaned Susan her strength. The injustice Susan had suffered had gone on too long. No one was calling for vengeance, or could even point to a discrete set of acts that needed avenging. There was no set of people to exact payment from. The crimes that needed correcting were woven into the fabric of Susan's society. That didn't change the purity or power of Susan's rage. It just meant that while Nyaya could resonate with it, empathize with it, recognize it, desire it even, she could do absolutely nothing about it. Susan, with all her strength and courage and beautiful rage, could no more protect her own than if she had been born with her hands already tied behind her back. And Nyaya, even if she were not imsed, could do nothing to unbind her.

Nyaya stumbled on a tree root for no reason. No, that wasn't true either. She stumbled on a tree root because that impotence resonated. It had happened to her only once. During the years before this imprisonment. When she still lived with her husband. He had taken her away from everything she held dear. Her children, her followers, the spirit she loved. He'd forced her to watch him do what he did best: destroy. And when her people cried out for revenge, Nyaya could do nothing but scream.

Nyaya realized that her breathing had shallowed; her lungs were filled with rage. Well that was ridiculous. There was no point in letting her husband still have this power over her. That was why she'd left him. She had once been the avatar of revenge. If she could not exact the deserved punishment, then her rage had no purpose. She had to let it go.

She turned left, then right, then right again, heading back towards the outskirts of town. If solitude and physical activity was not the solution, then perhaps the company of commiserators would help. 

Nyaya had been consort to the destroyer of universes since the beginning of time. But by the very nature of his role, Shiva was not an easy being to live with. Powerful, yes. For as long as she had known him, he had lead others to greatness. Trapping churned poison in his throat on earth, brokering the peace between Mercury's and Sogblen's rival gangs in this prison, their fellow beings were naturally drawn to his wisdom and charisma. Creative, absolutely. Not creative in the sense that Brahma was. No. He was an artist, and athlete, a dancer. Oh to watch him dance for pleasure. It was as terrifying and mesmerizing and graceful as a lion prowling at dusk. Faithful, in his own way. He had many consorts, and Nyaya was still in touch with them all. She had never known a single one of them to doubt how devoted he was to each of them. But, and here the all agreed as well, he was a terrible being to live with. Terrifying and temperamental, one could never quite be certain if he would demand that you bite your own tongue off, or if he would threaten to wash away your father's house in a meditation fueled deluge just to win your hand in marriage. As unromantic as a tornado, as demanding as the climb to Kailash and as constantly as caked in mud as an crocodile hunting its prey. He was an easy husband to be proud of, but not an easy creature to be tied to.

Nyaya slowed her steps as she approached Honesty's house, suddenly ashamed of her sweat dampened hair and dusty shoes. But Honesty had been another one of her husband's consorts, she'd stood by Nyaya's side when she'd left him, and had generally been a friend throughout this imprisonment. That was why Nyaya had sought her out. Not because she would understand the contours of her restlessness. In fact, Honesty almost certainly would not. But they had a shared history together, and sometimes, just that common ground was enough. 

As Nyaya desperately conjured a handkerchief to clean herself off with, Honesty came around the corner, hair combed and pinned, sari starched, impeccably turned out as always, even if she was just carrying a basket to go to the market. The short fair being made a small startled sound, as she almost ran in to Nyaya on the road. Then she looked her up and down, almost hiding her nose wrinkling in disgust with a smile that did not reach her eyes. "You are amazing, Nyaya," she said sweetly, "you know that? This heat. And you are still running. Would that we all had your discipline."

Nyaya shrugged and gave her co consort some space. She knew how much she sweat. There was no reason to offend her daintier sensibilities. "You know. Its just something to pass the time." 

Honesty wore a sari the color of a bouquet of chrysanthemums. It draped over her left arm hiding the worst of the scars she had earned when she had jumped into a fire to protect their shared husband's honor, giving her name to the practice of a millennia of women burning themselves on their husbands' funeral pyre. She had been beautiful once, short and fair, with skin the color of sandalwood, round of face with perfect almond shaped eyes, she could compete for the most admired in any pantheon you put her in. Now, in this prison, she chose to shed her vanity and embrace her disfigurement. An attempt to win Their affection. Here, Honesty was disfigured, but not cowed. Nyaya admired her for that. In her pride and her virtue and the care she took over her appearance, Honesty chose not to be beautiful. Instead, she became formidable.

Honesty continued to evaluate Nyaya's disheveled form, as if considering what to do with a disappointing child. When she had made up her mind, she adjusted the basket on her arm and started walking towards town. "Come to the market with me?"

Nyaya fell in stride with her. "Certainly. But why?" They both knew they didn't need to eat. Or shop, or work. Most basic necessities could just be summoned. They had not lost \emph{everything} in this new home of theirs.

Honesty shrugged and said mockingly, "You know. Its just something to pass the time." Before Nyaya could find a jest to cut the tension, Honesty scolded, "Really, sweetheart, you could make an effort. Get out more."

Nyaya brightened, understanding the source of her friend's displeasure. She had been isolated when she had left her husband, shunned by the community. For a long century, Honesty had been one of the only beings to stand by her side. It was natural that she still worried about Nyaya's social calendar. But that was a problem that she could address. "Gautama made me biscotti. Want one?"

It wasn't that Nyaya didn't interact with others in the community. She just did not interact in the circles that Honesty wished she would. But the offer of food mollified her. She examined the stacks of mound shaped cookies with care, made her selection and changed the subject. "Are you coming to Al Jahil's lecture tonight?" 

Nyaya tried not to roll her eyes. Al Jahil was Ahura Mazda's adopted name. The Wise Lord became The Ignorant. At some point in the distant past, he had had a vision that penance had been the way to convince Them to open the gates of this prison. He encouraged everyone to learn from what their daily visitors had to say. At another point, he'd had a vision of modesty. Everyone nodded gravely and changed their names. The community united and coalesced around his visions, where they had fought and squabbled before. He kept them all in order, she had to give him that. But he was the most showy and pretentious of them all. The hypocrisy repelled her. 

She had no idea if she was going to attend his lecture tonight. But she didn't want to upset Honesty. So she said, "Do the stars shine at night?" 

"Good." Honesty proceeded to rattle off the guest list and the gossip she had about the prominent members of the community who would be in attendance. Nyaya welcomed the distraction, even if she did not care two pennies for what was being said. Honesty had become conventional and proper. An enforcer and follower of the rules, which, according to Al Jahil, were Their rules. Nyaya could not imagine this woman beside her fighting her father to marry the man she loved or shaming him in front of the most power people in the land when they humiliated the man she had chosen. Nyaya shrugged. Honesty seemed happy with her choices, however Nyaya felt about them. This place. It changed people. She had to let that go.

They had been close once, before they'd entered this exile, commiserators and conspirators about the challenges of the man they had both married. Then, during their early days in this horrible place, before Honesty had learned convention, she had not deserted her. For that memory alone, Nyaya had sought her company. 

The first thing Nyaya had done, when They had thrust her in here, was to leave her husband. If her access to her followers no longer depended upon her playing the role of dutiful wife, then she did not need him. The others, of course, had disagreed. Even in those chaotic early days, when pantheons split apart and disoriented gods formed new alliances and fought among each other for dominance, on one thing they could all agree: if Juno could forgive Jupiter his peccadilloes, then what right did she have to not forgive the noble Shiva? 

She told herself that it did not matter. This was a new home, and a new life. She was starting over.

Then, the first thing Nyaya did with her fresh start was to evoke the full power of her rage to protect her community of bickering prisoners against the injustice of Their will. She had found the barriers of this prison deep in the forest that surrounded their community and bashed herself against it, certain that with enough power, it would crack. She had watched friends crumble and cry, unable the face the daily onslaught of suffering souls. She had watched others wither and die, unable to keep their followers faithful when they could not influence their lives. It was not fair. They had no right. They were upset, she granted, possibly rightfully so, she would grant that too. But They had no right to exile, torture and starve them. Her community was angry, frighted, sad. It did not matter that no one had asked her to take on their grief. She fed on the anger, nurtured it, channeled it. Retribution. It was who she was. No power on heaven or earth had ever been able to stop her when she was fueled by justified rage. 

Except Honesty, Sati, back then. Sati had begged her. She had looked so helpless and so scared. 'They should collectively beg for atonement rather than attempt to break through with force', she had argued, quoting Al Jahil's teachings. Nyaya had almost laughed out loud at the ridiculousness of the idea. 'There was no reason to believe that They were like them at all. Why would They listen to prayers and grant wishes when all of them were in prison for having done exactly that.' But Nyaya had looked at Sati's face and the derision had faded. Sati was afraid, it was true. In that heart stopping moment of clarity, Nyaya realized that Sati was afraid of \emph{her}. This woman had stood by her through her neighbor's scorn, defended her honor through their disdain. Like herself, Sati protected her own. And that woman was now asking her to stop. 'Very well,' Nyaya agreed, she would not fight. Centuries later, when Sati became Honesty, she put aside the name she had been known by all her life and become Nyaya, justice. 

An elderly rattle snake fell in with them as they approached the market square. "Honesty, you look delightful this morning."

"Grandfather!" Honesty exclaimed, "what a pleasure. How have you been keeping?" 

Grandfather, nee Blanc Dani, cleared his throat. He kept his eyes focused on the shapely short being in perfumed purple, pointedly ignoring her taller, darker companion. "Can't complain. Can not complain."  He grinned when Honesty offered him a biscotti. "I was out hunting with Vasuki yesterday." At the mention of her husband's snake, Grandfather shot Nyaya a pointed glance. Nyaya pushed down the bile the rose in her throat. There was no point in getting angry, she reminded herself. "He showed me a nest that he said you had woven for him. Is this true?"

Honestly beamed and slipped an arm through Nyaya's, ensuring that she could not escape. "Did he like it?" she asked coquetishly. "I'm so glad. I made it two weeks ago, but I haven't actually seen him since."

"My dear child," Grandfather boomed, "he could not stop talking about it. When can you make me one?"

Nyaya tried gently to pull away from her companion's grasp, there was no need for her to stand here and be snubbed by her husband's friend, but Honesty held fast. 

"Come find me before tonight's lecture and I'll measure you," Honesty cooed. "We'll both be there." 

She turned to Nyaya, who put her hands together in greeting. "Grandfather." 

The elderly snake raised his non-existent eyebrows in her direction. "I'm certain it will be my pleasure." Then he turned to Nyaya and asked "And will Ben Yosef be there?"

A knot formed in Nyaya's stomach. Any public event with Ben Yosef and herself there was certain to go poorly. Even if they barely saw each other, there was no shortage to the whispers of why she had actually left her husband. A public event with the two of them and her husband or his friends was destined to be a catastrophe. 

Honesty saved Nyaya the embarrassment with a titter of feigning innocence. "I highly doubt it, Grandfather. Ben Yosef has become such a hermit. I doubt he even knows of the lecture." 

Grandfather sighed and looked genuinely disappointed. "That is too bad," he shook his head. "Al Jahil told me that he had a vision of reconciliation. My guess is that this is the subject of tonight's lecture." He paused and looked thoughtfully past Nyaya's shoulder. "Ah well. It's too bad that Ben Yosef is just so hard to reach." He bent down and kissed Honesty on the cheek then drifted away without another word to Nyaya.

Nyaya ground her teeth. Ben Yosef. This place had completely broken his spirit. She had watched it break his spirit, and she had raged. Once a helpful, companionable being, known for his strange ability to be friends with everyone without ever taking sides, she'd watched him retreat into his own head and isolate himself from the very community that seemed make him whole. She had never asked Ben Yosef for the full details, but she knew that Al Jahil had played a key role in his transformation. 

Why had Nyaya not exacted payment from Al Jahil as he so richly deserved? She honestly could not say. Perhaps this new place had changed her too, for the worse. She could have chosen to lash out and punish the creature for being a bully. She certainly would have, once. But her anger had been tempered by the knowledge that taking down Al Jahil would have destroyed the fragile peace that he had created. Destroying him would also hurt Gautama, hurt Honesty. So she had not acted, not out of a lack of desire but out of a lack of ability. She had no choice. Instead of exacting her revenge from Al Jahil, she kept her distance from Ben Yosef. She told herself it was to keep tongues from wagging, it was unseemly for a woman such as her to be close to a former lover, but in truth she gave him a wide berth to keep her own heart from breaking. She was not proud of what she had done. She hoped that he would eventually forgive her, that he would eventually find his own way out. 

Honesty beamed. "Did you hear that?"

"Hear what?" Nyaya had not heard anything to merit this much optimism.

"Reconciliation, Nyaya, reconciliation!" Honesty squeezed two her hands in both her own. "If reconciliation is the best way into Their heart... Nyaya, we can end this! Bring Ben Yosef here tonight. I'll work on our husband. What greater need for reconciliation is there in this community than the three of you!"

Nyaya started shaking her head, wide eyed, and pulling away. This was sheer madness. How dare they bring him into their games. "No, Honesty, you can't... I won't do it." Honesty, better than almost anyone else, knew what had passed between them. Why she had left her husband. A public reconciliation. That was an impossible ask, and she certainly wouldn't drag Ben Yosef into it. 

"It could be our salvation, Nyaya. Don't you see." Honesty brought Nyaya's knuckles to her lips and looked up at her with large childlike eyes. "Please."

Nyaya sagged. To have to choose between her two oldest and dearest friends. It was not fair. But it was not the type of injustice she could do anything about. Her anger could not help her here. Instead, she shrugged all four of her shoulders, turned her back on Honesty, and walked towards Ben Yosef's hut. 

When she had turned the corner from the market square, she started running. Honesty knew. Of all people Honesty knew what she was asking of her. It was unforgivable of her to ask her to do this in spite of that knowledge. Over two millenia ago, Nyaya had fallen in love with a handsome wisp of a spirit she met on a journey to help a king avenge himself for the abduction and murder of his family. Her husband had not been more difficult than usual of late, or more demanding, or even more distant. He'd just, well, he'd just continued to be himself. Nyaya had been tired. Suddenly, she found herself away from watchful eyes and responsibilities of the court, free to do as she wished. The spirit had been young, and diferent, and new. Recently, Nyaya spent all of her time associating with the beings who made their home on Kailash. It had been centuries since she has socially met a deity of a different pantheon, and even longer since she had encountered a being who did not belong to a pantheon at all. This spirit was a taste of a foreign fruit, and reminder of a simpler past. He was a pleasant traveling companion, full of amusing stories he had heard from others in distant lands she had not had the time to visit. He was eager to listen to her talk about her life. He was sweet, and curious, and easy to amaze. He watched with interest, admiration even, when she launched a war to devastate a valley and re-throne a king. He'd been unironically enthralled by her power to change the course of human history. He was so blissfully unbothered by how inconsequential his life had been.

So they took the long way home, stopping to mete out justice for the theft of cattle for a local landowner or whimsically picking sides in a conflict between local warlords. They stopped in between on isolated clifftops or in scenic groves to do, well, whatever it was that pleased them both. It thrilled her teach him the pleasures of her existence, to watch him discover what he'd never thought to consider. She showed him the patterns of her life: the strength that she gathered from her followers' prayers; the delicate care and balance needed to  grant boons, too few, and they stopped believing, too many, they killed each other off; the surge of power she got stepping onto the earth and performing a miracle. On a fancy, she let him choose which yogi she would grant inhuman strength or the ability to shape shift to. And she'd laughed when he'd furrowed his brows to consider the nuances of such a relatively inconsequential decision. She wanted him to want what she had. She could keep him hidden as a lover in the court at Kailash, but never publicly. Not if he did not have any power. But she did not want him as a secret. She loved his sweet innocence, and she loved how he looked up to her. She loved how his breath felt when he whispered in her ear. But the more she loved him, the more she wanted him as a peer. 

Honesty knew all of this. She had confessed everything to her in a flury of tears after they had first arrived in this place. Honesty knew too that their husband had found them at the foot of the mountain. Sati had been there. Their husband had locked Nyaya up in a birdcage that hung around the neck of Garuda as he flew through the heavens; imprisoned, but able to see what he would do next. He chose a Persian king who did not even know his name, then stood by while he ravaged her people, their people, destroyed their farmlands and desecrated their ways. She could do nothing to answer her followers' calls for revenge, and somehow, her husband had convinced the court at Kailash to stand by. But that had not been the end of his cruelty. He husband had left her hanging around the great eagle's neck, until the cries for revenge for a brother's murder, became tears for one's father's death, became the grief of a people over generational land and wealth destroyed. And that sad righteous anger, woven into the fabric of a society, lacking a leader to point a finger at, she could do nothing about. 

Nyaya whimpered, suddenly remembering Susan. This place. She hated that it did this to her. They had no right to dredge up her deepest, most painful memories for what? Because it had suited Their whim to punish her thus so many centuries ago? She did not even know if that was actually Their motivation. 

Nyaya's feet found ran faster along the path to Ben Yosef's house. She did not know what she was going to do about Honesty's request, but she had to see Ben Yosef. Her beautiful, innocent spirit. In his grief over their separation, he had done exactly what she had hoped he would. The sweet fool had taken the form of a carpenter’s son and walked among men as an avatar of love, of all things. When his sensitive soul cracked under the pressures of this place, and he had built his hut as far from the center of town and its people as he could. She had to see him first. She would figure out how to protect him later.

\begin{center} ***** \end{center}

When Ben Yosef paused from chopping his cedar logs to wipe the sweat from his eyes, he felt a hand tug at the axe was leaning on. When he looked up, Nyaya crouched before him, offering him a wooden cup of clear yellowish liquid with one hand, cleaning and storing his tools with another, and tidying the chopping block of splinters and sticks with her remaining two. 

She looked ridiculous. Completely naked except for a pair of dusty white running shoes, a dark green ribbon tying her hair in a high pony tail and a bright pink belt binding her white garland of skulls to her waist. But that was not why he wanted to laugh. In his misery and guilt over his visitor this morning, he had turned to prayer, solitude and physical exertion while he waited for his suffering to pass. Relief flooded him as he found himself in the company of a friend, leaving him feeling slightly giddy. 

Ben Yosef grinned. "What's this? Mana from heaven?"

Nyaya shook her head and stood up, pressing the cup in his hands. She stood several inches taller than him and broader in the shoulder. They were both muscular people, but where he had always been wiry, she was built like a warrior. "Nothing so fancy. Plum wine." She led him towards the yard where she had laid out a large woven mat between the kitchen and the chicken coop. 

"Plum wine?" he asked, and wracked his brain wondering who grew plums in the community. "Where did you get plum wine?"

Nyaya gestured towards the barrel where he collected rain water. Ben Yosef suddenly snickered. Water to wine. It was so unoriginal as to be absolutely hilarious. When Nyaya chuckled, Ben Yosef couldn't control himself any longer. He stood in the middle of his yard, convulsing with laughter. 

Nyaya put an arm around him and lead him to the mat and sat him down. "Breathe," she reminded him, and patted him on the shoulder. Then she sat opposite him, where she had placed a similar cup for her self, and removed a cloth between them uncovering several platters of flat breads, honey, dried fruit and a variety of goat cheeses. "Drink," she said, "eat."

Ben Yosef tried to steady himself by sipping on the sweet and sour liquid. It was a surprisingly difficult task. Of late, the days when he did not feel up to seeking out the companionship of others greatly out numbered the days when he did. Sometimes, months would pass between his visits to market day or a tea stall or a simple game of chess. To find a friend here suddenly, when he hadn't realized how much he had needed one, it undid him.

After several mouthfuls of dried apricots and crumbled goat cheese drizzled with honey, he found the composure to say, "I'm being a terrible host. What brings you all the way out to my place today?"

Nyaya covered her mouth to spit out an olive pit before answering, "I heard the rate at which you were chopping logs. It sounded like you needed the company." 

Ben Yosef frowned. As desperately as he wanted that to be the truth, he doubted it. They had been lovers once. Arguably, that had been the source of all this troubles. When they found each other again in this prison, they had become close, but never what they once had been. Initially, he had written off her hesitancy to the proximity of her husband. Then, when left him, she had needed a friend more than anything else. He had stood by her then. She deserved a friend, even if she could not find it in her heart to forgive her husband as he had. But when Ben Yosef had enthusiastically taken up Al Jahil's teaching, they had grown distant. She respected his right to interpret Al Jahil's visions quite so literally, but she would not. She would have no part of his self abnegation and aestheticism. She had not wanted to talk to him about his penance since then. It was strange that she would want to start now.

The morning's guilt lay raw and heavy on his chest. Nyaya's intentions weren't worth trying to figure out, he decided. "I had a visitor" he began, then immediately felt foolish. Of course he had had a visitor. Everyone had a visitor. Every day. "What have I done, Nyaya," he started again. He needed so badly to unburden himself to someone -- the words came out almost before he thought them. "I said that the meek would inherit earth. Turn the other cheek, I'd said," his voice cracked, "I told them to forgive everything, including their own killers." He clenched and unclenched his fists. "I've taught them, not to love each other, but to allow themselves to be abused."

Nyaya stretched out a coffee colored hand and wrapped it around one of his tea colored fists. She looked uspeakably sad. "Let me guess. A South American coffee farmer who lost a son to the local organized violence, but still supports the local politicians who do nothing to stop it, because, ... I don't know, because it's the right thing to do?"

Ben Yosef flinched and pulled his hand away. "No." He thought of Kalpana and how naturally she had swallowed her own grief over her daughter's sale in order to forgive her husband for his mistakes. There was virtue in that, yes, but also an appalling lack of self awareness and agency. Sometimes, you don't have to forgive your neighbor if you love them enough to keep them from hurting you in the first place. "They are more subtle than that, Nyaya. You cannot capture their humanity in a thumbnail sketch like that. That's not fair." 

Ben Yosef watched Nyaya struggle to control her face, and waited for her to lash out with a scathing comment belittling his personal decision to bear witness as his penance. He waited, but it didn't come. Instead, she asked "So who was he?"

"She," Ben Yosef corrected, and the realized that he couldn't proceed. "She had lived as a domestic worker and farmer on the banks of the Ganges. But I cannot tell you her story. The last thing she asked of me as she left was for me to preserve her dignity. I cannot tell you her story and honor her request at the same time." 

Nyaya said nothing but rose to refill their cups. She came back to sit next to him, bare knees almost touching, sipping her drink and waiting for him to continue. It felt good to have her beside him. She was older than him and stronger, certainly. She could also be extremely cynical and frequently unpredictable. But that didn't bother him. They were, he realized, opposites in so many ways. He drew comfort from the fact that in spite of all their differences, they still cared for each other. 

Ben Yosef drained his cup. Nyaya rose to get him another. "She accepted everything that had ever happened to her as her fate," Ben Yosef began again. "She forgave everyone who had every harmed her because she should. She never questioned whether something was fair or just. It never occurred to her to be angry or to demand more."

Nyaya rested a comforting hand on his knee. He stared at the contrast in their skin color, her red black skin an oddly shaped knot on the walnut wood of his knee. "Kalpana had been one of yours," he continued. "She grew up lighting oil lamps in your honor. How did she never think to be more like you instead of all like me." His voice cracked on the last word so he bit his tongue and struggled against the guilt that threatened to overwhelm him. He deserved to be punished if this is where his talk of forgiveness had led. He had been so foolish, so short sighted, so unwise. 

Nyaya shifted to put an arm around him and changed the subject. "Why don't you spend more time in town, Ben Yosef?" she asked. "Al Jahil would welcome you. You don't have to be alone."

The question confused him. Nyaya had never been a proponent of Al Jahil and his associates. Why encourage the association now? The question also made him uncomfortable for reasons he did not fully understand. Ben Yosef roused himself to shake off the discomfort, walking over the the barrel with a large ewer to reduce the frequency of the trips. As distant as they had become of late, Nyaya could still expose his soul with a well chosen word. "This is my penance," he replied at last, reaching for the safe answer. "The company of others is a pleasure. I show my remorse through self denial." 

Nyaya stared at him. Ben Yosef ignored her, focusing intently on his drink. He would not let himself be bullied out of his beliefs by her cynicism, or shaken in his faith by her questions. They may have been lovers once, and then friends, but she had chosen to put distance between the two of them. She had no right to return and question his decisions. He whispered, "Just like going into town is your penance. You despise Al Jahil and his teaching, and yet you insist on keeping his company." 

He immediately regretted his words. Nyaya flinched and pulled away from him. More importantly, he could not believe that he has spoken in spite. That had never been his way, not even when he had been an anonymous wandering spirit. He spent too much time alone, of late. Too much time in his own head, too much time too sick to leave his house. But had the solitude made him mean? Had he allowed this place to change him so much?

Ben Yosef put down his drink and looked at his guest directly in the eye. "I'm sorry." he said. "I'm drunk and upset. I don't mean what I say. Please don't be angry."

When Nyaya didn't react, he tried again. "You may be right. The solitude is getting to me. I should go into town more. Otherwise, I'll become angry, like you." 

That elicited a response. Nyaya gave him a rueful smile, and reached a hand out for a peace offering. "You sweet innocent fool. You know nothing of anger. You never will." 

He took her hand. Then he leaned over and kissed her. This was probably a bad idea, some part of him knew, but he did it anyway. He was lonely and scared, too tired and sick to maintain his penance. 

Nyaya stiffened at first, either surprised, or trying to decide how she wanted to proceed, he couldn't tell. 

"They will talk", she said, sounding like she had made up her mind. Ben Yosef started pulling away, an apology already on his lips for having crossed a line, but Nyaya held him close. "Let them," she said, and kissed him back.

It wasn't, he realized, at all like it had been before. It wasn't just the meagerness of the surroundings, though. They'd made love in all types of places, from the lush inner chambers of an emperor's own harem to the damp stony floor of an underground cave system. Their surroundings had never bothered them at all. It was how she approached him. Or maybe it was him. It had been several millennia, and even beings such as themselves changed eventually. Where he remembered her to be hungry, she was malleable. Where she used to be commanding, she was demur. Where he was used to following her lead, he found himself facing a vacuum that he struggled to fill. So they muddled through in this strange new cadence in the dusty ground in front of his round earthen hut until he climaxed, as much from the memory of who they used to be and the relief of Nyaya's proximity, as from anything they were doing in the present. 

Nyaya rolled over and nesteled Ben Yosef snugly in three of her arms. The sun beat down on their sweaty mud stained bodies. He closed his eyes, and let her fourth hand gently trace the scar on his side.

"Do you think," she asked eventually, "that your guest would have been better served if it had occurred to her to be angry?" 

Ben Yosef lay still, breathing in her musk. He had implied that, hadn't he, that Kalpana had suffered because she had never considered objecting. It wasn't that simple, he knew. If Kalpana had been angry, it would have changed none of the facts about her life. Everything that had happened to her would still have happened. She just would have just been angry as well.

"My visitors," Nyaya continued when he didn't respond, "are always fighters." She paused and wrapped herself around him for comfort or courage. "But they are never fighting battles that they have the slightest chance of winning." She sighed and gave voice to a thought that had been haunting him for centuries. "I don't think my way, what I stood for, who I used to be, is right for the world any more."

Ben Yosef opened his eyes and looked at her. His goddess, the proud fierce queen he had first met over two millenia ago, lay in his arms uncertain. He had seen many expressions on that dark round face of hers, from grief to laughter, from love to hatred, and every possible form of anger and indignation imaginable, but he had never seen her uncertain. He wanted to comfort her. "But you are Nyaya now, justice. A better form of who you used to be." 

Nyaya shifted and turned away from him, blinking tears away from her eyes. Ben Yosef propped himself on an elbow and shook her gently by a shoulder. She didn't respond. "Nyaya," he whispered in her ear, and still got nothing. 

He sat up and contemplated the dark expanse of her back, expanding and contracting in time to her breath. "I think Al Jahil is wrong," he said suddenly, giving voice to thoughts he ever admitted to himself. Whether it was the wine or the passion that gave him courage, he could not tell. Perhaps it came from a need to find common ground with his erstwhile lover and mentor, a woman who had made him everything he was. "You asked me why I don't go to town more often. There are many reasons, but this is one of them. I think Al Jahil is wrong, but I don't have the courage to face him." Nyaya stirred, turning to face him. Encouraged, he continued "I don't think They want repentance. We've tried that for centuries. Its gotten us nowhere. I think They want to see that we've grown. You cannot forgive someone who will not change." He should have taught that as his gospel when he had the chance, he thought. To forgive one another, but not unconditionally. To forgive each other and expect change. Because without change there could be no justice. His face brightened. "Nyaya," he held out a hand for her, "We've both changed, haven't we. This prison has taught us both the error of our ways. We could be so much better together as who we are now. Forgiveness and Justice, two sides of the same coin." 

Nyaya was standing over him before he could finish the sentence. She towered before him in her full glory, barefoot and bloody with her wild bushy hair falling past her hips. Darker and more ominous than the cleaner, more pleasant form she had adopted for this prison. Ben Yosef cowered at the suddenness of her anger. He could not help himself. "Two sides of the same coin, Ben Yosef, really?" Her voice dripped with scorn. "Shiva and Shakti, Ben Yosef? Is that what you want for me again?"

When he didn't answer, when he couldn't answer, she turned from him and blew out great angry lung fulls of air, her garland of heads swaying with every breath. No, he did not want her to be tied to him as she had been to her husband. How could she even think that of him. He had only meant, he didn't know what he'd meant. He'd been trying to figure that out himself.

"Al Jahil is giving a lecture tonight," Nyaya continued, in a complete non-sequitur. Ben Yosef looked up at her, confused. Nyaya had reverted back from the apparition she had briefly become, her garland of bloody heads devoid of all flesh or blood, a tamer version of the the awesome force he had once stumbled upon in the Indus valley. "Apparently he's had a vision of reconciliation. Several people in the community think that this is our new way forward. They," Ben Yosef watched Nyaya struggle for control over her emotions and lose. "They want us to attend. To be reconciled with my husband" she spat.

Ben Yosef was horrified. Terrified. Indignant. Devastated. "Is that why you came. To take me to this," he remembered Al Jahil's lampooning posters from the early days of their imprisonment, "this public humiliation?"

Nyaya raised her head and barked a short cynical cry of laughter to the sky. Then her face softened and she took his hands in two of her own and put the other two on his shoulders. "As you said. I do not like Al Jahil or his friends. Why would I come here to do their bidding?"

"Then why did you come?"

Nyaya shrugged. "I can't tell you. I really can't. Maybe I'd spent so long being who I'm not to please them that I needed to see a friend who knew me." She bent down to kiss him on the forehead, then brought over the ewer of wine and their two discarded cups. They drank to their friendship. To a friendship, and no more. 

"You're not wrong, you know," Nyaya said after she had drained her second cup. "About Al Jahil and the error of our ways." Ben Yosef looked at her cautiously. "Forgiveness and Justice would have made a better pair." He opened his mouth to argue his point again, but she stopped him. "Except that I'm not Justice." 

"But Nyaya! You just said..."

"I know what I said," she interrupted. "If I could still walk the earth as the avatar of retribution, it would be a terrible thing." She paused and gave him a meaningful look. Ben Yosef stared back at her blankly. "But I cannot walk among men any more, and I serve no one pretending to be what I am not."

She extended a hand and clasped his. "I've made a decision, Ben Yosef. Will you support me?" He blinked in surprise. He had no idea what she could possibly be talking about. Then he shook it off and said "Of course," because at the end of the day, he knew he should.

Nyaya transformed. She grew darker, inhumanly dark, as she had when she had been angry. She stood before him, the pure midnight black of moonless night. Her sanitized garland of skulls grew flesh and skin and hair. It dripped blood, each face locked in an expression of pure terror. This time, the intentionality of the change, and the certainty of her aspect, took his breath away. He was again the small lost spirit who had wandered upon the caravan of a major deity. He loved her. And he was terrified for her.  "You are reclaiming Kali," he whispered, alarmed.

"I am," she agreed, and looked at him expectantly. 

"But," a thousand unvoiced terrors flew through his mind. He did his best to ignore them and speak for sense and reason. "You just said that this aspect is a terrible thing."

Nyaya, no, Kali, looked disappointed. The very thought of disappointing her tore his younger self to pieces. "That is not what I'd said, Ben Yosef, but," she sighed, spreading all twenty of her fingers before her, "perhaps you are right. It doesn't matter though. This is who I've always been." 

"But why return to it, Kali, when you have shown that you are capable of change?" The fears had stopped fluttering in his head and had settled down, starting to take shape. The other prisoners, for one, they would tear her apart.

"Change?" Kali laughed bitterly. "We've all changed, Ben Yosef. You. Me. Ahura Mazda. He changed from a strong, proud monodeus to a two faced politician whose only fire is in his bullying rhetoric. We're all capable of change."

Ben Yosef flinched and stiffled an urge to look over his shoulder to see if anyone had heard her. "That was uncharitable," he muttered. She was so brazen, divisive. He had promised his support to his oldest and dearest friend. But she wasn't making it easy. 

Something in his tone must have moved her, because when she next spoke, she sounded remorseful , "It was," she agreed, "thank you." She picked up a stone between her fingers and examined it closely before she spoke again. "There is nothing here that retribution can fix. There's no one to dole punishment out to."  She dropped the pebble in the bowl of discarded olive pits. "So, I'll have to learn to be more like you." 

Kali looked down and studied the length of her shadow, frowning. Then she said, "I should leave soon, or I will be late for the lecture." 

As she turned to leave, Ben Yosef scrambled after her, confused. "You're attending?" Like that? he wanted to add, but bit his tongue. "I thought you said..." but he trailed off. She hadn't said that she wouldn't attend. He did not understand why she would support the farce the others were planning. He hesitated, "Do you want me to come with you?" 

"No, of course not. Why should you attend?"

"You are going to fight Al Jahil?" he asked, incredulous. He could imagine her doing it, she was certainly strong enough. It was a terribly idea. It would thrust them all into chaos.

Kali cupped his face in her hand and brushed his cheek with her thumb. "I suppose I am," she said softly, "Lend me your strength, sweet spirit, so that I may win." 

\begin{comment}
	

Ben Yosef did not see Kali for long time after that visit, though he made an effort to visit the market square more regularly after that day. It had occurred to him, late that night after he had brought in the goats and tidied up the yard, that while he was no longer able to affect the lives that his visitors had lead, they were affecting his. The problem was not that they could forgive any wrong ever done to them, it was that they never felt they had a choice to do anything but accept what came. He had raised meekness up as a virtue, but he had taught so much more than that. 

If he were to convince Them that he had changed, he told himself, he needed to walk away from the errors in his gospels. He swallowed his fear, entered a tea stall packed with chattering gods and ordered a cinnamon and saffron drink. He sat quietly, listening to the buzz of gossip permeating the air.
 
She had been shameful, just perfectly shameful, everyone agreed. 

But what a spectacle, some hooted, to see her come as she had, dressed in that hideous form. 

She was absolutely crazy, others pronounced. Completely unhinged and untrustworthy. 

Perhaps, but absolutely worth coming out to see, someone leered. 

A stain on the community, to humiliate a good man like that. 

Just disrespectful, to say that at Their altar. 

Ben Yosef's stomach clenched, remembering the words that had once been said about him. He drained his glass and rose to deposit it in the tray of dirty dishes. 

Poor Al Jahil. It was a beautiful wedding he'd prepared. All that effort. I hope he's not taking it too hard. 

Ben Yosef pushed through the crowd, having heard enough. 

What did she say, before she walked out? Did you hear? I was too far back. 

I heard that she told them both that she forgave them. 

Ben Yosef grinned, then quickly hid his expression with a cough. He would go home, and go about his business. Kali would visit him again when she was ready. It would take her time to find her courage, he knew from his own experience. But then she would find him.

You must have misheard. Why would she forgive either of them? They've done absolutely nothing wrong.


\begin{comment}
If they were going to attend, they would have to leave soon. "You don't have to go if you don't want to. I very likely won't." 

When she rose to leave, Ben Yosef put a hand out to stop her. "I won't go. What ever Al Jahil wants to say today, I don't think I want to be a part of it." 



"I am," she agreed, and then, as if she expected this would mollify him, "and I'm taking a page from your book and forgiving our neighbors." 

Ben Yosef shook his head still reeling from the magnitude of the revelation and trying to follow the logic. "But they haven't done anything wrong" he managed. 

"Haven't they?" Nyaya smiled and then suddenly looked concerned and looked him in the eye. "Does this frighten you?"

Ben Yosef shook his head abruptly. This frightened him. How could it not? He had gone against the will of the community once, and he never would again. Why could Nyaya, no Kali, never find a way to fit in and get along. Never mind what They would make of this defiance. Why did she always have to fight?  "Of course not," he said out loud. "But our friends will tear you apart."

"I know," Maha Kali said somberly. "And that is why I need your help. I will have the strength not to let them beat me down. But I will need your help to not get angry. Please?" 

Ben Yosef nodded. To taken aback to do anything else. 

To his surprise, Mahakali, gave him a girlish blush and bent down to peck him on the cheek. "Thank you," she said, and then "You've always been my better self."

...

Late that night, alone with the dregs of the barrel, Ben Yosef thought about the events of the day. They had sat together drinking. Talking about the time before their imprisonment of their exploits on earth, and the period of their lives they had shared together. They'd drunk drunk together for hours, until Maha Kali had lain down in the yard and gone to sleep. 

Ben Yosef had left her there, and gone back to his workshop. The cedar logs had lost their appeal, so he turned to the other tasks; curing planks, polishing handles, fitting together the forest marquetry for an end table. It was fine, skilled work that required care and delicacy. It didn't matter that it was dark. It was an easy enought thing to summon a light. Penance, he decided, didn't always have to mean doing things the hardest way possible.

Eventually, when he was certain that she would not encounter any neighbors if she walked home, he went to wake his guest up. She lay there, sweat and blood and dirt mixing to stain her body with mud. With her hair matted and snarled, still stinking of alcohol, she could have passed for a female version of her husband. The thought gave him pause. How different the world could have been if they had met at a different point in time. If it truly could have been the two of them who were the opposite sides of a coin, and not \emph{that} being. He loved her, he realized, though that wasn't knew. Though how deeply he still felt for her still, possibly was.  But she had staked her position about their relationship, and that was an end to the matter. He shook her gently awake, and told her to go home, which she did.

Her decision to revert to her old name still made him uncomfortable. Her defiance of Their superiority terrified him. More immediately, their neighbors would tear them apart, and whatever else, they were still trapped in this prison with the others for all eternity. She wasn't afraid anymore, yet he still was. He didn't know what to do with that.

Yet she'd called him her better self, and that comforted him. They were friends, only, and that disappointed him, but they had been friends now for longer than she and Shiva had been intertwined. Well \emph{that} was an interesting realization, he thought, and took another gulp of wine, spitting out the dregs that had drifted into his mouth. And she, in her own twisted definition of the word, had forgiven Shiva for his torments, and perhaps even, she had forgiven Them. She had simply stopped letting her husband's and Their actions affect her any longer. If an all powerful force could be forgiven, maybe there was something of himself that had rubbed off onto her, just as a bit of Shiva had. He hoped so, and he hoped that he was right. That somewhere in the wide gap between their two personas was the way out of this hell.
\end{comment}

\begin{center}*****\end{center}

The man blubbered. 

Ben Yosef patiently put a thick slab of yak butter at the bottom of the bowl, the poured hot black tea over it. He pressed it into his guest's hand. "Drink, brother. You have had a long journey. This will give you strength." 

The familiar smell of the butter tea elicited a response, and the man cupped the warm bowl in his hands. He continued to cry for several minutes before he could take a sip. 

Ben Yosef waited. This was the penance. To wait, to listen. To bear witness. The man had died young. Himalayan, by the look of him, dressed in the saffron of a Buddhist monk's robe. Ben Yosef braced himself for the knowledge of his guest's grief.

And when it came, over plates of preserved tripe and preserved vegetables, Ben Yosef struggled to not let it overwhelm him. He shrunk the interview into exactly the type of thumbnail sketch he had chastised Nyaya, no Kali, for giving yesterday. He had to. He was \emph{going} to make it into town today, and he could not do that if Tinley's story overwhelmed him. 

Tinley had come from a poor family, but a good, religious family. He had attended the village school for a year or two, then transferred to a monastery when money got tight. Tinley respected his family, as any good child does. But he does not remember them. The monastery became his family and his life, exactly as it should have been. He respected and loved his brethren, performed his chores without complaint, enjoyed going out into the community to find other children who could be educated and helped just as he had been. The respect the lay community held for him and his brethren gave him meaning, and in turn he passed that respect and adoration onto his Khenpo, and man by the name of Rabten. Therefore, when it ended, when his home shuttered its windows and turned the monks out to fend for themselves, Tinley was lost. Ben Yosef could not get out of him why the monastery closed so suddenly. But the man's reticence, combined with his blind adoration for his religious leader made him suspect malfeasance on Rabten's part. The brethren scattered. Tinley could not say where. But Tinley could not bring himself to leave. He stayed in the area, begging for alms in the same community that had venerated him. Ben Yosef could not, would not recapture Tinley's realization that he had become an object of pity in his own community. He needed to make it into town today, and memories of the man's shame would undo him. Eventually, when the rumors spread that the Khenpo had bankrupted the monastery to pay off his gambling debts in Shigatse, Tinley spent three days denouncing the rumors on street corners before hanging himself. He had known, Ben Yosef concluded after the interview. He must have known all along what his mentor had done. Whether his guest could not admit what he knew because that would require forgiveness or whether his guest occupied a space where it was impossible for him to have done wrong was academic. His guest had chosen to absorb the fruits of his mentor's human failings onto himself in the worst possible way, out of some twisted interpretation of the virtue of charity. 

Ben Yosef shuddered as he shut his gate after his guest. His temple throbbed. He wanted to enter the dark of his hut with his rosary and pray for Their forgiveness for all the harm he had done to the world. But he shouldn't.

Nyaya, Kali, had shown him that he had become twisted and angry in his solitude, which was not who he wanted to be. He needed to go into town and interact with other beings. So he closed his eyes and dug his scarred palms into the gate posts until his headache cleared. Then he gathered a basket and walked to the market. 

\begin{center}*****\end{center}

It was not the most successful of trips, but it was a start. As the dirt path widened to a street, he realized that many of his fellow market goers dressed for the social gathering. At the very least, they bathed and groomed themselves, appearing in something nicer than a loin cloth. He, on the other hand, looked like a hermit who had not left his cave for six months. 

And then there was the noise and the clutter of other beings. He had no memory of the market days being \emph{this} busy. He was not prepared to be humiliated in front of a crowd \emph{this} large. 

Everyone one around him was chattering about the events of the night before. Every stall was abuzz with an air of scandal and the thrill of gossip. Someone had behaved deliciously poorly. It turned his stomach and left him gasping for air. 

But he had vowed that he would come, and he was here. He skirted the edge of the crowd until he found Ugly, once Ōkuninushi, and traded a bi-colored wooden ring for a bundle of vetiver clippings. While Ugly seemed as pleased to see him as anyone else in town, Ben Yosef merely mumbled his courtesies, and left the throng as soon as he could, with as little eye contact as possible. 

Only when he paused half way home to wipe his palms dry and steady his breathing did it occur to him that Kali had been at the lecture yesterday. She had gone to fight Al Jahil. Was she the subject of the breathless gossip? No matter. He wasn't going back to find out. One trip into town after such a period of isolation was enough. If his fellow prisoners were to ostracize her, he could learn about it later. 

By the time he passed his grazing goats, the long solitary walk home had stitched his composure together again. He mapped out where he would plant the clippings and how he would fence it off to keep his flock from grazing on the feed before it was ready, and occupied his mind calculating the lengths of wood he would need for the job. 

As a result, he did not notice the dark heap of bruises curled at his gate unti he was almost upon her. 

"Nyaya!" he cried and bent down over. Kali lay still, bleeding from multiple cuts, an arm bent at an awkward angle, elbows and ankles and face swollen and darkening with fresh bruises. "Kali," he said again, more urgently, despairing of where he could touch her without causing her pain. She was completely naked, he realized, without even her necklace of skulls. What had happened to her?
	
When she failed to answer his calls, he bit his lip and cradled her in his arms, and carried her through his gate. 

Kali groaned. Ben Yosef did not understand why the sound flooded him with such relief. They were not mortals. They could not be destroyed with physical violence such as this. But her silence. Her appearing in this wounded form at his doorstep. It unsettled him. 

"What happened?" he asked. 

"The djinn," she muttered. "They surrounded me. I didn't fight back." 

"Why?" he asked, but she did not reply. Ben Yosef lay her down carefully on his bed and waited. 

Kali did not answer. Neither did she start healing herself. Kali's wounds were serious. Her head and face were badly battered, she had two obviously broken bones, her midriff was bruised and bleeding in a way that made him worry about internal damage. Blood had soaked his arms in the brief act of bringing her inside, and was now seeping into his bedding. 

Kali's wounds would have left a mortal fighting for their life. But they were not a mortal. Beings as them could be hung on a cliff side, with an eagle pecking out their liver on a daily basis, and survive for centuries. He would just have to regrow it and close the wound every day. It would hurt, Ben Yosef could not image the interminable torture, but it would not kill him. To do that, one needed to forge special weapons imbued with a piece of their own essence. Such weapons did not exist in this prison. He was certain of that. In the chaos of the early days of their interment, before Al Jahil had figured out how to unite them to a common purpose, they'd banded into warring tribes, eager to take their frustrations out on each other. No one, no matter how hard they had tried or how creative they had been, had managed to forged the lethal weapons of old. That part of their powers that they no longer seemed to have.

So Kali would not die of her wounds. Of that, Ben Yosef was fairly certain. Just as she was certain that she was not comfortable. Why had she not fought back? She was a highly skilled warrior in any form of combat. And having let herself be wounded, why had she come to his gate in this state instead of healing herself and going home? He bent over and put his mouth near Kali's ear. "Can you hear me," he whispered. "You have to heal." Kali gave no sign of having registered his presence. Her breathing continued, slow and ragged, as if she had broken ribs and were sleeping. 

Ben Yosef sighed. If she were not going to do anything to help herself, then he would have to. He padded out of the room, drew water from the well, salted it to taste like tears, found a pile of clean rags, and started to wipe down her wounds. Her feet were caked with clay an mud. The first cloth came back completely filthy dark brown. So did the second. The third he used to gently was away the caked and drying blood that had collected from a wound further up her leg. By the fifth, he was certain that her feet had no broken skin. Her third toe was swollen, possibly broken, but he could not see any other damage to her feet.

Why would the djinn ambush her? Was this the fruit of the scandal she had caused last night? What had she done? And then, an unshakable certainty entered his mind and cast it long shadow from the doorway. She had said that she would have to be more like him. Oh, in Their acursed name! Had she too come to the conclusion that forgiveness and brotherly love meant taking whatever violence was heaped upon you without protest?

"Heal," he whispered. "There's no reason for you to stay like this."

Kali remained impassive. Ben Yosef sighed and moved his cloth up to her legs. Kali's left calf and right ankle had deep gashes that seemed to be the source of much of the blood on her feet. The calf wound was deep, but only oozing blood now that she wasn't walking. The ankle was more concerning. Someone had slashed through to the tendon, attempting to hamstring her. Further up the right shin, a large bruise gleamed purple hinting at the possibility of a fracture within. As the ankle was still bleeding badly, Ben Yosef supposed that this was the wound that needed the most attention. He reached for a long strip of cloth and began wrapping it to the best of his abilities. 

But the wound was ugly. On a mortal it would have needed stitches at least. The padding soaked through as soon as he tighten the knot on the bandage. He undid it to try again, but noticed that cotton padding he had put in place to collect the blood had, instead pressed the wound wide open. Flashes of bone peaked out as he attempted to wash the wound clean.

He would have to make a tourniquet, he realized, then despaired. There were so many wounds, and he was so ill prepared. "Heal!" he snapped, and pressed hard on the exposed flesh of her ankle. Kali's eyes snapped open and she cried out in pain. "You have to heal yourself," he said more gently, removing his bloody fingers. "You aren't convincing anyone of anything by bleeding on my bed." He paused when he saw the bleeding stop then the tendon, muscle and skin stitch back together. She had done a perfect job. The joint bore no scar or swelling or any other indication of the recent wound. "Thank you," he opened his mouth to say, then stopped. That was it. She wasn't tending to any of her other wounds. Her calf still oozed blood, her arm and rib still remained broken, her toe remained bruised. Kali had healed her ankle then had returned back to her absolute stillness. 

Ben Yosef stormed out the hut, as furious with himself as he was with her. He had tended the sick and dying for years. The key was to remain patient and impassive. That had always been his way. Why had he not been able to reproduce that here? 

Because she was being stubborn and unreasonable, a part of him retorted. She was not sick or dying. She did not need his attentions at all. Instead, she lay there, bloody and filthy, refusing to do the one thing that could actually make her whole. She reminded him of too many of this recent guests, and he could not take it.

At his work bench, Ben Yosef lifted the heavy splitting maul over his head and then stopped. Kali was resting inside. The noise of his wood chopping would disturb her. He buried the head of the axe into the ground with a soft thud.

It was more than that, and he knew it. If Kali had appeared at his doorstep badly wounded any time before yesterday, he would have found the strength to be gentle with her. But yesterday he had kissed her. Yesterday, he began to hope that they could be again what they once had been. Before Kali's husband had marched into their pavilion and dragged her out by her hair. He had followed her heavenly prison on foot for months, over hundreds of miles, though mountains and jungles, across raging monsoon fed rivers. He saw what Alexander had done to her people. When the Gatherer had let his armies in, the foreigner had stolen their cattle and trampled their fields; killed their sons and heirs; disrespected their customs, disrupted their trade and dictated how they spoke to their neighbors. He saw her people's prayers go unanswered. The proud kings and warriors they had just served and blessed and empowered, reduced to groveling before an invader in order to maintain a small piece of their dignity and lands. He had heard her agony as prayer after prayer went unanswered, both by her, helpless and trapped in a gilded cage, but also by the court at Kailash. Why had their gods forsaken them, her people had cried, and Ben Yosef has fled, unable to face the agony.

He had spent three centuries stewing in his own anger and heartbreak. He hated the Gatherer more than he had every hated anything in his short life. He hated his cruelty, his vindictiveness, his blatant abuse of power. In his darkest moments, he had wanted, so desperately desired to have a empire greater than all the kingdoms of the world so that he could rival the Gatherer for power, defeat him, and win back the woman he had loved. It had been, unquestioningly, the darkest point in his life, and it had cost him greatly to walk away from it. 

Ben Yoser groaned and sank down on his haunches, burying the heels of his hands in his eyes. Seeing Kali so powerless again, remembering how much he had once loved it. It made him weak. It sapped him of all resolve he had found for himself all those millenia ago in the Taurus mountains. He had known the folly of kissing her last night. It could never have lead to anything good, even without the added complication of Kali foolishly confronting Al Jahil. But he had done it anyway, hadn't her. Imprudent as it had been, they \emph{had} made love last night. And now he was here. She had been attacked by a group of djinn, and he desperately wanted revenge. That simply could not be. He had to live with that.

\begin{center}***********\end{center}

"Yoo hoo? Ben Yosef?" 

Ben Yosef groaned. There was someone at his gate, sounding far too cheerful for the circumstances he found himself in at the moment. He rose from his haunches, dusted himself off, and went to see to the intruder.

A plump bald man stood bowing and smiling at his gate, awkwardly carrying several bags. Ben Yosef stared, the name of the being escaping him for the moment.

Unperturbed, the new comer bobbed his head in greeting and asked "Is Nyaya here? I uh..." he gestured to the bags in his arms and several others on the ground beside him, obscured by the gate, "I brought her garland." 

"Gautama!" Ben Yosef blurted, suddenly remembering the name. "Come in, please." He rushed to open the gate and help with the baggage. Relief washed over him at the prospect of having an ally, even if in the form of someone he barely knew. 

The round man bustled about the yard, flustered and full of nervous energy, explaining which bags contained heads that were dirty, and which had heads that were damaged, and which were in good condition, and his thoughts on what could be done in terms of repairs. 

"Gautama," Ben Yosef interrupted, noting that in all his obvious concern for Kali's welfare, the nervous man had not asked after her. "She's inside, resting. Will you stay for lunch?"

"Oh, how stupid of me!" Gautama exclaimed, then bustled outside the gate to get one last basked. "Sometimes, really, if my head were not attached to my shoulders..." he pushed the basket to Ben Yosef, and declared. "For the host." 

Inside was a tray of small pastries and several bunches of small peach colored fruit. "Pistachios," Gautama offered helpfully, "from my garden, and baklava from this morning." Then he bowed and turned politely, but slowly to go.

Ben Yosef recognized the social dance of an unexpected guest desiring an invitation to stay, but not having the place to request it. He took a deep breath and mastered his unease around strangers. "Let me fetch you some water, at least," was the proper and accepted response. He had been the master of his emotions once. A calm, implacable deity a midst the rhetoric and the fury of his contemporaries. He led Gautama to a shaded part of the yard, and laid out a wide array of refreshments. 

Gautama hemmed and hawed and made the usual polite excuses expected of an unexpected guest. Ben Yosef waited for a pause in the flurry of words, then asked "How did you know she was here?"

Gautama blushed, his wheat colored cheeks turning blotchy and brown. "Well," he stammered. "You you know that we are neighbors." Ben Yosef nodded, though he had not known that. "Whe-when I didn't see he-er on her run this morning, I went searching." He paused to find the most politic way of proceeding. "None of her other friends knew where she was." 

Ben Yosef absorbed the news cautiously. Kali had developed a life here in this prison. Of course she had. He had too, of a sort. It was never reasonable to expect her to remain exactly as he had remembered her. If in the intervening millennia, he had made the decision to manifest to humanity as a deity, then he had no right to be jealous of her having a rich and full friend circle that he did not know about. The logic did absolutely nothing to abate the jealousy however. It was a sin, he reminded himself. The emotion would have to go.

"Do you know what happened?" Gautama asked. 

Surprisingly, that question was easier to answer than tackling his unwanted feelings. "She said the djinn got to her." 

Gautama nodded as if that meant something. "I had wondered last night." He sighed remorsefully.

It was Ben Yosef's turn to ask the questions. "Al Jahil's lecture. Is that what this is about? I knew she was going to make a scene, but..." he trailed off. He felt ashamed to admit that he had not been there for her because she had told him to stay at home to keep him safe. 

Gautama's face became even rounder in surprise. "You weren't there last night." Ben Yosef shook his head. Gautama poured himself tea and settled in for a story. "Do you remeber the temples to the old gods?" 

Ben Yosef shook his head. "Before my time." 

"Hm. Imagine a room, large enough to fit a thousand, with open walls on three sides. Gold as far as the eye can see. Gold ceiling, gold pillars, cloth of gold curtains. A raised dais in the middle where a statue to a god would normally sit. That is what Al Jahil had built for the spectacle."

Ben Yosef nodded, and shifted in his seat. The image made him uncomfortable. That was exactly the impression he believed they needed to walk away from if the community was ever to get Their forgiveness. Moreover, that was what Al Jahil had taught for centuries. The community needed to show that they had left the old ways behind. 

"When Nyaya arrived, her co-wives were all waiting for her. Their husband, The Gatherer was on the dais with Al Jahil. Four djinn stood at the four corners, an honor guard to the ceremonies to come. It was clear that the reconciliation Al Jahil had in mind was a remarriage between Nyaya and him. Although, to be honest, I do not know what they were thinking. I do not understand they possibly thought they could coerce Nyaya into that union again.

When Nyaya appeared as she did, everyone was shocked. None of us had dressed like that for many many centuries. Not that there was anything wrong with her appearance," he hesitated to add. "It was just... unconventional. Honesty, in particular, looked furious. She stormed up to the top of the dias and pulled Al Jahil aside for a conference. The other co-wives, however, did an excellent job. By the time Honesty and Al Jahil were done talking, Nyaya was veiled and garlanded, perfumed and decorated. The perfect image of a bride."

Ben Yosef bristled. He tried to imagine Kali, the queen of retribution whom he had once loved, submitting herself to the adornments of a docile wife. Nyaya had wrapped convention around her like armoor, but she was Kali again, wasn't she? Insecurity and indignation flared inside his chest. Even when they were first lovers, Kali had submitted to her husband when he had found them at the base of Kailash. And last night, she'd been... she had been more wife than ruler with him. He didn't have a place to put this sudden lust for revenge. What they had done to her was wrong, he wanted to protest. How well did he really know her, he asked himself instead. This wasn't his place.

"Al Jahil spoke for a long time about the need to put aside our differences to come together for the greater good, the value of sacrifice, and our need to show Them the lengths to which we are all willing to go to bend to Their will." Ben Yosef must have reacted involuntarily, because Gautama carefully added, "He didn't really say anything he hasn't said many times already.

"Then he announced that the first communal act of reconciliation and sacrifice was to be this reunion between two estranged pillars of our community. At this point Nyaya raised her hand to interrupt him. Very few people outside the dais heard what she said, but everyone up there reacted as if she had publicly thrashed her husband. Al Jahil went white as a sheet of paper, The Gatherer as dark as night. The four djinn looked like they would burst into flame on the spot. Then she walked off the stage and into the night. 

"Those closest to the stage claim that she told both men that she vowed to thwart their will in every way possible, and dared them to stop her. But I was too far back to have been able to hear."

A soft chuckling sound eminated from the doorway. "Is that what they're saying." 

Both Ben Yosef and Gautama jumped to their feet. 

"Kali!"

"Nyaya!"

Gautama gave Ben Yosef a confused look at his form of address, which gave the new unwanted tenant in his heart a surge of pleasure. She hadn't told him. Kali hadn't told him her new name. Jealousy was a sin, he chastised himself agin, and went to give Kali a hand.

"I'm okay, Ben Yosef," Kali said with a reassuring smile. "Thank you for letting me rest." She unfolded herself from where she was sitting against the door frame and brushed past his offered hand in a clear rebuff of any protective thought he may have had.

"Gautama. Thank you so much for gathering my heads. You truly needn't have." She lent in to give him a warm embrace, then stopped. Her friend stood with his mouth agape. Staring.

Kali was as naked as he had found her this morning. She looked odd without the garland of heads to break up the monotone blackness of her skin. She was blacker than the night without the accents the garland provided. A blob of matte paint on parchement, a goddess shaped hole moving through his yard. 

They were both staring, Ben Yosef realized, and moved to provide another seat for Kali to join them. 

Kali gave a wicked smirk and said "You get to stare once." She spread her arms and slowly turned around for inspection, lifting her hair off her back to show the clean smooth surface. 

\emph{She's put herself completely together}, Ben Yosef realized with relief, and set a cup of tea down for her.

Properly chastised, Gautama blushed. "I'm not a man," he grumbled. "I just don't understand why."

"I'm not Nyaya anymore," Kali said brightly, folding her legs under her and reaching for the basket of pistachios. "I'm going back to who I used to be." Her face maintained a studied calm, while two of her hands shucked the pink fruit, and the other two cleaned an deposited the nuts in a neat pile.

It hurt Ben Yosef see his queen recede into a menial domestic task. The simple confession, he realized, had been hard on her. "So what did you say?" he asked, changing the subject. 

Kali shrugged all four of her shoulders and looked up from her task. "I reminded them both that they had neglected to get my consent. I told them that they were free to do what they wanted to do, but they would have to do it without me." 

Ben Yosef felt let down. "Is that all?"

Kali guffawed. "All? If I could show you their faces. I don't think anyone had ever chastened those two before. Certainly, no one has ever said no to my husband. Fought with him, absolutely. Even defeated him on occasion. But I don't think anyone has just walked away." Kali beamed with pride. "They were devastated. I had no idea it would work so well." 

"Is that what you meant when you said you needed to be more like me?" Ben Yosef was confused. He had taught non-violence certainly, but this wasn't turning the other cheek.

Kali rolled an unshucked nut between her palms, embarrassed. "Oh, I had an entire speech prepared. I had every intention of matching Al Jahil's rhetoric with something as florid a 'Father forgive them, they know not what they do.'" She met Ben Yosef's eye, then added. "Well, maybe not that florid. But...." she paused and chewed her lip. "I don't know. I froze. I couldn't do it."

Ben Yosef understood. She had bit her tongue off once to show deference to her husband. Some habits were hard to break out of. "You were scared," he began, and offered her his hand. He wanted to embrace her, but they were not alone.

"No," she protested. "I was too angry." 

"What I don't understand," Gautama, who had been silently brooding over his tea this entire time, cut in, "is why you went running this morning." 

"What?" Kali turned to Gautama, confused, but smiling out of long habit. 

"Did you not once stop to think that what you did had upset the entire community. That you had put yourself in danger?" There was not stammer in the being's voice anymore. Concern for his friend had lent him authority.

Kali dropped the smile. She answered him with cold rage. "As you alluded earlier. I am not a woman."

Gautama had been a deity too, once. His belief system espoused strict discipline and a strict system of hierarchies. "And yet," he replied. Uncowed.

"I run," Kali insisted, her words precise and sharp, "because I have always run. I am vengeance personified. I have nothing to fear."

Gautama glared at her, concern replaced with anger. Ben Yosef recognized a debate about to explode. Kali, he was learning, valued he independence, and while he did not know Gautama well at all, he had a guess at the source of his line of questioning. 

"Kali," he started, in a soothing voice. "I think what we are actually asking," he looked up at Gautama, who continued to glower, neither protesting nor meeting his eyes, "is why you didn't fight back."

A moment of agony flickered across Kali's face before she care placed a pleasant smile over it. "I told you," she said, with a forced pleasantness, "I'm changing my ways." She tipped the basket towards her to as if to find more pistachios. "Baklava!" she cried, pretending she hadn't overheard what Gautama had brought as a host gift. "Ben Yosef, you've never had Gautama's cooking. \emph{This} is better than ambrosia." She busied herself serving the two men seated beside her. "You should realize that when Gautama gives you food, he's made it with his own hands." She turned to her friend all smiles and pleasantries. She had slipped back into Nyaya's demeanor. "When was the last time you conjured food, Gautama? Three hundred years ago? More?"

Ben Yosef had not realized that Gautama had also shed some of his adopted form to become more of the god that he once was until he saw him transform back. The change was subtle, and probably unconscious. The being shrank slightly in size and lost an aura of wisdom and authority. He blushed and smiled and bobbed his round bald head, admitting that he did not recall the date.

Ben Yosef allowed the two friends to talk as they would, conjuring tea and and flat breads as needed to keep their plates full. He did not have the will to intervene beyond what politeness required. He had recognized the agony on Kali's face before she hid it away, and it worried him. He had seen it that morning when Tinley had retold the rumors of his Khenpo's misdeeds. The poor man had been bound by bonds of duty and morality to forgive his spiritual leader. But he had been hurt beyond his comprehension and was adrift. Blinded by grief and anger, the only path forward he had seen was to hurt himself.

\begin{center} *************** \end{center}

Ben Yosef went to the market again a few days days later. This time, he had bathed, scrubbing him self down with an herbal scented soap, and rubbed his hair with a mildly scented oil. He put on a clean, simple wrap that fell to his ankles and arranged the pleats in a pleasing manner. 

He paused at the edge of the market square and contemplated the task ahead. The street ahead still thronged with people, but, he told himself, there were fewer people than after Kali's scene. His mission, he knew, was about a third of the way in. He pulled out the map he had drawn of the market square to familiarize himself with the path he had to take. All he had to do was to follow the eastern most path until he reached the tea stall then turn left. He object should be right after the central fountain.

His hands were shaking. Ben Yosef put the map away and went to wipe his sweaty palms on his loin cloth, then stopped. No. This was important. He would do this right. He conjured a handkerchief, dried his hands, folded the cloth neatly until it fit into the palm of his hand. Then, with one hand holding his items for trade and the other fooled into thinking it was occupied, he took a deep breath and marched into the center of the market.

"Ben Yosef!" Honesty cried in delight. "It is so good to see you out and about again. What brings you here?"

He looked at the short plump woman running the stall before him, dressed in a confection of pinks and lavenders, smelling sweetly of wisteria. Ben Yosef knew her to be well connected to Al Jahil. She held sway in this community. But to see her in her stall, she was the sweetest being one could ever imagine talking to. "I'm looking for a small bag," he explained, "something decorative, perhaps embroidered or macrame?" As he spoke, he pulled out the small wooden box with the marquetry hummingbird he had brought along, to indicate the value of the object he wished to acquire.

Honesty's eyes went wide for a moment as she appraised the offered trade. "Oh my," she said at last. "This is quite the offering. You must be looking for something special." 

Ben Yosef squeezed the handkerchief hidden in his palm. He had thought long and hard before coming to this stall. When Kali had left his house with Gautama, she had left her skulls in is care. She needed to figure out how she felt about them, she had said. He had let the comment pass, but he worried. After two days of indecision, he resolved to visit her. He needed a host gift, and Honesty traded in the best hand made crafts in the entire community.

"Let's see what we have here," Honesty said as she finished searching through her collection of bags and purses. She laid out three small draw string bags, each richly enough embroidered to suite an emperor of the appropriate age. The first was green silk with a bright blue cotton lining. Bright white sashiko cranes flew over ripe millet highlighted in gold thread, a pastoral scene from a different time. Another was cloth of gold with a simple white lining. The green night stared back at him, magnificent in his forgiveness of Gawain's lie. The last was a made of a deep sturdy piece of purple kente, absolutely covered with flowers in every shade of under the sun; the more he looked, the more flowers seemed to emerge from the spaces in between. As he examined his choices, Honesty pulled out a fourth option. "Personally, though, this is my favorite." 

She place before him a small black velvet purse, perfectly round with a platinum clasp discretely decorated with two small diamonds. It was stunning in its smooth lines and elegant simplicity, exactly the type of accessory the Nyaya would have used. Ben Yosef's heart jumped to his throat. Was he being that obvious? Surely, she couldn't know. She was simply doing her job. "No thank you," he said weakly. "I like these other choices better." It wasn't a lie. The selection before him would have thrilled the queen he had once traveled with.

"I see." Honesty said smoothly, putting away the velvet bag. "May I ask what this is for?" she asked with the haughty attentiveness of clerk who recognizes a client with outmoded tastes. "Perhaps I can help you find something suitable that is a bit more... modern?"

Ben Yosef panicked. He had not been ready to face this grilling about his motivations. He had not practiced a lie, and he did not want to confess the truth. "On second thought," he mumbled, "perhaps I shouldn't get anything after all. I'm sorry to have troubled you." 

Honesty put a hand protectively over the box he had offered to trade. "Its no trouble at all," she began with just a touch of sympathy. "There's nothing to be ashamed of, you know. You've spent a long time out of our society. I'd be only too happy to help you make a good impression." 

Whether she was trying to make a sale or pump him for information, Ben Yosef could not tell, and at the moment did not care. The press and the noise of the market was overwhelming. He wanted nothing more than to sit someplace quiet and peaceful where he didn't have to worry about how others judged his actions. He stuck out a shaky palm to get his box back.

Honesty's fingers lingered thoughtfully on the container. "With craftsmanship like this," she started thoughtfully. "How would you like to share the stall with me? It's positively sinful to keep goods like this from the community at large."

She kept to fingers firmly on the lid of the little box and looked at Ben Yosef expectantly. It took him five full seconds to realized that she expected an answer. "I'll think about it," he mumbled, hoping that would be enough. 

"Perfect!" Honest announced brightly, and began wrapping his own offering in a piece of pale yellow tissue paper, as if it had been something he had purchased. "I'll see you here tomorrow then. You don't need to bring much to start with. Two or three items will suffice." She placed the wrapped box into his outstretched hand and smiled warmly. "Its been a pleasure."

Ben Yosef turned quickly and walked away, not fully comprehending how he had committed himself to seeing this being again tomorrow. He pushed the thought out of his panicking brain and put all his efforts on not running for his life.

\begin{center} *************** \end{center}

Kali's house sat on the cliff on the north side of town that faced the ocean. As Ben Yosef approached it, the cobblestones that paved the majority of the community's streets gave way to smooth asphalt shaded on both sides by deodar cedar. After half a kilometer, the evergreens gave way to a twelve foot tall wrought iron gate, designed to look like it might guard a entry to an emporer's inner sanctum, but in reality entirely scalable and purely for show.

Ben Yosef let himself in by door built into the structure for use on non-ceremonial occasions. 

Beyond the gate, a white gravel path lead to a sprawling multi-level house of marble, steel and glass. The path wandered through a carefully manicured garden, showcasing a blend of flowers that would never have appeared together on earth as he knew it. Birds of paradise stood proudly surrounded by tulips, bannana trees shaded clusters of white and black irises, rose bushes competed with hibiscus, and monks hood and and angel's trumpets stood a respectful distance away from the path, reminding the visitor whose domain they were entering.

Almost upon entering the garden, one could look into the Kali's living room. The floor to ceiling windows offered little privacy to the occupants. The furniture consisted of several sleek, pale beige recamiers, shaped like ovals or jelly beans or simple rounded squares. In the center of the room sat a round sheet of glass, balanced atop a curved marble stand. Next to the coffee table, an off white fur rug provided the slightest bit of contrast against a snow white carpet. This was the house of someone who knew the power that expensive tastes could have, even in a community where money had no meaning.

Kali lay atop the largest of her divans, a fierce black mar in the elegant white room. Her loose wild hair tumbled down from the cushion to the carpet. She had one leg folded under her, the other was bent to support the book she read. Two of her hands were folded behind her head to give it a comfortable angle, a third turned the pages while the last idly stirred the cocktail that sat on the carpet beside her.

Naked, bloody and unkempt, she looked... like a grizzly bear in an English garden; out of place and lost. Her face and hands were caked with blood, but she wore no symbols of her power-- no garland adorned her chest, no weapons glistened in the her arms, no hand crafted diamond studded watch told her the time of day. She was no longer Nyaya, the sophisticated divorcee that dared the community to scorn her. Nor was she Kali, the awesome queen whose wild power was tethered by her will alone. She was... Ben Yosef didn't know who she was. Moreover, he worried that Kali didn't either. He took a deep breath and knocked on the front door. If she was lost, then he would be her friend. If she didn't need his help, then... he would be whatever she wanted him to be.

The door opened of its own accord, and in lieu of being greeted by a butler, Ben Yosef found himself in a small skylit foyer with a small shelf for his sandals beside a low fountain to wash his feet and a table with a small covered steamer holding fragrant warm wash cloths for his face and hands. By the time Ben Yosef had finished wiping dry his feet on the soft white floor mat, Kali made a stunning entrance from the other end of the entrance way. 

"Impeccable timing," she beamed, and offered him two of her hands and her warmest smile. "I was just setting out a few plate for lunch," she lied. "Come in. Join me." She stood before him, filthy and uncombed with the air of someone greeting a client for a working lunch. Ben Yosef blinked and stammered the standard politeness about not staying for a meal, furiously trying to reconcile the contradiction in her appearance and her bearing so as not to cause offense. As if a rhino, fresh from a mud bath, had offered him a tray of hors d'oeuvres balanced on its tusks. Some juxtapositions were just hard to reconcile.

He followed her into the breakfast room, where several square black plates lay tastefully arranged on a round marble table: a perfectly shaped pyramid of caviar on a bed of buffalo milk creme fraiche; fans of red and white sashimi in a coral reef of green papaya, star fruit and sweet potato; delicate tapioca and cheese puff pastries drizzled in tamarind sauce; perfectly seared medalions of eland glistened on a bed of yellow pickled radishes. 

Ben Yosef gaped. All of them had changed their abodes countless times over the centuries, shifting with the mood of the community or adopting the images their guests brought them of the societies that still worshiped them. He had understood that Kali had adopted modern human trappings of power since her separation, her wealth a subtle threat against anyone who would move against her. \emph{Stay with me, and enjoy my table. Turn on me, and I will buy the good will of every friend you thought you had.} But his, Ben Yosef was appalled. Last time he had visited her, she had lived in a marble palace: bright intricate murals painted her ceilings; her walls were paneled with wood inlayed with mother of pearl or precious stone or heavily brocaded curtains; her floors were covered in intricate mosaics; and gold dripped from her ceilings in the form of large chandeliers. Not every room was in an artistic style that her people would recognize, but every room had a similarity of purpose, a love of color and art and material that seemed, well, universal. 

Compared to that, his current minimalist surroundings felt like a barren desert. The room he stood in was seamless. Not a joint, not a knob, not a handle showed. It was all elegant lines and delicate balance. Sterile. What was not white was black. And everything was spotless. Ben Yosef dared not touch anything for fear that his fingerprints to leave a smudge on the pristine surface. But had been the essence of Nyaya's power, he recalled. The ability to make you envy and fear and feel ashamed at the same time.

"You've never been to this house, have you?" Kali asked, mistaking his shock for amazement. "Let me give you a tour." They passed from the lounge, which had been visible from the front walkway to the office, a long high-ceilinged skylit room with pristine white desk centered against floor to ceiling windows on the narrow wall and inbuilt white bookshelves reaching running up the full 14 feet of most of the longer sides. In one corner, a black spiral staircase wound down to a dimly lit vault. 

As he descended, he noticed the the volumes lining the adjacent walls became older, more fragile and rarer with each step. Some, he recognized, like a copy of \emph{The Battle of Maldon} or the original writings of Sei Shonagon, and others were manuscripts he had never known existed, like Hardy's \emph{The Poor Man and the Lady} or the writings of the philosopher Yang Zhu. As they landed in the dimly lit gallery, he say something that took his breath away. In a glass case prominently displayed on the left hand wall were a collection of papyrus scrolls, perfectly preserved as if they had been transcribed yesterday: a half dozen gnostic gospels. Around the room, other class cases displayed other lost documents, Mayan codices destroyed by colonial fires; Zeno's manuscripts lost to the mysts of time; \emph{The Book of Toth} that he had not realized actually had ever existed; and the wisdom of the unfortunate physcian Hua Tuo that everyone believed had not survived the author's death sentence. Ben Yosef acknowledged and admired all of these, but he kept gravitating towards the lost gospels -- his stories, the last tributes of prayer, devotion and benediction he had exchanged with humanity before being locked away in this prison. His eyes welled with tears. Where had she found these? He had thought these documents had been lost forever, either to time or persecution or both. He to read them again, to remember his glory and the memory of guiding his people. But he stopped, palmed frozen mid air, less than an inch from the pane of glass.

After that, he followed Kali numbly through the rest of her house, the music room, the courtyard, the shrine, the aviary and the roman style bath. All of it was sleek and austere, modern and imposing. But Ben Yosef did not have it in him to be moved. His soul had been stilled, enchanted, enthralled by the glimpse of his lost stories. 

Kali paused as they entered her bedroom. "What do you think?" she asked, gesturing at the large north facing glass wall the overlooked the sea. 

Ben Yosef shook himself from his reverie. Kali was beaming with pride, and he was being a very rude guest. "It is stunning," he replied looking out through the large glass panes. It truly was. From this vantage, he could see the entire horizon, from east to west. "The sunsets must be stunning."

Kali smirked. "They are." She moved to sit on her bed and gestured for him to join her.

Ben Yosef cleared his throat and fumbled at his belt. This had not been the purpose of his visit. He pulled out the small wooden box he had offered to Honesty earlier in that day. "I, um, I brought you something." He could feel his palms begin to sweat. He quickly put the box down on small floating shelf that served as a bedside table and stepped back.

It looked wrong, he suddenly realized with shame, like a mass of algae on an otherwise pristine lake. The box had been to Kali's aesthetic, intricate and warm and full of colors. It didn't belong on this uncluttered world where wealth meant and ostentatious pretense of how few possessions you had.

"It's stunning," Kali murmured, picking it up to admire it from all sides. "I'd heard people talk about your abilities, but this..." she trailed off, at a loss for words. She reached for the wooden clasp holding the box shut. Ben Yosef bit his lips and pushed down both the joy and the shame that battled for space in his heart. This was the moment he'd been leading up to. And he had no idea how she would react to the gesture.

Kali pulled out a long rosary of wooden beads. One hundred and eight distinct pea sized wooden heads, each carved from a warm maple, dark mahogany based on a thin layer redheart to signify the blood. He had spent the last three days carefully carving and fitting and stringing each one. Opposite the knot, with fifty four terrified heads on either side, hung a single holly dove. He did not know if he had gone too far. He had no desire to control or subdue her as her husband once had. A different version of Shiva and Shakti was never his goal. He had only wanted to remind her that he was by her side. He had no idea how she would take it. 

"You want me to wear a garland again," she asked after a long critical pause.

"I," Ben Yosef licked his lips and considered his words carefully. "I want you to be happy," he concluded.

Kali silently considered the rosary for several seconds, idly fingering the dove as she did. A variety of conflicting emotions flitted across her face, and Ben Yosef wished he had a glimpse into the battles she was raging with herself. He had made the mistake of fighting his demons alone for too long. He knew what it was to be lost.

Suddenly, she made up her mind and slipped it over her head. As she pulled her hair out from under it, something about her carriage changed. "Thank you, little spirit," she said, walking over to the mirror to arrange the beads. Her stride, Ben Yosef noted, was differently assertive. Nyaya's sheen of diplomacy had been shed like an old snakeskin. Kali's power was closer to the surface now, a sheathed sword on a belt, not a pistol holstered under a bulging suit.

"Well," she said, approvingly, with a royal sweep of an arm. "For this, I owe you a present in return." 

"My queen is too kind," Ben Yosef mumbled and bowed instinctive, recognition suddenly sparking in his mind. This was not Nyaya, the socialite lawyer and dispenser of Al Jahil's justice. This was Kali, wife of Shiva and a queen of the triumvirate. She did not dole justice or uphold other people's rules. She was revenge.

\begin{center} ******************* \end{center} 

Ben Yosef found himself climbing into a small black convertible with the snarling face of a large hunting cat emblazoned on the wheel. He wondered at comfort of his small low seat, a little hollow space that looked like it should have felt a lot more cramped than it did. The jumped as the car came to life with a low growl when Kali turned the key. He wondered where she planned to take him. Most of the roads in the community were dirt or cobble. It would ruin the undercarriage of a low riding vehicle such as this. 

But Kali just backed the car up and drove it past the ornate gates of her domain. She turned suddenly left as they passed the paved, cedar shaded road, and Ben Yosef shut his eyes in terror. Left took them up a steep shale covered slope, peppered with boulders and the occasional manzanita bush. He braced either for impact, or for the car to lose traction on the slippery rock and go tumbling down the mountain. When nothing happened, he cautiously loosened his death grip on the door handle and slowly opened his eyes. 

There were driving. On smooth black asphalt road that had not been there on his morning climb. He turned around and looked back the way they had come. Smooth black asphalt stretched behind him for fifty yards. Beyond the shale and manzanita showed no sign of their presence. He blinked. As the car moved, the asphalt rolled up behind it, leaving the underlying mountain side pristine in their wake. He spun around. Ahead of him, road laid itself before them as if placed by invisible hands, swerving around the largest boulders and leaving the car well clear of any branches that could scratch the paint.

He looked at Kali, expecting to see an expression of intense concentration for this complicated trick. Instead, she took her eyes off the road and gave him a broad, confident grin. "Where shall we go, little spirit?" she asked.

His heart fluttered, transported back through the millenia to a much younger age when she would turn to him with a list of supplications she had received during the night and let him choose who she would grace with her attention. King or priest, general or duke, the choice was entirely his. "Where shall we go, little spirit?" she would ask, and whatever he chose, for that one instant the great wild queen would simply obey. The choice itself was irrelevant, he knew that, even then. It simply amused her to have her course set by the careless whims of a wandering waif. But the ability to choose itself had imbued with significance. As if he could chart the course of their new relationship simply by stating where they would travel next. He closed his eyes and laced the fingers of his left hand into one of her right and rolled the question around in his mind. They had gone so many places together. A coastal near vast white sand beaches had been raided yet again by their neighbors to the north. The priest had cried out in agony of the violation of his temple and the theft of his lord's youngest wife. Kali had answered, as Ben Yosef had asked her to. It had taken two weeks for the wife to be returned, the temple's relics to be restored, and the perpetrators' heads to be displayed on the city walls. Kali had orchestrated the entire affair from atop a low mountain range, where three river met in a frothy fall. He would pick and alkanet during the days and bathe in turbid pools of the many tierd falls. At night, she would tell him of her triumphs, and he would paint her feet red.

They had gone to some islands once, far off the coast from the lands that Kali primarily called home. The island had converted, it seemed, possibly for convenience, only a generation or two ago, and now, the island's new prince needed her strength to punish the ever grasping hand of the main land that wanted to take and take and take and would not leave them alone. Ben Yosef had known that the task was beneath her attention, that Kali liked to reserve herself for more powerful armies or more atrocious wrongs. But the cerulean waters had called to him, and the jeweled fish that had darted in and out of bright underwater coral forests were more delightful than anything he had ever seen before. Kali had indulged him, as he knew she would. It took her barely a month to subdue the mainland. Every night, she would come home after midnight, ravenous and victorious. Every night, he would feed her fresh stews and listen to her stories of destruction and gore until she drifted off to sleep. Then he would rise with the first birdsong and dive into the warm dark waters to swim with the turtles and the whales.

He remembered a yogi too, praying on an island in the middle of a river over five miles wide and pregnant with monsoon rain. He had not realized that dolphins could live in fresh water or that cobras could swim. He had watched a mother elephant ...

But the car had stopped. Ben Yosef opened his eyes and looked around him, the reality of his imprisonment sweeping away his memories like a handful of autumn leaves. They had parked on the cliff face about a mile from where Kali had built her house. Next to them gurgled a brown little creek that dribbled itself over the mossy rock face into a muddy swamp that eventually gave way to a rocky beach. Behind them lay the main bulk of the community. Ahead, the ocean, and just beyond the horizon, he knew, the walls of this eternal prison.

"Where shall we go, little spirit," Kali asked again, oblivious to his disappointment. Ben Yosef bit his lip. He had to pick somewhere, he knew. This was her gift to him, he realized. An homage to what they had once shared, pale and tepid as it now had to be. This was her gift, he reminded himself, and he would not disappoint her. He drew himself together and scanned the landscape before him. He wanted someplace private, that was a certainty, going back into town was not what he wanted. But where was he to take her? 

He spotted a small finger of land that turned into an island with the rising tide, and pointed. There, at least, they could sit, and talk, and listen to the waves come in.

"As you will," Kali stated, then revved the engine so loudly that Ben Yosef jumped. He found himself pressed to the door as the car spun on a dim and hurtled across the barren cliff tops and into the woods below. Ben Yosef cowered, covering his head with his arms as young branched and vines snapped back from the impromptu road and whistle past his ears. They were going fast. Fast enough that the wind was a low dull roar and felt like a desert storm he should not be caught out in. Faster, much faster, than a body should be traveling off road in an open carriage on a thickly wooded mountainside. Ben Yosef curled himself closer to his knees.

Beside him, Kali whooped. He turned his head to see all four of her arms in the air, grasping joyfully at branches and vines that looked like the would behead her, only to swing suddenly out of reach and fall back into place as the car sped past, swaying softly in their wake. "Are you being a turtle, little spirit," Kali teased, as she reached for a sprig of red honeysuckle that had snagged itself obligingly on the side view mirror. She dropped the flowers neatly beside him. They smelled glorious.

"I suppose I am," he admitted, and straightened his body cautiously to look around. It would be a long while until he would feel the thrill of driving a Jaguar down a wooded mountainside in spring again, he told himself. He should take advantage of it, while he had the chance. Yes, it would be extremely uncomfortable to find himself wrapped around a tree trunk alongside two tons of sports car, but that was not likely to happen, was it, with Kali at the wheel. He should let himself enjoy the moment while it lasted. Kali offered him a hand, and he squeezed it, for dear life. Then he tilted back his head and let the sweet mossy smell of the mountain carry him away.

\begin{center} ******************* \end{center} 

The sun was low in the sky by the time they had reached the promontory he had indicated. Kali pulled the keys from the ignition and looked around thoughtfully. It was not such a bad spot, Ben Yosef told himself. The ocean smelt fresh today, void of the rotten salty smell of beached seaweed. And it was private, for the most part. In the distance, a mermaid sunned herself on a rock, and the squawking of the gulls was not entirely unpleasant. The tide had started to come in, however, and now they had to figure out a way to make it over to the tip of the promontory that was quickly becoming an island.

"Would you like a turn, little spirit?" Kali dangled the keys off the tip of a finger between them. 

Ben Yosef looked confused. He'd never driven before. 

Kali smirked, "I hear you've developed a reputation for walking on water since we last traveled together. Show me what you can do." 

Ben Yosef felt incredibly dense. Why was she asking him to take her car out over the water? "To the island?" he replied. 

"What ever you want," Kali shrugged and got out of the car. Ben Yosef slid over. 

The car came to life when his foot hit the gas, agile and responsive to his very will. He spent a minute driving over the pebbled shore, practicing laying and rolling up road as he had seen Kali do. This was a fantastic vehicle, and he would not want to damage it out of carelessness. The car purred like a lion and moved like a cobra. It was everything Kali's mount should be.

"Impressive," he muttered under his breath, and beside him, his passenger positively glowed with pride.

It struck him, suddenly, that while for all of their time together, wandering the the deserts and the mountains of the lands where they had met, he had assumed that she had been amusing herself with him, he may have been completely wrong. He had been inexperienced and young, he'd wielded absolutely no power. When they had met, he was the exact opposite of everything she represented. He'd just assumed that she had taken him on a pet for her own amusement. He'd been okay with that; it seemed to make them both happy. But now, as he watched her studying him from corner of his eye, another possibility dawned on him. The tour of her domain, the detailed stories of her conquests, the many times she had let him watch as she had fulfilled a supplicant's prayers. Even, even the placement of the lost Gnostic gospels where he could admire them in her library. Kali had wanted to impress him, hadn't she. It was as simple as that.

Well, two can play at that game, he thought, and pointed the car straight towards the horizon.

It was easy, Ben Yosef realized, after a brief moment of adjust for the motion of the waves. Easier, even, since he didn't have to worry about laying down pavement. In a vehicle that wanted to a be an extension of his body, driving on the ocean was no different than walking. 

He chuckled for the joy of it, and put more pressure on the gas.

\hlfix{There had been a moment, many years ago, when they had met a muni grieving the death of his only son. A hunting accident, the official story had been. The king had been granted sound seeking arrows by a powerful yogi for his strength and his virtue, and the boy had been unfortunate enough to be fetching water next to a prized deer. The king had apologized personally, and promised that the appropriate parties would be punished. The muni had not believed it for a second. Only a fool would use blessed arrows on a hunting expedition, and yogis did not give boons to fools. The current king had come to power barely a decade, killing or exiling the entire royal family to do so. Over the last year, rumor spoke of an exiled prince hiding in the jungles gathering information and men. His son had not been the target  of the magical arrows, the muni was certain, but in his grief, the father could not think of a boon greater than \emph{may he lose his favorite son in turn}. Kali had been offended. The man had spent a hundred days letting no part of his body touch the ground but a single big toe to earn the right to call upon the goddess in his time of need. When people called on the triumvirate with that type of devotion, they asked for boons like immortality. She had scoffed at the smallness of the request and let Ben Yosef grant the wish. He had tried to convince the man that declaring the tyrant's line completely barren was a more suitable punishment, but the grieving father would not be swayed. In the end, he had no choice but to help the man curse the king and die of grief. It had broken his heart. Kali had just shrugged and walked away from the whole mess. His squeamishness and mercy would not serve him well, she had declared. She had given him a chance to wield her power, and he had disappointed her.}{is this story distracting?}

Well, Ben Yosef decided. He would not disappoint her again. He put back his shoulders, summoned a large wave and drove through the solid blue arch as it crested towards the shore. He had his way, and it was different from hers. He would teach her the power and grace of how he chose to rule. 

He looked over to see how Kali was taking the ride. She had reached both right arms out of the car to plow two furrows in the wall of water they drove beside. He had never seen an expression of such childlike amazement on her warrior's face. Neither Nyaya or Kali had ever given the impression that there was ever anything in the world worth having that they could not buy to summon at will. It had made her jaded.

Ben Yosef grinned, sat up a little taller, turned left and drove the car into the sunset.

\begin{center} ******************* \end{center} 

It wasn't hard after that to make it into market more frequently. Maybe not every day, but every two or three was a cadence that he could sustain with hardly any effort. 

Granny Weaver had a watering hole on the far side of the market, where he could sip a tall glass of cold brewed cinnamon and saffron and watch the town go by. She was always busy by mid-morning, but if he came out right after he let his goats out to pasture, he could enjoy an hour of solitude while Granny Weaver spun her protective web.


\end{document}
