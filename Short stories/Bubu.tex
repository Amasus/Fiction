\documentclass{article}
% Uncomment the following line to allow the usage of graphics (.png, .jpg)
%\usepackage[pdftex]{graphicx}
% Comment the following line to NOT allow the usage of umlauts
\usepackage[utf8]{inputenc}
\usepackage{fullpage}
% Start the document
\begin{document}
She had everything; a comfortable home, a good family, land and a store she enjoyed working in, she even had a child growing inside her. She had everything. Her current situation terrified her.

She would meet him at a hotel near the train station, the type that rents rooms by the hour. She told her family that she was visiting her sister in the city. They were close, she visited her frequently. She had stayed with them a month last autumn during the dengue fever outbreak, first nursing her brother in law, then her niece, then finally her brother, who prefered his sisters' care to the company of the single men he shared rooms with. Her infant nephew had strong fatelines. He stayed within the mosquito net during the episode, and never fell ill. Several years earlier, when her niece was six months old, her father-in-law gave her permission to fly across the country to baby sit. Her sister was attending her first conference since becoming a mother. She had not set up a child care regime yet. Her father-in-law thought it would be good practice, back when there had been any hope of having children. It had been her first flight. She had worried how she would breathe, since the air was so much thinner at high altitudes. It surprised her to learn that the cabins were pressurized. So she spent a wonderful week with her niece in the hot lowlands at the feet of the coastal mountains, returning home more worldly, more in love with her niece, and more eager to return to her household chores and her position in the family business.

She stepped off the train and ducked into the station bathroom. In the sodden privacy of the stall, she carefully removed the coral, the conch and the iron from her wrists. She was doing this for her family, she reminded her trembling hands as they wrapped the bracelets in her handkerchief and laid them in the bottom of her bag. She took out a piece of tissue and rubbed at the vermillion in the part of her hair. It was almost impossible to remove all the traces of red from the hairline like this. She did the best she could and reparted her hair to hide the rest,  emerging from the bathroom an almost single woman.

He met her at the station, under the large clock by the ticket counter. They smiled, greeted each other warmly, if a bit awkwardly, and walked to the hotel together. He was a pleasant looking man, short, but from a family of tall men, not too dark, with a thick head of hair. He came from a good family, well respected in his village. His brother owned a business in the town she caught the train into the city from. He had recently started supplying her father-in-law's business. That was how they had met, at the family store. It would have been better if he had been completely unrelated to her family, but between her responsibilities at the house, and those at the shop, it was impossible for her to get time away. Only in the movies, and perhaps in the city, could one have an affair with a complete stranger. It was not as if she could ask her city dwelling sister to arrange someone for her. The man by her side was a school teacher living and teaching in a village about half an hour's rickshaw ride from the far side of the train tracks. Gossip would find it difficult to travel the near two hours between their houses, changing modes of transporation four times to wag its tongue, but it might be happy to sit at a tea stall near the station and mingle. Still, this was the best she could probably do.

He was a good man, for one who was willing to have a relationship with a married woman. A couple months ago, she had convinced her husband to meet him and his brother, the sugar supplier, for a movie in town. The movie had been syrupy sweet, about lovers meeting across class and caste lines, eloping, then resolving their decisions with their families. Afterwards, they sat over tea and sweets, Her husband and his brother talking about their common business interests, leaving her free to investigate this man beside her. 

They had both enjoyed the movie. 

She did not watch movies in town very frequently, perhaps a few times a year. She enjoyed them in the evening on the TV at home. 

He frequented this theater approximately monthly, perhaps a little more. 

She used to sew during her time in front of the TV. She had a small tailoring business. She'd given it up years ago, her brother-in-law did not like the noise of the machine over the TV.

His sister also had a small blouse tailoring business. He hoped that his sister would be able to make the investment their father had made in her training pay in her husband's home.

She had given hers up after marriage, when the pressures from the chores at home and the family business left her with little time for private responsibilities. 

He understood completely, and admitted that his sister may have to do the same.  He spent this early mornings tutoring children from the village, his mornings and afternoons at the schools and his evenings were spent between tasks for his family and helping his brother establish his business. It meant that he had to travel to the city frequently on Saturdays.

She used to tutor children too, when she lived with her mother. Her students were much younger than his. She had never graduated high school. Did he have any family in the city? 

No. Did she? 

Yes, now. She had grown up in a distant village, but now two of her siblings had established themselves there. No, no one was left with her parents, her siblings had all scattered to different parts of the state. They were a close family though. They still managed to get together several times a year, on one excuse or another. 

He came from a small family. It was just the three siblings. His father had grown up elsewhere, moved here after college, when he had gone into business with a friend. Isn't it a pity how the traditional large families have declined over the generations?

Her husband told her the time. If they did not leave now, they would miss the last bus. She bade the company goodnight. It had been a pleasure meeting both men. 

Her husband was always polite and pleasant company, with the exception of a few months a nearly a year ago, but that was her doing, more than his. He was the same tonight. They discussed their business plans made that night, the foreman's mother's health, and their nephew's schooling, before falling into a comfortable silence well before the bus reached their stop. She was glad the man she had just met that night was so close to his family. They seemed to share a taste in movies, and shared some common experiences. He did not look down upon her for her education, in fact he had been very polite, even warm all night. The conversation had been very pleasant. She had married on less. 

That he did not have much family in the area offered her some safety. They both had reasons to travel to the city regularly, that would give her some privacy. He would suit her purposes, she decided by the time she had walked the mile home to find her father-in-law waiting up for them by the television. She gave him his nightly medications and helped him with his mosquito net before retiring. All that needed to be done now was to ascertain that she could suit some purpose for this stranger. She lay awake for several hours after her husband had fallen asleep contemplating whether or not she should take steps to obtain that information.

It had been surprisingly easy, she recalled, contemplating in the dingy hotel lobby. She was not a frequenter of hotels, usually staying with friends and family on the few occasions she had to travel. There had been the hotel of her sister's conference, of course, where she babysat, as well as the pilgrimage she had gone on with her mother-in-law, and the trip with her sister and husband to the ocean front during the holidays a few years ago. Even with this limited experience, she could tell this hotel was dingy. The floor looked like it was perhaps swept daily and mopped with far less frequency, she could see betel juice stains in the corners of the room, and a list of beers, cigarettes and other \emph{necessities} available for purchase from reception. A cockroach scuttled into a crack underneath the stairs. 

Her afternoon's companion was making the arrangements. The hardest part of gauging his interest was figuring out what to say. She considered what women in similar situations said in stories and movies, sifting through the lines for things she could say without destroying her dignity. When she had the scene planned out in her head, she arranged to accompany her husband or brother-in-law into town. That part was easy. It was harder to guess when the supplier's brother would also be there. It took about a month before they coincided. She had started to despair that she would have to start the process over. Eventually they did meet. She saw him sitting at the tea stall across from the rickshaw stand, talking to one of his ex-students. She took a moment to wish him good evening, and found herself fumbling with her words. She had no practise with this. She had not even thought to like anyone before she married. She had been too busy with her parents' farm to bother. Up until recently, she had been too occupied with her family's household and business to need anything else. Was going to lure him to her with her inexperience? If only it was not so crucial that she succeed. 

The teacher bought her a cup of tea over her objections. She sat and talked with them while her brother-in-law haggled over glass bottles. The ex-student told her about the beautiful set of atlases his teacher had in his classroom. It was those atlases that had encouraged him to study geography in college. He now worked with a surveying company, traveling all over the state. It was a good job, he was grateful for the inspiration. 

She listened politely. She should not be away from her brother-in-law for long, not if she wanted to return again. She finished her drink, and bid her company good evening. She had not had a chance to talk to the teacher as she wished.

He smiled at her. It had been nice to run into her again. He could usually found in town on Wednessdays and Fridays. Should he help her find her brother-in-law?

No, she could manage. She knew which shops he had to visit. She found her brother-in-law with a large crate of bottles, relieved him of the package, and headed for the vegetable stalls, he for the pharmacist. 

They reunited for the journey home. Where had she slipped off early in the evening?

She had run into an old client of hers. They had fallen to gossiping. It was an easy lie, her brother-in-law had never been interested in her sewing business. 

It had been difficult to convince her father-in-law to let her visit town every ten days. They only needed to go in every two weeks. Why this sudden desire for the urban? She approached her husband with the matter. He did not question her or remark on the oddness of the request. He simply promised to lobby her cause to his father. For the next two months, he took her to town every ten days. It took that long until she worked up the courage to mention to the teacher that she would be visiting the city in a week. He was a keen man, and took the opportunity to suggest meeting at the station for some privacy. 
His keenness made her uncomfortable.  

His eargerness elevated her heart rate now as he indicated that he had the key. She followed him up the stairs. What was she doing with a man so eager to join with a married woman? He was an inspirational teacher who cared for his family, but what security did that grant her? She wished there was another way. She discretely removed the pin at her shoulder securing her sari, and stepped nervously over the threshold. 

She was on a train for her sister's house an hour and a half later. They had arranged to meet again before she returned home, this time at a town halfway between the city and the town they got off at. It would mean he would accompany her on the way home. That would have to be endured, she \emph{was} telling him that she wanted him. Under different circumstances, it would not have been so bad. He was a pleasant man, someone whose company she would ordinarily welcome, excepting the obvious character flaw. She would have to think of a way to end this once she had what she wanted. The earlier she thought that through, the better.

Her niece was delighted to see her. She missed being at school, and told her all about the teachers. She showed off her English and brought out all her school books. When her brother awoke, they brought in the laundry while the boy toddled beside them on the roof and took turns watching him and helping her sister with dinner. Her sister complained about politics at work and gossiped about the neighbors. She listened. For her own part, she talked about her father-in-law's health and a recent argument she had with her sister-in-law, about their neighbor's cow being ill and the prospering fish stock in their pond. She did not tell her about the afternoon's events. Her sister knew her situation, of course. At least, she knew what had instigated this drastic course of action, but she could not bring herself to update her.

Her brother-in-law came home. He asked after her family's health, and talked about recent developments at work. He had been passed up for that promotion, but was thinking of enrolling in night school. 

He sat with his daughter's school work, while she and her sister caught up on the doings of the rest of the family. Their remaining sister was applying for the position at the prestigeous mission school just outside the city. Her academic carreer was phenomenal, she was clearly the most successful of all the siblings. 

Still, a position at such a good school would make it difficult for her parents to find a good match for her. 

Her sister chided her small mindedness. 

It was not small minded. She was concerned, that was all. 

Her sister reminded her that she was capable of finding a husband for herself. 

She wished her the best. It would be good to have her closer to the city, instead of living in the north of the state. 

Their youngest brother was with their parents for the college break. He had called two days ago. He found it easier to study for his exams in the quiet of his parents' house. 

She was glad that he was taking his exams seriously again. He had not always been to dedicated to his studies.

She fed her niece in front of the television while her sister put the boy to bed. Then she put the girl to sleep and neatened the apartment, giving her sister some time with her husband. Eventually, she joined the adults for dinner.
 
This was such a comfortable routine. She had been welcome in her sister's household for as long as her sister had a household to call her own. Her brother-in-law was all that one could hope for, he never made any of her siblings feel like they were anywhere other than their own home. She knew where books and clothes and keys lived in this house, what her niece's school bus schedule was, how much of a tab the family had run up at the local grocers. There had been differences over the years, of course, that was true of all families. But the siblings had always been close, she had always been very fond of this sister. They had grown up a large family in a small house. It was impossible not to share everything. A year ago, she could not have imagined ever keeping a secret from this family. Yet as she mopped the floor where they had eaten, she could not find the courage to say the words. He sister returned from doing dishes, her brother-in-law had hung the mosquito nets. It was time to go to bed.

\vspace{.5cm}

The boy who tended the herd failed to show up today for unknown reasons of his own. Therefore, it fell to her to bring the goats in from the rain. The stench of their shed was overpowering today. She would have to tell someone soon.

Her rendezvous with the teacher had lasted a little over two months. They had met in the city only that once. In part, this had been a relief for her. She did not have enough cash saved from her sewing days to pay for an indefinite number of hotel rooms. She had insisted on paying her part, she did not want to be beholden to his man afterwards. For the rest, she had been terrified. He seemed to have an endless network of college friends who would turn a blind eye, or colleagues away on vacation whose houses he could use. She did not like it; it was too public. She did not like trailing her desperation around for his circle of associates to see. In the end, after assurances of discretion, she relented. It was true-- she was desperate.  She had wondered, sometimes, in the quiet hours of the night what she was doing. Did his inexhaustible network mean that he had done this before? What had she tangled herself into? If he had done this before, he had kept his other women quiet as well. She had not been able to uncover any stories about his bad habits in the questioning she had done before embarking on this endeavour.

She ended their meetings a week ago, as soon as she was certain she had what she had come for. It was risky, she knew. She may lose her prize, at which point she would have to go through a similar ordeal again. She had contemplated staying in her current situation until her position was more certain, but the humiliation of it all was more than she wished to bear. Rightly or wrongly, she chose the riskier path that offered immediate solace. He had taken her breaking of their relationship easily. She gave the excuse of no longer being able to withstand the increasing questions of her family, that she was not a brave woman. It was not entirely false. Her sudden and sometimes frequent absences had not gone unnoticed by her mother-in-law, but the woman shied from trouble by nature. As long as she completed her chores, the old woman would not complain to her father-in-law. He had laughed at her when she expressed her regrets and derided her when she claimed cowardice, but he had let her go. It had stung, but it did not matter. If he disliked her, perhaps he would not see fit to pursue her afterwards. Perhaps she could break this off as cleanly, and go back to the quiet of her family. Perhaps she could preserve some dignity.

The rain had prevented her from going to the store today. Her husband came home muddy and soaked, inspite of his umbrella. His father and brother would be along shortly. They were visiting one of their neighbor's. Their boy wanted a position at the shop. She drew and heated water for all three men to wash with as they came home. Dinner would be ready soon. He probably should not turn on the TV, the rain kept the solar panels from charging much today. He could tell her about the takings at the store instead.

She had been hopeful when she first came to her husband's home seven years ago. Unlike her sisters, she did not want to live in the city. She had grown up on farm, and that was where she was comfortable. Her husband's house was remote enough that her siblings were not likely to visit her often. She resigned herself to doing all the traveling that her family would permit and let the matter stand. Her father-in-law had a successful local business selling dried goods and homeopathic medicines. The family had a modest amount of land. She could help them prosper. Her father-in-law had been the first to see the advantages of her work ethic. Her sister-in-law, the elder brother's wife, was better educated than she, having graduated from a well known boarding school, showed no interest in the business side of the family's affairs, choosing to stay by her mother-in-law's side. She had not such compunctions. Her father-in-law took her under his wing, and she spent two evenings a week at the store. She quickly became his favorite. This caused some tensions with the other women in the household of course, but her mother had advised her on how to win them over. She redoubled her efforts around the house, which eventually earned her mother-in-law's acceptance. Her sister-in-law quieted a when she gave birth to a boy. The six year old, and his three year old brother sat in the center of their grandfather's vision, the pride and joy of the household.

The boys had not diminished her role in the family. She could never accuse the two charming bundles of trouble of that. Even if they had, she would have  gladly moved over to let them bask in their grandfather's love. But they had not. One could say that their births had indicated a shift from her playing a central role in the household to a more central role at the business. It was not where she had forseen her life going when she had married, but one cannot argue with what is written in one's fate. There was not much she could complain about, her husband was happy with her new position in the household. There was even talk, a few years ago, when his father's health had given them all a scare, that the land would be given to the elder brother, and the business to the younger. Her husband had pointed out to her that this was, at least in part, due to her active presence in the shop. 

The elder of her two brothers, a clerk for a lawyer, told her she should be proud that her father-in-law held her in such esteem. But she was not her brother, she did not live in the city. Her ambitions had never been to own a successful business. She was simply grateful that it had not come to that. Her father-in-law's heart recovered. He had to be careful of his diet and activity in the future, but he had shown no further signs of illness.

She quizzed her niece and nephew on their school work and mended the boy's school uniform while their mother put the youngest to sleep. She would soon not be able to help her neice any more. She usually liked working with the children, but then, she'd found that she enjoyed her morning's tutoring the village kids in her ancestral home a pleasure beyond the money it earned the family. Some said she had a gift for reaching them in a way they understood. Today, however, her heart was not in their studies. She was tired; her mind on the conversation she would have to have with her husband. When her sister-in-law returned, she excused herself and went to bed. There was still mending and accounting to be done. She would tend to them in the morning.

She had not thought it worth bringing up until nearly two years ago. Attempt after attempt had failed. Then she started hearing the unavoidable whispers during the holidays. Her neighbors thought that it was her fault. The thought had crossed her mind as well. When her sister-in-law advised her on how to best capture it, she went to her husband. She wanted to be tested. She remembered the horror on his face clearly, the way he had fumbled and searched for words, the stuttered protests, the unfinished sentences. She had thought then that he had reacted so strongly against the prospect of making her inadequecy so public. She had actually tried to console and reassure him. In the end, he said that he would discuss the matter with his father. 

He kept his word, though the men took their time deliberating. Her father-in-law handed down the verdict a month later that no woman of his house would be examined by city doctors. They should simply try harder. If it was not to be, the shame should remain within the house. 

She could not accept that. She had to know. She had heard that doctors could do things for people like her, that if she were lucky, with a little help, they  could make it as if nothing had ever been wrong. When she saw her sister during the Christian holidays, she asked her to make an appointment for her. Her sister knew how to find good doctors, she trusted her to make the arrangement. Of course she wanted an explanation. She contemplated lying, at least saying that she was acting with her father-in-laws blessings. But they did not lie to each other. She told her of the whispers and her husband's shame, she bound her to keep the matter quiet, even, no, especially from ther rest of the family. Her sister understood. She was glad she had someone to confide in.

It was raining hard when her husband woke her by joining her in bed. Was everyone else asleep? 

Yes, his father and brother had just turned in. Why had she gone to bed so early? His mother was worried about her.

She had something to tell him. She wanted him to know before the rest of the family found out.

He stared at her, as she had stared at him over a year ago. But at that point, she had no inkling of what was coming. Now, he knew, or should have known this day would come. He had told her to bring it about. He spat a word at her. It was the same word she had once told him she did not wish to become. She let it soak into the bedsheet between them. That word could destroy her, it was best not to touch it now. Her husband left for the roof. It was raining, she called after him. He was well aware of that fact. 

He had kept her from storming out of this room a year ago, a few minutes after he found her trying to leave the house. She had made arrangement to visit her sister again for a few days, coinciding with the doctor's appointment she had arranged. Her husband found her as she was putting her clothes in her bag. She had no idea how he found out. Perhaps the doctor had called the house. The doctor had been suspicious of an appointment made under such strange circumstances; a woman wanting to be checked, claiming to be married, but coming in alone, and appointment made by her sister. The entire matter smelled to him of something immoral, or at least illegal, nothing he would want his practise associated with. The implications had made her skin crawl. She left him with her husband's number, and their address. What else could she do?

However her husband had learned of the plans, he reminded her that his father had wished her not to see a doctor. 

She begged and pleaded with him. She told him how this did not need to be shameful anymore, that there was a good chance a good doctor could fix the problem without anyone finding out. She asked him to come with her. She begged him to let her go and ask his father to forgive her disobedience. She pleaded him with her desire for a child, that he allow his wife to fulfill this one natural dream. 

That was when he told her. It was not her fault. It was he who was deficient. 

She stared at him, her jaw dragging on the floor. He had known. The entire family had known. How long had he known and stayed silent, letting her think it was her fault. How long had the entire family conspired against her? She did not ask the questions, though her father had yelled them into the phone later that day when she phoned him, in tears, telling him what she had learned. He had married her to a family of toilet cleaners, he yelled. They had tricked him. They must have known before the wedding. Why else had the dowry been so affordable? This was grounds for divorce. He was going to call her brother and have him look into legal proceedings. She calmed him down and asked to speak to her mother. She did not want a divorce. She was twenty-six. She would never marry again. She wanted a family.

She would never have a child married to this man. She stopped staring at her husband and wiped the tears that had sprung to her eyes.

That was not true. There was another way. 

She did not believe her husband's suggestion. His mouth spoke, but it spouted nonsense. He proposed the impossible.

He did not  understand why she had to take that attitude. Children were born under those circumstances every day. He would accept the child as his own. What was her problem? It was a simple solution.

Her problem? Her problem was that she was not a whore. She was his wife. She would not be pimped around the neighborhood by him.

He slapped her. The shock of the act hurt as much as the blow itself. Her cheek bled where she bit it in surprise. She reached for her half packed bag and headed for the door. He stopped her there, hands clasped, pleading forgiveness. 

She would go to her sister's, as planned. She would not go to the doctor. She would return in a few days. She stepped past her husband to recieve her father-in-law's blessings for her journey.

Her husband returned, soaked, but with a cooler head. She rose to give him a towel to dry off and a change of dry clothes. She should go to her sister's for a few days.

No. Who knew what would happen if she allowed him to push her out of her house. She reminded him that she had done this for him. She reminded him of his professed desire for children. 

He had changed his mind. He climbed under the mosquito net and turned his back on her. 

She did not blow out the kerosene lamp. This was not the end of the discussion. Was this what the months of cajoling and pressure had been for? So he could change his mind?

He had made a mistake. He did not want this as much as he thought he had. MIstakes happen. It was not too late to change his mind. 

Never. She had not put herself through that torment on his whim. This was her child. She had earned it.

Her husband did not respond. The conversation was over. She blew out the flame and listened to the downpour and wondered how this had gone so poorly. 

When she returned from her sister's that time, he raised the proposal again. She refused. She was not that type of woman.

He spent the next week telling her how much he wanted children as well. He told her that she could rescue him from his shame. She could make up for his lack. Only she could make their family whole. She refused. They could live quite comfortably on what they had. She would put more time in at the store. His father's health would not last forever.

Eventually he stopped pleading, but he did not drop the subject. He asked her to sit with him when he watched TV, pointing out how women attracted men in the films. One could learn a lot from watching movies. It was indecent and prurient, humiliating to hear him speak to her like that with the rest of the family around. She stopped watching TV in the evenings, finding reasons to stay in the kitchen late into the evening.

Then came a period of undone chores. His clothes, that he usually washed, he left for her. The garden they tended together, was hers to see to alone. When their chickens fell ill, it was not due to his not bringing home feed, but her laziness and lack of resourcefulness. He made her unwelcome at the store, laughing with the customers about the difficulty in training women to keep a store orderly, teasing her lack of education, accusing her of misplacing objects she had not touched, that she was certain he had hidden. Every week he reminded her that their family would be happier if there were children running around. He would feel certain about his future. 

Eventually, her brother-in-law accused her of theft. He did not want to go to their father, his health would not take this betrayal. But if she did not return the money to the store, he would have little choice. 

She went to her husband. If she did as he asked, would he clear her name?

He did not know what she was talking about.

The money. Would he return the money. 

Oh, that? Of course. He was not a theif.

Neither was she. Would he tell his brother that.

Yes, yes, of course. She had nothing to worry about. She must understand. The family needed children. It was wrong of her to prevent them from having them.

Would he promise again that he would accept the child, and take her back.

Yes, of course. Her questions were growing tiresome.

\vspace{.5 cm}

She did not tell the rest of the family for months, though she told her sister-- everything. Her sister told her that she was welcome to stay with her for as long as she wanted, whenever she wanted. She was grateful, but she could not leave her home, not if there was any possibility that she would not be allowed back.

In all honesty, her family did not find out at first because she said anything. Her mother-in-law questioned her about it four months later. She was starting to show. She made the announcement in front of everyone, in front of her husband, that she was carrying his child. She did not know how he would react. They had not talked about it during these past few months. It was not that he neglected her completely, he simply neglected this part of her. Not once had he asked about her health. She had no idea how he would react to this announcement. 

He remained impassive. The other two men clamoured for a paternity test. Impossible they said. She was untrustworthy, her brother-in-law claimed, hinting at the alleged theft. They piled insults on her father's head, threatened to keep her closer to the house, promised to not let her work at the store.

She thought she would break when her husband quietly claimed paternity. The eyes that looked from her to her husband held less scorn for a passing beggar. 

Watch her tightly, her brother-in-law said, as he left the room. His wife followed, with a look that would have seared a rice paddy. 

Her mother-in-law would see to his medication, her father-in-law announced. He moved his feet from her, when she bent to touch them before departing. His mother-in-law did not say a word.

Her husband met her much later that night, tired and tense. It would be better if she spent the rest of her pregnancy with her sister. 

Was he asking her to leave her house?

No. He would take her and the child back as soon as her health permitted. There was no question of that. 

She could have the child in the hospital in town as easily as she could in the city. 

Why was she so deliberately stubborn? It would be easier for him to smooth matters at home without her being a constant reminder of the possibility of dishonor. He would take her back. He gave his word. Was that not good enough?

She did not trust him. She knew the value of his word. But there was wisdom in his request. She called her sister.

A week later, she stepped off a train in the city. He came with her to see her safely to her sister's house. He would promise to take her back in front of her brother-in-law today.

He was a good man, her husband. She wanted to trust him. In a different world, she would be so happy. She had everything; a comfortable home, a good family, land and a store she enjoyed working in, she even had a child growing inside her. She had everything. Her current situation terrified her.


 
\end{document}
