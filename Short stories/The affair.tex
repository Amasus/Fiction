\documentclass{article}
\usepackage{fullpage, verbatim}


%*****************
% Annotations
\usepackage{soul}
\usepackage[colorinlistoftodos,textsize=footnotesize]{todonotes}
\newcommand{\hlfix}[2]{\texthl{#1}\todo{#2}}
\newcommand{\hlnew}[2]{\texthl{#1}\todo[color=green!40]{#2}}
\newcommand{\sanote}{\todo[color=green!30]}
\newcommand{\egnote}{\todo[color=violet!30]}
\newcommand{\newstart}{\note{The inserted text starts here}}
\newcommand{\newfinish}{\note{The inserted text finishes here}}
\setstcolor{red}
%***************************

\begin{document}
She had everything; a comfortable home, a good family, land and a store she enjoyed working in, she even had a child growing inside her. She had everything. Her current situation terrified her.

\egnote{timeline: 8mo ago, with backstory flashbacks}
It had only been eight months or so ago that she had met him at a hotel near the train station in the city, the type that rented rooms by the hour. She had told her family that she was visiting her sister, which wasn't false. She had two sisters in the city, and two brothers as well. But she didn't need to specify. There was one sibling everyone congregated at for family occasions that didn't rise to the level of taking a train to the village where they'd all grown up. This sister lived in a modest flat, convenient to a commuter line with the added benefit of lacking any of the in-law related drama that made her brother's larger house less inviting. Besides, she and her sister were barely a year apart in age. They'd always been close. She tried to get away as often as she could, as frequently as the livestock and the family store would allow. 

She had stayed with them a month the previous autumn during the dengue fever outbreak, first nursing her brother-in-law, then her niece, then finally her second brother, who preferred his sisters' care to the company of the single men he shared rooms with. Her infant nephew had strong fate lines. He stayed within the mosquito net during the episode, and never fell ill. Several years earlier, when her niece was six months old, her father-in-law gave her permission to fly across the country to baby sit. Her sister was attending her first conference since becoming a mother. She had not set up a child care regime yet. Her father-in-law thought it would be good practice, for her to spend some time taking care of an infant, back when there had been a pretense of a hope of having children herself\egnote{tracking conflict: first mention of problems with fertility}. It had been her first flight. She had worried how she would breathe, since the air was so much thinner at high altitudes. It surprised her to learn that the cabins were pressurized. So she spent a wonderful week with her infant niece in the hot lowlands at the feet of the coastal mountains, returning home more worldly, more in love with her niece and more eager to return to her household chores and her position in the family business.

\egnote{timeline: back to 8mo ago}
So no one batted an eyelash when she said that she would visit her sister again for a couple days. It was the worst time of the year\egnote{worst for what? worst for farming, but best for visiting, which is the current context}, weather wise. The fields were fallow; there were no day laborer's to manage. Her husband had volunteered to manage the store. It was, he argued, a reasonable time for her to take a few days away. She made the arrangements with her sister to see her in the late afternoon, and then with this man after lunch. She stepped off the train and ducked into the station bathroom. In the sodden privacy of the stall, she carefully removed all signs of her married life. She was doing this for her family, she reminded her trembling hands as they wrapped the jewelry in her handkerchief and laid them in the bottom of her bag. She took out a piece of tissue and rubbed at the marks of marriage on her face. It was almost impossible to remove all the traces of telltale color like this. She did the best she could and re-parted her hair to hide the rest, emerging from the bathroom an almost single woman.

Her assignation met her at the station, under the large clock by the ticket counter. They smiled, greeted each other warmly, if a bit awkwardly. He fumbled in his pockets until he found a small newspaper wrapped package. He handed it over and looked at her expectantly. She didn't quite know what to do. She had not brought something for him in return. Would he think less of her for not having thought of it? She'd never done this before. She did not know the rules. But he kept looking at her with that polite expectant smile, and the thin package in her hands was not even tied closed in twine. It would not be particularly expensive she told herself, she would not be under any obligation by accepting this empty handed. She unwrapped a small carved wooden comb. There were birds carved in the handle, with crests depicted by white ovals, plastic, made to look like mother of pearl. It wasn't bad looking. The type of thing her sister might pick up on a whim for her from a market stall on the way home from work. She smiled brightly and assured him that it was perfect and that she loved it. Which wasn't completely a lie. He smiled, relieved, put a hand low behind her back\egnote{,} and lead her towards the hotel. 

\egnote{timeline: flashbacks to first meeting}
He was a pleasant looking man, though nothing spectacular, if it came to that. He was short, but from a family of tall men\egnote{still her POV? So far this feels like a first meeting, so how does she know about his family? brief mention that she already knows him, maybe?}, not too dark, with a thick head of hair. He came from a good family, well respected in his village. All in all, she could have done a lot worse. His brother owned a business\egnote{any reason not to mention that it's a sugar warehouse here?} in the town she caught the train into the city from. He\egnote{pronoun clarity: establish that it's the busniess/brother supplying, not `him'} had recently started supplying her father-in-law's business. That was how they had met, at her family store. It would have been better if she had chosen someone had been completely unrelated to her family, but between her responsibilities at the house, and those at the shop, it was impossible for her to get time away. This had been easier. He had come by to settle accounts with his brother one month on a whim. She'd served them all tea. When her father-in-law and this man's brother had gone to the back room to talk business, she stayed in the front to watch the store. She tried flirting with him. He seemed receptive. It was as simple as that. Only in the movies, and perhaps in the city, could one have an affair with a complete stranger. It was not as if she could ask her city dwelling sister to arrange someone for her in the same way she would happily arrange a marriage for their youngest siblings. 

\egnote{back to 8mo ago, then flashbacks to relationship development}
No, the man walking by her side was good enough for her purposes. He was a school teacher living and teaching in a village about half an hour's cycle ride from the far side of the train station in town. It was about as far from his house as it was from the station. His brother had to take the bus for over half an hour to manage the family's sugar warehouse every day. It was a good thing that he lived so far from town as well. Gossip would find it difficult to travel the near three hours\egnote{the distances and times on different modes of transport between the various different places is complicated} between their houses, changing modes of transportation four times to wag its tongue, she assured herself, while knowing, absolutely that it would be perfectly happy to sit at a tea stall near the train station between their houses and mingle. Still, a good man, from a good family, whose orbit intersected her small world in spite of the relative distances wherein their day to day lives occurred; this was the best she could possibly expect to do.

More than that, he seemed a good man\egnote{just used `good man' - change that or this to escalate here?}, for one who was willing to have a relationship with a married woman. Less than a month after meeting him at her father-in-law's shop, she had convinced her husband to meet him and his brother, the sugar supplier, for a movie in town. He\egnote{her husband, or the man? Puts a very differen implication on  the `arrangement' - is this an arragement with her husband, or the affair? Future context shows it's the former, but I initially thought it was the latter.} had resisted at first. Between the threshing and new livestock, there was just too much to do to afford the luxury of a leisurely afternoon on the town. But she had reminded him of their arrangement, that she would be perfectly happy to go back on her part of the agreement, if that was what he really wanted, and he had relented. To his credit, he did not even sulk during the hour long bus ride into town.

The movie had been syrupy sweet, about lovers meeting across class lines and societal fault lines, eloping, then resolving their decisions with their families. Normally it was the type of movie she would watch with her sister-in-law, in a stolen hours the wives and their mother-in-law could take for themselves once or twice a week after the children were asleep, and the men had gone out to play cards with friends. Her sister-in-law loved a good love story. And truth be told, she didn't mind them either. Her mother-in-law did, however. So she kept it a secret pleasure, for the sake of domestic peace. All the same, given the circumstances, she got no enjoyment out of this movie, even though everything indicated that it should have been exactly what would have made for a delightful afternoon. \egnote{the audience still doesn't entirely understand the conflict, so probably make it clear that her non-enjoyment of the movie is for the same reason as the main conflict and not (say) because it's awkward to go to romance movies with her husband or something}

Afterwards, they sat in a small eatery over tea and snacks. Her husband dominated the conversation with his sugar supplier, discussing prices and futures, supply lines and gas prices, business expansion plans and rumors about how other merchants in their community. This left her to entertain and investigate the fourth member of their party. She pulled her plastic chair closer to the small table they shared, wiped the sweat off her hand and looked at him. She had never done this before. She had no idea where to begin. Best so start with what she knew, and hope for the best. She put on a bright smile, leaned over to pass a plate of savory finger food to the man she had set her sights on, and asked him how he had enjoyed the show.

He had, mostly for the pleasure of the company he was in. 

She blushed at that.

He usually went to see action films with his friends. This made for a nice change of pace.

She did not watch movies in town very frequently, perhaps a few times a year. She enjoyed them in the evening on the TV at home. 

He frequented this theater approximately monthly, perhaps a little more. 

She used to sew during her time in front of the TV. She had a small tailoring business. She'd given it up years ago, her brother-in-law did not like the noise of the machine over the TV.

His sister also had a small blouse tailoring business. He hoped that his sister would be able to make the investment their father had made in her training pay in her husband's home.

Oh, was his sister about to get married?

They hadn't picked someone out yet. But she had just graduated from highschool. Her\egnote{their, to make it clear it's him and his sister and not her father?} father was actively looking.

She wished his family luck in finding a good match. Then admitted that she had given up her sewing business after marriage, when the pressures from the chores at home and the family business left her with little time for private responsibilities. 

He understood completely, and admitted that his sister may\egnote{might} have to do the same.  He spent this early mornings tutoring children from the village, his mornings and afternoons at the schools and his evenings were spent between tasks for his family and helping his brother establish his business. It meant that he had to travel to the city frequently on Saturdays.

She used to tutor children too, when she lived with her mother. Her students were much younger than his. She had never graduated high school. 

That was understandable. He sounded sympathetic. His sister had a few friends who did not either. It was hard, what with ailing parents and rising school fees. The degree didn't mean anything. One of his sister's friends was doing quite well for herself, married to a construction foreman in the city.

She bit her lip, feeling foolish for bringing up the awkward subject so early in their relationship. She changed the subject. Did he have any family in the city? 

No. Did she? 

Yes, now. She had grown up in the north of the state, but now four of her siblings had established themselves there. Yes, there were still two siblings left with her parents, a brother in college and a sister still in school. She was the only one who had moved this far south. They were a close family though. The siblings still managed to get together several times a year, on one excuse or another. Her father's two brothers still lived nearby, and they always congregated for the holidays. Did he come from a large family?

No, it was just the three siblings. But they were close. He hoped they would be able to marry his sister to someone close to home. He would miss her if she moved far away. His father had grown up elsewhere, moved here after college, when he had gone into business with a friend. He had two uncles still in his father's familial village, but they didn't frequently make the long journey across state lines to visit. Isn't it a pity how the traditional large families have declined over the generations for so many people? He envied her her large, close extended family.
\egnote{pace slows a bit in this dialog. Can maybe trim it a bit?}

Her husband told her the time. If they did not leave now, they would miss the last bus. She bade the company goodnight. It had been a pleasure meeting both men. 

Her husband could usually be relied upon to be polite and pleasant to her, with the exception of a few months a nearly a year ago, but that was her doing, more than his. He was no different that night. They discussed their business plans made that night, the foreman's mother's health, and their nephew's schooling, before falling into a comfortable silence well before the bus reached their stop, which she spent reviewing what she had learned that evening.

She was glad the man she had just met\egnote{this sounds like it's referring to their first meeting, rather than `the person she'd just been talking to'} that night was so close to his family. Her family was so important to her, it also being important to him made this ordeal\egnote{unclear at this point what `this ordeal' is, or that there's an ordeal at all. Assuming setting something up is the purpose of the movie excursion, maybe add more tension to making it happen? So far it's unclear how strong/fraught her motivation is. i.e. make it clear, at the beginning of this flashback, that she's setting something up with intent. Build on the reference to her arrangement with her husband?} easier somehow. They seemed to have different choices in movies. But that was okay. They seemed to share other common experiences, and that would be enough. She thought about the awkward moment when they had discussed her education. He had not looked down upon her at all. Perhaps he was just being polite, but many people would not have seen the need to be under the circumstances. The entire conversation had been very pleasant. He had been nothing but polite and courteous to her the entire time they'd spent together. She had married on less knowledge of a man.

That he did not have much family in the area offered her some safety. If something went wrong, there wouldn't be a gang of brothers and uncles and older nephews to come after her, raking her family's name through the mud. Or worse. She liked that he had to travel to the city regularly, that would give her some privacy. It would be hard to convince her own family to let her go to her sister's as often as she would need to see this man, but not impossible. She would just have to be creative. 

He would suit her purposes, she decided by the time she had walked the mile home to find her father-in-law waiting up for them by the television. She gave him his nightly medications and helped him get ready for bed before retiring herself. Her husband was already asleep by the time she had checked on all the animals, locked up the kitchen and climbed into bed herself. She lay there for several hours staring up at the ceiling. This man would suit her purposes. She had come this far with a few others over the last several months\egnote{previously, before the dialog section, `she had never done this before'. Reword here or there for consistency, or to make clear what `this' beyond feeling someone out}, and none of them had quite made her comfortable. Even if he wasn't the perfect choice, she decided, he was good enough that it wasn't worth expending all that effort to try again. The question that remained was, she supposed, how could he\egnote{she} make him want her as well? How, for that matter, did one go about attracting the attentions of a man.\egnote{s/./?} She had been handed a husband to marry, though two of her sisters had chosen their husbands themselves. She contemplated the idea of asking them how they had gone about it before rejecting the idea altogether. Even if she could face the shame of what she was doing enough to explain it to one of them, the skills necessary for finding a man to build a life with, she could only imagine, were very different from the skills needed for this. Or at least, that's what the movies she watched with her sister-in-law seemed to indicate.

\egnote{timeline: 4mo after movie}
In the end, it hadn't been as hard or complicated as all that. About four months after that afternoon at the movies, she found herself standing in a dingy hotel lobby. She was not a frequenter of hotels, she could hardly afford to be. On the rare occasion she went anywhere other than to visit her family, she stayed with friends or family connections somewhere near where they needed to be. There had been the hotel of her sister's conference, of course, where she babysat her niece. And she had gone on that pilgrimage she had gone on\egnote{recommend removing `she had gone on'} with her mother-in-law, but she wasn't sure that she should count the housing provided by the religious association as a hotel. Oh, and there was the time she hotel all of her siblings and spouses had rented on the trip to the ocean front during the holidays a few years ago. But that was the extent of her experience. Even so, she could tell the hotel the man had taken her to was dingy. The floor looked like it was perhaps swept daily and mopped with far less frequency. A layer of grime covered the floor, visible even over the dull grey tiles. Dark brownish red stains speckled the corners of the room and near the trash bins, betel juice tobacco spit, or something similar, she imagined. A lone person manned the reception desk. He sat, surly or bored, on small plastic stool, scrolling on his phone by the light of an intermittently flickering fluorescent bulb. A poster behind him listed available beers, cigarette brands and other \emph{necessities} available for purchase upon request. 

She stood a few steps back, taking in the scene, while her afternoon's companion made the arrangements for a room. A lone ceiling fan spun laconically and ineffectually against the heat. A cockroach scuttled into a crack underneath the stairs. This was, indeed, a dismal, dingy hotel.

\egnote{no time reference about when she managed to meet him again}
But the process of getting her\egnote{`herself', physically making the trip this time? or `them', in arranging for things to get to this point?} to the hotel had not been as hard as she had feared. If she were honest, the hardest part of attracting his interest had been in figuring out what to say. It had been surprisingly easy to convince her brother-in-law\egnote{characters: presumably this is her husband's brother, not her wife's husband. We haven't seen him yet, introduce him as `her husband's brother'? Why doesn't English have more clarifying words for this...} to allow her to accompany him into town on the days when he needed to get supplies for the shop. She spent the long silent bus rides in reviewing scenes from a wide variety of romance films and books that she had watched or read over the years, sifting through the lines for things she could say without destroying her dignity. Everything was infuriatingly awkward. She did not know how to toss her hair flirtatiously, or bat her eyelashes, or pout in an enticing manner.\egnote{I really like this sentence} Nor did she have the time, or inclination, to learn. 

Fortunately, on the first trip to town, she could not find her teacher anywhere. They had not arranged to meet before hand and her she quickly realized that just hoping to run into him in the crowded market square was not the strongest course of action. But what choice did she have? The other option was to ask her husband to act as go between. But her modesty prevented her from being able to make that request of him, irrespective of the arrangement they had come to.

By the second trip into town, she had the scene planned out in her head.\egnote{refer back to how her planned scene didn't work? That she did her part, but the rest didn't fall in line or something? Otherwise this intent doesn't really go anywhere} She followed her brother-in-law around dutifully, carrying packages and keeping track of errands that still needed to be run. She slipped away as often as she possibly could to scan the market square for him, even pretending to have spotted a long lost friend at one point as an excuse to walk past tea stalls and poke her head into eateries to see if she could spot him. Nothing. By the time they had gathered up their parcels and packages and made their way back to the bus stop, she had started despairing of ever finding him again; that she would have to start the entire process over again because of this flaw in her plan.

On the third trip into town, however, they did meet. She saw him sitting at the tea shop across from the shoe repair stall, talking to one of his ex-students.\egnote{how does she know it's a student at this point? I see the purpose to establish doubts about age gap, but it seems like she can't know this yet} The student looked like he was her age, somewhere in his mid twenties. It gave her pause. He taught the early years of the secondary school, he must have been at least ten years older than his students, more likely fifteen. It didn't matter, she told herself. She only needed him for one thing. The age difference was irrelevant. 

She took a moment to collect her thoughts and re-imagine how this meeting was supposed to go. Then she walked over to their table to wish them good day, and found herself fumbling with her words. This embarrassed her, which made her fumble even more. She had no practice with this, she bemoaned\egnote{bemoaned sounds like it was out loud. `bemoaned / lamented internally', `in her head', `to herself'? (none of those seem like quite the right voice)}.  She had not even thought to like anyone before she married. She had been too busy with her parents' farm to bother. How on earth was she supposed to achieve her goal? 

And after marriage? Up until recently, she had been too occupied with her family's household and business to have developed skills in anything else at all, let alone in anything alluring. What on earth was she going to entice him with? What value did she have to give him? If only it was not so crucial that she succeed. 

The teacher smiled warmly at her approach and brought over another chair. He insisted that she join them as she stuttered the required polite objections. He waved over a serving boy and ordered her a cup of tea. She sat with her hand\egnote{hands?} demurely folded in her lap and waited. 

Before the awkward silence extended itself too long, the younger man spoke up. He proudly told her about the beautiful set of atlases his teacher had in his classroom. It was those atlases, he claimed, that had encouraged him to study geography in college. The teacher blushed appropriately, insisting in the obligatory manner that it was the student's work ethic that lead him to where he was. But the younger man protested. He boasted that he now worked with a surveying company, traveling all over the state. It was a good job, well paid, the travel was an added benefit, a lot of opportunity for growth. He was grateful for the inspiration. 

She found herself drawn into the details of the geographer's story. The size of the classroom, the strictness of the teacher during lessons and his approachability out side of school. It was easy to flatter and show interest in the mentor through the younger man's admiration. 

As she finished her drink, she forced herself to remember that her brother-in-law was across the crowded square haggling over glass bottles. She should not be away from him for long, not if she wanted to return again. She put her hand over her glass when he offered to buy her another, and made her excuses. She wished she had more time. She had spent most of her time talking to the younger man. Had that had the intended effect? Had she accidentally snubbed him instead? She wished she could have had some time alone with him.

He rose as she did and smiled warmly at her. It had been a genuine pleasure to run into her again. He could usually \egnote{be} found in town on Wednesdays and Saturdays. Did she come to help her brother-in-law frequently?

She looked down quickly to hide her pleasure at the interest. No, she didn't have a regular schedule, she came when she was needed. But Wednesdays and Saturdays. She would keep that in mind.

Should he help her find her brother-in-law, he offered chivalrously.

No, she could manage. She knew which shops he had to visit. He should enjoy his evening with his guest. She nodded to both men and went back to her normal life.

Five minutes later, she found her brother-in-law struggling with a large crate of bottles, several packages of corks, a parcel of gauze rolls and other sundries. She relieved him of several packages, and helped him organize the rest of his load for maneuverability. Then she headed for the vegetable stalls, while he when in the other direction towards the pharmacist. 

They reunited at dusk for the journey home. Where had she slipped off early in the evening, he asked, mildly annoyed at being left alone to manage for so long. 

She had run into an old client of hers, she apologized. They had fallen to gossiping. It was an easy lie, her brother-in-law had never been interested in her sewing business. He wouldn't recognize any of the clients who had visited his house in the early days of her marriage. 

On the bus ride home, she contemplated her options. Meeting him twice a week would be excessive, she was certain of that. But\egnote{and} she didn't think she could convince her father-in-law of being allowed to make the journey every week. They went into town for supplies every two weeks. She could easily insist on accompanying whoever made the fortnightly supply run. But she feared that he would lose interest in her if she waited that long. 

As she had suspected, it had been\egnote{was} difficult to convince her father-in-law to let her visit town every ten days. Why this sudden desire for the urban? She had no answer prepared for that, so she stayed silent. Instead, she went to her husband, asking him to intervene in the matter. It did not suprise her at all the he did not question her desire or remark on the oddness of the request. He could not make any promises of course, but he would do what he could. He would lobby her cause to his father. 

For the next two months, he\egnote{which he?} took her to town every ten days. She found the teacher in the same tea shop every single time. Reading a book, or a newspaper. Idling away time. Waiting for her. They always drank exactly one cup of tea together, making it last perhaps a little longer than normal. They talked of each other's siblings, of their nieces and nephews, movies they had watched, celebrities they enjoyed. All safe, pleasant, reliable topics. 

They kept up this ruse for two months because that was how long it took for her to work up the courage to mention to him that she would be visiting the city in a week. He was a keen man. She did not have to spell out the implications of this trip. The city meant anonymity and the ability to meet beyond the reach of extended community and wagging tongues. He immediately pointed out that he was free on that date, and wouldn't mind a chance to take a trip up to the city himself. Would her schedule allow for them to meet? 

She blushed and looked down. She couldn't help herself. This was exactly what she had wanted, but it was so wrong. It would, she muttered. Staring at her hands.

He suggested a time that worked for both of them, and then said that they should meet at the station for some privacy. 

She agreed. This was, after all what she had been working towards this long while. And yet. His keenness made her uncomfortable.  

Now, at the hotel, his eagerness as he indicated that he had the room key elevated her heart rate. She followed him up the stairs trying not to think of where she was or the type of people who frequented establishments like this. What was she doing with a man so eager spend a few hours in a hotel room with a married woman? This was never where she had imagined herself to ever end up. 

But she had a goal, she reminded herself. And there was only one way to achieve it. This uncomfortable situation would end soon. And then? She could go back to her old, normal life, to her family. 

If scandal and rumors would let her, a voice in the back of her head despaired. They would, she reassured herself. They had to. She had been so careful. She had chosen a man who was an inspirational teacher who cared for his family. That gave her a certain amount of security. It would have to be enough.

Still, she wished there was another way. But there wasn't. And she was here now. She discretely removed the pin that held her clothing modestly to her body, and stepped nervously over the threshold. 

She was on the commuter train for her sister's house an hour and a half later. She had taken a shower to wash his scent off of her. It would not be appropriate to show up at her family's house smelling of what they had done. He hadn't objected. In fact, he'd wanted to watch. But when she protested, she could not imagine the type of person who could have an audience in such a private setting, he had respected her wished. A good sign, she reminded herself, over and over again, that in spite of his kinks\egnote{kink in the current vernacular sense? If not, probably different word - quirks?}, he did not do anything she was uncomfortable with. Maybe this would work after all. 

They had arranged to meet again in a few days on her return home. This time, he told her to get off at a town halfway between the city and the town where he lived. It would mean he would accompany her on the way home. 

People would talk, she protested. She could not bring shame on her family. 

They wouldn't, he assured her. Strangers got onto trains and sat next to each other all the time. He would read a paper. She could scroll on her phone, or do whatever she wanted. They didn't have to talk. He just wanted to be next to her was all.

It was sweet, or at least, it would have been a very sweet request under different circumstances. But in this situation, it was something that would have to be endured. If she was telling him that she wanted him, she would have to allow him some of his fancies. Objectively, aside from the fear for being found out, it was not such a bad prospect. He was a pleasant man: respectful, interested in her life, clean and tidy in his habits. He was someone whose company she would ordinarily welcome, excepting the one character flaw of sleeping with a married woman. 

No, whatever this current relationship was, she would have to think of a way to end this once she had what she wanted. The earlier she thought that through, the better.

When she reached her sister's house, her niece was delighted to see her. Her school had shut down unexpectedly due to inclement weather, and her parents were both at work. She'd been languishing at home with no one to keep her company but her baby brother. She was old enough to take care of the toddler, and proud of the trust her parents had placed in her. But he was so annoying. They couldn't leave the house because of the house\egnote{typo - heat?}, and he'd been cranky all day. She'd just convinced him to take a name\egnote{nap?} in the main room, which left her exiled in the kitchen. She was bored. And lonely. Could she ask her aunt to go in there and quietly fetch her favorite doll and some colored pencils? Please?

She returned, with the requested doll, a pencil case full of colored pencils and a notebook still half filled with blank pages, and settled in the kitchen to keep her niece company. She was such a responsible girl, she deserved some coddling every once in a while. After the ceremonial purse searching to see what her aunt had brought her (a small wooden hair pin)\egnote{wait is this the same comb he gave her at the station? If it is and you want people to notice, it's probably too subtle. If it's not and you don't want people to think it is, you're probably fine, I only noticed while making the strict timeline}, the girl placed herself cross legged on the floor to comb her doll's hair. She in turn, started washing and peeling cucumbers for a mid morning snack. Her niece babbled happily about all aspects of her life since her aunt had last visited. About her new teachers, the ones she liked and the ones she hated. About her friends, who was still with whom, and who was not\egnote{now?} a sworn enemy of their gaggle. When she placed a platter of salted vegetables and other easily comestibles she could scrounge from her sister's pantry between them, the girl kept talking between happy mouthfuls. She showed off her English and brought out all her school books. English was not her favorite subject, but her father insisted that she pay special attention in that class. He would even buy her doll a new outfit if she did well this year. The aunt, never having spent much time in urban settings, had little use for English. So she let her niece show off her superior skills and complimented her on her brilliance. 

When the toddler awoke, they brought in the laundry while the boy toddled beside them on the roof. Her niece showed off new tricks she had learnt with her jump rope between the clothes lines while she folded laundry and made certain the boy did not get up to anything dangerous.

Eventually, her sister came home from work. Her daughter gave a desultory rummage through her pockets for a piece of candy she knew wouldn't be there, kissed her mother, and went off to do her homework. Mother and aunt took turns watching the baby and preparing dinner. Her sister complained about politics at work and gossiped about the neighbors. She listened. For her own part, she talked about her father-in-law's health and a recent argument she had with her sister-in-law, about their neighbor's cow being ill\egnote{,}\egnote{if argument is only about the cow and the fish are another conversation point} and the prospering fish stock in their pond. She did not tell her about the afternoon's events. Her sister knew her situation, of course.\egnote{it's not yet `of course' for the audience} At least, she knew what had instigated this drastic course of action, but she could not bring herself to update her.

Her brother-in-law came home. He asked after her family's health, and talked about recent developments at his work. He had been passed up for that promotion, but was thinking of enrolling in night school. 

He sat with his daughter's school work, while she and her sister caught up on the doings of the rest of the family. Their youngest sister, still in school, was applying for the position at the prestigious mission school for 11th and 12th grade just outside the city. Her academic career was phenomenal, she was clearly the most successful of all the siblings. 

Even so, she pointed out while pouring the extra water out of the rice pot, a position at such a good school would make it difficult for their parents to find a good match for her. 

Her sister chided her small mindedness. 

It was not small minded, she retorted. She was concerned, that was all. 

Her sister reminded her that their youngest sibling was capable of finding a husband for herself. 

She decided to change the subject, wishing her the best in her applications. It would be good to have her closer to the city, instead of living in the north of the state. 

Their youngest brother was with their parents for the college break. He had called two days ago. He found it easier to study for his exams in the quiet of his parents' house. 

She was glad that he was taking his exams seriously again. He had not always been to dedicated to his studies.

She fed her niece in front of the television while her sister put the boy to bed. Then she put the girl to sleep and neatened the apartment, giving her sister some time with her husband. Eventually, she joined the adults for dinner.
 
This was such a comfortable routine. She had been welcome in her sister's household for as long as her sister had had a household to call her own. Her brother-in-law was all that one could hope for, he never made any of her\egnote{his wife's?} siblings feel like they were anywhere other than their own home. She knew where books and clothes and keys lived in this house, what her niece's school bus schedule was, how much of a tab the family had run up at the local grocers. There had been differences over the years, of course, that was true of all families. But all of the siblings had always been close, and she had always been very fond of this sister. They had grown up a large family in a small house. It was impossible not to share everything. 

A year ago, she could not have imagined ever keeping a secret from this family. Yet as she mopped the floor around where they had eaten, she could not find the courage to say the words. Her sister stood to do dishes. She took them from her and helped her dry. But she did not say a word about the morning. 

Her brother-in-law came in to tell them that he had hung the mosquito nets. It was time to go to bed.

\vspace{.5cm}

\egnote{timeline: when is this? probably want to establish in first paragraph, not second}
The boy who tended the herd failed to show up today\egnote{`today' may confuse timeline comprehension} for unknown reasons of his own. Therefore, it fell to her to bring the goats in from the rain. The stench of their shed was overpowering today. She did not know if it was because the boy was not cleaning the stalls properly, or if it was just her. There had to be a way to find out\egnote{a sneaky/surreptitious/casual? way to find out?}. Maybe she could send her sister in law to feed the herd tonight and see if she said anything.

Her rendezvous with the teacher had lasted a little over three months. They had met in the city only that once. In part, this had been a relief for her. She did not have enough cash saved from her sewing days to pay her share of an indefinite number of hotel rooms. She had always insisted on paying her part, she did not want to be beholden to his man afterwards. But the fact of the matter was, the expense would quickly deplete her saving\egnote{savings}. She could have asked her husband, she had suspected. He had always been willing to take her side in all other aspects of this ordeal. However, she'd never asked for financial help. That felt like crossing a line somewhere.

So it\egnote{`it' is only going to the city once? or the end of the affair? should specify that} was a relief, but only in part. Because the other alternative had been terrifying\egnote{clarify what the alternative is (I think it's location hopping?) and why it's terrifying (discovery?) briefly before giving all the detail? The following description is a little long to get through to understand the `other alternative'}. The teacher seemed to have an endless network of college friends who would turn a blind eye, or colleagues away on vacation whose houses he could use. So they spent their brief liaisons in a variety of suburban flats, small town apartments and even, a several petrifying\egnote{unclear wording. `On several petrifying, but brief, occasions'?}, but brief, in cow sheds in her own village. Her husband had covered for her frequent absences every time. She never asked what he told the rest of their family, and he never offered. Whatever he had said, they never asked her about where she had been. This worried her, but, if he was going to manage this his way, it was not her place to question it.  

She did not like meeting at the teacher's associates houses; it was too public, to dangerous. Too sordid. She did not like trailing her desperation around for his circle of friends and acquaintances to see. It didn't matter, he assured her. None of them knew her, and, unless she wanted otherwise, none of them ever would. Besides, their discretion was beyond reproach. They had all been in the same position. 

That took her aback. What type of woman had she become to depend on the honor of men who managed their affairs with other women as easily as they managed their green grocer's bills? But in the end, she had relented. It was true after all -- she was desperate.  

She had wondered, sometimes, in the quiet hours of the night what she was doing. If the man she secretly met had such a wide circle of well experienced friends, did this mean that he had done this before? Who had she entangled herself to\egnote{with?}? But she had done her homework. She had asked about his history before embarking on this misadventure. If he had done this before, he had kept his other women quiet as well. She had not been able to uncover any stories about any bad habits, let alone any affairs, when she'd chosen him.

All told, it was a great relief when she finally ended their meetings a week ago. She called it off, as soon as she was certain she had gotten what she had come for. She told him that she had enjoyed their time together very much, but that her family was, in spite of all their precautions, starting to get suspicious. She had always told him that she would not risk her family's name for this. She hoped he understood. 

It was a small lie, designed to let him down easily. Which it did.

Of course he understood. She had always been clear. Indeed, he admired the way she wanted to protect her family's honor through all of this. It made her different. Attractive. 

They parted on amicable terms, vowing to be strangers if they ever met again by chance. It was risky, she knew. On so many fronts. It was so soon. She may\egnote{might} lose her prize, at which point she would have to go through a similar ordeal again. And while she trusted him not to seek out gossip about her, she did not know what he would do if gossip found him. Would she be able to keep a second ordeal as quiet as she had kept this one. 

Initially, she had contemplated staying in her current situation until her position was more certain, but the humiliation of it all was more than she wished to bear. Sneaking out of her house, neglecting her responsibilities, copulating \egnote{in} a stranger's bedroom. She couldn't keep it up any more. Rightly or wrongly, she chose the riskier path that offered immediate solace. 

So she kept herself from gagging at the smell of the goat stalls and went to the kitchen to get herself a glass of water.

The rain had prevented her from going to the store today. Therefore, when her husband came home a few hours later, he found her chopping vegetables on the veranda, as far as she could be from the newly nauseating smells of her mother-in-law's cooking as she could be without arousing suspicion. 

He had come home muddy and soaked, in spite of his umbrella. The rain was simply torrential. His father and brother would be along shortly, he told her. They were visiting one of their neighbors. The neighbor's boy wanted a position at the shop. The kid was both intelligent and honest, by all accounts. But at the moment, the didn't have need for more hands at the store. Anyway, they would discuss it over dinner.

She listened carefully and nodded. She had not told her husband about the new state of affairs. Indeed, other than the fact that she had not asked him to make excuses for her in about a week, she had gone out of her way to not give any outward sign of anything having changed at all. She wasn't exactly certain why. After all. They had an arrangement. They had both agreed on this course. It had been his idea in the first place. But somehow, she could not bring herself to tell him. Just like asking for financial support to see this through, she sensed that she would be crossing a line somewhere.

Instead, she told him that his mother was almost done with dinner. She dropped off the vegetables she had working on, and excused herself to draw and heat
water for all three men to wash with when they came home. 

When she returned, she found her husband fiddling with the wires at the back of the TV. She reminded him gently that he probably should not turn it on, the rain kept the solar panels from charging much today, and they wanted to conserve energy for the fridge. Instead\egnote{instead of what? of telling him, presumably, context makes it sound like instead of telling him not to turn the TV on}, she sat down next to him and asked him about the takings at the store instead. The talked, companionably, waiting for the water to heat and the other men to arrive home. It was an easy conversation between them, as it usually was. And yet, she did not tell him.

\egnote{timeline: flashback to beginning and early days of marriage}
She had been hopeful when she first came to her husband's home seven years ago. Unlike her sisters, she did not want to live in the city. When it came time to find her a husband, she had explicitly asked to be married to a rural family. It did not matter if it was large or small. She did not need to be the eldest wife. She had grown up on farm, and that was where she was comfortable. She just wanted to live her married life out in the same general environment that she had come to love. 

Her husband's house was remote enough that her parents and siblings were not likely to visit her often. And even as her siblings left home and started migrating to the large city about half way between her parent's home and her husband's, it quickly became clear that even they would not leave the comforts of their electricity and running water to visit her as frequently as she wished to see them. So she resigned herself to doing all the traveling that her new family would permit to and let the matter stand. Her father-in-law had a successful local business selling dried goods and homeopathic medicines. They family had a modest amount of land, enough to grow rice and vegetables for the family with a little left over to take to market. She was pleased with the set up, even if it meant living far from family. It was familiar. She could help them prosper. All in all, her parents could have done worse by her. \egnote{this section feels like beginning-of-story scene setting. Reword, probably, or maybe move to the beginning?}

Her father-in-law had been the first to see the advantages of her work ethic. With her parents, she had tutored young children in the morning, gone to school herself in the late morning, and then did what needed doing around the house and lands all afternoon and evening. This left her a few hours after supper to study by lamp light. She'd never made it through high school, but that didn't matter for her future. She replaced her formal education with several hours a day with a neighbor, learning to sew, and then started bringing in enough money to buy her own chickens for the house. There was no reason she could not do for her husband's home what she had done for her father. 

Her sister-in-law, by contrast, the elder brother's wife, was better educated than she, having graduated from a well known boarding school. But she showed no interest in anything other than stay by her mother-in-law's side. She was a good cook and a very competent housekeeper, but that was where her interests ended. When she married into the household as the second wife, the animals, the garden, it all fell to her.

Her sewing business ended in six months. Whether they didn't like the traffic of people coming to the house, or whether they didn't like her earning money on her own, she couldn't tell, but after the first few months, she had to put her sewing machine away. Instead, her father-in-law took her under his wing, and she spent two evenings a week at the store. Organized and orderly, she made up for her inability to keep books by her ability to know where goods were in the shop, and what needed to be restocked without referring to a ledger. She quickly became his favorite. A third son, he would jokingly call her sometimes, which embarrassed her to no end. 

This caused some tensions with the other women in the household of course, and the easy conversation she had come to expect from her mother-in-law and sister-in-law became stilted and, eventually, tense. She told her mother of this progression of events over the holidays, worrying that she had done something wrong to have destroyed what had seemed a very fortunate marital situation. 

They are just jealous, her mother had said. Why wouldn't they be. She had one foot in their world, and the other in their husbands'. She was lucky. 

Jealousy, she knew how to tackle. She deferred to them in all matters domestic, even ones where she was perfectly competent or knew better than her sister-in-law. She made certain their husbands knew of their contributions to the smooth running of the household. She stayed quiet when the men spoke about business over dinner. She could always tell them her opinion in private, to her husband in their bedroom or the others at the store. There was no need for her to contribute to the conversation in the larger setting. Eventually, her mother-in-law was appeased by her docility and deference. 

Her sister-in-law took longer. But then, she gave birth to a boy. Something she\egnote{confusing pronoun} had not managed to do, even five years later. The five year old, and then, his three year old brother sat in the center of their grandfather's vision, the pride and joy of the household. Their mother basked in the glow of that light, easing the tension between them, and restoring her to the elder, and thus more important wife.

For her part, she did not feel that the boys had not\egnote{delete} diminished her role in the family. She could never accuse the two charming bundles of trouble of that. Even if they had, she would have gladly moved over to let them bask in their grandfather's love. But they had not. 

Instead, one could say that their births had indicated a shift from her playing a central role in the household to a more central role at the business. As mother and grandmother mooned over the two boys, she slowly ceded ground in the house and spent more and more time in the garden or looking after the livestock. When her husband suggested that she spend every afternoon at the store, is seemed like a natural idea. She wasn't needed around the house as much, and her time at the store was more profitable than her investments in the garden. 

It was not where she had foreseen her life going when she had married, of course, but one cannot argue with what is written in one's fate. She had always imagined herself with a large brood of children of her own, wiping noses and chasing them off to school, serving out large ladle fulls of porridge when they returned home famished from an evening herding cattle. But, if that was not for her, there was not much she could complain about with what she had. Her husband was happy with her new position in the household. There was even talk, a few years ago, when his father's health had given them all a scare, that the land would be given to the elder brother, and the business to the younger. Her husband had pointed out to her that this was, at least in part, due to her active presence in the shop. He'd been proud of her, of that she was certain. Very few wives could boast of that. But somewhere, she felt ungrateful for not being happy with what she had been dealt.

The elder of her two brothers, a clerk for a lawyer, told her she should be proud that her father-in-law held her in such esteem. But she was not her brother, she did not live in the city. Her ambitions had never been to own a successful business. She was a farm girl. She had always been a farm girl. She had wanted a farm wife's life. Therefore, when her father-in-law's heart recovered, she was doubly grateful: first, that this man she loved had recovered, but second, and equally importantly, that she could continue with a life that she wanted. The doctor told them that he had to be careful of his diet and activity in the future. She made note of his instructions carefully, and took it upon herself to ensure that he followed them to the letter.

\egnote{timeline: abrupt end to flashback. Indicate that we're back to `now'?}
After dinner, she sat with one of the neighbors' daughters and her nephew as they did their school work. She quizzed the girl for an upcoming test while she mended the boy's school uniform while his mother put the youngest to sleep. Whether it was because she was feeling off or because the girl was quickly exceeding the limits of her own knowledge, she found herself snapping at her student frequently. Which was not like her. She had loved working with the younger children when she was in her parents home. Whether it was the tutoring that gave her the dream of having a large family, or the experience of watching out for her many younger siblings that made her a patient tutor, she did not know, but she was never impatient or unkind to her students. Eventually, she closed the book and told the frazzled child to go home. She had done enough for tonight. She would be better served on her test by a good night's sleep.

Then she put away the half patched uniform and excused herself for the night. She was tired; her mind on the conversation she would have to have with her husband. She told her nephew to behave himself until his mother returned and went to bed. There was\egnote{were} still more chores that needed to be done. She would tend to them in the morning.

\egnote{timeline: flashback to two years ago and fertility discoveries}
She had not thought the issue worth bringing up until nearly two years ago. They had been married five years. Attempt after attempt had failed. Her siblings had children. Her sister-in-law had children. Wives that had come into the neighborhood years after her had children. But she did not. 

Eventually she started hearing the unavoidable whispers during the holidays. Her neighbors thought that it was her fault. Either that she was cold and ambitious, choosing to spend time with her father-in-law \egnote{add `rather'?} than her husband, which is disgusting and unkind, or worse, that there was something wrong with her, and that she was a person to be spurned, or pitied, or both. Though, to be honest, the second thought had crossed her mind as well. It wasn't that they weren't intimate. They were. It was just nothing ever happened.

When her sister-in-law pulled her aside one day to suggest that they were possibly doing it wrong, and advised her on how to best capture it, she went to her husband. She couldn't take the humiliation and whispers any more. She wanted to be tested. She wanted to know why they hadn't had a family and what they could do to change that. She remembered the horror on his face when she'd made this demand clearly, the way he had fumbled and searched for words, the stuttered protests, the unfinished sentences. She had thought then that he had reacted so strongly against the prospect of making her inadequacy so public. Yes, she'd be potentially be putting the family name to shame. But it wouldn't be his blood that was at fault. She'd been touched that it at affected him so much. She'd spent that night consoling and cajoling and reassuring him. In the end, he said that he would discuss the matter with his father. For the time being, she let that be enough.

He kept his word, though the men took their time deliberating. Then, about a month later, while her sister-in-law was putting the boys to sleep, her father-in-law came out to the veranda where she and her mother-in-law were mending baskets. He wanted to discuss something with them, he announced and called them into his bedroom. Her husband and his brother were already congregated there. She had tried to make eye contact with her husband to gauge the situation, but he wouldn't look at her. Bad then. She took her place beside him and awaited whatever decision the men had come to. 

They had discussed it, in detail, her father began, from all angles. The conclusion they came to was inescapable. No woman of his house would be examined by city doctors. If word got out about the reason for such extreme measure, the shame on the family would be too great. He could not permit it. She and his son should simply try harder.

Her husband looked like he wanted to be swallowed whole by the earth itself. His brother gave him a look that was equal parts disgust and pity. His father sat stone faced and determined in his verdict. 

She was stunned. A part of her wanted to reach out to her husband to comfort him. Whatever he had given to try to win for her this treatment possibility, it had clearly cost him. She did not like to think that she was the source of his humiliation. 

But most of her simply just didn't understand her father-in-law's decision. Yes, this outcome had always been in the realm of possibilities. She was not so naive to think otherwise. There were some families, even in her neighborhood who would find it perfectly reasonable to return a wife deficient in this manner back to her parents. But those types of families did not let their wives run the family store when the men had to step away. Her father-in-law was not a conservative man. She could not understand why he would react like this. Why could he not see that anything the doctor found would look worse for her than for his family.\egnote{s/./?} The fault would not lie in his line, but belong to an outsider they had brought in. There was very little risk to him or his line.

She let herself brood on the contradiction for a long while. Bringing it up at home again would do no good. Yet, she wanted this. She had heard that doctors could do marvelous things for people like her, that if she were lucky, with a little help, they could make it as if nothing had ever been wrong. She could have a family. It wouldn't matter if they were all girls. They would be hers.

When she saw her sister during the Christian holidays, she confided her suspicions, worries and desires. She asked her to make an appointment for her. Her sister knew how to find good doctors, she lived in the city. She'd brough her parents in for treatment before and nursed them to health in her own home. She trusted her to make the right arrangements. 

On the train ride up, she had considered how to frame the request. She could, of course, have simply said that this was an action she was taking with her father-in-law's blessings. But they did not lie to each other. If she wanted her sister's help, her sister would have to know everything. So she told her of the whispers and her husband's shame, her father-in-laws stony verdict and reminded her of her own yearning to be a mother. She bound her sister to keep the matter quiet, even, no, especially from the rest of their family. Her sister understood. Not viscerally, of course, she'd never gone through anything remotely similar, but they knew each other well. Her sister could read the anguish on her face. She would do what she could to support her. 

It was good to have someone to confide in, she confessed. 

Her sister had smiled and handed her a plate of freshly fried dumplings.

\egnote{timeline: back to `now', again a little abrupt}
It was raining hard when her husband woke her by joining her in bed. Along with all the other changes happening in her body, she was sleeping very lightly, waking up frequently at night and having vivid dreams.

Was everyone else asleep? she asked,\egnote{haven't used `said', `asked', etc. before much. Rephrase to keep consistency?} rolling over to greet him.

Yes, his father and brother had just turned in. He had stayed up later talking to his mother. Why had she gone to bed so early? His mother was worried about her.

She sighed and propped herself on an elbow. She couldn't keep it secret much longer. And it wouldn't be right for him to find out because someone in the family had come to him with their suspicions. Who knew what stories would be spun and believed under those circumstances. No. He had better find out directly from her. And the present seemed like an appropriate time.

He jut stared at her, dumb founded, absorbing the revelation. It reminded her of how she had stared at him, a little over a year ago, uncomprehending that such a thing could even enter the realm of possibility. But at that point, she had no inkling of what was coming or why. She'd been kept completely, and intentionally in the dark. In his case, he should have known this day would come. They had planned it together. No, had told her to bring it about. He knew every step she had taken to execute the plan. She'd kept no secrets from him.  

He spat a word at her. She let it soak into the bed sheet between them. It was the shock, she reminded himself. She was tired and half asleep. He had caught her off guard. She must not have told him gently enough. 

He stormed out of the room, announcing that he was going to the roof. 

It was raining, she called out after him.

He was well aware of that fact.

She lay back on her pillow, with her face still wet from the spray of his anger. That word, it could destroy her. How could he use it on her now. After everything. It was the exactly same word she had told him a year ago that she did not wish to become. A word he had assured her then that she would never be.

\egnote{timeline: flashback to a year ago, broaching the plan}
It had been raining that night too, when he had kept her from storming out of this very room. In actuality, she\egnote{he?} had found her trying to leave the house entirely, in secret, without making a scene. She had made arrangements with her sister to come up for a few days, coinciding with the doctor's appointment they had arranged. Her husband found her as she was putting her clothes in her bag. She had no idea how he found out. She had made arrangements to go to her sister's house as usual: found someone to tend the animals and garden; made certain that the men were able to run the shop without her; seen to it that her mother-in-law did not have anything extra she needed doing around the house in the next few days. But she had not told them why she was going. She had enough latitude in the household that she frequently did not need to give a reason. As long as her absences were infrequent enough to not disrupt the regular running of affairs.

Perhaps the doctor had called the house. As much as she had not wanted to, in the end, she had given the doctor her husband's number. It seemed like there was no other way to assuage the man's concerns. The doctor had been suspicious of an appointment made under such strange circumstances; a woman wanting to be checked, claiming to be married, but coming in alone, and appointment made by her sister. The entire matter smelled to him of something immoral, or at least illegal, nothing he would want his practice associated with. The implications had made her skin crawl. So she left him with enough information to upset the entire set of carefully laid out plans, her husband's name, occupation, that of her father-in-law as well, their phone number and address. What else could she do?

Whether it was due to the doctor's phone call, or some other overlooked detail had given her away, her husband had appeared, panicked and distraught. What was she doing, he asked her. His father had explicitly admonished her not to see a doctor. 

She tried to calmly continue packing her clothing until he placed himself physically in the way. Then she pleaded with him to move. She told him how, in this day and age, this did not need to be shameful anymore, that there was a good chance a good doctor could fix the problem without anyone finding out. She asked him to come with her, it would be easier if they faced it together. Less scandal, certainly, but also, he'd be there when she found out. Less time to worry. Wouldn't that be better? She begged him to let her go and ask his father to forgive her disobedience. She pleaded \egnote{add to or with} him about her desire for a child, that he allow his wife to fulfill this one natural dream. She'd been a good wife to his family. Why would he deny her this one thing?

That was when he told her. He'd whispered it at first, so she hadn't fully heard him. But even after he'd said it louder, she still couldn't comprehend. He had to say it a third time. It was not her fault. It was he who was deficient. 

She stared at him then, as he had just stared at her now, her jaw dragging on the floor. So many emotions struggled to make themselves known to her that they snarled themselves together, and she stopped being able to think altogether. Like with a knot in a bobbin shuttle, her mind ground to a halt and she just stared. After a moment, she turned on her heel and tried to storm out of the room. He stopped her.

He was the one begging this time.\egnote{`this time' implies current. `He had been the one begging then', maybe?} For her to stay, not to go to her family. Not to tell them of his shame. Not to drag his, no their, family through this. It could all be avoided, he claimed. They could make this right without anyone knowing. They'd find a way, he was certain.

How? She would never have a child married to this man. She stopped resisting her husband and wiped the tears that had sprung to her eyes.

That was not true, he said after a pause. There was another way. 

She looked up at him with disbelief mixed with hope. There was only one way children were brought into this world. And he could not do his part. What other way?

She did not believe her husband when he told her his plan. His mouth uttered words, but nonsense reached her ear. She could not understand how he could even contemplate such an idea. She was not that type of woman. He was proposing the impossible.

Her indignation upset him. He did not understand why she had to take that attitude. Children were born under those circumstances every day. What right did she have to be so high and mighty? He said that he would accept the child as his own. What else mattered? It was a simple solution. What was her problem?

Her problem? Her problem was that she was not a whore. She was his wife. She would not be pimped around the neighborhood by him. She'd spat the word at him then, as he had today. She was a decent woman. She was a moral woman. She had given everything she had to this family. She would not let him turn her into that.

He slapped her. The shock of the act hurt as much as the blow itself. The inside of her cheek bled where she bit it in surprise. She picked up her half packed bag and headed for the door, calmly this time. With all the dignity of a wife who had earned a place of honor in household she'd made prosper. 

He stopped her again, hands clasped, pleading forgiveness. He had not meant to hit her. He didn't know what had come over him. This was not who he was. After six years, surely she must know that. Would she stay. Please?

She eventually relented. She would stay, eventually. But she would leave today as planned, too. She would go to her sister's house, as planned. She would stay there for a few days, as planned. She would not, however, go to the doctor. That was pointless under the circumstances. She did not ask his permission for this. She told him what was to be accepted. Then, she stepped past him to receive her father-in-law's blessings for her journey.

They talked about it at her sister's house, after the children were asleep. Her sister, her brother-in-law and herself huddled in the kitchen around a mosquito coil, considering her options. He\egnote{`he' being her husband, and not the brother-in-law?} had known, that much he had admitted. The entire family had known. Her father-in-law had known. They had all known when they had forbidden her to seek treatment. But no one thought fit to let her know, as if the matter did not concern her.

How long had he known and stayed silent, letting her think it was her fault.\egnote{s/./?} How long had the entire family conspired against her? If this was how they were going to treat her, after everything she had done for them, there was no reason for her to take the fall for their family honor. The next morning, she would tell her parents. 

Her father yelled at her through the phone. He had married her to a family of toilet cleaners, he yelled. They had tricked him. They must have known before the wedding. Why else had the dowry been so affordable? This was grounds for divorce. He was going to call her brother and have him look into legal proceedings. He was incensed, and furious. If he were a younger man, he would take the train down himself to give her father-in-law a piece of his mind.

Far from making her feel protected, these threats of violence unnerved her. She had promised that she would return to the household. She still had to live with her husband. She calmed her father down and asked to speak to her mother. She did not want a divorce. She was twenty-six. She would never marry again, surely her mother knew that. She wanted a family. They had to find a solution that would allow her to have that.

\egnote{timeline: back to `now'. This time change is particularly abrupt}
Several hours later her husband returned to their bedroom, soaked, but with a cooler head. She rose to give him a towel to dry off and a change of dry clothes. 

She should go to her sister's for a few days, he said eventually.

No. Panic rose in her chest and froze into a stony armor. Who knew what would happen if she allowed him to push her out of her house. Honest wives were sent away from their marital homes on all sorts of innocent pretenses never to be allowed to return. Her husband had never before given her reason to fear that type of treachery, but he had not taken this news well. No. She could not trust this alliance that far. 

She reminded him that she had done this for him. That this had been his idea all along. To save face for the family as a whole, but also to keep neighbors from talking about him as well\egnote{`also...as well' redundant}. She reminded him that he too wanted children. That everyone needs a family to help look after them in their old age. She reminded him that he had met and approved of the father of the child. She reminded him of ever detail of the agreement they had come to. Except for what he had done to her to make her comply. That, she suspected, would not help her argument at all. She reminded him that he had promised, at the very outset, that he would accept this child as his own. 

He had changed his mind, he said abruptly. He climbed under the mosquito net and turned his back on her. The conversation, as far as he was concerned, was over.

She stood there, watching his tense shallow breathing. He was not going to fall asleep any time soon. She climbed into bed, but did not blow out the lamp. This was not the end of the discussion. Did he remember, she soothed, how much he had wanted this a year ago? How he had begged and pleaded an cajoled her? How he had won her over to his point of view? That wasn't the truth of the matter, from her perspective, but at the moment, the truth didn't matter. What mattered was that her husband honor his word. He had been so clever in coming up with this scheme. Why throw all of that away by changing his mind?

He had made a mistake, her husband retorted, without facing her. He did not want this as much as he thought he had. Mistakes happen. It was not too late to change his mind. 

Never, she thought. She had not put herself through that torment on his whim. This was her child. She had earned it. He did not get to take that away from her.

Her husband remained silent, and she did not have words to win him over. She stared at his back for a very long time, adrift in her own worries and fears. Eventually, his breathing deepened and evened. His body twitched as it gave in to sleep. The conversation was, indeed, over. She blew out the flame and listened to the downpour outside. She had been so careful about everything. How on earth did things turn out so poorly. 

\egnote{timeline: a year ago, convincing her about the affair}
When she returned from her sister's a year ago, her husband raised the proposal again. She continued to adamantly refuse. She was not that type of woman. She wanted a family. Desperately. But not at that price. It would not happen.

He spent the next week telling her how much he wanted children as well. Every time they passed the neighborhood elementary school, he would regale her with stories of the mischief he had been up to at that age, how his father had dealt with the situation, how he would deal with it differently. She listened politely, laughed even, at the antics, but remained unmoved. 

He told her that she could rescue him from his shame. He told her of the woman he was supposed to have married before her. How her father has insisted on a fertility test before agreeing to anything. Of how humiliating and upsetting it had been to be rejected at the same moment that he was learning about his shame. Of how much it still bothered him. That she could make up for his defect. She could save him, if only she were willing. Only she could make their family whole. 

She listened politely, feigning sympathy. Still, she refused. 

They could live quite comfortably on what they had, she had argued. She would put more time in at the store. His father's health would not last forever. Eventually, they would take over the store, and without little mouths to feed and wash and school, she would have the time to make the business grow. They could put money by for their old age, so that it would not matter that they did not have children to care for them.

Eventually he stopped pleading, but he did not drop the subject. He asked her to sit with him when he watched TV, pointing out how women attracted men in the films. One could learn a lot from watching movies. It was indecent and prurient, humiliating to hear him speak to her like that with the rest of the family around.\egnote{`with the rest of the family around'? Do they know about this, or does he just think they won't notice?} For those few weeks, she stopped watching TV in the evenings, finding reasons to stay in the kitchen late into the evening.

Then came a period of undone chores. His clothes, especially his under things, which he usually washed, he left for her. The garden they tended together, was hers to see to alone. When their chickens fell ill, it was not due to his not bringing home feed, but her laziness and lack of resourcefulness. He made her unwelcome at the store, laughing with the customers about the difficulty in training women to keep a store orderly, teasing her lack of education, accusing her of misplacing objects she had not touched, that she was certain he had hidden. 

He stopped touching her in bed. They did not talk about their days anymore. If she tried to make conversation, he reminded her that their family would be happier if there were children running around. That he would feel certain about his future. That all he was asking was for his wife to do her duty. 

Eventually, her brother-in-law accused her of theft. He did not want to go to their father, he had said, his health would not take this betrayal. But if she did not return the money to the store, he would have little choice. 

She went to her husband. She had not stolen the money, and if, indeed, money had actually gone missing, she had given her family no reason to suspect her of theft. There was only one possible source for this new assault. If she did as he asked, she said to him, would he clear her name?

He did not know what she was talking about, he said calmly.

The money. Would he return the money. 

Oh, that? Of course. He was not a thief.

Neither was she. Would he tell his brother that.

Yes, yes, of course. He waved away her concerns like an annoying mosquito. She had nothing to worry about. She must understand. The family needed children. It was wrong of her to prevent them from having them.

Would he promise again that he would accept the child, and take her back after she had done what he had asked.

Yes, of course. He looked bored, and shooed her out of the room. Her questions were growing tiresome.

\vspace{.5 cm}

\egnote{timeline: clarify which timeline we're in}
She did not tell the rest of her husband's family for months, though she told her sister everything. 

Her sister told her that she was welcome to stay with her for as long as she wanted, whenever she wanted. She also promised not to tell their parents or siblings until she was ready to say anything. 

She was grateful, but she could not leave her home, not if there was any possibility that she would not be allowed back.

Her sister understood, and sympathized, and told her that if she ever needed her to come down,\egnote{pronoun confusion. Clarify that this is the sister offering to visit her in the country} just to be there, she would be happy to take a few days of vacation. 

That was comforting. To know that she could have someone in her corner, in the same room as her, and not in the far away city. But as hard as the next few weeks were, she did not want to call her on that favor. She wanted to save it for when she really needed it.

At the end of the day, her marital family did not find out at all because she said anything.\egnote{technically they did find out because she said something, but her hand was forced} Her mother-in-law questioned her about it several months later. She was starting to show, in spite of her best attempts to hide her changing body in the folds of her clothes. 

So, in order to prevent her mother-in-law from telling the story however she would tell it, she pre-empted her at dinner that night. She made the announcement in front of everyone, in front of everyone\egnote{intentional duplicate?}, her husband, father-in-law, brother-in-law, mother-in-law, sister-in-law and nephews that she was carrying her husband's child. She did not know how he would react. They had not talked about it at all during these past few months. Their relationship had been cold since that rainy night when she had first told him. It was not that he had neglected her completely. It was just that he seemed to neglect this part of her\egnote{this new part of her? as in, specifically her pregnacy?}. He walked with her to the store as was their habit, and was as appreciative as ever when she served him food. He did not change his behavior so that anyone in the family would notice that something was amiss. Yet, he had not done any of the things a pregnant wife could reasonably expect from her husband. Not once had he asked about her health. 

No, she had not discussed this announcement with him before making it. She had no idea how he would react. She held her breath.

He remained impassive. Moreover, he remained silence\egnote{silent, in silence}. In the absence of his agreement, the other two men in the family clamored for a paternity test. Impossible they said. She was untrustworthy, her brother-in-law claimed, hinting at the alleged theft. They piled insults on her father's head, threatened to keep her closer to the house, promised to not let her work at the store. They threatened to send her back to her parents, or worse, to a home for single mothers. Better to have the scandal of a wife gone missing than one of a wife who would dishonor her husband. 

Tears filled her eyes and threatened to overflow. How could her husband sit quietly while they said things like this to her. Had he ever given any value to her honor in his family? She wanted him to say something. To show her that everything she had sacrificed because he had forced her to was not just water for a rock garden. 

But the insults continue, and her husband said nothing. This, after all was what comes of taking a wife from a family where the mother had married against her parent\egnote{parents'?} wishes. She thought she would break when her husband quietly claimed paternity. 

The insulting patter stopped. Everyone looked from her to her husband. Their eyes, she knew, had held less scorn for a passing beggar. She locked eyes with her husband and held his gaze.

Watch her tightly, her brother-in-law told him, as rose and left the half finished meal. His wife followed, silent, shepherding her children with her; shooting her an angry look that could have boiled a river. 

Her mother-in-law would see to his medication, her father-in-law announced, she did not need to appear in his sight any more. He asked both her and her husband to leave the room. She complied, but he refused to grant her his blessings as departed. His mother-in-law did not say a word.

Her husband met her much later that night, tired and tense. It would be better if she spent the rest of her pregnancy with her sister, he said.

Was he asking her to leave her house, she demanded. She was hurt. Too tired to be indignant or angry.

No, he promised. He would take her and the child back as soon as her health permitted. There was no question of that. 

She could have the child in the hospital in town as easily as she could in the city, she protested.

Her husband gave in to his frustration. Why was she so deliberately stubborn? Didn't she see that it would be easier for him to smooth matters at home without her being a constant reminder of the possibility of dishonor. He would take her back. He gave his word. Was that not good enough?

It certainly was not, she thought, after the last promise he had broken. She did not trust him at all, anymore. She had learned the value of his word. But she did not say any of this.

Instead, she forced herself to see the wisdom in his request. The next morning, she called her sister.

A week later, she stepped off a train in the city. Her husband came with her to see her safely to her sister's house. He would promise to take her back in front of her brother-in-law today.

He meant well, her husband. She wanted to trust him, as she had trusted him for so many years. In a different world, she would be so happy. She had everything; a comfortable home, a good family, land and a store she enjoyed working in, she even had a child growing inside her. She had everything. Her current situation terrified her.

\egnote{the ending is fairly abrupt, and unresolved. If that's not your goal, maybe add some possible resolutions, or just musings about them? `Maybe he would do as he said, take her back, accept her child. Maybe he would not. Maybe she would have to discover how to live with just herself and her child...etc.'}


 
\end{document}
