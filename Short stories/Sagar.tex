\documentclass{amsart}
\usepackage{fullpage}


%*****************
% Annotations
\usepackage{soul}
\usepackage[colorinlistoftodos,textsize=footnotesize]{todonotes}
\newcommand{\hlfix}[2]{\texthl{#1}\todo{#2}}
\newcommand{\hlnew}[2]{\texthl{#1}\todo[color=green!40]{#2}}
\newcommand{\sanote}{\todo[color=green!30]}
\newcommand{\egnote}{\todo[color=violet!30]}
\newcommand{\newstart}{\note{The inserted text starts here}}
\newcommand{\newfinish}{\note{The inserted text finishes here}}
\setstcolor{red}
%***************************



\begin{document}
This is a story about a woman. Who lives in a village. And it is a tragedy. It is not, however, in spite of the subject matter, a Satyajit Ray play or a Tagore poem. It is too ... well, judge for yourself.

Once upon a time, there was a woman, who lived with her family in a village on an island. It doesn't matter much how you decorate her daily life with details, but paint this woman on a rich canvas. If you like, you can start with a thatched roof and the chickens and the cows. She has several children, and family near by. The children play with their cousins and neighbors in dusty bare feet in the fields between the houses. The woman tends a garden and feeds her family on fresh vegetables, milk and eggs from her livestock. That is a perfectly fine beginning. But do not make it the end. For there is so much more to this woman than that.

Perhaps you continue on to imagine her island. Maybe it is a place where verdant is too pale a green to describe what you see. In winter, when the rice paddies are full, the fields by the side of the road literally glow emerald. Neat squares of swaying stalks, neatly planted in rows so thick you can barely see the water underneath. Later in the year, a different field might burst into yellow as the mustard flowers bloom. Banana leaves line the roads, interspersed with bamboo groves. On the horizon, the occasional coconut or palmira stretches for the sky. 

Along the road, every house has a pond, and every pond has fish. Boys run home after school, strip out of their shirts and shorts and dive into the waters. They splash and play and call to each other before their mothers call them in for a meal. If they are lucky, they will eat the fish caught in the traps they had laid out that morning before dragging their heavy feet to school. They cheerfully comply. They know they will see each other in the fields soon when they gather to herd the life stock and pick up yesterday's game of red rover. Earlier in the day, much earlier, as the sun rose and dew still clung to the surrounding reeds, storks fished in the ponds, white against the muddy waters, their beaks sending ripples across the glassy surface as they trap a frog or a minnow. If you are lucky, if the pond is sheltered by a grove of trees, kept for their shade, rather than their ability to provide fruit or sap or bark, you may see tropical birds of red and blue and yellow flit from the pond to their nest above, gullet fat with water bugs for their growing family, chest full of the most delightful song. Who would not rise with a smile if this were their wake up call every morning?

And at night? This is not a peaceful apline village where a hush falls over the houses like a blanket with the dark. No. As dusk brings with it a reprieve from the dazzling sun, the frog and crickets and water insect send up a greeting to their mother the moon that puts the morning birdsong to shame. This is no lullaby to drift off to. It is a riot and a cacophony to compete with the blaring sirens and the grating honking that plagues the city six hours away. Every pond at every house hosts its own ecosystem of frogs in the water reeds and crickets in the bushes and night crawlers in between, each annoucing to each other that night has fallen, and they have awoken, and this is their time to strive, to seek, to live.

But do not worry for your rest, dear reader. In a few hours, the bats and the owls will come out on their hunt. They will patrol the night skies and quiet the raucuous crowds. Be asured that some time before midnight, you will be able to string your hammock between the cowshed and the palmyra tree and rock yourself to sleep while gazing at the stars. And oh, what stars! I assume that you visit this tropical island from a distant overdeveloped land where cities, and even its surroundings are corrupted and polluted by the noise and light that accompanies the hedonistic consumption and need for power that accompanies the word "progress". I assume that you must go far into the mountains to see the white cloud of the milky way on a moonless night, and even then the horizon might glow red from the greed of your distant cities. The island has no such glow. It has no such need for power. It is yet untouched by electric lights, and the nights are as pure as they were a thousand years ago.

It is thus, in this paradise, that the woman raises her family. She is neither short nor tall, neither pretty or ugly. She is a matron, like many other matrons, face worn with the cares of the family and years of tending animals in the sun. Some know her as Such and Such's wife, a few know her childhood name. The author does not. Like all women fortunate enough to be written of by the literati who live on tropical islands, the author only knows her as Pishi or So and So's mother. Her daughters have names, and so does her husband. The woman, however, is content to have given up her name to be known by her family. Do not let this dismay you. Do not pity her for this loss of identity, because it is no loss. She has raised a good family, and she is proud of her work. They respect her, and she supports them. This was everything she was raised to be. Envy her instead. Who among us would not wish to be known and called upon by the fulfillment of our life's work? 

Perhaps the author knows the family. Perhaps the author carried one of the daughters around on her back, both barefoot on the cracked earth, spun her around until both passenger and carrier were too dizzy to stand or giggle or be decent. Perhaps the author read stories to her once, when she grew too old or big for piggy back rides, sitting on the stairs of a neighbor's hut, elbow to elbow, knees knocking in rapt attention. Did the older girl get jelous when the younger sister was still small enough to climb onto the author's back? Perhap, but more likely she did not, for jealousy was not a trait bred amongst these woman, and the author was free with her affection, and her stories of distant lands and strange cultures that her sister was too young to absorb.

Or perhaps the author never met the girls at all. It doesn't matter. There are so many island paradises, each teaming with mothers and daughters. This could be any one of them. The names and personal ties do not matter as much as the story.

And in this story, the younger daughter does, exactly as it is the tragedy of all daughters to do. She grows up. Let us say that she was a beauty, for which mother would not say that of her child. Clear dusky skin. Almond eyes. Black shiny hair that falls past her waist in a braid as thick as a rope. Let her have it all. Give her also, if you would like, a dimple when she smiles, or an exceptionally round face or pleasingly oval one, or, if you prefer, the perfect aquiline nose or button nose. By all accounts, give her beauty. She is young. Let her revel in it. She'll have little else.

Why? Because, dear reader, this daughter, born on this idyllic island and raised under a thatched roof with chickens and cows for friends, and lovely birdsong for a morning choir, did not have the luxury of wealth or the security of finding employment after graduating from school. That should not require saying, but it does. She did not even have the luxury of a father who could provide for her. 

Oh, she had a father. Do not get me wrong. He was still alive. And he was not the sort of villian in so many stories set in idyllic rural communities, drunk all the time, or a philanderer who beat his wife and children. No. He was a good man. In his youth, he worked for construction firms in nearby towns, hauling bricks and cement in baskets on his head. Digging ditches, putting up walls. It was a hard job, but it paid well, or at least, well enough to feed and clothe and educate his children, if he was judicious. And he was. Every week, he brought home his pay, and he gave it to his wife, who was educated enough to keep track of household accounts. Didn't I say that he was a good, forward thinking man? It would never cross his mind that he should control the family by controlling the pursestrings. They had little enough in life without diminishing it with such foolish rancour. And his wife, receiving the week's pay, would add to it what she could make from the vegetables she sold at the market, and the milk  and eggs she could coax from the livestock. It was a good job, and a good life. The sort of life one comes to expect in a pleasant pastoral story about a girl who grows up with goats on a distant mountain side. Why should not a tropical island contain the same?

But I had mentioned, had I not, that it was a hard job? We cannot forget that detail in this pastoral scene. Before too long, the father's back gave out. He could no longer carry baskets of wet cements, or lift shovelfuls of rock and sand out of trenches. He could not balance pyramids of bricks on his head, and walk the length of the construction site. There were days, even, when he could not sit on the bus that took him from his home to the boat to the mainland.

What were they to do? Do not despair, dear reader, for they certainly did not. Life is to be survived one day at a time, and in doing so, one never has the energy to despair. Despairing requires either weakness of will or the larger perspective of a third person omniscient author. Certainly, you, who have recently lived through a world changing pandemic, can appreciate that. You faced each trouble as it came, climbed over it, and then faced the next, because it was already upon you. There was no time for you to despair. And so it was for them. This was one setback among many, and they were a resourceful, tight knit family. Why should they not overcome this as they had all the others?

The husband and wife had loved each other once, and like many good marriages, that love had turned into a friendship as solid and dependable as the brick and mortar houses he used to build. They would get through this together, and put the needs of the family first. And she was a practical woman, who knew that she was married to a forward thinking man. If he could not bring money into the household, then she must, because the vegetables and the cattle could not. But while she was educated enough to keep the household accounts, she had no marketable skills. None, that is, except the skill that all women have. 

Do I shock you? Have I spoiled the remote island paradise with the scent of the city dockyards? Have you realized how much of morning birdsong is actually bickering over seed? Did it suddenly strike you that the night sky is so brilliant because the families eat dinner by kerosine lamp? Did you finally notice the sting of mosquitos and the sweltering heat of the sun? Had you not noticed that an island in the tropics would also be humid and sticky? Then you are naive. While you have certainly had more years of schooling than the husband in this story, it would seem that it is you who is the more backwards. For he did not see a problem with this arrangement. It was work that she could do, and it would pay well, better, even, than the money he had brought home. And they had greater need now, for she wanted to also provide what little medical care the island had to offer to her husband. There was no more shame in a woman providing for her family in the best way possible than there was for a man to do the same. There was more danger, perhaps, but she would work far from the village, and not in the large city close by where many of her neighbors did seasonal work, but at a large town centered around a major railroad junction, where strangers were plentiful, but also where she had just enough friends to get started. There were many things to consider, certainly, but they were a tight family, and close knit couple. They could maneuver even this. 

And so the years passed, until it was time for the younger daughter to marry. I will not pretend that the family managed to keep their secret the entire time. Having smashed the paradise I had attempted to build, there is no use denying that the same petty prudishness that affects city dwellers also takes root in rural minds. And this is why it is so important that the daughter be beautiful. Because while the bedridden father succeeded in holding together the family's dignity in the eyes of his neighbors, he could not succeed in arranging a marriage to an entirely desirable man. I do not mean to say that our beauty had to marry a one eyed dwarf, or a twice divorced gambler. No, nothing of the sort. But the match was not entirely desirable. Perhaps he was not as well off as they had hoped for, or better yet, he lived so far away that they could neither visit, nor ensure her safety in distant lands.  Perhaps he was simply extremely old. Whatever the case, the dissatisfaction with the prospective son-in-law was more substantial than the standard, almost obligatory, complaints of too dark, or too short, or slightly cross-eyed. One of those stock simply would not consider marriage to a girl whose mother did that.

Perhaps it was for this reason, perhaps because it was for some other, that the mother felt the need to make the wedding special. Perhaps it was to put her family back in the good graces of her neighbors. Perhaps it had just been a long hard year on many other fronts and she had a need to find some joy somewhere. One can speculate, but one will never know. 

Whatever her rationale, she threw caution to the wind, and went by train to the town where she worked to buy all that was needed for a spectacular wedding just beyond what her family could rightly afford. And when she returned, she worked herself to the bone preparing for her daughter's big day. Well, maybe it is better to say that she worked herself sick. For a few days before the wedding, she came down with a fever and a wheezing cough. 

I apologize, dear reader, if this sets off emotions and fears from recent events that I should have warned you of earlier. These have been a trying few years, and the words fever and cough hit home in a way that they simply did not when, as children, we lived comfortably in a post penicillin age of medicine. Alas, I regret to inform you that your initial instincts are not wrong. This story is not set in a distant tubercular past, though, I suppose, it certainly could be.

But the wedding day approached, and while the woman's cough got worse, it never got so bad that she could not hide it with some effort. And truly, what effort will a mother not go through to embark her beloved daughter on her new life with all the dignity and love she deserves. Can you blame her? She could not do less. 

The house filled itself to the brim with people. The husband's family comes from distant lands to see the stunning new addition to their family, bringing its own entourage of honored guests in tow. Children weave between the legs of the guests, playing tag, or trying to recover a stolen knick knack, crushing the garlands for the ceremony or toppling the presents for the priest. Someone, a kindly aunt, an older cousin, catches them and chases them out into the fields before they get in trouble. Old men and women sit contentedly, and unmasked, in a room filled with the aroma of burning ghee listening to the priest prepare the couple. Young men mill outside, clustered near the light spilling from the house. It is late enough now that the crickets have ceased their uproar, they certainly cannot be heard above the hum of the generator. Conversation with old friends is more pleasant in the pale moonlight, is it not? There is music, surely, but live. Did I not say that this wedding went just beyond what they could rightly afford? Two musicians shall we say, then? A singer who accompanied himself on a harmonium, and tablist. One must have a tablist, even if a sitar is an extravagance. 


In a tent erected not too far away, guests are escorted to their meals in groups of 20 or 30 or fifty. The sit elbow to elbow in rows, either comfortably cross legged on jute mats or on perched on rickety plastic chairs to eat off of banana leaf platters as servers walk down aisles just wide enough for their feet and buckets and ladles, serving course after course in proper order. The fries, then the eggplant, the sukto then together another vegetable and a fish or a meat. Or both? Did she manage both? Clever woman if she did. On her budget, that would be a coup. The guests eat their fill, seconds and thirds offered freely, they untie their drawstrings to make room for the chutney, then the yogurt and the sweets. Each seating takes about half an hour, maybe a bit more. How many are there, three, four, five? They must be finished before the ceremony is over. Before the conches blow loud and long, and the ululation fills the midnight sky. 

And when morning comes, the wife rises early to prepare and serve breakfast for her new in-laws, still hiding her cough, still ignoring her fever. It is important to send the guests off with a stomach full of good, memorable food. It is for the honor of her family. She must make a good first impression for her daughter's sake. She prepares the breakfast and brings it out to the tent, for there are too many to serve comfortably in the house. Someone else, a sister-in-law, a daughter, a friend, has cleaned out the detritus of the previous night and laid fresh banana leaves and perhaps even touched up the rice paste decorations from the night before. She serves them all thrice, as a good hostess should, and as good guests, they each take three servings from her hands. They eat and chat politely with the husband, pitying him as an invalid, but never to his face. 

Eventually, it is time to leave. Her beautiful daughter, her pride and joy, steps onto the rickshaw her husband has hired for the occasion. The new in-laws pile in behind her, or onto another vehicle to follow. They are decorated, garlanded or painted to announce the procession of a newly married couple. Someone picks up a tune and several voices join in. Her daughter will not look at her, she can tell that she is scared. While the woman longs to give her one last embrace, she holds back. Her daughter is no longer her's. She belongs to someone else now. It would not be her place. 

She steadies her emotions and sighs. As the caravan leaves, and she forces a smile on her face and waves them off. She knows that she has done well. She has done all she could and more. She hopes that the gods above will see to the rest. She wipes a tear from her eye, and takes to her bed. 

Do I need to tell you the rest? We all know the stories of social gatherings that turn into super spreader events. They fill us with an impotent rage. The months of staring at our computer screens, hoping for answers our leaders just could not provide. The jobs, the opportunities, connections we lost. The friendships, the relationships.  The mind numbing ennui of isolation, the helplessness of watching our kids suffer at home, solitude robbing them of crucial developmental benchmarks. For what? So that someone could have a party? How selfish, how short sighted. How.... How could they? How could they? How could they not have waited just a few months. Just a few months to allow all of our personal hells to have ended sooner. You are angry. I understand. I am angry too. 

We have all seen the news reports of poor rural hospitals overwhelmed during a pandemic, turning away patients who they suspected of having the disease because they simply lacked the capacity to help any more. People in rural communities boarding trains and buses carrying their loved ones too weak to walk, hoping that in the city, in the distant magical city, where even tiger's milk can be found, perhaps there, praying that there one will fine a doctor who can save this beloved. 

There was a time, we all lived it, when we could not turn on the evening news without witnessing stories of people crying because the trains could not take more stretchers. When the hospitals, even in the city, were too full. They did not have the ventilators, the bed space, the floor space even for influx of bodies waiting to die. Perhaps you saw the body bags piled outside the mortuaries. Perhaps you were wise enough to simply turn off the news at that point, perhaps you never turned it on at all because these tragedies were in a distant land, and there was enough hardship at home. You did not need to go seeking trouble that you could do nothing about. Good for you. Had I not reminded you, that in times of trouble, we, all human beings, find an amazing strength to solve the problem that is in front of us, take one day at a time, brace every evening for the tragedy that tomorrow will bring, ignoring everything else in order to survive. In that moment, while surviving, despair does not even cross our minds.

\end{document}