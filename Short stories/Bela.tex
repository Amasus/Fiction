\documentclass{amsart}
\usepackage{fullpage}


\usepackage{soul}
\usepackage[colorinlistoftodos,textsize=footnotesize]{todonotes}
\newcommand{\hlfix}[2]{\texthl{#1}\todo{#2}}
\newcommand{\hlnew}[2]{\texthl{#1}\todo[color=green!40]{#2}}
\newcommand{\sanote}{\todo[color=violet!30]}
\newcommand{\note}{\todo[color=green!40]}
\newcommand{\newstart}{\note{The inserted text starts here}}
\newcommand{\newfinish}{\note{The inserted text finishes here}}
\setstcolor{red}



\begin{document}
\begin{center} \emph{For my foremothers. \\  And yours}  \\ *** \end{center}
	
"I didn't know Baba was active in the student Independence day riots."

"Arre, what do you say? Your baba was a prominent student leader. And let me tell you, that man could walk. I remember one year, five of us had set out to put hoarding up around town for some rally the next day. It was May. Dhiren, Shubendu and Kaushik all gave up by 11. I stuck it out until lunch, and then went home. But Anirban, he just kept going. He took all of our papers, our buckets of glue, our brushes, and just kept going. I don't think I saw him back at the hostel until well after dinner."

"How many flyers did he put up?" Tapas asked, enthralled.

Anjan Kaku ignored him. "He'd walked all the way from College Street to Hazra More and west into Khiddirpur. That man could just do it."

Bela smiled. She stood at a small table in the narrow veranda of her ancestral home putting together paan for the few remaining guests her brother entertained in the adjoining room. Stories of her father's ability to walk were legendary in the family. He would walk when we was angry, walk when he didn't have bus fare, when he had some place to go, when he had nothing to do. She could imagine her father as a young man balancing five twine wrapped bundles of leaflets in one hand and three buckets of glue in the other, crisscrossing his city in well worn sandals, a mad dog under the noon day sun, to get the word out for a cause he cared about.

"He never talked about any of this." Tapas, middle aged and balding, was seeing his father in a new light.

"I'm not surprised," Anjan Kaku continued. "After your thakurda bailed him out from prison after the protests, I'm certain he was given a lecture he would not forget any time soon." 

Bela heard Soumik, the only one of her cousins who had stayed after the ceremony, cough in disapproval. She tensed, expecting conflict and scattered a neatly constructed pile of seeds across four sets of leaves.

"Ooof," she tisked and folded the edges of the paan leaves up to recover what she could of the mess.

Her father had never talked about his student days, though the fact that he had been a revolutionary came as no surprise to her. But, she was certain, his actions had been family scandal for all of his siblings. Soumik would likely have a very different version of family lore surrounding those events in his head. 

She forced her attention to the wayward seeds and waited for the tense exchange or the awkward silence. 

"But Baba got his Masters in Engineering from Calucutta University, didn't he?" her husband's baritone cut through the heat like the first drops of monsoon rain. "Top of his class in 1942?"

"That's right," Anjan Kaku beamed. "Now there was an interesting story." 

"Third," Soumik corrected tersely.

"Arre," Anjan Kaku was not to be deterred from his story. "He had tuberculosis during his exams. I'd like to see you do better." He paused to collect himself. "Anyway, do you know how he got in?"

"I'd assumed," said Tapas, "that he applied like everyone else."

"Dur. You think a man with a prison record and no degree just applies to get into Calcutta University?"

Soumik snorted. "What then?" Tapas asked, poorly hiding his annoyance. 

Bela started quickly folding her tray of leaves. Best to go in there soon, give them something else to do with their mouths. Ease the tension that was brewing in the room.

"Well, four months after the Independence day protests, he shows up at my father's house in Barakpore." Anjan Kaku paused for dramatic effect. "He'd apparently walked all the way there, from this very house. Thirty kilometers. Can you imagine?"

"Why?" Tapas asked.

"He'd come to beg for a hand out because he'd realized he'd thrown away all other chances of getting a degree," Soumik sneered.

Anjan Kaku ignored him. "Because he wanted to learn. ... And, he wanted to be close to Hirendranath Mukherjee. The man's fire was insatiable."

"But Dr. Mukherjee was a Professor of History," Tapas asked. 

Soumik scoffed.

"He was the head of the All India Student Federation. I'd imagine that his home department didn't matter for this case," Dhiren spoke with the authority imbued in a man only when he became an Alipore judge. A childhood friend of her brother, Bela still thought of him as round faced kid in shorts she'd catch stealing her mother's narus. Now, as an adult, he didn't seem to speak more than was absolutely necessary.

"Exactly so," Anjan Kaku continued. "I don't think you father had given up a single iota of his political beliefs when he showed up at my father's house that day. He'd just put them away momentarily. He told my father that he would do anything if he would exert his influence on the admissions committee."

Bela arranged the stuffed leaves neatly on an engraved bronze plate, putting one aside in a simple steel bowl. She walked into the conversation, eager to give them something other than the fat to chew. She knew, without even looking up at him the exact shape of the derision Soumik wore on his face.

"Bela," he husband said, hiding his relief at her entrance from all but her. He rose, and waved her in, his manners always a little more western than her family understood. But they had married her to an Oxford man, and it came in handy at times. 

The conversation skid to a halt.

Bela lowered her eyes, smiled and nodded to Himesh. Then she made her rounds of the guests, pausing at each man to ask if he needed anything as she offered them the stuffed leaf. The group of guests all sat shirtless in white bottoms in a half circle around her brother, who perched at the edge of the bed that had been pushed up against the wall to make room for the extra chairs. 

Only Dhiren wore his poite. The rest, either too opposed to the religion or of the wrong caste, sat bare chested as men could in the comfort of their own homes.

She went to Anjan Kaku first, to honor his age and his lifelong friendship with her father. He was 81, just a year younger than her father had been. For most of Bela's adult life, he'd held minor appointments in the CPI-M government, and supplemented his earnings by publishing pamhlets and editorials in both Times of India and Anandabazaar. Over the last five or so years, she'd watched her father's friend set slowly die off. Either through cancer or heart disease, or simple the hard living that accompanied a person growing up, however affluent, in the 1920 and 30s in India. Anjan Kaku was the only one of the many friends that had regularly visited during her childhood who could still attend the ceremony in person. She was glad for both his company, and his good health. As she handed him his paan, she noticed that he still had a paunch that hung over his drawstrings. It was endearing, as far as leftist physiques went. He smiled warmly at Bela through a mouth full of missing teeth and kept the paan in his hand, waiting for others to be served.

Bela proceeded counterclockwise. Dhiren sat with the self assurance of a man who had done what he has set out to do, and the world respected his achievements. He looked completely different from the scabby kneed boy Bela had chased out of the house with a broom for breaking her mother's rice urn. Still round of face, he sat completely erect, but with an ease that commanded respect. His thinning hair smelt of imported pomade. His feet lacked the tan lines of a sandal, not because it was winter, but because he only ever wore western shoes. He was here for Tapas, and she knew that. He had always been there for Tapas, even though he was of a higher caste and a much higher class than she and her brother had grown up in, he never paid heed to any comments about how he was ``slumming it" in the friendship he had with her brother. She didn't even mind when he barely acknowledged her as he popped a leaf from the tray directly into his mouth. She just moved on, knowing that her brother had a friend in his time of need.

Tapas, in contrast, sat on the burlap mourning cloth tied under his dhuti (as was demanded by custom) with the apologetic slouch commonly found in well educated men who worked in the US (which definitely was not). He and Bela had lived in different countries for nearly a quarter century at this point, and this had caused them to drift apart. Not because they didn't care for each other. He visited Calcutta every year or two, and they wrote each other, when they could. But that wasn't enough to keep a relationship going. Not when they were both so busy. Everyone knew that Tapas earned more than the rest of them combined, and then some. But he refused to throw his weight around. Instead, he was perfectly content to let Dhiren dominate the room with his poise and his well chosen comments, when Tapas clearly deserved the limelight, both because of his position as host, and the power his wealth should have given him. The false humility frustrated Bela to no end. But Tapas was her brother, and she was not going to change him. He would continue to carry himself like a man who didn't want to be noticed. If she needed him to do otherwise, she would just have to work around him.

Bela approached Soumik next. Three years her elder, he was the youngest son of her father's eldest brother. Bela had seven other cousins, none of whom she had grown up with. Soumik was the only one she knew at all, and she had only met him while they were both in their thirties. Her relationship with her extended family was sporadic. Almost completely because her father's relationship with his siblings had been non-existant for much of her life. But Soumik, for all his faults, took care of her family and kept in touch. She needed him, she reminded herself. Today was not the day to pick a fight. Yet, he sat with his legs outstretched in his chair, as if he owned the place, and was not a guest in her father's home. His attitude bothered her, and not merely because it simply didn't match reality. She needed to find a plan for her mother's future, to get her out from under this man's thumb. That much was clearl. But she couldn't let that get in the way today. Bela smiled pleasantly as she approached her cousin with two remaining pann. For his part, Soumik put away his sneer for her, exchanging a tight smile and a nod for the comestible and let her pass without comment.

Her husband, she served last. He, of all the people in the room, was the only one who knew what she was feeling today. He was the only one who understood what she had been through this last week, nursing her father and comforting her mother through the final days. Not because she loved him. They had an arranged marriage. He was an Oxford trained civil engineer, building bridges and damns and whatever else, wherever Birla sent him. They got along well enough, Bela knew, in fact, better than many, under the circumstances. But they were not in love. No, he understood her situation because he was the only person she had bothered to tell. There had been regular midnight calls from the public telephone in the hospital lobby during the last week, when Bela, exhausted from fetching syringes, or medication, or bandages, or endless cups of tea, finally gave herself a moment to process, called her husband in their Delhi flat to update him on how her father was doing, and to let him know whether he should book a flight for the morning. He took the last remaining digestive, and muttered a somber "thank you." Bela rolled her eyes. He did it, she was fairly certain, to annoy her. To extract a grudging smile on a difficult day. Well, it wouldn't hurt to concede to his game. The man was too western by far.

Digestives delivered, Bela walked to the door, removing herself, and intrusion, from the men's conversation. They  resumed immediately upon her exit. "Your father clearly did not have a place to stay when he showed up at our door," Anjan Kaku hurried to finish before he popped the paan in his mouth. "My father told him that he needed a cow herd."

"You don't say," Tapas exclaimed, his mouth filled with spices and spit.

"Your father agreed immediately. He kept our cattle for over a year before his application to Calcutta University was accepted."

Bela stepped into the veranda and shook her head. She had done what she could to calm the situation. It was out of her hands.

"He was the best herder we'd had in generations." The room chuckled warmly.

\begin{center} *** \end{center}

"Ma?" Bela asked, sticking her head into the small kitchen at the other end of the veranda. Her mother sat on a chipped wooden piri over a heavy grindstone, rolling a heavy pestle over mountains of soaked lentils to be made into dried dumplings for the year. "How are you doing? I brought you a paan." 

Her mother motioned to a spot on the floor beside her with an elbow without breaking her rhythm. "You're a sweet girl, Monik. I'll have it in a bit."

Bela bent down to place the bowl, her left hand awkwardly pinning the unfamiliar drape of her sari in place. \emph{You don't need to make boris for a family of 20 anymore, Ma} she wanted to say. \emph{It's just you now.} But she didn't. Her mother was a woman of her generation. Her parents had educated her to fifth grade, and she had spent the rest of her life as a home maker. First in her own home, and then in her marital home. She was grieving in her own way. Bela found it hard to respect her mother for being the woman that she was, quite, docile, servile even. But she had always been her father's daughter. Today was not the day, Bela told herself. It was a kindness to let her keep busy with what brought her comfort. Instead, she said, "Let me get these dishes for you," and kissed the top of her head.

At that her mother paused. "Should I make tea, do you think? I used to run a tea shop. It wouldn't be any trouble." 

Bela pinned her face in place and stared studiously at the task of piling dirty dishes. She remembered when her mother had to run a tea shop. Her mother had been raised in genteel society. She always let her father do the grocery shopping. She was not the type of wife who would ever have left her domestic realm. Women ran tea stalls all over India in 2001, but they didn't in the 1940s. And even then, the ones that do aren't born to the types of families her mother had been born to. 

They must truly have needed her mother to bring in an income, Bela knew. The circumstances must have been extraordinary, but she'd never inquired about the details. As a very young child, Bela remembered being hungry every month when rent was due. She'd remembered her father being a union organizer, black listed in factory after mill after dock side. But it had all been so long ago. She barely remembered it at all, and Tapas certainly did not. All she knew was that her father eventually found work, and their situation changed. They'd moved on. Why was her mother recalling it now?

"No, Ma." Bela replied when she could trust herself to speak. "They just had paan. They won't have tea now."

Her mother considered this for a while, grunted, and went back to her lentils.

Bela took the dishes and walked across the courtyard to the shared bathroom in the house. With the door closed, she pressed her forehead to the cool thick concrete walls and let out a slow shuddering breath.

Her father was gone. He has left the world after eighty two years of stubbornness and discipline. Somehow, the people he left in his wake were supposed to figure out how to carry on. The family had brought him to the crematorium early this morning, and only a few mourners remained in the house after the initial ceremonies has finished in the early afternoon.

It wasn't that they couldn't. Of course they could figure it out. Her father had educated his children well, seen to it that they had good jobs and were married into families that could provide for them. As for her mother? Himesh and Tapas would figure out whether or not she would come live with them in Delhi, or if Tapas would buy her one of the new flats being built in Salt Lake. Either way, she would be provided for. 

But the loss. The overwhelming sense of loss and foreboding she had lived with this past week or more. Knowing that everyone would be okay did nothing to dissipate the swarming loss that enveloped her like flies on a dying dog.

Bela sighed and dipped a metal bucket into the open concrete cistern that Kolkata Corporation filled with water twice a day. She caught her reflection in the bathroom mirror as she started lifting and paused.

It was hard to recognize herself. Her hair was the same salt an pepper she expected, and her eyes had more bags under them than usual, but that was unsurprising under the circumstances. No, her bearing was foreign to her. She looked like a village housewife. The mourning rites dictated that she wear a white, unstarched sari draped in the traditional Bengali style. Over the course of the day, the long piece of cloth had grown increasingly limp, and now it clung to her as if she had nothing else of worth in the world. This was the intended effect, she supposed. To enforce a simplicity that was half ostentation, but only half. To enforce that grief was the great equalizer. A force that could reduce a professor of anthropology at Delhi University to a creature that could be mistaken for a daily woman in some babu's home. It wasn't who she was, or how she wanted to honor her father.

She was not, in fact, a housewife or a maid, but a professor at a Delhi University. A fact that was due entirely to her father's will. She pressed the heels of her hands hard against her eyeball, pushing back tears that threatened to overwhelm. How many other fathers, in the 1970s, had encouraged their daughters to get PhDs, and abroad, at that? No other girl in Diamond Harbor, certainly. And how many in Calcutta? She owed everything she was to him. Her father had believed in education more than anything else. And he believed in treating his children equally, long before it was fashionable. He had been stubborn, and idealistic, often to a fault. But everything she was, she owed to him. And somehow, but some unfair twist of fait, that stubborn idealism was gone from her life now. She had no idea how she was going to plug that whole.

Bela transformed her wail of despair into a grunt as she heaved the heavy water bucket up and out of the cistern and set it on the floor with a small splash. She did the dishes.

\begin{center} *** \end{center}

By the time she returned to the kitchen, the men were arguing. 

Soumik was saying, "You have to be kidding me, Ujjal, Anirban Kaku could have been so much more than he was. I understand not speaking ill of the dead, but I cannot praise a man who wasted all that potential."

Tapas defended his father in a low growl. "My father chose not to work for the British. He chose swaraj over shoe licking. There is more principle in that one decision than Boro Jethu showed in his entire life." 

"Gentlemen," Himesh switched to perfect Oxbridge English, hoping, Bela was certain, to evoke an authority that would shock the company into listening. "There is no reason to insult the dead. I'm certain we can..."

The trick failed. Soumik, who bore an inexplicable hatred of her father, just spoke over him. "Anirban Kaku was a dock worker, after nearly a decade of unemployment. My father grew the family leather business into something that could support us for generations to come. He provided livelihoods for thousands of employees."

"Livelihoods?" Tapas asked. Unlike Soumik, who raised his voice when agitated, Tapas' voice grew low and slow when angered. Bela had to strain to hear. "Let's talk livelihoods, shall we? Twelve hour days. Minimal protection against chemical burns. All so that they can stuff their extended family into a ten foot by ten foot room? Don't talk to me about livelihoods. My father was a union leader. He fought people like your father to provide a dignified salary."

"You have no idea who he was, do you?" Bela imagined her cousin's face getting redder and redder with every statement. "Your father ran with hoodlums and criminals." 

"A man agitating for reasonable pay is not a criminal." 

"He is if he brings a weapon onto the factory floor. Oh wait. That's your type of agitation as well, isn't it. Arming rioters. You disgust me."

Bela held her breath. She knew that her brother had been involved in that Naxalbari uprising. A strong leftist streak ran through them both. How could it not, raised by their father as they both had been. The Naxal movement had been violent and ugly, just as independence had been. But that was no reason for Soumik to accuse her family of supporting violence.

"No one here has ever been convicted of anything." Dhiren asserted with unquestionable authority. When the men did not speak over him, he added more pointedly, "Not of inciting violence, not of employing child labor." 

A stony silence fell over the room, neither cousin wanting to engage with a judge on skeletons in their family closet. Bela did not understand the source of Soumik's vitriol. Yes, his side of the family were factory owners and her side union leaders. But why could that not be put aside for a single day?

Eventually, Soumik emerged, taking the long way around the courtyard to walk to the bathroom, choosing to stay in the shade. 

"There's no shame in being a dock worker," Anjan Kaku said. Bela wondered if he were trying to comfort Tapas. "None of you remember the Naval Uprising, I suppose..." 

Bela quickly put down the pile of wet dishes and stepped out of the kitchen door to listen. She knew about the Naval uprising in 1946, of course, and the role the students unions had played in it. But she'd learned it in her history books. She didn't know her father had ever played a role.

Her father used to tell her stories as a child, during the hot afternoons when she was supposed to be asleep, but rarely ever was. It had been their special time, to gossip about her friends at school, to complain about her brother, to hear him talk about the adventures of his youth. She thought she knew everything. Why had he never mentioned this. She crept forward, holding her breath.

"Didi," the building's charwoman called as she came down the stairs. Bela turned to answer. "Boro Boudi is calling you. She's on the roof."

Bela sighed and headed up the stairs. She wanted to hear about this new adventure her father had never mentioned, but she could not leave her mother waiting. Not in general, and never on a day like today. The story would have to wait. "Coming", she called, for her mother's benefit. Tapas, she hoped, could tell her later. 

The house was, by almost every sense of the word, her ancestral home. Except that she had not grown up there. Soumik had, the youngest son of her father's eldest brother. As had the rest of her cousins. But she and Tapas had grown up in Diamond Harbor with the rest of the dock children, and not in this traditional Bagh Bazaar home. Bela admired the building, and not just because of the heritage it symbolized. Thick cement walls, high ceilings and wooden slats across the windows keep the rooms naturally cool and dark all year long, in a way that neither the one room laborers quarters of her youth, nor the expensive flat she currently occupied could, at least not without AC. She had not had a red oxide floor growing up. Her Delhi flat was beautifully tiled. Therefore, in this house, walking on the smooth red veranda, she felt connected to a past that had always followed her around like a shadow in the corner of her vision. But of all the parts of the house, she loved the courtyard the best. Every room in the house opened onto it. It was literally the beating heart of the building. Seven steps to cross in one direction, ten in the other. She imagined her cousin yelling up and down to each other across open central space, and where grandmothers sat sunning themselves in winter, teaching crows to eat bread from their hands. It was where goats would be killed for festivals, or where large shipments of leather brought to be sorted and dispatched to the factory. 

She imagined this history, because she had never lived it. Her father had been estranged from his family for all of her childhood. A difference in political opinion, as far as she could tell. It was only about fifteen years ago, after he'd retired, that her father approached his eldest brother about moving back. All of the cousins had grown up. They'd either married, or moved to Bombay, or lived abroad. Only Soumik remained, and he spent most of his time managing the Nadia factory. Boro Jethu was going to sell the place to a real estate prospector who wanted to convert the building into 4 different living units, two on each floor. Boro Jethu arranged it so that her parents could occupy one of them. 

So her parents lived in two rooms on the ground floor of the house where her father had been born. It was not a very pleasant set of rooms, darker and more mosquito infested than the others due to it's position relative to the street. But it was nicer than where she and Tapas had grown up. It bothered her, every time she came home to her four bedroom Delhi home that she had not been allowed to raise her father from the squalor in which he had raised her. But her parents lived simply, and this was what they had wanted. 

Walking up the stairs today, it struck Bela that uncounted generations of her forefathers had passed in this house, just as her father had. And that he was going to be the last.

"Ma," she called, coming to the top of the stairs. Bela found her mother squatting on north side of the roof, placing small evenly spaced mounds of spiced lentil batter onto an old sari in the sun. 

"Come help me with the boris, Monik. Its getting late." Bela moved the batter to the middle of the sari, and started from the other end.

After a few rows, her mother asked, "What was the shouting about downstairs."

"Soumik had a go at Baba for being a layabout in Diamond Harbor. Tapas got upset. Honestly, I'm surprised it took the so long to get started." Soumik was the only one of her cousins Bela knew well at all, and that was only because he came by her parents rooms with some frequency during the year. She should be grateful for his attentions. With her away in Delhi, and her brother in New York, someone needed to check on them regularly. But Soumik was so unpleasant about the past. Sometimes she had wondered if it was worth it all. But her Baba unconditionally refused to move. 

Bela's mother interrupted her sifting through her regrets. "Did Anjan say that your Baba was a hero of the Naval revolt?" 

Bela startled. "Yes. He did. How did you know?"

Bela did not get an answer immediately. After three more rows, her mother spoke again, "Anjan wasn't allowed to do as much during troubles in the 40s as he wanted. His father found him a position as a clerk in some government office to keep him out of trouble." Bela balked. Anjan Kaku, working for the British? It was inconceivable. "So he lived vicariously through your father's misadventures. And exaggerated his deeds greatly." 

When her mother did not elaborate, Bela urged "So what did Baba do?"

"Nothing." Her mother's voice was suddenly hard and cold. "I made sure of that."

"Ma?" Bela asked. She turned to see her sitting back on her haunches, pressing the heel of her soiled hand to her forehead. 

She reached over and touched her shoulder. "My father visited one day, less than a year after we moved out there. He told him that if your father did not stop his activities immediately, he would notify the authorities. Your father didn't completely stop everything, of course. But he did enough."

"Authorities?" Bela was genuinely confused. What on earth could her father have been doing that her mother would have turned her own husband over to the British authorities?

"Monik, your father was an arms smuggler." She sighed heavily and got to her feet. "Can you finish this up, sona? I need to feed the cleaning woman." 

Bela watched her mother go. 

Two new facts struggled for dominance in her mind. Her mother had once stymied her father. Her mother, who Bela had never seen raise her voice or speak back to anyone in her life, the woman who had always been content to feed and comb their hair and not have an opinion on anything outside her home had stopped her father's revolutionary zeal in its tracks. It was like watching a kitten stop an avalanche. 

And her father? The man she had idolized for her entire life? The man who had taught her to only pick on people bigger than her, who taught her to help the underdog and the value of collective action and the importance of standing strong against the oppressor. That man, who had supported her wanting to study the social sciences because it was that important to understand our history in order not to repeat it. She couldn't wrap her head around it. Her father was an arms smuggler. 

\begin{center} *** \end{center}

It had been dark for hours. Throughout the late afternoon, the guests had slowly wandered out of the house. Soumik first, soon after the altercation. Then a relative came for Anjan Kaku. Finally, her husband and Tapas walked Dhiren to the larger street where his driver would be waiting. Her mother was busy with the dishes they had all left behind. For the first time all day, Bela had a moment to herself.

She put he elbows on the low wall on the east, the side that overlooked the river, and lit a cigarette. Bela took one long slow drag and closed her eyes. She needed just one moment to be away from the chaos of the day. She held her breath and let the nicotine ease her nerves. She imagined her office, a small neat room on the fifth floor of the social sciences building where the anthropology faculty had their space. Ever since the children were small, this room had been her refuge, her escape from domestic constriction to a world where he work had meaning. Here, her honor was not tied to the brilliance of the meal she served. Her neighbors did not gossip about the quality of silk in the sari she wore to work. Men invited her into the room, not to serve them paan, but to challenge their ideas.

She imagined the steel shelves behind her desk. The shelves stacked with color coded cardboard file folders, each tied neatly with ribbon: pink for university administration, green for course notes, yellow for student work, and clear plastic for manuscripts. But the bottom two shelves, that was her treasure. Gore Vidal mixed with Bankim Chandra and the obligatory Rushdie. She had an entire other steel book case for academic texts and journals. These two shelves had nothing to do with work. They were just for her. 

She remembered the smell of the potted tulsi plant she kept by the windows, and recalled the manuscript she knew sat in the middle of her desk. Moving back to India after her PhD had been the logical choice. India wanted native professors with experience abroad. Delhi University's offer was far superior to anything she would have gotten as a post doc in second tier American school, and her husband had a very lucrative position at Tata Steel. But as an Indian academic, she could not afford to go to the conferences abroad that she wanted to, even if she could get the visas. She missed corresponding with the wider community. Which was why this particular manuscript was so exciting. She'd secured a grant for Commonwealth Country academics to travel to the UK, and ...

"Didi?" 

Bela opened her eyes and exhaled. Tapas stood at the top of the stairs.

"You know you shouldn't smoke here," her brother chided. What will the neighbors say Bela completed in her head. Even in 2001 Calcutta, it was not appropriate for women to smoke cigarettes in public. A biri, sure. The poor must have their outlets. But any woman who could afford cigarettes should know better. She could have protested, or just given in. Instead, she offered her brother a smoke and a light.

"Is Dhiren off safely?" Her brother nodded as he lit a match. Bela put the ashtray between them. When he didn't add anything, she filled the awkwardness with "It was good of him to come."

Her brother puffed nervously, fiddling with the matches. Bela waited, and looked out over the river. Groups of men would be sitting and chatting on the ghat, she knew, even if she could not see them from this vantage. A train passed northward to become a commuter line to the suburbs. "Didi," he finally blurted out. "Did you know all of that about Baba's youth?"

Bela shrugged. As the elder, she had always been her father's favorite, she knew that. He had always treated them both the same. The same ammount of clothes for the pujas, the same number of tutors for school, the same expectations on housework and grades, the same encouragement for extra curricular activities. But, in the hottest part of the hottest afternoons, when Bela could not sleep and the house was completely still, she got her father's time in a way her sleeping brother did not. They would lie on their sides or propped on their elbows or flat on their backs talking. Baba would tell her stories. Stories of his childhood. Stories from the newspaper. Stories from work. Stories of his days in the Independence movement. There wasn't a single story of his life Bela didn't know, or so she had thought this morning.

"Most of them. But not all," she said, not wanting to hurt Tapas. "It was really nice to hear him lauded like that." 

"He was a brave man," her brother agreed. "An elephant."

Their father had always admired the elephant. An herbivore by nature, it was the bravest of the beasts. It was non aggressive in its daily life, but fiercely protecting its herd. Yet, put in the hands of the wrong human being (even before she started school, Bela knew that this was an allegory for the British) it could become a weapon of war. Bela had always thought that her father had told this story to teach them to pick their battles carefully. But now... had her father thought of himself as twisted into a weapon of war?

"I don't understand," he muttered. "I don't understand how Baba, how that man, a freedom fighter, union organizer, student leader. How could a man like that suddenly loose courage and pack me off to the US in the middle of everything."

Everything. By that, he meant the Naxal movement. Bela knew that her brother had been active in the violent rural uprising that had lead, in part, to Indira Gandhi declaring a state of emergency. But she had been at Berkeley when her brother was in college. All she knew first hand of the affair was that she got a telegram one morning asking her get her brother a family visa as soon as possible. Her father never told her why, and her brother would not talk about it.

"People change," she said carefully. Her brother wasn't the only college student to be shipped abroad by worried parents. Corpses of activists were showing up bloody and mutilated on their parents' door steps. No one wanted their child to be next. "West Bengal was a terrible place back then. Maybe he was scared he'd loose you?"

"Scared?" Tapas gave a bitter laugh. "Baba lead a hundred striking sailors face down guns and batons of their commanding officers. Have you ever known Baba to be scared?"

Bela had to admit that she hadn't. And her brother had a point. Nothing done during the seventies was that different from what had been done during the forties. Not even with the disappearing of activists and the open secret of police torture. Yes, it was Tapas's life on the line this time, and not his own, but that would that really make her father scared?


"I don't think so, Didi," Tapas continued. "Baba was always closer to you than he was to me, but when I went to college, when I became politically active, that changed." He took a deep lungful of smoke and let out slowly, putting this thoughts together. "It wasn't that he was proud of me. He'd always been proud of me. He... I don't know. He started sharing things with me. Does that make sense?" 

Bela nodded. She had been away in graduate school when her brother went to college. She hadn't been around to observe any of this transformation. As far as she was aware, as a child, her father had favored her slightly as a child, and then he and her brother had grown distant when he was suddenly sent to the US.

"And then suddenly, like someone had turned off a light, he closed up and stopped talking to me. Then you brought me to the US, and I never got him back." Tapas made a fist and brought it down slowly on the low concrete wall. Then he stared at his hand as if it was slowly opening and closing of it's own free will. 

They were all grieving. Each and every one of them in this house. And Bela's heart was too full with her own loss, from the exhaustion of the days ceremonies and taking care of the lingering guests, of what she had just learnt about her father, of the sheer unfairness of it all to have words to help her brother. She offered him another cigarette instead.

"For two years, I can talk to him about, well, anything. I can ask his help about how to organize. He'd proofread our pamphlets. He even started introducing me to old contacts of his to help us with our cause. I'd never seen that side of him. He never liked to talk about it with me growing up. And then suddenly..." Tapas's voice broke, and he fumbled with the matches to light up again.

It was hard to see her brother's grief. Not his loss of a man who he loved and respected all his life, but his grief at realizing the he didn't fully know the man he looked up to for so long. She was about to say something trite about everyone having secrets and that this was only normal, when a thought bubbled up to the surface of her mind and burst free. Her mother, as much as she could not imagine it, had found the strength to threaten imprisonment to her husband. If her mother had the resourcefulness to go around a man like her father, what mountains would she move to keep her son alive? 

"Ujjal," she began cautiously. "Were you by any chance," she paused, not wanting to force the issue, but needing to know, all the same, "were you by any chance involved with weapons dealing?"

Tapas suddenly ground out his half cigarette in the ash tray. "It was a long time ago, Didi. Best not to talk about it." 

He turned on his heel and went down the stairs.

\begin{center} *** \end{center}

"Ujjal, would you like more chutney," her mother asked. Bela, Tapas and her husband sat in a straight line on the veranda near the kitchen. Her mother sat opposite them, serving. It had been an long hard day for everyone, and her mother would not eat until they had finished. Tapas shook his head. "Himesh?" Another polite refusal. 

When her mother pulled a disappointed face at not being able to feed her son and son-in-law to her heart's content, Bela interceded.  "Ma," she offered. "Why don't you start. I'll clean up here."

Her mother begrudgingly agreed and she helped her mother carry the serving dishes back into the kitchen. 

When she emerged, having served her mother, Tapas was wondering out loud, "Why did Baba not come back here earlier? No one was against independence once we had it. I don't understand why he wouldn't make peace with his family over their differences." 

Bela quickly bent down to stack dishes and wipe the floor where they had eaten to hide her face from her brother. Tapas was a PhD materials scientist for IBM. He had more patents to his name than she had books to her. In so many ways, he was absolutely brilliant. But in others, he seemed to have less common sense than a goat.

Himesh answered for her. "Arre bhai, he was a union leader. Your thakurda owned three factories. Some differences cannot be patched."

But Tapas shook his head, insisting. "Maybe, but its not just that. I remember when Thakurda died. He took Baba's and Boro Jethu's hands and begged them to make peace. Thakurda said that wanted Baba to live at home. Boro Jethu refused him, on his death bed. So did Baba." 

Himesh shrugged. "Your Baba was always stubborn. Maybe he couldn't bring himself to make peace with his brother. That's been known to happen you know."

Tapas persisted, "Then why do it later? What changed in Baba? I just don't understand." 

Himesh gave him a sympathetic look. "You're overthinking it, bhai. It's late, you are tired. People grow old and mellow. Its as simple as that." 

Tapas opened his mouth to argue, and Bela quickly stacked the dishes and turned to leave. It has been a long, hard emotional day. She had cremated her father, long before she was ready to say goodbye. Then she had learned more about him and her brother than she had ever wanted to know. It wasn't fair, and she was done with it. Her brother needed answers about the man they had both lost. She understood that. But she did not have the capacity nor the physical strength to uncover any more family skeletons. She wanted her brother to stop asking. She wanted him to be content grieving the man he knew and let her do the same. She wanted. She didn't know what she wanted. She wanted the world to stop turning. She wanted time to roll back. She wanted not to be here anymore. 

"One moment," Himesh cut her brother off. He walked back to the room that had hosted guests all day, now rearranged to allow the family to spend the night. "Bela," he called as he walked. "Make some time to meet me here."

When she walked into main room, bereft of dishes and with clean dry hands, she found her husband sitting on the bed with his brief case open beside him. He reached in and handed her a pink file folder tied with a red ribbon. "What's this?" she asked.

"A manuscript," her husband replied. "It came in the mail the day after you left. So I packed it." 

Bela stared at the package in her hands, uncomprehending. "Open it," her husband said, so she did. He was right. It was, indeed, a manuscript. A review request from an journal she sometimes published in.

She looked up at her husband, confused. "I don't understand," she said. "I don't have time..."

But he cut her off. "Yes you do." He nodded towards the stairs. "Go to the roof. No one will bother you there. I hate seeing you like this." When she hesitated, he added, "I'll make sure Ma is comfortable and keep Tapas out from underfoot. Go."

Suddenly Bela grinned. Document in hand, the first tendrils of normalcy finally could penetrate the fog of grief and duty that had consumed her day. She felt giddy, like a school girl with a novel after her year end exams. A few hours to herself to do exactly as she pleased. She grabbed a pen from her husband's suitcase. A jute mat and hurricane lamp would be on the roof. That was all she needed. She turned on her heel and all but ran to the stairs.  

\begin{center} *** \end{center}

It was well past midnight when Bela finally made it to bed. She'd read the manuscript until she was too tired to work anymore, then come downstairs to rifle through the small set of her father's books that stood in the veranda. She'd decided that she wasn't up for anything more serious than a Feluda mystery, but it was comfortable to read the stories she and her father had read to each other in college. To imagine the past to be as solid and uncomplicated as it had been when she'd first experienced it.

She could hear her husband and brother's asynchronous snoring from the base of the stairs across the courtyard. They had taken the bed, she knew. So she felt her way along the wall until she encountered the low ribbon holding up the mosquito net for herself and her mother. She followed it until her toes felt the edge of the jute mat, untucked the netting, and crawled in next to her mother on the floor.

Her mother stirred, "Monik? is that you?"

"Yes Ma, go back to sleep." 

Her mother sighed. Then turned on her side to face her. "I've been thinking," she said, "I don't want to go to Salt Lake." 

Bela fumbled for her mothers hand and squeezed it. "That's okay," she said. They'd discussed living arrangements briefly over dinner with her mother, laying out the two main options as they'd seen them. At the time, her mother, not speaking any Hindi, wanted to stay in Calcutta. But nothing had been set in stone. "You can come live with us. We have more than enough room."

"No," her mother insisted. "I don't want that either. I want to stay here. If you don't want to support that, I can open a tea stall again. The rent here is not that much. And it's just me this time." 

Bela rolled over and propped herself up on her elbow. It was too dark to even make out the contours of her mother's shape. But she wanted to speak in her direction. "You won't have to start a tea stall, Ma. We're all here to support you." When her mother didn't respond, she continued. "Why do you want to stay here, Ma? What does this place hold for you?" Her mother didn't answer, but Bela could hear her breathing grow shallow. After a while she asked, "Ma?"

"When Ujjal asked," her mother's voice was almost a whisper, "why your baba never made peace with your thakurda ... he wasn't wrong. The reasons don't make sense." Her mother sighed. Bela tensed. She didn't want to learn more secrets. But she didn't want to leave her mother carrying this grief by herself any more either. She waited. Eventually her mother found her courage. "You can't tell anyone this, you understand?" 

Bela shook her head. "Of course not. Not my husband, not Tapas, not the children. Not Ujjal. No one." 

Her mother heaved a stuttering sigh again. Belas was certain she was crying. "Within a year of independence, when you were about a year old, your father started organizing the dock workers. He was a charismatic man, you know that. But he'd always had more principle than common sense. To make a union stand on its own two legs, people need to be paid off. Cops need to look the other way, floor managers need to allow votes to happen. Local merchants need to be convinced that they are willing to support the cause.

"But you know your father. He could never bring himself to pay a bribe. And he certainly wouldn't pay a bribe with money he raised from the masses. It went against everything he believed in."

Her mother paused again, failing in her courage to continue the story. Bela stroked the soft crepe-like skin on the back of her mother's hand. "So what did he do?" she prompted.

"He stole from your Boro Jethu. One thousand Rupees. That was a lot of money back then." Bela let out a long slow breath. It certainly was. Possibly a month's salary for all the men in one of her Thakurda's factories. 

"So that's why," she said with finality, "your father could not come back to this house for so long."

Curiosity battled with shock in her brain. There was so much she didn't know about her father. She wanted to remember him the way he was, a freedom fighter and a man of high principles, not a thief who supported violence. On one hand, it was too much. On the other hand, this was only half the story. She had promised not to tell Tapas, but she wanted to know the rest, for his sake. "So what changed, Ma? In eighty seven. When you moved back?"

Her mother sighed again, then whispered, "Ever since your Thakurda died, your father was so filled with remorse. He wanted to make it right with his family." She squeezed Bela's hand. "You understand, don't you," she pleaded. "What its like to loose a father. How it can change you?"

Bela did. She nodded, realized her mother couldn't see her, and squeezed back. "I do. So, what happened? He made peace with Boro Jethu?"

Her mother tisked softly. "Of course not. Stubbornness runs in this family, and there is nothing to be done about that." Bela waited for her mother to find her courage again. "I sold my wedding jewelry to pay him back." 

"Ma!" Bela yelped. 

Her mother fumbled for her face and covered her mouth. "Hush. You'll wake the others." 

Bela caught herself. She couldn't believe it. Her mother had sold her wedding jewelry to pay a debt for her father. She was stunned. A woman's wedding jewelry was all she had. It was her pride and her honor, yes, but it was also her nest egg, her safety net in case her husband died, or worse, he left her to fend for herself. Bela had insisted on having a civil marriage so that her mother would not have to touch her jewelry to pay for it. And this. To hear that she had sold it to pay back a debt incurred by theft. It was just too much.

"Will you be quiet?" Her mother asked. Bela took several deep breaths to gain control of herself and nodded. Her mother removed her hand. 

"But Ma," she whispered. "Why didn't you ask for help? Ujjal and I would have gladly pitched in to let Baba make peace with his family. It wouldn't even have been a hardship." 

Her mother gave a wry grunt. "Oh my sona girl. You think I could ask you to pay for your father's misdeeds? No. Not in this life."

So they lay there, hand in hand, nestled up to a secret too big for the mosquito netting. In the bed beside them, Tapas and Himesh snored. Oblivious of what sacrifices had been made around them. Bela understood, she thought. Her mother had given up too much to be allowed to live in this space. A space that should have been hers from her first day as a new bride. She could not leave it now. She wondered what she would have to tell Himesh to give her mother this one request. 

"What was the story?" her mother asked, "that your father liked to read to you so much? The one about stealing from the rich to give to the poor?" 

"Robin Hood," Bela supplied. 

"Robin Hood," her mother repeated. Then gave a wry laugh. "Your father was always too romantic to be of any practical use. He imagined that being an outlaw was a good thing. I don't think it ever occurred to him that criminals are the villians of all those stories he liked to read to us." 

That made Bela think. "So why did you not turn him in, Ma? Like you threatened to do for the ..." the words stuck in her throat. She could not bring herself to say that her father had been an arms smuggler. 

Her mother spoke as if the answer were obvious. "I had you, sona. I couldn't brand you as the daughter of a criminal. What would that do for your future?" Then suddenly, her mother let out a sob. "Your Thakurda had stopped supporting us. Which I couldn't object to. No one should have to support a son who is a thief. And I..." she paused to catch her breath. "I couldn't go to my parents and ask for help. How could I explain why my father in law wasn't supporting his own son?"

Bela waited for her mother to let out a few more shuddering sobs. Eventually, she moved to dry her eyes. 

"I was so alone, sona. I had no one to turn to. Your father wasn't going to give up his principles to provide for you, and I couldn't let you starve. So I did the only thing I could. I started selling tea to strangers. I know it wasn't easy for you, Monik. And when Ujjal was born, it was harder still. I chose to raise you in poverty because I thought it was better than letting people talk. I thought I was protecting you, Monik. From your father. From your relatives. From everyone around you. I just wanted to give you the space to grow up unmarred."

\end{document}