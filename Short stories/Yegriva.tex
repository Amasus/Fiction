I heard my grandmother calling to me, as she hadn't since I was a child. "Yegriva, wake up, little gull. You have to drink something."

I said something, which, in my mind, sounded like "Later Baba. Let me sleep," but came out an incoherent groan. I turned my fevered forehead to rest against her cool hand, shutting my eyes with renewed fervour. Baba Lesha smiled at me. Her black black eyes sparkling like school of fish in the night sea. Her hair the color of sea foam. She sang to me. She sang the same lullabies that all mothers and grandmothers sang once, the same ones I sang to my three babes. They were songs of the cliffs and the oceans and waves. The tunes reminded me of the raucous screech of seagulls and the noble sight of herds of whales migrating along the horizon. They were the songs I grew up by. Tunes of wide open spaces, salt spray, wind and the color blue. I began to cry.

In my weakened state, I struggled to make sense of the sudden grief that overwhelmed me. I was at home, with Baba Lasha, was I not? What was there to grieve for? It was as it had been for years, simply myself and my Baba, two fisher women in a house with a small garden on a peninsula that a man could cross in a hundred paces. We ate what the tide brought in, dug for mussles, and sold trinkets made of their shells at the fair every summer solstice. The winters were cold and harsh, but Baba Lasha was warm, and her stews, ... well, my husband once claimed he married me for the recipe. I had returned, miraculously, to that simple world of my childhood, to live as I had before my marriage, before my children, before the cliffs fell and the wars started, and the long wearisome march of exile had worn my feet to the bone.

I buried my face deeper into Baba Lasha's side to forget the years in between.

She smelt wrong. Even as she grew older, and less sure footed, even in the autumn, when my grandmother spent her days digging for onions and turnips in the garden while I tended to the nets and scoured the tide pools, my grandmother smelt more of the sea than earth. She could not help it, living on such a thin promontory of land nearly an island at high tide, the sea owned us. We lived at his mercy and prospered by his benevolence. We paid him obeisance, and he was generous. He had marked us with his scent, anointed us, so that when we forayed inland for any reason, the cliffs would know whose servants we were, and the land could not touch us. The sea would keep us safe as long as we remained loyal, my grandmother would tell me every day as we searched the tide pools for our food. I would nod somberly, and thank our watery master dutifully for each piece of flesh or shell or rock or wood I put into my basket. We would leave a piece of the best meat uncooked on the alter all day, then put it out with the nets in the evening before the tide came in to take our tithe.

Why did my grandmother smell of the earth? She had not come with us on this grand folly inland. "The ocean has kept me safe my entire life," she had announced to my husband when it was time for us to go. "He will keep me in my old age." My husband, grumbled, but accepted this pious argument, offered her his blessings and left. I stayed behind to argue, sending Mirna, my oldest out with her two siblings to finish gathering our belongings. "You are an old woman," I cried. "Your eyes have grown dim. How will you live alone?" She looked at me sadly, her once night black eyes dimmed by heavy clouds that would never blow ashore, and I understood. She had no intention of living alone, no more than she had of dying away from her home. I wept then. I clung to her soft plump shoulders and soaked them with my eyes own brine. She ran her fingers through my hair and sang to me then, as she was singing to me now. Eventually, I calmed, and Dyren, my youngest and only son, having grown hungry, was brought in by Mirna.

I fed my wailing babe. When the child had fallen asleep, I put him in his sling, and walked with Mirna to our pile of provisions and belongings. It was larger than that of most of the other houses. My husband was a good man, a strong man. His arm was true and his spear flew straight both on water and on land. We never lacked for food. I sorted through our baskets, dividing the dried edibles into two halves. Then I went through our seed store, dividing that into three piles. I took half our dried fish, and two piles of our seed store and stood, balancing the several baskets carefully against my body so as not to wake my son. Then I turned back towards my grandmother's house.

Mirna stared at me open mouthed. I nodded for her to come with me, handing her a basket of talapia that pushed against her brother's sleeping feet. "But mother, what will we eat?" she protested.

"Your father will hunt." I answered tightly. "We will not starve."

My daughter made a sound half way between a gag and a gasp. "Meat, mother? We cannot eat meat. The ocean will reject us."

I took a deep breath, and looked around. We were quite alone. The villagers were out of earshot, packing the mules and carts with our collective belongings. A few gulls circled overhead, as always, but it was not clear what they heard, or if they reported their knowledge back to the ocean faithfully. The long stretch of narrow land to my grandmother's house lay empty. Still, I lowered my voice to say the words that I knew should not be said. One could never be certain what spirits lingered in the air, or if a crab listened from under a rock. "We are rejecting the ocean, Mirna. Now that we are going, we will never return. The cliffs to the east have crumbled, taking half the city of Selvand with it. The raiders from the north have razed the rest and slain our king. They are marching west and our men have decided that we cannot defend ourselves. Our priests say we will only go inland until this crisis passes, but they know as well as you or I, that once you turn you back on the ocean, he will not take you back. Our men have decided to turn to the land for our safety. They will hunt from it to feed us as well."

My daughter looked at me angrily, as befitted the blasphemy I had just uttered. It changed neither my pace nor my resolve. "We share our food with Baba Lasha. She has no one to fish for her." Mirna followed, truculent, rebellion and outrage painted all over her face, but she did follow.

The raiders would not come this far, I told myself. Other villages further to the east were emptying as well. No one could stand against them. After Selvand fell, we simply did not have the men to defend ourselves. I was lucky, my husband was one of the few who did come back from that battle. He has seen how easily the northmen had overrun the great pearl of the Selvic people. He had no hope for our safety behind the simple wooden barricades surrounding our houses meant to keep out wolves. Many other villages had come to the same conclusion. The Selvic tribes were streaming inland. The raiders would grow weary of the clusters of empty houses strung along the coast long before they reached my grandmother's promontory. She would be safe. She still had her strength if not her sight. With the seed I was giving her and a little luck, she could live off he land. Of all the ocean's subjects, she alone stayed true. She would do her obeisance, and the ocean would provide.

My grandmother said nothing when we showed her the food we had brought for her. She followed us silently to the corner where we put the baskets and counted them with her hands. She ran her fingers through the bags of seeds, one by one, identifying each, then measuring their weight in her palms. "You will be beaten for this," she told me at last, after she had taken stock of her store. "It doesn't matter," I murmured.

She cupped my cheeks in her hands and kissed my forehead, gently, as she used to when I was a girl, and I had done something to please her. I remembered that kiss the next day as I carried my sleeping son across my front in the dark and foreign woods, trailing behind the bulk of my people due to the heavy and painful bruises on my back and legs.

"Yegriva," Baba Lasha called out as I crossed her threshold. I turned back, though it is bad luck to start a journey in hesitation. My daughter, angry at my betrayal, walked on. "You are a good girl," my grandmother said. "Take this to remember me by. I always meant for you to inherit it." She took a goat hair rope from around her neck, lifting it above her head. From it, dangled a brilliant white crystal, nearly the size of a gull's egg, and just as smooth. It glinted and sparkled in the sun. I had loved this trinket as a child, as had my children after me. My grandfather had given it to her once, long long ago. He had found the perfect stone in his crab net, already with a hole through it, as if asking to be made into a necklace for a woman he loved. Baba Lasha said she would no be parted from it for as long as she lived.

"I have no use for trinkets where I am going," I protested, and thought "Live Baba. Take my food and thrive. It is not your time yet."

"You will settle down eventually," my grandmother said, smiling with reassurance and wisdom, pressing the stone into my hand as if she were merely giving me a remedy for colic.

I found myself paralysed by my emotions. I could neither accept the present nor say goodbye. I stood, staring at her.

The door to my grandmother's hut closed abruptly in my face. Baba Lasha was gone. Reluctantly, I turned my steps towards my waiting family.

*************

"Have you fled too? Did the raider's find you?" I murmured into my grandmother's earth scented body. "Why do you smell of moss?"

"Wake up, Yegriva," Baba Lasha's eyes grew grave and still, like the reflection of water at the bottom of a well. I tried to focus on her voice, there was an urgency in it that frightened me, but it slipped away like an eel in open waters.

"Where am I?" I asked at last. "How did you find me?"

"There's no time," my grandmother urged. "Where's Dyren?" That thought gave my mind mooring. My son, now five was a wild and willful boy. Strong of body, like his father, my people praised. And strong of will, like his mother, my husband cursed. Still, an only son, he was the pearl of his father's affections, and a source of constant worry for me. In a sudden panic for his safety, my eyes sprang open.

I gasped for air. A pain beyond comprehension radiated through my body. It consumed me, leaving no space for thought. I lay still for what seemed an eternity, breathing in short shallow breaths, as that seemed to hurt the least. After some time, I identified the source of the pain. It was worse in my back than anywhere else, but my legs ached, and my head throbbed as if a blacksmith had mistook it for an anvil. And I was cold, no, wet. I was soaked to the skin, the bed I lay in was wet. With what? I made a feeble effort to move, but the slightest attempt to move any part of my body attached to my back sent lightning bolts of pain across my back, turning my stomach and tunnelling my vision.

I lay with my eyes closed until the wave of nausea and dizziness passed. Then I opened my eyes and tried to take stock of my surroundings. There was something, I thought feebly, I needed to do, someone had been here with me. Who? Moving only my eyes, I found myself in a dim space. A draft reminded me of how cold and wet I was. Where ever I was, my shelter was not complete. I could hear water dripping somewhere near me. The sound made me realize my thirst. The need for water competed with the pain in my body, urging me to move again. I struggled to lift my head, only to lose consciousness.

**********

I next became aware of the foul stench of bile. This time, even lying still, I could not shake the darkness blurring the edges of my vision. I had to get help, I realized, or I would not be long in this world. Where was I? Had there been someone with me? I tried to call out, but only managed a moan. I tried again, louder, then waited. The space I lay in seemed darker now, though that may have been due to my fading vision. The dripping sound was louder now, in more places, all around me. I was, if anything, colder. I forced my eyes to focus on my surrounding, beginning with the bed I lay on.

It was brown. And green. Covered in dead leaves and moss. Water pooled and beaded on the dead leaves and formed a light spray over hair thin fronds of the tiny forest growing under my nose. The forest of moss below me seemed as varied as the forests I had wandered through in my long exile, a patch of light green fronds tangled with a cluster of darker shoots. Here and there a single red stemmed stalk towered above the rest, a foreigner, a giant, an invader. If I looked closely, I thought I could see where their tiny roots took hold of the long dead tree bark. I breathed as deeply as I dared, taking in the dark smells of old rotting wood and dirt freshly made by the industrious roots of my bedding. It was beautiful, I thought, in my tired delirium, then smiled to myself. We had been walking for four years, living off fruits of the forest, pausing at first, for a season to farm, then moving on when the harvest failed. Begging at villages during the winter, starving when they chased us away. Learning, reluctantly, slowly, the ways of the land; that herds of deer migrate as fish do, that rats and rabbits can be dug for in winter, as mussles and crabs. We learned, and hated every lesson. It was ironic, that only at the end of my days, defeated by this new solid master, I would learn to love him, too late to make offerings or ask for protection.

Something cold and wet dropped onto my temple, rolled down my cheek and down my nose. I caught it on the tip of my tongue. Water. It was raining. I was thirsty. My tongue, heavy and dry, but now greedy with its first taste of water reached out of its own accord and sucked what it could from the leaves and moss around me. There was not enough water to slake my thirst, but enough to awaken in me a desire to live. I opened my mouth and grazed off the forest of moss below me. We had spent long lean springs eating new shoots and their tubers. The moss tasted clean and green. It would likely not kill me.

When I had eaten and drank all I could reach, given my limited mobility, I took stock of my situation. There wasn't much to it. I was in a forest, in the rain, with unknowns wounds. I had little knowledge of how I had come here, and less of where the rest of my people were. I knew I had to get out of the rain, but I could not move. Only with the greatest of effort had I willed my arm towards the ancient dead tree trunk. Bringing bits of moss and the occasional grub to my mouth was agony. Moving my body was unthinkable. I needed warmth, but first, I needed strength.

Exhausted, more than a little frightened, but too cold to sleep, I wondered where I was, and how I had come to be here. During the slow deliberate feeding process, I had learned some of the limits of my body. I shifted slowly to get a different perspective on my surroundings, hoping for a clue. Further down the tree trunk that had become my trough, I saw a small cluster of mushrooms white in the growing twilight. I remembered. Then immediately wish I hadn't.

*****************

We had camped last winter in a set of caves as close to the sea as we dared, with the wars still raging along the coast. Then spring came, and with it, the danger of bears awakening from their winter sleep. The men consulted. Here again, our village had been lucky. We still had more than a half dozen men able to fight and hunt for us. In our long wanderings, we had met groups of just women and children, no livestock, no seed, starving and sick, begging in the lean spring. More mouths to feed when we could not feed our own. We stoned them and sent them away.

We still had men to hunt and boys to grow into men. Our women were still fertile, our men became better hunters with every season, and our boys clever at stealing eggs from nests and fish or fowl from neighboring farms. We were destitute, but we would survive.

The men decided that the winter had been too hard, that too many of us were ill. It was not wise to leave our current shelter. So we stayed, and I nursed my frail beautiful Lyta, my middle child who should have become a woman by now, but remained thin, quiet, and still a girl. I thanked my husband for his kindness, for I am certain he had a say in the decision to stay. He was not the type of man to let his family suffer. That first winter of wandering, when our people suffered and died as they never had before, he would not allow me to be stoned and turned into the cold, though so many in the group, including my eldest, Mirna, desired it.

My husband accepted my thanks, and told me to give a larger tithe than usual tonight. We prayed to the land as we had prayed to the sea, though we begrudged him every morsel of food that did not go to our children. What else could we do? We knew of no other way to placate our master. The land rejected our offerings and continued to rain troubles on our head. Whereas our tithes at home would bring nets full of shimmering fish, our new offerings only brought flies, rats and the smell of petrification. Still, I did as I was told.

By the time Lyta has recovered her strength, Borno, a strong boy well into his transition to manhood, had been bitten by a snake. His foot had swollen to the size of a small gourd. We would rest a while longer, it was decided, there were others who would benefit from the extra rest, and we could not burden our few remaining mules with Borno's weight, as well as the weight he carried on his shoulders.

And so, when it came time for the bears to come down from the hills, we still slept in their shelter. It happened in the evening, when the men came back, exhausted after an unsuccessful chase for a buck. The boys had returned with only two fish, and the rat traps had been full, so we added more water and a few more ferns to the stock, and spread the meager fare around.

I found my husband rummaging through our belongings, and asked if I could help. He took out the small pouch of Selvic strength that had been given to him while he served in the king's army. It contained shreds of desiccated mushrooms that grew on the briny crags of the north coast. When chewed to a paste and held under a warrior's tongue, it made him as brave and strong as ten men, if he was of true Selvic blood. Others, it made cower in terror. I had seen it done. It is administered as a test to boys as they enter manhood.

"We hunt with this tomorrow," my husband said.

I looked at him, shocked. This was a drastic measure. "There is a war to the north. We need that to defend ourselves." Only two of our village veterans had survived. We did not have much strength.

"We need food," he barked back. "We cannot wander forever living off rats and reeds." I said nothing, but locked eyes with him for a moment. He was as sick of this exile as I was. But this was rash, risky and foolish. Our stock of supplies, weapons, medicines, clothes, everything we had brought with us from home had dwindled, or broken or been worn away with use. We had nothing left to protect ourselves with. To waste something we would surely need when, not if, the raiders came further inland, was folly beyond words. I saw my husband's jaw tighten, unhappy with my standing in his way.

I lowered my eyes, but said, "Spring is here. There will be berries, eggs and birds soon. We will make it to another winter."

My husband tossed the pouch on top of the small heap of our belongings, then busied himself sharpening his spears. I stood beside him, stubbornly, not wanting him to have his way. He honed the wood carefully, meticulously, letting each curling shaving fall to a neat pile before blowing on the point and inspecting his work. Exile had changed him. He could no longer be the jaunty hunter and proud warrior honoured by his king to lead men. There was no room for show or anger or pride away from the oceans. Only grim determination and minute attention to detail.

He put down his finished spear and sighed. Without looking at me, he jerked his head towards the contentious pouch. "I have seen these growing high in the south facing mountains. I will send the boys to forage when the bears are gone. Then you will dry it."

I nodded, relieved for the first time in I could not recall how long, and touched his hand to tell him that I trusted him. He was a good man, and he had kept us safe all this time. He looked up and opened his mouth to speak, but the words never came out. Dyren, our youngest, and only, son, came screaming back to the cave, along with a friend. "B-b-b-bear," they cried, "Bear, bear, bear!"

The two boys had been tasked with laying out the tithes for the evening. They had gone a ways into the hills to avoid fouling our camp. By doing so, they had disturbed a large black bear. The creature, running faster than anything so lumbering should be able to manage came to a sudden halt when he saw our band of people. Lyta grabbed Dyren and pulled him deep into our cave. The other boys mother did the same. Myrna, who had been washing Borno's wounded foot picked him up and helped him hobble back to shelter. Two men with spears took positions behind them to guard their path.

My husband, tired and hungry, seeing the bear's hesitation, grabbed his newly sharpened weapon and ran forward. There would be meat tonight. Others, slower to react, or less brave, followed. The bear ran when faced with one man. When cornered by many, rose on its hind legs to fight. The creature was massive, taller and stronger than any giant that had ever waded into the battles of old. I watched from our shelter as the men worried and tired it with small wounds. It, in turn, slashed out paws as large as paddles trimmed with claws sharper than knives. Then the unthinkable happened. My husband fell.

Emboldened by its victory, the bear sprang through the hole created by his victim and ran down the hill, towards our camp.

Without thinking, my right hand grabbed a fish knife, my left slipped the pouch of Selvic strength around my neck, my feet took me out of the cave. I ground a large pinch of strength between my teeth, softening it with my saliva. The strength was not made for women. I had no idea what it would do for me. But I had no choice. With my husband gone, my people would not hesitate to turn me out. My eldest, Mirna, now a woman of her own right, would be the first and loudest advocate of my expulsion. Then who would keep Dyren out of trouble, or nurse Lyta through her frequent illnesses. These must have been my worries, but at the moment, I could only see the bear that would destroy us.

I moved the paste from between my teeth to under my tongue. I turned my feet to intercept the path of the bear. Men with spears chased it, yelling. At the animal or at me, I could not tell. It made no difference. Without warning, the ground shifted and the world took on a strange quality, as if looking through air again for the first time after a long dive. I stumbled, and felt dizzy. Something warm and hard pressed between my breasts. I was aware of the bear thundering towards me. The mens' voices, not far behind, were panicked now. I could not get up in time to run from the oncoming animal. The Selvic strength had failed me. I had no courage, no strength. I wished only that I had some means of defending myself, like the fish that grow large, spiky and inedible when faced with a predator they cannot fight.

The hard warm object between my breasts burned with a red hot heat. I felt the weight of several hundred pounds of angry animal crash down on me.

*********************

I was so cold. A painful shiver brought me out of my unpleasant memory into my more unpleasant predicament. The water than ran down my cheeks tasted salty. Well, this wouldn't do, I told myself sternly. I may have lost everything, been turned out by my people, but I was still alive. If I wanted to improve my situation, crying would not help. I flexed my shoulders and whimpered with pain. I clawed at the rotten trunk that had provided me with sustenance. It would give me shelter as well.

The rotten wood came free painfully in my hands, and the soil was soft and crumbling. A child could have dug itself a shelter under the rotten log before me. But I was weak. It took hours to make a shallow recess large enough for my torso only. Every few handfuls had me gasping in pain. I cannot count how many times dislodging a stubborn clod of dirt made my eyes go dark. I stopped then, gasping for breath and fighting back the fear that I would die before completing my shelter. My fingers were raw and bleeding by the time I had finished my work, but I could not see them. It had grown fully dark by the time I squeezed my body under the rotten tree. I pulled my knees excruciatingly to my chest, and hid my body behind the pile of freshly dug earth. The newly exposed dirt was warm and dry. The work had warmed my wet body so I no longer shivered. I was exhausted and longed for sleep. A pack of dogs howled in the night. I prayed, out of long, stubborn, immutable habit to the ocean to keep me safe from predators for what remained of the night.

I awoke when bright sunlight streamed past the wet bulwark I had dug for my protection last night. My wounds had crusted over during the night, and I could feel the fresh bare beginnings of a scab crack an splinter as I tried to move. I lay still and sobbed in my dim den, nothing more than a fevered, starving wounded animal, hiding from the wolves. I must have fallen asleep again, for the next thing I knew, the sun no longer shone directly into my hovel, and I was hungry. I fed myself on the same diet of ferns as I had the night before, and continued digging. A dead tree, I learned, was a wonder of life. I found snails and earthworms and innumerable grubs in the soil and around my home. They were bitter, or slimy, or earthy in flavour. I did not care. I closed my mind to the disgusting shape of my prey, and swallowed. If the food did not kill me, my fever certainly would. I chased the birds away every morning, and kept the vermin for myself.

The days grew into weeks, or months, I could not tell, but my body grew stronger, and I ventured out of my burrow. My back, shoulders and legs were covered with wounds half the size of my palm, as if great buebles had formed over the back side of my body then burst all at once. I could stand, but I only as a hunchback.  I also sported deeper cuts and livid bruises all over my body, but most in the region of my head and shoulders. Well, I shuddered, as I recalled, they had stoned me, to chase me away would have been acceptable. To kill me would have been preferred.

When I could walk on two feet, I learned that my watery god had heard me pray. A small river passed not five minutes from where I had dug myself in. Its water was frigid with ice melt, so cold that it made my head ache to drink from it directly. But a river meant life. On its banks grew grasses and plants I recognized; reeds I could, with time, weave into nets, roots I could dig for, shallows I could scour for fish. If only, I prayed, throwing half a raw grouse egg into the flowing frigid water, I had a blade, I could hunt and live off the fresh water as I had off the ocean. It was a mad prayer, but the water had made me giddy with hope. The river would carry my tithe to my father the ocean, and he would hear me. He must. Why else would this river be here, if not to be my salvation?

Days passed, but my back didn't straighten. The rains came and flodded my little borrow, time and time again. I found pile of boulders, each the size of a house standing in the middle of the forest. There was room enough between them for a man to squeeze into their shelter. I could still see the sky through leaves of a half dead massive oak long ago hollowed out by fire or disease. The rain would get in, but there would be a dry patch large enough for me to lay lengthwise and bring a few belongings in. I closed my eyes in the shadow of the rocks and laughed dryly. Belongings. I still did not have a blade. I had the flint in my belt, the Selvic strength I had taken, the clothes on my back, and a dozen strange needle like horns I'd found in scattered in a clearing not far from where I had made my first home.

I shuddered and pushed the bundle of needle sharp horns into a dark corner of the cave. I honestly didn't know why I'd gathered them. I suppose I thought that they would be useful for spearing fish, but day after day, I'd made up excuses to not bring them with me to the river. They frightened me. They seemed to have been shed by some large animal, and any creature equipped with a miriad of horns like that, I did not want to encounter. I saw no trace of a fight, or footprints, or any clue as to why these were shed in the clearing, except. ...

I sat painfully and abruptly upright. I had flint on me. If I gathered kindling, I could have a fire for the first time since my injuries. That was a thought to keep me going. But I could not shake the other ... The base of the horns, round and smooth, with a strangely thin leather still attached were almost the exact same size of the burst buebles on my back....

*******************

A few days after the discovery of my new home, I climbed to the top of a rocky hill, and peered over the canopy of trees. There was smoke in the horizon, a lot of it, and not too far off. The wars still raged to the north. Then I realized, with surprising distance, that village being burned was perhaps a day away. After the looters have left, I could sift through what they had left. Perhaps find the essentials I needed to survive, even a kitchen knife or rusty wood axe was better than what I had. I also realized that I would need to climb this hill daily, to keep an eye on where the fires moved. I wondered if I would have to leave my new shelter before winter.

I climbed down the hill, and checked the few traps I had laid. They were full. Cheerful, for the first time in a long time, I walked home, four rats swinging by their tails in one hand, the other clutching a rough staff I used to walk. 

As twilight fell, I heard the sound of wild dogs in the night. It was a familiar sound, we had heard it as we huddled in the cliffs last winter, our men lighting fires and posting extra sentries to keep the packs at bay. I had heard them over the last few days as well, helpless at night, shivering under my tree. I placed a long branch as thick as my wrist into the fire. At least tonight I would not be helpless, I thought, and turned to my rats roasting on the spit. 

The sound did not retreat at is did on other nights. If anything, it became louder. Whether driven by the fighting in the north, or drawn by the smell of roasting meat, as I listened, it seemed the pack circled my cave. 

I climbed up and out, hoping to see by what remained of the light, wanting to maintain the higher ground. I regretted my decision immediately. There were six of them lean, athletic, mottled black and brown, two on either side of me below me, and two behind. I was a cripple armed only with a burning brand. I swung it quickly in the air around me to keep the flame alive. Dogs can't climb, I told myself. As long as I remained up here, and they below me, ... And then I heard the distinct sound of claws scrambling up rock.

I heard the yelp of a creature cinging itself on my half cooked dinner. My stomach turned. I was finished. Deserted by my people, alone and crippled in the wild, squatting in the hunting grounds of a pack of wild dogs, I was not meant to survive these odds. I tried to brace myself for an end to my struggle. Yet as the burnt muzzle emerged from the same hole in my roof I had climbed out of, I threw my brand at it with all my strength and vaulted my self onto the branch of the oak sheltering my home. Without thinking I climbed a few more limbs and then slid into the hollow of the tree.

Six pairs of paws scrapped and scrabbled uselessly at the wooden walls of my new fort as I cowered, head in my hands on the inside. The baying was maddening. I screamed when a paw found a crack in the wood. It was quickly replaces by a shiny black nose and the sound fervent gnawing of wood. I sat, cornered, terrified, clasping my hands tight against my ears to keep the baying and the sound of wood cracking out. From the depths of the maelstrom, a thought floated to the surface. I still had the Selvic strength. If I was to die tonight, I would die fighting. I had thought that it had not effected me last time, but those needles... hadn't I wished to become a pufferfish? I had more time to prepare this time. I would do it right.

I transfered the mash under my tongue, and flet the heat of my grandmother's locket flare against my chest. I waited for the first wave of dizziness to pass, then gathered my courage and stood up. 

I straightened my crooked back, and reached up to find a handhold to climb out of the trap I had found myself in. My weakenend body rebelled, my field of vision narrowed, my knees gave out and I blacked out.

*******************************

I dreamt. Against the baying of the dogs and burning at my chest I say Myrna's face twisted in horror as the men of my village pulled the bear off my back. "Witch!" she screamed. I felt a sharp pain at my temple. "Demon," other voices echoed, and I felt stones pelt my back like hailstones. 

I turned to search for my children. I had to leave, I knew, but not without Dryen. My five year old, the pearl of my husband's heart. How would he survive without me. In my dream, it seemed I searched for minutes to find him, though I would not have survived if I had. I watched him struggle against Lyta's hold, my pale sickly middle daughter, both childrens' faces stained by tears. She held him back with all her strength and shouted at me to leave. 

I held out my hands to my two youngest children and said something plaintive. "Please" or "I love you" or "Come with me." 

Lyta's face contorted with rage, matching her sister's. She bent down to pick up a stone, and Dryen slipped out of her grasp. That knocked the breath out of me. My gentle frail Lyta, the quiet helpful girl who never left my side had taken up arms against me. Dryen ran between the legs of his elders into the crowd, crying, with the only desire to protect his mother. Stones that fell short of me fell on his back, and I knew I had only one option. 

I turned heel and ran.

In my dream, I found myself immediately in my little underground hovel. But this was a home, not a burrow. Still underground, it had strings of garlic and ropes of carrots hanging from the rafters, carved from the rotting tree that formed my roof, enormous now, wide enough to cover a modest cottage. Dogs bayed outside, but I barred the door. Snows came and melted. I built a fire, stayed warm and dry, cooked my fish, reached up to the ceiling to pick the fruits of my year's harvest. 

I heard the sharp sound of wood splitting open and a sudden tearing pain in my chest. Realizing that I was finished in the world beyond my dream, I thrust myself deeper into the fantasy of warmth and safety I had woven in my unconciousness. I was a child at my grandmother's house again. It was late summer, and her garden was bursting with vegetables of every color and shape. Its midmorning. I climb up onto the roof to select a gourd and handfuls of colorful beans for the day's stew. My grandmother's legs are no longer steady enough to reach the vines trained to climb up and over our cottage. I love it up here. I put a piece of dried fish on the ridge of the roof and I lay on my front steep slope, pressing my bare toes against the same shingles that support the runner beans, letting the salt breeze blow against my face, waiting for my favourite gull to pay me visit. Life could go on forever in this easy pace. Nothing ever need change.

**********************
