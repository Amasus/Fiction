\documentclass{article}
\usepackage{fullpage, verbatim}

\begin{document}

\emph{The ground trembles. The mountains blocking the northern sky do not notice, but the high tableland does. A vole freezes, mid-forage. A stalk of wild rice gracefully bends its head down to meet a naked hand. The rodent sniffs the night air. No predators. The ground twitches. The stalk springs back, upright and swaying. Indignant at failing to spread its seed. The dirt rolls back in a smooth, silent silken motion, indifferent to anything that had made the mistake of thinking it a home. A maw forms, occupying a quarter of the cold fertile plateau; as even and perfect as if sculpted by an artist, as symmetric as a mouth opening to intake a gasp of surprise, as smooth as the surface of an eye. A quartz face is revealed, dazzlingly white and glinting in the moonlight, smooth sleek until it ends sharply in a perfect circular hole, pitch black and so deep, one could imagine seeing the stars on the other end side of the earth through it--if there were anyone to look over the edge. From it comes a sigh of lament. 'It is gone.' The earth quavers again, and the eye is covered again, as quickly as smoothly as it formed. But for the broad stripe of crushed grasses and scattering of small dead creatures, there is no sign of disturbance. The earth sighs again. Resigned this time. 'The agony is about to begin.'}

\vspace{.25cm}

A woman shifts in her bed, moans as if in pain, then shifts and resettles herself under an embroidered mosquito net. A swan stretches its wings at her movement, then rests its head lightly on its back again and sleeps. It is too warm to bury its head amidst its feathers. A loose arm over a pillow displays four small bruises, arranged in a pattern that would hint at a fifth on the other side. If one were to follow the limb in one direction, one could see minute traces of blood and flesh under the nails, not hers. In the other direction one would encounter a torso covered with more bruises and a few open wounds, or might, if modesty did not require the flesh to be appropriately covered, even in sleep. There is, however, a hint of red peering out between the waist and the sash of the sari--not enough to stain the snow white fabric, but darker than the bright red border. It is simply a statement written on the pale olive skin. Not the beginning of a story, or an end, but a non-descript clause lost in the middle of a tome.

The woman shifts again, moans again, louder, and this time, opens her eyes. For a moment, she is confused, uncertain of the source of this new pain. But when it passes, a weary recognition lines her face. She gets up, careful not to uncover more than a sliver of her belly and no leg above the ankle, untucks the mosquito net, stands and stretches. At no point does she cross an invisible line dividing the bed, even though she knows without looking that the other half is empty. She'd never imagined that she could depend on him in times like this, even if she'd wanted to. The swan looks up, torn between annoyance and concern. Then leaves the stifled air within the mosquito netting to sleep in the cooler air on the veranda. "It’s not fair," she says to it, as it passes. "To be married to the creator of all things, and not be allowed to create one single child." The bird bobs its head in acknowledgement, but it has heard this complaint many times, and has no comfort to offer its mistress. 

After some time, the woman picks up a veena, and follows her companion outside. It will be a long uncomfortable night, waiting for her body to rid itself of the dead. The instrument will give her something to do with her hands.

\section{Oxford, UK}

Ana dropped her bag at the table where her flat mate was waiting. Wednesday evenings at Jude the Obscure had become a tradition, mostly because it was near the new math building. Both Ada and Ana spent more time at work or travelling than in the Edwardian town house they shared. This was a chance to keep in touch through busy schedules. As Ana became more and unhappy with her boss and her department, Ada started smirking and pointing out the irony of the choice. Sometimes, Ada’s smugness really got under her skin. Not today. Today, she needed a drink.

Ana’s flat mate sat nursing a glass of white wine, while a large orange juice sat across the table for Anna, water beading on the surface of the glass. "Sorry I'm late. It’s been a day."

"I can't wait to hear." Ada's voice was simultaneously reserved, patient and slightly sardonic. She was so infuriatingly British. The fact of the matter was, Ada had heard all of Ana’s complaints before. As a good friend, she knew she will hear them again. Ada passed Ana a basket of soggy cold chips. 

Ana nibbled a cold piece of potato that managed to be both oily and dry at the same time, sipped her juice and pulled a face. Well, there was at least one part of her day she could fix. She went to the bar and got a shot of vodka added to her drink. 

Returning, she said with false cheer, "Well, I discovered a new particle today."

Ada pointedly looked away from her augmented drink. "And..."

"And," Ana continued, as if prompting a particularly dull student, "someone is supposed to be proud of me."

Ada tipped her glass gently towards her. "Mirror?"

"Oh bugger off! I know you are all about women defining the standards of our own success. But seriously. There is a limit."

"Fine." Ada deliberately rested her chin on the knuckles of her right hand and put a look of rapt attention on her face. "Tell me about it."

Ana opened her mouth to speak. Contemplated her words. Closed her mouth, thought and opened it again before throwing a napkin across the table at her friend. Both women giggled. The one time Ada had tried to explain pre-Islamic Nigerian politics to her, it had nearly ruined their friendship. Ana couldn't pronounce the dynastic names, and Ada, Ayodele, accused her of some colonial oppression or another. Never mind that Ana wasn't British. In fact, she had a colonial history or her own. It was just that it simply didn't matter to her. After that, trying to have a conversation about high energy physics and particles that may or may not exist seemed, well, unwise.

"Okay, love. Fess up." Ada put a chocolate colored hand on Ana's olive one. Ana still hadn't gotten used to the northern English verbal mannerism. While cold by every other measure of social interaction, anyone from a bus driver to your mother could use the word as a casual attempt at intimacy. Ana's head kept trying to replace it with `mi amor' and it rankled every time. Ada was from Liverpool. If she'd had a day like this, she could take a train to her family. Celebrate or cry. Maybe just sit in front of the telly and silently hold hands with someone she loved. Because there was nothing to be said. Ana's family, including Marco, were in Buenos Aires. She missed them. Oh God did she miss them today.

"Ana?" Ada prompted when she doesn't answer. Ana furiously struggled for an angle. This was too delicate, and she didn't know how to face it just yet.

"The gas lighter doesn't think it’s significant," she said instead.

"Your new particle? But that's ridiculous. How many people discover subatomic particles every day?"

"More than you think." Ana shrugged, letting the inaccuracy slide. "And it’s not a real particle. People have been studying CPT asymmetry for decades, including gas lighter. I'm just the last in a long line of people publishing on it. It’s a made up particle in a fictional theory." Barely more than a phase difference her boss had called it.

Ada's eyebrows quirked in her defensive. "And did \emph{he} discover a particle? I bet he's just jealous."

"No," Ana sighed. She braced herself to reveal the next piece of today's travesty. "He's angry that I didn't take the position at Hull."

"The Lecturer position?" Ada gasped, utterly taken aback. A lectureship was a permanent position, and as rare as a duck's tooth. It meant an end to the post doc purgatory and the start of an independent academic life. A chance to dig in and do some serious work. Ana wilted inside. She couldn't even make her politically radical flat mate understand. Maybe this was the final sign that she just wasn't good enough to run with this pack of extremely competent, inhumanly ambitious people. She didn't want what these people wanted. She wasn't `serious' like them. She wanted a family, and she wanted to be with Marco. After flushing their first attempt down a toilet this morning, she just wanted to go home to him. 

"Why?" Ada had recovered and was fully sympathetic now. She placed her wine glass to the corner of the table, and reached across to take both of Ana's hands in hers. "Ana, what happened?"

"Hull's cold" Ana whined. "I'm so sick of these gray summers and dark winters." 

"Mmm hmm." Ada pursed her lips patiently and waited for Ana to get to the actual point. These complaints were as old. As boring and uninformative as they were oft repeated.

"Hull is hours from an international airport," she began again, trying desperately not to sound shrill. "I want a family. How am I going to do that if I have to get on a plane for twenty hours every month to try."

Ana saw the penny drop for Ada. Her eyes flicked from the altered juice to her devastated face. "Oh sweetheart," she cooed. Then she got up decisively, grabbed her purse with one hand and tugged at Ana with the other. "Let's go home. I'll put the kettle on, then I'll make us tea. We can eat in front of the telly. You need to put your feet up."

\section{}
An owl lands on the harmonium. The swan spreads its wings threateningly, protecting both its mistress's instruments and her solitude when she works. The owl gives a desultory hoot and hops to the threshold, but not outside the room. The bright blue curtain hanging in the doorway cascades over its brilliant white shoulders. The swan doesn't hiss, but keeps its eyes on the intruder. 

But the commotion is enough to disturb the mistress. She had had a hard morning, and found it impossible to give her music her full attention. She puts down her flute in annoyance, sees the owl, then brightens immediately. 

Wrinkled and uncombed, she jumps up with a rustle of silk and runs to the front door. "Laks! Oh it’s so good to see you."

"Hi sis. Well, aren't you in a state. Is this a bad time?" Her sister is immaculately turned out, as always. In spite of the long journey, her bright red sari remains crisply pleated, her hair remains perfectly pinned and devoid of feathers, the khol around her eyes unsmudged by sweat, wind or dust. 

"Not at all. I'm all alone here. Will you stay the night?"

Laks contemplates the question while reaching over to hug her sister. Sara lets her, against her better judgement. As soon as her sister touches her, she knows, as Sara knew she would. Laks pulls back to look her sister in the eye. Concern, worried, but not chastising. When she doesn't speak, Sara relaxes. Her sister was there for her. This would not be about her relationship with her husband. It had been a hard morning.

Finally Laks gives her sister a peck on the cheek and says "Well, I'm parched. Let me get the drinks. Music room, studio or library?"

"Studio." Sara mumbles. She can't face her instruments again. The studio had a window opening to the west. It caught the early afternoon light perfectly. She chases the birds out into the courtyard, and settles herself by an easel.

Laks returns a few minutes later with two glasses of lemonade and a plate of salted cucumbers. She pauses behind her sister to admire her charcoal sketch of a rainy day. Even half complete, she can see the droop of the banana leaves, and feel the water dripping off the skeleton of a thatched roof. She can practically smell the wet earth. "You are so talented," she whispers.

Sara starts and puts down the charcoal, waving away the picture modestly. They cross the room to a wide couch, and Sara settles herself against her sister’s shoulder, plate of cucumbers resting on her belly. Laks strokes the top of her head for a while, then idly starts undoing her braid. "What happened?" 

Sara shrugs. The two sisters sit in companionable silence. Laks did not want to pressure her sister, and Sara had not processed everything that had happened that day yet. It was too momentous for even her to comprehend. And she did not know if she could tackle this alone. But now, with her sister's steadying presence, she could contemplate unpacking recent events. She searched for an angle.

When the plate of cucumbers lies empty on Sara's stomach, but before Laks threatens to get up, Sara asks, "Do you ever wonder why we are as we are?"

Laks shifts to look at her quizzically, then shakes her head. "I don't know. These deep philosophical questions have always been your department." 

"No, I don't mean like that." Sara gets up to face her sister. This was going to be a serious conversation, better conducted face to face. Immediately she regrets the loss of physical contact and takes her sister's hand. "What do I mean?" She pauses to find better words. "Have you ever wondered why we can't be different?"

Laks frowns. "Sure. Every time I see you draw or perform. I wish I had a fraction of your talent." She squeezes her hand and smiles "But there's no point in being jealous. I'm me. You're my sister. It’s just how we are made."

Sara doesn't loosen her grip. The conversation was headed where she didn't want it to, but she didn't quite have the energy to direct it elsewhere. "I am. Jealous I mean. Of you." 

"Oh sis..."

"I mean look at you. Everything you touch blooms. Every venture you put your hand to succeeds." 

Laks laughs. "That because I know I can't carry a tune, and thus never try."

"That's not what I mean. You and your husband. You're a pair. You fix things together. You make them go on. You work with him, hand in hand even. You could have children if you wanted to, but you don't. You have so much more ..." Sara feels the tears welling in her eyes and stops... . This is not where she wanted the conversation to go. She feels the exhaustion and the pain of the miscarriage closing in on her. Lying devastated in the empty aviary at dawn, half sobbing, half screaming in anger and frustration. How many times had she tried for and lost this child? Why could she not make anything for herself, separate from what he could take away from her? She had been pounding the earth with her fists, when suddenly it happened. Her eye had opened. A pinprick of light exploded in her body. And then before the pain could consume her, with muscles and power she didn't know she had, she pushed it out into the world where it expanded with terrifying speed, dark and hot and all consuming. She had screamed in horror of the thing throbbing and floating in the air before her. Then it simply drifted away. Sara doesn't remember much more between screaming and waking up in the mid-day sun, covered in sweat and blood. She wanted to believe that it was a pain fueled dream, but she had wanted this so much... She was terrified that it wasn't.

"Oh Sara." Laks bundles her sister up in her arms and rocks her gently. "I wish I knew what you say. You are the one good with words." She lets Sara sob quietly in her arms for a while, flooded with relief at having a friend nearby. Then she pushes her away and says "You've just lost a pregnancy, and you've been alone all day. It makes sense that you want your life to be different. I'm here now. Read me something. It'll make you feel better."

Several verses, and three courses of a meal later, Sara feels better. Her sister is such a better cook than she will ever be. Just watching her take command of the kitchen makes Sara feel certain that she will be provided for. Now, with a full stomach, she feels brave enough to try again. "Laks, if I could find a way to get away from all this, would you come with me?"

Laks stops making neat piles of her fish bones at the side of her plate and looks up, alarmed. "Sweetheart, there's nowhere to go that he couldn't find you."

"But what if there was," Sara insists. "What if I could prove to you that there was such a place. Would you come?"

"Hush dearest. This is crazy talk. I know it’s been tough. You hate being a housewife, you've wanted a child for so long and you've had your hopes dashed so many times for so very many reasons. But right now, this is the hormones and trauma talking." Laks rests her left hand on Sara's shoulder. "You need sleep. Running away won't solve anything."

"No. Listen." Sara raises her voice and shrugs off the hand. Laks flinches. "I'm sorry. I just..." She takes a deep breath, and opens her eye a fraction. She can feel it, somewhere, burning black with heat and growing at an unimaginable rate. It is real. "Look, I've done something. Just listen." When Laks continues to look skeptical, she adds "Please?"

Sara tells her sister about the previous night. The argument and the beating. The miscarriage. Then, before her sister can interrupt with sympathy, she takes a deep breath and tells her of the black thing that came out of her like an afterbirth and floated away. When she's done, her sister sits in stunned silence. Sara knows that this is not something that she should have done, and her sister has always been one for playing by the rules. After several second of nervous silence, Sara asks "Laks? Say something. Are you mad at me?"

"This isn't..." she stutters. "I mean. No, I'm not mad. But," she struggles for meaning, "this isn't right. You shouldn't be able to do that. It’s not who we are."

"Laks, \emph{this} isn't who we are. Look at us. You were always the domestic one. Cooking and sewing and gardening. When mother went mad over our brother, you ran the household. Better than mother did, even. You could take the craziest idea and make it profitable." 

Laks smiled slyly. "Remember the elephants?"

"Oh do I!" Sara snorted with amusement. "Between that and the cobra, I swore I'd never touch another land animal." 

"Vasuki. Hoo boy, was father mad." Both women chuckle then sigh, the tension broken.

"But seriously, look at me," Sara continues, "I was given every opportunity to study and make something of myself. We used to fight like cats and dogs over it. Why wasn't I helping out with the animals? It wasn't fair that you never had a tutor come to the house for you. And where are we now? I'm stuck at home all day while you go out and make the world beautiful. I'm certain this isn't how we were supposed to end up."

Laks tucks a lose strand of hair behind her sister's ear. "Sara, you’re frustrated. I get that..."

But Sara can't take the platitudes any more. The horror and the hope of her creation is too enormous. The opportunity presented too momentous and fragile to ignore. She presses on. "Look, I know you are happy where you are. You don't want to upset that. But this could be a chance for us to start over. Be who we both want to be. It's probably horribly selfish of me even to ask.... But will you think about it?"

Laks frowns and stares at her empty plate for a long time. "Sure. I'll think about it. But before you dream too big, tell me this. Do you know if our husbands can find it? And if they can, can father destroy it?"

Her sister's words knock the wind out of Sara. Her father. She hadn't even thought of that.


\section{Liverpool, UK}

"'Tia Ada, how \emph{are} you?" Francesca's chipper voice flounced into the room, and she dropped her backpack on the floor near the bed."

\emph{"Fran! How lovely! Do your parents know you are here?"} Ayodele thought. She knew the words would appear on the screen above her bed. It was frustrating that the technology couldn't convey the wave of warmth filling her body to bursting. The implant had an option to have a female voice speak for her, but it sounded like a perfectly respectable young woman from Oxford, not a black grandmother from Manchester. Not only did it get the nuances wrong, it raised her blood pressure if she heard it go on for too long. It was better just to have the text.

Fran rolled her eyes. "I'm going to uni next year? I have my own web business. I even paid for my own ticket! I'm not a kid anymore."

\emph{"Do your parents know?"}

"Yes, Tia." Fran gave the requisite sulk. "I told you I'd be coming up on Saturday. I'm going to the game tonight, then crashing here. Remember?" she looked a little worried, as if, in her infirmity, Ayodele might be losing her memory too.

She most certainly was not. \emph{"Goodness. Is it Saturday already? It’s so hard to keep track when one's stuck in bed all day. Well. Give us a hug then."} Fran smelled so young. Due to the stroke, Ayodele couldn't feel the warmth of the embrace across most of her body. But she could see and smell. Fran favored a shampoo that smelt of strawberries and a deodorant that reminded her of her grandmother's talcum powder. She couldn't imagine a fresher or more youthful combination. 

Fran pulled away suddenly. "Wait. I forgot. Mama sent you these." She reached down and brought out a bouquet of lavender and hyacinth. She placed it on her pillow. They smelled fantastic. "They're from the greenhouse. Just a mo." Fran bounced into the kitchen of the Ayodele’s studio and returned with the vase her mother had insisted Ayodele keep specifically for this purpose, and a plate full of chocolate biscuits Ayodele insisted her nurse keep in stock specifically for the enjoyment of her god-daughter's daughter. The Fran flicked a celebrity quiz show from her watch to the screen and settled down on the chair in the corner of the bedroom. The show wasn't exactly her cup of tea, but the company was good. She soon backgrounded the noise and fell asleep. 

It really was special, to have Ana's grandkids surrounding her. Ayodele knew how lucky she was. Ayodele had never married, so she had no family of her own to speak of any more. But Ana. Ana had always stepped in. After rejecting the offer from Hull, Ana had scrapped together a tutoring position at St. Peter's college for a year until she got a position at Manchester. She tried for a few more years to get a position in Argentina, but no one seemed interested. So by the time Ayodele was an assistant professor at Howard University, Marcos had given up his law practice and moved to Manchester as an advocate for refugees. Ayodele's mother got diagnosed with lymphoma when Ayodele was up for tenure. Ana, already a lecturer and mother of two, took the hour long train journey a couple times a week for the day to day care needs, while her brother, Mobo, handled everything else. Ayodele visited when she could, but the increasingly pointed e-mails from her mother just made it difficult. Several years later, when her father passed of a heart attack when she was visiting the University of Lagos on a sabbatical, Ayodele only managed to come to the funeral. Marcos and Mobo took care of the arrangements while Ana sat holding her mother's hand. Somewhere in between, Ana and Marcos Cerilo-Gusman had been accepted into the Seyi clan. 

She'd always been jealous of Ana for her accomplishments. There aren't a lot of university professors in any field who can claim to have come up with the seed that revolutionized the way the world worked 50 years later. Ayodele had given up on so many relationships pursuing that dream, and it seemed to just fall into Ana's lap. Certainly, Ana had suffered. What woman of color didn't. But somehow, she manage to have a family and a career. As much as she loved Ana, a primal part of Ayodele's brain screamed at the injustice.

Mobo was diagnosed with early onset Alzheimer’s about the time Ayodele was thinking about retiring. While the neuroscience boom that was somehow inspired by Ana's new particle had already let paralytics interact with the world, there was no cure for the degenerative disease. Mobo had never been the type of man who could be faithful to a woman, so neither of his wives could be depended on, and his three children all claimed to be too busy. So Ayodele retired and returned to Liverpool to take her of him. When the stress proved too much, she took the train to Manchester. The Cerilo-Gusman's welcomed her home. Within a few months, any trace of jealousy had disappeared. 

Ana had passed a few years ago. But that made no difference to the rest of the family. She was Tia Ada, end of story.

When Ayodele awoke, the sun was low in the sky. Out of the corner of her eye, she could see the blue projected screen where Fran was doing her homework. The nurse had come and gone. She could see her liquid dinner on the bedside table. Fran must have offered to feed her.

She asked her implant to ding to get Fran's attention. \emph{When's your game?}

Fran pouted. "My friends ditched me. One of them is grounded and the other is breaking up with his boyfriend." She shrugged. "Don't feel like going it alone."

\emph{In my day, kids snuck out of windows to go to Manchester Liverpool matches.}

Fran put on a shocked expression. "'Tia Ada! Would you really want me going around with kids who climb up and down drainpipes?!"

Ayodele again found herself cursing the Londoner who thought that a single female voice for all of the UK would be sufficient. She wanted to laugh. She wanted to fill the small room with the sound of a grandmother's joy. All she could manage was a single tear rolling down her right cheek.

"Anyway," Fran continued, “I have this dumb ethics essay to write. Don't suppose you could help, could ya?"

\emph{"Go on."}

"Well, you know the climate trials that ended when I was about 9? Where everyone had to contribute to a cleanup fund based on what had been emitted when?"

\emph{"I've only had a stroke love. I'm not daft."}

"Right. Sorry. Well, anyhow ... our Ethics teacher wants us to write about what would happen if we could trace the guilt for World War II back in the same way. Would Einstein be convicted, Nuremburg style, for example. Which is dumb, since Nan's discovery doesn't work like that."

\emph{"Because God, if he exists, is indifferent to human action."}

"What?"

\emph{"Never mind. It’s an old argument your Nan and I used to have. What have you written so far?"}

Fran pouted. "That Ms. Morrisey has completely misunderstood the point and potential of Professor Cerilo-Gusman's discovery." When Ayodele did not respond, she offered "That's probably not going to fly, is it?"

\emph{"Probably not."}

Fran came over to perch on the bed. "So what do I do?"

\emph{"Well, what do you know about your Nan's discovery?"}

"Well," Fran thought for a moment. "It somehow allows you to follow the path of a subatomic particle back through time, for a while at least. So you can see where atoms and stuff came from originally. Isn't that kinda how your implant works? It can see which neurons are firing, and therefore can read your mind?"

Ayodele was impressed. Fran was 17, and already could explain it as well as she could. \emph{"So what about time travel?"}

"What about it." Fran sounded belligerent. Defensive. "Nan never thought this would enable that."

\emph{"No, but others did. Do in fact."} Fran's face remained set. \emph{"I'm not asking you to defile your Nan's work. Just see it from other people's points of view."} Fran wasn't budging. Ayodele understood. While the climate trials were ongoing, Ana had been drawn into the middle of political controversy. Ever apolitical, she'd been forced to speak to defend her science to poticians and policy makers. It has been hard on Ana. Fran, well, it had started before she'd been born, hadn't it. That harried version of her Nan was all she knew. \emph{"From the point of view of other scientists your Nan respected, then. Talk about the responsibilities we bear for our forefather's actions. Time travel. Be creative."} Fran reluctantly gave in. 

The girl slid back onto the cushioned chair and went back to work on her screen. Ayodele asked her implant to pull up her e-mail, and turned to the day's messages. As the years had worn on, she got fewer and fewer requests for collaborations or reviews or letters of recommendation. Instead, today she got an announcement for an ex-student's 60th birthday conference. Well, going was out of the question, but she could at least write him a nice note.

Eventually the alarm for dinner went off. Fran saved her work and took the liquid meal to the kitchenette. Some bottles went in the microwave, others came out of the fridge. One had to keep it interesting somehow. "'Tia Ada?"

\emph{"Yes love."}

"Nan's work was really important, right? I mean, we're using it everywhere, and still debating it all over the place."

\emph{"Certainly."}

"Then why didn't she ever get the Nobel prize for it?"

Ayodele could just see the face Ana would pull at this question. Instead of speaking for her friend, she just thought \emph{"Because mathematicians don't win Nobels."}

Fran sighed. It expressed Ana's exasperation as eloquently as if the girl had suddenly been possessed by her ghost. "The Fields Medal then." 

\emph{"Because by the time they'd realized its import, she was too old."}

"Tia." The cheek Ayodele heard was all Fran. It was pure teenage whine and annoyance and an older generation that just doesn't understand. "You know what I mean."

\emph{"And so do you, love."}

\section{}
\emph{The universe spins. Gasses and dust hum, vibrate, dance, coallesce into nebulous communities. Gossip and radiate. Form stars, or planets, rocks and elements. Or dissipate in singles and small clusters to find other commuinties to orbit around. The heat and pressure of creation has eased. The immobilizing cold of old age is inconcievably far away. This is a moment of youth and potential. Creation and jubilation, when chance meetings and opportune meldings are more powerful than the stern dictates of fate. }

\emph{Something obstructs this casual chaos. For an instant, in the vast tepid spaces between nebulae and galaxies, a barrier is formed. Anything drifting aimlessly between graviational wells is suddenly there. A new nebula. But perfectly shaped between two arcing walls. Beyond it, a vaccuum so hard that it hurts even the vaccua of space. It glows hot and ultra-violet within itself, bursting with the energy of formation. }

\emph{The glowing eye surveys the blackness. It is a beacon, seeking understanding, a flashlight searching for clues, a mother peering into a child's face to ferret out the secrets hidden in its pockets.}

\emph{Protons resonate as one, disappointed. "This is exactly the same." The walls of the eye shaped nebula relax, as if ready to release its captive constituents. Then it pauses, hesitates and forms again, tighter, as if squinting. A small star starts coalescing in the center. A bright pupil born out of pressure and confusion. "No" the star rumbles. "Not the same." The hot gas flares, bigger brighter, a hope that what it sees truly is. "There is history here. And knowledge. Time is not the same." }

\emph{The eye relaxes. The walls release their captives. The star shatters back into its myriad particles, and, with a final resonating vibration, every inter galactic particle is carefully placed back where it was found. "I only hope it is enough."}

\vspace{5 pt}

A woman's scream sounds from somewhere above her. Sara jerks and pulls out of her meditation. She rubs her eyes and tries desperately to reorient herself to this reality, vaguely aware of her swan hastily waddling outside. Bother, she'd knocked over her inkwell all over her desk and the scroll she was working on. Hastily, she scrabbles for the blotting rags to contain the growing pool of dark indigo before it took over the entire writing surface. From above, she hears her swan hiss and beat its wings against the peacock's indignant cries on the thatch above.

Sara rolls her eyes and wipes her now inky hands as clean as she possibly can in a few seconds. A morning's work completely ruined. She should have gone to the prayer room to check on her creation, she knew. But she was not expecting visitors. She had not expected to be interrupted. Resigned at all the work she would have to redo, she rises, thinks about straightening her sari, then decides against it, given the state of her hands.  Instead, she places a bright smile carefully on her face, she walks to the doorway to meet her brother.

"Kartik!" she says brightly, "How good of you to stop by." Sara had been enjoying her solitude. Laks had stayed for two days, until she had cried all the tears she wished to cry, and until she was sure that her body was healing properly from the miscarriage. A week ago, Laks had awoken just before dawn to find Sara already dressed and bent over a pile of empty scrolls, alone in the glow of an oil lamp. At the door of the library, the swan dozed, sleepily guarding its mistress from anyone daring to intrude on her creativity. 

Laks had smiled and padded off towards the kitchen. By the time she returned, dawn had lit the window well enough to work without the help of lamplight, but Sara had not moved to dim it. So Laks bribed the guardian swan with a spare idli and set breakfast on the ground beside the kneeling scribe. She snuffed the flame, and refilled the bowl for her sister to use again at dusk, then brought a few more inkwells for future use, and arranged the tumble of empty scrolls into neat piles an easy arms reach away from the low desk. Sara had already finished one scroll and hung it out to dry. Laks and looked at it and laughed softly. At that, Sara had looked up. "You don't like it?" she asked, uncertain.

"I am not a literary critic," Laks protested. "Its just good to see you working again."

Sara shrugged, relieved, and turned back to her writing, but Laks was quicker. A glass of thick amchoor topped salted buffalo cream appeared before her. "Eat," she said. "You are still recovering." Sara took the cup, sipped, and reread her last stanza. Laks shook her head in exasperation. But also relief. If her sister was writing, she would be okay.

"I'll see myself out then," she said. Sara did not acknowledge her, let alone see her out. She did, however, hear her call out, "Come again soon," as her owl took flight.

In the week since Laks had left, in theory, Sara had done nothing but write. In actuality, her writing came second to the ever growing needs of her creation. The universe was young and active an needy. Stars formed constantly needing her guidance. And if she turned away for too long, they died again, just as quickly, and their remains demanded even more of her attention. In the other universe, the one she occupied, the triumverate had divided up the work between them, so that no one person had to spend too much time watching. Instead, she was creator, preserver and destroyer all in one. It was exhilarating, to be everything for her own creation, to be so important and needed for it's survival. And it was exhausting. At any other time, a week was ample time to write out a simple epic. And now, she was barely a quarter finished, her creation pulling her away whenever it felt the need. 

At first, she wondered if she had just made a copy of what the triumverate had done. She didn't want this to be a copy of her husband's work. After the first few days, however, Sara had started noticing the differences. Time flowed differently in her creation somehow. Hers was imbued with something uniquely hers: history, knowledge, stories. As particles moved in her creation, they remembered where they came from. They left a trail in time, as thin as a spider's web, but a solid as steel. If you were clever, and more than a little patient, you could trace the threads back to the very beginning of time. It was wonderful. And it was hers. She desperately wanted to share this with someone. And she could not she knew. A universe is too fragile. It could be cracked as easily as an egg, if one knew how. And her family knew definitely how. She could not let her creation, \emph{her} creation, be destroyed. So she would guard it and guide it, as the trinity did their own. And she would hide it from them with every bit of guile she could muster.

But her brother was looking at her oddly, his perfectly oiled hair falling in waves just past his shoulders, twirling his spear thoughtfully on its end. That would not do. He could not have reason to think anything was amiss. Sara looked down at the mess of her hands in embarrassment. "Your peacock startled me," she offers in apology. 

"Really, Sara. You have always been so careless." That annoys her. For all the bloody dusty messes her brother had dragged through her childhood home growing up. She raises her chin to where the mounts squabbled on the roof. "Can you take him to the aviary? I just had that rethached." 

Kartik gives a jaunty shrug, completely unabashed. "Certainly, sister. Only Brahma is seeing to his mount now. I'll tend to mine as soon as he is done."

Sara's stomach sinks. Her husband. Why was her husband home? And in her brother's company? When they had last ... Sara takes a deep breath and pushes away the memory of their last argument. When she had last seen him, he had given her the impression that he would be away on a long meditation. A meditation was a solitary, sacred affair. It could take anywhere from months to years. He'd only been away for a week. Had Kartik gone to fetch him? If so, why? One god did not disturb another god's purification without a \emph{very} good reason. What could possibly be the emergency?

Sara shakes herself out of her reverie. She has no time for this. Her husband is home, and he would be hungry. She looks desperately at her filthy hands and heads to the well to draw water for tea.

%\begin{center} ... \end{center}

"Ah, the dutiful wife appears," Brahma drawls as she approaches her library with a tray full of tea and savories and sweets. She does not miss the sarcasm in his tone. Why do they have to be in her library? The room is in complete disarray. Her music room or studio would have done nicely, as would his study. Sara swore he did this specifically to humiliate her.

"Lunch will be ready soon," she responds flatly, placing her husband's tea cup carefully beside him. He sits, bearded, fair and many headed, languidly on a carpet propped on an elbow against a pillow. One head scrutinizes her every movement, while another looks irritably in her brother's direction, while a third reads one of her scrolls. He must have picked it up at random from the collection of finished writing, beside her desk, scattering the pile in the process. She would just have to reorganize when he went to bed. Sara sighs. 

"We won't be staying for lunch, sister," Kartik replies from the window. He holds himself stiffly upright, back to his brother-in-law, as if the two had been arguing. "We have urgent business to attend to. We cannot stay here long." 

So there was a reason for this unexpected interruption. She bites her tongue, knowing that asking the question would not get her answers. "A man must eat," Brahma protests, "and I have yet to pack." 

Kartik turns impatiently and takes the tea from Sara's hands. "Fine," he retorts. "Eat, and gather your belongings." He turns to her. "Sara, how quickly can you serve a simple lunch?" 

As she speeds up the stairs to pack her husband's ceremonial lotus, rosary, veda scroll and water pot, she feels her universe tug at the edge of he consciousness as it had done countless times while she wrote. A universe needs the constant attention of its creators, else it withers and warps, grows into something it should not become. The triumverate took it in turns, creating, preserving and destroying what needed to be born, kept alive, or killed. She, at this moment, has no one. It was one thing to interrupt her writing when she was alone in the house. It was exhausting to put down her writing at a moments notice to tend to the growing thing's every need. But now, with her creation pulling at her attention like the cry of a new born baby, and the pressures of her domestic duties. Sara didn't think she could manage. 

Sara quickly rummages through her husbands belonging to pack what he would need for a meeting of, how long? Had he said? She put some extra clothes in the pack, just in case. She ignored the frantic pull for her attention fraying the edges of her consciousness. Her creation would have to wait. Just for a little bit. Until she'd fed her husband and her brother and they had set off on whatever urgent mission called them here. It would have to figure out how to sooth itself for the moment. And after that, she would get help. She would go to the only person she knew who could get her out of this bind.

\section{Pheonix, Arizona, USA}
Fran buzzed in her next interview. 

When the prison guards leave, she finds herself facing a short muscular young man in his 20s of hispanic origin. She looked down at his file. Honduran refugee, it said. His name was Carlos Iglesias, and that she was seeing him because of a charge of inciting violence. Reading further on in the file, she saw the he was 26 years old, and had been in this facility for the last nine years on charges of aggravated assault and possession of illicit substances. She sighed and told herself to remain impartial and professional, but really, the man's file was a caricature of himself. 

When she looked back across the table, Fran struggled to separate this man from the dozen others she interviewed this month.  Short, tall, chubby or chiseled, young or old, they all blended into the same person: Hispanic, mentally ill, violent.

Fran took a sip of her coffee to regain composure. She hated this part of her job. She wasn’t a tiny woman, but she wasn’t large either. The man across the desk from her was easily a head taller than her. She knew that her boss promised to give her the men that would react well to her for as long as he could. She didn’t quite know what that meant, but suspected that this was why she never interviewed the white or African American inmates. 

When she looked across the desk, she saw the prisoner was fidgeting in his chair, staring at his knees. Good, he was nervous. It might mean that he wouldn’t cause trouble.

"Good afternoon, Carlos" she recited the first part of her script in a crisp professional tone. "My name is Ms. Gusman-Pe\~{n}a. I'm here to conduct your psychological scan. Our interaction is being recorded. Do you know why you are here?" 

The prisoner sat, recalcitrant, sulky, and silent. Fran sighed with relief. He was going to stone wall her. It didn’t matter. He wouldn’t be able to stonewall the scanner. It would mean extra work on her part, of course, without verbal cues to contextualize the psychological readout, but at least it meant that he wouldn’t call her names, coconut, or perra or worse.

She recited the next part of her script. "You are here because of your involvement with the fight that broke out in the yard on Monday morning. Are you aware of this incident?" Still nothing. Fran pushed on. "You have a right to remain silent, but I must warn you that your silence will guarantee a scan. The procedure is not painful, but it may reveal to us other information that may affect further disciplinary actions and the future of your incarceration here. Is there anything you wish to say?"

She finished her speech and counted slowly to 60, as instructed. This was a shit job, and she was going to burn out in it before she hit 30. This was not how she’d imagined her life playing out. She had read psychology in uni, hoping to do some good in the world. She met Michael, an American causal archeologist studying abroad year at UCL, did the trans-Atlantic long distance thing for a few years, then married as soon as he had a good enough job and moved to the American southwest where she helped prison authorities sort out troublemakers. Her job was to identify and direct those who wanted to deal with their mental health issues find their way to medication or counseling. 

She was helping people, she told herself. And she wasn’t in danger. The restraints saw to that, and Michael would never have let her take this job if she was in any danger. But the looks her clients gave her… It got under her skin. 

When she got to 50, Mr. Iglesias said "The boys told me to be quiet. So I'll just take the scan, miss."

\emph{A polite one,} a corner of her brain scoffed. She finished counting and carefully placed the electrodes in the correct position. Then she retreated sat behind her screen where the read out was being recorded. 

She proceeded with the interview. First the banal, where were you born, what are your parent’s names, where did you go to school? Then build up to the reason the man sat before her. The man remained silent, and the electrodes picked up on the reflexive reactions he had to her questions. Sometimes a prisoner could fool the scanner by conjuring up a lewd thought at just the right moment, but generally speaking, the procedure picked up on the actual answers to her questions. It was like asking someone not to think of an elephant. Do it correctly, and that’s all they \emph{can} think of.

It was 3:30 when the guards had escorted her prisoner out. Her calendar indicated that this was her last interview for the day. She had half an afternoon and four reports to write. Fran rubbed her temples, trying to come up with four different narratives for the four different men she’d spoken to today. But they blended together before her eyes. Why were their stories always so damned similar?

“Well isn’t that fascinating,” her friend Meghan had said recently. She’d flown back to Cardiff to be at her best friend’s wedding. Meghan was a little high, to calm her nerves for the ceremony, but it always made her judgmental.

Fran had simply glared. The two women had gone separate ways after college, and Fran knew Meghan didn’t approve of all her decisions. But it had been Meghan’s wedding day, so she bit her tongue. Meghan, on the other hand, blundered on. “You know how racist the American system is. I mean, they can literally read people’s minds, and yet they still imprison one in 20 black men.”

Fran tisked, annoyed at herself.  Why was she thinking of that now? Meghan and her bleeding heart politics had led her to be a social worker in rural Wales. She had no idea about urban life, or race. Fran simply had a job to do. Except for a few years during the climate trials when it felt like her Nan herself was on trial, her family had remained strictly apolitical. She was definitely now a racist.

Fran sighed and got up. She wasn’t being productive. She’d had four grueling interviews, reading the private thoughts of horrible men. She needed to go home and take a bath.

\vspace{.5 cm}

Michael was already making dinner when she pulled into the driveway. The kitchen of their 3 bedroom suburban home smelled deliciously of sautéed garlic, onions and rendered fat. "Evening. You're home early," he said with warm surprise.

She wearily put down her bag, took off her heels and accepted the proffered glass of rose. "I just couldn't hack it at work. I had four scans on my calendar today. It was simply brutal."

Michael stopped chopping carrots, dried his hands, then came over to give her a backrub. She closed her eyes and let him pamper her. "Go get changed," he whispered in her ear. "I'm making lamb chili."

"Mmm," she moaned, then lazily asked "why are you home so soon? Not that I'm complaining."

Michael moved back to the chopping board. "I have a trade show this evening, remember? I'll be out late. I figured I'd cut home early and surprise you with dinner. Otherwise, I wouldn't see you today."

She didn't remember, but instead of admitting it, she stole a piece of chopped celery, and ran to the bedroom to change.

By the time she'd showered, wafts of cooked lamb and paprika beckoned her back downstairs. The table was set for two, and Michael was in the living room, playing the jazz classics of a century ago. "What's your trade show this evening?" she asked, joining him on the couch. Her hair was wet, and he was dressed for work. As much as she might want, curling up with him wouldn't be practical.

"It’s the Four Corner's Collector event," he reminded her. 

"Snazzy" she said absently and picked up the bowl of pistachios on the coffee table. She should have guessed. Michael’s work fell roughly into two categories, helping the local tribes detect, uncover and trace the history of newly found artifacts, and collector’s events, where anyone with the interest and enough cash could purchase anything from arrow heads to urns to wall murals discovered outside of specified and regulated archeological sites. His good suits only came out for the latter. The practice made the British in her uneasy. She came from a country famed for stealing artifacts. But Michael did not see the problem with it. 

Somehow, being able to trace the origins of a piece of work back with great accuracy suddenly made archeology profitable again. It was, Fran supposed, like what it must have been like in the UK in the 1920s, when everyone wanted to take a piece of Egyptian pyramid home with them. The fact that this now put bread on her table deeply disturbed her. But she'd never get Michael to change jobs for such a nebulous reason.

"You know," Michael said, taking the bowl of nuts from her hand, after she sat silently picking at pistachio shells for a while, "if work is bothering you that much, we could start a family. Take a few years off to watch the kids, then see what's out there when you're ready."

Fran shuddered involuntarily. Whatever was wrong with work, she couldn't see herself raising a child who would grow up with the same names she got called there. "I don't know," she said instead, "I'll think about it." Then to change the subject, added with a show of great enthusiasm, "Dinner smells wonderful. Is it ready yet?"


\section{}
A swan glides to a perfect watery landing, the droplets forming its wake leap out and back to leave the rider completely dry. Sara lifts her lengha the tiniest fraction to step onto the bank. She walks to her mother's house.

She finds her in the back, feeding the chickens, collecting the eggs, cleaning and mending the coop. A lion, sunning itself nearby opens one eye to observe her. Sara, in turn, watches her mother's industry in awe. Somehow she could accomplish two to three times as much in any given day than either her sister or herself. The woman never stopped. 

"Mum?"

The woman turns around, surprised. "Saraswati! What an unexpected surprise! Come in. Come in. I've just finished stewing some pithes. You can have some while they are still hot." Her mother dropped her tools and basket, washed her hands, and started fussing Sara towards the house. 

Sara struggles gently to untangle herself from her mother's arms and fails. "Thank mum, but I'm fine. I'm just ate."

"Nonsense. You need to keep your strength up."

Sara sighs and puts her arm around her mother's waist. She gave her a little squeeze. This isn't how she had intended to tell her. "No mom. I don't anymore." 

Her mother stops moving. This complete stillness in her is so unexpected that the lion gets up from sunning itself and pads protectively towards its mistress. For one brief moment, the plump pleasant matron flickers to reveal a dark blood stained terror. Then it is gone. The matron regains control of herself and looks at her daughter with concern. "How are you doing?"

Sara is shocked by the sudden image of anger manifest, but flattered and comforted too. "It's okay mum. I'm okay. That isn't why I came to see you."

Her mother tuts in a way that makes it clear that she doesn't believe a word of it, and leads Sara to the kitchen to sit before piles of fruit, yogurt, dumplings and a cup of steaming hot sweet tea.

"So," her mother says, sitting down with a plate of salted guava. "To what do I owe the honor of this visit?"

Sara sculpts her yogurt with her spoon. Even with her gift of eloquence, this was hard. She chooses her words carefully to buy herself more time. "I've started a new project. And I'd like your perspective."

Her mother sits up a little. She's taken the bait and assumes that Sara is talking about a new artistic endeavor. She is honored to be consulted. "Certainly hon. What can I do?" Then, before Sara could answer, interjects "Though I don't like you working so hard right after a miscarriage. You should be resting. You look completely worn out."

Sara shrinks back a little. She has to be careful. It would be hard to slip anything past her mother. She was right, Sara had been working herself to the bone. The last time a universe had been created, the triumvirate had divided up the work. One to create what needed making, one to maintain what had been made, and one to recycle anything whose purpose had been fulfilled, raw material for the creator again. But now, it is just her, creating and observing, making sure that everything runs smoothly. It takes all of her time to do the work of two men. 

"Yes mom," she lies, "I'll be careful." She gives her a weak smile and checks that the deceit still holds. Then continues "I've been wondering what it would be like if we could start over, reinvent ourselves." Then she adds cautiously "Not do some of the things we've been made to do." 

Parvati pretends not to notice the allusion. She keeps the focus on her daughter. "Saraswati, I know your marriage has been hard." She pauses and swallows. Faint outlines of a skull garlanded black goddess flicker in and out of Sara's vision. "But is fantasy really productive?"

"I don't know. But I have to try. I can't cope the way you do." Sara shifts to familiar ground. How many times had she been here, in this very room, crying to her mother about the newest outrage, a beating, a locking away of her scrolls, deciding that her aviary was the most auspicious place for a month's meditation, bringing a host of kings and generals home on notice to be dined and put up for the night while he sorts out their rewards for their oblations. "I have to do something. Have something of my own to hold on to. If it doesn't work, I promise, I'll drop it." The last is probably a lie. But it wins her mother over.

"Okay, so... reinvent myself." Parvati frowns and thinks intently for a while. Sara sips her tea and waits. After what feels like centuries, Parvati comes out of her reverie and sighs. "I don't know. I can't do it." 

This makes Sara instantly and reflexively angry, in spite of all her commitments to herself to stay calm in this conversation. She needs to have buy in from her mother before she can ask her for advice. This was too important. "Why not mom! Why can't you? Why can't you imagine being something other than what" she lowers her voice as she jabs in the direction of her father's rooms "he, \emph{they}, made you."

It is now her mother's turn to react in anger. But there is no image of a wild haired black terror before her now. This is worse. It is a mother's anger at a child. Love and pain mixed with a certain sense that a wrong has been committed and must be corrected. Her voice drips with sarcasm and rage. "Because, pet, I'm not like you. Unlike you, they literally made me. You were there," she spits, and Sara recoils in guilt. "Don't pretend you don't know. You woke your husband up so they could make me. Greater than them, my foot! I kill because they tell me to. I live when I am allowed to. They cut me up and scatter me as needed. Reimagine my role?! I can no more do that than a hammer can contemplate watering my beans."

Sara, sitting at her mother's feet, is both terrified and heartbroken. She feels like, \emph{is}, a small scolded child. She knows her mother's story, and her part in it. How could she not? They had collectively come to her, praised and prayed and begged. They were all powerful men, kings and leaders; men who would have normally written her out of any story. Had she let the flattery go to her head? Had she been swayed by the urgency of the crisis? Whatever the cause, Sara had woken her husband and let them craft a living breathing woman as a tool. And she knew the cruelty of that act even as she had done it.

She takes a deep breath and stands up. Crosses over to her mother, now shaking slightly with emotion. Embraces her. "I know mum, I'm sorry. I know I was part of it, and I know what you've been through. And I couldn't have borne it. But I... I want to fix it. For you, for me. For all of us." Her mother refuses to look at her. She is too busy taking deep breaths to calm herself. Sara doesn't know if she's been heard. She bends her body so that her face is directly in her mother's field of vision. She sees that Parvati's eyes are shiny with tears. "You are also Shakti. You cannot forget that."

Parvati snorts and moves away "Pheh. Your father's better half. Light and darkness. I wouldn't exist without him." 

"Or," Sara offers tentatively "he wouldn't exist without you?"

Something Sara said while she was trembling with rage come back to Parvati. She turns to look at her daughter, suspicious, wary, and more than a little alarmed. "You said you wanted to fix it. This project of yours. What exactly have you started?"

\section{Pheonix, Arizona, USA}

"Hello, this is Meghan."

Hearing her best friend's sleepy Welsh lilt took down a wall inside Fran she hadn't realized she'd put up. She'd been a stranger in America for three years now, stumbling through different customs and expectations. She so needed to be home right now.

"Hello? Fran? Are you there?"

"Um." Now that she had placed the call, she didn't quite know how to what to say. "Um. Hi Meghan. Is this too late?"

There's a yawn and a rustle on the other of the line. "It's gone midnight. What's going on?"

Fran sighed. "I don't know. I think... I just needed to hear a friend's voice."

"Well," Meghan began. Fran could hear her force herself awake and leave the bedroom. "I told you that Ffloyd got a promotion last week. He's so much happier now that he's in a different department and not working for Enid anymore. He's in charge of sales for Swansea, Neath Port Talbot and Bridge End, which means he'll be travelling a lot more. We're still trying to figure out how that works. I'm really excited for him, but I don't want to cut down on hours. I mean, I just went back to work, but I don't think we can afford a nanny. Anyway, we'll work it out. I'm talking to my mum about coming down for long weekends. It'll be okay, and you probably don't want to hear about my troubles. Um... what else... Oh. Dylan just grew a tooth! It's the cutest thing. Well, I'm not sleeping because his gums hurt and he's crying all night, and Ffloyd just got the promotion... But a tooth! My goodness..."

"I think I want a divorce," Fran interrupted, suddenly blurting out the reason for the call. There was a stunned silence on the phone. 

"Oh. ... Really? ... Are you sure?"

"Of course I'm bloody sure. And no, I'm not going to sleep on it." Now that she'd started, she couldn't stop. "It's been coming for months, maybe years. Anyone but an idiot should have seen it coming. But I didn't, and now I do. So that's that I guess. I just can't live with him anymore. I simply cannot." Fran was surprised at how calm she sounded on the phone. When she left work, she'd been in tears. She'd told the car to drive, somewhere, anywhere, then curled up in the back seat to think. When she realized that she couldn't think, she'd called Meghan. 

"Okay." Meghan exhaled loudly. Fran could hear the tap running, probably Meghan putting the kettle on. "Do you want to tell me what happened?"

"Nothing happened. It’s not like that. It’s not like the movies where he hit me, or we had an argument and I stormed off, or I discovered someone else's underwear in his suitcase. We just... I don’t know ... don't work." Fran paused. Meghan remained silent. This made Fran nervous. Meghan was opinionated and prone to make sweeping judgements about the most mundane events on idealistic grounds. What was she thinking? Why didn’t she say something? But the sleepy welsh woman refused to answer, and Fran had to continue, "He's been pressuring me to start a family for a while now. Been getting pretty insistent about it, in fact. He knows I hate my job, you know I hate my job, God, and everyone knows I hate my job. He's trying to help, I know, but his solution is that I quit for a few years to raise his kids."

"Okay, that’s gross!" Meghan agreed, "I don't think my grandmother did that. But,” she hesitated, “I thought you always wanted kids.”

Fran relaxed a little. Meghan wouldn’t judge her poorly. But there was something else. She sucked in her breath, thinking. The words hung half formed in her mind. Arizona, America, was too different. It wasn’t her home. Try as she might, she just didn’t, no she was made to feel she didn’t belong. But it was Michael’s home, wasn’t it? And he simply thought that Fran was not trying hard enough. 

But that wasn’t entirely correct either. Exhausted, Fran gave up and simply said "I just don't think we're compatible. I mean, where did it go wrong? It seemed so perfect in college. We were both in causal fields. We both dreamed of making the world better. He wanted to study the past, and I wanted study the mind. He was gorgeous, and he was nice, and he loved the fact that my Nan had given this gift to the world. It was supposed to be perfect." Fran let out a despondent sigh. "What happened?"

Meghan snorts softly on her end of the line. 

"What?"

"Nothing."

"No seriously, what?"

"No, I promised you at your hen party that I wouldn't." 

Fran put her head in her hands. "That," she sighed, "go ahead. Rip the plaster off. You've been wanting to for years now."

"Micheal Bernstetdt is an American. He's a product of his culture, materialistic and shallow. He never wanted what you wanted. He studied causal archeology because it was where the money was. Everyone but you could see that. I'm sorry Fran, you know I never liked him -- aside from a pair of gorgeous hazel eyes and a hair that must take an hour in front of a mirror to get right, I have no idea why you did."

This got Fran irrationally defensive. "He's a good man, and he loves me..."

"And you," Meghan continued, "I'm sorry, Fran, but you asked me to rip the bandage off, have spent too much of your life obsessed about your Nan. The causal revoltion has been phenomenal, but she didn't start it, and you don't own it. You can't take it personally every time someone uses it do to something you don't aprove of." 

Meghan finished and waited awkwardly for Fran's responce. Fran held the phone numbly to her ear trying to process this sudden attack on her character. The car hummed smoothly off US 17, turning east on Route 66.

"You know," Fran said finally, "you could be American too."

"Huh?"

"That was very direct." The women laugh, and the tension of the moment evaporates away. 

After a while, Meghan askes "So, what next?"

"I dunno," Fran admits. "I mean, you're probably right. My Nan is my hero. I just want to make her proud." Never mind that Ana was apolitical. "It just hurts to see the potential of causal techonology be wasted on helping rich collectors buy expensive baubles or incarcerate men with abnormally low IQs." She thought of the 15 inmates she'd interviewed this week, some for parole, some for discipline. God she hated her job.

"Will divorcing Micheal fix that?"

"Yes. No. Maybe? Ugh, I don't know. I just have to do something. I need to get out of this rut and change what I'm doing. I can't do that with Micheal."

After a pause, Meghan asks, "What about your job?"

"I hate it? I'd give notice if I had a plan? This is shit! I'm only 27. I'm too young to be having a mid-life crisis." 

Fran can hear Meghan chuckling on the other end of the line. "Listen, we have a spare room here. I was going to offer it to my mum, but if you need a place to crash, recenter yourself, you can always come here." A pause, then more hesitantly "And... I could really use a hand."

\section{}
Durga buzzes around the kitchen, ten arms a blur of action. Making tea, arranging plates of sweets, folding napkins. Stacking trays and dishes. "You have to take this to them. Hurry."

Sara is confused. "Who? You have guests? Who's here?"

"Your father has been in conference with them all day. Your husband and your brother-in-law. If they know about what you've done... Well, you need to know what they are saying, and they need to know that you are here."

Sara falters. A meeting of the triumverate? Here? Now? It can't be a coincidence. But.. "They can't know, can they?" Can they?

Durga gives her daughter an impatient look. "There isn't a thing in this universe your father isn't aware of."

"But that's just it mom. This \emph{isn't} this universe. Its different. That's why its safe."

Durga shoots back a look of scorn and pity at her daughter and shoves a tray of food into her arms. "Just go."

Sara thinks furiously as she balances the tray of food towards her father's part of the house. They can't be aware of her creation. It was a different universe, and she had been so careful to not let them in. She had created it. Made both then energy and matter from her own body. Her husband did not have, could not have, any role. She had taken care to imbue every boson and fermion with her essence from the beginning so that she could be woven into the fabric of planets and stars later. Shakti and Vishnu were one. That way she could be part of it all without needing to constantly supervise as Vishnu had. It still needed a lot of care, but she could manage the periodic interventions required. At least until she got her sister or mother's help. As for her father, her creation was so young, there had been no need for him. That was a problem to be solved later, but he could not know yet. She did not undertand. 

Vasuki lays curled at the threshold of her father's study. He raises his head and sways threateningly at Sara as she approaches. "Fine," she says. "Go tell him that mother has sent up a tray of food for our guests."

Vasuki bobs and slithers under the curtain hanging in the doorway to deliver the message. Sara leans in close to listen. 

Vishnu's calm tenor voice is speaking. "We cannot decide this unless all three of us are in agreement. And Brahma is right. As both father and husband, he has the greater claim. You are only her father."

The room rumbles with the distant beat of angry drums. Sara can tell her father is pacing back and forth across the room, unhappy. "A compromise then," his deep baritone rasps. "Separate living arrangements." 

Sara exhales in relief. Laks must have intervened to save her marriage and said something to their father. She is annoyed at her sister, but that can come later. They haven't gathered to discuss her project. They don't know.

"Out of the question!" Brahma's three heads retort at once. "We practically..." he stops suddenly, and Sara hears the quiet rasping of her father's voice. 

A few moments later, Shiva emerges from beyond the curtain. He is as gaunt and filthy as ever. Dirt and ash steak his face and in the depressions between his ribs. Dreadlocked hair lies piled any which way atop his head. She knows he neither eats nor bathes as often as he should, but sometimes, especially after long absences, the result repulses her. Now he also seems unaware that he has left the company of men-- he has not even bothered to tie a leopard skin around his loins to protect her modesty. "You should not be here," he says without ceremony, and takes the tray from her hands. 

Sara obeys out of habit. "No, father," she mumbles and turns to leave.

As she is about to turn the corner, he orders "Stay the night. I need to talk to you later."

\section{Butner, North Carolina, USA}

“I’m sorry Fran, but the decision’s been made. We are sending your report on to the hearing, but no one from PADRES is going to testify in Raleigh.” Greg sounded calm and imperturbable on the phone. He’d made up his mind. She was going to try to change it anyway.

“Then talk to the legal team again,” she insisted over the phone, wishing she were there in person. “There are less than a handful of activists in this state who have worked on prisoner psychiatric scans for as long as I have.” 

Greg cleared his throat. “And Dr. Raghev Pal is one of those elite. He is a very good spokesman for the cause.”

“Dr. Pal is an academic!” Fran shot back. “He’s never set foot in a state prison.” She looked at her watch. It was after 2:30. There was no way that she could pop by the office and have this conversation in person. “I have. And I’ve testified in tens of appellate courts. You won’t find a better candidate.”

“Dr. Pal is the head of psychiatry at UNC’s School of Medicine. Just think of the optics, Fran. No one is denying your expertise. We are submitting your report in full to the state legislature. Don’t be difficult.”

“Did you just say optics, Greg? You mean the state legislature is going to take a balding Asian man more seriously than a Hispanic matron?” 

As soon as the words were out of her mouth, she knew she’d gone too far. “No Fran, that is not what I’m saying." Gerg's voice was cold. This conversation was over. "Make sure the legal team has the final version of your report by first thing Friday.” The phone blinked off as Fran’s car pulled into the school parking lot. She got out and slammed the door, fuming. Not denying her expertise, like hell. Greg didn't have the courage to talk about race or disability or discrimination. Neither did Pal, which is why he was testigyin. Fran had worked with PADRES lobbying against psychological scans of prisoners for nearly two decades, but they'd alway fought their battles on privacy grounds, even though the evidence ...

Her watch blinged, and Fran took a deep breath. Her daughter, Amanda, had just started kindergarten. Today of all days, she could not be late. The alarm meant she had ten minutes before pickup, enough for a quick around the block to clear her head.

She hadn’t in a million years expected herself to be raising a child in the US. After spending a few months with Meghan and Dylan, she’d had every intention of settling down in Manchester after her divorce went through. But the privacy laws in the UK were very different than in the US, and the standards of care for mental health were behind the times. She simply did not feel comfortable conducting psychological scans in NHS hospitals. 

As she hadn’t given up her US citizenship after the divorce, she returned to what she knew best. Some of her grandfather’s lawyer friends helped her find a job with PADRES, and somehow, between one thing and another, she never left. 

Ten years later, still single, she had decided to have Amanda. Now here she was, one of two women in dress pants in the school parking lot, waiting to pick her daughter up from her first day at school. 

The school doors opened and teachers led rows of orderly children outside, like beads on a string. Fran was impressed, and, irrationally, more than a little proud. 

In ones and twos, the children pointed out their parents and were dismissed, holding hands with new found friends, or unabashedly running back to a mother’s open arms. When Amanda was dismissed however, she ran past her classmates, tore past Fran, and headed straight to the car.

When Fran got there, the five year old was pounding the paneling, demanding that it take her home. She could see a full blown tantrum on the horizon. 

“Hard day, pumpkin?” 

Amanda immediately stopped yelling, dried her eyes, and turned her back to her mother. 

\emph{This is new}, Fran thought, then told the car to start driving. The vehicle turned towards the park where Amanda regularly had a playdate with a friend from work. The mothers took turns watching the kids and working on a park bench. 

“No!!!” Amanda yelled. “I don’t want to go to the park. I want to go home. I want to go to my dad’s!”

Fran was more confused than alarmed. “You don’t have a father, sweet heart. You know that.” She’d always been very frank with her daughter about her birth. 

The looming tantrum hit. “Yes I do! Yes I do! Yes I do!” Fists and snot flew all over the inside of the car. “Everyone has a dad. The teacher said so. I want to go to my dad’s house. Take me there. NOW!”

Fran took a deep breath and let it out slowly. So that was it. She shook her head, and rummaged in her phone for her colleague’s number. The playdate and the report for the legislature would have to wait.


\section{}
\emph{The universe is middle aged now. Planets, stars, galaxies dance merrily in their orbits. Some stars are still being born, other die, the clutter and creases of old age is visible, but not ruinous yet. Life teams in places, glittering specks of self determination spinning around suns, or sometimes, rarely, swimming between the stars. }

\emph{It is a quite time. A time of contentment. One was done with the energy and hubub of creation. Now is a chance to reap the benefits of one's labors, rest on one's laurels, watch one's family thrive and fourish and grow. It has had a good life, with few catastrophies or regrets. If it could, the universe would smile.}

\emph{Perhaps it is because of this complacency that the universe does not recognize the danger when it comes. It start small, a black hole where no star had died. The quiet sound of drums that dies as it hits the inter galactic void.} 

\emph{Even after it appears, the disturbance does not feel disasterous or foreign. It certainly does not feel foreign. Just a little bit of one's self that isn't quite working as it should. More like a painful blister than a tumor, and certainly not an infection or intrusion. Something that should, by all reasonable expectations, pass.}

\emph{But the discrepancy is voracious. A pinprick of a hole snags into a tear, then a slash, then a ragged edge of cloth that cannot be patched. The black hole feeds, it seems, on the very vaccuum itself, drawing in the surrounding galaxies around it, like a woman gathering up her skirts. By the time the universe is aware of this violation of physical laws, it is too late. The nullity eats and it grows and it consumes, expands, devours until it is a solid mass of nothingess against the background of stars. Clearly visible as terrible black eye arching across half the night sky to any onlooking lifeform. Anywhere.}

\emph{The drum beats are audible now, even across the vaccuum of space. It is a primal call, known alike to stars and planets and microbes. It was, after all, there at the time of creation and before. It was present in the creator herself.}

\emph{The universe responds as it knows it must. This is the tandavam, the beginning and the end. All things must obey, just as all things must be born and die. It knows that. So it submits. Reluctantly. Sobbing. It lets itself be consumed. It does not want to go. If only it had a little more time.}

\section{Butner, North Carolina, USA}

"And so, it is my pleasure to turn the stage over to a woman who needs no introduction, the next President of the United States, Senator Hanako Perez." The crowd erupts into applause as the small East Asian looking woman, dressed casually in jeans and a cardigan walks onto the stage. This is a small campaign event for Senator Perez as she weaves her way across the country preparing for the Democratic primaries. But for Fran, this is an opportunity to really focus the discussion on prison reform. 

"Thank you Ms. Gusman-Pe\~{n}a," the Senator spoke softly, too softly for the microphones to pick up, while firmly shaking Fran's hand, "for your service to our country." 

The words send a shiver down Fran's spine. She was suddenly aware of the intimacy of the moment, as well as the audience for hundreds. Fran had never served in the armed forces. She hadn’t even given up her British passport. Yet Senator Perez had looked directly into her eyes and recognized her life's work. \emph{So this is what it feels like to be acknowledged}, she thought. "Thank you, Senator", she replied. That woman absolutely had her vote.

Stepping off the stage she took a moment to look around the audience. Fran had filled the high school auditorium with friends and allies, colleagues, neighbors and opponents. But she had filled it with people who were at least interested in the discussion. Butner was a small town that had had a federal prisons in it for over a century. Now mostly a bedroom community for Durham and Raleigh, the federal penitentiary was the largest local employer. Bringing a serious presidential candidate here felt like a major step forward.

Her ten year old daughter waved her fists in the air, mouthing "GO MOM!" as Fran came to her seat. Then Amanda curled up on her arm to listen to the senator talk.

\vspace{.5 cm}

"Mom?" Amanda asked a couple hours later as they walked home along the bike path. It was late October. The summer heat had finally eased to make a long walk a pleasant activity. "I'm thinking of running for student council."

"That's sudden." Fran replied. "I thought you said it was a popularity contest." Amanda had never quite felt like she fit in at school. Rather than bemoan the fact that her family was different from those of her friends, by the end of kindergarten, she’d simply made it known that she didn’t care for her peers.

"It is. But I was thinking.... Running for president is too, isn't it? I mean, kinda."

Fran laughed. She liked this new side of her daughter. "Maybe kinda. There's a lot of policy and research and negotiating involved too, you know."

"Sure." Amanda swung her arms wildly by her side, full of nervous energy. "But that's because being president is important. The student council doesn't do anything. It’s just something to do."

"Huh" Fran probed, "So why do you want to run for it?"

Amanda hesitated, suddenly shy and awkward for a moment. Then she stood on tiptoe, pecked Fran on the cheek and said "Because I want to be like you. Make a difference, you know. Fix things."

Fran didn't know whether to laugh, or cry. She wanted to encourage this new found confidence, but also tell her daughter sternly to find a different, less heart breaking, line of work. Instead, she squeezed her hand and said "Sleep on it, love. You have the rest of your life to figure it out."

"I know." Amanda kicked a stone off the gravel path. "I will. But I won't change my mind." Then she let go of her hand and started skipping ahead on the path.

\section{}
Sara awakens with a cry from the nightmare of her creation's demise. The sound of the particles vibrating in time, and drum beat in what should have been hard vacuum. Some monster had discovered her most closely guarded secret. It had entered as quietly as a possum stealing eggs from a nest, broken the shell with a claw, found the resonant frequency of her galaxies and shattered them all to smithereens. Her beautiful creation. Weeping. She'd heard her star and nebulae weeping. Her mother moves beside her, still asleep. Sara takes a deep breath to remind herself of where she is. In bed. At her mother's house. It was just a nightmare. Her creation was okay. It was just a dream. She would just take a moment to check on it. 

As she gathers the mosquito netting to find a quiet place to meditate, she is startled by her father's body sitting cross legged on the floor beside the bed, reeking of hashish. He too is meditating, naked except for Vasuki, sleeping, loosely coiled around his shoulders and draped across his filthy body to hide his genitals from view.

And suddenly, she realizes, her dream was real. The drum beat was a tandavum. Only one man new how to shatter stars like that. One one being knew that dance. Her eyes well with tears and her head buzzes with anger. She can barely breathe, let alone think. "You!" she whispers, screams in hushed tones, unable to disguise the hatred in her voice.

Her father opens his eyes to look at her.

"Why!" Sara yells, unable to stop herself now. "How?" Sara is sobbing now. She cannot control her emotions. "There was no place for you there! I was creator and preserver in one. I buried my own dead and obsolete with love. You could not have gotten in!"

Her father clears his throat. It is a low rumble, felt in her chest more than her ears. A warning to pay heed or face the consequences. "You were the light. But I am the darkness. There cannot be one without the other." Sara gaped. Astounded. Hearing the words without comprehension. "What you made. It would have tipped our balance. It had to be destroyed. For your own good." 

Shiva rolls to a kneeling position, then rises and pads out of the room. Sara stares after him, stupefied.

She is not aware of her surroundings until her mother raises herself on an elbow and puts and arm on her shoulder. Only then does comprehension fully dawn on Sara. She collapses on her, gasping and sobbing in small keening sounds. All those lives. All that effort. All the hopes she had built for herself and her family. It was too much. "Sweet heart. Hush now," Parvati starts. "Its..."

But Sara can't take it any more. "Don't" she screams. "Don't defend him. Of all the things in this wretched creation, \emph{don't} defend that horrid man." 

"I wasn't." Durga is harder now. "Listen to me girl." It takes Sara a long moment to control her breathing and dry her nose and eyes. When she is ready, Durga continues "What you made, you made it out of your hopes and dreams. It was based on everything you knew." 

Sara nods pathetically. "But it wasn't good enough" she whines. "I couldn't make a home for us."

"It was perfect, for what it was." Her mother strokes her wet cheeks. "But it couldn't have been a home for us. Everything you know was built by them. When you start from that and make a small change, it is already, essentially, built by them. Even if you hadn't overlooked some small loophole, the patterns and the rhythms and the resonances would have been too familiar to them. They would have sensed it." Durga pauses, searching for words. "It was too close to what we now have. They would have stumbled across it eventually. "

Sara buries her fist against a pillow and screams, long and gutteral and low. It wasn't fair. Her beautiful peoples. Each and every one of them. They worshiped her, and she should have protected them. All of them taken before their time. How could he? 

"Hush, you stupid girl," Durga persists. Not angry, but not sympathetic either. "Its not over." Sara stops and rolls over, hugging the pillow to her chest. "You have been trained in every creative endeavor our husbands could imagine. If anyone can create a universe that does not play by their rules, that works differently than they do, my money is on you. Stop feeling sorry for yourself and get to work."


\end{document}
