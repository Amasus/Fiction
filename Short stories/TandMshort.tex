\documentclass{article}

\begin{document}

The young Duke Makala was a boy of fifteen when I first came to Cortan. Even then, he was as smooth as silk and  as confident as a king. He would inherit the lands his father had carved out for Marsea, he would add to Cortan's glory and wealth, if not by his military strategic abilities, then by his ability to inspire men to greatness. He was a small man. Duke Lukos, his younger by two years, and a sterner, more dogmatic man, was almost of a size with him.

I tried no harder than any other sergeant in Cortan's army for either of the Duke's attention, though by the end of my second year in Cortan, I found myself inexplicably graced with unexpected moments of Duke Makala's time. It was not unusual for the duke to draw me away from a dicing table to sit across a chess board from him. He had the habit of leaving me in the middle of a game, extracting from me the promise to meet him again to finish what we had started. His game was always unconventional, often surprising, rarely successful, but almost always challenging. He left me looking forward to our next encouter. Duke Makala had a habit of making queer friendships. He had grown attached to a young orphan from Cortan's White Tower, a girl of five when she started first appearing at the Duke's table. I did not question the Duke's habit, or his favours towards me. Those first years in Cortan were difficult for me. I found myself alone in a harsh foreign duchy, far from the comforts of home, yet with no with abatement from the risk of ruining my father with my secret. My position in charge of a tent full of ambitious boys presented me with dangers more often than I would like. I responded by forcing myself to be scrupulously fair to those in my care. I had no room for personal favourites if I wished my true nature to remain concealed. I found middle aged bell ringer from the temple to share my secret with. Dim witted and large, he is called miscast for his shortcomings. Yet Fidelo was loyal and warm in his own way. We shared our outcast status when I returned from missions to the border or when loneliness and fear threatened to overwhelm me. I made certain to pay the man well for his time and cursed the Maker for seeing fit to make me miscast as he did.

I remember being pleased, though not entirely surprised, when, as my second summer in Cortan drew to a close, the young duke invited me to join him on a small lion hunt. I was more surprised when, still smelling of lion's blood, the duke asked me to ride with him, leaving the rest of our company to celebrate his kill. He wanted to spar. With the thrill of the hunt still surging in his blood, and the growl of the great beast thrashing in the grasses still echoing in my skull, we dropped our spears and drew swords in the hot dusty midmorning sun. The taste of victory over death ran rampant through our bodies. Anything was possible in this world where our physical strengths reigned supreme. Makala moved like a dancer more than a fighter, as the Black Riders he trained to fight with did. The young duke may not have been a member of that elite group of mounted archers yet, but he was fast, agile,and fearless, even for a member of Marsea's army. In comparison, I was slow and stoic. A larger man, I would press an advantage when I saw one, but I would not tire myself uselessly on a spry opponent. The dry grasses and crunched underfoot, the sun beat on my thick black hair, the dry heat stung my nostrils, Makala glistened with sweat and the exuberance of his first kill, prowling just beyond my reach until he chose to close in to exchange a few blows. I blocked and parried and circled and waited, enjoying flexing my muscles in this excersize in warfare and the false sense of danger that accompanied it. Makala danced against me, drawing me forward at times, pushing me back at others. He was playing in the late summer heat, and so was I. I found myself at one point backed against a thorn tree. When I moved to step around it, I snagged my shirt on a bramble. The young duke swept out my legs, landing me firmly on my back on a bed of thorns. It had been foolish of me to get so close to the offending tree. I had focused too much on the delight of the fight, and not enough on my surroundings. I prepared my self for the demand that I yeild. Instead the duke tossed aside his sword, lept upon me as if to wrestle me and kissed me full upon my lips. The exuberance drained from my body as I lay under him, suddenly feeling more endangered than I would  had the lion we hunted sat atop my chest. My duke, grinning, counted down the seconds to my defeat. The ambitious third son of an ungifted barron does not decry his duke as miscast, especially when he fears the same label upon himself. Duke Makala's hold on me was weak at best. I swallowed my immediate terror and threw the laughing duke onto the dusty yellow ground and wrestled him as he desired. I did not dare defy him in this isolated compromised position. If he knew, if he would tell, not only would I be exiled or stoned, but my father would be ruined. I had spent the last six years of my life training myself not to feel desire when wrestling and sparring and drinking with men and boys I would later share tents with. I had trained myself not to notice the ripple of shoulders, or breathe the acrid smell of sweat, or linger on the thought of dark lips meeting mine. But the duke was laughing, drunk on the success of his hunt that the pleasure of our bodies grinding and struggling against each other. We rolled and grappled each other among the brambly twigs, yellow dust stinging as it caked itself onto our torn skin. I would never had dreamt of letting my desires lose if my lips did not still throb with the memory of the kiss they had just met. Duke Makala was nearly a head smaller than me, wrestling not his strength. I told myself that the duke wanted from me only what I wished from my bell puller, wrapped my legs around his narrow waist and pushed his heaving chest to the yellow earth and called for him to yeild. The young duke's body went limp. He was still laughing, his chest and shoulders rippling silently with delight under my thighs, a motion that would become to familiar to me in the years to come. The ecstacy of his joy and the ferocity of my terror became one and the same. I was only aware of our shared solitude. The dark smell of lion's blood mingled with the sharper aroma of my duke's sweat to awaken an urgent need I had made a long habit of leaving unacknowledged. The duke still laughed as I helped him to his feet. His fearless reveling in our illicit sparring captivated every last drop of my attention. I stepped out of the cautious shadow I had learned to live my life in. We were alone in this grassy clearing. No one would know if I returned to those dark parted lips the kiss the duke had taken from me. I still remember the relief I felt when he returned my embrace. I gave myself to the pleasure of the moment and whimpered in his arms. We fell to the ground again, our tumbling becoming more urgent, our clothes scattered across the clearing as we danced to a different tune. We moved quietly, secretly, united in our delight and our shame. The sun beat down upon our bare prespiring skin, caked with yellow mud and straw where flesh did not desperately stick to flesh bound by desire and  sweat. I licked at the salt crusted on his neck, felt his hot breath drying the moisture beading on my chest, dug my nails into his supple narrow back and brought my mouth down hard onto his, in a need to be closer to him than the physical barriers of our bodies would allow. The young duke still tasted of the spice of the dried meat we had consumed before dawn. My shame sat with me, as it always did, but further than usual, dwarfed by the delight of a consenting conspirer. When I could control my desires no more, I rolled the young duke over and mounted him. He moaned when I pushed myself inside his small muscled buttox. "You have done this before," he gasped as a I moved, breaking our silent passion with his surprise. I could not respond in my shame. I had not realized that I introduced my duke to the world of fear and shame our corruputed natures forced us to live in. I rubbed his engorged member to the rythm of my need until we both released the malcreated tension the thrill of the lion's hunt that planted in us.

We lay still in the sun baked yellow earth afterwords, I silent in my shame, Makala ebulient in his exhileration. He spoke of his plans for the duchy when he would rule. Foligno's duke had angered the new regent consort. It would be Cortan's turn to rise soon. His father had great plans for the duchy, and Makala would ride with the Black Rider's. Cortan's Black Rider's were far superior to Foligno's. Next year, when he rode with them, he would reveal the Folignian commander to be the fool that he truly was. I listened and picked flakes of yellow grey mud from the young duke's shoulder. 

"You are silent, Timmon," the young duke said at long last. "Are you always so silent with your lovers?"

Lovers. What an inappropriate word. What did we miscast know of love? We were only granted a corrupted overpowering need. "I have never had the pleasure of a lover."

Duke Makala sat up to give me a queer look. "You have one now. I will make you a general in my court if you stay with me." He stretched and rose, walking naked through the yellow grasses, picking our scattered clothes off the brambles and grasses with a well muscled ease.

What will you do to me if I do not stay with you, your grace, I thought. I said nothing but watched his agile, sun kissed body disappear behind layers of cotton and leather. This man could be the making of me if I were careful, if my desire for him did not prove to be the ruin of me first. It was an impossible loathesome situation I found myself in. I had corrupted the innocence of the young duke. If I were a noble, moral man, I would advise him to shun my company and seek happiness in the company of women and his father's pride. But the Maker had seen fit to make me miscast, and I could not tear my eyes from his narrow lithe back and the dark lips that had kissed me.

"Are you coming?" Duke Makala asked, startling me. He laughed and tossed me a bundle of clothes. "Stay if you like. I'll tell the others that a brush farie mesmerized you and stole you away to her lair."

\vspace{.5cm}

Back with our companions, Makala gave no sign than anything had passed between us beyond a friendly spar. I followed his lead and did the same. It was best not to contradict the story my duke had spun, and I watched him carefully for clues. We spent the day slowly making our way back to the barron on who's land we had killed the great beast. When we arrived, we ate and drank and made ourselves comfortable. That night, I watched Makala put away more goat's meat than our prey could have done, then gather up a handful of our companion's to visit the brothels. He extended the invitation to me, in the same manner as he had to all his other comrades, but I refused him. Under the best of circumstances, brothels held no allure for me. With Makala as drunk and exuberant as he was, it could prove disasterous. I watched the Duke and his friend's disappear into the night, gathered up a handful of olives and retired to my straw palate.

We did not meet in private for over a week in Cortan. I buried myself in my work during the days, focusing all my attention and energies on the boys in my care, keeping my ears open for the faintest hint of a whisper about Duke Makala's nature, or my own for that matter, telling myself that the absence of Makala's attentions could only be a boon to me. I would become a general by my own means, in Cortan, or elsewhere. Pursuing our encouter on the lion hunt would be nothing but folly. My nights were different. The lion hunt haunted my dreams. I was the lion, stalking the duke. I watched him from a distance, invisible in the tall grasses the color of my rich tawny fur. He searched for me, alone, his companions far off in the distance. A hare made a noise behind him, the duke turned to see and I sprang. We rolled together, man and beast in the dry yellow lands, wrestling, grappling, thrashing, but silent, as if all sound had disappeared from the world. In my dreams, Makala was as strong as a pair of horses. Sometimes, he bested me, driving his knife deep into my stomach, leaving me to bleed painfully to death in the desolate grasslands. At times, I would pin him to the ground, and his face would change from terror to joy, his chest and shoulders rippling with laughter. The I would bend down to kiss those beautiful dark lips, only to find myself feasting on his flesh. Whatever the dream, the effect was the same. I would sit up at night with cold sweat dripping down my back, terrified of being found out. 

Then, one evening, Makala found me in a ring of men watching a pair of dogs fight. He stepped between the rows of shouting and cheering men to my side. "I have an errand to run tonight," he said in a low voice close to my ear. "Accompany me."

I sighed and tore my eyes off the fight. I had money riding on the white dog with the black eye and chewed off ear. He did well in this fight. I would rather have watched him succeed and back him in the next fight, than spend an evening alone with the duke.  He could make me a general, or he could ruin me I thought, and followed him out of the yard. "Dear Charmina's cat has run away. The poor girl is completely distraught," he explained as we left the barracks.

I balked in my surprise. "I beg your pardon, Makala. Why are you searching for her grace's cat, and not the palace guard?" Why was I accompanying him on this impossible mission?

Makala shrugged of my incredulity and answered in complete seriousness "I try to be a good brother, Timmon. My siblings trust me. Charmina cannot go to mother on this matter. The Duchess is still angry at my sister for having been found ungifted by the Tower over a year ago, and father does not have time for such matters. So it falls upon me. Did you have money in that fight?"

"Yes," I replied, trying not to sound too disappointed. "On the white one."

"Well," Makala said, his voice filling with mirth, "I thank you for coming with me. I'll make certain you are given your winnings." 

My winnings were the least of my worries. Makala walked the streets of the city with a firm determination, not searching or calling, as if he knew where he intended to go. My duke wanted me for more than an extra pair of hands in catching a ducal pet. It seemed unlikely that we were searching for an animal at all. The uncertainty filled me with a strange fear flecked with a nervous, shameful anticipation.

At length we came to a small wooden door in an alcove at the base of the west tower. Makala lit two torches, handed me one, and pulled out a set of keys. This was the oldest part of the castle, the west tower the shortest of the nine towers of Cortan. The stones of the stairs winding to the top well worn by the soles of guardsmen. But the floor of the alcove had not been similarly smoothed by feet. It lacked even the marks left by the frequent opening of a heavy iron banded door. But the door slid smoothly and silently on its hinges, it had been oiled recently in expectation of use. The young duke waited for me to stoop my head to enter the dark cramped passageway beyond before closing the door, leaving us bereft of light and the knowledge of the outside world. Before us lay a set of deep narrow stairs. Cobwebs hung on the ceiling, and the light of our torch disturbed a few bats. My hand went to my dagger.

"Put that away, Timmon," Makala laughed. "I do not expect to find the cat held hostage by an army or armed rats. Be careful of the steps, they are somewhat uneven I've been told."

I started to wonder if the young duke had lost hold of his senses. How would a cat have locked itself away in this dark tunnel? My distrustful silence must have spoken volumes. "You have never been in this part of the castle?" my companion asked. "This passage leads to the Tower, if you follow it far enough. There is an easier entrance behind the temple you may be more familiar with."  

I was. We had entered the long tunnel that connected the castle to the Tower and provided an escape from the city in case of seige or attack. I asked the obvious question. "Why do you expect to find her grace's cat down here?"

"This is where Lukos claimed to have put her. Keep your eyes to the ground. She is grey, and likely cowering."

I did not understand. "It seems a lot of trouble to go through for a prank to play on one's sister."

Makala gave me a questioning look, then laughed. "Don't be daft, Timmon. Lukos did not have to steal the keys to come down here. He came in by the temple, or got one of his gifted friends to smuggle the cat through the Tower. But those entrances are too public. We would have been seen."

I felt my heart hammering in my throat. My presence on this errand was about more than just a grey cat. I tried to come up with a list of reasons why I could not accompany the young duke any further, when my companion interrupted me. "There is a store room through there. See if you can find her. I'll go to the next one." I turned right to find myself in a small room filled with barrels and crates, large lengths of thick rope, odds and ends of old carpenters tools, and a few wooden maces and half rotten sheilds. It was less of a store room and more of a midden heap of forgotten and neglected junk. I felt ridiculous bending over barrels to peer into the shadows for a small cowering form, lifting heavy crates with rusted hinges, scanning the darkness for signs of movement, chasing after creatures that turned out to be rats or moles. I came up with a hundred excuses to give the duke for abandoning my search, to leave via the temple and return to my companions and dogs. I told myself I would finish this room, then leave.

I found Makala in the corridor, and shook my head. "I'll take the next room on the right," he said before I could speak. "There is another room on your left ahead."

My protests died in my throat. I woud follow my duke's commands, whatever he asked me to do. I found the room I had been directed to. It was barely more than an alcove, with a sealed well in the center, cluttered with rusting buckets and a few rotten barrels. I had just started the process of searching for a frightened pet in the bottom of the rotten barrels when I heard heard a voice echo down the dark corridor. 

I ran back to where Makala searched to find the young duke triumphantly holding a wriggling sack at arms length, laughing at me for the look of concern on my face, and the blade in my hand. "It's only a cat, Timmon, there is no danger here. One might think that you are afraid of the dark. Come in here."

"I am sworn to protect you, your grace," I said weakly, and followed where he beckoned. Unlike the other rooms I had seen, this was a dry, well maintained store room, lined with shelves containing boxes of incense, brass oil lamps and glass bottles. Piles of furs lay on a raised wooden platform to protect them from the dank packed earth under our feet. An open crate revealed piles of dark woolen cloth of good quality. A few barrels sat stacked in a corner, smelling of the fragrant oils used in temple ceremonies. Makala put the wriggling sack atop one of these and settled himself on a pile of furs. "What is this place?" I asked.

"The temple uses this as a store room for excess supplies. I'm afraid Charmina's cat soiled Father Anglius's stock of robe fabric. I'll have to make certain that both Charmina and Lukos confess to this little crime. Sit." I perched nervously on the edge of the wooden platform where the young duke indicated, and looked around the room, carefully keeping my eyes off my companion. I should have left long before it came to this. "I like you, Timmon," Makala continued, "but you have been avoiding me since we returned from the hunt. It's a poor show of affection for a lover, I'd think."

I swallowed hard. What could I say to not offend the young amourous duke? Alone in the dark storeroom, I did not dare admitting my reservations. He held my future in these lands in his hands. Neither could I confess my fears. "I dream of you every night," I whispered at last. At least it was not a lie.

"Really?" Makala sounded impressed. He slid off the pile of furs to put his arms around my waist. "What do you dream?"

I cannot be responsible for corrupting your nature, your grace, I wanted to say. I could not imagine why he would want to lead the life of fear and secrets I led. But he was smiling when I turned to him, the same open cresent of dark lips and white teeth that had called to me on the desolate grasslands and haunted my dreams. We were alone, many feet underground, somewhere between the castle and the city, with only a frightened cat to witness our shame.

"I dream I am a lion," I said, then turned to devour my prey.

\vspace{.5cm}

We met in private several times after that, sometimes in unused sheds in the barracks, riding out beyond the city walls to a desolate field or pasture at other times, sometimes at Makala's bidding, at other times instigated by my weakness and inability to control my desires. We never used the same place twice, we never entered or left our rendevouz point together. Makala was always delighted to meet me, I always terrified of the mark this would leave on my soul for corrupting the next duke of Cortan. Eventually, I could take the shame no more and stopped attending morning services at the temple altogether. I could not face a chance encounter with a confessor or the question of why I had not confided in the temple. I could not keep this secret from the Maker's hands atop every other secret about my true miscast nature. 

I still visited Fidelo, my bell puller. He felt safer, he was loyal to me, he could neither make me, nor could he betray me. It was the path of an unambitious soldier, the one I knew was safer, but did not have the discipline to take. Ambition fought for my heart against my desire to keep my body, and my father's name intact. The slow witted ugly man's appeal paled before the memory the young, ambitious, lively duke. Fidelo had nothing to offer me but safety. That was hardly enough to satisfy me, physically, or otherwise, though for my family's sake, I told myself it would have to do. 

Yet I went to him the night before I marched east for the spring's campaign. Makala had offered to make me a captain a few days before. Our nightly games of chess had drawn an audience as it frequently did. As I had won the game we had just finished, he demanded that I pay for a round of drinks for our company. He was in an exuberant mood that night, already drunk on something, I could not tell what, perhaps just the pleasure he reaped from living. 

"I could make you a captain, Timmon, would you like that?" he said, setting up the next game. "Say the word, and I will speak to father. Tell me you owe me a favour, and I will see that it happens before you march." 

I felt the blood drain from my face while I clenched my teeth in anger. I had hoped that Makala would advance my career, but not like this. This was too public, too obvious. If word got out that I was his favourite, how long would it take for jealous eyes to peer into the dark shadows and secluded spaces in and around the castle to find the two of us lying together in shame? And I would not be his whore, exchanging my body for my rank. Miscast though I was, I had my pride. The name Romino meant something, even in these Preserver neglected barren lands.

Makala finished setting up his two rows of white figures. "Will you not play another game?" he asked innocently, glancing at the emptiness on my half of the board.

"I will be promoted on my own merits, your grace," I bit off. The room around us hushed in tense anticipation as Makala and I stared at each other for a long moment. 

Eventually, he shrugged. "As you wish," he said, and turned his attentions to a newly made sergent with the Black Riders, a man a few years younger than me, name Carmine Dielo. A jealous fury raged in my chest. I may only have been a member of the infantry, a man more proficient at reading maps and scouting than riding and archery, but that was no reason to cast me aside so easily. I slammed down my cup and rose noisily from the game, leaving the warmth and company of the hall for the damp chill of the winter night.

It was better that I seek comfort from Fidelo before I marched than attempt to approach the young duke. We had not spoken since that day. I had angered him, he had offended me. With any luck, I thought, as I paid my man for his time, I may have released myself forever from my misbegotten bond with the young duke.

\vspace{.5cm}

I returned from the campaign a self made captain. A strange restlessness gnawed at my heels during my time away from Cortan. Only the respect and admiration of my companions and superiors would gain me any peace, and that only temporary. My father had fought beside Duke Ergino to gain new lands for Cortan. He had earned his name in this duchy and been honoured by the duke. I would make my way in Cortan's army as my father had, by honour and courage, not by being his son's pet dog. 

There was a wreckless courage in my manner of fighting, a unconventionality to my scouting missions and an edge of anger to every other part of my military life during those three months away. Commander Lazaro, who I served under, a ruthless man himself admired these qualities. When I returned from Vestali dell with accurate numbers of the Harapi forces we would face and a Harapi scout bound and gagged to my horse, my promotion was all but guaranteed.

I returned to Cortan sated, promoted, and eager to visit my father and tell him of my achievements. The old man would be pleased, and I longed for a change from these dry yellow lands that had been my home for nearly three years to see the fertile valleys of Deyalorn and my youth. If I went before the Day of Unions, I could take my sister Imelda to the stream in my father's woods. I could catch fresh fish for her one last time before she married. My nephew Cheppe was barely more than an infant when I had seen him last, perhaps he would be old enough to teach the game of chess. On the last day of our march back to Cortan, I filled my head with plans of the gifts I would bring my family from these hot dry lands, orange marmalades and fine cottons, perhaps even a horse for my father's stables. 

The dreams disappeared when a page found me soon after I enetered the barrack gates. He handed me an unsigned, unsealed piece of paper with the words "Base of west tower, sundown." There was only one possible source for this missive. I resolved not to meet Makala tonight. It was for the best that we had argued before I left for the campaign. It would be better for all if he simply forgot me. I tossed the note into a fire and went in search of Fidelo to meet that night in our usual place and time. 

I learned at the temple that my man had left suddenly a month ago. He now lived with his sister in the distant barrony of Quiro. This was Makala's doing, I was certain. The young duke had not finished with me yet, as much as I wished to rid myself of him. 

I found myself waiting in the alcove where the duke and I had started our search for the duchess's cat at the appointed time. It felt too public and conspicuous, though that part of the castle was deserted. Three minutes passed by like as many hours before Makala found me there. He gestured silently for me to follow him. We went up the empty staircase, down a narrow passage, and up another smaller set of stairs that led into a large comfortable bedroom. Beside the bed, stood a large round table set with a game of chess. The shelves held an odd combination of trinkets and weapons. The floor was richly carpeted, the windows opened to the west. I could see the deep reds and purples of the evening sky from my vantage point.

"Well, what do you think?" Makala asked, throwing himself languidly onto a cushioned chair. 

"Where are we?"

"In Duke Ippolito's room. You entered through the once secret passage built for my grandfather's safety. The passage won't be guarded until the castle's gates close. These are my rooms now. This and the room through that wall." He gestured to the south with his chin. "We can meet here without fear."

He paused expectantly for my answer. "You have done well for yourself, Makala."

The young duke tensed and sat upright in his chair. "Is that all you have to say after an absence of three months?"

He did not raise his voice, but I knew I had upset him. I proceeded carefully. I did not need a vengeful duke to destroy my father's name. "Makala, your grace, this affair of ours cannot continue. The damage it would do to both our houses, I cannot bear that on my concience."

Makala studied me for a long moment then said, "You are still angry at me, is that it? I humiliated you the last time we met. I apologize. You became a captain on your own merits, I congradulate you. I will not insult your abilities again. Will that do?" He bowed pentinently before me. 

It was an awkward moment, for the third son of an ungifted barron to raise the heir to Cortan to his feet. He should never have bowed. I was no longer angry with him. I simply wanted our liasons to end. "If not that, what is it?" Makala asked, when I offered him nothing but my forgiveness. "I spent the last three months praying like a woman that the Preserver will keep you safe and the Destroyer will grant you strength. I have never felt so lost as when I did not know of how you fared in battle. Now that you have returned, victorious and whole, how can you have nothing to say to me."

I gazed dumbly at the young duke, my wits sucked dry by his earnest confession. I had nothing to say to him. I had missed him as well. Seeing him again, that much was obvious. The restlessness of the last three months sourced as strong by my desire to earn Makala's respect as it had been to honour my father's name. 

The words caught themselves against a lump in my throat. It would be better if our liasons ended. 

"I love you, Timmon," Makala said suddenly. "You are breaking my heart."

The desparate words of my beautiful duke frightened me into speech. "You must be mistaken, your grace. Miscasts cannot know love. It is not in our nature."

"In the Maker's name, don't bring formalities into this room," Makala began, then stopped suddenly. He turned to me with a strange curiosity. "You truly believe that fragile yarn about miscasts and love?"

"I would not have said it if I did not." I spoke coldly and formally, retreating to the common wisdom about me like us. Any other tactic would lead to a breaking of promises and a further fall from grace. The temple's logic was simple enough. Idiot children and deaf and blind did not seem to know or understand the concept of love as the those the Maker had made whole did. How could one as warped as I hope to cast my eyes upon the heir to Cortan?

Makala backed slowly away from me to pour two cups of wine. He spoke in a calm quiet tone. "I thought you intellegent, Timmon. You love your father as well as any son, and better than most. I love my siblings more than anything else in the world. Yet you believe the withered words of our speakers more than your senses. Very well." He handed me a cup. "Can we at least be friends, in this room, if not beyond these walls? Will you join me for a game tonight?"

I had never thought of my nature as Makala had put it to me that night. Of course I loved my family, my father, my siblings, my nephews. Life in Cortan away from them had felt like a long exile at times from everything I knew and loved. But the Makers hands spoke with such certainty about the nature of the miscast, I could not doubt the truth of their words. I smiled warmly at the young duke and joined him across the chess board. 

We talked and drank and played that night, as old friends would have. I revelled in his company, in his complete and devoted attention. He had joined the ranks of the black riders in my absense, he would train with them now that the army had returned. He told me his plans for the games at the castle to celebrate his coming of age in a month. He listened to my tales of Deyalorn, and of how desperately I missed my family. He promised to see what he could do to get me leave to spend with my father, and I let him win the game at hand.

Makala chased me from the room well before the guard would be at his post. When he kissed me goodnight, a small gentle peck on the cheek, he planted in me a seed of hope that life could be different from this point forward than anything I had hitherto dreamed possible. 

\end{document}
